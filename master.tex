%This document wants to the explain the quran package with some examples.
\documentclass{report}
\usepackage[a4paper,margin=2.2cm]{geometry}
\usepackage{hyperref}
\hypersetup{
    bookmarks=true,         % show bookmarks bar?
    unicode=true,          % non-Latin characters in Acrobat’s bookmarks
    pdftoolbar=true,        % show Acrobat’s toolbar?
    pdfmenubar=true,        % show Acrobat’s menu?
    pdffitwindow=false,     % window fit to page when opened
    pdfstartview={FitH},    % fits the width of the page to the window
    pdftitle={Quran},    % title
    pdfauthor={Author},     % author
    pdfsubject={Subject},   % subject of the document
    pdfcreator={Creator},   % creator of the document
    pdfproducer={Producer}, % producer of the document
    pdfkeywords={keyword1, key2, key3}, % list of keywords
    pdfnewwindow=true,      % links in new PDF window
    colorlinks=true,       % false: boxed links; true: colored links
    linkcolor=blue,          % color of internal links (change box color with linkbordercolor)
    citecolor=green,        % color of links to bibliography
    filecolor=magenta,      % color of file links
    urlcolor=cyan           % color of external links
}
\usepackage{forloop}
\usepackage{quran}
%The xepersian package automatically load bidi, and I've loaded it because I want to set a font that supports Arabic letters
\usepackage{xepersian}

% This macro set the main text font for non-latin letter, and it can scale font.
\settextfont[Scale=2]{Scheherazade}

\def\surna[#1]{\centerline{\hss\surahname[#1]\hss\lr{\surahname[#1]}\hss}}
\def\test#1{
    \par
    \surna[#1]
    \quransurah*[#1]
    \bigskip
}

\title{Quran Translation}
\author{Abdulla Yusufali}
\begin{document}
\maketitle
\tableofcontents
\newpage
% For typesetting بِسمِ اللَّهِ الرَّحمٰنِ الرَّحيمِ use below macro
\chapter{\textmalayalam{അല്‍ ഫാത്തിഹ ( പ്രാരംഭം )}}
\begin{Arabic}
\Huge{\centerline{\basmalah}}\end{Arabic}
\flushright{\begin{Arabic}
\quranayah[1][1]
\end{Arabic}}
\flushleft{\begin{malayalam}
പരമകാരുണികനും ദയാപരനുമായ അല്ലാഹുവിന്റെ നാമത്തില്‍.
\end{malayalam}}
\flushright{\begin{Arabic}
\quranayah[1][2]
\end{Arabic}}
\flushleft{\begin{malayalam}
സ്തുതിയൊക്കെയും അല്ലാഹുവിന്നാണ്. അവന്‍ മുഴുലോകരുടെയും പരിപാലകന്‍.
\end{malayalam}}
\flushright{\begin{Arabic}
\quranayah[1][3]
\end{Arabic}}
\flushleft{\begin{malayalam}
പരമകാരുണികന്‍. ദയാപരന്‍.
\end{malayalam}}
\flushright{\begin{Arabic}
\quranayah[1][4]
\end{Arabic}}
\flushleft{\begin{malayalam}
വിധിദിനത്തിന്നധിപന്‍.
\end{malayalam}}
\flushright{\begin{Arabic}
\quranayah[1][5]
\end{Arabic}}
\flushleft{\begin{malayalam}
നിനക്കു മാത്രം ഞങ്ങള്‍ വഴിപ്പെടുന്നു. നിന്നോടു മാത്രം ഞങ്ങള്‍ സഹായം തേടുന്നു.
\end{malayalam}}
\flushright{\begin{Arabic}
\quranayah[1][6]
\end{Arabic}}
\flushleft{\begin{malayalam}
ഞങ്ങളെ നീ നേര്‍വഴിയിലാക്കേണമേ.
\end{malayalam}}
\flushright{\begin{Arabic}
\quranayah[1][7]
\end{Arabic}}
\flushleft{\begin{malayalam}
നീ അനുഗ്രഹിച്ചവരുടെ വഴിയില്‍. നിന്റെ കോപത്തിന്നിരയായവരുടെയും പിഴച്ചവരുടെയും വഴിയിലല്ല.
\end{malayalam}}
\chapter{\textmalayalam{അല്‍ ബഖറ ( പശു )}}
\begin{Arabic}
\Huge{\centerline{\basmalah}}\end{Arabic}
\flushright{\begin{Arabic}
\quranayah[2][1]
\end{Arabic}}
\flushleft{\begin{malayalam}
അലിഫ് - ലാം - മീം. ‎
\end{malayalam}}
\flushright{\begin{Arabic}
\quranayah[2][2]
\end{Arabic}}
\flushleft{\begin{malayalam}
ഇതാണ് വേദപുസ്തകം. ഇതില്‍ സംശയമില്ല. ‎ഭക്തന്മാര്‍ക്കിതു വഴികാട്ടി. ‎
\end{malayalam}}
\flushright{\begin{Arabic}
\quranayah[2][3]
\end{Arabic}}
\flushleft{\begin{malayalam}
അഭൌതിക സത്യങ്ങളില്‍ വിശ്വസിക്കുന്നവരാണവര്‍. ‎നമസ്കാരം നിഷ്ഠയോടെ നിര്‍വഹിക്കുന്നവരും നാം ‎നല്‍കിയതില്‍ നിന്ന് ചെലവഴിക്കുന്നവരുമാണ്. ‎
\end{malayalam}}
\flushright{\begin{Arabic}
\quranayah[2][4]
\end{Arabic}}
\flushleft{\begin{malayalam}
നിനക്ക് ഇറക്കിയ ഈ വേദപുസ്തകത്തിലും നിന്റെ ‎മുമ്പുള്ളവര്‍ക്ക് ഇറക്കിയവയിലും ‎വിശ്വസിക്കുന്നവരുമാണവര്‍. പരലോകത്തില്‍ ‎അടിയുറച്ച ബോധ്യമുള്ളവരും. ‎
\end{malayalam}}
\flushright{\begin{Arabic}
\quranayah[2][5]
\end{Arabic}}
\flushleft{\begin{malayalam}
അവര്‍ തങ്ങളുടെ നാഥന്റെ നേര്‍വഴിയിലാണ്. വിജയം ‎വരിക്കുന്നവരും അവര്‍ തന്നെ. ‎
\end{malayalam}}
\flushright{\begin{Arabic}
\quranayah[2][6]
\end{Arabic}}
\flushleft{\begin{malayalam}
എന്നാല്‍ സത്യനിഷേധികളോ; അവര്‍ക്കു നീ താക്കീതു ‎നല്‍കുന്നതും നല്‍കാതിരിക്കുന്നതും തുല്യമാണ്. അവര്‍ ‎വിശ്വസിക്കുകയില്ല. ‎
\end{malayalam}}
\flushright{\begin{Arabic}
\quranayah[2][7]
\end{Arabic}}
\flushleft{\begin{malayalam}
അല്ലാഹു അവരുടെ മനസ്സും കാതും അടച്ചു ‎മുദ്രവെച്ചിരിക്കുന്നു. അവരുടെ കണ്ണുകള്‍ക്ക് മൂടിയുണ്ട്. ‎അവര്‍ക്കാണ് കൊടിയ ശിക്ഷ. ‎
\end{malayalam}}
\flushright{\begin{Arabic}
\quranayah[2][8]
\end{Arabic}}
\flushleft{\begin{malayalam}
ചില ആളുകള്‍ അവകാശപ്പെടുന്നു: "അല്ലാഹുവിലും ‎അന്ത്യദിനത്തിലും ഞങ്ങള്‍ വിശ്വസിച്ചിരിക്കുന്നു.” ‎യഥാര്‍ഥത്തിലവര്‍ വിശ്വാസികളേയല്ല. ‎
\end{malayalam}}
\flushright{\begin{Arabic}
\quranayah[2][9]
\end{Arabic}}
\flushleft{\begin{malayalam}
അല്ലാഹുവിനെയും വിശ്വാസികളെയും ‎വഞ്ചിക്കുകയാണവര്‍. എന്നാല്‍ ‎തങ്ങളെത്തന്നെയാണവര്‍ വഞ്ചിക്കുന്നത്; മറ്റാരെയുമല്ല. ‎അവരത് അറിയുന്നില്ലെന്നുമാത്രം. ‎
\end{malayalam}}
\flushright{\begin{Arabic}
\quranayah[2][10]
\end{Arabic}}
\flushleft{\begin{malayalam}
അവരുടെ മനസ്സുകളില്‍ രോഗമുണ്ട്. അല്ലാഹു ആ ‎രോഗം വര്‍ധിപ്പിച്ചു. ഇനി അവര്‍ക്കുള്ളത് നോവേറിയ ‎ശിക്ഷയാണ്; അവര്‍ കള്ളം ‎പറഞ്ഞുകൊണ്ടിരുന്നതിനാലാണത്. ‎
\end{malayalam}}
\flushright{\begin{Arabic}
\quranayah[2][11]
\end{Arabic}}
\flushleft{\begin{malayalam}
‎“നിങ്ങള്‍ ഭൂമിയില്‍ കുഴപ്പമുണ്ടാക്കരുതെ"ന്ന് ‎ആവശ്യപ്പെട്ടാല്‍ അവര്‍ പറയും: "ഞങ്ങള്‍ നന്മ ‎ചെയ്യുന്നവര്‍ മാത്രമാകുന്നു.“ ‎
\end{malayalam}}
\flushright{\begin{Arabic}
\quranayah[2][12]
\end{Arabic}}
\flushleft{\begin{malayalam}
അറിയുക; അവര്‍ തന്നെയാണ് കുഴപ്പക്കാര്‍. പക്ഷേ, ‎അവരതറിയുന്നില്ല. ‎
\end{malayalam}}
\flushright{\begin{Arabic}
\quranayah[2][13]
\end{Arabic}}
\flushleft{\begin{malayalam}
‎“മറ്റുള്ളവര്‍ വിശ്വസിച്ചപോലെ നിങ്ങളും വിശ്വസിക്കുക" ‎എന്ന് ആവശ്യപ്പെട്ടാല്‍ അവര്‍ ചോദിക്കും: "വിഡ്ഢികള്‍ ‎വിശ്വസിച്ചപോലെ ഞങ്ങളും വിശ്വസിക്കണമെന്നോ?" ‎എന്നാല്‍ അറിയുക: അവര്‍ തന്നെയാണ് വിഡ്ഢികള്‍. ‎പക്ഷേ, അവരതറിയുന്നില്ല. ‎
\end{malayalam}}
\flushright{\begin{Arabic}
\quranayah[2][14]
\end{Arabic}}
\flushleft{\begin{malayalam}
സത്യവിശ്വാസികളെ കണ്ടുമുട്ടുമ്പോള്‍ അവര്‍ പറയും: ‎‎"ഞങ്ങളും വിശ്വസിച്ചിരിക്കുന്നു." അവരും അവരുടെ ‎പിശാചുക്കളും മാത്രമായാല്‍ അവര്‍ പറയും: "ഞങ്ങള്‍ ‎നിങ്ങളോടൊപ്പം തന്നെയാണ്. ഞങ്ങള്‍ അവരെ ‎പരിഹസിക്കുക മാത്രമായിരുന്നു." ‎
\end{malayalam}}
\flushright{\begin{Arabic}
\quranayah[2][15]
\end{Arabic}}
\flushleft{\begin{malayalam}
അല്ലാഹു അവരെ പരിഹാസ്യരാക്കുകയും ‎അതിക്രമങ്ങളില്‍ അന്ധരായി അലയാന്‍ ‎വിട്ടിരിക്കുകയുമാണ്. ‎
\end{malayalam}}
\flushright{\begin{Arabic}
\quranayah[2][16]
\end{Arabic}}
\flushleft{\begin{malayalam}
അവരാണ് നേര്‍വഴി വിറ്റ് വഴികേട് ‎വിലയ്ക്കെടുത്തവര്‍. അവരുടെ കച്ചവടം ഒട്ടും ‎ലാഭകരമല്ല. അവര്‍ക്കു നേര്‍മാര്‍ഗം നഷ്ടപ്പെട്ടിരിക്കുന്നു. ‎
\end{malayalam}}
\flushright{\begin{Arabic}
\quranayah[2][17]
\end{Arabic}}
\flushleft{\begin{malayalam}
അവരുടെ ഉപമ ഇവ്വിധമാകുന്നു: ഒരാള്‍ തീകൊളുത്തി. ‎ചുറ്റും പ്രകാശം പരന്നപ്പോള്‍ അല്ലാഹു അവരുടെ ‎വെളിച്ചം അണച്ചു. എന്നിട്ടവരെ ഒന്നും കാണാത്തവരായി ‎കൂരിരുളിലുപേക്ഷിച്ചു. ‎
\end{malayalam}}
\flushright{\begin{Arabic}
\quranayah[2][18]
\end{Arabic}}
\flushleft{\begin{malayalam}
ബധിരരും മൂകരും കുരുടരുമാണവര്‍. ‎അതിനാലവരൊരിക്കലും നേര്‍വഴിയിലേക്കു ‎തിരിച്ചുവരില്ല. ‎
\end{malayalam}}
\flushright{\begin{Arabic}
\quranayah[2][19]
\end{Arabic}}
\flushleft{\begin{malayalam}
അല്ലെങ്കില്‍ മറ്റൊരുപമ: മാനത്തുനിന്നുള്ള പെരുമഴ. ‎അതില്‍ ഇരുളും ഇടിമുഴക്കവും മിന്നല്‍പ്പിണരുമുണ്ട്. ‎മേഘഗര്‍ജനം കേട്ട് മരണഭീതിയാല്‍ അവര്‍ ചെവികളില്‍ ‎വിരലുകള്‍ തിരുകുന്നു. അല്ലാഹു സത്യനിഷേധികളെ ‎സദാ വലയം ചെയ്യുന്നവനത്രെ. ‎
\end{malayalam}}
\flushright{\begin{Arabic}
\quranayah[2][20]
\end{Arabic}}
\flushleft{\begin{malayalam}
മിന്നല്‍പ്പിണരുകള്‍ അവരുടെ കാഴ്ചയെ ‎കവര്‍ന്നെടുക്കുന്നു. അതിന്റെ ഇത്തിരിവെട്ടം ‎കിട്ടുമ്പോഴൊക്കെ അവരതിലൂടെ നടക്കും. ‎ഇരുള്‍മൂടിയാലോ അവര്‍ അറച്ചുനില്‍ക്കും. അല്ലാഹു ‎ഇച്ഛിച്ചിരുന്നെങ്കില്‍ അവരുടെ കേള്‍വിയും കാഴ്ചയും ‎അവന്‍ കെടുത്തിക്കളയുമായിരുന്നു. തീര്‍ച്ചയായും ‎അല്ലാഹു എല്ലാറ്റിനും കഴിവുറ്റവന്‍ തന്നെ. ‎
\end{malayalam}}
\flushright{\begin{Arabic}
\quranayah[2][21]
\end{Arabic}}
\flushleft{\begin{malayalam}
ജനങ്ങളേ, നിങ്ങളെയും മുന്‍ഗാമികളെയും സൃഷ്ടിച്ച ‎നിങ്ങളുടെ നാഥന് വഴിപ്പെടുക. നിങ്ങള്‍ ‎ഭക്തരായിത്തീരാന്‍. ‎
\end{malayalam}}
\flushright{\begin{Arabic}
\quranayah[2][22]
\end{Arabic}}
\flushleft{\begin{malayalam}
അവന്‍ നിങ്ങള്‍ക്കായി ഭൂമിയെ വിരിപ്പാക്കി. ‎ആകാശത്തെ മേലാപ്പാക്കി. മാനത്തുനിന്ന് മഴ വീഴ്ത്തി. ‎അതുവഴി നിങ്ങള്‍ക്കു കഴിക്കാനുള്ള കായ്കനികള്‍ ‎കിളിര്‍പ്പിച്ചുതന്നു. അതിനാല്‍ നിങ്ങള്‍ അല്ലാഹുവിന് ‎സമന്മാരെ സങ്കല്‍പിക്കരുത്. നിങ്ങള്‍ എല്ലാം ‎അറിയുന്നവരായിരിക്കെ. ‎
\end{malayalam}}
\flushright{\begin{Arabic}
\quranayah[2][23]
\end{Arabic}}
\flushleft{\begin{malayalam}
നാം നമ്മുടെ ദാസന് ഇറക്കിക്കൊടുത്ത ഈ വേദം ‎നമ്മുടേതുതന്നെയോ എന്ന് നിങ്ങള്‍ ‎സംശയിക്കുന്നുവെങ്കില്‍ ഇതുപോലുള്ള ‎ഒരധ്യായമെങ്കിലും കൊണ്ടുവരിക. അല്ലാഹുവിനു ‎പുറമെ നിങ്ങള്‍ക്ക് സഹായികളോ സാക്ഷികളോ ‎ഉണ്ടെങ്കില്‍ അവരെയും വിളിച്ചുകൊള്ളുക. നിങ്ങള്‍ ‎സത്യസന്ധരെങ്കില്‍! ‎
\end{malayalam}}
\flushright{\begin{Arabic}
\quranayah[2][24]
\end{Arabic}}
\flushleft{\begin{malayalam}
നിങ്ങള്‍ക്കതു ചെയ്യാന്‍ സാധ്യമല്ലെങ്കില്‍ -നിങ്ങള്‍ക്കതു ‎സാധ്യമല്ല; തീര്‍ച്ച- നിങ്ങള്‍ നരകത്തീയിനെ ‎കാത്തുകൊള്ളുക. മനുഷ്യരും കല്ലുകളും ഇന്ധനമായ ‎നരകാഗ്നിയെ. സത്യനിഷേധികള്‍ക്കായി ‎തയ്യാറാക്കപ്പെട്ടതാണത്. ‎
\end{malayalam}}
\flushright{\begin{Arabic}
\quranayah[2][25]
\end{Arabic}}
\flushleft{\begin{malayalam}
സത്യവിശ്വാസം സ്വീകരിക്കുകയും ‎സല്‍ക്കര്‍മങ്ങളനുഷ്ഠിക്കുകയും ചെയ്യുന്നവരെ ‎ശുഭവാര്‍ത്ത അറിയിക്കുക: അവര്‍ക്ക് താഴ്ഭാഗത്തൂടെ ‎അരുവികളൊഴുകുന്ന സ്വര്‍ഗീയാരാമങ്ങളുണ്ട്. അതിലെ ‎കനികള്‍ ആഹാരമായി ലഭിക്കുമ്പോഴൊക്കെ അവര്‍ ‎പറയും: "ഞങ്ങള്‍ക്കു നേരത്തെ നല്‍കിയതു ‎തന്നെയാണല്ലോ ഇതും." സത്യമോ, സമാനതയുള്ളത് ‎അവര്‍ക്ക് സമ്മാനിക്കപ്പെടുകയാണ്. അവര്‍ക്കവിടെ ‎വിശുദ്ധരായ ഇണകളുണ്ട്. അവരവിടെ ‎സ്ഥിരവാസികളായിരിക്കും. ‎
\end{malayalam}}
\flushright{\begin{Arabic}
\quranayah[2][26]
\end{Arabic}}
\flushleft{\begin{malayalam}
കൊതുകിനെയോ അതിലും നിസ്സാരമായതിനെപ്പോലുമോ ‎ഉപമയാക്കാന്‍ അല്ലാഹുവിന് ഒട്ടും സങ്കോചമില്ല. ‎അപ്പോള്‍ വിശ്വാസികള്‍ അതു തങ്ങളുടെ നാഥന്റെ ‎സത്യവചനമാണെന്നു തിരിച്ചറിയുന്നു. എന്നാല്‍ ‎സത്യനിഷേധികള്‍ ചോദിക്കുന്നു: "ഈ ഉപമ കൊണ്ട് ‎അല്ലാഹു എന്താണ് ഉദ്ദേശിക്കുന്നത്?“ അങ്ങനെ ഈ ഉപമ ‎കൊണ്ട് അവന്‍ ചിലരെ വഴിതെറ്റിക്കുന്നു. പലരേയും ‎നേര്‍വഴിയിലാക്കുന്നു. എന്നാല്‍ ധിക്കാരികളെ മാത്രമേ ‎അവന്‍ വഴിതെറ്റിക്കുന്നുള്ളൂ. ‎
\end{malayalam}}
\flushright{\begin{Arabic}
\quranayah[2][27]
\end{Arabic}}
\flushleft{\begin{malayalam}
അല്ലാഹുവുമായി കരാര്‍ ഉറപ്പിച്ചശേഷം അതു ‎ലംഘിക്കുന്നവരാണവര്‍; അല്ലാഹു കൂട്ടിയിണക്കാന്‍ ‎കല്‍പിച്ചതിനെ വേര്‍പെടുത്തുന്നവര്‍; ഭൂമിയില്‍ ‎കുഴപ്പമുണ്ടാക്കുന്നവര്‍. നഷ്ടം പറ്റിയവരും അവര്‍തന്നെ. ‎
\end{malayalam}}
\flushright{\begin{Arabic}
\quranayah[2][28]
\end{Arabic}}
\flushleft{\begin{malayalam}
എങ്ങനെ നിങ്ങള്‍ അല്ലാഹുവിനെ നിഷേധിക്കും? ‎നിങ്ങള്‍ക്ക് ജീവനില്ലായിരുന്നു. പിന്നെ അവന്‍ നിങ്ങള്‍ക്കു ‎ജീവനേകി. അവന്‍ തന്നെ നിങ്ങളെ മരിപ്പിക്കും. വീണ്ടും ‎ജീവിപ്പിക്കും. അവസാനം അവങ്കലേക്കുതന്നെ ‎നിങ്ങളെല്ലാം തിരിച്ചുചെല്ലും. ‎
\end{malayalam}}
\flushright{\begin{Arabic}
\quranayah[2][29]
\end{Arabic}}
\flushleft{\begin{malayalam}
അവനാണ് ഭൂമിയിലുള്ളതെല്ലാം നിങ്ങള്‍ക്കായി ‎സൃഷ്ടിച്ചത്. കൂടാതെ ഏഴാകാശങ്ങളെ ക്രമീകരിച്ച് ‎ഉപരിലോകത്തെ അവന്‍ സംവിധാനിച്ചു. എല്ലാ ‎കാര്യങ്ങളും അറിയുന്നവനാണവന്‍. ‎
\end{malayalam}}
\flushright{\begin{Arabic}
\quranayah[2][30]
\end{Arabic}}
\flushleft{\begin{malayalam}
നിന്റെ നാഥന്‍ മലക്കുകളോടു പറഞ്ഞ സന്ദര്‍ഭം: ‎‎"ഭൂമിയില്‍ ഞാനൊരു പ്രതിനിധിയെ ‎നിയോഗിക്കുകയാണ്." അവരന്വേഷിച്ചു: "ഭൂമിയില്‍ ‎കുഴപ്പമുണ്ടാക്കുകയും ചോര ചിന്തുകയും ‎ചെയ്യുന്നവരെയോ നീ നിയോഗിക്കുന്നത്? ഞങ്ങളോ ‎നിന്റെ മഹത്വം കീര്‍ത്തിക്കുന്നു. നിന്റെ വിശുദ്ധി ‎വാഴ്ത്തുകയും ചെയ്യുന്നു." അല്ലാഹു പറഞ്ഞു: ‎‎"നിങ്ങളറിയാത്തവയും ഞാനറിയുന്നു." ‎
\end{malayalam}}
\flushright{\begin{Arabic}
\quranayah[2][31]
\end{Arabic}}
\flushleft{\begin{malayalam}
അല്ലാഹു ആദമിനെ എല്ലാ വസ്തുക്കളുടെയും പേരുകള്‍ ‎പഠിപ്പിച്ചു. പിന്നീട് അവയെ മലക്കുകളുടെ മുന്നില്‍ ‎പ്രദര്‍ശിപ്പിച്ച് അവന്‍ കല്‍പിച്ചു: "നിങ്ങള്‍ ഇവയുടെ ‎പേരുകള്‍ പറയുക, നിങ്ങള്‍ സത്യം പറയുന്നവരെങ്കില്‍?" ‎
\end{malayalam}}
\flushright{\begin{Arabic}
\quranayah[2][32]
\end{Arabic}}
\flushleft{\begin{malayalam}
അവര്‍ പറഞ്ഞു: "കുറ്റമറ്റവന്‍ നീ മാത്രം. നീ ‎പഠിപ്പിച്ചുതന്നതല്ലാതൊന്നും ഞങ്ങള്‍ക്കറിയില്ല. എല്ലാം ‎അറിയുന്നവനും യുക്തിമാനും നീ മാത്രം." ‎
\end{malayalam}}
\flushright{\begin{Arabic}
\quranayah[2][33]
\end{Arabic}}
\flushleft{\begin{malayalam}
അല്ലാഹു പറഞ്ഞു: "ആദം! ഇവയുടെ പേരുകള്‍ അവരെ ‎അറിയിക്കുക." അങ്ങനെ ആദം അവരെ, ആ ‎പേരുകളറിയിച്ചു. അപ്പോള്‍ അല്ലാഹു ചോദിച്ചു: ‎‎"ആകാശഭൂമികളില്‍ ഒളിഞ്ഞുകിടക്കുന്നതൊക്കെയും ‎ഞാനറിയുന്നുവെന്ന് നിങ്ങളോട് പറഞ്ഞിരുന്നില്ലേ? ‎നിങ്ങള്‍ തെളിയിച്ചു കാണിക്കുന്നവയും ‎ഒളിപ്പിച്ചുവെക്കുന്നവയും ഞാനറിയുന്നുവെന്നും?" ‎
\end{malayalam}}
\flushright{\begin{Arabic}
\quranayah[2][34]
\end{Arabic}}
\flushleft{\begin{malayalam}
നാം മലക്കുകളോടു പറഞ്ഞ സന്ദര്‍ഭം: "നിങ്ങള്‍ ആദമിന് ‎സാഷ്ടാംഗം ചെയ്യുക." അവരൊക്കെയും സാഷ്ടാംഗം ‎പ്രണമിച്ചു; ഇബ്ലീസൊഴികെ. അവന്‍ വിസമ്മതിച്ചു; ‎അഹങ്കരിക്കുകയും ചെയ്തു. അങ്ങനെ അവന്‍ ‎സത്യനിഷേധികളില്‍ പെട്ടവനായി. ‎
\end{malayalam}}
\flushright{\begin{Arabic}
\quranayah[2][35]
\end{Arabic}}
\flushleft{\begin{malayalam}
നാം പറഞ്ഞു: "ആദമേ, നീയും നിന്റെ ഇണയും ‎സ്വര്‍ഗത്തില്‍ താമസിക്കുക. വിശിഷ്ട വിഭവങ്ങള്‍ ‎വേണ്ടുവോളം തിന്നുകൊള്ളുക. പക്ഷേ, ഈ ‎വൃക്ഷത്തോടടുക്കരുത്. അടുത്താല്‍ നിങ്ങളിരുവരും ‎അതിക്രമികളായിത്തീരും.“ ‎
\end{malayalam}}
\flushright{\begin{Arabic}
\quranayah[2][36]
\end{Arabic}}
\flushleft{\begin{malayalam}
എന്നാല്‍ പിശാച് അവരിരുവരെയും അതില്‍നിന്ന് ‎തെറ്റിച്ചു. അവരിരുവരെയും ‎അവരുണ്ടായിരുന്നിടത്തുനിന്നു പുറത്താക്കി. അപ്പോള്‍ ‎നാം കല്‍പിച്ചു: "ഇവിടെ നിന്നിറങ്ങിപ്പോവുക. പരസ്പര ‎ശത്രുതയോടെ വര്‍ത്തിക്കും നിങ്ങള്‍. ഭൂമിയില്‍ ‎നിങ്ങള്‍ക്ക് കുറച്ചുകാലം കഴിയാന്‍ ഇടമുണ്ട്; കഴിക്കാന്‍ ‎വിഭവങ്ങളും." ‎
\end{malayalam}}
\flushright{\begin{Arabic}
\quranayah[2][37]
\end{Arabic}}
\flushleft{\begin{malayalam}
അപ്പോള്‍ ആദം തന്റെ നാഥനില്‍ നിന്ന് ചില വചനങ്ങള്‍ ‎അഭ്യസിച്ചു. അതുവഴി പശ്ചാത്തപിച്ചു. അല്ലാഹു ‎അതംഗീകരിച്ചു. തീര്‍ച്ചയായും ഏറെ മാപ്പരുളുന്നവനും ‎ദയാപരനുമാണവന്‍. ‎
\end{malayalam}}
\flushright{\begin{Arabic}
\quranayah[2][38]
\end{Arabic}}
\flushleft{\begin{malayalam}
നാം കല്‍പിച്ചു: "എല്ലാവരും ഇവിടം വിട്ട് പോകണം. ‎എന്റെ മാര്‍ഗദര്‍ശനം നിങ്ങള്‍ക്ക് അവിടെ വന്നെത്തും. ‎സംശയമില്ല; എന്റെ മാര്‍ഗം പിന്തുടരുന്നവര്‍ ‎നിര്‍ഭയരായിരിക്കും; ദുഃഖമില്ലാത്തവരും". ‎
\end{malayalam}}
\flushright{\begin{Arabic}
\quranayah[2][39]
\end{Arabic}}
\flushleft{\begin{malayalam}
‎"എന്നാല്‍ അതിനെ അവിശ്വസിക്കുകയും നമ്മുടെ ‎തെളിവുകളെ തള്ളിപ്പറയുകയും ചെയ്യുന്നവരോ, ‎അവരാകുന്നു നരകാവകാശികള്‍. അവരതില്‍ ‎സ്ഥിരവാസികളായിരിക്കും." ‎
\end{malayalam}}
\flushright{\begin{Arabic}
\quranayah[2][40]
\end{Arabic}}
\flushleft{\begin{malayalam}
ഇസ്രയേല്‍ മക്കളേ, ഞാന്‍ നിങ്ങള്‍ക്കേകിയ അനുഗ്രഹം ‎ഓര്‍ത്തുനോക്കൂ. നിങ്ങള്‍ എനിക്കുതന്ന വാഗ്ദാനം ‎പൂര്‍ത്തീകരിക്കൂ. നിങ്ങളോടുള്ള പ്രതിജ്ഞ ഞാനും ‎നിറവേറ്റാം. നിങ്ങള്‍ എന്നെ മാത്രം ഭയപ്പെടുക. ‎
\end{malayalam}}
\flushright{\begin{Arabic}
\quranayah[2][41]
\end{Arabic}}
\flushleft{\begin{malayalam}
ഞാന്‍ ഇറക്കിയ വേദത്തില്‍ വിശ്വസിക്കുക. അതു ‎നിങ്ങളുടെ വശമുള്ള വേദങ്ങളെ ശരിവെക്കുന്നതാണ്. ‎അതിനെ ആദ്യം നിഷേധിക്കുന്നവര്‍ നിങ്ങളാകരുത്. ‎എന്റെ വചനങ്ങള്‍ തുച്ഛ വിലയ്ക്കു വില്‍ക്കരുത്. ‎എന്നോടുമാത്രം ഭക്തി പുലര്‍ത്തുക. ‎
\end{malayalam}}
\flushright{\begin{Arabic}
\quranayah[2][42]
\end{Arabic}}
\flushleft{\begin{malayalam}
സത്യവും അസത്യവും കൂട്ടിക്കലര്‍ത്തി ‎ആശയക്കുഴപ്പമുണ്ടാക്കരുത്. ബോധപൂര്‍വം സത്യം ‎മറച്ചുവെക്കരുത്. ‎
\end{malayalam}}
\flushright{\begin{Arabic}
\quranayah[2][43]
\end{Arabic}}
\flushleft{\begin{malayalam}
നമസ്കാരം നിഷ്ഠയോടെ നിര്‍വഹിക്കുക, സകാത്ത് ‎നല്‍കുക, നമിക്കുന്നവരോടൊപ്പം നമിക്കുക. ‎
\end{malayalam}}
\flushright{\begin{Arabic}
\quranayah[2][44]
\end{Arabic}}
\flushleft{\begin{malayalam}
നിങ്ങള്‍ ജനങ്ങളോട് നന്മ കല്‍പിക്കുകയും സ്വന്തം ‎കാര്യത്തിലത് മറക്കുകയുമാണോ? അതും വേദം ‎ഓതിക്കൊണ്ടിരിക്കെ? നിങ്ങള്‍ ഒട്ടും ‎ആലോചിക്കുന്നില്ലേ? ‎
\end{malayalam}}
\flushright{\begin{Arabic}
\quranayah[2][45]
\end{Arabic}}
\flushleft{\begin{malayalam}
സഹനത്തിലൂടെയും നമസ്കാരത്തിലൂടെയും ‎ദിവ്യസഹായം തേടുക. നമസ്കാരം വലിയ ഭാരം തന്നെ; ‎ഭക്തന്മാര്‍ക്കൊഴികെ. ‎
\end{malayalam}}
\flushright{\begin{Arabic}
\quranayah[2][46]
\end{Arabic}}
\flushleft{\begin{malayalam}
നിശ്ചയമായും തങ്ങളുടെ നാഥനുമായി സന്ധിക്കുമെന്നും; ‎അവസാനം അവനിലേക്കു തിരിച്ചുചെല്ലുമെന്നും ‎അറിയുന്നവരാണവര്‍. ‎
\end{malayalam}}
\flushright{\begin{Arabic}
\quranayah[2][47]
\end{Arabic}}
\flushleft{\begin{malayalam}
ഇസ്രയേല്‍ മക്കളേ, ഞാന്‍ നിങ്ങള്‍ക്കു നല്‍കിയ ‎അനുഗ്രഹങ്ങള്‍ ഓര്‍ക്കുക; നിങ്ങളെ മറ്റാരെക്കാളും ‎ശ്രേഷ്ഠരാക്കിയതും. ‎
\end{malayalam}}
\flushright{\begin{Arabic}
\quranayah[2][48]
\end{Arabic}}
\flushleft{\begin{malayalam}
ആര്‍ക്കും ആരെയും സഹായിക്കാനാവാത്ത; ‎ആരില്‍നിന്നും ശിപാര്‍ശയോ മോചനദ്രവ്യമോ ‎സ്വീകരിക്കാത്ത; കുറ്റവാളികള്‍ക്ക് ഒരുവിധ സഹായവും ‎ലഭിക്കാത്ത ആ ദിന ത്തെ കരുതിയിരിക്കുക. ‎
\end{malayalam}}
\flushright{\begin{Arabic}
\quranayah[2][49]
\end{Arabic}}
\flushleft{\begin{malayalam}
ഫറവോന്റെ ആള്‍ക്കാരില്‍നിന്ന് നിങ്ങളെ നാം രക്ഷിച്ചത് ‎ഓര്‍ക്കുക: ആണ്‍കുട്ടികളെ അറുകൊല ചെയ്തും ‎പെണ്‍കുട്ടികളെ ജീവിക്കാന്‍ വിട്ടും അവന്‍ നിങ്ങളെ ‎കഠിനമായി പീഡിപ്പിക്കുകയായിരുന്നു. അതില്‍ ‎നിങ്ങള്‍ക്കു നിങ്ങളുടെ നാഥനില്‍ നിന്നുള്ള കടുത്ത ‎പരീക്ഷണമുണ്ടായിരുന്നു. ‎
\end{malayalam}}
\flushright{\begin{Arabic}
\quranayah[2][50]
\end{Arabic}}
\flushleft{\begin{malayalam}
ഓര്‍ക്കുക: സമുദ്രം പിളര്‍ത്തി നിങ്ങള്‍ക്കു നാം ‎വഴിയൊരുക്കി. അങ്ങനെ നിങ്ങളെ നാം രക്ഷപ്പെടുത്തി. ‎നിങ്ങള്‍ നോക്കിനില്‍ക്കെ ഫറവോന്റെ ആള്‍ക്കാരെ നാം ‎വെള്ളത്തിലാഴ്ത്തി. ‎
\end{malayalam}}
\flushright{\begin{Arabic}
\quranayah[2][51]
\end{Arabic}}
\flushleft{\begin{malayalam}
ഓര്‍ക്കുക: മൂസാക്കു നാം നാല്‍പത് രാവുകള്‍ അവധി ‎നിശ്ചയിച്ചു. അദ്ദേഹം സ്ഥലം വിട്ടതോടെ നിങ്ങള്‍ ‎പശുക്കുട്ടിയെ ഉണ്ടാക്കി. നിങ്ങള്‍ ‎അതിക്രമികളാവുകയായിരുന്നു. ‎
\end{malayalam}}
\flushright{\begin{Arabic}
\quranayah[2][52]
\end{Arabic}}
\flushleft{\begin{malayalam}
എന്നിട്ടും നാം നിങ്ങള്‍ക്കു പിന്നെയും മാപ്പേകി. നിങ്ങള്‍ ‎നന്ദിയുള്ളവരാകാന്‍. ‎
\end{malayalam}}
\flushright{\begin{Arabic}
\quranayah[2][53]
\end{Arabic}}
\flushleft{\begin{malayalam}
ഓര്‍ക്കുക: മൂസാക്കു നാം വേദം നല്‍കി. ‎സത്യാസത്യങ്ങളെ വേര്‍തിരിച്ചുകാണിക്കുന്ന ‎പ്രമാണവും. അതിലൂടെ നിങ്ങള്‍ നേര്‍വഴിയിലാകാന്‍. ‎
\end{malayalam}}
\flushright{\begin{Arabic}
\quranayah[2][54]
\end{Arabic}}
\flushleft{\begin{malayalam}
ഓര്‍ക്കുക: മൂസ തന്റെ ജനത്തോടോതി: "എന്റെ ജനമേ, ‎പശുക്കിടാവിനെ ഉണ്ടാക്കിവെച്ചതിലൂടെ നിങ്ങള്‍ ‎നിങ്ങളോടുതന്നെ കൊടിയ ക്രൂരത കാണിച്ചിരിക്കുന്നു. ‎അതിനാല്‍ നിങ്ങള്‍ നിങ്ങളുടെ സ്രഷ്ടാവിനോട് ‎പശ്ചാത്തപിക്കുക. നിങ്ങള്‍ നിങ്ങളെത്തന്നെ ഹനിക്കുക. ‎അതാണ് നിങ്ങളുടെ കര്‍ത്താവിങ്കല്‍ നിങ്ങള്‍ക്കുത്തമം." ‎പിന്നീട് അല്ലാഹു നിങ്ങളുടെ പശ്ചാത്താപം സ്വീകരിച്ചു. ‎അവന്‍ പശ്ചാത്താപം സ്വീകരിക്കുന്നവനും ‎ദയാപരനുമല്ലോ. ‎
\end{malayalam}}
\flushright{\begin{Arabic}
\quranayah[2][55]
\end{Arabic}}
\flushleft{\begin{malayalam}
ഓര്‍ക്കുക: നിങ്ങള്‍ പറഞ്ഞ സന്ദര്‍ഭം: "മൂസാ, ദൈവത്തെ ‎നേരില്‍ പ്രകടമായി കാണാതെ ഞങ്ങള്‍ നിന്നില്‍ ‎വിശ്വസിക്കുകയില്ല." അപ്പോള്‍ ഒരു ഘോരഗര്‍ജനം ‎നിങ്ങളെ പിടികൂടി; നിങ്ങള്‍ നോക്കിനില്‍ക്കെ. ‎
\end{malayalam}}
\flushright{\begin{Arabic}
\quranayah[2][56]
\end{Arabic}}
\flushleft{\begin{malayalam}
പിന്നെ മരണശേഷം നിങ്ങളെ നാം ‎ഉയിര്‍ത്തെഴുന്നേല്‍പിച്ചു. നിങ്ങള്‍ നന്ദിയുള്ളവരാകാന്‍. ‎
\end{malayalam}}
\flushright{\begin{Arabic}
\quranayah[2][57]
\end{Arabic}}
\flushleft{\begin{malayalam}
നിങ്ങള്‍ക്കു നാം മേഘത്തണലൊരുക്കി. മന്നും ‎സല്‍വായും ഇറക്കിത്തന്നു. നിങ്ങളോടു പറഞ്ഞു: ‎‎"നിങ്ങള്‍ക്കു നാമേകിയ വിശിഷ്ട വിഭവങ്ങള്‍ ഭക്ഷിക്കുക." ‎അവര്‍ ദ്രോഹിച്ചത് നമ്മെയല്ല. പിന്നെയോ ‎തങ്ങള്‍ക്കുതന്നെയാണവര്‍ ദ്രോഹം വരുത്തിയത്. ‎
\end{malayalam}}
\flushright{\begin{Arabic}
\quranayah[2][58]
\end{Arabic}}
\flushleft{\begin{malayalam}
ഓര്‍ക്കുക: നാം നിങ്ങളോടു പറഞ്ഞു: "നിങ്ങള്‍ ഈ പട്ടണ ‎ത്തില്‍ പ്രവേശിക്കുക. അവിടെനിന്ന് ആവശ്യമുള്ളത്ര ‎വിശിഷ്ട വിഭവങ്ങള്‍ തിന്നുകൊള്ളുക. എന്നാല്‍ ‎നഗരകവാടം കടക്കുന്നത് വണക്കത്തോടെയാവണം. ‎പാപമോചനവചനം ഉരുവിട്ടുകൊണ്ടും. എങ്കില്‍ നാം ‎നിങ്ങള്‍ക്ക് പാപങ്ങള്‍ പൊറുത്തുതരും. സുകൃതികള്‍ക്ക് ‎അനുഗ്രഹങ്ങള്‍ വര്‍ധിപ്പിച്ചുതരും." ‎
\end{malayalam}}
\flushright{\begin{Arabic}
\quranayah[2][59]
\end{Arabic}}
\flushleft{\begin{malayalam}
എന്നാല്‍ ആ അക്രമികള്‍, തങ്ങളോടു പറഞ്ഞതിനെ മാറ്റി ‎മറ്റൊന്ന് സ്വീകരിച്ചു. അതിനാല്‍ ആ അക്രമികള്‍ക്കുമേല്‍ ‎നാം മുകളില്‍നിന്ന് ശിക്ഷയിറക്കി. അവര്‍ അധര്‍മം ‎പ്രവര്‍ത്തിച്ചതിനാല്‍. ‎
\end{malayalam}}
\flushright{\begin{Arabic}
\quranayah[2][60]
\end{Arabic}}
\flushleft{\begin{malayalam}
ഓര്‍ക്കുക: മൂസ തന്റെ ജനതക്കുവേണ്ടി കുടിനീരുതേടി. ‎നാം കല്‍പിച്ചു: "നീ നിന്റെ വടികൊണ്ട് ‎പാറമേലടിക്കുക." അങ്ങനെ അതില്‍നിന്ന് പന്ത്രണ്ട് ‎ഉറവകള്‍ പൊട്ടിയൊഴുകി. എല്ലാ വിഭാഗം ജനങ്ങളും ‎തങ്ങള്‍ കുടിവെള്ളമെടുക്കേണ്ടിടം തിരിച്ചറിഞ്ഞു. നാം ‎നിര്‍ദേശിച്ചു: "അല്ലാഹു നല്‍കിയ വിഭവങ്ങളില്‍നിന്ന് ‎തിന്നുകയും കുടിക്കുകയും ചെയ്യുക. ഭൂമിയില്‍ ‎നാശകാരികളായിക്കഴിയരുത്." ‎
\end{malayalam}}
\flushright{\begin{Arabic}
\quranayah[2][61]
\end{Arabic}}
\flushleft{\begin{malayalam}
നിങ്ങള്‍ പറഞ്ഞതോര്‍ക്കുക: "ഓ മൂസാ, ഒരേതരം ‎അന്നംതന്നെ തിന്നു സഹിക്കാന്‍ ഞങ്ങള്‍ക്കാവില്ല. ‎അതിനാല്‍ താങ്കള്‍ താങ്കളുടെ നാഥനോട് പ്രാര്‍ഥിക്കുക: ‎അവന്‍ ഞങ്ങള്‍ക്ക് മണ്ണില്‍ മുളച്ചുണ്ടാകുന്ന ചീര, ‎കക്കിരി, ഗോതമ്പ്, പയര്‍, ഉള്ളി മുതലായവ ‎ഉത്പാദിപ്പിച്ചുതരട്ടെ." മൂസ ചോദിച്ചു: "വിശിഷ്ട ‎വിഭവങ്ങള്‍ക്കുപകരം താണതരം സാധനങ്ങളാണോ ‎നിങ്ങള്‍ തേടുന്നത്? എങ്കില്‍ നിങ്ങള്‍ ഏതെങ്കിലും ‎പട്ടണത്തില്‍ പോവുക. നിങ്ങള്‍ തേടുന്നതൊക്കെ ‎നിങ്ങള്‍ക്കവിടെ കിട്ടും." അങ്ങനെ അവര്‍ നിന്ദ്യതയിലും ‎ദൈന്യതയിലും അകപ്പെട്ടു. ദൈവകോപത്തിനിരയായി. ‎അവര്‍ അല്ലാഹുവിന്റെ തെളിവുകളെ ‎തള്ളിപ്പറഞ്ഞതിനാലും പ്രവാചകന്മാരെ അന്യായമായി ‎കൊന്നതിനാലുമാണത്. ധിക്കാരം കാട്ടുകയും പരിധിവിട്ട് ‎പ്രവര്‍ത്തിക്കുകയും ചെയ്തതിനാലും. ‎
\end{malayalam}}
\flushright{\begin{Arabic}
\quranayah[2][62]
\end{Arabic}}
\flushleft{\begin{malayalam}
ഈ ദൈവദൂതനില്‍ വിശ്വസിച്ചവരോ യഹൂദരോ ‎ക്രൈസ്തവരോ സാബിഉകളോ ആരുമാവട്ടെ, ‎അല്ലാഹുവിലും അന്ത്യദിനത്തിലും വിശ്വസിക്കുകയും ‎സല്‍ക്കര്‍മങ്ങള്‍ പ്രവര്‍ത്തിക്കുകയും ചെയ്യുന്നവര്‍ക്ക് ‎അവരുടെ നാഥന്റെ അടുക്കല്‍ അര്‍ഹമായ ‎പ്രതിഫലമുണ്ട്. അവര്‍ ഭയപ്പെടേണ്ടതില്ല. ‎ദുഃഖിക്കേണ്ടതുമില്ല. ‎
\end{malayalam}}
\flushright{\begin{Arabic}
\quranayah[2][63]
\end{Arabic}}
\flushleft{\begin{malayalam}
ഓര്‍ക്കുക: നിങ്ങളോടു നാം കരാര്‍ വാങ്ങി. ‎നിങ്ങള്‍ക്കുമീതെ മലയെ ഉയര്‍ത്തുകയും ചെയ്തു. നാം ‎നിങ്ങള്‍ക്കു നല്‍കിയ വേദത്തെ ബലമായി ‎മുറുകെപ്പിടിക്കാന്‍ നിര്‍ദേശിച്ചു. അതിലെ നിര്‍ദേശങ്ങള്‍ ‎ഓര്‍ക്കാനും. നിങ്ങള്‍ ഭക്തരാകാന്‍. ‎
\end{malayalam}}
\flushright{\begin{Arabic}
\quranayah[2][64]
\end{Arabic}}
\flushleft{\begin{malayalam}
എന്നാല്‍ പിന്നെയും നിങ്ങള്‍ പിന്തിരിഞ്ഞു പോയി. ‎നിങ്ങള്‍ക്ക് അല്ലാഹുവിന്റെ അനുഗ്രഹവും കാരുണ്യവും ‎ഇല്ലായിരുന്നെങ്കില്‍ നിങ്ങള്‍ നഷ്ടം ‎പറ്റിയവരാകുമായിരുന്നു.‎
\end{malayalam}}
\flushright{\begin{Arabic}
\quranayah[2][65]
\end{Arabic}}
\flushleft{\begin{malayalam}
സാബത്ത്നാളി ല്‍ നിങ്ങളിലെ അതിക്രമം കാണിച്ചവരെ ‎നിങ്ങള്‍ക്ക് നന്നായറിയാമല്ലോ. അവരോട് നാം വിധിച്ചു: ‎‎"നിങ്ങള്‍ നിന്ദ്യരായ കുരങ്ങുകളാവുക." ‎
\end{malayalam}}
\flushright{\begin{Arabic}
\quranayah[2][66]
\end{Arabic}}
\flushleft{\begin{malayalam}
അങ്ങനെ ആ സംഭവത്തെ നാം അക്കാലക്കാര്‍ക്കും ‎പില്‍ക്കാലക്കാര്‍ക്കും ഗുണപാഠമാക്കി. ഭക്തന്മാര്‍ക്ക് ‎സദുപദേശവും. ‎
\end{malayalam}}
\flushright{\begin{Arabic}
\quranayah[2][67]
\end{Arabic}}
\flushleft{\begin{malayalam}
ഓര്‍ക്കുക: മൂസ തന്റെ ജനത്തോടു പറഞ്ഞു: "അല്ലാഹു ‎നിങ്ങളോട് ഒരു പശുവെ അറുക്കാന്‍ കല്‍പിച്ചിരിക്കുന്നു." ‎അവര്‍ ചോദിച്ചു: "നീ ഞങ്ങളെ പരിഹസിക്കുകയാണോ?" ‎മൂസ പറഞ്ഞു: "അവിവേകികളില്‍ പെടാതിരിക്കാന്‍ ‎ഞാന്‍ അല്ലാഹുവില്‍ അഭയം തേടുന്നു." ‎
\end{malayalam}}
\flushright{\begin{Arabic}
\quranayah[2][68]
\end{Arabic}}
\flushleft{\begin{malayalam}
അവര്‍ പറഞ്ഞു: "അത് ഏതിനമായിരിക്കണമെന്ന് ‎ഞങ്ങള്‍ക്കുവേണ്ടി താങ്കള്‍ താങ്കളുടെ നാഥനോട് ‎അന്വേഷിക്കുക." മൂസ പറഞ്ഞു: "അല്ലാഹു ‎അറിയിക്കുന്നു: “ആ പശു പ്രായം കുറഞ്ഞതോ ‎കൂടിയതോ ആവരുത്. വയസ്സൊത്തതായിരിക്കണം." ‎അതിനാല്‍ കല്‍പന പാലിക്കുക." ‎
\end{malayalam}}
\flushright{\begin{Arabic}
\quranayah[2][69]
\end{Arabic}}
\flushleft{\begin{malayalam}
അവര്‍ പറഞ്ഞു: "താങ്കള്‍ താങ്കളുടെ നാഥനോട് ‎ഞങ്ങള്‍ക്കുവേണ്ടി അന്വേഷിക്കുക, അതിന്റെ നിറം ‎ഏതായിരിക്കണമെന്ന്." മൂസ പറഞ്ഞു: "കാണികളില്‍ ‎കൌതുകമുണര്‍ത്തുന്ന തെളിഞ്ഞ മഞ്ഞനിറമുള്ള ‎പശുവായിരിക്കണമെന്ന് അല്ലാഹു ‎നിര്‍ദേശിച്ചിരിക്കുന്നു." ‎
\end{malayalam}}
\flushright{\begin{Arabic}
\quranayah[2][70]
\end{Arabic}}
\flushleft{\begin{malayalam}
അവര്‍ പറഞ്ഞു: "അത് ഏതു തരത്തില്‍ പെട്ടതാണെന്ന് ‎ഞങ്ങള്‍ക്കു വിശദീകരിച്ചുതരാന്‍ നീ നിന്റെ ‎നാഥനോടപേക്ഷിക്കുക. പശുക്കളെല്ലാം ഏറക്കുറെ ‎ഒരുപോലിരിക്കുന്നതായി ഞങ്ങള്‍ക്കുതോന്നുന്നു. ‎ദൈവമിച്ഛിക്കുന്നുവെങ്കില്‍ തീര്‍ച്ചയായും ഞങ്ങള്‍ ‎അതിനെ കണ്ടെത്തുക തന്നെ ചെയ്യും." ‎
\end{malayalam}}
\flushright{\begin{Arabic}
\quranayah[2][71]
\end{Arabic}}
\flushleft{\begin{malayalam}
മൂസ പറഞ്ഞു: "അല്ലാഹു അറിയിക്കുന്നു: നിലം ‎ഉഴുതാനോ വിള നനയ്ക്കാനോ ഉപയോഗിക്കാത്തതും ‎കലകളില്ലാത്തതും കുറ്റമറ്റതുമായ പശുവായിരിക്കണം ‎അത്." അവര്‍ പറഞ്ഞു: "ശരി, ഇപ്പോഴാണ് നീ ശരിയായ ‎വിവരം തന്നത്." അങ്ങനെ അവരതിനെ അറുത്തു. ‎അവരത് ചെയ്യാന്‍ തയ്യാറാകുമായിരുന്നില്ല. ‎
\end{malayalam}}
\flushright{\begin{Arabic}
\quranayah[2][72]
\end{Arabic}}
\flushleft{\begin{malayalam}
ഓര്‍ക്കുക: നിങ്ങള്‍ ഒരാളെ കൊന്നു. എന്നിട്ട് ‎പരസ്പരാരോപണം നടത്തി കുറ്റത്തില്‍നിന്ന് ‎ഒഴിഞ്ഞുമാറി. എന്നാല്‍ അല്ലാഹു നിങ്ങള്‍ ‎മറച്ചുവെക്കുന്നതിനെ വെളിക്കു കൊണ്ടുവരുന്നവനത്രെ. ‎
\end{malayalam}}
\flushright{\begin{Arabic}
\quranayah[2][73]
\end{Arabic}}
\flushleft{\begin{malayalam}
അപ്പോള്‍ നാം പറഞ്ഞു: "നിങ്ങള്‍ അതിന്റെ ഒരു ‎ഭാഗംകൊണ്ട് ആ ശവശരീരത്തെ അടിക്കുക." അവ്വിധം ‎അല്ലാഹു മരിച്ചവരെ ജീവിപ്പിക്കുന്നു. നിങ്ങള്‍ ‎ചിന്തിക്കാനായി അവന്‍ തന്റെ തെളിവുകള്‍ നിങ്ങള്‍ക്കു ‎കാണിച്ചുതരുന്നു. ‎
\end{malayalam}}
\flushright{\begin{Arabic}
\quranayah[2][74]
\end{Arabic}}
\flushleft{\begin{malayalam}
അതിനുശേഷം പിന്നെയും നിങ്ങളുടെ മനസ്സ് കടുത്തു. ‎അത് പാറപോലെ കഠിനമായി. അല്ല; അതിലും കൂടുതല്‍ ‎കടുത്തു. ചില പാറകളില്‍നിന്ന് ഉറവകള്‍ ‎പൊട്ടിയൊഴുകാറുണ്ട്. ചിലത് പൊട്ടിപ്പിളര്‍ന്ന് വെള്ളം ‎ചുരത്താറുമുണ്ട്. ദൈവഭയത്താല്‍ ‎നിലംപതിക്കുന്നവയുമുണ്ട്. നിങ്ങള്‍ ‎ചെയ്യുന്നതിനെക്കുറിച്ചൊന്നും അല്ലാഹു അശ്രദ്ധനല്ല. ‎
\end{malayalam}}
\flushright{\begin{Arabic}
\quranayah[2][75]
\end{Arabic}}
\flushleft{\begin{malayalam}
വിശ്വസിച്ചവരേ, നിങ്ങളുടെ സന്ദേശം ഈ ജനം ‎സ്വീകരിക്കുമെന്ന് നിങ്ങളിനിയും പ്രതീക്ഷിക്കുന്നുവോ? ‎അവരിലൊരു വിഭാഗം ദൈവവചനം കേള്‍ക്കുന്നു. ‎നന്നായി മനസ്സിലാക്കുന്നു. എന്നിട്ടും ബോധപൂര്‍വം ‎അവരതില്‍ കൃത്രിമം കാണിക്കുന്നു. ‎
\end{malayalam}}
\flushright{\begin{Arabic}
\quranayah[2][76]
\end{Arabic}}
\flushleft{\begin{malayalam}
സത്യവിശ്വാസികളെ കണ്ടുമുട്ടുമ്പോള്‍ അവര്‍ പറയും: ‎‎"ഞങ്ങളും വിശ്വസിച്ചിരിക്കുന്നു." അവര്‍ ‎തനിച്ചാകുമ്പോള്‍ പരസ്പരം പറയും: "അല്ലാഹു ‎നിങ്ങള്‍ക്കു വെളിപ്പെടുത്തിത്തന്ന കാര്യങ്ങള്‍ നിങ്ങള്‍ ‎ഇക്കൂട്ടര്‍ക്ക് പറഞ്ഞുകൊടുക്കുകയോ? അതുവഴി ‎നിങ്ങളുടെ നാഥങ്കല്‍ നിങ്ങള്‍ക്കെതിരെ ന്യായവാദം ‎നടത്താന്‍. നിങ്ങള്‍ തീരെ ആലോചിക്കുന്നില്ലേ?" ‎
\end{malayalam}}
\flushright{\begin{Arabic}
\quranayah[2][77]
\end{Arabic}}
\flushleft{\begin{malayalam}
അവരറിയുന്നില്ലേ: അവര്‍ രഹസ്യമാക്കുന്നതും ‎പരസ്യമാക്കുന്നതും അല്ലാഹുവിനറിയാമെന്ന്. ‎
\end{malayalam}}
\flushright{\begin{Arabic}
\quranayah[2][78]
\end{Arabic}}
\flushleft{\begin{malayalam}
അവരില്‍ ചിലര്‍ നിരക്ഷരരാണ്. വേദഗ്രന്ഥമൊന്നും ‎അവര്‍ക്കറിയില്ല; ചില വ്യാമോഹങ്ങള്‍ ‎വെച്ചുപുലര്‍ത്തുന്നതല്ലാതെ. ഊഹിച്ചെടുക്കുക ‎മാത്രമാണവര്‍ ചെയ്യുന്നത്. ‎
\end{malayalam}}
\flushright{\begin{Arabic}
\quranayah[2][79]
\end{Arabic}}
\flushleft{\begin{malayalam}
അതിനാല്‍ സ്വന്തം കൈകൊണ്ട് പുസ്തകമെഴുതി അത് ‎അല്ലാഹുവില്‍നിന്നുള്ളതാണെന്ന് ‎അവകാശപ്പെടുന്നവര്‍ക്കു നാശം! തുച്ഛമായ ‎കാര്യലാഭങ്ങള്‍ക്കുവേണ്ടിയാണ് അവരതു ചെയ്യുന്നത്. ‎തങ്ങളുടെ കൈകൊണ്ട് എഴുതിയുണ്ടാക്കിയതിനാല്‍ ‎അവര്‍ക്കു നാശം! അവര്‍ സമ്പാദിച്ചതു കാരണവും ‎അവര്‍ക്കു നാശം! ‎
\end{malayalam}}
\flushright{\begin{Arabic}
\quranayah[2][80]
\end{Arabic}}
\flushleft{\begin{malayalam}
അവരവകാശപ്പെടുന്നു: "എണ്ണപ്പെട്ട ഏതാനും ‎നാളുകളല്ലാതെ നരകം ഞങ്ങളെ സ്പര്‍ശിക്കുകയില്ല." ‎ചോദിക്കുക: "നിങ്ങള്‍ അല്ലാഹുവുമായി വല്ല കരാറും ‎ഉണ്ടാക്കിയിട്ടുണ്ടോ? എങ്കില്‍ അല്ലാഹു തന്റെ കരാര്‍ ‎ലംഘിക്കുകയില്ല; തീര്‍ച്ച. അതോ, അല്ലാഹുവിന്റെ ‎പേരില്‍ നിങ്ങള്‍ക്കറിയാത്തത് ആരോപിക്കുകയാണോ?" ‎
\end{malayalam}}
\flushright{\begin{Arabic}
\quranayah[2][81]
\end{Arabic}}
\flushleft{\begin{malayalam}
എന്നാല്‍ അറിയുക: ആര്‍ പാപം പ്രവര്‍ത്തിക്കുകയും ‎പാപച്ചുഴിയിലകപ്പെടുകയും ചെയ്യുന്നുവോ അവരാണ് ‎നരകാവകാശികള്‍. അവരതില്‍ ‎സ്ഥിരവാസികളായിരിക്കും. ‎
\end{malayalam}}
\flushright{\begin{Arabic}
\quranayah[2][82]
\end{Arabic}}
\flushleft{\begin{malayalam}
സത്യവിശ്വാസം സ്വീകരിക്കുകയും സല്‍ക്കര്‍മങ്ങള്‍ ‎പ്രവര്‍ത്തിക്കുകയും ചെയ്യുന്നവര്‍ ആരോ അവരാണ് ‎സ്വര്‍ഗാവകാശികള്‍. അവരതില്‍ ‎നിത്യവാസികളായിരിക്കും. ‎
\end{malayalam}}
\flushright{\begin{Arabic}
\quranayah[2][83]
\end{Arabic}}
\flushleft{\begin{malayalam}
ഓര്‍ക്കുക: ഇസ്രയേല്‍ മക്കളില്‍നിന്ന് നാം ഉറപ്പുവാങ്ങി: ‎അല്ലാഹുവിനല്ലാതെ നിങ്ങള്‍ വഴിപ്പെടരുത്; ‎മാതാപിതാക്കളോടും അടുത്ത ബന്ധുക്കളോടും ‎അനാഥകളോടും അഗതികളോടും നല്ല നിലയില്‍ ‎വര്‍ത്തിക്കണം; ജനങ്ങളോട് നല്ലതു പറയണം; നമസ്കാരം ‎നിഷ്ഠയോടെ നിര്‍വഹിക്കണം; സകാത്ത് നല്‍കണം. ‎പക്ഷേ, പിന്നീട് നിങ്ങള്‍ അവഗണനയോടെ ‎പിന്തിരിഞ്ഞുകളഞ്ഞു; നിങ്ങളില്‍ അല്പം ചിലരൊഴികെ. ‎
\end{malayalam}}
\flushright{\begin{Arabic}
\quranayah[2][84]
\end{Arabic}}
\flushleft{\begin{malayalam}
പരസ്പരം ചോര ചിന്തില്ലെന്നും വീടുകളില്‍നിന്ന് ‎പുറന്തള്ളുകയില്ലെന്നും നാം നിങ്ങളില്‍നിന്ന് ‎ഉറപ്പുവാങ്ങിയതോര്‍ക്കുക. നിങ്ങളത് സ്ഥിരീകരിച്ചു. ‎നിങ്ങളതിന് സാക്ഷികളുമായിരുന്നു. ‎
\end{malayalam}}
\flushright{\begin{Arabic}
\quranayah[2][85]
\end{Arabic}}
\flushleft{\begin{malayalam}
എന്നിട്ടും പിന്നെയുമിതാ നിങ്ങള്‍ സ്വന്തക്കാരെ ‎കൊല്ലുന്നു. സ്വജനങ്ങളിലൊരു വിഭാഗത്തെ അവരുടെ ‎വീടുകളില്‍നിന്ന് ആട്ടിപ്പുറത്താക്കുന്നു. കുറ്റകരമായും ‎ശത്രുതാപരമായും നിങ്ങള്‍ അവര്‍ക്കെതിരെ ‎ഒത്തുചേരുന്നു. അവര്‍ നിങ്ങളുടെ അടുത്ത് ‎യുദ്ധത്തടവുകാരായെത്തിയാല്‍ നിങ്ങളവരോട് ‎മോചനദ്രവ്യം വാങ്ങുന്നു. അവരെ തങ്ങളുടെ വീടുകളില്‍ ‎നിന്ന് പുറന്തള്ളുന്നതുതന്നെ നിങ്ങള്‍ക്കു നിഷിദ്ധമത്രെ. ‎നിങ്ങള്‍ വേദപുസ്തകത്തിലെ ചിലവശങ്ങള്‍ ‎വിശ്വസിക്കുകയും ചിലവശങ്ങള്‍ ‎തള്ളിക്കളയുകയുമാണോ? നിങ്ങളില്‍ അവ്വിധം ‎ചെയ്യുന്നവര്‍ക്കുള്ള പ്രതിഫലം ഐഹികജീവിതത്തില്‍ ‎നിന്ദ്യത മാത്രമായിരിക്കും. ഉയിര്‍ത്തെഴുന്നേല്‍പു നാളില്‍ ‎കൊടിയ ശിക്ഷയിലേക്ക് അവര്‍ തള്ളപ്പെടും. നിങ്ങള്‍ ‎ചെയ്യുന്നതിനെപ്പറ്റി അല്ലാഹു ഒട്ടും അശ്രദ്ധനല്ല. ‎
\end{malayalam}}
\flushright{\begin{Arabic}
\quranayah[2][86]
\end{Arabic}}
\flushleft{\begin{malayalam}
പരലോകത്തിനു പകരം ഇഹ ലോകജീവിതം ‎വാങ്ങിയവരാണവര്‍. അതിനാല്‍ അവര്‍ക്ക് ശിക്ഷയില്‍ ‎ഇളവ് ലഭിക്കുകയില്ല. അവര്‍ക്ക് ഒരുവിധ സഹായവും ‎കിട്ടുകയുമില്ല. ‎
\end{malayalam}}
\flushright{\begin{Arabic}
\quranayah[2][87]
\end{Arabic}}
\flushleft{\begin{malayalam}
നിശ്ചയമായും മൂസാക്കു നാം വേദം നല്‍കി. ‎അദ്ദേഹത്തിനുശേഷം നാം തുടരെത്തുടരെ ദൂതന്മാരെ ‎അയച്ചുകൊണ്ടിരുന്നു. മര്‍യമിന്റെ മകന്‍ ഈസാക്കു ‎നാം വ്യക്തമായ അടയാളങ്ങള്‍ നല്‍കി. ‎പരിശുദ്ധാത്മാവിനാല്‍ അദ്ദേഹത്തെ പ്രബലനാക്കുകയും ‎ചെയ്തു. നിങ്ങളുടെ ഇച്ഛക്കിണങ്ങാത്ത കാര്യങ്ങളുമായി ‎ദൈവദൂതന്‍ നിങ്ങള്‍ക്കിടയില്‍ വന്നപ്പോഴെല്ലാം നിങ്ങള്‍ ‎ഗര്‍വിഷ്ഠരായി ധിക്കരിക്കുകയോ? അവരില്‍ ചിലരെ ‎നിങ്ങള്‍ തള്ളിപ്പറഞ്ഞു. ചിലരെ കൊല്ലുകയും ചെയ്തു. ‎
\end{malayalam}}
\flushright{\begin{Arabic}
\quranayah[2][88]
\end{Arabic}}
\flushleft{\begin{malayalam}
അവര്‍ പറഞ്ഞു: "ഞങ്ങളുടെ മനസ്സുകള്‍ ‎അടഞ്ഞുകിടക്കുകയാണ്." അല്ല; സത്യനിഷേധം കാരണം ‎അല്ലാഹു അവരെ ശപിച്ചിരിക്കയാണ്. അതിനാല്‍ ‎അവരില്‍ അല്‍പം ചിലരേ വിശ്വസിക്കുന്നുള്ളൂ. ‎
\end{malayalam}}
\flushright{\begin{Arabic}
\quranayah[2][89]
\end{Arabic}}
\flushleft{\begin{malayalam}
തങ്ങളുടെ വശമുള്ള വേദത്തെ സത്യപ്പെടുത്തുന്ന ഗ്രന്ഥം ‎ദൈവത്തില്‍നിന്ന് അവര്‍ക്ക് വന്നെത്തി. അവരോ, ‎അതിനുമുമ്പ് അത്തരമൊന്നിലൂടെ അവിശ്വാസികളെ ‎പരാജയപ്പെടുത്താനായി പ്രാര്‍ഥിക്കാറുണ്ടായിരുന്നു. ‎എന്നിട്ടും അവര്‍ക്ക് നന്നായറിയാവുന്ന ആ ഗ്രന്ഥം ‎വന്നെത്തിയപ്പോള്‍ അവരതിനെ തള്ളിപ്പറഞ്ഞു! ‎അതിനാല്‍ ദൈവശാപം ആ സത്യനിഷേധികള്‍ക്കത്രെ. ‎
\end{malayalam}}
\flushright{\begin{Arabic}
\quranayah[2][90]
\end{Arabic}}
\flushleft{\begin{malayalam}
അല്ലാഹു അവതരിപ്പിച്ചതിനെ തള്ളിക്കളഞ്ഞതിലൂടെ ‎അവര്‍ സ്വയംവിറ്റുവാങ്ങിയത് എത്ര ചീത്ത. അതിനവരെ ‎പ്രേരിപ്പിച്ചതോ, ദൈവം തന്റെ ഔദാര്യം തന്റെ ‎ദാസന്മാരില്‍ താനിഷ്ടപ്പെടുന്നവര്‍ക്ക് നല്‍കിയതിലെ ‎അമര്‍ഷവും. അതിനാലവര്‍ കൊടിയ ‎ദൈവികകോപത്തിനിരയായി. സത്യനിഷേധികള്‍ക്ക് ‎ഏറെ നിന്ദ്യമായ ശിക്ഷയാണുള്ളത്. ‎
\end{malayalam}}
\flushright{\begin{Arabic}
\quranayah[2][91]
\end{Arabic}}
\flushleft{\begin{malayalam}
അല്ലാഹു ഇറക്കിത്തന്നതില്‍ വിശ്വസിക്കുക ‎എന്നാവശ്യപ്പെട്ടാല്‍ അവര്‍ പറയും: "ഞങ്ങള്‍ക്ക് ‎ഇറക്കിത്തന്നതില്‍ ഞങ്ങള്‍ വിശ്വസിക്കുന്നു." അതിന് ‎പുറത്തുള്ളതിനെ അവര്‍ തള്ളിക്കളയുന്നു. അത് അവരുടെ ‎വശമുള്ളതിനെ ശരിവെക്കുന്ന ‎സത്യസന്ദേശമായിരുന്നിട്ടും. ചോദിക്കുക: നിങ്ങള്‍ ‎വിശ്വാസികളെങ്കില്‍ പിന്നെ എന്തിനാണ് നിങ്ങള്‍ ‎അല്ലാഹുവിന്റെ പ്രവാചകന്മാരെ ‎കൊന്നുകൊണ്ടിരുന്നത്? ‎
\end{malayalam}}
\flushright{\begin{Arabic}
\quranayah[2][92]
\end{Arabic}}
\flushleft{\begin{malayalam}
വ്യക്തമായ തെളിവോടെ മൂസ നിങ്ങളുടെ അടുക്കല്‍ ‎വന്നു. എന്നിട്ടും പിന്നെയും നിങ്ങള്‍ പശുക്കുട്ടിയെ ‎ദൈവമാക്കി. നിങ്ങള്‍ അതിക്രമം കാട്ടുകയായിരുന്നു. ‎
\end{malayalam}}
\flushright{\begin{Arabic}
\quranayah[2][93]
\end{Arabic}}
\flushleft{\begin{malayalam}
ഓര്‍ക്കുക: നിങ്ങള്‍ക്കു മീതെ പര്‍വതത്തെ ‎ഉയര്‍ത്തിക്കൊണ്ട് നിങ്ങളോടു നാം ഉറപ്പുവാങ്ങി. “നാം ‎നിങ്ങള്‍ക്കു നല്‍കിയത് ശക്തമായി മുറുകെപ്പിടിക്കുക. ‎ശ്രദ്ധയോടെ കേള്‍ക്കുക." അവര്‍ പറഞ്ഞു: “ഞങ്ങള്‍ ‎കേള്‍ക്കുകയും ധിക്കരിക്കുകയും ചെയ്തിരിക്കുന്നു." ‎സത്യനിഷേധം നിമിത്തം പശുഭക്തി അവരുടെ ‎മനസ്സുകളില്‍ അള്ളിപ്പിടിച്ചിരിക്കുന്നു. പറയുക: "നിങ്ങള്‍ ‎വിശ്വാസികളെങ്കില്‍ നിങ്ങളുടെ വിശ്വാസം ‎നിങ്ങളോടാവശ്യപ്പെടുന്നത് വളരെ ചീത്ത തന്നെ." ‎
\end{malayalam}}
\flushright{\begin{Arabic}
\quranayah[2][94]
\end{Arabic}}
\flushleft{\begin{malayalam}
പറയുക: “ദൈവത്തിങ്കല്‍ പരലോകരക്ഷ ‎മറ്റാര്‍ക്കുമില്ലാതെ നിങ്ങള്‍ക്കു മാത്രം ‎പ്രത്യേകമായുള്ളതാണെങ്കില്‍ നിങ്ങള്‍ മരണം ‎കൊതിക്കുക; നിങ്ങള്‍ സത്യസന്ധരെങ്കില്‍!" ‎
\end{malayalam}}
\flushright{\begin{Arabic}
\quranayah[2][95]
\end{Arabic}}
\flushleft{\begin{malayalam}
അവരൊരിക്കലും അതാഗ്രഹിക്കുകയില്ല. കാരണം ‎നേരത്തെ ചെയ്തുകൂട്ടിയ ചീത്ത പ്രവൃത്തികള്‍ തന്നെ. ‎അതിക്രമികളെ നന്നായി തിരിച്ചറിയുന്നവനാണ് ‎അല്ലാഹു. ‎
\end{malayalam}}
\flushright{\begin{Arabic}
\quranayah[2][96]
\end{Arabic}}
\flushleft{\begin{malayalam}
ജീവിതത്തോട് മറ്റാരെക്കാളും കൊതിയുള്ളവരായി ‎നിനക്കവരെ കാണാം; ബഹുദൈവ വിശ്വാസികളെക്കാളും ‎അത്യാഗ്രഹികളായി. ആയിരം കൊല്ലമെങ്കിലും ‎ആയുസ്സുണ്ടായെങ്കില്‍ എന്ന് അവര്‍ ഓരോരുത്തരും ‎ആഗ്രഹിക്കുന്നു. എന്നാല്‍ ആയുര്‍ദൈര്‍ഘ്യം ശിക്ഷയില്‍ ‎നിന്ന് രക്ഷപ്പെടുത്തുകയില്ല. അവര്‍ ചെയ്യുന്നതൊക്കെയും ‎സൂക്ഷ്മമായി വീക്ഷിക്കുന്നവനാണ് അല്ലാഹു. ‎
\end{malayalam}}
\flushright{\begin{Arabic}
\quranayah[2][97]
\end{Arabic}}
\flushleft{\begin{malayalam}
പറയുക: ആരെങ്കിലും ശത്രുത പുലര്‍ത്തുന്നത് ജിബ്രീലി ‎നോടാണെങ്കില്‍ അവരറിയണം; ജിബ്രീല്‍ നിന്റെ മനസ്സില്‍ ‎വേദമിറക്കിയത് ദൈവനിര്‍ദേശപ്രകാരം മാത്രമാണ്. ‎അത് മുന്‍ വേദങ്ങളെ സത്യപ്പെടുത്തുന്നു. സത്യവിശ്വാസം ‎സ്വീകരിക്കുന്നവര്‍ക്ക് നേര്‍വഴി നിര്‍ദേശിക്കുന്നു. ‎സുവാര്‍ത്ത അറിയിക്കുകയും ചെയ്യുന്നു. ‎
\end{malayalam}}
\flushright{\begin{Arabic}
\quranayah[2][98]
\end{Arabic}}
\flushleft{\begin{malayalam}
ആരെങ്കിലും അല്ലാഹുവിന്റെയും മലക്കുകളുടെയും ‎അവന്റെ ദൂതന്മാരുടെയും ജിബ്രീലിന്റെയും ‎മീകാഈലി ന്റെയും ശത്രുവാണെങ്കില്‍ അറിയുക: ‎നിസ്സംശയം അല്ലാഹു സത്യനിഷേധികളോട് ‎വിരോധമുള്ളവനത്രെ. ‎
\end{malayalam}}
\flushright{\begin{Arabic}
\quranayah[2][99]
\end{Arabic}}
\flushleft{\begin{malayalam}
ഉറപ്പായും നിനക്കു നാം വ്യക്തമായ വചനങ്ങളാണ് ‎അവതരിപ്പിച്ചിരിക്കുന്നത്. കുറ്റവാളികളല്ലാതെ അതിനെ ‎തള്ളിക്കളയുകയില്ല. ‎
\end{malayalam}}
\flushright{\begin{Arabic}
\quranayah[2][100]
\end{Arabic}}
\flushleft{\begin{malayalam}
അവര്‍ ഏതൊരു കരാറിലേര്‍പ്പെട്ടാലും അവരിലൊരു ‎വിഭാഗം അതിനെ തള്ളിക്കളയുകയാണോ? അല്ല; ‎അവരിലേറെ പേരും സത്യനിഷേധികളാകുന്നു. ‎
\end{malayalam}}
\flushright{\begin{Arabic}
\quranayah[2][101]
\end{Arabic}}
\flushleft{\begin{malayalam}
അവരുടെ അടുത്ത് ദൈവദൂതന്‍ വന്നെത്തി. അദ്ദേഹം ‎അവരുടെ വശമുള്ളതിനെ സത്യപ്പെടുത്തുന്നവനായിരുന്നു. ‎എന്നിട്ടും വേദം കിട്ടിയവരിലൊരുകൂട്ടര്‍ ആ ‎ദൈവികഗ്രന്ഥത്തെ പിറകോട്ട് വലിച്ചെറിഞ്ഞു. ‎അവര്‍ക്കൊന്നും അറിയാത്തപോലെ. ‎
\end{malayalam}}
\flushright{\begin{Arabic}
\quranayah[2][102]
\end{Arabic}}
\flushleft{\begin{malayalam}
സുലൈമാന്റെ ആധിപത്യത്തിനെതിരെ പിശാചുക്കള്‍ ‎പറഞ്ഞുപരത്തിയതൊക്കെയും അവര്‍ പിന്‍പറ്റി. ‎യഥാര്‍ഥത്തില്‍ സുലൈമാന്‍ അവിശ്വാസി ആയിട്ടില്ല. ‎അവിശ്വസിച്ചത് ആ പിശാചുക്കളാണ്. അവര്‍ ജനങ്ങള്‍ക്ക് ‎മാരണം പഠിപ്പിക്കുകയായിരുന്നു. ‎ബാബിലോണിയയിലെ ഹാറൂത്, മാറൂത് എന്നീ ‎മലക്കുകള്‍ക്ക് ഇറക്കിക്കൊടുത്തതിനെയും അവര്‍ ‎പിന്‍പറ്റി. അവരിരുവരും അതാരെയും ‎പഠിപ്പിച്ചിരുന്നില്ല: “ഞങ്ങളൊരു പരീക്ഷണം; അതിനാല്‍ ‎നീ സത്യനിഷേധിയാകരുത്" എന്ന് ‎അറിയിച്ചുകൊണ്ടല്ലാതെ. അങ്ങനെ ജനം ‎അവരിരുവരില്‍നിന്ന് ഭാര്യാ-ഭര്‍ത്താക്കന്മാര്‍ക്കിടയില്‍ ‎വിടവുണ്ടാക്കുന്ന വിദ്യ പഠിച്ചുകൊണ്ടിരുന്നു. എന്നാല്‍ ‎അല്ലാഹുവിന്റെ അനുവാദമില്ലാതെ അവര്‍ക്ക് ‎അതുപയോഗിച്ച് ആരെയും ദ്രോഹിക്കാനാവില്ല. ‎തങ്ങള്‍ക്കു ദോഷകരവും ഒപ്പം ഒട്ടും ‎ഉപകാരമില്ലാത്തതുമാണ് അവര്‍ ‎പഠിച്ചുകൊണ്ടിരുന്നത്. ആ വിദ്യ സ്വീകരിക്കുന്നവര്‍ക്ക് ‎പരലോകത്ത് ഒരു വിഹിതവുമില്ലെന്ന് അവര്‍ക്കുതന്നെ ‎നന്നായറിയാം. അവര്‍ സ്വന്തത്തെ വിറ്റുവാങ്ങിയത് എത്ര ‎ചീത്ത? അവരതറിഞ്ഞിരുന്നെങ്കില്‍. ‎
\end{malayalam}}
\flushright{\begin{Arabic}
\quranayah[2][103]
\end{Arabic}}
\flushleft{\begin{malayalam}
അവര്‍ സത്യവിശ്വാസം സ്വീകരിക്കുകയും ദോഷബാധയെ ‎സൂക്ഷിക്കുകയുമാണെങ്കില്‍ അല്ലാഹുവിങ്കലുള്ള ‎പ്രതിഫലം അത്യുത്തമമാകുമായിരുന്നു. ‎അവരതറിഞ്ഞിരുന്നെങ്കില്‍. ‎
\end{malayalam}}
\flushright{\begin{Arabic}
\quranayah[2][104]
\end{Arabic}}
\flushleft{\begin{malayalam}
വിശ്വസിച്ചവരേ, നിങ്ങള്‍ “റാഇനാ" എന്നു പറയരുത്. ‎പകരം “ഉന്‍ളുര്‍നാ" എന്നുപറയുക. ശ്രദ്ധയോടെ ‎കേള്‍ക്കുകയും ചെയ്യുക. സത്യനിഷേധികള്‍ക്ക് ‎നോവേറിയ ശിക്ഷയുണ്ട്. ‎
\end{malayalam}}
\flushright{\begin{Arabic}
\quranayah[2][105]
\end{Arabic}}
\flushleft{\begin{malayalam}
വേദക്കാരിലെയും ബഹുദൈവവിശ്വാസികളിലെയും ‎സത്യനിഷേധികള്‍ നിങ്ങളുടെ നാഥനില്‍ നിന്ന് നിങ്ങള്‍ക്ക് ‎ഒരു ഗുണവും ലഭിക്കുന്നത് തീരെ ഇഷ്ടപ്പെടുന്നില്ല. ‎എന്നാല്‍ അല്ലാഹു തന്റെ കാരുണ്യത്താല്‍ ‎താനിച്ഛിക്കുന്നവരെ പ്രത്യേകം അനുഗ്രഹിക്കുന്നു. ‎അല്ലാഹു അതിമഹത്തായ അനുഗ്രഹമുള്ളവന്‍ തന്നെ. ‎
\end{malayalam}}
\flushright{\begin{Arabic}
\quranayah[2][106]
\end{Arabic}}
\flushleft{\begin{malayalam}
ഏതെങ്കിലും വേദവാക്യത്തെ നാം ദുര്‍ബലമാക്കുകയോ ‎മറപ്പിക്കുകയോ ആണെങ്കില്‍ പകരം തത്തുല്യമോ ‎കൂടുതല്‍ മികച്ചതോ നാം കൊണ്ടുവരും. നിനക്കറിയില്ലേ, ‎അല്ലാഹു എല്ലാ കാര്യത്തിനും കഴിവുറ്റവനാണെന്ന്. ‎
\end{malayalam}}
\flushright{\begin{Arabic}
\quranayah[2][107]
\end{Arabic}}
\flushleft{\begin{malayalam}
നിനക്കറിയില്ലേ, തീര്‍ച്ചയായും അല്ലാഹുവിനു ‎തന്നെയാണ് ആകാശ ഭൂമികളുടെ സമ്പൂര്‍ണാധിപത്യം. ‎അല്ലാഹുവല്ലാതെ നിങ്ങള്‍ക്കൊരു രക്ഷകനോ ‎സഹായിയോ ഇല്ല. ‎
\end{malayalam}}
\flushright{\begin{Arabic}
\quranayah[2][108]
\end{Arabic}}
\flushleft{\begin{malayalam}
അല്ല; നേരത്തെ മൂസയോട് തന്റെ ജനം ഉന്നയിച്ചതു ‎പോലുള്ള ചോദ്യങ്ങള്‍ നിങ്ങളുടെ പ്രവാചകനോട് ‎ചോദിക്കാനാണോ നിങ്ങളാഗ്രഹിക്കുന്നത്? സംശയമില്ല; ‎സത്യവിശ്വാസത്തിനുപകരം സത്യനിഷേധം ‎സ്വീകരിക്കുന്നവര്‍ നേര്‍വഴിയില്‍നിന്ന് ‎തെറ്റിപ്പോയിരിക്കുന്നു. ‎
\end{malayalam}}
\flushright{\begin{Arabic}
\quranayah[2][109]
\end{Arabic}}
\flushleft{\begin{malayalam}
വേദക്കാരില്‍ ഏറെപ്പേരും ആഗ്രഹിക്കുന്നു, നിങ്ങള്‍ ‎സത്യവിശ്വാസികളായ ശേഷം നിങ്ങളെ ‎സത്യനിഷേധികളാക്കി മാറ്റാന്‍ സാധിച്ചെങ്കിലെന്ന്! ‎അവരുടെ അസൂയയാണതിനു കാരണം. ഇതൊക്കെയും ‎സത്യം അവര്‍ക്ക് നന്നായി വ്യക്തമായ ശേഷമാണ്. ‎അതിനാല്‍ അല്ലാഹു തന്റെ കല്‍പന നടപ്പാക്കും വരെ ‎നിങ്ങള്‍ വിട്ടുവീഴ്ച കാണിക്കുക. സംയമനം പാലിക്കുക. ‎തീര്‍ച്ചയായും അല്ലാഹു എല്ലാ കാര്യങ്ങള്‍ക്കും ‎കഴിവുറ്റവന്‍ തന്നെ. ‎
\end{malayalam}}
\flushright{\begin{Arabic}
\quranayah[2][110]
\end{Arabic}}
\flushleft{\begin{malayalam}
നിങ്ങള്‍ നിഷ്ഠയോടെ നമസ്കരിക്കുക. സകാത്ത് നല്‍കുക. ‎നിങ്ങള്‍ ചെയ്യുന്ന ഏതു നന്മയുടെയും സദ്ഫലം ‎നിങ്ങള്‍ക്ക് അല്ലാഹുവിങ്കല്‍ കണ്ടെത്താം. നിങ്ങള്‍ ‎ചെയ്യുന്നതൊക്കെയും ഉറപ്പായും അല്ലാഹു കാണുന്നുണ്ട്. ‎
\end{malayalam}}
\flushright{\begin{Arabic}
\quranayah[2][111]
\end{Arabic}}
\flushleft{\begin{malayalam}
ജൂതനോ ക്രിസ്ത്യാനിയോ ആവാതെ ആരും ‎സ്വര്‍ഗത്തിലെത്തുകയില്ലെന്ന് അവര്‍ അവകാശപ്പെടുന്നു. ‎അതവരുടെ വ്യാമോഹം മാത്രം. അവരോട് പറയൂ: ‎നിങ്ങള്‍ തെളിവു കൊണ്ടുവരിക; നിങ്ങള്‍ ‎സത്യസന്ധരെങ്കില്‍. ‎
\end{malayalam}}
\flushright{\begin{Arabic}
\quranayah[2][112]
\end{Arabic}}
\flushleft{\begin{malayalam}
എന്നാല്‍ ആര്‍ സുകൃതവാനായി സര്‍വസ്വം ‎അല്ലാഹുവിന് സമര്‍പിക്കുന്നുവോ അവന് തന്റെ ‎നാഥന്റെ അടുത്ത് അതിനുള്ള പ്രതിഫലമുണ്ട്. അവര്‍ക്ക് ‎ഒന്നും ഭയപ്പെടാനില്ല. ദുഃഖിക്കാനുമില്ല. ‎
\end{malayalam}}
\flushright{\begin{Arabic}
\quranayah[2][113]
\end{Arabic}}
\flushleft{\begin{malayalam}
ക്രിസ്ത്യാനികളുടെ നിലപാടുകള്‍ക്ക് ‎ഒരടിസ്ഥാനവുമില്ലെന്ന് യഹൂദര്‍ പറയുന്നു. യഹൂദരുടെ ‎വാദങ്ങള്‍ക്ക് അടിസ്ഥാനമൊന്നുമില്ലെന്ന് ‎ക്രിസ്ത്യാനികളും വാദിക്കുന്നു. അവരൊക്കെ ‎വേദമോതുന്നവരാണുതാനും. വിവരമില്ലാത്ത ‎ചിലരെല്ലാം മുമ്പും ഇവര്‍ വാദിക്കും വിധം ‎പറഞ്ഞിട്ടുണ്ട്. അതിനാല്‍, അവര്‍ ‎ഭിന്നിച്ചുകൊണ്ടിരിക്കുന്ന കാര്യങ്ങളില്‍ ‎ഉയിര്‍ത്തെഴുന്നേല്‍പു നാളില്‍ അല്ലാഹു വിധി ‎കല്‍പിക്കുന്നതാണ്. ‎
\end{malayalam}}
\flushright{\begin{Arabic}
\quranayah[2][114]
\end{Arabic}}
\flushleft{\begin{malayalam}
അല്ലാഹുവിന്റെ പള്ളികളില്‍ അവന്റെ നാമം ‎പ്രകീര്‍ത്തിക്കുന്നത് വിലക്കുകയും പള്ളികളുടെ തന്നെ ‎നാശത്തിന് ശ്രമിക്കുകയും ചെയ്യുന്നവനേക്കാള്‍ കടുത്ത ‎അക്രമി ആരാണ്? പേടിച്ചുകൊണ്ടല്ലാതെ അവര്‍ക്കവിടെ ‎പ്രവേശിക്കാവതല്ല. അവര്‍ക്ക് ഈ ലോകത്ത് കൊടിയ ‎അപമാനമുണ്ട്. പരലോകത്ത് കഠിന ശിക്ഷയും. ‎
\end{malayalam}}
\flushright{\begin{Arabic}
\quranayah[2][115]
\end{Arabic}}
\flushleft{\begin{malayalam}
കിഴക്കും പടിഞ്ഞാറും അല്ലാഹുവിന്റേതാണ്. അതിനാല്‍ ‎നിങ്ങള്‍ എങ്ങോട്ടു തിരിഞ്ഞു പ്രാര്‍ഥിച്ചാലും ‎അവിടെയൊക്കെ അല്ലാഹുവിന്റെ സാന്നിധ്യമുണ്ട്. ‎അല്ലാഹു അതിരുകള്‍ക്കതീതനാണ്. എല്ലാം ‎അറിയുന്നവനും. ‎
\end{malayalam}}
\flushright{\begin{Arabic}
\quranayah[2][116]
\end{Arabic}}
\flushleft{\begin{malayalam}
ദൈവം പുത്രനെ വരിച്ചിരിക്കുന്നുവെന്ന് അവര്‍ ‎വാദിക്കുന്നു. എന്നാല്‍ അവന്‍ അതില്‍നിന്നെല്ലാം എത്ര ‎പരിശുദ്ധന്‍. ആകാശഭൂമികളിലുള്ളതെല്ലാം ‎അവന്റേതാണ്. എല്ലാം അവന്ന് വഴങ്ങുന്നവയും. ‎
\end{malayalam}}
\flushright{\begin{Arabic}
\quranayah[2][117]
\end{Arabic}}
\flushleft{\begin{malayalam}
ഇല്ലായ്മയില്‍നിന്ന് ആകാശ ഭൂമികളെ ‎ഉണ്ടാക്കിയവനാണവന്‍. അവനൊരു കാര്യം ‎തീരുമാനിച്ചാല്‍ “ഉണ്ടാവുക" എന്ന വചനം മതി. ‎അതോടെ അതുണ്ടാകുന്നു. ‎
\end{malayalam}}
\flushright{\begin{Arabic}
\quranayah[2][118]
\end{Arabic}}
\flushleft{\begin{malayalam}
അറിവില്ലാത്തവര്‍ ചോദിക്കുന്നു: "അല്ലാഹു ഞങ്ങളോട് ‎നേരില്‍ സംസാരിക്കാത്തതെന്ത്? അല്ലെങ്കില്‍ ഞങ്ങള്‍ക്ക് ‎ഒരടയാളമെങ്കിലും കൊണ്ടുവരാത്തതെന്ത്?" ‎ഇവരിപ്പോള്‍ ചോദിക്കുന്നപോലെ ഇവരുടെ ‎മുന്‍ഗാമികളും ചോദിച്ചിരുന്നു. ഇരുവിഭാഗത്തിന്റെയും ‎മനസ്സുകള്‍ ഒരുപോലെയാണ്. തീര്‍ച്ചയായും അടിയുറച്ചു ‎വിശ്വസിക്കുന്നവര്‍ക്ക് നാം തെളിവുകള്‍ ‎വ്യക്തമാക്കിക്കൊടുത്തിട്ടുണ്ട്. ‎
\end{malayalam}}
\flushright{\begin{Arabic}
\quranayah[2][119]
\end{Arabic}}
\flushleft{\begin{malayalam}
നിസ്സംശയം, നിന്നെ നാം സത്യസന്ദേശവുമായാണ് ‎അയച്ചത്. ശുഭവാര്‍ത്ത അറിയിക്കുന്നവനും മുന്നറിയിപ്പ് ‎നല്‍കുന്നവനുമായി. അതിനാല്‍ നരകാവകാശികളെപ്പറ്റി ‎നിന്നോടു ചോദിക്കുകയില്ല. ‎
\end{malayalam}}
\flushright{\begin{Arabic}
\quranayah[2][120]
\end{Arabic}}
\flushleft{\begin{malayalam}
ജൂതരോ ക്രൈസ്തവരോ നിന്നെ സംബന്ധിച്ച് ‎സംതൃപ്തരാവുകയില്ല; നീ അവരുടെ ‎മാര്‍ഗമവലംബിക്കുംവരെ. പറയുക: സംശയമില്ല. ‎ദൈവിക മാര്‍ഗദര്‍ശനമാണ് സത്യദര്‍ശനം. നിനക്കു ‎യഥാര്‍ഥ ജ്ഞാനം ലഭിച്ചശേഷം നീ അവരുടെ ഇച്ഛകളെ ‎പിന്‍പറ്റിയാല്‍ പിന്നെ അല്ലാഹുവിന്റെ പിടിയില്‍നിന്ന് ‎നിന്നെ രക്ഷിക്കാന്‍ ഏതെങ്കിലും കൂട്ടാളിയോ ‎സഹായിയോ ഉണ്ടാവുകയില്ല. ‎
\end{malayalam}}
\flushright{\begin{Arabic}
\quranayah[2][121]
\end{Arabic}}
\flushleft{\begin{malayalam}
നാം ഈ വേദഗ്രന്ഥം നല്‍കിയവര്‍ ആരോ അവരിത് ‎യഥാവിധി പാരായണം ചെയ്യുന്നു. അവരിതില്‍ ‎ആത്മാര്‍ഥമായി വിശ്വസിക്കുന്നു. അതിനെ ‎നിഷേധിക്കുന്നവരോ, യഥാര്‍ഥത്തില്‍ അവര്‍ തന്നെയാണ് ‎നഷ്ടംപറ്റിയവര്‍. ‎
\end{malayalam}}
\flushright{\begin{Arabic}
\quranayah[2][122]
\end{Arabic}}
\flushleft{\begin{malayalam}
ഇസ്രയേല്‍ മക്കളേ, ഞാന്‍ നിങ്ങള്‍ക്കേകിയ ‎അനുഗ്രഹങ്ങളോര്‍ക്കുക; നിങ്ങളെ സകല ജനത്തേക്കാളും ‎ശ്രേഷ്ഠരാക്കിയതും. ‎
\end{malayalam}}
\flushright{\begin{Arabic}
\quranayah[2][123]
\end{Arabic}}
\flushleft{\begin{malayalam}
ആര്‍ക്കും മറ്റുള്ളവര്‍ക്കായി ഒന്നും ചെയ്യാനാവാത്ത; ‎ആരുടെയും പ്രായശ്ചിത്തം സ്വീകരിക്കാത്ത; ആര്‍ക്കും ‎ആരുടെയും ശിപാര്‍ശ ഉപകരിക്കാത്ത; ആര്‍ക്കും ‎ഒരുവിധ സഹായവും ലഭിക്കാത്ത ആ നാളിനെ ‎സൂക്ഷിക്കുക. ‎
\end{malayalam}}
\flushright{\begin{Arabic}
\quranayah[2][124]
\end{Arabic}}
\flushleft{\begin{malayalam}
ഓര്‍ക്കുക: ഇബ്റാഹീമിനെ അദ്ദേഹത്തിന്റെ നാഥന്‍ ‎ചില കല്‍പനകളിലൂടെ പരീക്ഷിച്ചു. അദ്ദേഹം ‎അതൊക്കെയും നടപ്പാക്കി. അപ്പോള്‍ അല്ലാഹു അരുളി: ‎‎"നിന്നെ ഞാന്‍ ജനങ്ങളുടെ നേതാവാക്കുകയാണ്." ‎ഇബ്റാഹീം ആവശ്യപ്പെട്ടു: "എന്റെ മക്കളെയും." ‎അല്ലാഹു അറിയിച്ചു: "എന്റെ വാഗ്ദാനം ‎അക്രമികള്‍ക്കു ബാധകമല്ല." ‎
\end{malayalam}}
\flushright{\begin{Arabic}
\quranayah[2][125]
\end{Arabic}}
\flushleft{\begin{malayalam}
ഓര്‍ക്കുക: ആ ഭവന ത്തെ നാം മാനവതയുടെ മഹാസംഗമ ‎സ്ഥാനമാക്കി; നിര്‍ഭയമായ സങ്കേതവും. ഇബ്റാഹീം ‎നിന്ന ഇടം നിങ്ങള്‍ നമസ്കാര സ്ഥലമാക്കുക. ത്വവാഫ് ‎ചെയ്യുന്നവര്‍ക്കും ഭജനമിരിക്കുന്നവര്‍ക്കും തലകുനിച്ചും ‎സാഷ്ടാംഗം പ്രണമിച്ചും പ്രാര്‍ഥിക്കുന്നവര്‍ക്കുമായി ‎എന്റെ ഭവനം വൃത്തിയാക്കിവെക്കണമെന്ന് ‎ഇബ്റാഹീമിനോടും ഇസ്മാഈലിനോടും നാം ‎കല്‍പിച്ചു. ‎
\end{malayalam}}
\flushright{\begin{Arabic}
\quranayah[2][126]
\end{Arabic}}
\flushleft{\begin{malayalam}
ഇബ്റാഹീം പ്രാര്‍ഥിച്ചത് ഓര്‍ക്കുക: "എന്റെ നാഥാ! ‎ഇതിനെ നീ ഭീതി ഏതുമില്ലാത്ത നാടാക്കേണമേ! ഇവിടെ ‎പാര്‍ക്കുന്നവരില്‍ അല്ലാഹുവിലും അന്ത്യദിനത്തിലും ‎വിശ്വസിക്കുന്നവര്‍ക്ക് ആഹാരമായി കായ്കനികള്‍ ‎നല്‍കേണമേ." അല്ലാഹു അറിയിച്ചു: "അവിശ്വാസിക്കും ‎നാമതു നല്‍കും. ഇത്തിരി കാലത്തെ ജീവിതസുഖം ‎മാത്രമാണ് അവന്നുണ്ടാവുക. പിന്നെ നാമവനെ നരക ‎ശിക്ഷക്കു വിധേയനാക്കും. അത് ചീത്ത താവളം തന്നെ." ‎
\end{malayalam}}
\flushright{\begin{Arabic}
\quranayah[2][127]
\end{Arabic}}
\flushleft{\begin{malayalam}
ഓര്‍ക്കുക: ഇബ്റാഹീമും ഇസ്മാഈലും ആ ‎മന്ദിരത്തിന്റെ അടിത്തറ കെട്ടിപ്പൊക്കുകയായിരുന്നു. ‎അന്നേരമവര്‍ പ്രാര്‍ഥിച്ചു: "ഞങ്ങളുടെ നാഥാ! ഞങ്ങളില്‍ ‎നിന്ന് നീയിത് സ്വീകരിക്കേണമേ; നിശ്ചയമായും നീ എല്ലാം ‎കേള്‍ക്കുന്നവനും അറിയുന്നവനുമല്ലോ". ‎
\end{malayalam}}
\flushright{\begin{Arabic}
\quranayah[2][128]
\end{Arabic}}
\flushleft{\begin{malayalam}
‎"ഞങ്ങളുടെ നാഥാ! നീ ഞങ്ങളിരുവരെയും നിന്നെ ‎അനുസരിക്കുന്നവരാക്കേണമേ! ഞങ്ങളുടെ ‎സന്തതികളില്‍നിന്ന് നിന്നെ വഴങ്ങുന്ന ഒരു സമുദായത്തെ ‎ഉയര്‍ത്തിക്കൊണ്ടുവരേണമേ! ഞങ്ങളുടെ ‎ഉപാസനാക്രമങ്ങള്‍ ഞങ്ങള്‍ക്കു നീ കാണിച്ചു തരേണമേ! ‎ഞങ്ങളുടെ പശ്ചാത്താപം സ്വീകരിക്കേണമേ; സംശയമില്ല, ‎നീ പശ്ചാത്താപം ഉദാരമായി സ്വീകരിക്കുന്നവനും ‎കരുണാമയനും തന്നെ. ‎
\end{malayalam}}
\flushright{\begin{Arabic}
\quranayah[2][129]
\end{Arabic}}
\flushleft{\begin{malayalam}
‎"ഞങ്ങളുടെ നാഥാ! നീ അവര്‍ക്ക് അവരില്‍ നിന്നു തന്നെ ‎ഒരു ദൂതനെ നിയോഗിക്കേണമേ! അവര്‍ക്കു നിന്റെ ‎വചനങ്ങള്‍ ഓതിക്കേള്‍പ്പിക്കുകയും വേദവും ‎വിജ്ഞാനവും പഠിപ്പിക്കുകയും അവരെ ‎സംസ്കരിക്കുകയും ചെയ്യുന്ന ദൂതനെ. നിസ്സംശയം, നീ ‎പ്രതാപിയും യുക്തിജ്ഞനും തന്നെ." ‎
\end{malayalam}}
\flushright{\begin{Arabic}
\quranayah[2][130]
\end{Arabic}}
\flushleft{\begin{malayalam}
ആരെങ്കിലും ഇബ്റാഹീമിന്റെ മാര്‍ഗം വെറുക്കുമോ? ‎സ്വയം വിഡ്ഢിയായവനല്ലാതെ. ഈ ലോകത്ത് നാം ‎അദ്ദേഹത്തെ മികവുറ്റവനായി തെരഞ്ഞെടുത്തിരിക്കുന്നു. ‎പരലോകത്തും അദ്ദേഹം സച്ചരിതരിലായിരിക്കും. ‎
\end{malayalam}}
\flushright{\begin{Arabic}
\quranayah[2][131]
\end{Arabic}}
\flushleft{\begin{malayalam}
നിന്റെ നാഥന്‍ അദ്ദേഹത്തോട് “വഴിപ്പെടുക" എന്ന് ‎കല്‍പിച്ചു.അപ്പോള്‍ അദ്ദേഹം പറഞ്ഞു: ‎‎“സര്‍വലോകനാഥന് ഞാനിതാ വഴിപ്പെട്ടിരിക്കുന്നു." ‎
\end{malayalam}}
\flushright{\begin{Arabic}
\quranayah[2][132]
\end{Arabic}}
\flushleft{\begin{malayalam}
ഇബ്റാഹീമും യഅ്ഖൂബും തങ്ങളുടെ മക്കളോട് ‎ഇതുതന്നെ ഉപദേശിച്ചു: "എന്റെ മക്കളേ, അല്ലാഹു ‎നിങ്ങള്‍ക്ക് നിശ്ചയിച്ചുതന്ന വിശിഷ്ടമായ ജീവിത ‎വ്യവസ്ഥയാണിത്. അതിനാല്‍ നിങ്ങള്‍ ‎മുസ്ലിംകളായല്ലാതെ മരണപ്പെടരുത്." ‎
\end{malayalam}}
\flushright{\begin{Arabic}
\quranayah[2][133]
\end{Arabic}}
\flushleft{\begin{malayalam}
‎“എനിക്കുശേഷം നിങ്ങള്‍ ആരെയാണ് വഴിപ്പെടുക"യെന്ന് ‎ആസന്നമരണനായിരിക്കെ യഅ്ഖൂബ് തന്റെ മക്കളോടു ‎ചോദിച്ചപ്പോള്‍ നിങ്ങളവിടെ ഉണ്ടായിരുന്നോ? അവര്‍ ‎പറഞ്ഞു: "ഞങ്ങള്‍ അങ്ങയുടെ ദൈവത്തെ തന്നെയാണ് ‎വഴിപ്പെടുക. അങ്ങയുടെ പിതാവായ ‎ഇബ്റാഹീമിന്റെയും ഇസ്മാഈലിന്റെയും ‎ഇസ്ഹാഖിന്റെയും നാഥനായ ആ ഏക ദൈവത്തെ. ‎ഞങ്ങള്‍ അവനെ അനുസരിച്ച് ജീവിക്കുന്നവരാകും." ‎
\end{malayalam}}
\flushright{\begin{Arabic}
\quranayah[2][134]
\end{Arabic}}
\flushleft{\begin{malayalam}
ഏതായാലും അത് കഴിഞ്ഞുപോയ ഒരു സമുദായം. ‎അവര്‍ക്ക് അവര്‍ ചെയ്തതിന്റെ ഫലമുണ്ട്. നിങ്ങള്‍ക്ക് ‎നിങ്ങള്‍ ശേഖരിച്ചുവെച്ചതിന്റെയും. അവര്‍ ‎പ്രവര്‍ത്തിച്ചിരുന്നതിനെപ്പറ്റി നിങ്ങളോട് ആരും ‎ചോദിക്കുകയില്ല. ‎
\end{malayalam}}
\flushright{\begin{Arabic}
\quranayah[2][135]
\end{Arabic}}
\flushleft{\begin{malayalam}
അവര്‍ പറയുന്നു: "നിങ്ങള്‍ നേര്‍വഴിയിലാകണമെങ്കില്‍ ‎ജൂതരോ ക്രിസ്ത്യാനികളോ ആവുക."പറയുക: "അല്ല. ‎ശുദ്ധ മാനസനായ ഇബ്റാഹീമിന്റെ മാര്‍ഗമാണ് ‎സ്വീകരിക്കേണ്ടത്. അദ്ദേഹം ബഹുദൈവ ‎വാദിയായിരുന്നില്ല." ‎
\end{malayalam}}
\flushright{\begin{Arabic}
\quranayah[2][136]
\end{Arabic}}
\flushleft{\begin{malayalam}
നിങ്ങള്‍ പ്രഖ്യാപിക്കുക: ഞങ്ങള്‍ അല്ലാഹുവിലും ‎അവനില്‍നിന്ന് ഞങ്ങള്‍ക്ക് ഇറക്കിക്കിട്ടിയതിലും ‎ഇബ്റാഹീം, ഇസ്മാഈല്‍, ഇസ്ഹാഖ്, യഅ്ഖൂബ്, ‎അവരുടെ സന്താനപരമ്പരകള്‍ എന്നിവര്‍ക്ക് ‎ഇറക്കിക്കൊടുത്തതിലും മൂസാക്കും ഈസാക്കും ‎നല്‍കിയതിലും മറ്റു പ്രവാചകന്മാര്‍ക്ക് തങ്ങളുടെ ‎നാഥനില്‍നിന്ന് അവതരിച്ചവയിലും ‎വിശ്വസിച്ചിരിക്കുന്നു. അവരിലാര്‍ക്കുമിടയില്‍ ‎ഞങ്ങളൊരുവിധ വിവേചനവും കല്‍പിക്കുന്നില്ല. ഞങ്ങള്‍ ‎അല്ലാഹുവിനെ അനുസരിച്ച് ജീവിക്കുന്നവരത്രെ. ‎
\end{malayalam}}
\flushright{\begin{Arabic}
\quranayah[2][137]
\end{Arabic}}
\flushleft{\begin{malayalam}
നിങ്ങള്‍ വിശ്വസിച്ചപോലെ അവരും ‎വിശ്വസിക്കുകയാണെങ്കില്‍ അവരും ‎നേര്‍വഴിയിലാകുമായിരുന്നു. അവര്‍ ‎പിന്തിരിയുകയാണെങ്കില്‍ പിന്നെ അവര്‍ കടുത്ത ‎കിടമത്സരത്തില്‍ തന്നെയായിരിക്കും. അവരില്‍നിന്ന് ‎നിന്നെ കാക്കാന്‍ അല്ലാഹുമതി. അവന്‍ എല്ലാം ‎കേള്‍ക്കുന്നവനും അറിയുന്നവനുമല്ലോ. ‎
\end{malayalam}}
\flushright{\begin{Arabic}
\quranayah[2][138]
\end{Arabic}}
\flushleft{\begin{malayalam}
അല്ലാഹുവിന്റെ വര്‍ണം സ്വീകരിക്കുക. ‎അല്ലാഹുവിന്റെ വര്‍ണത്തെക്കാള്‍ വിശിഷ്ടമായി ആരുടെ ‎വര്‍ണമുണ്ട്? അവനെയാണ് ഞങ്ങള്‍ വഴിപ്പെടുന്നത്. ‎
\end{malayalam}}
\flushright{\begin{Arabic}
\quranayah[2][139]
\end{Arabic}}
\flushleft{\begin{malayalam}
ചോദിക്കുക: അല്ലാഹുവിന്റെ കാര്യത്തില്‍ നിങ്ങള്‍ ‎ഞങ്ങളോട് തര്‍ക്കിക്കുകയാണോ? അവന്‍ ഞങ്ങളുടെയും ‎നിങ്ങളുടെയും നാഥനല്ലോ. ഞങ്ങള്‍ക്ക് ഞങ്ങളുടെ ‎കര്‍മഫലം. നിങ്ങള്‍ക്ക് നിങ്ങളുടേതും. ഞങ്ങള്‍ ‎ആത്മാര്‍ഥമായും അവന് മാത്രം ‎കീഴൊതുങ്ങിക്കഴിയുന്നവരാണ്. ‎
\end{malayalam}}
\flushright{\begin{Arabic}
\quranayah[2][140]
\end{Arabic}}
\flushleft{\begin{malayalam}
ഇബ്റാഹീമും ഇസ്മാഈലും ഇസ്ഹാഖും യഅ്ഖൂബും ‎അദ്ദേഹത്തിന്റെ സന്താനങ്ങളും ജൂതരോ ‎ക്രിസ്ത്യാനികളോ ആയിരുന്നുവെന്നാണോ നിങ്ങള്‍ ‎വാദിക്കുന്നത്? ചോദിക്കുക: നിങ്ങളാണോ ഏറ്റം ‎നന്നായറിയുന്നവര്‍? അതോ അല്ലാഹുവോ? ‎അല്ലാഹുവില്‍ നിന്ന് വന്നെത്തിയ തന്റെ വശമുള്ള ‎സാക്ഷ്യം മറച്ചുവെക്കുന്നവനെക്കാള്‍ വലിയ അക്രമി ‎ആരുണ്ട്? നിങ്ങള്‍ ചെയ്തുകൊണ്ടിരിക്കുന്നതിനെപ്പറ്റി ‎ഒട്ടും അശ്രദ്ധനല്ല അല്ലാഹു. ‎
\end{malayalam}}
\flushright{\begin{Arabic}
\quranayah[2][141]
\end{Arabic}}
\flushleft{\begin{malayalam}
അത് കഴിഞ്ഞുപോയ ജനസമുദായം. അവരുടെ കര്‍മഫലം ‎അവര്‍ക്ക്. നിങ്ങള്‍ സമ്പാദിച്ചത് നിങ്ങള്‍ക്കും. അവര്‍ ‎പ്രവര്‍ത്തിച്ചുകൊണ്ടിരുന്നതിനെപ്പറ്റി നിങ്ങളോടാരും ‎ചോദിക്കുകയില്ല. ‎
\end{malayalam}}
\flushright{\begin{Arabic}
\quranayah[2][142]
\end{Arabic}}
\flushleft{\begin{malayalam}
മൂഢന്മാര്‍ ചോദിക്കുന്നു: "അന്നോളം അവര്‍ ‎തിരിഞ്ഞുനിന്നിരുന്ന ഖിബ്ല യില്‍ നിന്ന് അവരെ ‎തെറ്റിച്ചതെന്ത്?" പറയുക: "കിഴക്കും പടിഞ്ഞാറും ‎അല്ലാഹുവിന്റേതുതന്നെ. അല്ലാഹു ‎അവനിച്ഛിക്കുന്നവരെ നേര്‍വഴിയില്‍ നയിക്കുന്നു." ‎
\end{malayalam}}
\flushright{\begin{Arabic}
\quranayah[2][143]
\end{Arabic}}
\flushleft{\begin{malayalam}
ഇവ്വിധം നിങ്ങളെ നാം ഒരു മിത ‎സമുദായമാക്കിയിരിക്കുന്നു. നിങ്ങള്‍ ലോകജനതക്ക് ‎സാക്ഷികളാകാന്‍. ദൈവദൂതന്‍ നിങ്ങള്‍ക്കു ‎സാക്ഷിയാകാനും. നീ നേരത്തെ തിരിഞ്ഞുനിന്നിരുന്ന ‎ദിക്കിനെ ഖിബ്ലയായി നിശ്ചയിച്ചിരുന്നത്, ദൈവദൂതനെ ‎പിന്‍പറ്റുന്നവരെയും പിന്‍മാറിപ്പോകുന്നവരെയും ‎വേര്‍തിരിച്ചറിയാന്‍ വേണ്ടി മാത്രമാണ്. അത് ഏറെ ‎പ്രയാസകരമായിരുന്നു; ദൈവിക ‎മാര്‍ഗദര്‍ശനത്തിനര്‍ഹരായവര്‍ക്കൊഴികെ. അല്ലാഹു ‎നിങ്ങളുടെ വിശ്വാസത്തെ ഒട്ടും പാഴാക്കുകയില്ല. ‎അല്ലാഹു ജനങ്ങളോട് അളവറ്റ ദയാപരനും ‎കരുണാമയനുമാകുന്നു. ‎
\end{malayalam}}
\flushright{\begin{Arabic}
\quranayah[2][144]
\end{Arabic}}
\flushleft{\begin{malayalam}
നിന്റെ മുഖം അടിക്കടി മാനത്തേക്ക് ‎തിരിഞ്ഞുകൊണ്ടിരിക്കുന്നത് നാം കാണുന്നുണ്ട്. ‎അതിനാല്‍ നിനക്കിഷ്ടപ്പെടുന്ന ഖിബ്ലയിലേക്ക് നിന്നെ നാം ‎തിരിക്കുകയാണ്. ഇനിമുതല്‍ മസ്ജിദുല്‍ഹറാമിന്റെ ‎നേരെ നീ നിന്റെ മുഖം തിരിക്കുക. നിങ്ങള്‍ ‎എവിടെയായിരുന്നാലും നിങ്ങള്‍ അതിന്റെ നേരെ മുഖം ‎തിരിക്കുക. വേദം നല്‍കപ്പെട്ടവര്‍ക്ക് ഇത് തങ്ങളുടെ ‎നാഥനില്‍ നിന്നുള്ള സത്യമാണെന്ന് നന്നായറിയാം. അവര്‍ ‎പ്രവര്‍ത്തിക്കുന്നതിനെപ്പറ്റി അല്ലാഹു ഒട്ടും അശ്രദ്ധനല്ല. ‎
\end{malayalam}}
\flushright{\begin{Arabic}
\quranayah[2][145]
\end{Arabic}}
\flushleft{\begin{malayalam}
നീ ഈ വേദക്കാരുടെ മുമ്പില്‍ എല്ലാ തെളിവുകളും ‎കൊണ്ടുചെന്നാലും അവര്‍ നിന്റെ ഖിബ്ലയെ ‎പിന്‍പറ്റുകയില്ല. അവരുടെ ഖിബ്ലയെ നിനക്കും ‎പിന്‍പറ്റാനാവില്ല. അവരില്‍തന്നെ ഒരുവിഭാഗം മറ്റു ‎വിഭാഗക്കാരുടെ ഖിബ്ലയെയും പിന്തുടരില്ല. ഈ ‎സത്യമായ അറിവ് ലഭിച്ചശേഷവും നീ അവരുടെ ‎തന്നിഷ്ടങ്ങളെ പിന്‍പറ്റിയാല്‍ ഉറപ്പായും നീയും ‎അതിക്രമികളുടെ കൂട്ടത്തില്‍ പെട്ടുപോകും. ‎
\end{malayalam}}
\flushright{\begin{Arabic}
\quranayah[2][146]
\end{Arabic}}
\flushleft{\begin{malayalam}
നാം വേദം നല്‍കിയ ജനത്തിന് അദ്ദേഹത്തെ തങ്ങളുടെ ‎മക്കളെ അറിയുന്നപോലെ അറിയാം. എന്നിട്ടും ‎അവരിലൊരുകൂട്ടര്‍ അറിഞ്ഞുകൊണ്ടുതന്നെ സത്യം ‎മറച്ചുവെക്കുകയാണ്. ‎
\end{malayalam}}
\flushright{\begin{Arabic}
\quranayah[2][147]
\end{Arabic}}
\flushleft{\begin{malayalam}
ഇത് നിന്റെ നാഥനില്‍ നിന്നുള്ള സത്യസന്ദേശമാണ്. ‎അതിനാല്‍ അതേപ്പറ്റി നീ സംശയാലുവാകരുത്. ‎
\end{malayalam}}
\flushright{\begin{Arabic}
\quranayah[2][148]
\end{Arabic}}
\flushleft{\begin{malayalam}
ഓരോ വിഭാഗത്തിനും ഓരോ ദിശയുണ്ട്. അവര്‍ ‎അതിന്റെ നേരെ തിരിയുന്നു. നിങ്ങള്‍ നന്മയിലേക്കു ‎മുന്നേറുക. നിങ്ങള്‍ എവിടെയായിരുന്നാലും അല്ലാഹു ‎നിങ്ങളെയെല്ലാം ഒന്നിച്ചുകൊണ്ടുവരും. അല്ലാഹു ‎എല്ലാറ്റിനും കഴിവുറ്റവന്‍ തന്നെ. ‎
\end{malayalam}}
\flushright{\begin{Arabic}
\quranayah[2][149]
\end{Arabic}}
\flushleft{\begin{malayalam}
നീ ഏതുവഴിയില്‍ സഞ്ചരിച്ചാലും ‎മസ്ജിദുല്‍ഹറാമിന്റെ നേരെ മുഖം തിരിക്കുക. ‎കാരണം അത് നിന്റെ നാഥനില്‍ നിന്നുള്ള ‎സത്യനിഷ്ഠമായ നിര്‍ദേശമാണ്. നിങ്ങള്‍ ‎പ്രവര്‍ത്തിക്കുന്നതിനെപ്പറ്റി അല്ലാഹു ഒട്ടും അശ്രദ്ധനല്ല. ‎
\end{malayalam}}
\flushright{\begin{Arabic}
\quranayah[2][150]
\end{Arabic}}
\flushleft{\begin{malayalam}
നീ എവിടെനിന്നു പുറപ്പെട്ടാലും നിന്റെ മുഖം ‎മസ്ജിദുല്‍ഹറാമിന്റെ നേരെ തിരിക്കുക. നിങ്ങള്‍ ‎എവിടെയായിരുന്നാലും അതിന്റെ നേര്‍ക്കാണ് മുഖം ‎തിരിക്കേണ്ടത്. നിങ്ങള്‍ക്കെതിരെ ജനങ്ങള്‍ക്ക് ഒരു ‎ന്യായവും ഇല്ലാതിരിക്കാനാണിത്. അവരിലെ ‎അതിക്രമികള്‍ക്കൊഴികെ. നിങ്ങളവരെ പേടിക്കരുത്. ‎എന്നെ മാത്രം ഭയപ്പെടുക. എന്റെ അനുഗ്രഹം നിങ്ങള്‍ക്ക് ‎തികവോടെ തരാനാണിത്; നിങ്ങള്‍ നേര്‍വഴി ‎പ്രാപിക്കാനും. ‎
\end{malayalam}}
\flushright{\begin{Arabic}
\quranayah[2][151]
\end{Arabic}}
\flushleft{\begin{malayalam}
നാം നിങ്ങള്‍ക്ക് നിങ്ങളില്‍ നിന്നുതന്നെ ദൂതനെ ‎അയച്ചുതന്നപോലെയാണിത്. അദ്ദേഹമോ നിങ്ങള്‍ക്ക് ‎നമ്മുടെ സൂക്തങ്ങള്‍ ഓതിത്തരുന്നു. നിങ്ങളെ ‎സംസ്കരിക്കുന്നു. വേദവും വിജ്ഞാനവും പഠിപ്പിക്കുന്നു. ‎നിങ്ങള്‍ക്ക് അറിയാത്ത കാര്യങ്ങള്‍ നിങ്ങള്‍ക്ക് ‎അറിയിച്ചുതരികയും ചെയ്യുന്നു. ‎
\end{malayalam}}
\flushright{\begin{Arabic}
\quranayah[2][152]
\end{Arabic}}
\flushleft{\begin{malayalam}
അതിനാല്‍ നിങ്ങള്‍ എന്നെ ഓര്‍ക്കുക. ഞാന്‍ നിങ്ങളെയും ‎ഓര്‍ക്കാം. എന്നോടു നന്ദി കാണിക്കുക. നന്ദികേട് ‎കാണിക്കരുത്. ‎
\end{malayalam}}
\flushright{\begin{Arabic}
\quranayah[2][153]
\end{Arabic}}
\flushleft{\begin{malayalam}
വിശ്വസിച്ചവരേ, നിങ്ങള്‍ ക്ഷമയിലൂടെയും ‎നമസ്കാരത്തിലൂടെയും ദിവ്യസഹായം തേടുക. ‎തീര്‍ച്ചയായും ക്ഷമിക്കുന്നവരോടൊപ്പമാണ് അല്ലാഹു. ‎
\end{malayalam}}
\flushright{\begin{Arabic}
\quranayah[2][154]
\end{Arabic}}
\flushleft{\begin{malayalam}
ദൈവമാര്‍ഗത്തില്‍ വധിക്കപ്പെടുന്നവരെ “മരിച്ചവരെ"ന്ന് ‎പറയാതിരിക്കുക. അല്ല; അവര്‍ ജീവിച്ചിരിക്കുന്നവരാണ്. ‎പക്ഷേ, നിങ്ങളത് അറിയുന്നില്ല. ‎
\end{malayalam}}
\flushright{\begin{Arabic}
\quranayah[2][155]
\end{Arabic}}
\flushleft{\begin{malayalam}
ഭയം, പട്ടിണി, ജീവധനാദികളുടെ നഷ്ടം, വിളനാശം ‎എന്നിവയിലൂടെ നാം നിങ്ങളെ പരീക്ഷിക്കുകതന്നെ ‎ചെയ്യും. അപ്പോഴൊക്കെ ക്ഷമിക്കുന്നവരെ ശുഭവാര്‍ത്ത ‎അറിയിക്കുക. ‎
\end{malayalam}}
\flushright{\begin{Arabic}
\quranayah[2][156]
\end{Arabic}}
\flushleft{\begin{malayalam}
ഏതൊരു വിപത്തു വരുമ്പോഴും അവര്‍ പറയും: ‎‎“ഞങ്ങള്‍ അല്ലാഹുവിന്റേതാണ്. അവനിലേക്കുതന്നെ ‎തിരിച്ചുചെല്ലേണ്ടവരും." ‎
\end{malayalam}}
\flushright{\begin{Arabic}
\quranayah[2][157]
\end{Arabic}}
\flushleft{\begin{malayalam}
അവര്‍ക്ക് അവരുടെ നാഥനില്‍ നിന്നുള്ള അതിരറ്റ ‎അനുഗ്രഹങ്ങളും കാരുണ്യവുമുണ്ട്. അവര്‍ തന്നെയാണ് ‎നേര്‍വഴി പ്രാപിച്ചവര്‍. ‎
\end{malayalam}}
\flushright{\begin{Arabic}
\quranayah[2][158]
\end{Arabic}}
\flushleft{\begin{malayalam}
തീര്‍ച്ചയായും സ്വഫായും മര്‍വ യും അല്ലാഹുവിന്റെ ‎അടയാള ങ്ങളില്‍പെട്ടവയാണ്. അതിനാല്‍ ‎അല്ലാഹുവിന്റെ ആദരണീയ ഭവന ത്തിങ്കല്‍ ഹജ്ജോ ‎ഉംറയോ നിര്‍വഹിക്കുന്നവര്‍ അവയ്ക്കിടയില്‍ ‎പ്രയാണം നടത്തുന്നത് കുറ്റകരമാകുന്ന പ്രശ്നമേയില്ല. ‎സ്വയം സന്നദ്ധരായി സുകൃതം ചെയ്യുന്നവര്‍ ‎മനസ്സിലാക്കട്ടെ: അല്ലാഹു എല്ലാം അറിയുന്നവനും ‎നന്ദിയുള്ളവനുമാണ്. ‎
\end{malayalam}}
\flushright{\begin{Arabic}
\quranayah[2][159]
\end{Arabic}}
\flushleft{\begin{malayalam}
നാം അവതരിപ്പിച്ച വ്യക്തമായ തെളിവുകളും ‎മാര്‍ഗനിര്‍ദേശങ്ങളും വേദപുസ്തകത്തിലൂടെ ‎വിശദീകരിച്ചിരിക്കുന്നു. എന്നിട്ടും അവയെ ‎മറച്ചുവെക്കുന്നവരെ ഉറപ്പായും അല്ലാഹു ശപിക്കുന്നു. ‎ശപിക്കുന്നവരൊക്കെയും അവരെ ശപിക്കുന്നു. ‎
\end{malayalam}}
\flushright{\begin{Arabic}
\quranayah[2][160]
\end{Arabic}}
\flushleft{\begin{malayalam}
പശ്ചാത്തപിക്കുകയും നിലപാട് നന്നാക്കിത്തീര്‍ക്കുകയും ‎മറച്ചുവെച്ചത് വിശദീകരിച്ചുകൊടുക്കുകയും ‎ചെയ്യുന്നവരെയൊഴികെ. അവരുടെ പശ്ചാത്താപം ഞാന്‍ ‎സ്വീകരിക്കുന്നു. ഞാന്‍ പശ്ചാത്താപം സ്വീകരിക്കുന്നവനും ‎ദയാപരനും തന്നെ. ‎
\end{malayalam}}
\flushright{\begin{Arabic}
\quranayah[2][161]
\end{Arabic}}
\flushleft{\begin{malayalam}
സത്യത്തെ തള്ളിക്കളയുകയും ‎സത്യനിഷേധികളായിത്തന്നെ മരണമടയുകയും ‎ചെയ്യുന്നവര്‍ക്ക് അല്ലാഹുവിന്റെയും മലക്കുകളുടെയും ‎മുഴുവന്‍ മനുഷ്യരുടെയും ശാപമുണ്ട്. ‎
\end{malayalam}}
\flushright{\begin{Arabic}
\quranayah[2][162]
\end{Arabic}}
\flushleft{\begin{malayalam}
അവരത് എക്കാലവും അനുഭവിക്കും. അവര്‍ക്ക് ‎ശിക്ഷയിലൊട്ടും ഇളവുണ്ടാവില്ല. മറ്റൊരവസരം ‎അവര്‍ക്ക് ലഭിക്കുകയുമില്ല. ‎
\end{malayalam}}
\flushright{\begin{Arabic}
\quranayah[2][163]
\end{Arabic}}
\flushleft{\begin{malayalam}
നിങ്ങളുടെ ദൈവം ഏകദൈവം. അവനല്ലാതെ ദൈവമില്ല. ‎അവന്‍ പരമ കാരുണികന്‍. ദയാപരന്‍. ‎
\end{malayalam}}
\flushright{\begin{Arabic}
\quranayah[2][164]
\end{Arabic}}
\flushleft{\begin{malayalam}
ആകാശഭൂമികളുടെ സൃഷ്ടിപ്പില്‍; രാപ്പകലുകള്‍ മാറിമാറി ‎വരുന്നതില്‍; മനുഷ്യര്‍ക്കുപകരിക്കുന്ന ചരക്കുകളുമായി ‎സമുദ്രത്തില്‍ സഞ്ചരിക്കുന്ന കപ്പലുകളില്‍; അല്ലാഹു ‎മാനത്തുനിന്ന് മഴവീഴ്ത്തി അതുവഴി, ജീവനറ്റ ഭൂമിക്ക് ‎ജീവനേകുന്നതില്‍; ഭൂമിയില്‍ എല്ലായിനം ജീവികളെയും ‎പരത്തിവിടുന്നതില്‍; കാറ്റിനെ ചലിപ്പിക്കുന്നതില്‍; ‎ആകാശഭൂമികള്‍ക്കിടയില്‍ ആജ്ഞാനുവര്‍ത്തിയായി ‎നിര്‍ത്തിയിട്ടുള്ള കാര്‍മേഘത്തില്‍; എല്ലാറ്റിലും ‎ചിന്തിക്കുന്ന ജനത്തിന് അനേകം തെളിവുകളുണ്ട്; ‎സംശയമില്ല. ‎
\end{malayalam}}
\flushright{\begin{Arabic}
\quranayah[2][165]
\end{Arabic}}
\flushleft{\begin{malayalam}
ചിലയാളുകള്‍ അല്ലാഹു അല്ലാത്തവരെ അവന്ന് ‎സമന്മാരാക്കിവെക്കുന്നു. അവര്‍ അല്ലാഹുവെ ‎സ്നേഹിക്കുന്നപോലെ ഇവരെയും സ്നേഹിക്കുന്നു. ‎സത്യവിശ്വാസികളോ, പരമമായി സ്നേഹിക്കുന്നത് ‎അല്ലാഹുവിനെയാണ്. അക്രമികള്‍ക്ക് പരലോകശിക്ഷ ‎നേരില്‍ കാണുമ്പോള്‍ ബോധ്യമാകും, ശക്തിയൊക്കെയും ‎അല്ലാഹുവിനാണെന്നും അവന്‍ കഠിനമായി ‎ശിക്ഷിക്കുന്നവനാണെന്നും. ഇക്കാര്യം ഇപ്പോള്‍ തന്നെ ‎അവര്‍ കണ്ടറിഞ്ഞിരുന്നെങ്കില്‍. ‎
\end{malayalam}}
\flushright{\begin{Arabic}
\quranayah[2][166]
\end{Arabic}}
\flushleft{\begin{malayalam}
പിന്തുടരപ്പെട്ടവര്‍ പിന്തുടരുന്നവരി ല്‍നിന്ന് ‎ഒഴിഞ്ഞുമാറുകയും ശിക്ഷ നേരില്‍ കാണുകയും ‎അന്യോന്യമുള്ള ബന്ധം അറ്റുപോവുകയും ചെയ്യുന്ന ‎സന്ദര്‍ഭം! ‎
\end{malayalam}}
\flushright{\begin{Arabic}
\quranayah[2][167]
\end{Arabic}}
\flushleft{\begin{malayalam}
അനുയായികള്‍ അന്ന് പറയും: "ഞങ്ങള്‍ക്ക് ഒരു ‎തിരിച്ചുപോക്കിന് അവസരമുണ്ടായെങ്കില്‍ ഇവരിപ്പോള്‍ ‎ഞങ്ങളെ കൈവെടിഞ്ഞപോലെ ഇവരെ ഞങ്ങളും ‎കൈവെടിയുമായിരുന്നു." അങ്ങനെ അവരുടെ ‎ചെയ്തികള്‍ അവര്‍ക്ക് കൊടിയ ഖേദത്തിന് ‎കാരണമായതായി അല്ലാഹു അവര്‍ക്ക് ‎കാണിച്ചുകൊടുക്കും. നരകത്തീയില്‍നിന്നവര്‍ക്ക് ‎പുറത്തുകടക്കാനാവില്ല. ‎
\end{malayalam}}
\flushright{\begin{Arabic}
\quranayah[2][168]
\end{Arabic}}
\flushleft{\begin{malayalam}
മനുഷ്യരേ, ഭൂമിയിലെ വിഭവങ്ങളില്‍ അനുവദനീയവും ‎ഉത്തമവുമായത് തിന്നുകൊള്ളുക. പിശാചിന്റെ ‎കാല്‍പ്പാടുകളെ പിന്‍പറ്റരുത്. അവന്‍ നിങ്ങളുടെ ‎പ്രത്യക്ഷ ശത്രുവാണ്. ‎
\end{malayalam}}
\flushright{\begin{Arabic}
\quranayah[2][169]
\end{Arabic}}
\flushleft{\begin{malayalam}
ചീത്തകാര്യങ്ങളിലും നീചവൃത്തികളിലും ‎വ്യാപരിക്കാനാണ് അവന്‍ നിങ്ങളോട് കല്‍പിക്കുന്നത്. ‎ദൈവത്തിന്റെ പേരില്‍ നിങ്ങള്‍ക്കറിയാത്ത കാര്യങ്ങള്‍ ‎കെട്ടിപ്പറയാനും. ‎
\end{malayalam}}
\flushright{\begin{Arabic}
\quranayah[2][170]
\end{Arabic}}
\flushleft{\begin{malayalam}
അല്ലാഹു ഇറക്കിത്തന്ന സന്ദേശം പിന്‍പറ്റാന്‍ ‎ആവശ്യപ്പെട്ടാല്‍ അവര്‍ പറയും: "ഞങ്ങളുടെ പൂര്‍വ ‎പിതാക്കള്‍ പിന്തുടര്‍ന്നുകണ്ട പാതയേ ഞങ്ങള്‍ ‎പിന്‍പറ്റുകയുള്ളൂ." അവരുടെ പിതാക്കള്‍ ‎ചിന്തിക്കുകയോ നേര്‍വഴി പ്രാപിക്കുകയോ ‎ചെയ്യാത്തവരായിരുന്നിട്ടും! ‎
\end{malayalam}}
\flushright{\begin{Arabic}
\quranayah[2][171]
\end{Arabic}}
\flushleft{\begin{malayalam}
സത്യനിഷേധികളോടു സംസാരിക്കുന്നവന്റെ ഉപമ ‎വിളിയും തെളിയുമല്ലാതൊന്നും കേള്‍ക്കാത്ത ‎കാലികളോട് ഒച്ചയിടുന്ന ഇടയനെ പോലെയാണ്. അവര്‍ ‎ബധിരരും മൂകരും കുരുടരുമാണ്. അവരൊന്നും ‎ആലോചിച്ചറിയുന്നില്ല. ‎
\end{malayalam}}
\flushright{\begin{Arabic}
\quranayah[2][172]
\end{Arabic}}
\flushleft{\begin{malayalam}
വിശ്വസിച്ചവരേ, നാം നിങ്ങള്‍ക്കേകിയ ‎വിഭവങ്ങളില്‍നിന്ന് വിശിഷ്ടമായത് ആഹരിക്കുക. ‎അല്ലാഹുവോട് നന്ദി കാണിക്കുക. നിങ്ങള്‍ അവനുമാത്രം ‎വഴിപ്പെടുന്നവരാണെങ്കില്‍! ‎
\end{malayalam}}
\flushright{\begin{Arabic}
\quranayah[2][173]
\end{Arabic}}
\flushleft{\begin{malayalam}
നിങ്ങള്‍ക്ക് അവന്‍ നിഷിദ്ധമാക്കിയത് ഇവ മാത്രമാണ്: ‎ശവം, രക്തം, പന്നിമാംസം, അല്ലാഹുവല്ലാത്തവരുടെ ‎പേരില്‍ അറുക്കപ്പെട്ടത്. എന്നാല്‍ ‎നിര്‍ബന്ധിതാവസ്ഥയിലുള്ളവന് അതില്‍ ഇളവുണ്ട്. ‎പക്ഷേ ഇത് നിയമലംഘനമാഗ്രഹിച്ചാവരുത്. ‎അത്യാവശ്യത്തിലധികവുമാവരുത്. അല്ലാഹു ഏറെ ‎പൊറുക്കുന്നവനും ദയാപരനുമല്ലോ. ‎
\end{malayalam}}
\flushright{\begin{Arabic}
\quranayah[2][174]
\end{Arabic}}
\flushleft{\begin{malayalam}
വേദഗ്രന്ഥത്തില്‍ അല്ലാഹു അവതരിപ്പിച്ച കാര്യങ്ങള്‍ ‎മറച്ചുപിടിക്കുകയും അതിനു വിലയായി തുച്ഛമായ ‎ഐഹികതാല്‍പര്യങ്ങള്‍ നേടിയെടുക്കുകയും ‎ചെയ്യുന്നവര്‍, തങ്ങളുടെ വയറുകളില്‍ തിന്നുനിറക്കുന്നത് ‎നരകത്തീയല്ലാതൊന്നുമല്ല. ഉയിര്‍ത്തെഴുന്നേല്‍പുനാളില്‍ ‎അല്ലാഹു അവരോട് മിണ്ടുകയില്ല. അവരെ ‎ശുദ്ധീകരിക്കുകയുമില്ല. അവര്‍ക്ക് നോവേറിയ ‎ശിക്ഷയുണ്ട്. ‎
\end{malayalam}}
\flushright{\begin{Arabic}
\quranayah[2][175]
\end{Arabic}}
\flushleft{\begin{malayalam}
സന്മാര്‍ഗം വിറ്റ് ദുര്‍മാര്‍ഗം വാങ്ങിയവരാണവര്‍. ‎പാപമോചനത്തിനുപകരം ശിക്ഷയും. നരകശിക്ഷ ‎ഏറ്റുവാങ്ങാനുള്ള അവരുടെ ധാര്‍ഷ്ട്യം അപാരം തന്നെ! ‎
\end{malayalam}}
\flushright{\begin{Arabic}
\quranayah[2][176]
\end{Arabic}}
\flushleft{\begin{malayalam}
പരമസത്യം വ്യക്തമാക്കുന്ന വേദപുസ്തകം അല്ലാഹു ‎ഇറക്കിത്തന്നു. എന്നിട്ടും വേദഗ്രന്ഥത്തിന്റെ കാര്യത്തില്‍ ‎ഭിന്നിച്ചവര്‍ തങ്ങളുടെ മാത്സര്യത്തില്‍പെട്ട് ‎സത്യത്തില്‍നിന്ന് ഏറെ ദൂരെയായിരിക്കുന്നു. അതാണ് ‎ഇതിനൊക്കെയും കാരണം. ‎
\end{malayalam}}
\flushright{\begin{Arabic}
\quranayah[2][177]
\end{Arabic}}
\flushleft{\begin{malayalam}
നിങ്ങള്‍ കിഴക്കോട്ടോ പടിഞ്ഞാറോട്ടോ ‎മുഖംതിരിക്കുന്നതല്ല പുണ്യം. പിന്നെയോ, ‎അല്ലാഹുവിലും അന്ത്യദിനത്തിലും മലക്കുകളിലും ‎വേദഗ്രന്ഥത്തിലും പ്രവാചകന്മാരിലും വിശ്വസിക്കുക; ‎സമ്പത്തിനോട് ഏറെ പ്രിയമുണ്ടായിരിക്കെ അത് ‎അടുത്ത ബന്ധുക്കള്‍ക്കും അനാഥകള്‍ക്കും അഗതികള്‍ക്കും ‎വഴിയാത്രക്കാര്‍ക്കും ചോദിച്ചുവരുന്നവര്‍ക്കും അടിമ ‎മോചനത്തിനും ചെലവഴിക്കുക; നമസ്കാരം ‎നിഷ്ഠയോടെ നിര്‍വഹിക്കുക; സകാത്ത് നല്‍കുക; ‎കരാറുകളിലേര്‍പ്പെട്ടാലവ പാലിക്കുക; ‎പ്രതിസന്ധികളിലും വിപദ്ഘട്ടങ്ങളിലും യുദ്ധരംഗത്തും ‎ക്ഷമ പാലിക്കുക; ഇങ്ങനെ ചെയ്യുന്നവരാണ് ‎പുണ്യവാന്മാര്‍. അവരാണ് സത്യം പാലിച്ചവര്‍. അവര്‍ ‎തന്നെയാണ് യഥാര്‍ഥ ഭക്തന്മാര്‍. ‎
\end{malayalam}}
\flushright{\begin{Arabic}
\quranayah[2][178]
\end{Arabic}}
\flushleft{\begin{malayalam}
വിശ്വസിച്ചവരേ, കൊല്ലപ്പെടുന്നവരുടെ കാര്യത്തില്‍ ‎പ്രതിക്രിയ നിങ്ങള്‍ക്ക് നിയമമാക്കിയിരിക്കുന്നു: ‎സ്വതന്ത്രന് സ്വതന്ത്രന്‍; അടിമക്ക് അടിമ; സ്ത്രീക്ക് സ്ത്രീ. ‎എന്നാല്‍ കൊലയാളിക്ക് തന്റെ സഹോദരനില്‍നിന്ന് ‎ഇളവു ലഭിക്കുകയാണെങ്കില്‍ മര്യാദ പാലിയില്‍ അതം ‎ഗീകരിക്കുകയും മാന്യമായ നഷ്ടപരിഹാരം നല്‍കുകയും ‎വേണം. നിങ്ങളുടെ നാഥനില്‍ നിന്നുള്ള ഒരിളവും ‎കാരുണ്യവുമാണിത്. പിന്നെയും പരിധി വിടുന്നവര്‍ക്ക് ‎നോവേറിയ ശിക്ഷയുണ്ട്. ‎
\end{malayalam}}
\flushright{\begin{Arabic}
\quranayah[2][179]
\end{Arabic}}
\flushleft{\begin{malayalam}
ബുദ്ധിശാലികളേ, പ്രതിക്രിയയില്‍ നിങ്ങള്‍ക്കു ‎ജീവിതമുണ്ട്. നിങ്ങള്‍ ഭക്തിയുള്ളവരാകാനാണിത്. ‎
\end{malayalam}}
\flushright{\begin{Arabic}
\quranayah[2][180]
\end{Arabic}}
\flushleft{\begin{malayalam}
നിങ്ങളിലാര്‍ക്കെങ്കിലും മരണമടുത്തുവെന്നറിഞ്ഞാല്‍ ‎നിങ്ങള്‍ക്കു ശേഷിപ്പു സ്വത്തുണ്ടെങ്കില്‍ ‎മാതാപിതാക്കള്‍ക്കും അടുത്ത ബന്ധുക്കള്‍ക്കും ന്യായമായ ‎നിലയില്‍ ഒസ്യത്ത് ചെയ്യാന്‍ നിങ്ങള്‍ ബാധ്യസ്ഥരാണ്. ‎ഭക്തന്മാര്‍ക്കിത് ഒഴിച്ചുകൂടാനാവാത്ത കടമയത്രെ. ‎
\end{malayalam}}
\flushright{\begin{Arabic}
\quranayah[2][181]
\end{Arabic}}
\flushleft{\begin{malayalam}
ഒസ്യത്ത് കേട്ടശേഷം ആരെങ്കിലും അത് മാറ്റിമറിച്ചാല്‍ ‎കുറ്റം മാറ്റിമറിച്ചവര്‍ക്കാണ്. നിസ്സംശയം, അല്ലാഹു ‎എല്ലാം കേള്‍ക്കുന്നവനും അറിയുന്നവനുമാണ്. ‎
\end{malayalam}}
\flushright{\begin{Arabic}
\quranayah[2][182]
\end{Arabic}}
\flushleft{\begin{malayalam}
അഥവാ, ഒസ്യത്ത് ചെയ്തവനില്‍നിന്ന് വല്ല അനീതിയോ ‎തെറ്റോ സംഭവിച്ചതായി ആരെങ്കിലും ‎ആശങ്കിക്കുന്നുവെങ്കില്‍ അയാള്‍ ബന്ധപ്പെട്ടവര്‍ക്കിടയില്‍ ‎ഒത്തുതീര്‍പ്പുണ്ടാക്കുന്നതില്‍ തെറ്റൊന്നുമില്ല. അല്ലാഹു ‎ഏറെ പൊറുക്കുന്നവനും ദയാപരനും തന്നെ. ‎
\end{malayalam}}
\flushright{\begin{Arabic}
\quranayah[2][183]
\end{Arabic}}
\flushleft{\begin{malayalam}
വിശ്വസിച്ചവരേ, നിങ്ങള്‍ക്ക് നോമ്പ് ‎നിര്‍ബന്ധമാക്കിയിരിക്കുന്നു. നിങ്ങളുടെ ‎മുമ്പുണ്ടായിരുന്നവര്‍ക്ക് നിര്‍ബന്ധമാക്കിയിരുന്ന ‎പോലെത്തന്നെ. നിങ്ങള്‍ ഭക്തിയുള്ളവരാകാന്‍. ‎
\end{malayalam}}
\flushright{\begin{Arabic}
\quranayah[2][184]
\end{Arabic}}
\flushleft{\begin{malayalam}
നിര്‍ണിതമായ ഏതാനും ദിനങ്ങളില്‍. നിങ്ങളാരെങ്കിലും ‎രോഗിയോ യാത്രക്കാരനോ ആണെങ്കില്‍ മറ്റു ‎ദിവസങ്ങളില്‍ അത്രയും എണ്ണം തികയ്ക്കണം. ഏറെ ‎പ്രയാസത്തോടെ മാത്രം നോമ്പെടുക്കാന്‍ കഴിയുന്നവര്‍ ‎നോമ്പുപേക്ഷിച്ചാല്‍ പകരം പ്രായശ്ചിത്തമായി ‎ഒരഗതിക്ക് ആഹാരം നല്‍കണം. എന്നാല്‍ ആരെങ്കിലും ‎സ്വയം കൂടുതല്‍ നന്മ ചെയ്താല്‍ അതവന് നല്ലതാണ്. ‎നോമ്പെടുക്കലാണ് നിങ്ങള്‍ക്കുത്തമം. നിങ്ങള്‍ ‎തിരിച്ചറിയുന്നവരെങ്കില്‍. ‎
\end{malayalam}}
\flushright{\begin{Arabic}
\quranayah[2][185]
\end{Arabic}}
\flushleft{\begin{malayalam}
ഖുര്‍ആന്‍ ഇറങ്ങിയ മാസമാണ് റമദാന്‍. അത് ‎ജനങ്ങള്‍ക്കു നേര്‍വഴി കാണിക്കുന്നതാണ്. സത്യമാര്‍ഗം ‎വിശദീകരിക്കുന്നതും സത്യാസത്യങ്ങളെ ‎വേര്‍തിരിച്ചുകാണിക്കുന്നതുമാണ്. അതിനാല്‍ ‎നിങ്ങളിലാരെങ്കിലും ആ മാസത്തിന് ‎സാക്ഷികളാകുന്നുവെങ്കില്‍ ആ മാസം ‎വ്രതമനുഷ്ഠിക്കണം. ആരെങ്കിലും രോഗത്തിലോ ‎യാത്രയിലോ ആണെങ്കില്‍ പകരം മറ്റു ദിവസങ്ങളില്‍നിന്ന് ‎അത്രയും എണ്ണം തികയ്ക്കണം. അല്ലാഹു നിങ്ങള്‍ക്ക് ‎എളുപ്പമാണാഗ്രഹിക്കുന്നത്. പ്രയാസമല്ല. നിങ്ങള്‍ ‎നോമ്പിന്റെ എണ്ണം പൂര്‍ത്തീകരിക്കാനാണിത്. നിങ്ങളെ ‎നേര്‍വഴിയിലാക്കിയതിന്റെ പേരില്‍ നിങ്ങള്‍ ‎അല്ലാഹുവിന്റെ മഹത്വം കീര്‍ത്തിക്കാനും അവനോട് ‎നന്ദിയുള്ളവരാകാനുമാണിത്. ‎
\end{malayalam}}
\flushright{\begin{Arabic}
\quranayah[2][186]
\end{Arabic}}
\flushleft{\begin{malayalam}
എന്റെ ദാസന്മാര്‍ എന്നെപ്പറ്റി നിന്നോടു ചോദിച്ചാല്‍ ‎പറയുക: ഞാന്‍ അടുത്തുതന്നെയുണ്ട്. എന്നോടു ‎പ്രാര്‍ത്ഥിച്ചാല്‍ പ്രാര്‍ഥിക്കുന്നവന്റെ പ്രാര്‍ഥനക്ക് ‎ഞാനുത്തരം നല്‍കും. അതിനാല്‍ അവരെന്റെ ‎വിളിക്കുത്തരം നല്‍കട്ടെ. എന്നില്‍ വിശ്വസിക്കുകയും ‎ചെയ്യട്ടെ. അവര്‍ നേര്‍വഴിയിലായേക്കാം. ‎
\end{malayalam}}
\flushright{\begin{Arabic}
\quranayah[2][187]
\end{Arabic}}
\flushleft{\begin{malayalam}
നോമ്പിന്റെ രാവില്‍ നിങ്ങളുടെ ഭാര്യമാരുമായുള്ള ‎ലൈംഗികബന്ധം നിങ്ങള്‍ക്ക് അനുവദിച്ചിരിക്കുന്നു. ‎അവര്‍ നിങ്ങള്‍ക്കുള്ള വസ്ത്രമാണ്; നിങ്ങള്‍ അവര്‍ക്കുള്ള ‎വസ്ത്രവും. നിങ്ങള്‍ നിങ്ങളെത്തന്നെ വഞ്ചിക്കുക ‎യായിരുന്നുവെന്ന് അല്ലാഹു അറിഞ്ഞിട്ടുണ്ട്. എന്നാല്‍ ‎അല്ലാഹു നിങ്ങളുടെ പശ്ചാത്താപം സ്വീകരിച്ചിരിക്കുന്നു. ‎നിങ്ങള്‍ക്ക് മാപ്പേകിയിരിക്കുന്നു. ഇനിമുതല്‍ നിങ്ങള്‍ ‎അവരുമായി സഹവസിക്കുക. അല്ലാഹു അതിലൂടെ ‎നിങ്ങള്‍ക്കനുവദിച്ചത് തേടുക. അപ്രകാരംതന്നെ ‎തിന്നുകയും കുടിക്കുകയും ചെയ്യുക. പ്രഭാതത്തിന്റെ ‎വെള്ള ഇഴകള്‍ കറുപ്പ് ഇഴകളില്‍നിന്ന് വേര്‍തിരിഞ്ഞു ‎കാണുംവരെ. പിന്നെ എല്ലാം വര്‍ജിച്ച് രാവുവരെ ‎വ്രതമാചരിക്കുക. നിങ്ങള്‍ പള്ളികളില്‍ ‎ഭജനമിരിക്കുമ്പോള്‍ ഭാര്യമാരുമായി വേഴ്ച പാടില്ല. ‎ഇതൊക്കെയും അല്ലാഹുവിന്റെ അതിര്‍വരമ്പുകളാണ്. ‎അതിനാല്‍ നിങ്ങളവയോടടുക്കരുത്. ഇവ്വിധം അല്ലാഹു ‎അവന്റെ വചനങ്ങള്‍ ജനങ്ങള്‍ക്ക് ‎വിവരിച്ചുകൊടുക്കുന്നു. അവര്‍ സൂക്ഷ്മത ‎പാലിക്കുന്നവരാകാന്‍. ‎
\end{malayalam}}
\flushright{\begin{Arabic}
\quranayah[2][188]
\end{Arabic}}
\flushleft{\begin{malayalam}
നിങ്ങളുടെ ധനം നിങ്ങള്‍ അന്യായമായി അന്യോന്യം ‎അധീനപ്പെടുത്തി ആഹരിക്കരുത്. ബോധപൂര്‍വം ‎കുറ്റകരമായ മാര്‍ഗത്തിലൂടെ അന്യരുടെ സ്വത്തില്‍നിന്ന് ‎ഒരു ഭാഗം തിന്നാനായി നിങ്ങള്‍ അതുമായി ‎ഭരണാധികാരികളെ സമീപിക്കുകയുമരുത്. ‎
\end{malayalam}}
\flushright{\begin{Arabic}
\quranayah[2][189]
\end{Arabic}}
\flushleft{\begin{malayalam}
അവര്‍ നിന്നോട് ചന്ദ്രക്കലയെക്കുറിച്ചു ചോദിക്കുന്നു. ‎പറയുക: അത് ജനങ്ങള്‍ക്ക് കാലം ‎കണക്കാക്കാനുള്ളതാണ്. ഹജ്ജിനുള്ള അടയാളവും. ‎നിങ്ങള്‍ നിങ്ങളുടെ വീടുകളില്‍ പിന്‍ഭാഗത്തൂടെ ‎പ്രവേശിക്കുന്നതില്‍ പുണ്യമൊന്നുമില്ല. അല്ലാഹുവെ ‎സൂക്ഷിച്ചു ജീവിക്കുന്നതിലാണ് യഥാര്‍ഥ പുണ്യം. ‎അതിനാല്‍ വീടുകളില്‍ മുന്‍വാതിലുകളിലൂടെ തന്നെ ‎പ്രവേശിക്കുക. അല്ലാഹുവോട് ഭക്തിപുലര്‍ത്തുക. ‎എങ്കില്‍ നിങ്ങള്‍ക്കു വിജയം വരിക്കാം. ‎
\end{malayalam}}
\flushright{\begin{Arabic}
\quranayah[2][190]
\end{Arabic}}
\flushleft{\begin{malayalam}
നിങ്ങളോട് യുദ്ധം ചെയ്യുന്നവരോട് ദൈവമാര്‍ഗത്തില്‍ ‎നിങ്ങളും യുദ്ധം ചെയ്യുക. എന്നാല്‍ പരിധി ‎ലംഘിക്കരുത്. അതിക്രമികളെ അല്ലാഹു ഇഷ്ടപ്പെടുന്നില്ല. ‎
\end{malayalam}}
\flushright{\begin{Arabic}
\quranayah[2][191]
\end{Arabic}}
\flushleft{\begin{malayalam}
ഏറ്റുമുട്ടുന്നത് എവിടെവെച്ചായാലും നിങ്ങളവരെ ‎വധിക്കുക. അവര്‍ നിങ്ങളെ പുറത്താക്കിയിടത്തുനിന്ന് ‎നിങ്ങളവരെയും പുറന്തള്ളുക. മര്‍ദനം കൊലയെക്കാള്‍ ‎ഭീകരമാണ്. മസ്ജിദുല്‍ ഹറാമിനടുത്തുവെച്ച് അവര്‍ ‎നിങ്ങളോടേറ്റുമുട്ടുന്നില്ലെങ്കില്‍ അവിടെ വെച്ച് നിങ്ങള്‍ ‎അവരോട് യുദ്ധം ചെയ്യരുത്. അഥവാ, അവര്‍ നിങ്ങളോടു ‎യുദ്ധം ചെയ്യുകയാണെങ്കില്‍ നിങ്ങളവരെ വധിക്കുക. ‎അതാണ് അത്തരം സത്യനിഷേധികള്‍ക്കുള്ള പ്രതിഫലം. ‎
\end{malayalam}}
\flushright{\begin{Arabic}
\quranayah[2][192]
\end{Arabic}}
\flushleft{\begin{malayalam}
എന്നാല്‍ അവര്‍ വിരമിക്കുകയാണെങ്കിലോ, അറിയുക: ‎അല്ലാഹു ഏറെ മാപ്പേകുന്നവനും ദയാമയനുമാകുന്നു. ‎
\end{malayalam}}
\flushright{\begin{Arabic}
\quranayah[2][193]
\end{Arabic}}
\flushleft{\begin{malayalam}
മര്‍ദനം ഇല്ലാതാവുകയും “ദീന്‍" ‎അല്ലാഹുവിന്റേതായിത്തീരുക യും ചെയ്യുന്നതുവരെ ‎നിങ്ങളവരോടു യുദ്ധം ചെയ്യുക. എന്നാല്‍ അവര്‍ ‎വിരമിക്കുക യാണെങ്കില്‍ അറിയുക: ‎അതിക്രമികളോടല്ലാതെ ഒരുവിധ കയ്യേറ്റവും പാടില്ല. ‎
\end{malayalam}}
\flushright{\begin{Arabic}
\quranayah[2][194]
\end{Arabic}}
\flushleft{\begin{malayalam}
ആദരണീയ മാസത്തിനുപകരം ആദരണീയ മാസം തന്നെ. ‎ആദരണീയമായ മറ്റു കാര്യങ്ങള്‍ ‎കയ്യേറ്റത്തിനിരയായാലും അവ്വിധം പ്രതിക്രിയയുണ്ട്. ‎അതിനാല്‍ നിങ്ങള്‍ക്കെതിരെ ആരെങ്കിലും ‎അക്രമമഴിച്ചുവിട്ടാല്‍ അതേവിധം നിങ്ങളവരെയും ‎നേരിടുക. അല്ലാഹുവെ സൂക്ഷിക്കുക. അറിയുക, ‎സൂക്ഷ്മത പുലര്‍ത്തുന്നവരോടൊപ്പമാണ് അല്ലാഹു. ‎
\end{malayalam}}
\flushright{\begin{Arabic}
\quranayah[2][195]
\end{Arabic}}
\flushleft{\begin{malayalam}
അല്ലാഹുവിന്റെ മാര്‍ഗത്തില്‍ ചെലവഴിക്കുക. നിങ്ങള്‍ ‎നിങ്ങളുടെ കൈകളാല്‍ നിങ്ങളെത്തന്നെ ‎ആപത്തിലകപ്പെടുത്തരുത്. നന്മ ചെയ്യുക. തീര്‍ച്ചയായും ‎നന്മ ചെയ്യുന്നവരെ അല്ലാഹു സ്നേഹിക്കും. ‎
\end{malayalam}}
\flushright{\begin{Arabic}
\quranayah[2][196]
\end{Arabic}}
\flushleft{\begin{malayalam}
നിങ്ങള്‍ അല്ലാഹുവിനായി ഹജ്ജും ഉംറയും തികവോടെ ‎നിര്‍വഹിക്കുക. അഥവാ, നിങ്ങള്‍ ഉപരോധിക്കപ്പെട്ടാ ല്‍ ‎നിങ്ങള്‍ക്ക് സാധ്യമായ രീതിയില്‍ ബലിനടത്തുക. ‎ബലിമൃഗം അതിന്റെ സ്ഥാനത്ത് എത്തുവോളം നിങ്ങള്‍ ‎തലമുടിയെടുക്കരുത്. അഥവാ, ആരെങ്കിലും രോഗം ‎കാരണമോ തലയിലെ മറ്റെന്തെങ്കിലും പ്രയാസം മൂലമോ ‎മുടി എടുത്താല്‍ പ്രായശ്ചിത്തമായി നോമ്പെടുക്കുകയോ ‎ദാനം നല്‍കുകയോ ബലിനടത്തുകയോ വേണം. നിങ്ങള്‍ ‎നിര്‍ഭയാവസ്ഥയിലാവുകയും ഉംറ നിര്‍വഹിച്ച് ഹജ്ജ് ‎കാലംവരെ സൌകര്യം ഉപയോഗപ്പെടുത്തുക ‎യുമാണെങ്കില്‍ സാധ്യമായ ബലി നല്‍കുക. ‎ആര്‍ക്കെങ്കിലും ബലി സാധ്യമായില്ലെങ്കില്‍ പത്ത് നോമ്പ് ‎പൂര്‍ണമായി അനുഷ്ഠിക്കണം. മൂന്നെണ്ണം ഹജ്ജ് ‎വേളയിലും ഏഴെണ്ണം തിരിച്ചെത്തിയ ശേഷവും. ‎കുടുംബത്തോടൊത്ത് മസ്ജിദുല്‍ഹറാമിന്റെ അടുത്ത് ‎താമസിക്കാത്തവര്‍ക്കുള്ളതാണ് ഈ നിയമം. അല്ലാഹുവെ ‎സൂക്ഷിക്കുക. അറിയുക: അല്ലാഹു കഠിനമായി ‎ശിക്ഷിക്കുന്നവനാണ്. ‎
\end{malayalam}}
\flushright{\begin{Arabic}
\quranayah[2][197]
\end{Arabic}}
\flushleft{\begin{malayalam}
ഹജ്ജ്കാലം ഏറെ അറിയപ്പെടുന്ന മാസങ്ങളാണ്. ഈ ‎നിര്‍ണിത മാസങ്ങളില്‍ ആരെങ്കിലും ഹജ്ജില്‍ ‎പ്രവേശിച്ചാല്‍ പിന്നെ സ്ത്രീപുരുഷവേഴ്ചയോ ‎ദുര്‍വൃത്തിയോ വഴക്കോ പാടില്ല. നിങ്ങള്‍ എന്തു ‎സുകൃതം ചെയ്താലും അല്ലാഹു അതറിയുക തന്നെ ‎ചെയ്യും. നിങ്ങള്‍ യാത്രക്കാവശ്യമായ ‎വിഭവങ്ങളൊരുക്കുക. എന്നാല്‍ യാത്രക്കാവശ്യമായ ‎വിഭവങ്ങളിലേറ്റം ഉത്തമം ദൈവഭക്തിയത്രെ. ‎വിചാരശാലികളേ, നിങ്ങളെന്നോട് ഭക്തിയുള്ളവരാവുക. ‎
\end{malayalam}}
\flushright{\begin{Arabic}
\quranayah[2][198]
\end{Arabic}}
\flushleft{\begin{malayalam}
അതോടൊപ്പം നിങ്ങള്‍ക്ക് നിങ്ങളുടെ നാഥന്റെ ‎അനുഗ്രഹങ്ങള്‍ തേടുന്ന തില്‍ തെറ്റൊന്നുമില്ല. നിങ്ങള്‍ ‎അറഫ യില്‍ നിന്ന് മടങ്ങിക്കഴിഞ്ഞാല്‍ മശ്അറുല്‍ ‎ഹറാമി നടുത്തുവച്ച് അല്ലാഹുവെ സ്മരിക്കുക. അവന്‍ ‎നിങ്ങള്‍ക്ക് കാണിച്ചുതന്നപോലെ അവനെ ‎പ്രകീര്‍ത്തിക്കുകയും ചെയ്യുക. ഇതിനുമുമ്പ് നിങ്ങള്‍ ‎വഴിപിഴച്ചവരായിരുന്നല്ലോ. ‎
\end{malayalam}}
\flushright{\begin{Arabic}
\quranayah[2][199]
\end{Arabic}}
\flushleft{\begin{malayalam}
പിന്നീട് ആളുകള്‍ മടങ്ങുന്നതെവിടെനിന്നോ അവിടെനിന്ന് ‎നിങ്ങളും മടങ്ങുക. അല്ലാഹുവോട് പാപമോചനം ‎തേടുക. നിശ്ചയമായും അല്ലാഹു ഏറെ ‎പൊറുക്കുന്നവനും ദയാപരനും തന്നെ. ‎
\end{malayalam}}
\flushright{\begin{Arabic}
\quranayah[2][200]
\end{Arabic}}
\flushleft{\begin{malayalam}
നിങ്ങള്‍ ഹജ്ജ് കര്‍മങ്ങള്‍ നിര്‍വഹിച്ചുകഴിഞ്ഞാല്‍ ‎അല്ലാഹുവെ ഓര്‍ക്കുക. നിങ്ങള്‍ നിങ്ങളുടെ പിതാക്കളെ ‎ഓര്‍ക്കുംപോലെ. അല്ല, അതിലും കൂടുതലായി അവനെ ‎സ്മരിക്കുക. ചില ആളുകള്‍ പ്രാര്‍ഥിക്കുന്നു: "ഞങ്ങളുടെ ‎നാഥാ! ഞങ്ങള്‍ക്ക് നീ ഈ ലോകത്തുതന്നെ എല്ലാം ‎തരേണമേ." അവര്‍ക്ക് പരലോകത്ത് ഒന്നുമുണ്ടാവില്ല. ‎
\end{malayalam}}
\flushright{\begin{Arabic}
\quranayah[2][201]
\end{Arabic}}
\flushleft{\begin{malayalam}
മറ്റുചിലര്‍ പ്രാര്‍ഥിക്കുന്നു: "ഞങ്ങളുടെ നാഥാ! ‎ഞങ്ങള്‍ക്കു നീ ഈ ലോകത്ത് നന്മ നല്‍കേണമേ, ‎പരലോകത്തും നന്മ നല്‍കേണമേ, നരകശിക്ഷയില്‍ നിന്ന് ‎ഞങ്ങളെ നീ രക്ഷിക്കേണമേ." ‎
\end{malayalam}}
\flushright{\begin{Arabic}
\quranayah[2][202]
\end{Arabic}}
\flushleft{\begin{malayalam}
അവര്‍ സമ്പാദിച്ചതിന്റെ വിഹിതം അവര്‍ക്കുണ്ട്. ‎അല്ലാഹു അതിവേഗം കണക്കുനോക്കുന്നവനാകുന്നു. ‎
\end{malayalam}}
\flushright{\begin{Arabic}
\quranayah[2][203]
\end{Arabic}}
\flushleft{\begin{malayalam}
നിര്‍ണിതനാളുകളി ല്‍ നിങ്ങള്‍ ദൈവസ്മരണയില്‍ ‎മുഴുകുക. ആരെങ്കിലും ധൃതി കാണിച്ച് രണ്ടുദിവസം ‎കൊണ്ടുതന്നെ മതിയാക്കി മടങ്ങിയാല്‍, അതില്‍ ‎തെറ്റൊന്നുമില്ല. ആരെങ്കിലും പിന്തിമടങ്ങുന്നുവെങ്കില്‍ ‎അതിലും തെറ്റില്ല. ഭക്തിപുലര്‍ത്തുന്നവര്‍ക്കുള്ളതാണ് ഈ ‎നിയമം. നിങ്ങള്‍ അല്ലാഹുവോട് ഭക്തിയുള്ളവരാവുക. ‎അറിയുക: നിങ്ങളെല്ലാം അവന്റെ സന്നിധിയില്‍ ‎ഒരുമിച്ചു കൂട്ടപ്പെടുന്നവരാണ്. ‎
\end{malayalam}}
\flushright{\begin{Arabic}
\quranayah[2][204]
\end{Arabic}}
\flushleft{\begin{malayalam}
ചില മനുഷ്യരുണ്ട്. ഐഹിക ജീവിതത്തെ സംബന്ധിച്ച ‎അവരുടെ സംസാരം നിന്നില്‍ കൌതുകമുണര്‍ത്തും. ‎തങ്ങളുടെ ഉദ്ദേശ്യശുദ്ധി ബോധ്യപ്പെടുത്താന്‍ അവര്‍ ‎അല്ലാഹുവെ സാക്ഷിനിര്‍ത്തും. വാസ്തവത്തിലവര്‍ ‎സത്യത്തിന്റെ കൊടും വൈരികളത്രെ. ‎
\end{malayalam}}
\flushright{\begin{Arabic}
\quranayah[2][205]
\end{Arabic}}
\flushleft{\begin{malayalam}
അധികാരം ലഭിച്ചാല്‍ അവര്‍ ശ്രമിക്കുക ഭൂമിയില്‍ ‎കുഴപ്പമുണ്ടാക്കാനാണ്; കൃഷിനാശം വരുത്താനും ‎മനുഷ്യകുലത്തെ നശിപ്പിക്കാനുമാണ്. എന്നാല്‍ അല്ലാഹു ‎കുഴപ്പം ഇഷ്ടപ്പെടുന്നില്ല. ‎
\end{malayalam}}
\flushright{\begin{Arabic}
\quranayah[2][206]
\end{Arabic}}
\flushleft{\begin{malayalam}
‎“അല്ലാഹുവെ സൂക്ഷിക്കുക" എന്ന് അവനോട് ‎ആരെങ്കിലും പറഞ്ഞാല്‍ അഹങ്കാരം അവനെ ‎അതിനനുവദിക്കാതെ പാപത്തില്‍ തന്നെ ‎ഉറപ്പിച്ചുനിര്‍ത്തുന്നു. അവന് നരകം തന്നെ മതി. അത് ‎എത്ര ചീത്ത ഇടം! ‎
\end{malayalam}}
\flushright{\begin{Arabic}
\quranayah[2][207]
\end{Arabic}}
\flushleft{\begin{malayalam}
മറ്റുചില മനുഷ്യരുണ്ട്. അവര്‍ അല്ലാഹുവിന്റെ പ്രീതി ‎പ്രതീക്ഷിച്ച് സ്വന്തത്തെ സമ്പൂര്‍ണമായി സമര്‍പ്പിക്കുന്നു. ‎അല്ലാഹു തന്റെ അടിമകളോട് അതീവ ദയാലുവാണ്. ‎
\end{malayalam}}
\flushright{\begin{Arabic}
\quranayah[2][208]
\end{Arabic}}
\flushleft{\begin{malayalam}
വിശ്വസിച്ചവരേ, നിങ്ങള്‍ പൂര്‍ണമായി ഇസ്ലാമില്‍ ‎പ്രവേശിക്കുക. പിശാചിന്റെ കാല്‍പ്പാടുകളെ ‎പിന്‍പറ്റരുത്. അവന്‍ നിങ്ങളുടെ പ്രത്യക്ഷ ശത്രുവാണ്. ‎
\end{malayalam}}
\flushright{\begin{Arabic}
\quranayah[2][209]
\end{Arabic}}
\flushleft{\begin{malayalam}
വ്യക്തമായ തെളിവുകള്‍ വന്നെത്തിയശേഷവും നിങ്ങള്‍ ‎സത്യമാര്‍ഗത്തില്‍നിന്ന് വഴുതിപ്പോവുകയാണെങ്കില്‍ ‎അറിയുക: അല്ലാഹു പ്രതാപിയും യുക്തിമാനുമാണ്. ‎
\end{malayalam}}
\flushright{\begin{Arabic}
\quranayah[2][210]
\end{Arabic}}
\flushleft{\begin{malayalam}
മേഘമേലാപ്പിനുകീഴെ അല്ലാഹുവും മലക്കുകളും ‎അവരുടെ അടുത്ത് വരികയും കാര്യം ‎തീരുമാനിക്കുകയും ചെയ്യണമെന്നാണോ അവര്‍ ‎പ്രതീക്ഷിക്കുന്നത്? അന്ത്യവിധിക്കായി എല്ലാ കാര്യങ്ങളും ‎തിരിച്ചെത്തുന്നത് അല്ലാഹുവിങ്കലേക്കു തന്നെ. ‎
\end{malayalam}}
\flushright{\begin{Arabic}
\quranayah[2][211]
\end{Arabic}}
\flushleft{\begin{malayalam}
ഇസ്രയേല്‍ മക്കളോട് ചോദിക്കുക, എത്രയെത്ര വ്യക്തമായ ‎തെളിവുകളാണ് നാം അവര്‍ക്കു നല്‍കിയതെന്ന്. ‎അല്ലാഹുവിന്റെ അനുഗ്രഹം വന്നെത്തിയശേഷം അതിനെ ‎മാറ്റിമറിക്കാന്‍ ശ്രമിക്കുന്നവന്‍ അറിയട്ടെ: അല്ലാഹു ‎കഠിനമായി ശിക്ഷിക്കുന്നവനാണ്. ‎
\end{malayalam}}
\flushright{\begin{Arabic}
\quranayah[2][212]
\end{Arabic}}
\flushleft{\begin{malayalam}
സത്യനിഷേധികള്‍ക്ക് ഈ ലോകജീവിതം ഏറെ ‎ചേതോഹരമായി തോന്നിയിരിക്കുന്നു. ‎സത്യവിശ്വാസികളെ അവര്‍ പരിഹസിക്കുകയാണ്. ‎എന്നാല്‍ ഉയിര്‍ത്തെഴുന്നേല്‍പുനാളില്‍ ‎ദൈവഭക്തന്മാരായിരിക്കും അവരെക്കാള്‍ ഉന്നതന്മാര്‍. ‎അല്ലാഹു അവനിച്ഛിക്കുന്നവര്‍ക്ക് കണക്കില്ലാതെ ‎വിഭവങ്ങള്‍ നല്‍കുന്നു. ‎
\end{malayalam}}
\flushright{\begin{Arabic}
\quranayah[2][213]
\end{Arabic}}
\flushleft{\begin{malayalam}
ആദിയില്‍ മനുഷ്യരാശി ഒരൊറ്റ സമുദായമായിരുന്നു. ‎പിന്നീട് അവര്‍ക്കിടയില്‍ ഭിന്നതയുണ്ടായപ്പോള്‍ ‎ശുഭവാര്‍ത്ത അറിയിക്കുന്നവരും മുന്നറിയിപ്പ് ‎നല്‍കുന്നവരുമായി അല്ലാഹു പ്രവാചകന്മാരെ ‎നിയോഗിച്ചു. അവര്‍ക്കിടയില്‍ അഭിപ്രായ ‎വ്യത്യാസമുള്ള കാര്യങ്ങളില്‍ തീര്‍പ്പുകല്‍പിക്കാനായി ‎അവരോടൊപ്പം സത്യവേദ പുസ്തകവും ‎അവതരിപ്പിച്ചു. വേദം ലഭിച്ചവര്‍ തന്നെയാണ് ‎വ്യക്തമായ തെളിവുകള്‍ വന്നെത്തിയശേഷവും അതില്‍ ‎ഭിന്നിച്ചത്. അവര്‍ക്കിടയിലെ കിടമത്സരം കാരണമാണത്. ‎എന്നാല്‍ സത്യവിശ്വാസികളെ അവര്‍ ഭിന്നിച്ചകന്നുപോയ ‎സത്യത്തിലേക്ക് അല്ലാഹു തന്റെ ഹിതമനുസരിച്ച് ‎വഴിനടത്തി. അല്ലാഹു അവനിച്ഛിക്കുന്നവരെ ‎നേര്‍വഴിയിലേക്കു നയിക്കുന്നു. ‎
\end{malayalam}}
\flushright{\begin{Arabic}
\quranayah[2][214]
\end{Arabic}}
\flushleft{\begin{malayalam}
അല്ല; നിങ്ങളുടെ മുന്‍ഗാമികളെ ബാധിച്ച ‎ദുരിതങ്ങളൊന്നും നിങ്ങള്‍ക്കു വന്നെത്താതെതന്നെ നിങ്ങള്‍ ‎സ്വര്‍ഗത്തിലങ്ങ് കടന്നുകളയാമെന്ന് കരുതുന്നുണ്ടോ? ‎പീഡനങ്ങളും പ്രയാസങ്ങളും അവരെ ബാധിച്ചു. ‎ദൈവദൂതനും കൂടെയുള്ള വിശ്വാസികളും “ദൈവ ‎സഹായം എപ്പോഴാണുണ്ടാവുക"യെന്ന് ‎വിലപിക്കേണ്ടിവരുമാറ് കിടിലംകൊള്ളിക്കുന്ന അവസ്ഥ ‎അവര്‍ക്കുണ്ടായി. അറിയുക: അല്ലാഹുവിന്റെ ‎സഹായം അടുത്തുതന്നെയുണ്ടാകും. ‎
\end{malayalam}}
\flushright{\begin{Arabic}
\quranayah[2][215]
\end{Arabic}}
\flushleft{\begin{malayalam}
അവര്‍ ചോദിക്കുന്നു: അവരെന്താണ് ‎ചെലവഴിക്കേണ്ടതെന്ന്? പറയുക: നിങ്ങള്‍ ‎ചെലവഴിക്കുന്ന നല്ലതെന്തും മാതാപിതാക്കള്‍ക്കും ‎അടുത്ത ബന്ധുക്കള്‍ക്കും അനാഥകള്‍ക്കും അഗതികള്‍ക്കും ‎വഴിപോക്കര്‍ക്കുമാണ് നല്‍കേണ്ടത്. നിങ്ങള്‍ നല്ലതെന്തു ‎ചെയ്താലും തീര്‍ച്ചയായും അല്ലാഹു അതെല്ലാമറിയും. ‎
\end{malayalam}}
\flushright{\begin{Arabic}
\quranayah[2][216]
\end{Arabic}}
\flushleft{\begin{malayalam}
യുദ്ധം നിങ്ങള്‍ക്ക് നിര്‍ബന്ധമാക്കിയിരിക്കുന്നു; അത് ‎നിങ്ങള്‍ക്ക് അനിഷ്ടകരം തന്നെ. എന്നാല്‍ ഗുണകരമായ ‎കാര്യം നിങ്ങള്‍ക്ക് അനിഷ്ടകരമായേക്കാം. ‎ദോഷകരമായത് ഇഷ്ടകരവുമായേക്കാം. അല്ലാഹു ‎അറിയുന്നു. നിങ്ങളോ അറിയുന്നുമില്ല. ‎
\end{malayalam}}
\flushright{\begin{Arabic}
\quranayah[2][217]
\end{Arabic}}
\flushleft{\begin{malayalam}
ആദരണീയ മാസത്തില്‍ യുദ്ധം ചെയ്യുന്നതിനെ സംബന്ധിച്ച് ‎അവര്‍ നിന്നോടു ചോദിക്കുന്നു. പറയുക: അതിലെ യുദ്ധം ‎അതീവഗുരുതരം തന്നെ. എന്നാല്‍ ദൈവമാര്‍ഗത്തില്‍ ‎നിന്ന് ജനങ്ങളെ വിലക്കുക, അവനെ നിഷേധിക്കുക, ‎മസ്ജിദുല്‍ഹറാമില്‍ വിലക്കേര്‍പ്പെടുത്തുക, അതിന്റെ ‎അവകാശികളെ അവിടെനിന്ന് പുറത്താക്കുക- ഇതെല്ലാം ‎അല്ലാഹുവിങ്കല്‍ അതിലും കൂടുതല്‍ ഗൌരവമുള്ളതാണ്. ‎‎“ഫിത്ന" കൊലയെക്കാള്‍ ഗുരുതരമാണ്. അവര്‍ക്കു ‎കഴിയുമെങ്കില്‍ നിങ്ങളെ നിങ്ങളുടെ മതത്തില്‍നിന്ന് ‎പിന്തിരിപ്പിക്കും വരെ അവര്‍ നിങ്ങളോട് യുദ്ധം ‎ചെയ്തുകൊണ്ടേയിരിക്കും. നിങ്ങളാരെങ്കിലും തന്റെ ‎മതത്തില്‍നിന്ന് പിന്മാറി സത്യനിഷേധിയായി ‎മരണമടയുകയാണെങ്കില്‍ അവരുടെ കര്‍മങ്ങള്‍ ‎ഇഹത്തിലും പരത്തിലും പാഴായതുതന്നെ. ‎അത്തരക്കാരെല്ലാം നരകത്തീയിലായിരിക്കും. അവരതില്‍ ‎സ്ഥിരവാസികളായിരിക്കും. ‎
\end{malayalam}}
\flushright{\begin{Arabic}
\quranayah[2][218]
\end{Arabic}}
\flushleft{\begin{malayalam}
എന്നാല്‍ സത്യവിശ്വാസം സ്വീകരിക്കുകയും അതിന്റെ ‎പേരില്‍ നാടുവെടിയുകയും അല്ലാഹുവിന്റെ ‎മാര്‍ഗത്തില്‍ ജിഹാദ് നടത്തുകയും ചെയ്യുന്നവരാണ് ‎ദിവ്യാനുഗ്രഹം പ്രതീക്ഷിക്കാവുന്നവര്‍. അല്ലാഹു ഏറെ ‎പൊറുക്കുന്നവനും ദയാപരനും തന്നെ. ‎
\end{malayalam}}
\flushright{\begin{Arabic}
\quranayah[2][219]
\end{Arabic}}
\flushleft{\begin{malayalam}
നിന്നോടവര്‍ മദ്യത്തെയും ചൂതിനെയും സംബന്ധിച്ച് ‎ചോദിക്കുന്നു. പറയുക: അവ രണ്ടിലും ഗുരുതരമായ ‎തിന്മയുണ്ട്. മനുഷ്യര്‍ക്ക് ചില ഉപകാരങ്ങളുമുണ്ട്. ‎എന്നാല്‍ അവയിലെ തിന്മയാണ് പ്രയോജനത്തെക്കാള്‍ ‎ഏറെ വലുത്. തങ്ങള്‍ ചെലവഴിക്കേണ്ടതെന്തെന്നും ‎അവര്‍ നിന്നോട് ചോദിക്കുന്നു. പറയുക: ‎‎“ആവശ്യംകഴിച്ച് മിച്ചമുള്ളത്." ഇവ്വിധം അല്ലാഹു ‎നിങ്ങള്‍ക്ക് വിധികള്‍ വിശദീകരിച്ചുതരുന്നു. നിങ്ങള്‍ ‎ചിന്തിക്കുന്നവരാകാന്‍; ‎
\end{malayalam}}
\flushright{\begin{Arabic}
\quranayah[2][220]
\end{Arabic}}
\flushleft{\begin{malayalam}
ഈ ലോകത്തെപ്പറ്റിയും പരലോകത്തെപ്പറ്റിയും. ‎അനാഥക്കുട്ടികളെ സംബന്ധിച്ചും അവര്‍ നിന്നോടു ‎ചോദിക്കുന്നു. പറയുക: അവര്‍ക്ക് നന്മ ‎വരുത്തുന്നതെല്ലാം നല്ലതാണ്. നിങ്ങള്‍ അവരോടൊപ്പം ‎താമസിക്കുന്നതിലും തെറ്റില്ല. അവര്‍ നിങ്ങളുടെ ‎സഹോദരങ്ങളാണല്ലോ. നാശമുണ്ടാക്കുന്നവനെയും നന്മ ‎വരുത്തുന്നവനെയും അല്ലാഹു വേര്‍തിരിച്ചറിയുന്നു. ‎ദൈവമിച്ഛിച്ചിരുന്നെങ്കില്‍ അവന്‍ നിങ്ങളെ ‎പ്രയാസപ്പെടുത്തുമായിരുന്നു. ഉറപ്പായും അല്ലാഹു ‎പ്രതാപിയും യുക്തിമാനുമാകുന്നു. ‎
\end{malayalam}}
\flushright{\begin{Arabic}
\quranayah[2][221]
\end{Arabic}}
\flushleft{\begin{malayalam}
സത്യവിശ്വാസം സ്വീകരിച്ചാലല്ലാതെ ബഹുദൈവ ‎വിശ്വാസിനികളെ നിങ്ങള്‍ വിവാഹം ചെയ്യരുത്. ‎സത്യവിശ്വാസിനിയായ ഒരടിമപ്പെണ്ണാണ് ബഹുദൈവ ‎വിശ്വാസിനിയെക്കാളുത്തമം. അവള്‍ നിങ്ങളില്‍ ‎കൌതുകമുണര്‍ത്തിയാലും ശരി. അപ്രകാരം തന്നെ ‎സത്യവിശ്വാസം സ്വീകരിക്കുവോളം ബഹുദൈവ ‎വിശ്വാസികള്‍ക്ക് നിങ്ങള്‍ മക്കളെ വിവാഹം ‎ചെയ്തുകൊടുക്കരുത്. സത്യവിശ്വാസിയായ അടിമയാണ് ‎ബഹുദൈവ വിശ്വാസിയെക്കാളുത്തമം. അവന്‍ ‎നിങ്ങളില്‍ കൌതുകമുണര്‍ത്തിയാലും ശരി. അവര്‍ ‎ക്ഷണിക്കുന്നത് നരകത്തിലേക്കാണ്. അല്ലാഹുവോ, ‎അവന്റെ ഹിതാനുസൃതം സ്വര്‍ഗത്തിലേക്കും ‎പാപമോചനത്തിലേക്കും ക്ഷണിക്കുന്നു. അവന്‍ തന്റെ ‎തെളിവുകള്‍ ജനങ്ങള്‍ക്കായി വിശദീകരിച്ചുകൊടുക്കുന്നു. ‎അവര്‍ കാര്യം മനസ്സിലാക്കി ഉള്‍ക്കൊള്ളാന്‍. ‎
\end{malayalam}}
\flushright{\begin{Arabic}
\quranayah[2][222]
\end{Arabic}}
\flushleft{\begin{malayalam}
ആര്‍ത്തവത്തെ സംബന്ധിച്ചും അവര്‍ നിന്നോടു ‎ചോദിക്കുന്നു. പറയുക: അത് മാലിന്യമാണ്. അതിനാല്‍ ‎ആര്‍ത്തവ വേളയില്‍ നിങ്ങള്‍ ‎സ്ത്രീകളില്‍നിന്നകന്നുനില്‍ക്കുക. ശുദ്ധിയാകുംവരെ ‎അവരെ സമീപിക്കരുത്. അവര്‍ ശുദ്ധി നേടിയാല്‍ ‎അല്ലാഹു നിങ്ങളോടാജ്ഞാപിച്ച പോലെ നിങ്ങളവരെ ‎സമീപിക്കുക. അല്ലാഹു പശ്ചാത്തപിക്കുന്നവരെ ‎സ്നേഹിക്കുന്നു. ശുചിത്വം പാലിക്കുന്നവരെയും ‎അവനിഷ്ടപ്പെടുന്നു. ‎
\end{malayalam}}
\flushright{\begin{Arabic}
\quranayah[2][223]
\end{Arabic}}
\flushleft{\begin{malayalam}
നിങ്ങളുടെ സ്ത്രീകള്‍ നിങ്ങളുടെ കൃഷിയിട മാകുന്നു. ‎അതിനാല്‍ നിങ്ങളാഗ്രഹിക്കുംവിധം നിങ്ങള്‍ക്ക് ‎നിങ്ങളുടെ കൃഷിയിടത്ത് ചെല്ലാവുന്നതാണ്. എന്നാല്‍ ‎നിങ്ങളുടെ ഭാവിക്കു വേണ്ടത് നിങ്ങള്‍ നേരത്തെ തന്നെ ‎ചെയ്തുവെക്കണം. നിങ്ങള്‍ അല്ലാഹുവെ സൂക്ഷിക്കുക. ‎അറിയുക: നിങ്ങള്‍ അവനുമായി കണ്ടുമുട്ടുകതന്നെ ‎ചെയ്യും. സത്യവിശ്വാസികളെ ശുഭവാര്‍ത്ത അറിയിക്കുക. ‎
\end{malayalam}}
\flushright{\begin{Arabic}
\quranayah[2][224]
\end{Arabic}}
\flushleft{\begin{malayalam}
നന്മ ചെയ്യുക, ഭക്തി പുലര്‍ത്തുക, ജനങ്ങള്‍ക്കിടയില്‍ ‎രഞ്ജിപ്പുണ്ടാക്കുക എന്നിവക്ക് തടസ്സമുണ്ടാക്കാനായി ‎ശപഥം ചെയ്യാന്‍ നിങ്ങള്‍ അല്ലാഹുവിന്റെ ‎പേരുപയോഗിക്കരുത്. അല്ലാഹു എല്ലാം ‎കേള്‍ക്കുന്നവനാണ്. സകലതും അറിയുന്നവനും. ‎
\end{malayalam}}
\flushright{\begin{Arabic}
\quranayah[2][225]
\end{Arabic}}
\flushleft{\begin{malayalam}
ബോധപൂര്‍വമല്ലാതെ പറഞ്ഞുപോകുന്ന ശപഥങ്ങളുടെ ‎പേരില്‍ അല്ലാഹു നിങ്ങളെ പിടികൂടുകയില്ല. എന്നാല്‍ ‎നിങ്ങള്‍ മനപ്പൂര്‍വം പ്രവര്‍ത്തിച്ചതിന്റെ പേരില്‍ ‎അല്ലാഹു പിടികൂടും. അല്ലാഹു ഏറെ പൊറുക്കുന്നവനും ‎ക്ഷമിക്കുന്നവനുമാണ്. ‎
\end{malayalam}}
\flushright{\begin{Arabic}
\quranayah[2][226]
\end{Arabic}}
\flushleft{\begin{malayalam}
തങ്ങളുടെ ഭാര്യമാരുമായി ബന്ധപ്പെടില്ലെന്ന് ശപഥം ‎ചെയ്തവര്‍ക്ക് നാലുമാസം വരെ കാത്തിരിക്കാം. അവര്‍ ‎മടങ്ങുന്നു വെങ്കില്‍ അല്ലാഹു ഏറെ പൊറുക്കുന്നവനും ‎ദയാപരനുമാകുന്നു. ‎
\end{malayalam}}
\flushright{\begin{Arabic}
\quranayah[2][227]
\end{Arabic}}
\flushleft{\begin{malayalam}
അഥവാ, അവര്‍ വിവാഹമോചനം തന്നെയാണ് ‎തീരുമാനിക്കുന്നതെങ്കില്‍ അല്ലാഹു എല്ലാം ‎കേള്‍ക്കുന്നവനും അറിയുന്നവനുമാണ്. ‎
\end{malayalam}}
\flushright{\begin{Arabic}
\quranayah[2][228]
\end{Arabic}}
\flushleft{\begin{malayalam}
വിവാഹമോചിതര്‍ മൂന്നു തവണ മാസമുറ ‎ഉണ്ടാവുംവരെ തങ്ങളെ സ്വയം നിയന്ത്രിച്ചു കഴിയണം. ‎അല്ലാഹു അവരുടെ ഗര്‍ഭാശയങ്ങളില്‍ ‎സൃഷ്ടിച്ചുവെച്ചതിനെ മറച്ചുവെക്കാന്‍ അവര്‍ക്ക് ‎അനുവാദമില്ല. അവര്‍ അല്ലാഹുവിലും ‎അന്ത്യദിനത്തിലും വിശ്വസിക്കുന്നവരെങ്കില്‍! ‎അതിനിടയില്‍ അവര്‍ ബന്ധം നന്നാക്കാന്‍ ‎ഉദ്ദേശിക്കുന്നുവെങ്കില്‍ അവരെ തിരിച്ചെടുക്കാന്‍ ‎അവരുടെ ഭര്‍ത്താക്കന്മാര്‍ ഏറ്റം അര്‍ഹരത്രെ. ‎സ്ത്രീകള്‍ക്ക് ബാധ്യതകളുള്ളതുപോലെത്തന്നെ ന്യായമായ ‎അവകാശങ്ങളുമുണ്ട്. എന്നാല്‍ പുരുഷന്മാര്‍ക്ക് ‎അവരെക്കാള്‍ ഒരു പദവി കൂടുതലുണ്ട്. അല്ലാഹു ‎പ്രതാപിയും യുക്തിമാനുമാകുന്നു. ‎
\end{malayalam}}
\flushright{\begin{Arabic}
\quranayah[2][229]
\end{Arabic}}
\flushleft{\begin{malayalam}
വിവാഹമോചനം രണ്ടു തവണയാകുന്നു. പിന്നെ ‎ന്യായമായ നിലയില്‍ കൂടെ നിര്‍ത്തുകയോ നല്ല നിലയില്‍ ‎ഒഴിവാക്കുകയോ വേണം. നേരത്തെ നിങ്ങള്‍ ഭാര്യമാര്‍ക്ക് ‎നല്‍കിയിരുന്നതില്‍ നിന്ന് യാതൊന്നും തിരിച്ചുവാങ്ങാന്‍ ‎പാടില്ല; ഇരുവരും അല്ലാഹുവിന്റെ നിയമപരിധികള്‍ ‎പാലിക്കാന്‍ കഴിയില്ലെന്ന് ആശങ്കിക്കുന്നുവെങ്കിലല്ലാതെ. ‎അവരിരുവരും അല്ലാഹുവിന്റെ നിയമപരിധികള്‍ ‎പാലിക്കുകയില്ലെന്ന് നിങ്ങള്‍ക്ക് ആശങ്ക ‎തോന്നുന്നുവെങ്കില്‍ സ്ത്രീ തന്റെ ഭര്‍ത്താവിന് വല്ലതും ‎നല്‍കി വിവാഹമോചനം നേടുന്ന തില്‍ ഇരുവര്‍ക്കും ‎കുറ്റമില്ല. അല്ലാഹുവിന്റെ നിയമപരിധികളാണിവ. ‎നിങ്ങളവ ലംഘിക്കരുത്. ദൈവികനിയമങ്ങള്‍ ‎ലംഘിക്കുന്നവര്‍ തന്നെയാണ് അതിക്രമികള്‍. ‎
\end{malayalam}}
\flushright{\begin{Arabic}
\quranayah[2][230]
\end{Arabic}}
\flushleft{\begin{malayalam}
വീണ്ടും വിവാഹമോചനം നടത്തിയാല്‍ പിന്നെ അവന് ‎അവള്‍ അനുവദനീയയാവുകയില്ല; അവളെ മറ്റൊരാള്‍ ‎വിവാഹം കഴിക്കുകയും അയാള്‍ അവളെ ‎വിവാഹമോചനം നടത്തുകയും ചെയ്താലല്ലാതെ. ‎അപ്പോള്‍ മുന്‍ഭര്‍ത്താവിനും അവള്‍ക്കും ‎ദാമ്പത്യത്തിലേക്ക് തിരിച്ചുവരുന്നതില്‍ വിരോധമില്ല; ‎മേലില്‍ ഇരുവരും ദൈവികനിയമങ്ങള്‍ പാലിക്കുമെന്ന് ‎കരുതുന്നുവെങ്കില്‍. ഇത് അല്ലാഹു നിശ്ചയിച്ച ‎നിയമപരിധികളാണ്. കാര്യമറിയുന്ന ജനത്തിന് അല്ലാഹു ‎അവ വിശദീകരിച്ചുതരികയാണ്. ‎
\end{malayalam}}
\flushright{\begin{Arabic}
\quranayah[2][231]
\end{Arabic}}
\flushleft{\begin{malayalam}
നിങ്ങള്‍ സ്ത്രീകളെ വിവാഹമോചനം ചെയ്യുകയും ‎അങ്ങനെ അവരുടെ അവധി എത്തുകയും ചെയ്താല്‍ ‎അവരെ ന്യായമായ നിലയില്‍ കൂടെ നിര്‍ത്തുക. ‎അല്ലെങ്കില്‍ മാന്യമായി പിരിച്ചയക്കുക. അവരെ ‎ദ്രോഹിക്കാനായി അന്യായമായി പിടിച്ചുവെക്കരുത്. ‎ആരെങ്കിലും അങ്ങനെ ചെയ്യുന്നുവെങ്കില്‍ അവന്‍ ‎തനിക്കുതന്നെയാണ് ദ്രോഹം വരുത്തുന്നത്. ‎അല്ലാഹുവിന്റെ വചനങ്ങളെ നിങ്ങള്‍ ‎കളിയായിട്ടെടുക്കാതിരിക്കുവിന്‍. അല്ലാഹു ‎നിങ്ങള്‍ക്കേകിയ അനുഗ്രഹങ്ങള്‍ ഓര്‍ക്കുക. അല്ലാഹു ‎നിങ്ങളെ ഉപദേശിക്കാനായി വേദപുസ്തകവും ‎തത്ത്വജ്ഞാനവും ഇറക്കിത്തന്നതും ഓര്‍ക്കുക. ‎അല്ലാഹുവോട് ഭക്തിയുള്ളവരാവുക. അറിയുക: ‎നിശ്ചയമായും അല്ലാഹു എല്ലാം അറിയുന്നവനാണ്. ‎
\end{malayalam}}
\flushright{\begin{Arabic}
\quranayah[2][232]
\end{Arabic}}
\flushleft{\begin{malayalam}
നിങ്ങള്‍ സ്ത്രീകളെ വിവാഹമോചനം ചെയ്തു. അവര്‍ ‎തങ്ങളുടെ അവധിക്കാലം പൂര്‍ത്തീകരിക്കുകയും ‎ചെയ്തു. പിന്നീട് ന്യായമായ നിലയില്‍ പരസ്പരം ‎ഇഷ്ടപ്പെടുകയാണെങ്കില്‍ അവര്‍ തങ്ങളുടെ ‎ഭര്‍ത്താക്കന്മാരെ വേള്‍ക്കുന്നത് നിങ്ങള്‍ വിലക്കരുത്. ‎നിങ്ങളില്‍ അല്ലാഹുവിലും അന്ത്യദിനത്തിലും ‎വിശ്വസിക്കുന്നവര്‍ക്കുള്ള ഉപദേശമാണിത്. അതാണ് ‎നിങ്ങള്‍ക്ക് ഏറെ ശ്രേഷ്ഠവും വിശുദ്ധവും. അല്ലാഹു ‎അറിയുന്നു; നിങ്ങള്‍ അറിയുന്നില്ല. ‎
\end{malayalam}}
\flushright{\begin{Arabic}
\quranayah[2][233]
\end{Arabic}}
\flushleft{\begin{malayalam}
മാതാക്കള്‍ തങ്ങളുടെ മക്കളെ രണ്ടുവര്‍ഷം ‎പൂര്‍ണമായും മുലയൂട്ടണം. മുലകുടികാലം ‎പൂര്‍ത്തീകരിക്കണമെന്ന് ഉദ്ദേശിക്കുന്നുവെങ്കിലാണിത്. ‎മുലയൂട്ടുന്ന സ്ത്രീക്ക് ന്യായമായ നിലയില്‍ ഭക്ഷണവും ‎വസ്ത്രവും നല്‍കേണ്ട ബാധ്യത കുട്ടിയുടെ ‎പിതാവിനാണ്. എന്നാല്‍ ആരെയും അവരുടെ ‎കഴിവിനപ്പുറമുള്ളതിന് നിര്‍ബന്ധിക്കാവതല്ല. ഒരു ‎മാതാവും തന്റെ കുഞ്ഞ് കാരണമായി ‎പീഡിപ്പിക്കപ്പെടരുത്. അപ്രകാരം തന്നെ കുട്ടി ‎തന്റേതാണെന്ന കാരണത്താല്‍ പിതാവും ‎പീഡിപ്പിക്കപ്പെടരുത്. പിതാവില്ലെങ്കില്‍ അയാളുടെ ‎അനന്തരാവകാശികള്‍ക്ക് അയാള്‍ക്കുള്ള അതേ ‎ബാധ്യതയുണ്ട്. എന്നാല്‍ ഇരുവിഭാഗവും പരസ്പരം ‎കൂടിയാലോചിച്ചും തൃപ്തിപ്പെട്ടും മുലയൂട്ടല്‍ ‎നിര്‍ത്തുന്നുവെങ്കില്‍ അതിലിരുവര്‍ക്കും കുറ്റമില്ല. ‎അഥവാ, കുട്ടികള്‍ക്ക് മറ്റൊരാളെക്കൊണ്ട് ‎മുലകൊടുപ്പിക്കണമെന്നാണ് നിങ്ങള്‍ ഉദ്ദേശിക്കുന്നതെങ്കില്‍ ‎അതിനും വിരോധമില്ല. അവര്‍ക്കുള്ള പ്രതിഫലം നല്ല ‎നിലയില്‍ നല്‍കുന്നുവെങ്കിലാണിത്. നിങ്ങള്‍ അല്ലാഹുവെ ‎സൂക്ഷിക്കുക. അറിയുക: അല്ലാഹു നിങ്ങള്‍ ‎ചെയ്യുന്നതെല്ലാം കണ്ടറിയുന്നവനാണ്. ‎
\end{malayalam}}
\flushright{\begin{Arabic}
\quranayah[2][234]
\end{Arabic}}
\flushleft{\begin{malayalam}
നിങ്ങളിലാരെങ്കിലും ഭാര്യമാരെ വിട്ടേച്ചു ‎മരിച്ചുപോയാല്‍ ആ ഭാര്യമാര്‍ നാല് മാസവും പത്തു ‎ദിവസവും തങ്ങളെ സ്വയം നിയന്ത്രിച്ചുനിര്‍ത്തേണ്ട ‎താണ്. അങ്ങനെ അവരുടെ കാലാവധിയെത്തിയാല്‍ ‎തങ്ങളുടെ കാര്യത്തില്‍ ന്യായമായ നിലയില്‍ അവര്‍ ‎പ്രവര്‍ത്തിക്കുന്നതില്‍ നിങ്ങള്‍ക്ക് കുറ്റമൊന്നുമില്ല. ‎നിങ്ങള്‍ ചെയ്യുന്നതെല്ലാം സൂക്ഷ്മമായി ‎അറിയുന്നവനാണ് അല്ലാഹു. ‎
\end{malayalam}}
\flushright{\begin{Arabic}
\quranayah[2][235]
\end{Arabic}}
\flushleft{\begin{malayalam}
ആ സ്ത്രീകളു മായി നിങ്ങള്‍ വിവാഹക്കാര്യം ‎വ്യംഗ്യമായി സൂചിപ്പിക്കുകയോ മനസ്സില്‍ ‎ഒളിപ്പിച്ചുവെക്കുകയോ ചെയ്യുന്നത് കുറ്റകരമല്ല. നിങ്ങള്‍ ‎അവരെ ഓര്‍ത്തേക്കുമെന്ന് അല്ലാഹുവിനു നന്നായറിയാം. ‎എന്നാല്‍ സ്വകാര്യമായി അവരുമായി ഒരുടമ്പടിയും ‎ഉണ്ടാക്കരുത്. നിങ്ങള്‍ക്ക് അവരോട് മാന്യമായ ‎നിലയില്‍ സംസാരിക്കാം. നിശ്ചിത അവധി എത്തുംവരെ ‎വിവാഹ ഉടമ്പടി നടത്തരുത്. അറിയുക: തീര്‍ച്ചയായും ‎നിങ്ങളുടെ മനസ്സിലുള്ളത് അല്ലാഹു അറിയുന്നുണ്ട്. ‎അതിനാല്‍ അവനെ സൂക്ഷിക്കുക. അറിയുക: അല്ലാഹു ‎ഏറെ പൊറുക്കുന്നവനും ക്ഷമിക്കുന്നവനുമാണ്. ‎
\end{malayalam}}
\flushright{\begin{Arabic}
\quranayah[2][236]
\end{Arabic}}
\flushleft{\begin{malayalam}
സ്ത്രീകളെ സ്പര്‍ശിക്കുകയോ അവരുടെ വിവാഹമൂല്യം ‎നിശ്ചയിക്കുകയോ ചെയ്യുംമുമ്പെ നിങ്ങളവരെ ‎വിവാഹമോചനം നടത്തുകയാണെങ്കില്‍ നിങ്ങള്‍ക്കതില്‍ ‎കുറ്റമില്ല. എന്നാല്‍ നിങ്ങളവര്‍ക്ക് മാന്യമായ നിലയില്‍ ‎ജീവിതവിഭവം നല്‍കണം. കഴിവുള്ളവന്‍ തന്റെ ‎കഴിവനുസരിച്ചും പ്രയാസപ്പെടുന്നവന്‍ തന്റെ ‎അവസ്ഥയനുസരിച്ചും. നല്ല മനുഷ്യരുടെ ‎ബാധ്യതയാണിത്. ‎
\end{malayalam}}
\flushright{\begin{Arabic}
\quranayah[2][237]
\end{Arabic}}
\flushleft{\begin{malayalam}
അഥവാ, ഭാര്യമാരെ സ്പര്‍ശിക്കും മുമ്പെ നിങ്ങള്‍ ‎വിവാഹബന്ധം വേര്‍പ്പെടുത്തുകയും നിങ്ങളവരുടെ ‎വിവാഹമൂല്യം നിശ്ചയിക്കുകയും ചെയ്തിട്ടുണ്ടെങ്കില്‍ ‎നിങ്ങള്‍ നിശ്ചയിച്ച വിവാഹമൂല്യത്തിന്റെ പാതി ‎അവര്‍ക്കുള്ളതാണ്. അവര്‍ ഇളവ് ‎അനുവദിക്കുന്നില്ലെങ്കിലും വിവാഹ ഉടമ്പടി ആരുടെ ‎കയ്യിലാണോ അയാള്‍ വിട്ടുവീഴ്ച ‎ചെയ്യുന്നില്ലെങ്കിലുമാണിത്. നിങ്ങള്‍ വിട്ടുവീഴ്ച ‎ചെയ്യലാണ് ദൈവഭക്തിയുമായി ഏറെ ‎പൊരുത്തപ്പെടുന്നത്. പരസ്പരം ഔദാര്യം കാണിക്കാന്‍ ‎മറക്കരുത്. അല്ലാഹു നിങ്ങള്‍ ചെയ്യുന്നതെല്ലാം ‎കണ്ടുകൊണ്ടിരിക്കുന്നവനാണ്; തീര്‍ച്ച. ‎
\end{malayalam}}
\flushright{\begin{Arabic}
\quranayah[2][238]
\end{Arabic}}
\flushleft{\begin{malayalam}
നിങ്ങള്‍ നമസ്കാരത്തില്‍ ശ്രദ്ധ പുലര്‍ത്തുക. ‎വിശേഷിച്ചും വിശിഷ്ടമായ നമസ്കാരം. ‎അല്ലാഹുവിന്റെ മുന്നില്‍ ഭക്തിയോടെ നിന്ന് ‎നമസ്കരിക്കുക. ‎
\end{malayalam}}
\flushright{\begin{Arabic}
\quranayah[2][239]
\end{Arabic}}
\flushleft{\begin{malayalam}
അരക്ഷിതാവസ്ഥയിലാണ് നിങ്ങളെങ്കില്‍ നടന്നുകൊണ്ടോ ‎വാഹനത്തിലിരുന്നുകൊണ്ടോ നമസ്കാരം ‎നിര്‍വഹിക്കുക. എന്നാല്‍ സുരക്ഷിതാവസ്ഥയിലായാല്‍ ‎നിങ്ങള്‍ക്ക് അറിവില്ലാതിരുന്നത് അല്ലാഹു നിങ്ങള്‍ക്ക് ‎പഠിപ്പിച്ചുതന്ന പോലെ നിങ്ങളവനെ സ്മരിക്കുക. ‎
\end{malayalam}}
\flushright{\begin{Arabic}
\quranayah[2][240]
\end{Arabic}}
\flushleft{\begin{malayalam}
നിങ്ങളില്‍ ഭാര്യമാരെ വിട്ടേച്ച് മരണപ്പെടുന്നവര്‍ ‎തങ്ങളുടെ ഭാര്യമാര്‍ക്ക് ഒരു കൊല്ലത്തേക്കാവശ്യമായ ‎ജീവിതവിഭവങ്ങള്‍ വസ്വിയ്യത്തു ചെയ്യേണ്ടതാണ്. ‎അവരെ വീട്ടില്‍നിന്ന് ഇറക്കിവിടരുത്. എന്നാല്‍ അവര്‍ ‎സ്വയം പുറത്തുപോകുന്നുവെങ്കില്‍ തങ്ങളുടെ കാര്യത്തില്‍ ‎ന്യായമായ നിലയിലവര്‍ ചെയ്യുന്നതിലൊന്നും നിങ്ങള്‍ക്ക് ‎ഉത്തരവാദിത്തമില്ല. അല്ലാഹു പ്രതാപിയും ‎യുക്തിമാനും തന്നെ. ‎
\end{malayalam}}
\flushright{\begin{Arabic}
\quranayah[2][241]
\end{Arabic}}
\flushleft{\begin{malayalam}
വിവാഹമോചിതര്‍ക്ക് ന്യായമായ നിലയില്‍ ‎ജീവിതവിഭവം നല്‍കണം. ഭക്തന്മാരുടെ ‎ബാധ്യതയാണിത്. ‎
\end{malayalam}}
\flushright{\begin{Arabic}
\quranayah[2][242]
\end{Arabic}}
\flushleft{\begin{malayalam}
ഇവ്വിധം അല്ലാഹു നിങ്ങള്‍ക്ക് തന്റെ കല്‍പനകള്‍ ‎വിശദീകരിച്ചുതരുന്നു. നിങ്ങള്‍ ചിന്തിച്ചറിയാന്‍. ‎
\end{malayalam}}
\flushright{\begin{Arabic}
\quranayah[2][243]
\end{Arabic}}
\flushleft{\begin{malayalam}
ആയിരങ്ങളുണ്ടായിട്ടും മരണഭയത്താല്‍ തങ്ങളുടെ ‎വീടുവിട്ടിറങ്ങിയ ജനത യുടെ അവസ്ഥ നീ ‎കണ്ടറിഞ്ഞില്ലേ? അല്ലാഹു അവരോട് കല്‍പിച്ചു: ‎‎"നിങ്ങള്‍ മരിച്ചുകൊള്ളുക." പിന്നെ അല്ലാഹു അവരെ ‎ജീവിപ്പിച്ചു. ഉറപ്പായും അല്ലാഹു മനുഷ്യരോട് ഉദാരത ‎പുലര്‍ത്തുന്നവനാണ്. എന്നാല്‍ മനുഷ്യരിലേറെ പേരും ‎നന്ദി കാണിക്കുന്നില്ല. ‎
\end{malayalam}}
\flushright{\begin{Arabic}
\quranayah[2][244]
\end{Arabic}}
\flushleft{\begin{malayalam}
നിങ്ങള്‍ അല്ലാഹുവിന്റെ മാര്‍ഗത്തില്‍ യുദ്ധം ചെയ്യുക. ‎തീര്‍ച്ചയായും അല്ലാഹു എല്ലാം കേള്‍ക്കുന്നവനും ‎അറിയുന്നവനുമാണ്. ഇക്കാര്യം നന്നായി മനസ്സിലാക്കുക. ‎
\end{malayalam}}
\flushright{\begin{Arabic}
\quranayah[2][245]
\end{Arabic}}
\flushleft{\begin{malayalam}
അല്ലാഹുവിന് ഉത്തമമായ കടം നല്‍കുന്നവരായി ‎ആരുണ്ട്? എങ്കില്‍ അല്ലാഹു അത് അയാള്‍ക്ക് ‎അനേകമിരട്ടിയായി തിരിച്ചുകൊടുക്കും. ധനം ‎പിടിച്ചുവെക്കുന്നതും വിട്ടുകൊടുക്കുന്നതും ‎അല്ലാഹുവാണ്. അവങ്കലേക്കുതന്നെയാണ് നിങ്ങളുടെ ‎മടക്കം. ‎
\end{malayalam}}
\flushright{\begin{Arabic}
\quranayah[2][246]
\end{Arabic}}
\flushleft{\begin{malayalam}
നീ അറിഞ്ഞിട്ടുണ്ടോ? മൂസാക്കുശേഷമുള്ള ഇസ്രയേലി ‎പ്രമാണിമാരുടെ കാര്യം? അവര്‍ തങ്ങളുടെ ‎പ്രവാചകനോടു പറഞ്ഞു: “ഞങ്ങള്‍ക്കൊരു രാജാവിനെ ‎നിശ്ചയിച്ചുതരിക. ഞങ്ങള്‍ ദൈവമാര്‍ഗത്തില്‍ ‎പടപൊരുതാം." പ്രവാചകന്‍ ചോദിച്ചു: “യുദ്ധത്തിന് ‎കല്‍പന കിട്ടിയാല്‍ പിന്നെ, നിങ്ങള്‍ യുദ്ധം ‎ചെയ്യാതിരിക്കുമോ?" അവര്‍ പറഞ്ഞു: ‎‎“ദൈവമാര്‍ഗത്തില്‍ ഞങ്ങളെങ്ങനെ പൊരുതാതിരിക്കും? ‎ഞങ്ങളെ സ്വന്തം വീടുകളില്‍ നിന്നും മക്കളില്‍നിന്നും ‎ആട്ടിപ്പുറത്താക്കിയിരിക്കെ?" എന്നാല്‍ യുദ്ധത്തിന് ‎കല്‍പന കൊടുത്തപ്പോള്‍ അവര്‍ പിന്തിരിഞ്ഞുകളഞ്ഞു; ‎ചുരുക്കം ചിലരൊഴികെ. അല്ലാഹു അക്രമികളെപ്പറ്റി ‎നന്നായറിയുന്നവനാണ്. ‎
\end{malayalam}}
\flushright{\begin{Arabic}
\quranayah[2][247]
\end{Arabic}}
\flushleft{\begin{malayalam}
അവരുടെ പ്രവാചകന്‍ അവരെ അറിയിച്ചു: “അല്ലാഹു ‎ത്വാലൂത്തിനെ നിങ്ങള്‍ക്ക് രാജാവായി ‎നിശ്ചയിച്ചിരിക്കുന്നു." അവര്‍ പറഞ്ഞു: ‎‎“അയാള്‍ക്കെങ്ങനെ ഞങ്ങളുടെ രാജാവാകാന്‍ കഴിയും? ‎രാജത്വത്തിന് അയാളെക്കാള്‍ യോഗ്യത ‎ഞങ്ങള്‍ക്കാണല്ലോ. അയാള്‍ വലിയ ‎പണക്കാരനൊന്നുമല്ലല്ലോ." പ്രവാചകന്‍ പ്രതിവചിച്ചു: ‎‎“അല്ലാഹു അദ്ദേഹത്തെ നിങ്ങളെക്കാള്‍ ഉല്‍കൃഷ്ടനായി ‎തെരഞ്ഞെടുത്തിരിക്കുന്നു. അദ്ദേഹത്തിന് കായികവും ‎വൈജ്ഞാനികവുമായ കഴിവ് ധാരാളമായി ‎നല്‍കിയിരിക്കുന്നു. അല്ലാഹു രാജത്വം ‎താനിച്ഛിക്കുന്നവര്‍ക്ക് കൊടുക്കുന്നു. അല്ലാഹു ഏറെ ‎വിശാലതയുള്ളവനാണ്. എല്ലാം അറിയുന്നവനും." ‎
\end{malayalam}}
\flushright{\begin{Arabic}
\quranayah[2][248]
\end{Arabic}}
\flushleft{\begin{malayalam}
അവരുടെ പ്രവാചകന്‍ അവരോടു പറഞ്ഞു: ‎‎“അദ്ദേഹത്തിന്റെ രാജാധികാരത്തിനുള്ള തെളിവ് ആ ‎പെട്ടി നിങ്ങള്‍ക്ക് തിരിച്ചുകിട്ടലാണ്. അതില്‍ നിങ്ങളുടെ ‎നാഥനില്‍ നിന്നുള്ള ശാന്തിയുണ്ട്; മൂസായുടെയും ‎ഹാറൂന്റെയും കുടുംബം വിട്ടേച്ചുപോയ ‎വിശിഷ്ടാവശിഷ്ടങ്ങളും. മലക്കുകള്‍ അതു ‎ചുമന്നുകൊണ്ടുവരും. തീര്‍ച്ചയായും നിങ്ങള്‍ക്കതില്‍ ‎മഹത്തായ തെളിവുണ്ട്. നിങ്ങള്‍ വിശ്വാസികളെങ്കില്‍!" ‎
\end{malayalam}}
\flushright{\begin{Arabic}
\quranayah[2][249]
\end{Arabic}}
\flushleft{\begin{malayalam}
അങ്ങനെ പട്ടാളവുമായി ത്വാലൂത്ത് പുറപ്പെട്ടപ്പോള്‍ ‎പറഞ്ഞു: “അല്ലാഹു ഒരു നദികൊണ്ട് നിങ്ങളെ ‎പരീക്ഷിക്കാന്‍ പോവുകയാണ്. അതില്‍നിന്ന് ‎കുടിക്കുന്നവനാരോ, അവന്‍ എന്റെ കൂട്ടത്തില്‍ ‎പെട്ടവനല്ല. അത് രുചിച്ചുനോക്കാത്തവനാരോ അവനാണ് ‎എന്റെ അനുയായി. എന്നാല്‍ തന്റെ കൈകൊണ്ട് ഒരു ‎കോരല്‍ എടുത്തവന്‍ ഇതില്‍ നിന്നൊഴിവാണ്." പക്ഷേ, ‎അവരില്‍ ചുരുക്കം ചിലരൊഴികെ എല്ലാവരും ‎അതില്‍നിന്ന് ഇഷ്ടംപോലെ കുടിച്ചു. അങ്ങനെ ‎ത്വാലൂത്തും കൂടെയുള്ള വിശ്വാസികളും ആ നദി ‎മുറിച്ചുകടന്നു മുന്നോട്ടുപോയപ്പോള്‍ അവര്‍ പറഞ്ഞു: ‎‎“ജാലൂത്തിനെയും അയാളുടെ സൈന്യത്തെയും ‎നേരിടാനുള്ള കഴിവ് ഇന്ന് ഞങ്ങള്‍ക്കില്ല." എന്നാല്‍ ‎അല്ലാഹുവുമായി കണ്ടുമുട്ടേണ്ടിവരുമെന്ന ‎വിചാരമുള്ളവര്‍ പറഞ്ഞു: “എത്രയെത്ര ‎ചെറുസംഘങ്ങളാണ് ദിവ്യാനുമതിയോടെ ‎വന്‍സംഘങ്ങളെ ജയിച്ചടക്കിയത്; അല്ലാഹു ‎ക്ഷമിക്കുന്നവരോടൊപ്പമാണ്." ‎
\end{malayalam}}
\flushright{\begin{Arabic}
\quranayah[2][250]
\end{Arabic}}
\flushleft{\begin{malayalam}
അങ്ങനെ ജാലൂത്തിനും സൈന്യത്തിനുമെതിരെ ‎പടവെട്ടാനിറങ്ങിയപ്പോള്‍ അവര്‍ പ്രാര്‍ഥിച്ചു: ‎‎"ഞങ്ങളുടെ നാഥാ! ഞങ്ങള്‍ക്കു നീ ക്ഷമ ‎പകര്‍ന്നുതരേണമേ! ഞങ്ങളുടെ പാദങ്ങളെ ‎ഉറപ്പിച്ചുനിര്‍ത്തേണമേ! സത്യനിഷേധികളായ ‎ജനത്തിനെതിരെ ഞങ്ങളെ നീ സഹായിക്കേണമേ." ‎
\end{malayalam}}
\flushright{\begin{Arabic}
\quranayah[2][251]
\end{Arabic}}
\flushleft{\begin{malayalam}
അവസാനം ദൈവഹിതത്താല്‍ അവര്‍ ശത്രുക്കളെ ‎തോല്‍പിച്ചോടിച്ചു. ദാവൂദ് ജാലൂത്തിനെ കൊന്നു. ‎അല്ലാഹു അദ്ദേഹത്തിന് അധികാരവും തത്ത്വജ്ഞാനവും ‎നല്‍കി. അവനിച്ഛിച്ചതൊക്കെ അദ്ദേഹത്തെ പഠിപ്പിച്ചു. ‎അല്ലാഹു ജനങ്ങളില്‍ ചിലരെ മറ്റുചിലരെക്കൊണ്ട് ‎പ്രതിരോധിച്ചില്ലായിരുന്നെങ്കില്‍ ഭൂമിയാകെ ‎കുഴപ്പത്തിലാകുമായിരുന്നു. ലോകത്തെങ്ങുമുള്ളവരോട് ‎അത്യുദാരനാണ് അല്ലാഹു. ‎
\end{malayalam}}
\flushright{\begin{Arabic}
\quranayah[2][252]
\end{Arabic}}
\flushleft{\begin{malayalam}
അല്ലാഹുവിന്റെ വചനങ്ങളാണിവ. നാമിതു ‎വേണ്ടതുപോലെ നിനക്ക് ഓതിക്കേള്‍പ്പിച്ചുതരികയാണ്. ‎തീര്‍ച്ചയായും നീ ദൈവദൂതന്മാരില്‍ പെട്ടവന്‍ തന്നെ. ‎
\end{malayalam}}
\flushright{\begin{Arabic}
\quranayah[2][253]
\end{Arabic}}
\flushleft{\begin{malayalam}
ആ ദൈവദൂതന്മാരില്‍ ചിലരെ നാം മറ്റുള്ളവരെക്കാള്‍ ‎ശ്രേഷ്ഠരാക്കിയിരിക്കുന്നു. അല്ലാഹു നേരില്‍ ‎സംസാരിച്ചവര്‍ അവരിലുണ്ട്. മറ്റുചിലരെ അവന്‍ ‎വിശിഷ്ടമായ ചില പദവികളിലേക്കുയര്‍ത്തിയിരിക്കുന്നു. ‎മര്‍യമിന്റെ മകന്‍ യേശുവിന് നാം വ്യക്തമായ ‎അടയാളങ്ങള്‍ നല്‍കി. പരിശുദ്ധാത്മാവിനാല്‍ ‎അദ്ദേഹത്തെ പ്രബലനാക്കി. അല്ലാഹു ഇച്ഛിച്ചിരുന്നെങ്കില്‍ ‎അവരുടെ പിന്‍മുറക്കാര്‍ അവര്‍ക്ക് വ്യക്തമായ തെളിവ് ‎വന്നെത്തിയശേഷവും പരസ്പരം പൊരുതുമായിരുന്നില്ല. ‎എന്നാല്‍ അവര്‍ പരസ്പരം ഭിന്നിച്ചു. അവരില്‍ ‎വിശ്വസിച്ചവരുണ്ട്. സത്യനിഷേധികളുമുണ്ട്. അല്ലാഹു ‎ഇച്ഛിച്ചിരുന്നെങ്കില്‍ അവര്‍ തമ്മിലടിക്കുമായിരുന്നില്ല. ‎പക്ഷേ, അല്ലാഹു അവനിച്ഛിക്കുന്നത് ചെയ്യുന്നു. ‎
\end{malayalam}}
\flushright{\begin{Arabic}
\quranayah[2][254]
\end{Arabic}}
\flushleft{\begin{malayalam}
വിശ്വസിച്ചവരേ, കൊള്ളക്കൊടുക്കകളോ ‎സ്നേഹസ്വാധീനമോ ശിപാര്‍ശയോ ഒന്നും നടക്കാത്ത ‎നാള്‍ വന്നെത്തുംമുമ്പെ, നാം നിങ്ങള്‍ക്കു നല്‍കിയവയില്‍ ‎നിന്ന് ചെലവഴിക്കുക. സത്യനിഷേധികള്‍ തന്നെയാണ് ‎അതിക്രമികള്‍. ‎
\end{malayalam}}
\flushright{\begin{Arabic}
\quranayah[2][255]
\end{Arabic}}
\flushleft{\begin{malayalam}
അല്ലാഹു; അവനല്ലാതെ ദൈവമില്ല. അവന്‍ എന്നെന്നും ‎ജീവിച്ചിരിക്കുന്നവന്‍; എല്ലാറ്റിനെയും ‎പരിപാലിക്കുന്നവന്‍; മയക്കമോ ഉറക്കമോ അവനെ ‎ബാധിക്കുകയില്ല. ആകാശഭൂമികളിലുള്ളതൊക്കെയും ‎അവന്റേതാണ്. അവന്റെ അടുക്കല്‍ അനുവാദമില്ലാതെ ‎ശിപാര്‍ശ ചെയ്യാന്‍ കഴിയുന്നവനാര്? അവരുടെ ‎ഇന്നലെകളിലുണ്ടായതും ‎നാളെകളിലുണ്ടാകാനിരിക്കുന്നതും അവനറിയുന്നു. ‎അവന്റെ അറിവില്‍നിന്ന് അവനിച്ഛിക്കുന്നതല്ലാതെ ‎അവര്‍ക്കൊന്നും അറിയാന്‍ സാധ്യമല്ല. അവന്റെ ‎ആധിപത്യം ആകാശഭൂമികളെയാകെ ‎ഉള്‍ക്കൊണ്ടിരിക്കുന്നു. അവയുടെ സംരക്ഷണം ‎അവനെയൊട്ടും തളര്‍ത്തുന്നില്ല. അവന്‍ അത്യുന്നതനും ‎മഹാനുമാണ്. ‎
\end{malayalam}}
\flushright{\begin{Arabic}
\quranayah[2][256]
\end{Arabic}}
\flushleft{\begin{malayalam}
മതകാര്യത്തില്‍ ഒരുവിധ ബലപ്രയോഗവുമില്ല. ‎നന്മതിന്മകളുടെ വഴികള്‍ വ്യക്തമായും ‎വേര്‍തിരിഞ്ഞുകഴിഞ്ഞിരിക്കുന്നു. അതിനാല്‍ ദൈവേതര ‎ശക്തികളെ നിഷേധിക്കുകയും അല്ലാഹുവില്‍ ‎വിശ്വസിക്കുകയും ചെയ്യുന്നവന്‍ മുറുകെപ്പിടിച്ചത് ‎ഉറപ്പുള്ള കയറിലാണ്. അതറ്റുപോവില്ല. അല്ലാഹു ‎എല്ലാം കേള്‍ക്കുന്നവനും അറിയുന്നവനുമാകുന്നു. ‎
\end{malayalam}}
\flushright{\begin{Arabic}
\quranayah[2][257]
\end{Arabic}}
\flushleft{\begin{malayalam}
അല്ലാഹു, വിശ്വസിച്ചവരുടെ രക്ഷകനാണ്. അവന്‍ ‎അവരെ ഇരുളുകളില്‍നിന്ന് വെളിച്ചത്തിലേക്ക് ‎നയിക്കുന്നു. എന്നാല്‍ സത്യനിഷേധികളുടെ ‎രക്ഷാധികാരികള്‍ ദൈവേതരശക്തികളാണ്. അവര്‍ ‎അവരെ നയിക്കുന്നത് വെളിച്ചത്തില്‍നിന്ന് ‎ഇരുളുകളിലേക്കാണ്. അവര്‍ തന്നെയാണ് ‎നരകാവകാശികള്‍. അവരതില്‍ ‎സ്ഥിരവാസികളായിരിക്കും. ‎
\end{malayalam}}
\flushright{\begin{Arabic}
\quranayah[2][258]
\end{Arabic}}
\flushleft{\begin{malayalam}
നീ കണ്ടില്ലേ; ഇബ്റാഹീമിനോട് അദ്ദേഹത്തിന്റെ ‎നാഥന്റെ കാര്യത്തില്‍ തര്‍ക്കിച്ചവനെ. കാരണം അല്ലാഹു ‎അവന്ന് രാജാധികാരം നല്‍കി. ഇബ്റാഹീം പറഞ്ഞു: ‎‎"ജീവിപ്പിക്കുകയും മരിപ്പിക്കുകയും ചെയ്യുന്നവനാണ് ‎എന്റെ നാഥന്‍." അയാള്‍ അവകാശപ്പെട്ടു: "ഞാനും ‎ജീവിപ്പിക്കുകയും മരിപ്പിക്കുകയും ചെയ്യുന്നുണ്ടല്ലോ!" ‎ഇബ്റാഹീം പറഞ്ഞു: "എന്നാല്‍ അല്ലാഹു സൂര്യനെ ‎കിഴക്കുനിന്നുദിപ്പിക്കുന്നു. നീ അതിനെ പടിഞ്ഞാറുനിന്ന് ‎ഉദിപ്പിക്കുക." അപ്പോള്‍ ആ സത്യനിഷേധി ഉത്തരംമുട്ടി. ‎അക്രമികളായ ജനത്തെ അല്ലാഹു ‎നേര്‍വഴിയിലാക്കുകയില്ല. ‎
\end{malayalam}}
\flushright{\begin{Arabic}
\quranayah[2][259]
\end{Arabic}}
\flushleft{\begin{malayalam}
അല്ലെങ്കിലിതാ മറ്റൊരു ഉദാഹരണം. തകര്‍ന്ന് കീഴ്മേല്‍ ‎മറിഞ്ഞുകിടക്കുന്ന ഒരു പട്ടണത്തിലൂടെ ‎സഞ്ചരിക്കാനിടയായ ഒരാള്‍. അയാള്‍ പറഞ്ഞു: ‎‎"നിര്‍ജീവമായിക്കഴിഞ്ഞശേഷം ഇതിനെ അല്ലാഹു ‎എങ്ങനെ ജീവിപ്പിക്കാനാണ്?" അപ്പോള്‍ അല്ലാഹു ‎അയാളെ നൂറുകൊല്ലം ജീവനറ്റ നിലയിലാക്കി. പിന്നീട് ‎ഉയിര്‍ത്തെഴുന്നേല്‍പിച്ചു. അല്ലാഹു ചോദിച്ചു: "നീ ‎എത്രകാലം ഇങ്ങനെ കഴിച്ചുകൂട്ടി?" അയാള്‍ പറഞ്ഞു: ‎‎"ഒരു ദിവസം; അല്ലെങ്കില്‍ ഒരു ദിവസത്തിന്റെ ഏതാനും ‎ഭാഗം." അല്ലാഹു പറഞ്ഞു: "അല്ല, നീ നൂറ് കൊല്ലം ഇങ്ങനെ ‎കഴിച്ചുകൂട്ടിയിരിക്കുന്നു. നീ നിന്റെ അന്നപാനീയങ്ങള്‍ ‎നോക്കൂ. അവയൊട്ടും വ്യത്യാസപ്പെട്ടിട്ടില്ല. എന്നാല്‍ നീ ‎നിന്റെ കഴുതയെ ഒന്ന് നോക്കൂ. നിന്നെ ജനത്തിന് ഒരു ‎ദൃഷ്ടാന്തമാക്കാനാണ് നാമിങ്ങനെയെല്ലാം ചെയ്തത്. ആ ‎എല്ലുകളിലേക്ക് നോക്കൂ. നാം അവയെ എങ്ങനെ ‎കൂട്ടിയിണക്കുന്നുവെന്നും പിന്നെ എങ്ങനെ മാംസം ‎കൊണ്ട് പൊതിയുന്നുവെന്നും." ഇങ്ങനെ സത്യം ‎വ്യക്തമായപ്പോള്‍ അയാള്‍ പറഞ്ഞു: "അല്ലാഹു ‎എല്ലാറ്റിനും കഴിവുറ്റവനാണെന്ന് ഞാനറിയുന്നു." ‎
\end{malayalam}}
\flushright{\begin{Arabic}
\quranayah[2][260]
\end{Arabic}}
\flushleft{\begin{malayalam}
ഓര്‍ക്കുക: ഇബ്റാഹീം പറഞ്ഞു: "എന്റെ നാഥാ! ‎മരിച്ചവരെ നീ എങ്ങനെ ജീവിപ്പിക്കുന്നുവെന്ന് എനിക്കു ‎കാണിച്ചുതരേണമേ." അല്ലാഹു ചോദിച്ചു: "നീ ‎വിശ്വസിച്ചിട്ടില്ലേ?" അദ്ദേഹം പറഞ്ഞു: "തീര്‍ച്ചയായും ‎അതെ. എന്നാല്‍ എനിക്കു മനസ്സമാധാനം ലഭിക്കാനാണ് ‎ഞാനിതാവശ്യപ്പെടുന്നത്." അല്ലാഹു കല്‍പിച്ചു: "എങ്കില്‍ ‎നാലു പക്ഷികളെ പിടിച്ച് അവയെ നിന്നോട് ‎ഇണക്കമുള്ളതാക്കുക. പിന്നെ അവയുടെ ഓരോ ഭാഗം ‎ഓരോ മലയില്‍ വെക്കുക. എന്നിട്ടവയെ വിളിക്കുക. ‎അവ നിന്റെ അടുക്കല്‍ ഓടിയെത്തും. അറിയുക: ‎അല്ലാഹു പ്രതാപിയും യുക്തിമാനുമാണ്." ‎
\end{malayalam}}
\flushright{\begin{Arabic}
\quranayah[2][261]
\end{Arabic}}
\flushleft{\begin{malayalam}
ദൈവമാര്‍ഗത്തില്‍ തങ്ങളുടെ ധനം ‎ചെലവഴിക്കുന്നവരുടെ ഉപമയിതാ: ഒരു ധാന്യമണി; അത് ‎ഏഴ് കതിരുകളെ മുളപ്പിച്ചു. ഓരോ കതിരിലും നൂറു ‎മണികള്‍. അല്ലാഹു അവനിച്ഛിക്കുന്നവര്‍ക്ക് ഇവ്വിധം ‎ഇരട്ടിയായി കൂട്ടിക്കൊടുക്കുന്നു. അല്ലാഹു ഏറെ ‎വിശാലതയുള്ളവനും സര്‍വജ്ഞനുമാണ്. ‎
\end{malayalam}}
\flushright{\begin{Arabic}
\quranayah[2][262]
\end{Arabic}}
\flushleft{\begin{malayalam}
അല്ലാഹുവിന്റെ മാര്‍ഗത്തില്‍ തങ്ങളുടെ ധനം ‎ചെലവഴിക്കുന്നു; എന്നിട്ട് ചെലവഴിച്ചത് ‎എടുത്തുപറയുകയോ ദാനം വാങ്ങിയവരെ ‎ശല്യപ്പെടുത്തുകയോ ചെയ്യുന്നുമില്ല; അത്തരക്കാര്‍ക്ക് ‎അവരുടെ നാഥന്റെ അടുക്കല്‍ അര്‍ഹമായ ‎പ്രതിഫലമുണ്ട്. അവര്‍ക്ക് പേടിക്കേണ്ടിവരില്ല. ‎ദുഃഖിക്കേണ്ടിയും വരില്ല. ‎
\end{malayalam}}
\flushright{\begin{Arabic}
\quranayah[2][263]
\end{Arabic}}
\flushleft{\begin{malayalam}
ദ്രോഹം പിന്തുടരുന്ന ദാനത്തെക്കാള്‍ ഉത്തമം നല്ലവാക്കു ‎പറയലും വിട്ടുവീഴ്ച കാണിക്കലുമാകുന്നു. അല്ലാഹു ‎സ്വയം പര്യാപ്തനും ഏറെ ക്ഷമയുള്ളവനും തന്നെ. ‎
\end{malayalam}}
\flushright{\begin{Arabic}
\quranayah[2][264]
\end{Arabic}}
\flushleft{\begin{malayalam}
വിശ്വസിച്ചവരേ, കൊടുത്തത് എടുത്തുപറഞ്ഞും സ്വൈരം ‎കെടുത്തിയും നിങ്ങള്‍ നിങ്ങളുടെ ദാനധര്‍മങ്ങളെ ‎പാഴാക്കരുത്. അല്ലാഹുവിലും അന്ത്യദിനത്തിലും ‎വിശ്വസിക്കാതെ ആളുകളെ കാണിക്കാനായി മാത്രം ‎ചെലവഴിക്കുന്നവനെപ്പോലെ. അതിന്റെ ഉപമയിതാ: ‎ഒരുറച്ച പാറ; അതിന്മേല്‍ ഇത്തിരി മണ്ണുണ്ടായിരുന്നു. ‎അങ്ങനെ അതിന്മേല്‍ കനത്ത മഴ പെയ്തു. അതോടെ ‎അത് മിനുത്ത പാറപ്പുറം മാത്രമായി. അവര്‍ ‎അധ്വാനിച്ചതിന്റെ ഫലമൊന്നുമനുഭവിക്കാനവര്‍ക്ക് ‎കഴിഞ്ഞില്ല. അല്ലാഹു സത്യനിഷേധികളായ ജനത്തെ ‎നേര്‍വഴിയിലാക്കുകയില്ല. ‎
\end{malayalam}}
\flushright{\begin{Arabic}
\quranayah[2][265]
\end{Arabic}}
\flushleft{\begin{malayalam}
ദൈവപ്രീതി പ്രതീക്ഷിച്ചും തികഞ്ഞ മനസ്സാന്നിധ്യത്തോടും ‎തങ്ങളുടെ ധനം ചെലവഴിക്കുന്നവരുടെ ഉദാഹരണമിതാ: ‎ഉയര്‍ന്ന പ്രദേശത്തുള്ള ഒരു തോട്ടം; കനത്ത മഴ ‎കിട്ടിയപ്പോള്‍ അതിരട്ടി വിളവു നല്‍കി. അഥവാ, അതിനു ‎കനത്ത മഴകിട്ടാതെ ചാറ്റല്‍ മഴ മാത്രമാണ് ‎ലഭിക്കുന്നതെങ്കില്‍ അതും മതിയാകും. നിങ്ങള്‍ ‎ചെയ്യുന്നതെല്ലാം കാണുന്നവനാണ് അല്ലാഹു. ‎
\end{malayalam}}
\flushright{\begin{Arabic}
\quranayah[2][266]
\end{Arabic}}
\flushleft{\begin{malayalam}
നിങ്ങളിലാര്‍ക്കെങ്കിലും ഈന്തപ്പനകളും മുന്തിരി ‎വള്ളികളുമുള്ള തോട്ടമുണ്ടെന്ന് കരുതുക. അതിന്റെ ‎താഴ്ഭാഗത്തൂടെ അരുവികളൊഴുകിക്കൊണ്ടിരിക്കുന്നു. ‎അതില്‍ എല്ലായിനം കായ്കനികളുമുണ്ട്. അയാള്‍ക്കോ ‎വാര്‍ധക്യം ബാധിച്ചിരിക്കുന്നു. അയാള്‍ക്ക് ദുര്‍ബലരായ ‎കുറേ കുട്ടികളുമുണ്ട്. അപ്പോഴതാ തീക്കാറ്റേറ്റ് ആ തോട്ടം ‎കരിഞ്ഞുപോകുന്നു. ഇങ്ങനെ സംഭവിക്കുന്നത് ‎നിങ്ങളാരെങ്കിലും ഇഷ്ടപ്പെടുമോ? ഇവ്വിധം അല്ലാഹു ‎നിങ്ങള്‍ക്ക് തെളിവുകള്‍ വിവരിച്ചുതരുന്നു. നിങ്ങള്‍ ‎ആലോചിച്ചറിയാന്‍. ‎
\end{malayalam}}
\flushright{\begin{Arabic}
\quranayah[2][267]
\end{Arabic}}
\flushleft{\begin{malayalam}
വിശ്വസിച്ചവരേ, നിങ്ങള്‍ സമ്പാദിച്ച ഉത്തമ ‎വസ്തുക്കളില്‍നിന്നും നിങ്ങള്‍ക്കു നാം ഭൂമിയില്‍ ‎ഉത്പാദിപ്പിച്ചുതന്നതില്‍ നിന്നും നിങ്ങള്‍ ചെലവഴിക്കുക. ‎കണ്ണടച്ചുകൊണ്ടല്ലാതെ നിങ്ങള്‍ക്കു തന്നെ ‎സ്വീകരിക്കാനാവാത്ത ചീത്ത വസ്തുക്കള്‍ ദാനം ‎ചെയ്യാനായി കരുതിവെക്കരുത്. അറിയുക: അല്ലാഹു ‎അന്യാശ്രയമില്ലാത്തവനും സ്തുത്യര്‍ഹനുമാണ്. ‎
\end{malayalam}}
\flushright{\begin{Arabic}
\quranayah[2][268]
\end{Arabic}}
\flushleft{\begin{malayalam}
പിശാച് പട്ടിണിയെപ്പറ്റി നിങ്ങളെ പേടിപ്പിക്കുന്നു. ‎നീചവൃത്തികള്‍ക്കു നിങ്ങളെ പ്രേരിപ്പിക്കുകയും ‎ചെയ്യുന്നു. എന്നാല്‍ അല്ലാഹു തന്നില്‍ നിന്നുള്ള ‎പാപമോചനവും അനുഗ്രഹവും നിങ്ങള്‍ക്ക് വാഗ്ദാനം ‎നല്‍കുന്നു. അല്ലാഹു വിശാലതയുള്ളവനും എല്ലാം ‎അറിയുന്നവനുമാണ്. ‎
\end{malayalam}}
\flushright{\begin{Arabic}
\quranayah[2][269]
\end{Arabic}}
\flushleft{\begin{malayalam}
അല്ലാഹു അവനിച്ഛിക്കുന്നവര്‍ക്ക് അഗാധമായ അറിവ് ‎നല്‍കുന്നു. അത്തരം അറിവ് നല്‍കപ്പെടുന്നവന്ന്, ‎കണക്കില്ലാത്ത നേട്ടമാണ് കിട്ടുന്നത്. എന്നാല്‍ ‎ബുദ്ധിമാന്മാര്‍ മാത്രമേ ഇതില്‍നിന്ന് ‎പാഠമുള്‍ക്കൊള്ളുന്നുള്ളൂ. ‎
\end{malayalam}}
\flushright{\begin{Arabic}
\quranayah[2][270]
\end{Arabic}}
\flushleft{\begin{malayalam}
നിങ്ങള്‍ എത്രയൊക്കെ ചെലവഴിച്ചാലും എന്തൊക്കെ ‎നേര്‍ച്ചയാക്കിയാലും അതെല്ലാം ഉറപ്പായും അല്ലാഹു ‎അറിയുന്നു. അക്രമികള്‍ക്ക് സഹായികളായി ‎ആരുമുണ്ടാവില്ല. ‎
\end{malayalam}}
\flushright{\begin{Arabic}
\quranayah[2][271]
\end{Arabic}}
\flushleft{\begin{malayalam}
നിങ്ങള്‍ ദാനധര്‍മങ്ങള്‍ പരസ്യമായി ചെയ്യുന്നുവെങ്കില്‍ ‎അതു നല്ലതുതന്നെ. എന്നാല്‍ നിങ്ങളത് ‎രഹസ്യമാക്കുകയും പാവങ്ങള്‍ക്ക് ‎നല്‍കുകയുമാണെങ്കില്‍ അതാണ് കൂടുതലുത്തമം. അത് ‎നിങ്ങളുടെ പല പിഴവുകളെയും മായ്ച്ചുകളയും. ‎നിങ്ങള്‍ ചെയ്യുന്നതൊക്കെയും നന്നായറിയുന്നവനാണ് ‎അല്ലാഹു. ‎
\end{malayalam}}
\flushright{\begin{Arabic}
\quranayah[2][272]
\end{Arabic}}
\flushleft{\begin{malayalam}
ജനങ്ങളെ നേര്‍വഴിയിലാക്കേണ്ട ബാധ്യതയൊന്നും ‎നിനക്കില്ല. എന്നാല്‍ അല്ലാഹു അവനിച്ഛിക്കുന്നവരെ ‎നേര്‍വഴിയിലാക്കുന്നു. നിങ്ങള്‍ നല്ലതെന്തെങ്കിലും ‎ചെലവഴിക്കുന്നുവെങ്കില്‍ അത് നിങ്ങളുടെ ‎നന്മക്കുവേണ്ടിത്തന്നെയാണ്. ദൈവപ്രീതി പ്രതീക്ഷിച്ച് ‎മാത്രമാണ് നിങ്ങള്‍ ചെലവഴിക്കേണ്ടത്. നിങ്ങള്‍ ‎നല്ലതെന്തു ചെലവഴിച്ചാലും അതിന്റെ പ്രതിഫലം ‎നിങ്ങള്‍ക്ക് പൂര്‍ണമായും ലഭിക്കും. നിങ്ങളൊട്ടും ‎അനീതിക്കിരയാവുകയില്ല. ‎
\end{malayalam}}
\flushright{\begin{Arabic}
\quranayah[2][273]
\end{Arabic}}
\flushleft{\begin{malayalam}
ഭൂമിയില്‍ സഞ്ചരിച്ച് അന്നമന്വേഷിക്കാന്‍ ‎അവസരമില്ലാത്തവിധം അല്ലാഹുവിന്റെ മാര്‍ഗത്തിലെ ‎തീവ്രയത്നങ്ങളില്‍ ബന്ധിതരായ ദരിദ്രര്‍ക്കുവേണ്ടി ‎ചെലവഴിക്കുക. അവരുടെ മാന്യത കാരണം അവര്‍ ‎ധനികരാണെന്ന് അറിവില്ലാത്തവര്‍ കരുതിയേക്കാം. ‎എന്നാല്‍ ലക്ഷണംകൊണ്ട് നിനക്കവരെ തിരിച്ചറിയാം. ‎അവര്‍ ആളുകളെ ചോദിച്ച് ശല്യംചെയ്യുകയില്ല. നിങ്ങള്‍ ‎നല്ലത് എത്ര ചെലവഴിച്ചാലും തീര്‍ച്ചയായും അല്ലാഹു ‎അതറിയുന്നവനാണ്. ‎
\end{malayalam}}
\flushright{\begin{Arabic}
\quranayah[2][274]
\end{Arabic}}
\flushleft{\begin{malayalam}
രാവും പകലും രഹസ്യമായും പരസ്യമായും തങ്ങളുടെ ‎ധനം ചെലവഴിക്കുന്നവര്‍ക്ക് അവരുടെ നാഥന്റെ ‎അടുക്കല്‍ അവരര്‍ഹിക്കുന്ന പ്രതിഫലമുണ്ട്. ‎അവര്‍ക്കൊന്നും പേടിക്കാനില്ല. അവര്‍ ‎ദുഃഖിക്കേണ്ടിവരികയുമില്ല. ‎
\end{malayalam}}
\flushright{\begin{Arabic}
\quranayah[2][275]
\end{Arabic}}
\flushleft{\begin{malayalam}
പലിശ തിന്നുന്നവര്‍ക്ക്, പിശാചുബാധയേറ്റ് ‎കാലുറപ്പിക്കാനാവാതെ വേച്ച് വേച്ച് ‎എഴുന്നേല്‍ക്കുന്നവനെപ്പോലെയല്ലാതെ ‎നിവര്‍ന്നുനില്‍ക്കാനാവില്ല. “കച്ചവടവും ‎പലിശപോലെത്തന്നെ" എന്ന് അവര്‍ ‎പറഞ്ഞതിനാലാണിത്. എന്നാല്‍ അല്ലാഹു കച്ചവടം ‎അനുവദിച്ചിരിക്കുന്നു. പലിശ വിരോധിക്കുകയും ‎ചെയ്തിരിക്കുന്നു. അതിനാല്‍ അല്ലാഹുവിന്റെ ഉപദേശം ‎വന്നെത്തിയതനുസരിച്ച് ആരെങ്കിലും പലിശയില്‍ നിന്ന് ‎വിരമിച്ചാല്‍ നേരത്തെ പറ്റിപ്പോയത് അവന്നുള്ളതുതന്നെ. ‎അവന്റെ കാര്യം അല്ലാഹുവിങ്കലാണ്. അഥവാ, ‎ആരെങ്കിലും പലിശയിലേക്ക് മടങ്ങുന്നുവെങ്കില്‍ ‎അവരാണ് നരകാവകാശികള്‍. അവരതില്‍ ‎സ്ഥിരവാസികളായിരിക്കും. ‎
\end{malayalam}}
\flushright{\begin{Arabic}
\quranayah[2][276]
\end{Arabic}}
\flushleft{\begin{malayalam}
അല്ലാഹു പലിശയെ ശോഷിപ്പിക്കുന്നു. ദാനധര്‍മങ്ങളെ ‎പോഷിപ്പിക്കുന്നു. നന്ദികെട്ടവനും കുറ്റവാളിയുമായ ‎ആരെയും അല്ലാഹു ഇഷ്ടപ്പെടുന്നില്ല. ‎
\end{malayalam}}
\flushright{\begin{Arabic}
\quranayah[2][277]
\end{Arabic}}
\flushleft{\begin{malayalam}
സത്യവിശ്വാസം സ്വീകരിക്കുകയും സല്‍ക്കര്‍മങ്ങള്‍ ‎പ്രവര്‍ത്തിക്കുകയും നമസ്കാരം നിഷ്ഠയോടെ ‎നിര്‍വഹിക്കുകയും സകാത്ത് നല്‍കുകയും ചെയ്തവര്‍ക്ക് ‎തങ്ങളുടെ നാഥന്റെ അടുക്കല്‍ അവരര്‍ഹിക്കുന്ന ‎പ്രതിഫലമുണ്ട്. അവര്‍ പേടിക്കേണ്ടതില്ല. ‎ദുഃഖിക്കേണ്ടിവരികയുമില്ല. ‎
\end{malayalam}}
\flushright{\begin{Arabic}
\quranayah[2][278]
\end{Arabic}}
\flushleft{\begin{malayalam}
വിശ്വസിച്ചവരേ, നിങ്ങള്‍ അല്ലാഹുവെ സൂക്ഷിക്കുക. ‎പലിശയിനത്തില്‍ ബാക്കിയുള്ളത് ഉപേക്ഷിക്കുക. നിങ്ങള്‍ ‎വിശ്വാസികളെങ്കില്‍! ‎
\end{malayalam}}
\flushright{\begin{Arabic}
\quranayah[2][279]
\end{Arabic}}
\flushleft{\begin{malayalam}
നിങ്ങള്‍ അങ്ങനെ ചെയ്യുന്നില്ലെങ്കില്‍ അറിയുക: ‎നിങ്ങള്‍ക്കെതിരെ അല്ലാഹുവിന്റെയും അവന്റെ ‎ദൂതന്റെയും യുദ്ധപ്രഖ്യാപനമുണ്ട്. നിങ്ങള്‍ ‎പശ്ചാത്തപിക്കുന്നുവെങ്കില്‍ നിങ്ങളുടെ മൂലധനം ‎നിങ്ങള്‍ക്കുതന്നെയുള്ളതാണ്; നിങ്ങള്‍ ആരെയും ‎ദ്രോഹിക്കാതെയും. ആരുടെയും ‎ദ്രോഹത്തിനിരയാകാതെയും. ‎
\end{malayalam}}
\flushright{\begin{Arabic}
\quranayah[2][280]
\end{Arabic}}
\flushleft{\begin{malayalam}
കടക്കാരന്‍ ക്ളേശിക്കുന്നവനെങ്കില്‍ ‎ആശ്വാസമുണ്ടാകുംവരെ അവധി നല്‍കുക. നിങ്ങള്‍ ‎ദാനമായി നല്‍കുന്നതാണ് നിങ്ങള്‍ക്കുത്തമം. നിങ്ങള്‍ ‎അറിയുന്നവരെങ്കില്‍.‎
\end{malayalam}}
\flushright{\begin{Arabic}
\quranayah[2][281]
\end{Arabic}}
\flushleft{\begin{malayalam}
നിങ്ങള്‍ ദൈവസന്നിധിയിലേക്ക് തിരിച്ചുചെല്ലുന്ന ‎നാളിനെ സൂക്ഷിക്കുക. അന്ന് ഓരോരുത്തര്‍ക്കും തങ്ങള്‍ ‎പ്രവര്‍ത്തിച്ചതിന്റെ പ്രതിഫലം പൂര്‍ണമായി ‎നല്‍കുന്നതാണ്. ആരും അനീതിക്കിരയാവില്ല. ‎
\end{malayalam}}
\flushright{\begin{Arabic}
\quranayah[2][282]
\end{Arabic}}
\flushleft{\begin{malayalam}
വിശ്വസിച്ചവരേ, നിശ്ചിത അവധി നിര്‍ണയിച്ച് നിങ്ങള്‍ ‎വല്ല കടമിടപാടും നടത്തുകയാണെങ്കില്‍ അത് ‎രേഖപ്പെടുത്തിവെക്കണം. എഴുതുന്നയാള്‍ ‎നിങ്ങള്‍ക്കിടയില്‍ അത് നീതിയോടെ കുറിച്ചുവെക്കട്ടെ. ‎ഒരെഴുത്തുകാരനും അല്ലാഹു അവനെ പഠിപ്പിച്ച പോലെ ‎എഴുതാന്‍ വിസമ്മതിക്കരുത്. അയാളത് ‎രേഖപ്പെടുത്തുകയും കടബാധ്യതയുള്ളവന്‍ ‎പറഞ്ഞുകൊടുക്കുകയും വേണം. അയാള്‍ അല്ലാഹുവെ ‎സൂക്ഷിക്കുകയും തന്റെ ഉത്തരവാദിത്വത്തില്‍ വീഴ്ച ‎വരുത്താതിരിക്കുകയും ചെയ്യട്ടെ. അഥവാ, കടക്കാരന്‍ ‎മൂഢനോ കാര്യശേഷി കുറഞ്ഞവനോ ‎പറഞ്ഞുകൊടുക്കാന്‍ കഴിവില്ലാത്തവനോ ആണെങ്കില്‍ ‎അയാളുടെ രക്ഷിതാവ് അയാള്‍ക്കുവേണ്ടി ‎നീതിനിഷ്ഠമായി വാചകം പറഞ്ഞുകൊടുക്കണം. ‎നിങ്ങളിലെ രണ്ടു പുരുഷന്മാരെ സാക്ഷിനിര്‍ത്തണം. ‎അഥവാ, രണ്ടു പുരുഷന്മാരില്ലെങ്കില്‍ നിങ്ങള്‍ക്കിഷ്ടമുള്ള ‎ഒരു പുരുഷനും രണ്ട് സ്ത്രീ യും ‎സാക്ഷികളായുണ്ടാവണം. അവരില്‍ ഒരുവള്‍ക്ക് ‎പിശകുപറ്റിയാല്‍ മറ്റവള്‍ ഓര്‍മിപ്പിക്കാനാണിത്. ‎സാക്ഷികളെ വിളിച്ചാല്‍ അവരതിന് വിസമ്മതിക്കരുത്. ‎ഇടപാട് ചെറുതായാലും വലുതായാലും അതിന്റെ ‎അവധി നിശ്ചയിച്ച് രേഖപ്പെടുത്താന്‍ വിമുഖത ‎കാണിക്കരുത്. അതാണ് അല്ലാഹുവിങ്കല്‍ ഏറ്റം ‎നീതിനിഷ്ഠം. സാക്ഷ്യത്തിന് കൂടുതല്‍ ‎കരുത്തുനല്‍കുന്നതും നിങ്ങള്‍ക്ക് സംശയം ‎തോന്നാതിരിക്കാന്‍ ഏറ്റം പറ്റിയതും അതുതന്നെ. എന്നാല്‍ ‎നിങ്ങള്‍ റൊക്കമായി നടത്തുന്ന കച്ചവട ‎ഇടപാടുകള്‍ക്കിതു ബാധകമല്ല. അത് ‎രേഖപ്പെടുത്താതിരിക്കുന്നതില്‍ തെറ്റൊന്നുമില്ല. എന്നാലും ‎നിങ്ങള്‍ കൊള്ളക്കൊടുക്കകള്‍ നടത്തുമ്പോള്‍ ‎സാക്ഷിനിര്‍ത്തണം. അതോടൊപ്പം എഴുത്തുകാരനോ ‎സാക്ഷിയോ പീഡിപ്പിക്കപ്പെടരുത്. അങ്ങനെ നിങ്ങള്‍ ‎ചെയ്യുന്നുവെങ്കില്‍ അത് അധര്‍മമാണ്. നിങ്ങള്‍ ‎അല്ലാഹുവെ സൂക്ഷിക്കുക. അല്ലാഹു നിങ്ങള്‍ക്കെല്ലാം ‎വിശദമായി പഠിപ്പിച്ചുതരികയാണ്. അല്ലാഹു എല്ലാ ‎കാര്യങ്ങളും നന്നായറിയുന്നവനത്രെ. ‎
\end{malayalam}}
\flushright{\begin{Arabic}
\quranayah[2][283]
\end{Arabic}}
\flushleft{\begin{malayalam}
അഥവാ, നിങ്ങള്‍ യാത്രയിലാവുകയും എഴുതാന്‍ ആളെ ‎കിട്ടാതിരിക്കുകയുമാണെങ്കില്‍ പണയവസ്തുക്കള്‍ ‎കൈമാറിയാല്‍ മതി. നിങ്ങളിലൊരാള്‍ മറ്റൊരാളെ ‎വല്ലതും വിശ്വസിച്ചേല്‍പിച്ചാല്‍ അയാള്‍ തന്റെ ‎വിശ്വാസ്യത പാലിക്കണം. തന്റെ നാഥനെ ‎സൂക്ഷിക്കുകയും വേണം. നിങ്ങള്‍ സാക്ഷ്യം ഒരിക്കലും ‎മറച്ചുവെക്കരുത്. ആരതിനെ മറച്ചുവെക്കുന്നുവോ, ‎അവന്റെ മനസ്സ് പാപപങ്കിലമാണ്. നിങ്ങള്‍ ‎ചെയ്യുന്നതെല്ലാം നന്നായറിയുന്നവനാണ് അല്ലാഹു. ‎
\end{malayalam}}
\flushright{\begin{Arabic}
\quranayah[2][284]
\end{Arabic}}
\flushleft{\begin{malayalam}
ആകാശഭൂമികളിലുള്ളതൊക്കെയും ‎അല്ലാഹുവിന്റേതാണ്. നിങ്ങളുടെ മനസ്സിലുള്ളത് ‎നിങ്ങള്‍ വെളിപ്പെടുത്തിയാലും ഒളിപ്പിച്ചുവെച്ചാലും ‎അല്ലാഹു അതിന്റെപേരില്‍ നിങ്ങളെ വിചാരണ ചെയ്യും. ‎അങ്ങനെ അവനിച്ഛിക്കുന്നവര്‍ക്ക് അവന്‍ മാപ്പേകും. ‎അവനിച്ഛിക്കുന്നവരെ അവന്‍ ശിക്ഷിക്കും. അല്ലാഹു ‎എല്ലാ കാര്യങ്ങള്‍ക്കും കഴിവുറ്റവനാണ്. ‎
\end{malayalam}}
\flushright{\begin{Arabic}
\quranayah[2][285]
\end{Arabic}}
\flushleft{\begin{malayalam}
ദൈവദൂതന്‍ തന്റെ നാഥനില്‍ നിന്ന് തനിക്ക് ‎ഇറക്കിക്കിട്ടിയതില്‍ വിശ്വസിച്ചിരിക്കുന്നു. അതുപോലെ ‎സത്യവിശ്വാസികളും. അവരെല്ലാം അല്ലാഹുവിലും ‎അവന്റെ മലക്കുകളിലും വേദപുസ്തകങ്ങളിലും ‎ദൂതന്മാരിലും വിശ്വസിച്ചിരിക്കുന്നു. ‎‎“ദൈവദൂതന്മാരില്‍ ആരോടും ഞങ്ങള്‍ വിവേചനം ‎കല്‍പിക്കുന്നില്ലെ"ന്ന് അവര്‍ സമ്മതിക്കുന്നു. ‎അവരിങ്ങനെ പ്രാര്‍ഥിക്കുകയും ചെയ്യുന്നു: "ഞങ്ങള്‍ ‎കേള്‍ക്കുകയും അനുസരിക്കുകയും ചെയ്തിരിക്കുന്നു. ‎ഞങ്ങളുടെ നാഥാ! ഞങ്ങള്‍ക്കു നീ മാപ്പേകണമേ. ‎നിന്നിലേക്കാണല്ലോ ഞങ്ങളുടെ മടക്കം." ‎
\end{malayalam}}
\flushright{\begin{Arabic}
\quranayah[2][286]
\end{Arabic}}
\flushleft{\begin{malayalam}
അല്ലാഹു ആരെയും അയാളുടെ കഴിവില്‍ കവിഞ്ഞതിന് ‎നിര്‍ബന്ധിക്കുന്നില്ല. ഒരുവന്‍ സമ്പാദിച്ചതിന്റെ സദ്ഫലം ‎അവന്നുള്ളതാണ്. അവന്‍ സമ്പാദിച്ചതിന്റെ ദുഷ്ഫലവും ‎അവന്നുതന്നെ. "ഞങ്ങളുടെ നാഥാ; മറവി ‎സംഭവിച്ചതിന്റെയും പിഴവു പറ്റിയതിന്റെയും പേരില്‍ ‎ഞങ്ങളെ നീ പിടികൂടരുതേ. ഞങ്ങളുടെ നാഥാ! ഞങ്ങളുടെ ‎പൂര്‍വികരെ വഹിപ്പിച്ചതുപോലുള്ള ഭാരം ഞങ്ങളുടെ ‎മേല്‍ നീ ചുമത്തരുതേ. ഞങ്ങളുടെ നാഥാ! ഞങ്ങള്‍ക്കു ‎താങ്ങാനാവാത്ത കൊടും ഭാരം ഞങ്ങളെ നീ ‎വഹിപ്പിക്കരുതേ. ഞങ്ങള്‍ക്കു നീ മാപ്പേകേണമേ! ‎പൊറുത്തു തരേണമേ. ഞങ്ങളോടു നീ കരുണ ‎കാണിക്കേണമേ. നീയാണല്ലോ ഞങ്ങളുടെ രക്ഷകന്‍. ‎അതിനാല്‍ സത്യനിഷേധികളായ ജനത്തിനെതിരെ ഞങ്ങളെ ‎നീ സഹായിക്കേണമേ." ‎
\end{malayalam}}
\chapter{\textmalayalam{ആലു ഇംറാന്‍ ( ഇംറാന്‍ കുടുംബം )}}
\begin{Arabic}
\Huge{\centerline{\basmalah}}\end{Arabic}
\flushright{\begin{Arabic}
\quranayah[3][1]
\end{Arabic}}
\flushleft{\begin{malayalam}
അലിഫ്-ലാം-മീം. ‎
\end{malayalam}}
\flushright{\begin{Arabic}
\quranayah[3][2]
\end{Arabic}}
\flushleft{\begin{malayalam}
അല്ലാഹു; അവനല്ലാതെ ദൈവമില്ല. അവന്‍ എന്നെന്നും ‎ജീവിച്ചിരിക്കുന്നവന്‍. എല്ലാറ്റിനെയും ‎പരിപാലിക്കുന്നവന്‍. ‎
\end{malayalam}}
\flushright{\begin{Arabic}
\quranayah[3][3]
\end{Arabic}}
\flushleft{\begin{malayalam}
സത്യസന്ദേശവുമായി ഈ വേദം നിനക്ക് ഇറക്കിത്തന്നത് ‎അവനാണ്. അത് മുന്‍വേദങ്ങളെ ശരിവെക്കുന്നു. ‎തൌറാത്തും ഇഞ്ചീലും അവന്‍ ഇറക്കിക്കൊടുത്തു. ‎
\end{malayalam}}
\flushright{\begin{Arabic}
\quranayah[3][4]
\end{Arabic}}
\flushleft{\begin{malayalam}
അത് ഇതിനു മുമ്പാണ്. ഇതെല്ലാം മനുഷ്യര്‍ക്ക് ‎വഴികാണിക്കാനുള്ളതാണ്. ശരിതെറ്റുകളെ ‎വേര്‍തിരിച്ചറിയാനുള്ള പ്രമാണവും അവന്‍ ‎ഇറക്കിത്തന്നു. അതിനാല്‍ ഇനിയും അല്ലാഹുവിന്റെ ‎വചനങ്ങളെ തള്ളിപ്പറഞ്ഞവരാരോ അവര്‍ക്ക് കഠിനമായ ‎ശിക്ഷയുണ്ട്. അല്ലാഹു പ്രതാപിയും ശിക്ഷ ‎നടപ്പാക്കുന്നവനുമാകുന്നു. ‎
\end{malayalam}}
\flushright{\begin{Arabic}
\quranayah[3][5]
\end{Arabic}}
\flushleft{\begin{malayalam}
മണ്ണിലോ മാനത്തോ അല്ലാഹുവിന്റെ കാഴ്ചയില്‍ ‎പെടാത്ത ഒന്നും തന്നെയില്ല; തീര്‍ച്ച. ‎
\end{malayalam}}
\flushright{\begin{Arabic}
\quranayah[3][6]
\end{Arabic}}
\flushleft{\begin{malayalam}
അവനാണ് ഗര്‍ഭാശയങ്ങളില്‍ അവനിച്ഛിക്കും വിധം ‎നിങ്ങളെ രൂപപ്പെടുത്തുന്നത്. അവനല്ലാതെ ദൈവമില്ല. ‎അവന്‍ പ്രതാപിയാണ്. യുക്തിമാനും. ‎
\end{malayalam}}
\flushright{\begin{Arabic}
\quranayah[3][7]
\end{Arabic}}
\flushleft{\begin{malayalam}
അവനാണ് നിനക്ക് ഈ വേദം ഇറക്കിത്തന്നത്. അതില്‍ ‎വ്യക്തവും ഖണ്ഡിതവുമായ വാക്യങ്ങളുണ്ട്. അവയാണ് ‎വേദഗ്രന്ഥത്തിന്റെ കാതലായ ഭാഗം. തെളിച്ചു ‎പറഞ്ഞിട്ടില്ലാത്ത ചില വാക്യങ്ങളുമുണ്ട്. മനസ്സില്‍ ‎വക്രതയുള്ളവര്‍ കുഴപ്പമാഗ്രഹിച്ച് ‎ആശയവ്യക്തതയില്ലാത്ത വാക്യങ്ങളുടെ പിറകെ ‎പോവുകയും അവയെ വ്യാഖ്യാനിക്കാന്‍ ശ്രമിക്കുകയും ‎ചെയ്യുന്നു. എന്നാല്‍ അവയുടെ ശരിയായ വ്യാഖ്യാനം ‎അല്ലാഹുവിനേ അറിയുകയുള്ളൂ. അറിവില്‍ പാകത ‎നേടിയവര്‍ പറയും: "ഞങ്ങളതില്‍ വിശ്വസിച്ചിരിക്കുന്നു. ‎എല്ലാം ഞങ്ങളുടെ നാഥനില്‍ നിന്നുള്ളതാണ്." ‎ബുദ്ധിമാന്മാര്‍ മാത്രമേ ആലോചിച്ചറിയുന്നുള്ളൂ. ‎
\end{malayalam}}
\flushright{\begin{Arabic}
\quranayah[3][8]
\end{Arabic}}
\flushleft{\begin{malayalam}
അവര്‍ പ്രാര്‍ഥിക്കുന്നു: "ഞങ്ങളുടെ നാഥാ, ഞങ്ങളെ നീ ‎നേര്‍വഴിയിലാക്കിയശേഷം ഞങ്ങളുടെ മനസ്സുകളെ ‎അതില്‍നിന്ന് തെറ്റിച്ചുകളയരുതേ! നിന്റെ പക്കല്‍ ‎നിന്നുള്ള കാരുണ്യം ഞങ്ങള്‍ക്കു നല്‍കേണമേ. സംശയമില്ല, ‎നീ തന്നെയാണ് അത്യുദാരന്‍". ‎
\end{malayalam}}
\flushright{\begin{Arabic}
\quranayah[3][9]
\end{Arabic}}
\flushleft{\begin{malayalam}
‎"ഞങ്ങളുടെ നാഥാ! തീര്‍ച്ചയായും ഒരു നാള്‍ നീ ‎ജനങ്ങളെയൊക്കെ ഒരുമിച്ചുകൂട്ടും. അതിലൊട്ടും ‎സംശയമില്ല. നിശ്ചയമായും അല്ലാഹു കരാര്‍ ‎ലംഘിക്കുകയില്ല." ‎
\end{malayalam}}
\flushright{\begin{Arabic}
\quranayah[3][10]
\end{Arabic}}
\flushleft{\begin{malayalam}
സത്യനിഷേധികള്‍ക്ക് അല്ലാഹുവിന്റെ ശിക്ഷയില്‍നിന്ന് ‎രക്ഷകിട്ടാന്‍ അവരുടെ സ്വത്തോ സന്താനങ്ങളോ തീരെ ‎ഉപകരിക്കുകയില്ല. അവരാണ് നരകത്തീയിലെ ‎വിറകായിത്തീരുന്നവര്‍. ‎
\end{malayalam}}
\flushright{\begin{Arabic}
\quranayah[3][11]
\end{Arabic}}
\flushleft{\begin{malayalam}
ഫറവോന്റെ ആള്‍ക്കാരുടെയും അവര്‍ക്ക് ‎മുമ്പുള്ളവരുടെയും അനുഭവം തന്നെയാണ് ‎ഇവര്‍ക്കുമുണ്ടാവുക. അവരെല്ലാം നമ്മുടെ ‎തെളിവുകളെ തള്ളിക്കളഞ്ഞു. അപ്പോള്‍ അവരുടെ ‎കുറ്റകൃത്യങ്ങള്‍ കാരണമായി അല്ലാഹു അവരെ ‎പിടികൂടി. അല്ലാഹു കഠിനമായി ‎ശിക്ഷിക്കുന്നവനാകുന്നു. ‎
\end{malayalam}}
\flushright{\begin{Arabic}
\quranayah[3][12]
\end{Arabic}}
\flushleft{\begin{malayalam}
സത്യനിഷേധികളോടു പറയുക: ഒട്ടും വൈകാതെ ‎നിങ്ങളെ കീഴടക്കി കൂട്ടത്തോടെ നരകത്തീയിലേക്ക് ‎നയിക്കും. അതെത്ര ചീത്ത സ്ഥലം! ‎
\end{malayalam}}
\flushright{\begin{Arabic}
\quranayah[3][13]
\end{Arabic}}
\flushleft{\begin{malayalam}
പരസ്പരം ഏറ്റുമുട്ടിയ രണ്ടു കൂട്ടരില്‍ നിങ്ങള്‍ക്ക് ‎ഗുണപാഠമുണ്ട്. ഒരു വിഭാഗം ദൈവമാര്‍ഗത്തില്‍ ‎പടവെട്ടുകയായിരുന്നു. മറു വിഭാഗം ‎സത്യനിഷേധികളും. സത്യനിഷേധികളുടെ ദൃഷ്ടിയില്‍ ‎സത്യവിശ്വാസികള്‍ തങ്ങളുടെ ഇരട്ടിയുള്ളതായാണ് ‎തോന്നിയത്. അല്ലാഹു അവനിച്ഛിക്കുന്നവരെ തന്റെ ‎സഹായത്താല്‍ കരുത്തരാക്കുന്നു. തീര്‍ച്ചയായും ‎ഉള്‍ക്കാഴ്ചയുള്ളവര്‍ക്കൊക്കെ ഇതില്‍ വലിയ ‎ഗുണപാഠമുണ്ട്. ‎
\end{malayalam}}
\flushright{\begin{Arabic}
\quranayah[3][14]
\end{Arabic}}
\flushleft{\begin{malayalam}
ഭാര്യമാര്‍, മക്കള്‍, സ്വര്‍ണത്തിന്റെയും വെള്ളിയുടെയും ‎കൂമ്പാരങ്ങള്‍, മേത്തരം കുതിരകള്‍, കന്നുകാലികള്‍, ‎കൃഷിയിടങ്ങള്‍ എന്നീ ഇഷ്ടവസ്തുക്കളോടുള്ള മോഹം ‎മനുഷ്യര്‍ക്ക് ചേതോഹരമാക്കിയിരിക്കുന്നു. ‎അതൊക്കെയും ഐഹികജീവിതത്തിലെ സുഖഭോഗ ‎വിഭവങ്ങളാണ്. എന്നാല്‍ ഏറ്റവും ഉത്തമമായ സങ്കേതം ‎അല്ലാഹുവിങ്കലാകുന്നു. ‎
\end{malayalam}}
\flushright{\begin{Arabic}
\quranayah[3][15]
\end{Arabic}}
\flushleft{\begin{malayalam}
പറയുക: ഇതിനേക്കാള്‍ ശ്രേഷ്ഠമായത് ‎ഞാനറിയിച്ചുതരട്ടെയോ? ഭക്തി പുലര്‍ത്തിയവര്‍ക്ക് ‎തങ്ങളുടെ നാഥന്റെ അടുക്കല്‍ താഴ്ഭാഗത്തൂടെ ‎ആറുകളൊഴുകുന്ന സ്വര്‍ഗീയാരാമങ്ങളുണ്ട്. അവരവിടെ ‎സ്ഥിരവാസികളായിരിക്കും. അവര്‍ക്കവിടെ ‎പരിശുദ്ധരായ ഇണകളുണ്ട്; ഒപ്പം ദൈവപ്രീതിയും. ‎അല്ലാഹു തന്റെ അടിമകളുടെ അവസ്ഥകളൊക്കെ ‎കണ്ടറിയുന്നവനാണ്. ‎
\end{malayalam}}
\flushright{\begin{Arabic}
\quranayah[3][16]
\end{Arabic}}
\flushleft{\begin{malayalam}
ഇങ്ങനെ പ്രാര്‍ഥിക്കുന്നവരാണവര്‍: "ഞങ്ങളുടെ നാഥാ, ‎ഞങ്ങളിതാ വിശ്വസിച്ചിരിക്കുന്നു. അതിനാല്‍ നീ ‎ഞങ്ങളുടെ പാപങ്ങള്‍ പൊറുത്തുതരേണമേ. ‎നരകശിക്ഷയില്‍നിന്ന് ഞങ്ങളെ നീ രക്ഷിക്കേണമേ." ‎
\end{malayalam}}
\flushright{\begin{Arabic}
\quranayah[3][17]
\end{Arabic}}
\flushleft{\begin{malayalam}
അവര്‍ ക്ഷമ പാലിക്കുന്നവരാണ്. സത്യസന്ധരാണ്. ‎ദൈവഭക്തരാണ്. ദൈവമാര്‍ഗത്തില്‍ ധനം ‎ചെലവഴിക്കുന്നവരാണ്. രാവിന്റെ അവസാന ‎യാമങ്ങളില്‍ പാപമോചനത്തിനായി പ്രാര്‍ഥിക്കുന്നവരും. ‎
\end{malayalam}}
\flushright{\begin{Arabic}
\quranayah[3][18]
\end{Arabic}}
\flushleft{\begin{malayalam}
താനല്ലാതെ ദൈവമില്ലെന്നതിന് അല്ലാഹു സാക്ഷ്യം ‎സമര്‍പ്പിച്ചിരിക്കുന്നു. മലക്കുകളും ജ്ഞാനികളുമെല്ലാം ‎അതിനു സാക്ഷ്യം വഹിച്ചിട്ടുണ്ട്. അവന്‍ നീതി ‎നടത്തുന്നവനത്രെ. അവനല്ലാതെ ദൈവമില്ല. പ്രതാപിയും ‎യുക്തിമാനുമാണവന്‍. ‎
\end{malayalam}}
\flushright{\begin{Arabic}
\quranayah[3][19]
\end{Arabic}}
\flushleft{\begin{malayalam}
ഉറപ്പായും അല്ലാഹുവിങ്കല്‍ മതമെന്നാല്‍ ഇസ്ലാംതന്നെ. ‎വേദപുസ്തകം ലഭിച്ചവര്‍ ഇതില്‍നിന്ന് തെന്നിമാറിയത് ‎അവര്‍ക്ക് അറിവ് വന്നെത്തിയശേഷം മാത്രമാണ്. ‎അവര്‍ക്കിടയിലെ കിടമത്സരം കാരണമാണത്. ‎ആരെങ്കിലും അല്ലാഹുവിന്റെ തെളിവുകളെ ‎തള്ളിക്കളയുന്നുവെങ്കില്‍ അറിയുക: അല്ലാഹു ‎അതിവേഗം വിചാരണ നടത്തുന്നവനാണ്. ‎
\end{malayalam}}
\flushright{\begin{Arabic}
\quranayah[3][20]
\end{Arabic}}
\flushleft{\begin{malayalam}
അഥവാ, അവര്‍ നിന്നോട് തര്‍ക്കിക്കുകയാണെങ്കില്‍ ‎പറയുക: "ഞാന്‍ എന്നെ പൂര്‍ണമായും അല്ലാഹുവിന് ‎സമര്‍പ്പിച്ചിരിക്കുന്നു; എന്നെ പിന്തുടര്‍ന്നവരും." ‎വേദഗ്രന്ഥം ലഭിച്ചവരോടും ‎അക്ഷരജ്ഞാനമില്ലാത്തവരോടും നീ ചോദിക്കുക: “നിങ്ങള്‍ ‎ദൈവത്തിന് കീഴ്പ്പെട്ടോ?" അവര്‍ കീഴ്പ്പെട്ടു കഴിഞ്ഞാല്‍ ‎ഉറപ്പായും അവര്‍ നേര്‍വഴിയിലായി. അവര്‍ ‎പിന്തിരിഞ്ഞു പോയാലോ അവര്‍ക്ക് സന്മാര്‍ഗം ‎എത്തിക്കേണ്ട ബാധ്യതയേ നിനക്കുള്ളൂ. അല്ലാഹു തന്റെ ‎ദാസന്മാരുടെ കാര്യം സൂക്ഷ്മമായി ‎വീക്ഷിക്കുന്നവനാണ്. ‎
\end{malayalam}}
\flushright{\begin{Arabic}
\quranayah[3][21]
\end{Arabic}}
\flushleft{\begin{malayalam}
അല്ലാഹുവിന്റെ തെളിവുകളെ തള്ളിപ്പറയുകയും ‎അന്യായമായി പ്രവാചകന്മാരെ കൊലപ്പെടുത്തുകയും ‎നീതി പാലിക്കാന്‍ കല്‍പിക്കുന്നവരെ വധിക്കുകയും ‎ചെയ്യുന്നവര്‍ക്ക് നോവേറിയ ശിക്ഷയുണ്ടെന്ന് ‎‎“സുവാര്‍ത്ത" അറിയിക്കുക. ‎
\end{malayalam}}
\flushright{\begin{Arabic}
\quranayah[3][22]
\end{Arabic}}
\flushleft{\begin{malayalam}
തങ്ങളുടെ കര്‍മങ്ങളെല്ലാം ഇഹത്തിലും പരത്തിലും ‎പാഴായിപ്പോയവരാണവര്‍. അവര്‍ക്കു സഹായികളായി ‎ആരുമുണ്ടാവില്ല. ‎
\end{malayalam}}
\flushright{\begin{Arabic}
\quranayah[3][23]
\end{Arabic}}
\flushleft{\begin{malayalam}
വേദവിജ്ഞാനത്തില്‍നിന്ന് ഒരു വിഹിതം കിട്ടിയ കൂട്ടരെ ‎ക്കുറിച്ച് നീ അറിഞ്ഞില്ലേ? അവര്‍ക്കിടയിലെ ‎തര്‍ക്കങ്ങളില്‍ തീര്‍പ്പു കല്‍പിക്കാന്‍ അല്ലാഹുവിന്റെ ‎വേദത്തിലേക്ക് അവരെ ക്ഷണിക്കുമ്പോള്‍ ഒരു വിഭാഗം ‎ഒഴിഞ്ഞു മാറി പിന്തിരിഞ്ഞുപോകുന്നു. ‎
\end{malayalam}}
\flushright{\begin{Arabic}
\quranayah[3][24]
\end{Arabic}}
\flushleft{\begin{malayalam}
നിര്‍ണിതമായ ഏതാനും നാളുകളല്ലാതെ നരകത്തീ ‎തങ്ങളെ തൊടില്ലെന്ന് വാദിച്ചതിനാലാണ് ‎അവരങ്ങനെയായത്. അവര്‍ സ്വയം കെട്ടിച്ചമച്ച വാദങ്ങള്‍ ‎അവരുടെ മതകാര്യത്തിലവരെ ‎വഞ്ചിതരാക്കിയിരിക്കുന്നു. ‎
\end{malayalam}}
\flushright{\begin{Arabic}
\quranayah[3][25]
\end{Arabic}}
\flushleft{\begin{malayalam}
ഒരു ദിനം നാമവരെ ഒരുമിച്ചുകൂട്ടും. അന്ന് അവരുടെ ‎അവസ്ഥ എന്തായിരിക്കും? അങ്ങനെ ‎സംഭവിക്കുമെന്നതിലൊട്ടും സംശയമില്ല. അന്ന് ‎ഓരോരുത്തര്‍ക്കും താന്‍ പ്രവര്‍ത്തിച്ചതിന്റെ പ്രതിഫലം ‎പൂര്‍ണമായി നല്‍കും. ആരോടും ഒട്ടും ‎അനീതിയുണ്ടാവില്ല. ‎
\end{malayalam}}
\flushright{\begin{Arabic}
\quranayah[3][26]
\end{Arabic}}
\flushleft{\begin{malayalam}
പറയുക: എല്ലാ ആധിപത്യങ്ങള്‍ക്കും ഉടമയായ ‎അല്ലാഹുവേ, നീ ഇച്ഛിക്കുന്നവര്‍ക്ക് നീ ‎ആധിപത്യമേകുന്നു. നീ ഇച്ഛിക്കുന്നവരില്‍ നിന്ന് നീ ‎ആധിപത്യം നീക്കിക്കളയുന്നു. നീ ഇച്ഛിക്കുന്നവരെ നീ ‎പ്രതാപികളാക്കുന്നു. നീ ഇച്ഛിക്കുന്നവരെ നീ ‎നിന്ദ്യരാക്കുകയും ചെയ്യുന്നു. സമസ്ത സൌഭാഗ്യങ്ങളും ‎നിന്റെ കൈയിലാണ്. തീര്‍ച്ചയായും നീ ‎എല്ലാകാര്യത്തിനും കഴിവുറ്റവന്‍ തന്നെ. ‎
\end{malayalam}}
\flushright{\begin{Arabic}
\quranayah[3][27]
\end{Arabic}}
\flushleft{\begin{malayalam}
നീ രാവിനെ പകലിലേക്ക് കടത്തിവിടുന്നു. പകലിനെ ‎രാവിലേക്കും കടത്തിവിടുന്നു. നീ ജീവനില്ലാത്തതില്‍നിന്ന് ‎ജീവനുള്ളതിനെ ഉണ്ടാക്കുന്നു. ജീവനുള്ളതില്‍ നിന്ന് ‎ജീവനില്ലാത്തതിനെ പുറപ്പെടുവിക്കുന്നു. നീ ‎ഇച്ഛിക്കുന്നവര്‍ക്ക് നീ കണക്കില്ലാതെ കൊടുക്കുന്നു. ‎
\end{malayalam}}
\flushright{\begin{Arabic}
\quranayah[3][28]
\end{Arabic}}
\flushleft{\begin{malayalam}
സത്യവിശ്വാസികള്‍ സത്യവിശ്വാസികളെയല്ലാതെ ‎സത്യനിഷേധികളെ ആത്മമിത്രങ്ങളാ ക്കരുത്. ആരെങ്കിലും ‎അങ്ങനെ ചെയ്യുന്നുവെങ്കില്‍ അവന് അല്ലാഹുവുമായി ‎ഒരു ബന്ധവുമില്ല. നിങ്ങള്‍ അവരുമായി കരുതലോടെ ‎വര്‍ത്തിക്കുകയാണെങ്കില്‍ അതിനു വിരോധമില്ല. ‎അല്ലാഹു അവനെപ്പറ്റി നിങ്ങള്‍ക്ക് താക്കീത് നല്‍കുന്നു. ‎അല്ലാഹുവിങ്കലേക്കാണല്ലോ നിങ്ങള്‍ ‎തിരിച്ചുചെല്ലേണ്ടത്. ‎
\end{malayalam}}
\flushright{\begin{Arabic}
\quranayah[3][29]
\end{Arabic}}
\flushleft{\begin{malayalam}
പറയുക: നിങ്ങളുടെ മനസ്സിലുള്ളത് നിങ്ങള്‍ ‎ഒളിപ്പിച്ചുവെച്ചാലും തെളിയിച്ചുകാട്ടിയാലും അല്ലാഹു ‎അറിയും. ആകാശഭൂമികളിലുള്ളതെല്ലാം അവനറിയുന്നു. ‎അല്ലാഹു എല്ലാ കാര്യങ്ങള്‍ക്കും കഴിവുറ്റവനാണ്. ‎
\end{malayalam}}
\flushright{\begin{Arabic}
\quranayah[3][30]
\end{Arabic}}
\flushleft{\begin{malayalam}
ഓര്‍ക്കുക: ഓരോ മനുഷ്യനും താന്‍ ചെയ്ത ‎നന്മയുടെയും തിന്മയുടെയും ഫലം നേരില്‍ ‎കണ്ടറിയും ദിനം വരാനിരിക്കുന്നു. ആ ദിനം തന്നില്‍ ‎നിന്ന് ഏറെ ദൂരെയായിരുന്നെങ്കിലെന്ന് ഓരോ മനുഷ്യനും ‎അന്ന് ആഗ്രഹിച്ചുപോകും. അല്ലാഹു തന്റെ ‎ശിക്ഷയെക്കുറിച്ച് നിങ്ങള്‍ക്ക് താക്കീത് നല്‍കുന്നു. ‎അല്ലാഹു തന്റെ അടിമകളോട് പരമദയാലുവാകുന്നു. ‎
\end{malayalam}}
\flushright{\begin{Arabic}
\quranayah[3][31]
\end{Arabic}}
\flushleft{\begin{malayalam}
പറയുക: നിങ്ങള്‍ അല്ലാഹുവെ സ്നേഹിക്കുന്നുവെങ്കില്‍ ‎എന്നെ പിന്തുടരുക. അപ്പോള്‍ അല്ലാഹു നിങ്ങളെയും ‎സ്നേഹിക്കും. നിങ്ങളുടെ പാപങ്ങള്‍ പൊറുത്തുതരും. ‎അല്ലാഹു ഏറെ പൊറുക്കുന്നവനും ‎പരമകാരുണികനുമാകുന്നു. ‎
\end{malayalam}}
\flushright{\begin{Arabic}
\quranayah[3][32]
\end{Arabic}}
\flushleft{\begin{malayalam}
പറയുക: അല്ലാഹുവെയും അവന്റെ ദൂതനെയും നിങ്ങള്‍ ‎അനുസരിക്കുക. നിങ്ങള്‍ വിസമ്മതിക്കുന്നുവെങ്കില്‍ ‎അറിയുക: അല്ലാഹു സത്യനിഷേധികളെ ‎സ്നേഹിക്കുകയില്ല; തീര്‍ച്ച. ‎
\end{malayalam}}
\flushright{\begin{Arabic}
\quranayah[3][33]
\end{Arabic}}
\flushleft{\begin{malayalam}
ആദം, നൂഹ്, ഇബ്റാഹീംകുടുംബം, ഇംറാന്‍കുടുംബം ‎ഇവരെയൊക്കെ നിശ്ചയമായും ലോകജനതയില്‍ നിന്ന് ‎അല്ലാഹു പ്രത്യേകം തെരഞ്ഞെടുത്തിരിക്കുന്നു. ‎
\end{malayalam}}
\flushright{\begin{Arabic}
\quranayah[3][34]
\end{Arabic}}
\flushleft{\begin{malayalam}
അവരെല്ലാം ഒരേ വംശപരമ്പരയിലെ സന്തതികളാണ്. ‎അല്ലാഹു എല്ലാം കേള്‍ക്കുന്നവനും അറിയുന്നവനുമത്രെ. ‎
\end{malayalam}}
\flushright{\begin{Arabic}
\quranayah[3][35]
\end{Arabic}}
\flushleft{\begin{malayalam}
ഓര്‍ക്കുക: ഇംറാന്റെ ഭാര്യ ഇങ്ങനെ പ്രാര്‍ഥിച്ച ‎സന്ദര്‍ഭം: "എന്റെ നാഥാ, എന്റെ വയറ്റിലെ കുഞ്ഞിനെ ‎നിന്റെ സേവനത്തിനായി സമര്‍പ്പിക്കാന്‍ ഞാന്‍ ‎നേര്‍ച്ചയാക്കിയിരിക്കുന്നു; എന്നില്‍നിന്ന് നീയിതു ‎സ്വീകരിക്കേണമേ. നീ എല്ലാം കേള്‍ക്കുന്നവനും ‎അറിയുന്നവനുമല്ലോ." ‎
\end{malayalam}}
\flushright{\begin{Arabic}
\quranayah[3][36]
\end{Arabic}}
\flushleft{\begin{malayalam}
പിന്നീട് ആ കുഞ്ഞിനെ പ്രസവിച്ചപ്പോള്‍ അവള്‍ പറഞ്ഞു: ‎‎"എന്റെ നാഥാ, ഞാന്‍ പ്രസവിച്ചത് ‎പെണ്‍കുഞ്ഞിനെയാണ്- അവള്‍ പ്രസവിച്ചത് ആരെയെന്ന് ‎നന്നായറിയുന്നവനാണ് അല്ലാഹു.-ആണ്‍കുഞ്ഞ് ‎പെണ്‍കുഞ്ഞിനെപ്പോലെയല്ലല്ലോ. ആ കുഞ്ഞിന് ഞാന്‍ ‎മര്‍യം എന്നു പേരിട്ടിരിക്കുന്നു. അവളെയും അവളുടെ ‎സന്താനപരമ്പരകളെയും ശപിക്കപ്പെട്ട പിശാചില്‍ നിന്ന് ‎രക്ഷിക്കാനായി ഞാനിതാ നിന്നിലഭയം തേടുന്നു." ‎
\end{malayalam}}
\flushright{\begin{Arabic}
\quranayah[3][37]
\end{Arabic}}
\flushleft{\begin{malayalam}
അങ്ങനെ അവളുടെ നാഥന്‍ അവളെ നല്ല നിലയില്‍ ‎സ്വീകരിച്ചു.മെച്ചപ്പെട്ട രീതിയില്‍ ‎വളര്‍ത്തിക്കൊണ്ടുവന്നു. സകരിയ്യായെ അവളുടെ ‎സംരക്ഷകനാക്കി. സകരിയ്യാ മിഹ്റാബി ല്‍ അവളുടെ ‎അടുത്തു ചെന്നപ്പോഴെല്ലാം അവള്‍ക്കരികെ ‎ആഹാരപദാര്‍ഥങ്ങള്‍ കാണാറുണ്ടായിരുന്നു. അതിനാല്‍ ‎അദ്ദേഹം ചോദിച്ചു: "മര്‍യം, നിനക്കെവിടെനിന്നാണിത് ‎കിട്ടുന്നത്?" അവള്‍ അറിയിച്ചു: "ഇത് അല്ലാഹുവിങ്കല്‍ ‎നിന്നുള്ളതാണ്. അല്ലാഹു അവനിച്ഛിക്കുന്നവര്‍ക്ക് ‎കണക്കറ്റ് കൊടുക്കുന്നു." ‎
\end{malayalam}}
\flushright{\begin{Arabic}
\quranayah[3][38]
\end{Arabic}}
\flushleft{\begin{malayalam}
അവിടെവെച്ച് സകരിയ്യാ തന്റെ നാഥനോട് പ്രാര്‍ഥിച്ചു: ‎‎"എന്റെ നാഥാ, എനിക്കു നീ നിന്റെ വകയായി ‎നല്ലവരായ മക്കളെ നല്‍കേണമേ. തീര്‍ച്ചയായും നീ ‎പ്രാര്‍ഥന കേള്‍ക്കുന്നവനല്ലോ." ‎
\end{malayalam}}
\flushright{\begin{Arabic}
\quranayah[3][39]
\end{Arabic}}
\flushleft{\begin{malayalam}
അങ്ങനെ അദ്ദേഹം “മിഹ്റാബില്‍" പ്രാര്‍ഥിച്ചുകൊണ്ടു ‎നില്‍ക്കെ മലക്കുകള്‍ അദ്ദേഹത്തെ വിളിച്ചു പറഞ്ഞു: ‎‎"നിശ്ചയമായും അല്ലാഹു നിന്നെ യഹ്യാ യെ സംബന്ധിച്ച ‎ശുഭവാര്‍ത്ത അറിയിക്കുന്നു. അല്ലാഹുവില്‍നിന്നുള്ള ‎വചന ത്തെ സത്യപ്പെടുത്തുന്നവനായാണ് അവന്‍ വരിക. ‎അവന്‍ നേതാവും ആത്മസംയമനം പാലിക്കുന്നവനും ‎സദ്വൃത്തരില്‍പ്പെട്ട പ്രവാചകനുമായിരിക്കും." ‎
\end{malayalam}}
\flushright{\begin{Arabic}
\quranayah[3][40]
\end{Arabic}}
\flushleft{\begin{malayalam}
സകരിയ്യാ ചോദിച്ചു: "എന്റെ നാഥാ! എനിക്കെങ്ങനെ ‎ഇനിയൊരു പുത്രനുണ്ടാകും? ഞാന്‍ ‎കിഴവനായിക്കഴിഞ്ഞു. എന്റെ ഭാര്യയോ വന്ധ്യയും." ‎അല്ലാഹു അറിയിച്ചു: "അതൊക്കെ ശരി തന്നെ. എന്നാല്‍ ‎അല്ലാഹു അവനിച്ഛിക്കുന്നതു ചെയ്യുന്നു." ‎
\end{malayalam}}
\flushright{\begin{Arabic}
\quranayah[3][41]
\end{Arabic}}
\flushleft{\begin{malayalam}
അദ്ദേഹം പറഞ്ഞു: "എന്റെ നാഥാ! എനിക്കു നീ ‎ഒരടയാളം കാണിച്ചുതന്നാലും." അല്ലാഹു അറിയിച്ചു: ‎‎"നിനക്കുള്ള അടയാളം മൂന്നുനാള്‍ ആംഗ്യഭാഷയിലല്ലാതെ ‎ജനങ്ങളോട് സംസാരിക്കാതിരിക്കലാണ്. നിന്റെ നാഥനെ ‎ആവോളം സ്മരിക്കുക. രാവിലെയും വൈകുന്നേരവും ‎അവന്റെ വിശുദ്ധി വാഴ്ത്തുക." ‎
\end{malayalam}}
\flushright{\begin{Arabic}
\quranayah[3][42]
\end{Arabic}}
\flushleft{\begin{malayalam}
മലക്കുകള്‍ പറഞ്ഞതോര്‍ക്കുക: "മര്‍യം, അല്ലാഹു നിന്നെ ‎പ്രത്യേകം തെരഞ്ഞെടുത്തിരിക്കുന്നു. വിശുദ്ധയും ‎ലോകത്തിലെ മറ്റേത് സ്ത്രീകളെക്കാളും ‎വിശിഷ്ടയുമാക്കിയിരിക്കുന്നു". ‎
\end{malayalam}}
\flushright{\begin{Arabic}
\quranayah[3][43]
\end{Arabic}}
\flushleft{\begin{malayalam}
‎"മര്‍യം, നീ നിന്റെ നാഥനോട് ഭക്തി പുലര്‍ത്തുക. ‎അവനെ സാഷ്ടാംഗം പ്രണമിക്കുക. തല ‎കുനിക്കുന്നവരോടൊപ്പം നമിക്കുക." ‎
\end{malayalam}}
\flushright{\begin{Arabic}
\quranayah[3][44]
\end{Arabic}}
\flushleft{\begin{malayalam}
നാം നിനക്ക് ബോധനംനല്‍കുന്ന അഭൌതിക ‎വിവരങ്ങളില്‍പെട്ടതാണിത്. തങ്ങളില്‍ ആരാണ് ‎മര്‍യമിന്റെ സംരക്ഷണം ഏറ്റെടുക്കേണ്ടതെന്ന് ‎നിശ്ചയിക്കാന്‍ അവര്‍ തങ്ങളുടെ എഴുത്താണികള്‍ ‎എറിഞ്ഞപ്പോള്‍ നീ അവരോടൊപ്പമുണ്ടായിരുന്നില്ല. ‎അക്കാര്യത്തിലവര്‍ തര്‍ക്കിച്ചുകൊണ്ടിരുന്നപ്പോഴും നീ ‎അവിടെയുണ്ടായിരുന്നില്ല. ‎
\end{malayalam}}
\flushright{\begin{Arabic}
\quranayah[3][45]
\end{Arabic}}
\flushleft{\begin{malayalam}
മലക്കുകള്‍ പറഞ്ഞതോര്‍ക്കുക: "മര്‍യം, അല്ലാഹു തന്നില്‍ ‎നിന്നുള്ള ഒരു വചന ത്തെ സംബന്ധിച്ച് നിന്നെയിതാ ‎ശുഭവാര്‍ത്ത അറിയിക്കുന്നു. അവന്റെ പേര്‍ ‎മര്‍യമിന്റെ മകന്‍ മസീഹ് ഈസാ എന്നാകുന്നു. അവന്‍ ‎ഈ ലോകത്തും പരലോകത്തും ഉന്നതസ്ഥാനീയനും ദിവ്യ ‎സാമീപ്യം സിദ്ധിച്ചവനുമായിരിക്കും". ‎
\end{malayalam}}
\flushright{\begin{Arabic}
\quranayah[3][46]
\end{Arabic}}
\flushleft{\begin{malayalam}
‎“തൊട്ടിലില്‍വെച്ചുതന്നെ അവന്‍ ജനത്തോടു ‎സംസാരിക്കും. പ്രായമായശേഷവും. അവന്‍ സദാ ‎സദ്വൃത്തനായിരിക്കും." ‎
\end{malayalam}}
\flushright{\begin{Arabic}
\quranayah[3][47]
\end{Arabic}}
\flushleft{\begin{malayalam}
അവള്‍ ചോദിച്ചു: "എന്റെ നാഥാ, എനിക്കെങ്ങനെ ‎പുത്രനുണ്ടാകും? എന്നെ ഒരു പുരുഷനും ‎തൊട്ടിട്ടുപോലുമില്ല!" അല്ലാഹു അറിയിച്ചു: "അത് ‎ശരിതന്നെ. എന്നാല്‍, അല്ലാഹു അവനിച്ഛിക്കുന്നത് ‎സൃഷ്ടിക്കുന്നു. അവന്‍ ഒരു കാര്യം തീരുമാനിച്ചാല്‍ ‎‎“ഉണ്ടാവുക" എന്നു പറയുകയേ വേണ്ടൂ. അപ്പോഴേക്കും ‎അതുണ്ടാവുന്നു." ‎
\end{malayalam}}
\flushright{\begin{Arabic}
\quranayah[3][48]
\end{Arabic}}
\flushleft{\begin{malayalam}
അവനെ അല്ലാഹു വേദവും തത്ത്വജ്ഞാനവും ‎തൌറാത്തും ഇഞ്ചീലും പഠിപ്പിക്കും. ‎
\end{malayalam}}
\flushright{\begin{Arabic}
\quranayah[3][49]
\end{Arabic}}
\flushleft{\begin{malayalam}
ഇസ്രയേല്‍ മക്കളിലേക്കു ദൂതനായി നിയോഗിക്കും. ‎അവന്‍ പറയും: "ഞാന്‍ നിങ്ങളുടെ നാഥനില്‍ നിന്നുള്ള ‎തെളിവുമായാണ് നിങ്ങളുടെ അടുത്തു വന്നിരിക്കുന്നത്. ‎ഞാന്‍ നിങ്ങള്‍ക്കായി കളിമണ്ണുകൊണ്ട് പക്ഷിയുടെ ‎രൂപമുണ്ടാക്കും. പിന്നെ ഞാനതിലൂതിയാല്‍ ‎അല്ലാഹുവിന്റെ അനുമതിയോടെ അതൊരു ‎പക്ഷിയായിത്തീരും. ജന്മനാ കണ്ണില്ലാത്തവനെയും ‎പാണ്ഡുരോഗിയെയും ഞാന്‍ സുഖപ്പെടുത്തും. ‎ദൈവഹിതമനുസരിച്ച് മരിച്ചവരെ ജീവിപ്പിക്കും. നിങ്ങള്‍ ‎തിന്നുന്നതെന്തെന്നും വീടുകളില്‍ സൂക്ഷിച്ചുവെച്ചത് ‎ഏതൊക്കെയെന്നും ഞാന്‍ നിങ്ങള്‍ക്കു വിവരിച്ചു തരും. ‎തീര്‍ച്ചയായും അതിലെല്ലാം നിങ്ങള്‍ക്ക് ‎അടയാളങ്ങളുണ്ട്; നിങ്ങള്‍ വിശ്വാസികളെങ്കില്‍! ‎
\end{malayalam}}
\flushright{\begin{Arabic}
\quranayah[3][50]
\end{Arabic}}
\flushleft{\begin{malayalam}
‎"തൌറാത്തില്‍ നിന്ന് എന്റെ മുമ്പിലുള്ളതിനെ ‎ശരിവെക്കുന്നവനായാണ് എന്നെ അയച്ചത്. നിങ്ങള്‍ക്ക് ‎നിഷിദ്ധമായിരുന്ന ചിലത് അനുവദിച്ചുതരാനും. ‎നിങ്ങളുടെ നാഥനില്‍ നിന്നുള്ള തെളിവുമായാണ് ഞാന്‍ ‎നിങ്ങളിലേക്ക് വന്നത്. അതിനാല്‍ നിങ്ങള്‍ ‎ദൈവഭക്തരാവുക. എന്നെ അനുസരിക്കുക. ‎
\end{malayalam}}
\flushright{\begin{Arabic}
\quranayah[3][51]
\end{Arabic}}
\flushleft{\begin{malayalam}
‎"നിശ്ചയമായും അല്ലാഹു എന്റെയും നിങ്ങളുടെയും ‎നാഥനാണ്. അതിനാല്‍ അവന്നുമാത്രം വഴിപ്പെടുക. ‎ഇതാണ് നേര്‍വഴി." ‎
\end{malayalam}}
\flushright{\begin{Arabic}
\quranayah[3][52]
\end{Arabic}}
\flushleft{\begin{malayalam}
പിന്നീട് ഈസാക്ക് അവരുടെ സത്യനിഷേധഭാവം ‎ബോധ്യമായപ്പോള്‍ ചോദിച്ചു: "ദൈവമാര്‍ഗത്തില്‍ ‎എനിക്കു സഹായികളായി ആരുണ്ട്?" ഹവാരികള്‍ ‎പറഞ്ഞു: "ഞങ്ങള്‍ അല്ലാഹുവിന്റെ സഹായികളാണ്. ‎ഞങ്ങള്‍ അല്ലാഹുവില്‍ വിശ്വസിച്ചിരിക്കുന്നു. ഞങ്ങള്‍ ‎അല്ലാഹുവെ അനുസരിക്കുന്നവരാണെന്ന് അങ്ങ് ‎സാക്ഷ്യം വഹിച്ചാലും". ‎
\end{malayalam}}
\flushright{\begin{Arabic}
\quranayah[3][53]
\end{Arabic}}
\flushleft{\begin{malayalam}
‎"ഞങ്ങളുടെ നാഥാ, നീ ഇറക്കിത്തന്നതില്‍ ഞങ്ങള്‍ ‎വിശ്വസിച്ചിരിക്കുന്നു. നിന്റെ ദൂതനെ ഞങ്ങള്‍ ‎പിന്തുടരുകയും ചെയ്തിരിക്കുന്നു. അതിനാല്‍ ‎സത്യത്തിന് സാക്ഷ്യം വഹിക്കുന്നവരുടെ കൂട്ടത്തില്‍ ‎ഞങ്ങളെയും നീ ഉള്‍പ്പെടുത്തേണമേ." ‎
\end{malayalam}}
\flushright{\begin{Arabic}
\quranayah[3][54]
\end{Arabic}}
\flushleft{\begin{malayalam}
സത്യനിഷേധികള്‍ ഗൂഢതന്ത്രം പ്രയോഗിച്ചു. ‎അല്ലാഹുവും തന്ത്രം പ്രയോഗിച്ചു. തന്ത്രപ്രയോഗങ്ങളില്‍ ‎മറ്റാരെക്കാളും മികച്ചവന്‍ അല്ലാഹു തന്നെ. ‎
\end{malayalam}}
\flushright{\begin{Arabic}
\quranayah[3][55]
\end{Arabic}}
\flushleft{\begin{malayalam}
അല്ലാഹു പറഞ്ഞതോര്‍ക്കുക: ഈസാ, ഞാന്‍ നിന്നെ ‎പൂര്‍ണമായി ഏറ്റെടുക്കും. നിന്നെ എന്നിലേക്ക് ‎ഉയര്‍ത്തും. സത്യനിഷേധികളില്‍ നിന്ന് അടര്‍ത്തിയെടുത്ത് ‎നിന്നെ നാം വിശുദ്ധനാക്കും. നിന്നെ പിന്‍പറ്റിയവരെ ‎ഉയിര്‍ത്തെഴുന്നേല്‍പുനാള്‍വരെ സത്യനിഷേധികളെക്കാള്‍ ‎ഉന്നതരാക്കും. പിന്നെ നിങ്ങളുടെയൊക്കെ തിരിച്ചുവരവ് ‎എന്റെ അടുത്തേക്കാണ്. നിങ്ങള്‍ ഭിന്നിച്ചിരുന്ന ‎കാര്യങ്ങളില്‍ അപ്പോള്‍ ഞാന്‍ തീര്‍പ്പു കല്‍പിക്കും. ‎
\end{malayalam}}
\flushright{\begin{Arabic}
\quranayah[3][56]
\end{Arabic}}
\flushleft{\begin{malayalam}
എന്നാല്‍ സത്യനിഷേധികളെ നാം ഇഹത്തിലും ‎പരത്തിലും കഠിനമായി ശിക്ഷിക്കും. അവര്‍ക്ക് ‎തുണയായി ആരുമുണ്ടാവില്ല. ‎
\end{malayalam}}
\flushright{\begin{Arabic}
\quranayah[3][57]
\end{Arabic}}
\flushleft{\begin{malayalam}
അതോടൊപ്പം, സത്യവിശ്വാസം സ്വീകരിക്കുകയും ‎സല്‍ക്കര്‍മങ്ങള്‍ പ്രവര്‍ത്തിക്കുകയും ചെയ്തവര്‍ക്കുള്ള ‎പ്രതിഫലം അല്ലാഹു പൂര്‍ണമായും നല്‍കും. ‎അക്രമികളെ അല്ലാഹു ഇഷ്ടപ്പെടുന്നില്ല. ‎
\end{malayalam}}
\flushright{\begin{Arabic}
\quranayah[3][58]
\end{Arabic}}
\flushleft{\begin{malayalam}
നിനക്കു നാം ഈ ഓതിക്കേള്‍പ്പിക്കുന്നത് ‎ദൈവവചനങ്ങളില്‍പ്പെട്ടതാണ്. യുക്തിപൂര്‍വമായ ‎ഉദ്ബോധനത്തില്‍നിന്നുള്ളവയും. ‎
\end{malayalam}}
\flushright{\begin{Arabic}
\quranayah[3][59]
\end{Arabic}}
\flushleft{\begin{malayalam}
സംശയമില്ല. അല്ലാഹുവിന്റെ അടുത്ത് ഈസാ ‎ആദമിനെപ്പോലെയാണ്. അല്ലാഹു ആദമിനെ മണ്ണില്‍നിന്ന് ‎സൃഷ്ടിച്ചു. പിന്നെ അതിനോട് “ഉണ്ടാവുക" എന്ന് ‎കല്‍പിച്ചു. അങ്ങനെ അദ്ദേഹം ഉണ്ടായി. ‎
\end{malayalam}}
\flushright{\begin{Arabic}
\quranayah[3][60]
\end{Arabic}}
\flushleft{\begin{malayalam}
ഇതെല്ലാം നിന്റെ നാഥനില്‍ നിന്ന് കിട്ടിയ ‎സത്യസന്ദേശമാണ്. അതിനാല്‍ നീ ‎സംശയാലുക്കളില്‍പ്പെടാതിരിക്കുക. ‎
\end{malayalam}}
\flushright{\begin{Arabic}
\quranayah[3][61]
\end{Arabic}}
\flushleft{\begin{malayalam}
നിനക്ക് യഥാര്‍ഥ ജ്ഞാനം വന്നെത്തിയശേഷം ‎ഇക്കാര്യത്തില്‍ ആരെങ്കിലും നിന്നോട് തര്‍ക്കിക്കാന്‍ ‎വരുന്നുവെങ്കില്‍ അവരോടു പറയുക: "നിങ്ങള്‍ വരൂ! ‎നമ്മുടെ ഇരുകൂട്ടരുടെയും മക്കളെയും സ്ത്രീകളെയും ‎നമുക്കു വിളിച്ചുചേര്‍ക്കാം. നമുക്ക് ഒത്തുചേര്‍ന്ന്, ‎കൂട്ടായി അകമഴിഞ്ഞ് പ്രാര്‍ഥിക്കാം: “കള്ളം ‎പറയുന്നവര്‍ക്ക് ദൈവശാപം ഉണ്ടാവട്ടെ" ‎
\end{malayalam}}
\flushright{\begin{Arabic}
\quranayah[3][62]
\end{Arabic}}
\flushleft{\begin{malayalam}
ഇത് സത്യസന്ധമായ സംഭവവിവരണമാണ്; തീര്‍ച്ച. ‎അല്ലാഹു അല്ലാതെ ദൈവമില്ല. ഉറപ്പായും അല്ലാഹു ‎തന്നെയാണ് പ്രതാപിയും യുക്തിമാനും. ‎
\end{malayalam}}
\flushright{\begin{Arabic}
\quranayah[3][63]
\end{Arabic}}
\flushleft{\begin{malayalam}
ഇനിയും അവര്‍ പിന്തിരിഞ്ഞുപോവുകയാണെങ്കില്‍ ‎ഓര്‍ക്കുക: തീര്‍ച്ചയായും അല്ലാഹു നാശകാരികളെപ്പറ്റി ‎നന്നായറിയുന്നവനാണ്. ‎
\end{malayalam}}
\flushright{\begin{Arabic}
\quranayah[3][64]
\end{Arabic}}
\flushleft{\begin{malayalam}
പറയുക: വേദവിശ്വാസികളേ, ഞങ്ങളും നിങ്ങളും ‎ഒന്നുപോലെ അംഗീകരിക്കുന്ന തത്ത്വത്തിലേക്കു വരിക. ‎അതിതാണ്: "അല്ലാഹു അല്ലാത്ത ആര്‍ക്കും നാം ‎വഴിപ്പെടാതിരിക്കുക; അവനില്‍ ഒന്നിനെയും ‎പങ്കുചേര്‍ക്കാതിരിക്കുക; അല്ലാഹുവെ കൂടാതെ നമ്മില്‍ ‎ചിലര്‍ മറ്റുചിലരെ രക്ഷാധികാരികളാക്കാതിരിക്കുക." ‎ഇനിയും അവര്‍പിന്തിരിഞ്ഞുപോകുന്നുവെങ്കില്‍ ‎പറയുക: "ഞങ്ങള്‍ മുസ്ലിംകളാണ്. നിങ്ങളതിന് ‎സാക്ഷികളാവുക." ‎
\end{malayalam}}
\flushright{\begin{Arabic}
\quranayah[3][65]
\end{Arabic}}
\flushleft{\begin{malayalam}
വേദവിശ്വാസികളേ, ഇബ്റാഹീമിന്റെ കാര്യത്തില്‍ ‎നിങ്ങളെന്തിനു തര്‍ക്കിക്കുന്നു? തൌറാത്തും ഇഞ്ചീലും ‎അവതരിച്ചത് അദ്ദേഹത്തിനുശേഷമാണല്ലോ. നിങ്ങള്‍ ഒട്ടും ‎ആലോചിക്കാത്തതെന്ത്? ‎
\end{malayalam}}
\flushright{\begin{Arabic}
\quranayah[3][66]
\end{Arabic}}
\flushleft{\begin{malayalam}
നിങ്ങള്‍ക്ക് അറിവുള്ള കാര്യത്തില്‍ നിങ്ങള്‍ ഒരുപാട് ‎തര്‍ക്കിച്ചു. ഇപ്പോള്‍ നിങ്ങളെന്തിന് അറിയാത്ത ‎കാര്യത്തിലും തര്‍ക്കിക്കുന്നു? അല്ലാഹു എല്ലാം ‎അറിയുന്നു. നിങ്ങളോ ഒന്നും അറിയുന്നുമില്ല. ‎
\end{malayalam}}
\flushright{\begin{Arabic}
\quranayah[3][67]
\end{Arabic}}
\flushleft{\begin{malayalam}
ഇബ്റാഹീം ജൂതനോ ക്രിസ്ത്യാനിയോ ആയിരുന്നില്ല. ‎വക്രതയില്ലാത്ത മുസ്ലിമായിരുന്നു. അദ്ദേഹം ഒരിക്കലും ‎ബഹുദൈവ വിശ്വാസിയായിരുന്നില്ല. ‎
\end{malayalam}}
\flushright{\begin{Arabic}
\quranayah[3][68]
\end{Arabic}}
\flushleft{\begin{malayalam}
തീര്‍ച്ചയായും ജനങ്ങളില്‍ ഇബ്റാഹീമിനോട് ഏറ്റം ‎അടുത്തവര്‍ അദ്ദേഹത്തെ പിന്‍പറ്റിയവരും ഈ ‎പ്രവാചകനും അദ്ദേഹത്തില്‍ വിശ്വസിച്ചവരുമാണ്. ‎അല്ലാഹു സത്യവിശ്വാസികളുടെ രക്ഷകനാകുന്നു. ‎
\end{malayalam}}
\flushright{\begin{Arabic}
\quranayah[3][69]
\end{Arabic}}
\flushleft{\begin{malayalam}
വേദക്കാരിലൊരു കൂട്ടര്‍ നിങ്ങളെ വഴിതെറ്റിക്കാന്‍ ‎കഴിഞ്ഞെങ്കിലെന്ന് കൊതിക്കുന്നു. സത്യത്തില്‍ അവര്‍ ‎അവരെത്തന്നെയാണ് വഴിതെറ്റിക്കുന്നത്. പക്ഷേ ‎അവരതറിയുന്നില്ല. ‎
\end{malayalam}}
\flushright{\begin{Arabic}
\quranayah[3][70]
\end{Arabic}}
\flushleft{\begin{malayalam}
വേദക്കാരേ, നിങ്ങളെന്താണ് ദൈവിക ദൃഷ്ടാന്തങ്ങളെ ‎തള്ളിപ്പറയുന്നത്? നിങ്ങളവയ്ക്ക് സാക്ഷ്യം ‎വഹിച്ചവരല്ലോ. ‎
\end{malayalam}}
\flushright{\begin{Arabic}
\quranayah[3][71]
\end{Arabic}}
\flushleft{\begin{malayalam}
വേദക്കാരേ, നിങ്ങളെന്തിനാണ് സത്യത്തെ അസത്യവുമായി ‎കൂട്ടിക്കുഴക്കുന്നത്? അറിഞ്ഞുകൊണ്ട് നിങ്ങളെന്തിന് ‎സത്യത്തെ മറച്ചുവെയ്ക്കുന്നു? ‎
\end{malayalam}}
\flushright{\begin{Arabic}
\quranayah[3][72]
\end{Arabic}}
\flushleft{\begin{malayalam}
വേദക്കാരിലൊരുകൂട്ടര്‍ പറയുന്നു: "ഈ വിശ്വാസികള്‍ക്ക് ‎അവതീര്‍ണമായതില്‍ പകലിന്റെ പ്രാരംഭത്തില്‍ നിങ്ങള്‍ ‎വിശ്വസിച്ചുകൊള്ളുക. പകലറുതിയില്‍ അതിനെ ‎തള്ളിപ്പറയുകയും ചെയ്യുക. അതുകണ്ട് ആ ‎വിശ്വാസികള്‍ നമ്മിലേക്ക് തിരിച്ചുവന്നേക്കാം". ‎
\end{malayalam}}
\flushright{\begin{Arabic}
\quranayah[3][73]
\end{Arabic}}
\flushleft{\begin{malayalam}
‎"നിങ്ങളുടെ മതത്തെ പിന്‍പറ്റുന്നവരെയല്ലാതെ ആരെയും ‎നിങ്ങള്‍ വിശ്വസിക്കരുത്". പറയുക: “അല്ലാഹുവിന്റെ ‎സന്മാര്‍ഗം മാത്രമാണ് യഥാര്‍ഥ സത്യപാത". “നിങ്ങള്‍ക്കു ‎തന്നത് മറ്റാര്‍ക്കെങ്കിലും നല്‍കുമെന്നോ നിങ്ങളുടെ ‎നാഥന്റെ അടുക്കല്‍ അവരാരെങ്കിലും നിങ്ങളോട് ‎ന്യായവാദം നടത്തുമെന്നോ നിങ്ങള്‍ ‎വിശ്വസിക്കരുതെ"ന്നും ആ വേദക്കാര്‍ പറഞ്ഞു. പറയുക: ‎അനുഗ്രഹങ്ങളെല്ലാം അല്ലാഹുവിന്റെ കയ്യിലാണ്. ‎അവനിച്ഛിക്കുന്നവര്‍ക്ക് അവനത് നല്‍കുന്നു. അല്ലാഹു ‎ഏറെ വിശാലതയുള്ളവനാണ്; എല്ലാം അറിയുന്നവനും. ‎
\end{malayalam}}
\flushright{\begin{Arabic}
\quranayah[3][74]
\end{Arabic}}
\flushleft{\begin{malayalam}
അല്ലാഹു അവനിച്ഛിക്കുന്നവരെ തന്റെ അനുഗ്രഹത്തിന് ‎പ്രത്യേകം തെരഞ്ഞെടുക്കുന്നു. അല്ലാഹു അതിമഹത്തായ ‎അനുഗ്രഹമുള്ളവനാണ്. ‎
\end{malayalam}}
\flushright{\begin{Arabic}
\quranayah[3][75]
\end{Arabic}}
\flushleft{\begin{malayalam}
വേദവിശ്വാസികളിലൊരു വിഭാഗം നീയൊരു ‎സ്വര്‍ണക്കൂമ്പാരം തന്നെ വിശ്വസിച്ചേല്‍പിച്ചാലും അത് ‎തിരിച്ചുതരുന്നവരാണ്. മറ്റൊരു വിഭാഗമുണ്ട്. കേവലം ‎ഒരു ദീനാര്‍ വിശ്വസിച്ചേല്പിച്ചാല്‍പോലും നിനക്ക് ‎അവരത് മടക്കിത്തരില്ല- നീ നിരന്തരം ‎പിന്തുടര്‍ന്നാലല്ലാതെ. അതിനു കാരണം അവരിങ്ങനെ ‎വാദിച്ചുകൊണ്ടിരിക്കുന്നതാണ്: "ഈ നിരക്ഷരരുടെ ‎കാര്യത്തില്‍ ഞങ്ങള്‍ക്ക് കുറ്റമുണ്ടാവാനിടയില്ല." അവര്‍ ‎ബോധപൂര്‍വം അല്ലാഹുവിന്റെ പേരില്‍ കള്ളം ‎പറയുകയാണ്. ‎
\end{malayalam}}
\flushright{\begin{Arabic}
\quranayah[3][76]
\end{Arabic}}
\flushleft{\begin{malayalam}
അല്ല; ആരെങ്കിലും തന്റെ പ്രതിജ്ഞ പാലിക്കുകയും ‎സൂക്ഷ്മത പുലര്‍ത്തുകയും ചെയ്യുന്നുവെങ്കില്‍ അറിയുക: ‎തീര്‍ച്ചയായും അല്ലാഹു സൂക്ഷ്മത പുലര്‍ത്തുന്നവരെ ‎ഇഷ്ടപ്പെടുന്നു. ‎
\end{malayalam}}
\flushright{\begin{Arabic}
\quranayah[3][77]
\end{Arabic}}
\flushleft{\begin{malayalam}
അല്ലാഹുവോടുള്ള പ്രതിജ്ഞയും സ്വന്തം ശപഥങ്ങളും ‎നിസ്സാര വിലയ്ക്ക് വില്‍ക്കുന്നവര്‍ക്ക് പരലോകത്ത് ഒരു ‎വിഹിതവുമുണ്ടാവില്ല. ഉയിര്‍ത്തെഴുന്നേല്‍പുനാളില്‍ ‎അല്ലാഹു അവരോട് മിണ്ടുകയില്ല. അവരെ ‎നോക്കുകയോ സംസ്കരിക്കുകയോ ഇല്ല. അവര്‍ക്ക് ‎നോവേറിയ ശിക്ഷയുണ്ട്. ‎
\end{malayalam}}
\flushright{\begin{Arabic}
\quranayah[3][78]
\end{Arabic}}
\flushleft{\begin{malayalam}
വേദം വായിക്കുമ്പോള്‍ നാവ് കോട്ടുന്ന ചിലരും ‎അക്കൂട്ടത്തിലുണ്ട്; അതൊക്കെ ‎വേദപുസ്തകത്തിലുള്ളതാണെന്ന് നിങ്ങള്‍ ‎ധരിക്കാനാണത്. എന്നാലതൊന്നും ‎വേദപുസ്തകത്തിലുള്ളതല്ല. അതൊക്കെ ദൈവത്തിങ്കല്‍ ‎നിന്നുള്ളതാണെന്ന് അവരവകാശപ്പെടും. യഥാര്‍ഥത്തില്‍ ‎അതൊന്നും ദൈവത്തിങ്കല്‍ നിന്നുള്ളതല്ല. അവര്‍ ‎ബോധപൂര്‍വം അല്ലാഹുവിന്റെ പേരില്‍ കള്ളം ‎പറയുകയാണ്. ‎
\end{malayalam}}
\flushright{\begin{Arabic}
\quranayah[3][79]
\end{Arabic}}
\flushleft{\begin{malayalam}
ഒരാള്‍ക്ക് അല്ലാഹു വേദപുസ്തകവും തത്ത്വജ്ഞാനവും ‎പ്രവാചകത്വവും നല്‍കുക; എന്നിട്ട് അയാള്‍ ജനങ്ങളോട് ‎‎“നിങ്ങള്‍ അല്ലാഹുവിന്റെ അടിമകളാകുന്നതിനുപകരം ‎എന്റെ അടിമകളാവുക" എന്ന് പറയുക; ഇത് ഒരു ‎മനുഷ്യനില്‍നിന്ന് ഒരിക്കലും സംഭവിക്കാവതല്ല. മറിച്ച് ‎അയാള്‍ പറയുക “നിങ്ങള്‍ വേദപുസ്തകം പഠിക്കുകയും ‎പഠിപ്പിക്കുകയും ചെയ്യുന്നതിലൂടെ കളങ്കമേശാത്ത ‎ദൈവഭക്തരാവുക" എന്നായിരിക്കും. ‎
\end{malayalam}}
\flushright{\begin{Arabic}
\quranayah[3][80]
\end{Arabic}}
\flushleft{\begin{malayalam}
നിങ്ങള്‍ മലക്കുകളെയും പ്രവാചകന്മാരെയും ‎രക്ഷകരാക്കണമെന്ന് അയാള്‍ ഒരിക്കലും നിങ്ങളോട് ‎കല്‍പിക്കുകയുമില്ല. നിങ്ങള്‍ മുസ്ലിംകളായ ശേഷം ‎സത്യനിഷേധികളാകാന്‍ ഒരു പ്രവാചകന്‍ നിങ്ങളോട് ‎കല്‍പിക്കുകയോ? ‎
\end{malayalam}}
\flushright{\begin{Arabic}
\quranayah[3][81]
\end{Arabic}}
\flushleft{\begin{malayalam}
ഓര്‍ക്കുക: അല്ലാഹു പ്രവാചകന്മാരോടിങ്ങനെ ഉറപ്പ് ‎വാങ്ങിയ സന്ദര്‍ഭം: "ഞാന്‍ നിങ്ങള്‍ക്ക് വേദപുസ്തകവും ‎തത്ത്വജ്ഞാനവും നല്‍കി. പിന്നീട് നിങ്ങളുടെ ‎വശമുള്ളതിനെ സത്യപ്പെടുത്തുന്ന ഒരു ദൈവദൂതന്‍ ‎നിങ്ങളുടെ അടുത്ത് വരികയാണെങ്കില്‍ ഉറപ്പായും ‎നിങ്ങള്‍ അദ്ദേഹത്തെ വിശ്വസിക്കുകയും ‎സഹായിക്കുകയും വേണം." അല്ലാഹു അവരോടു ‎ചോദിച്ചു: "നിങ്ങളിതംഗീകരിക്കുകയും അതനുസരിച്ച് ‎എന്നോടുള്ള കരാര്‍ ഒരു ബാധ്യതയായി ഏറ്റെടുക്കുകയും ‎ചെയ്തില്ലേ?" അവര്‍ അറിയിച്ചു: "അതെ, ‎ഞങ്ങളംഗീകരിച്ചിരിക്കുന്നു." അല്ലാഹു പറഞ്ഞു: "എങ്കില്‍ ‎നിങ്ങളതിന് സാക്ഷികളാവുക. ഞാനും നിങ്ങളോടൊപ്പം ‎സാക്ഷിയായുണ്ട്." ‎
\end{malayalam}}
\flushright{\begin{Arabic}
\quranayah[3][82]
\end{Arabic}}
\flushleft{\begin{malayalam}
അതിനുശേഷം ആരെങ്കിലും പിന്തിരിഞ്ഞുകളഞ്ഞാല്‍ ‎അവര്‍ തന്നെയാണ് കുറ്റവാളികള്‍. ‎
\end{malayalam}}
\flushright{\begin{Arabic}
\quranayah[3][83]
\end{Arabic}}
\flushleft{\begin{malayalam}
അല്ലാഹുവിന്റെ ജീവിതവ്യവസ്ഥയല്ലാത്ത ‎മറ്റുവല്ലതുമാണോ അവരാഗ്രഹിക്കുന്നത്? ‎ആകാശഭൂമികളിലുള്ളവരൊക്കെയും സ്വയം ‎സന്നദ്ധമായോ നിര്‍ബന്ധിതമായോ അവനുമാത്രം ‎കീഴ്പ്പെട്ടിരിക്കെ. എല്ലാവരുടെയും തിരിച്ചുപോക്കും ‎അവങ്കലേക്കു തന്നെ. ‎
\end{malayalam}}
\flushright{\begin{Arabic}
\quranayah[3][84]
\end{Arabic}}
\flushleft{\begin{malayalam}
പറയുക: ഞങ്ങള്‍ അല്ലാഹുവില്‍ വിശ്വസിച്ചിരിക്കുന്നു. ‎ഞങ്ങള്‍ക്ക് ഇറക്കിത്തന്നത്; ഇബ്റാഹീം, ഇസ്മാഈല്‍, ‎ഇസ്ഹാഖ്, യഅ്ഖൂബ്, യഅ്ഖൂബ്സന്തതികള്‍ ‎എന്നിവര്‍ക്ക് ഇറക്കിക്കൊടുത്തത്; മൂസാക്കും ഈസാക്കും ‎മറ്റു പ്രവാചകന്മാര്‍ക്കും തങ്ങളുടെ നാഥനില്‍നിന്ന് ‎വന്നെത്തിയത്- എല്ലാറ്റിലും ഞങ്ങള്‍ ‎വിശ്വസിച്ചിരിക്കുന്നു. അവരിലാരോടും ഞങ്ങളൊരു ‎വിവേചനവും കാണിക്കുന്നില്ല. ഞങ്ങള്‍ അല്ലാഹുവിന് ‎വഴിപ്പെട്ട മുസ്ലിംകളാണ്. ‎
\end{malayalam}}
\flushright{\begin{Arabic}
\quranayah[3][85]
\end{Arabic}}
\flushleft{\begin{malayalam}
ഇസ്ലാം അല്ലാത്ത ജീവിതമാര്‍ഗം ആരെങ്കിലും ‎ആഗ്രഹിക്കുന്നുവെങ്കില്‍, അവനില്‍നിന്നത് ‎സ്വീകരിക്കുകയില്ല. പരലോകത്തോ അവന്‍ ‎പരാജിതരിലുമായിരിക്കും. ‎
\end{malayalam}}
\flushright{\begin{Arabic}
\quranayah[3][86]
\end{Arabic}}
\flushleft{\begin{malayalam}
സത്യവിശ്വാസം സ്വീകരിച്ചശേഷം വീണ്ടും ‎സത്യനിഷേധികളായ ജനതയെ അല്ലാഹു എങ്ങനെ ‎നേര്‍വഴിയിലാക്കും? ദൈവദൂതന്‍ സത്യവാനാണെന്ന് ‎സ്വയം സാക്ഷ്യം വഹിച്ചവരാണിവര്‍. അവര്‍ക്ക് ‎വ്യക്തമായ തെളിവുകള്‍ വന്നെത്തിയിട്ടുണ്ട്. ‎അക്രമികളായ ആ ജനവിഭാഗത്തെ അല്ലാഹു ‎നേര്‍വഴിയിലാക്കുകയില്ല. ‎
\end{malayalam}}
\flushright{\begin{Arabic}
\quranayah[3][87]
\end{Arabic}}
\flushleft{\begin{malayalam}
അവര്‍ക്കുള്ള പ്രതിഫലം അല്ലാഹുവിന്റെയും ‎മലക്കുകളുടെയും മുഴുവന്‍ മനുഷ്യരുടെയും ശാപമാണ്. ‎ഉറപ്പായും അവര്‍ക്കതുണ്ടാവും. ‎
\end{malayalam}}
\flushright{\begin{Arabic}
\quranayah[3][88]
\end{Arabic}}
\flushleft{\begin{malayalam}
അവര്‍ എന്നെന്നും ശപിക്കപ്പെട്ടവരായി നിലനില്‍ക്കും. ‎ശിക്ഷയില്‍ അവര്‍ക്കൊരിളവുമില്ല. ശിക്ഷ ‎നടപ്പാക്കുന്നതില്‍ ഒട്ടും അവധി കിട്ടുകയുമില്ല. ‎
\end{malayalam}}
\flushright{\begin{Arabic}
\quranayah[3][89]
\end{Arabic}}
\flushleft{\begin{malayalam}
പിന്നീട് പശ്ചാത്തപിക്കുകയും ജീവിതം ‎നന്നാക്കിത്തീര്‍ക്കുകയും ചെയ്തവര്‍ക്കൊഴികെ. ‎അല്ലാഹു ഏറെ പൊറുക്കുന്നവനും ദയാപരനുമാണ്. ‎
\end{malayalam}}
\flushright{\begin{Arabic}
\quranayah[3][90]
\end{Arabic}}
\flushleft{\begin{malayalam}
സത്യവിശ്വാസം സ്വീകരിച്ചശേഷം സത്യനിഷേധികളായി ‎മാറുകയും തുടര്‍ന്ന് സത്യനിഷേധത്തില്‍ അടിക്കടി ‎വര്‍ധനവ് വരുത്തുകയും ചെയ്തവരുടെ പശ്ചാത്താപം ‎അല്ലാഹു ഒരിക്കലും സ്വീകരിക്കുകയില്ല. അവര്‍ ‎തന്നെയാണ് ദുര്‍മാര്‍ഗികള്‍. ‎
\end{malayalam}}
\flushright{\begin{Arabic}
\quranayah[3][91]
\end{Arabic}}
\flushleft{\begin{malayalam}
സത്യനിഷേധികളായി ജീവിക്കുകയും ‎സത്യനിഷേധികളായിത്തന്നെ മരിക്കുകയും ചെയ്തവരില്‍ ‎ആരെങ്കിലും ഭൂമി നിറയെ സ്വര്‍ണം പ്രായശ്ചിത്തമായി ‎നല്‍കിയാലും അവരില്‍നിന്നത് സ്വീകരിക്കുന്നതല്ല; ‎അവര്‍ക്ക് നോവേറിയ ശിക്ഷയുണ്ട്. അവര്‍ക്ക് ‎തുണയായി ആരുമുണ്ടാവില്ല. ‎
\end{malayalam}}
\flushright{\begin{Arabic}
\quranayah[3][92]
\end{Arabic}}
\flushleft{\begin{malayalam}
ഏറെ പ്രിയപ്പെട്ടവയില്‍ നിന്ന് ചെലവഴിക്കാതെ ‎നിങ്ങള്‍ക്ക് പുണ്യം നേടാനാവില്ല. നിങ്ങള്‍ ‎ചെലവഴിക്കുന്നതെന്തും നന്നായറിയുന്നവനാണ് അല്ലാഹു. ‎
\end{malayalam}}
\flushright{\begin{Arabic}
\quranayah[3][93]
\end{Arabic}}
\flushleft{\begin{malayalam}
എല്ലാ ആഹാരപദാര്‍ഥങ്ങളും ഇസ്രയേല്‍ മക്കള്‍ക്ക് ‎അനുവദനീയമായിരുന്നു. തൌറാത്തിന്റെ ‎അവതരണത്തിനുമുമ്പ് ഇസ്രയേല്‍ തന്റെമേല്‍ ‎നിഷിദ്ധമാക്കിയവയൊഴികെ. പറയുക: നിങ്ങള്‍ ‎തൌറാത്ത് കൊണ്ടുവന്ന് വായിച്ചു കേള്‍പ്പിക്കുക. ‎നിങ്ങള്‍ സത്യസന്ധരെങ്കില്‍. ‎
\end{malayalam}}
\flushright{\begin{Arabic}
\quranayah[3][94]
\end{Arabic}}
\flushleft{\begin{malayalam}
അതിനുശേഷവും ആരെങ്കിലും അല്ലാഹുവിന്റെ പേരില്‍ ‎കള്ളം കെട്ടിച്ചമക്കുകയാണെങ്കില്‍ അവര്‍ തന്നെയാണ് ‎അക്രമികള്‍. ‎
\end{malayalam}}
\flushright{\begin{Arabic}
\quranayah[3][95]
\end{Arabic}}
\flushleft{\begin{malayalam}
പറയുക: അല്ലാഹു അരുളിയത് സത്യം തന്നെ. അതിനാല്‍ ‎നിര്‍മല ഹൃദയനായ ഇബ്റാഹീമിന്റെ പാത നിങ്ങള്‍ ‎പിന്തുടരുക. അദ്ദേഹം ബഹുദൈവ വിശ്വാസികളില്‍ ‎പെട്ടവനായിരുന്നില്ല. ‎
\end{malayalam}}
\flushright{\begin{Arabic}
\quranayah[3][96]
\end{Arabic}}
\flushleft{\begin{malayalam}
തീര്‍ച്ചയായും മനുഷ്യര്‍ക്കായി ഉണ്ടാക്കിയ ‎ആദ്യദേവാലയം മക്കയിലേതുതന്നെ. അത് ‎അനുഗൃഹീതമാണ്. ലോകര്‍ക്കാകെ വഴികാട്ടിയും. ‎
\end{malayalam}}
\flushright{\begin{Arabic}
\quranayah[3][97]
\end{Arabic}}
\flushleft{\begin{malayalam}
അതില്‍ വ്യക്തമായ ദൃഷ്ടാന്തങ്ങളുണ്ട്. ‎ഇബ്റാഹീമിന്റെ പ്രാര്‍ഥനാസ്ഥലം; അവിടെ ‎പ്രവേശിക്കുന്നവന്‍ നിര്‍ഭയനായിരിക്കും. ആ ‎മന്ദിരത്തിലെത്തിച്ചേരാന്‍ കഴിവുള്ളവര്‍ അവിടെച്ചെന്ന് ‎ഹജ്ജ് നിര്‍വഹിക്കുകയെന്നത് മനുഷ്യര്‍ക്ക് ‎അല്ലാഹുവോടുള്ള ബാധ്യതയാണ്. ആരെങ്കിലും അതിനെ ‎ധിക്കരിക്കുന്നുവെങ്കില്‍ അറിയുക: അല്ലാഹു ‎ലോകരിലാരുടെയും ആശ്രയമാവശ്യമില്ലാത്തവനാണ്. ‎
\end{malayalam}}
\flushright{\begin{Arabic}
\quranayah[3][98]
\end{Arabic}}
\flushleft{\begin{malayalam}
ചോദിക്കുക: വേദക്കാരേ, നിങ്ങളെന്തിനാണ് ‎ദൈവത്തിന്റെ വേദവാക്യങ്ങള്‍ നിഷേധിച്ചുതള്ളുന്നത്? ‎നിങ്ങള്‍ ചെയ്യുന്നതിനെല്ലാം അല്ലാഹു സാക്ഷിയാണ്. ‎
\end{malayalam}}
\flushright{\begin{Arabic}
\quranayah[3][99]
\end{Arabic}}
\flushleft{\begin{malayalam}
പറയുക: വേദക്കാരേ, നിങ്ങളെന്തിനാണ് വിശ്വസിച്ചവരെ ‎ദൈവ മാര്‍ഗത്തില്‍നിന്ന് തടയുന്നത്? അതാണ് ‎നേര്‍വഴിയെന്ന് നിങ്ങള്‍ തന്നെ ‎സാക്ഷ്യപ്പെടുത്തിയിരിക്കെ നിങ്ങളെന്തിനത് ‎വികലമാക്കാന്‍ ശ്രമിക്കുന്നു? നിങ്ങള്‍ ‎ചെയ്തുകൊണ്ടിരിക്കുന്നതിനെപ്പറ്റിയൊന്നും അല്ലാഹു ‎തീരെ അശ്രദ്ധനല്ല. ‎
\end{malayalam}}
\flushright{\begin{Arabic}
\quranayah[3][100]
\end{Arabic}}
\flushleft{\begin{malayalam}
വിശ്വസിച്ചവരേ, വേദം കിട്ടിയവരിലൊരു ‎വിഭാഗത്തിന്റെ വാദം നിങ്ങള്‍ സ്വീകരിച്ചാല്‍, നിങ്ങള്‍ ‎സത്യവിശ്വാസികളായ ശേഷം നിങ്ങളെ വീണ്ടുമവര്‍ ‎അവിശ്വാസികളാക്കിമാറ്റും. ‎
\end{malayalam}}
\flushright{\begin{Arabic}
\quranayah[3][101]
\end{Arabic}}
\flushleft{\begin{malayalam}
നിങ്ങളെ ദൈവവചനങ്ങള്‍ ‎ഓതിക്കേള്‍പ്പിച്ചുകൊണ്ടിരിക്കെ, നിങ്ങളെങ്ങനെ ‎അവിശ്വാസികളാകും? നിങ്ങള്‍ക്കിടയില്‍ ‎ദൈവദൂതനുണ്ട്താനും. ആര്‍ അല്ലാഹുവെ ‎മുറുകെപ്പിടിക്കുന്നുവോ, അവന്‍ ഉറപ്പായും ‎നേര്‍വഴിയില്‍ നയിക്കപ്പെടും. ‎
\end{malayalam}}
\flushright{\begin{Arabic}
\quranayah[3][102]
\end{Arabic}}
\flushleft{\begin{malayalam}
വിശ്വസിച്ചവരേ, നിങ്ങള്‍ അല്ലാഹുവോട് ശരിയാംവിധം ‎ഭക്തി പുലര്‍ത്തുക. നിങ്ങള്‍ മുസ്ലിംകളായല്ലാതെ ‎മരിക്കരുത്. ‎
\end{malayalam}}
\flushright{\begin{Arabic}
\quranayah[3][103]
\end{Arabic}}
\flushleft{\begin{malayalam}
നിങ്ങളൊന്നായി അല്ലാഹുവിന്റെ പാശം ‎മുറുകെപ്പിടിക്കുക. ഭിന്നിക്കരുത്. അല്ലാഹു നിങ്ങള്‍ക്കു ‎നല്‍കിയ അനുഗ്രഹങ്ങളോര്‍ക്കുക: നിങ്ങള്‍ അന്യോന്യം ‎ശത്രുക്കളായിരുന്നു. പിന്നെ അവന്‍ നിങ്ങളുടെ ‎മനസ്സുകളെ പരസ്പരം കൂട്ടിയിണക്കി. അങ്ങനെ ‎അവന്റെ അനുഗ്രഹത്താല്‍ നിങ്ങള്‍ ‎സഹോദരങ്ങളായിത്തീര്‍ന്നു. നിങ്ങള്‍ തീക്കുണ്ഡത്തിന്റെ ‎വക്കിലായിരുന്നു. അതില്‍നിന്ന് അവന്‍ നിങ്ങളെ ‎രക്ഷിച്ചു. ഇവ്വിധം അല്ലാഹു അവന്റെ ദൃഷ്ടാന്തങ്ങള്‍ ‎നിങ്ങള്‍ക്ക് വിവരിച്ചുതരുന്നു; നിങ്ങള്‍ സന്മാര്‍ഗം ‎പ്രാപിച്ചവരാകാന്‍. ‎
\end{malayalam}}
\flushright{\begin{Arabic}
\quranayah[3][104]
\end{Arabic}}
\flushleft{\begin{malayalam}
നിങ്ങള്‍ നല്ലതിലേക്ക് ക്ഷണിക്കുകയും നന്മ ‎കല്‍പിക്കുകയും തിന്മ തടയുകയും ചെയ്യുന്ന ഒരു ‎സമുദായമായിത്തീരണം. അവര്‍ തന്നെയാണ് വിജയികള്‍. ‎
\end{malayalam}}
\flushright{\begin{Arabic}
\quranayah[3][105]
\end{Arabic}}
\flushleft{\begin{malayalam}
വ്യക്തമായ തെളിവുകള്‍ വന്നെത്തിയശേഷം ഭിന്നിച്ച് പല ‎കക്ഷികളായിപ്പിരിഞ്ഞവരെപ്പോലെ നിങ്ങളാവരുത്. ‎അവര്‍ക്ക് കൊടിയ ശിക്ഷയുണ്ട്. ‎
\end{malayalam}}
\flushright{\begin{Arabic}
\quranayah[3][106]
\end{Arabic}}
\flushleft{\begin{malayalam}
ചില മുഖങ്ങള്‍ പ്രസന്നമാവുകയും മറ്റുചില മുഖങ്ങള്‍ ‎ഇരുളുകയും ചെയ്യുന്ന ദിനമാണതുണ്ടാവുക. അന്ന് ‎മുഖം ഇരുണ്ടവരോട് ഇങ്ങനെ പറയും: "സത്യവിശ്വാസം ‎സ്വീകരിച്ചശേഷം സത്യനിഷേധികളാവുകയല്ലേ നിങ്ങള്‍ ‎ചെയ്തത്? അവ്വിധം സത്യനിഷേധികളായതിനാല്‍ ‎നിങ്ങളിന്ന് കൊടിയ ശിക്ഷ അനുഭവിച്ചുകൊള്ളുക." ‎
\end{malayalam}}
\flushright{\begin{Arabic}
\quranayah[3][107]
\end{Arabic}}
\flushleft{\begin{malayalam}
എന്നാല്‍ പ്രസന്നമായ മുഖമുള്ളവര്‍ അന്ന് ‎അല്ലാഹുവിന്റെ അനുഗ്രഹത്തിലായിരിക്കും. ‎അവരെന്നെന്നും അതേ അവസ്ഥയിലാണുണ്ടാവുക. ‎
\end{malayalam}}
\flushright{\begin{Arabic}
\quranayah[3][108]
\end{Arabic}}
\flushleft{\begin{malayalam}
ഇതൊക്കെയും അല്ലാഹുവിന്റെ വചനങ്ങളാണ്. നാമവ ‎നിനക്ക് യഥാവിധി ഓതിക്കേള്‍പ്പിക്കുന്നു. ‎ലോകജനതയോട് ഒരനീതിയും കാണിക്കാന്‍ അല്ലാഹു ‎ഉദ്ദേശിക്കുന്നില്ല. ‎
\end{malayalam}}
\flushright{\begin{Arabic}
\quranayah[3][109]
\end{Arabic}}
\flushleft{\begin{malayalam}
ആകാശഭൂമികളിലുള്ളതൊക്കെയും ‎അല്ലാഹുവിന്റേതാണ്. എല്ലാം ഒടുവില്‍ ‎മടങ്ങിയെത്തുന്നതും അവങ്കലേക്കു തന്നെ. ‎
\end{malayalam}}
\flushright{\begin{Arabic}
\quranayah[3][110]
\end{Arabic}}
\flushleft{\begin{malayalam}
മനുഷ്യസമൂഹത്തിനായി ഉയിരെടുത്ത ഉത്തമ ‎സമുദായമായിത്തീര്‍ന്നിരിക്കുന്നു നിങ്ങള്‍. നിങ്ങള്‍ നന്മ ‎കല്‍പിക്കുന്നു. തിന്മ തടയുന്നു. അല്ലാഹുവില്‍ ‎വിശ്വസിക്കുന്നു. ഇവ്വിധം വേദക്കാര്‍ ‎വിശ്വസിച്ചിരുന്നെങ്കില്‍ അവര്‍ക്കതെത്ര നന്നായേനെ! ‎അവരുടെ കൂട്ടത്തില്‍ വിശ്വാസികളുണ്ട്. എന്നാല്‍ ‎ഏറെപേരും കുറ്റവാളികളാണ്. ‎
\end{malayalam}}
\flushright{\begin{Arabic}
\quranayah[3][111]
\end{Arabic}}
\flushleft{\begin{malayalam}
നേരിയ ചില ശല്യമല്ലാതെ നിങ്ങള്‍ക്കൊരു ദ്രോഹവും ‎വരുത്താനവര്‍ക്കാവില്ല. അഥവാ, അവര്‍ നിങ്ങളോട് ‎യുദ്ധത്തിലേര്‍പ്പെടുകയാണെങ്കില്‍ ഉറപ്പായും അവര്‍ ‎പിന്തിരിഞ്ഞോടും. പിന്നെ അവര്‍ക്ക് എവിടെനിന്നും ഒരു ‎സഹായവും കിട്ടുകയില്ല. ‎
\end{malayalam}}
\flushright{\begin{Arabic}
\quranayah[3][112]
\end{Arabic}}
\flushleft{\begin{malayalam}
അല്ലാഹുവില്‍ നിന്നോ ജനങ്ങളില്‍ നിന്നോ എന്തെങ്കിലും ‎അവലംബം കിട്ടുന്നതൊഴികെ, അവര്‍ ‎എവിടെയായിരുന്നാലും അപമാനം അവരില്‍ ‎വന്നുപതിച്ചിരിക്കുന്നു. അവര്‍ അല്ലാഹുവിന്റെ ‎കോപത്തിനിരയാവുകയും അവര്‍ക്കുമേല്‍ ഹീനത്വം ‎വന്നുവീഴുകയും ചെയ്തിരിക്കുന്നു. അവര്‍ ദൈവിക ‎ദൃഷ്ടാന്തങ്ങളെ തള്ളിക്കളഞ്ഞതിനാലും അന്യായമായി ‎പ്രവാചകന്മാരെ കൊന്നുകൊണ്ടിരുന്നതിനാലുമാണിത്. ‎അവരുടെ ധിക്കാരത്തിന്റെയും അതിക്രമത്തിന്റെയും ‎ഫലവും. ‎
\end{malayalam}}
\flushright{\begin{Arabic}
\quranayah[3][113]
\end{Arabic}}
\flushleft{\begin{malayalam}
അവരെല്ലാം ഒരുപോലെയല്ല. വേദക്കാരില്‍ ‎നേര്‍വഴിയില്‍ നിലകൊള്ളുന്ന ഒരു വിഭാഗമുണ്ട്. ‎അവര്‍ രാത്രി വേളകളില്‍ സാഷ്ടാംഗം പ്രണമിച്ച് ‎അല്ലാഹുവിന്റെ വചനങ്ങള്‍ പാരായണം ചെയ്യുന്നു. ‎
\end{malayalam}}
\flushright{\begin{Arabic}
\quranayah[3][114]
\end{Arabic}}
\flushleft{\begin{malayalam}
അവര്‍ അല്ലാഹുവിലും അന്ത്യദിനത്തിലും ‎വിശ്വസിക്കുന്നു. നന്മ കല്‍പിക്കുന്നു. തിന്മ തടയുന്നു. ‎നല്ല കാര്യങ്ങളില്‍ ഉത്സുകരാകുന്നു. അവര്‍ സജ്ജനങ്ങളില്‍ ‎പെട്ടവരാണ്. ‎
\end{malayalam}}
\flushright{\begin{Arabic}
\quranayah[3][115]
\end{Arabic}}
\flushleft{\begin{malayalam}
അവരെന്തു നന്മ ചെയ്താലും അതിന്റെ ഫലം അവര്‍ക്കു ‎ലഭിക്കാതിരിക്കില്ല. അല്ലാഹു യഥാര്‍ഥ ഭക്തന്മാരെ ‎തിരിച്ചറിയുന്നവനാകുന്നു. ‎
\end{malayalam}}
\flushright{\begin{Arabic}
\quranayah[3][116]
\end{Arabic}}
\flushleft{\begin{malayalam}
എന്നാല്‍ സത്യനിഷേധികളോ, അവരുടെ സമ്പത്തും ‎സന്താനങ്ങളും അല്ലാഹുവിന്റെ ശിക്ഷയില്‍നിന്ന് അവരെ ‎തീരെ രക്ഷിക്കുകയില്ല. അവര്‍ നരകാവകാശികളാണ്. ‎അവരവിടെ നിത്യവാസികളായിരിക്കും. ‎
\end{malayalam}}
\flushright{\begin{Arabic}
\quranayah[3][117]
\end{Arabic}}
\flushleft{\begin{malayalam}
ഐഹികജീവിതത്തില്‍ അവര്‍ ചെലവഴിക്കുന്നതിന്റെ ‎ഉപമ കൊടുംതണുപ്പുള്ള ഒരു ശീതക്കാറ്റിന്റെതാണ്. അത് ‎സ്വന്തത്തോട് അതിക്രമം കാണിച്ച ഒരു ‎ജനവിഭാഗത്തിന്റെ കൃഷിയിടത്തെ ബാധിച്ചു. ‎അങ്ങനെയത് ആ കൃഷിയെ നിശ്ശേഷം നശിപ്പിച്ചു. ‎അല്ലാഹു അവരോട് ദ്രോഹമൊന്നും ചെയ്തിട്ടില്ല. അവര്‍ ‎തങ്ങളെത്തന്നെ ദ്രോഹിക്കുകയായിരുന്നു. ‎
\end{malayalam}}
\flushright{\begin{Arabic}
\quranayah[3][118]
\end{Arabic}}
\flushleft{\begin{malayalam}
വിശ്വസിച്ചവരേ, നിങ്ങളില്‍ പെട്ടവരെയല്ലാതെ നിങ്ങള്‍ ‎ഉള്ളുകള്ളികളറിയുന്നവരാക്കരുത്. നിങ്ങള്‍ക്ക് ‎വിപത്തുവരുത്തുന്നതില്‍ അവരൊരു വീഴ്ചയും ‎വരുത്തുകയില്ല. നിങ്ങള്‍ പ്രയാസപ്പെടുന്നതാണ് ‎അവര്‍ക്കിഷ്ടം. നിങ്ങളോടുള്ള വെറുപ്പ് അവരുടെ ‎വാക്കുകളിലൂടെതന്നെ വെളിവായിട്ടുണ്ട്. അവരുടെ ‎നെഞ്ചകം ഒളിപ്പിച്ചുവെക്കുന്നത് കൂടുതല്‍ ഭീകരമത്രെ. ‎നിങ്ങള്‍ക്കിതാ നാം തെളിവുകള്‍ നിരത്തിത്തന്നിരിക്കുന്നു; ‎നിങ്ങള്‍ ആലോചിച്ചറിയുന്നവരെങ്കില്‍. ‎
\end{malayalam}}
\flushright{\begin{Arabic}
\quranayah[3][119]
\end{Arabic}}
\flushleft{\begin{malayalam}
നോക്കൂ, നിങ്ങളുടെ സ്ഥിതി: നിങ്ങളവരെ സ്നേഹിക്കുന്നു. ‎അവരോ നിങ്ങളെ സ്നേഹിക്കുന്നുമില്ല. നിങ്ങള്‍ എല്ലാ ‎വേദങ്ങളിലും വിശ്വസിക്കുന്നു. നിങ്ങളെ ‎കണ്ടുമുട്ടുമ്പോള്‍ അവര്‍ പറയും: "ഞങ്ങളും ‎വിശ്വസിച്ചിരിക്കുന്നു." നിങ്ങളില്‍നിന്ന് ‎പിരിഞ്ഞുപോയാലോ നിങ്ങളോടുള്ള വെറുപ്പുകാരണം ‎അവര്‍ വിരല്‍ കടിക്കുന്നു. പറയുക: നിങ്ങള്‍ നിങ്ങളുടെ ‎വെറുപ്പുമായി മരിച്ചുകൊള്ളുക. ‎മനസ്സുകളിലുള്ളതൊക്കെയും അല്ലാഹു ‎നന്നായറിയുന്നുണ്ട്. ‎
\end{malayalam}}
\flushright{\begin{Arabic}
\quranayah[3][120]
\end{Arabic}}
\flushleft{\begin{malayalam}
നിങ്ങള്‍ക്ക് എന്തെങ്കിലും നന്മയുണ്ടാവുന്നത് അവര്‍ക്ക് ‎മനഃപ്രയാസമുണ്ടാക്കും. നിങ്ങള്‍ക്ക് വല്ല വിപത്തും ‎ബാധിക്കുന്നതോ, അതവരെ സന്തോഷിപ്പിക്കും. നിങ്ങള്‍ ‎ക്ഷമിക്കുകയും സൂക്ഷ്മത പാലിക്കുകയുമാണെങ്കില്‍ ‎അവരുടെ കുതന്ത്രം നിങ്ങള്‍ക്കൊരു വിപത്തും ‎വരുത്തുകയില്ല. അവര്‍ പ്രവര്‍ത്തിക്കുന്നതൊക്കെയും ‎സൂക്ഷ്മമായി അറിയുന്നവനാണ് അല്ലാഹു. ‎
\end{malayalam}}
\flushright{\begin{Arabic}
\quranayah[3][121]
\end{Arabic}}
\flushleft{\begin{malayalam}
സത്യവിശ്വാസികള്‍ക്ക് യുദ്ധത്തിന് ‎താവളമൊരുക്കാനായി നീ നിന്റെ കുടുംബത്തില്‍നിന്ന് ‎പുലര്‍ച്ചെ ഇറങ്ങിത്തിരിച്ച കാര്യം ഓര്‍ക്കുക. അല്ലാഹു ‎എല്ലാം കേള്‍ക്കുന്നവനും അറിയുന്നവനുമാണ്. ‎
\end{malayalam}}
\flushright{\begin{Arabic}
\quranayah[3][122]
\end{Arabic}}
\flushleft{\begin{malayalam}
ഓര്‍ക്കുക: നിങ്ങളിലെ രണ്ടു വിഭാഗം; ആ ‎ഇരുകൂട്ടരുടെയും രക്ഷാധികാരി അല്ലാഹുവാണ്. ‎എന്നിട്ടും അവര്‍ ഭയന്നോടാന്‍ ഭാവിച്ച സന്ദര്‍ഭം. ‎സത്യവിശ്വാസികള്‍ അല്ലാഹുവില്‍ ഭരമേല്‍പിക്കട്ടെ. ‎
\end{malayalam}}
\flushright{\begin{Arabic}
\quranayah[3][123]
\end{Arabic}}
\flushleft{\begin{malayalam}
നിങ്ങള്‍ നന്നെ ദുര്‍ബലരായിരിക്കെ ബദ്റി ല്‍ അല്ലാഹു ‎നിങ്ങളെ സഹായിച്ചിട്ടുണ്ട്. അതിനാല്‍ അല്ലാഹുവോട് ‎ഭക്തിയുള്ളവരാവുക. നിങ്ങള്‍ നന്ദിയുള്ളവരാകാന്‍. ‎
\end{malayalam}}
\flushright{\begin{Arabic}
\quranayah[3][124]
\end{Arabic}}
\flushleft{\begin{malayalam}
നീ സത്യവിശ്വാസികളോടു പറഞ്ഞ സന്ദര്‍ഭം: "നിങ്ങളുടെ ‎നാഥന്‍ മുവ്വായിരം മലക്കുകളെ ഇറക്കി നിങ്ങളെ ‎സഹായിക്കുന്നത് നിങ്ങള്‍ക്ക് മതിയാവില്ലേ?" ‎
\end{malayalam}}
\flushright{\begin{Arabic}
\quranayah[3][125]
\end{Arabic}}
\flushleft{\begin{malayalam}
സംശയം വേണ്ടാ, നിങ്ങള്‍ ക്ഷമയവലംബിക്കുകയും ‎സൂക്ഷ്മത പാലിക്കുകയുമാണെങ്കില്‍ ശത്രുക്കള്‍ ഈ ‎നിമിഷം തന്നെ നിങ്ങളുടെ അടുത്തുവന്നെത്തിയാലും ‎നിങ്ങളുടെ നാഥന്‍, തിരിച്ചറിയാന്‍ കഴിയുന്ന അയ്യായിരം ‎മലക്കുകളാല്‍ നിങ്ങളെ സഹായിക്കും. ‎
\end{malayalam}}
\flushright{\begin{Arabic}
\quranayah[3][126]
\end{Arabic}}
\flushleft{\begin{malayalam}
അല്ലാഹു ഇവ്വിധം അറിയിച്ചത് നിങ്ങള്‍ക്കൊരു ‎ശുഭവാര്‍ത്തയായാണ്; നിങ്ങളുടെ മനസ്സുകള്‍ ‎ശാന്തമാകാനും. യഥാര്‍ഥ സഹായം പ്രതാപിയും ‎യുക്തിമാനുമായ അല്ലാഹുവില്‍ നിന്നു മാത്രമേ ‎ലഭിക്കുകയുള്ളൂ. ‎
\end{malayalam}}
\flushright{\begin{Arabic}
\quranayah[3][127]
\end{Arabic}}
\flushleft{\begin{malayalam}
സത്യനിഷേധികളില്‍ നിന്നൊരു സംഘത്തെ ഉന്മൂലനം ‎ചെയ്യുകയോ ആശയറ്റ് പിന്തിരിയാന്‍ സ്വയം ‎പ്രേരിതരാകുമാറ് അവരെ ഒതുക്കുകയോ ചെയ്യാനാണത്. ‎
\end{malayalam}}
\flushright{\begin{Arabic}
\quranayah[3][128]
\end{Arabic}}
\flushleft{\begin{malayalam}
തീരുമാനമെടുക്കുന്നതില്‍ നിനക്കൊ രു പങ്കുമില്ല. ‎അല്ലാഹു ഒരുപക്ഷേ, അവരുടെ പശ്ചാത്താപം ‎സ്വീകരിച്ചേക്കാം. അല്ലെങ്കില്‍ അവരെ ശിക്ഷിച്ചേക്കാം. ‎തീര്‍ച്ചയായും അവര്‍ അക്രമികള്‍ തന്നെയാണ്. ‎
\end{malayalam}}
\flushright{\begin{Arabic}
\quranayah[3][129]
\end{Arabic}}
\flushleft{\begin{malayalam}
ആകാശഭൂമികളിലുള്ളതെല്ലാം അല്ലാഹുവിന്റേതാണ്. ‎അവനിച്ഛിക്കുന്നവര്‍ക്ക് അവന്‍ പൊറുത്തുകൊടുക്കുന്നു. ‎അവനിച്ഛിക്കുന്നവരെ ശിക്ഷിക്കുന്നു. അല്ലാഹു ഏറെ ‎പൊറുക്കുന്നവനാണ്; പരമ ദയാലുവും. ‎
\end{malayalam}}
\flushright{\begin{Arabic}
\quranayah[3][130]
\end{Arabic}}
\flushleft{\begin{malayalam}
വിശ്വസിച്ചവരേ, നിങ്ങള്‍ കൊള്ളപ്പലിശ തിന്നാതിരിക്കുക. ‎അല്ലാഹുവോട് ഭക്തിയുള്ളവരാവുക. നിങ്ങള്‍ ‎വിജയിച്ചേക്കാം. ‎
\end{malayalam}}
\flushright{\begin{Arabic}
\quranayah[3][131]
\end{Arabic}}
\flushleft{\begin{malayalam}
സത്യനിഷേധികള്‍ക്കായി ഒരുക്കിയ നരകത്തീയിനെ ‎സൂക്ഷിക്കുക. ‎
\end{malayalam}}
\flushright{\begin{Arabic}
\quranayah[3][132]
\end{Arabic}}
\flushleft{\begin{malayalam}
നിങ്ങള്‍ അല്ലാഹുവെയും അവന്റെ ദൂതനെയും ‎അനുസരിക്കുക. നിങ്ങള്‍ക്ക് ദിവ്യകാരുണ്യം ‎കിട്ടിയേക്കാം. ‎
\end{malayalam}}
\flushright{\begin{Arabic}
\quranayah[3][133]
\end{Arabic}}
\flushleft{\begin{malayalam}
നിങ്ങളുടെ നാഥനില്‍ നിന്നുള്ള പാപമോചനവും ‎ആകാശഭൂമികളോളം വിശാലമായ സ്വര്‍ഗവും ‎നേടാനായി നിങ്ങള്‍ ധൃതിയില്‍ മുന്നോട്ടുവരിക. ‎ഭക്തന്മാര്‍ക്കായി തയ്യാറാക്കിയതാണത്. ‎
\end{malayalam}}
\flushright{\begin{Arabic}
\quranayah[3][134]
\end{Arabic}}
\flushleft{\begin{malayalam}
ധന്യതയിലും ദാരിദ്യ്രത്തിലും ധനം ‎ചെലവഴിക്കുന്നവരും കോപം ‎കടിച്ചിറക്കുന്നവരുമാണവര്‍; ജനങ്ങളോട് വിട്ടുവീഴ്ച ‎കാണിക്കുന്നവരും. സല്‍ക്കര്‍മികളെ അല്ലാഹു ‎സ്നേഹിക്കുന്നു. ‎
\end{malayalam}}
\flushright{\begin{Arabic}
\quranayah[3][135]
\end{Arabic}}
\flushleft{\begin{malayalam}
വല്ല നീചകൃത്യവും ചെയ്യുകയോ, തങ്ങളോടുതന്നെ ‎എന്തെങ്കിലും അതിക്രമം കാണിക്കുകയോ ചെയ്താല്‍ ‎അപ്പോള്‍തന്നെ അല്ലാഹുവെ ഓര്‍ക്കുന്നവരാണവര്‍; ‎തങ്ങളുടെ പാപങ്ങള്‍ക്ക് മാപ്പിരക്കുന്നവരും. പാപങ്ങള്‍ ‎പൊറുക്കാന്‍ അല്ലാഹുവല്ലാതെ ആരുണ്ട്? ‎അവരൊരിക്കലും തങ്ങള്‍ ചെയ്തുപോയ തെറ്റുകളില്‍ ‎ബോധപൂര്‍വം ഉറച്ചുനില്‍ക്കുകയില്ല. ‎
\end{malayalam}}
\flushright{\begin{Arabic}
\quranayah[3][136]
\end{Arabic}}
\flushleft{\begin{malayalam}
അവര്‍ക്കുള്ള പ്രതിഫലം, തങ്ങളുടെ നാഥനില്‍ നിന്നുള്ള ‎പാപമോചനവും താഴ്ഭാഗത്തൂടെ അരുവികളൊഴുകുന്ന ‎സ്വര്‍ഗീയാരാമങ്ങളുമാണ്. അവരതില്‍ ‎സ്ഥിരവാസികളായിരിക്കും. സല്‍ക്കര്‍മികള്‍ക്കുള്ള ‎പ്രതിഫലം എത്ര അനുഗൃഹീതം. ‎
\end{malayalam}}
\flushright{\begin{Arabic}
\quranayah[3][137]
\end{Arabic}}
\flushleft{\begin{malayalam}
നിങ്ങള്‍ക്കുമുമ്പ് എന്തെല്ലാം നടപടിക്രമങ്ങള്‍ ഇവിടെ ‎കഴിഞ്ഞുപോയിട്ടുണ്ട്! അതിനാല്‍ നിങ്ങള്‍ ഭൂമിയില്‍ ‎സഞ്ചരിച്ച് സത്യനിഷേധികളുടെ ഒടുക്കം ‎എവ്വിധമായിരുന്നുവെന്ന് നോക്കിക്കാണുക. ‎
\end{malayalam}}
\flushright{\begin{Arabic}
\quranayah[3][138]
\end{Arabic}}
\flushleft{\begin{malayalam}
ഇത് സകല ജനത്തോടുമുള്ള വിളംബരമാണ്. ‎ദൈവഭക്തര്‍ക്കുള്ള മാര്‍ഗദര്‍ശനവും സദുപദേശവും. ‎
\end{malayalam}}
\flushright{\begin{Arabic}
\quranayah[3][139]
\end{Arabic}}
\flushleft{\begin{malayalam}
നിങ്ങള്‍ ദുര്‍ബലരോ ദുഃഖിതരോ ആവരുത്. നിങ്ങള്‍ ‎തന്നെയാണ് അത്യുന്നതര്‍; നിങ്ങള്‍ ‎സത്യവിശ്വാസികളെങ്കില്‍! ‎
\end{malayalam}}
\flushright{\begin{Arabic}
\quranayah[3][140]
\end{Arabic}}
\flushleft{\begin{malayalam}
നിങ്ങള്‍ക്കിപ്പോള്‍ ക്ഷതം പറ്റിയിട്ടുണ്ടെങ്കില്‍ മുമ്പ് ‎അവര്‍ക്കും ക്ഷതമേറ്റിട്ടുണ്ട്. ആ ദിനങ്ങള്‍ ‎ജനങ്ങള്‍ക്കിടയില്‍ നാം മാറ്റിമറിച്ചുകൊണ്ടിരിക്കും. ‎അല്ലാഹുവിന് സത്യവിശ്വാസികളെ ‎വേര്‍തിരിച്ചെടുക്കാനാണത്. നിങ്ങളില്‍നിന്ന് ‎രക്തസാക്ഷികളെ സൃഷ്ടിക്കാനും. അല്ലാഹു അക്രമികളെ ‎ഇഷ്ടപ്പെടുകയില്ല. ‎
\end{malayalam}}
\flushright{\begin{Arabic}
\quranayah[3][141]
\end{Arabic}}
\flushleft{\begin{malayalam}
അല്ലാഹുവിന് സത്യവിശ്വാസികളെ ‎കറകളഞ്ഞെടുക്കാനാണിത്. സത്യനിഷേധികളെ ‎തകര്‍ക്കാനും. ‎
\end{malayalam}}
\flushright{\begin{Arabic}
\quranayah[3][142]
\end{Arabic}}
\flushleft{\begin{malayalam}
അല്ല; നിങ്ങള്‍ വെറുതെയങ്ങ് സ്വര്‍ഗത്തില്‍ ‎കടന്നുകളയാമെന്ന് കരുതുന്നുണ്ടോ, നിങ്ങളില്‍നിന്ന് ‎ദൈവമാര്‍ഗത്തില്‍ സമരം നടത്തുന്നവരെയും ‎ക്ഷമയവലംബിക്കുന്നവരെയും തിരിച്ചറിഞ്ഞിട്ടല്ലാതെ? ‎
\end{malayalam}}
\flushright{\begin{Arabic}
\quranayah[3][143]
\end{Arabic}}
\flushleft{\begin{malayalam}
മരണത്തെ നേരില്‍ കാണുംമുമ്പെ നിങ്ങളത് ‎കൊതിക്കുന്നവരായിരുന്നു. എന്നാല്‍ ഇപ്പോള്‍ നിങ്ങള്‍ ‎നോക്കിനില്‍ക്കെ തന്നെ നിങ്ങളതിനെ നേരില്‍ ‎കണ്ടിരിക്കുന്നു. ‎
\end{malayalam}}
\flushright{\begin{Arabic}
\quranayah[3][144]
\end{Arabic}}
\flushleft{\begin{malayalam}
മുഹമ്മദ് ദൈവദൂതനല്ലാതാരുമല്ല. അദ്ദേഹത്തിനുമുമ്പും ‎ദൈവദൂതന്മാര്‍ കഴിഞ്ഞുപോയിട്ടുണ്ട്. അദ്ദേഹം ‎മരണപ്പെടുകയോ കൊല്ലപ്പെടുകയോ ചെയ്താല്‍ നിങ്ങള്‍ ‎പിന്‍തിരിഞ്ഞുപോവുകയോ? ആരെങ്കിലും ‎പിന്‍തിരിഞ്ഞുപോകുന്നുവെങ്കില്‍ അറിയുക: അവന്‍ ‎അല്ലാഹുവിന് ഒരു ദ്രോഹവും വരുത്തുകയില്ല. ‎അതോടൊപ്പം, നന്ദി കാണിക്കുന്നവര്‍ക്ക് അല്ലാഹു ‎അര്‍ഹമായ പ്രതിഫലം നല്‍കും. ‎
\end{malayalam}}
\flushright{\begin{Arabic}
\quranayah[3][145]
\end{Arabic}}
\flushleft{\begin{malayalam}
ദൈവഹിതമനുസരിച്ചല്ലാതെ ആര്‍ക്കും മരിക്കാനാവില്ല. ‎മരണസമയം സുനിശ്ചിതമാണ്. ആരെങ്കിലും ‎ഇഹലോകത്തിലെ പ്രതിഫലമാണ് ആഗ്രഹിക്കുന്നതെങ്കില്‍ ‎നാമവനത് നല്‍കും. ആരെങ്കിലും പരലോകത്തെ ‎പ്രതിഫലമാണ് കൊതിക്കുന്നതെങ്കില്‍ നാമവന് അതും ‎കൊടുക്കും. നന്ദി കാണിക്കുന്നവര്‍ക്ക് നാം നല്ല പ്രതിഫലം ‎നല്‍കും. ‎
\end{malayalam}}
\flushright{\begin{Arabic}
\quranayah[3][146]
\end{Arabic}}
\flushleft{\begin{malayalam}
എത്രയോ പ്രവാചകന്മാരുണ്ടായിട്ടുണ്ട്. ‎അവരോടൊപ്പം നിരവധി ദൈവഭക്തന്മാര്‍ ‎പോരാടിയിട്ടുമുണ്ട്. എന്നിട്ട് അല്ലാഹുവിന്റെ ‎മാര്‍ഗത്തില്‍ അനുഭവിച്ച ദുരിതങ്ങള്‍കൊണ്ടൊന്നും ‎അവര്‍ തളര്‍ന്നില്ല. അവര്‍ ദുര്‍ബലരാവുകയോ ‎കീഴടങ്ങുകയോ ചെയ്തില്ല. ക്ഷമാശീലരെ അല്ലാഹു ‎സ്നേഹിക്കുന്നു. ‎
\end{malayalam}}
\flushright{\begin{Arabic}
\quranayah[3][147]
\end{Arabic}}
\flushleft{\begin{malayalam}
അവരുടെ പ്രാര്‍ഥന ഇതുമാത്രമായിരുന്നു: "ഞങ്ങളുടെ ‎നാഥാ! ഞങ്ങളുടെ പാപങ്ങളും ഞങ്ങളുടെ കാര്യത്തില്‍ ‎സംഭവിച്ചുപോയ അതിരുകവിച്ചിലുകളും ഞങ്ങള്‍ക്കു നീ ‎പൊറുത്തുതരേണമേ. ഞങ്ങളുടെ പാദങ്ങളെ ‎ഉറപ്പിച്ചുനിര്‍ത്തേണമേ. സത്യനിഷേധികളായ ‎ജനത്തിനെതിരെ ഞങ്ങളെ സഹായിക്കേണമേ!" ‎
\end{malayalam}}
\flushright{\begin{Arabic}
\quranayah[3][148]
\end{Arabic}}
\flushleft{\begin{malayalam}
അതിനാല്‍ അല്ലാഹു അവര്‍ക്ക് ഐഹികമായ പ്രതിഫലം ‎നല്‍കി; കൂടുതല്‍ മെച്ചമായ പാരത്രിക ഫലവും. ‎സല്‍ക്കര്‍മികളെ അല്ലാഹു സ്നേഹിക്കുന്നു. ‎
\end{malayalam}}
\flushright{\begin{Arabic}
\quranayah[3][149]
\end{Arabic}}
\flushleft{\begin{malayalam}
വിശ്വസിച്ചവരേ, നിങ്ങള്‍ സത്യനിഷേധികളെ അനുസരിച്ചു ‎ജീവിച്ചാല്‍ അവര്‍ നിങ്ങളെ പിറകോട്ടു ‎തിരിച്ചുകൊണ്ടുപോകും. അങ്ങനെ നിങ്ങള്‍ എല്ലാം ‎നഷ്ടപ്പെട്ടവരായിത്തീരും. ‎
\end{malayalam}}
\flushright{\begin{Arabic}
\quranayah[3][150]
\end{Arabic}}
\flushleft{\begin{malayalam}
നിങ്ങളുടെ രക്ഷകന്‍ അല്ലാഹുവാണ്. സഹായികളില്‍ ‎ഏറെ നല്ലവനും അവന്‍ തന്നെ. ‎
\end{malayalam}}
\flushright{\begin{Arabic}
\quranayah[3][151]
\end{Arabic}}
\flushleft{\begin{malayalam}
സത്യനിഷേധികളുടെ മനസ്സുകളില്‍ നാം ഭയം ‎ഇട്ടുകൊടുക്കും. അല്ലാഹുവിന്റെ ‎പങ്കാളികളാണെന്നതിന് അവനൊരു തെളിവും ‎നല്‍കിയിട്ടില്ലാത്ത വസ്തുക്കളെ അവര്‍ അവന്റെ ‎പങ്കാളികളാക്കിയതിനാലാണിത്. നാളെ അവരുടെ ‎താവളം നരകമത്രെ. അക്രമികളുടെ വാസസ്ഥലം എത്ര ‎ചീത്ത! ‎
\end{malayalam}}
\flushright{\begin{Arabic}
\quranayah[3][152]
\end{Arabic}}
\flushleft{\begin{malayalam}
അല്ലാഹു നിങ്ങളോടുള്ള അവന്റെ വാഗ്ദാനം ‎നിറവേറ്റിയിരിക്കുന്നു. ആദ്യം അവന്റെ അനുമതി ‎പ്രകാരം നിങ്ങളവരുടെ ‎കഥകഴിച്ചുകൊണ്ടിരിക്കുകയായിരുന്നു. പിന്നെ, നിങ്ങള്‍ ‎ദുര്‍ബലരാവുകയും കാര്യനിര്‍വഹണത്തിന്റെ പേരില്‍ ‎പരസ്പരം തര്‍ക്കിക്കുകയും ചെയ്തു. നിങ്ങള്‍ക്ക് ഏറെ ‎പ്രിയപ്പെട്ടത് അല്ലാഹു നിങ്ങള്‍ക്ക് കാണിച്ചുതന്നശേഷം ‎നിങ്ങള്‍ അനുസരണക്കേട് കാണിച്ചു. നിങ്ങളില്‍ ഐഹിക ‎താത്പര്യങ്ങളുള്ളവരുണ്ട്. പരലോകം ‎കൊതിക്കുന്നവരുമുണ്ട്. പിന്നീട് അല്ലാഹു നിങ്ങളെ ‎അവരില്‍നിന്ന് പിന്‍തിരിപ്പിച്ചു; നിങ്ങളെ പരീക്ഷിക്കാന്‍. ‎അല്ലാഹു നിങ്ങള്‍ക്ക് മാപ്പേകിയിരിക്കുന്നു. അവന്‍ ‎സത്യവിശ്വാസികളോട് അത്യുദാരന്‍ തന്നെ. ‎
\end{malayalam}}
\flushright{\begin{Arabic}
\quranayah[3][153]
\end{Arabic}}
\flushleft{\begin{malayalam}
ഓര്‍ക്കുക: ആരെയും തിരിഞ്ഞുനോക്കാതെ നിങ്ങള്‍ ‎ഓടിക്കയറുകയായിരുന്നു. ദൈവദൂതന്‍ പിന്നില്‍നിന്ന് ‎നിങ്ങളെ വിളിക്കുന്നുണ്ടായിരുന്നു. അപ്പോള്‍ അല്ലാഹു ‎നിങ്ങള്‍ക്ക് ദുഃഖത്തിനുമേല്‍ ദുഃഖം പ്രതിഫലമായി ‎നല്‍കി. നിങ്ങള്‍ക്ക് കൈവിട്ടുപോയ നേട്ടത്തിന്റെയോ ‎നിങ്ങളെ ബാധിക്കുന്ന വിപത്തിന്റെയോ പേരില്‍ നിങ്ങള്‍ ‎ദുഃഖിതരാവാതിരിക്കാനാണിത്. നിങ്ങള്‍ ‎ചെയ്യുന്നതൊക്കെയും നന്നായറിയുന്നവനാണ് അല്ലാഹു. ‎
\end{malayalam}}
\flushright{\begin{Arabic}
\quranayah[3][154]
\end{Arabic}}
\flushleft{\begin{malayalam}
പിന്നെ, ആ ദുഃഖത്തിനുശേഷം അല്ലാഹു നിങ്ങള്‍ക്ക് എല്ലാം ‎മറന്ന് മയങ്ങിയുറങ്ങാവുന്ന ശാന്തി നല്‍കി. ‎നിങ്ങളിലൊരു വിഭാഗം ആ മയക്കത്തിന്റെ ശാന്തത ‎അനുഭവിച്ചു. മറ്റൊരു വിഭാഗം സ്വന്തത്തെപ്പറ്റി മാത്രം ‎ചിന്തിച്ച് അസ്വസ്ഥരായി. അവര്‍ അല്ലാഹുവെ ‎സംബന്ധിച്ച് സത്യവിരുദ്ധമായ അനിസ്ലാമിക ‎ധാരണയാണ് വെച്ചുപുലര്‍ത്തിയിരുന്നത്. അവര്‍ ‎ചോദിക്കുന്നു: "കാര്യങ്ങള്‍ തീരുമാനിക്കുന്നതില്‍ നമുക്ക് ‎വല്ല പങ്കുമുണ്ടോ?” പറയുക: "കാര്യങ്ങളെല്ലാം ‎അല്ലാഹുവിന്റെ അധീനതയിലാണ്.” അറിയുക: അവര്‍ ‎നിന്നോട് വെളിപ്പെടുത്താത്ത ചിലത് ‎മനസ്സുകളിലൊളിപ്പിച്ചുവെക്കുന്നുണ്ട്. അവര്‍ പറയുന്നു: ‎‎"കാര്യങ്ങള്‍ തീരുമാനിക്കുന്നതില്‍ നമുക്ക് ‎പങ്കുണ്ടായിരുന്നെങ്കില്‍ നാം ഇവിടെ വെച്ച് ‎നശിക്കുമായിരുന്നില്ല.” പറയുക: "നിങ്ങള്‍ നിങ്ങളുടെ ‎വീടുകളിലായിരുന്നാല്‍ പോലും വധിക്കപ്പെടാന്‍ ‎വിധിക്കപ്പെട്ടവര്‍ തങ്ങളുടെ മരണസ്ഥലത്തേക്ക് സ്വയം ‎പുറപ്പെട്ടുവരുമായിരുന്നു. ഇപ്പോള്‍ നടന്നതെല്ലാം, ‎നിങ്ങളുടെ നെഞ്ചകത്തുള്ളതിനെ അല്ലാഹു ‎പരീക്ഷിക്കാനും നിങ്ങളുടെ മനസ്സിലുള്ളത് ‎കറകളഞ്ഞെടുക്കാനുമാണ്. നെഞ്ചകത്തുള്ളതൊക്കെയും ‎നന്നായറിയുന്നവനാണ് അല്ലാഹു.” ‎
\end{malayalam}}
\flushright{\begin{Arabic}
\quranayah[3][155]
\end{Arabic}}
\flushleft{\begin{malayalam}
രണ്ടു കൂട്ടര്‍ ഏറ്റുമുട്ടിയ ദിവസം; നിങ്ങളില്‍നിന്ന് ‎പിന്തിരിഞ്ഞുപോയവരെ തങ്ങളുടെ തന്നെ ചില ‎ചെയ്തികള്‍ കാരണം പിശാച് ‎വഴിപിഴപ്പിക്കുകയായിരുന്നു. അല്ലാഹു അവര്‍ക്ക് ‎മാപ്പേകിയിരിക്കുന്നു. തീര്‍ച്ചയായും അല്ലാഹു ഏറെ ‎പൊറുക്കുന്നവനും ക്ഷമിക്കുന്നവനും തന്നെ. ‎
\end{malayalam}}
\flushright{\begin{Arabic}
\quranayah[3][156]
\end{Arabic}}
\flushleft{\begin{malayalam}
വിശ്വസിച്ചവരേ, നിങ്ങള്‍ ‎സത്യനിഷേധികളെപ്പോലെയാവരുത്. തങ്ങളുടെ ‎സഹോദരങ്ങള്‍ കച്ചവടത്തിന് ഭൂമിയില്‍ ‎സഞ്ചരിക്കുകയോ യുദ്ധത്തിനു പുറപ്പെടുകയോ ചെയ്ത് ‎മരണമടഞ്ഞാല്‍ അവര്‍ പറയും: "ഇവര്‍ ഞങ്ങളുടെ ‎അടുത്തായിരുന്നെങ്കില്‍ മരിക്കുകയോ ‎വധിക്കപ്പെടുകയോ ഇല്ലായിരുന്നു.” ഇതൊക്കെയും ‎അല്ലാഹു അവരുടെ മനസ്സുകളില്‍ ഖേദത്തിന് ‎കാരണമാക്കിവെക്കുന്നു. ജീവിപ്പിക്കുന്നതും ‎മരിപ്പിക്കുന്നതും അല്ലാഹുവാണ്. നിങ്ങള്‍ ‎ചെയ്യുന്നതൊക്കെയും കണ്ടറിയുന്നവനാണ് അല്ലാഹു. ‎
\end{malayalam}}
\flushright{\begin{Arabic}
\quranayah[3][157]
\end{Arabic}}
\flushleft{\begin{malayalam}
നിങ്ങള്‍ അല്ലാഹുവിന്റെ മാര്‍ഗത്തില്‍ ‎വധിക്കപ്പെടുകയോ മരിക്കുകയോ ആണെങ്കില്‍ ‎ഉറപ്പായും അതിലൂടെ നിങ്ങള്‍ക്ക് അല്ലാഹുവില്‍നിന്നുള്ള ‎പാപമോചനവും അനുഗ്രഹവും ലഭിക്കും. അവര്‍ ‎ഒരുക്കൂട്ടിവെക്കുന്നതിനെക്കാളൊക്കെ മഹത്തരം ‎അതാണ്. ‎
\end{malayalam}}
\flushright{\begin{Arabic}
\quranayah[3][158]
\end{Arabic}}
\flushleft{\begin{malayalam}
നിങ്ങള്‍ മരിക്കട്ടെ, അല്ലെങ്കില്‍ വധിക്കപ്പെടട്ടെ; ‎രണ്ടായാലും എല്ലാവരെയും ഒടുവില്‍ ഒരുമിച്ചുകൂട്ടുക ‎അല്ലാഹുവിങ്കലാണ്. ‎
\end{malayalam}}
\flushright{\begin{Arabic}
\quranayah[3][159]
\end{Arabic}}
\flushleft{\begin{malayalam}
അല്ലാഹുവിന്റെ അനുഗ്രഹം കൊണ്ടാണ് നീ അവരോട് ‎സൌമ്യനായത്. നീ പരുഷപ്രകൃതനും ‎കഠിനമനസ്കനുമായിരുന്നെങ്കില്‍ നിന്റെ ചുറ്റുനിന്നും ‎അവരൊക്കെയും പിരിഞ്ഞുപോകുമായിരുന്നു. ‎അതിനാല്‍ നീ അവര്‍ക്ക് മാപ്പേകുക. അവരുടെ ‎പാപമോചനത്തിനായി പ്രാര്‍ഥിക്കുക. കാര്യങ്ങള്‍ ‎അവരുമായി കൂടിയാലോചിക്കുക. അങ്ങനെ നീ ‎തീരുമാനമെടുത്താല്‍ അല്ലാഹുവില്‍ ഭരമേല്‍പിക്കുക. ‎തീര്‍ച്ചയായും അല്ലാഹു തന്നില്‍ ഭരമേല്‍പിക്കുന്നവരെ ‎ഇഷ്ടപ്പെടുന്നു. ‎
\end{malayalam}}
\flushright{\begin{Arabic}
\quranayah[3][160]
\end{Arabic}}
\flushleft{\begin{malayalam}
അല്ലാഹു നിങ്ങളെ തുണക്കുന്നുവെങ്കില്‍ പിന്നെ നിങ്ങളെ ‎തോല്‍പിക്കാനാര്‍ക്കും കഴിയില്ല. അവന്‍ നിങ്ങളെ ‎കൈവെടിയുന്നുവെങ്കില്‍ പിന്നെ നിങ്ങളെ സഹായിക്കാന്‍ ‎അവനെക്കൂടാതെ ആരാണുള്ളത്? അതിനാല്‍ ‎സത്യവിശ്വാസികള്‍ അവനില്‍ ഭരമേല്‍പിക്കട്ടെ. ‎
\end{malayalam}}
\flushright{\begin{Arabic}
\quranayah[3][161]
\end{Arabic}}
\flushleft{\begin{malayalam}
വഞ്ചന നടത്തുകയെന്നത് ഒരു ‎പ്രവാചകനില്‍നിന്നുമുണ്ടാവില്ല. ആരെങ്കിലും വല്ലതും ‎വഞ്ചിച്ചെടുത്താല്‍ ഉയിര്‍ത്തെഴുന്നേല്‍പുനാളില്‍ ‎അയാള്‍ തന്റെ ചതിക്കെട്ടുമായാണ് ‎ദൈവസന്നിധിയിലെത്തുക. പിന്നീട് എല്ലാ ‎ഓരോരുത്തര്‍ക്കും താന്‍ നേടിയതിന്റെ ഫലം ‎പൂര്‍ണമായി നല്‍കും. ആരോടും ഒരനീതിയും ‎കാണിക്കുകയില്ല. ‎
\end{malayalam}}
\flushright{\begin{Arabic}
\quranayah[3][162]
\end{Arabic}}
\flushleft{\begin{malayalam}
അല്ലാഹുവിന്റെ പ്രീതി പിന്തുടര്‍ന്നവന്‍ ‎ദൈവകോപത്തിലാണ്ടുപോയവനെപ്പോലെയാണോ? ‎അവന്റെ താവളം നരകമാണ്. അതെത്ര ചീത്ത സങ്കേതം! ‎
\end{malayalam}}
\flushright{\begin{Arabic}
\quranayah[3][163]
\end{Arabic}}
\flushleft{\begin{malayalam}
അവര്‍ അല്ലാഹുവിന്റെ അടുത്ത് പല പദവികളിലാണ്. ‎അവര്‍ ചെയ്യുന്നതൊക്കെ കണ്ടറിയുന്നവനാണ് അല്ലാഹു. ‎
\end{malayalam}}
\flushright{\begin{Arabic}
\quranayah[3][164]
\end{Arabic}}
\flushleft{\begin{malayalam}
തങ്ങളില്‍നിന്നുതന്നെയുള്ള ഒരു ദൂതനെ ‎നിയോഗിച്ചതിലൂടെ സത്യവിശ്വാസികളെ അല്ലാഹു ‎അതിയായി അനുഗ്രഹിച്ചിരിക്കുന്നു. അദ്ദേഹം അവരെ ‎അല്ലാഹുവിന്റെ വചനങ്ങള്‍ ഓതിക്കേള്‍പ്പിക്കുന്നു. ‎അവരെ സംസ്കരിച്ചെടുക്കുന്നു. വേദപുസ്തകവും ‎തത്ത്വജ്ഞാനവും പഠിപ്പിക്കുന്നു. അവരോ, അതിനു മുമ്പ് ‎വ്യക്തമായ വഴികേടിലായിരുന്നു. ‎
\end{malayalam}}
\flushright{\begin{Arabic}
\quranayah[3][165]
\end{Arabic}}
\flushleft{\begin{malayalam}
നിങ്ങളെ ഒരു വിപത്ത് ബാധിച്ചപ്പോഴേക്കും “ഇതെങ്ങനെ ‎സംഭവിച്ചു”വെന്ന് നിങ്ങള്‍ ചോദിക്കുന്നു. എന്നാല്‍ ‎ഇതിന്റെ ഇരട്ടി നിങ്ങള്‍ ശത്രുക്കേല്‍പിച്ചിട്ടുണ്ട്. ‎പറയുക: "ഇത് നിങ്ങളുടെ ഭാഗത്തുനിന്നുതന്നെ ‎സംഭവിച്ചതാണ്. തീര്‍ച്ചയായും അല്ലാഹു എല്ലാ ‎കാര്യത്തിനും കഴിവുറ്റവനാണ്.” ‎
\end{malayalam}}
\flushright{\begin{Arabic}
\quranayah[3][166]
\end{Arabic}}
\flushleft{\begin{malayalam}
രണ്ടു വിഭാഗം ഏറ്റുമുട്ടിയ ദിവസം നിങ്ങളെ ബാധിച്ച ‎വിപത്ത് ദൈവഹിതമനുസരിച്ചു തന്നെയാണ്. നിങ്ങളിലെ ‎യഥാര്‍ഥ വിശ്വാസികളാരെന്ന് വേര്‍തിരിച്ചറിയാന്‍ ‎വേണ്ടിയാണത്. ‎
\end{malayalam}}
\flushright{\begin{Arabic}
\quranayah[3][167]
\end{Arabic}}
\flushleft{\begin{malayalam}
കപടവിശ്വാസികളാരെന്ന് വ്യക്തമാകാനും. "നിങ്ങള്‍ ‎വരൂ! അല്ലാഹുവിന്റെ മാര്‍ഗത്തില്‍ യുദ്ധം ചെയ്യൂ; ‎അല്ലെങ്കില്‍ ചെറുത്തുനില്‍ക്കുകയെങ്കിലും ചെയ്യൂ” എന്ന് ‎കല്‍പിച്ചപ്പോള്‍ അവര്‍ പറഞ്ഞു: "യുദ്ധമുണ്ടാകുമെന്ന് ‎അറിഞ്ഞിരുന്നെങ്കില്‍ ഞങ്ങളും നിങ്ങളെ ‎പിന്തുടരുമായിരുന്നു.” അന്ന് അവര്‍ക്ക് ‎സത്യവിശ്വാസത്തേക്കാള്‍ അടുപ്പം ‎സത്യനിഷേധത്തോടായിരുന്നു. അവരുടെ ‎മനസ്സിലില്ലാത്തതാണ് നാവുകൊണ്ടവര്‍ പറയുന്നത്. ‎അവര്‍ മറച്ചുവെക്കുന്നതൊക്കെയും ‎നന്നായറിയുന്നവനാണ് അല്ലാഹു. ‎
\end{malayalam}}
\flushright{\begin{Arabic}
\quranayah[3][168]
\end{Arabic}}
\flushleft{\begin{malayalam}
യുദ്ധത്തിനുപോകാതെ വീട്ടിലിരുന്നവരാണവര്‍; എന്നിട്ട് ‎തങ്ങളുടെ സഹോദരങ്ങളെ പ്പറ്റി “ഞങ്ങള്‍ ‎പറഞ്ഞതനുസരിച്ചിരുന്നെങ്കില്‍ ‎വധിക്കപ്പെടുമായിരുന്നില്ല” എന്നു പറഞ്ഞവരും. ‎പറയുക: "എങ്കില്‍ നിങ്ങള്‍ നിങ്ങളില്‍നിന്ന് മരണത്തെ ‎തട്ടിമാറ്റുക; നിങ്ങള്‍ സത്യസന്ധരെങ്കില്‍!” ‎
\end{malayalam}}
\flushright{\begin{Arabic}
\quranayah[3][169]
\end{Arabic}}
\flushleft{\begin{malayalam}
അല്ലാഹുവിന്റെ മാര്‍ഗത്തില്‍ വധിക്കപ്പെട്ടവര്‍ ‎മരിച്ചുപോയവരാണെന്ന് കരുതരുത്. സത്യത്തിലവര്‍ ‎തങ്ങളുടെ നാഥന്റെ അടുക്കല്‍ ജീവിച്ചിരിക്കുന്നവരാണ്. ‎അവര്‍ക്ക് ജീവിത വിഭവം നിര്‍ലോഭം ‎ലഭിച്ചുകൊണ്ടിരിക്കും. ‎
\end{malayalam}}
\flushright{\begin{Arabic}
\quranayah[3][170]
\end{Arabic}}
\flushleft{\begin{malayalam}
അല്ലാഹു തന്റെ അനുഗ്രഹത്തില്‍ നിന്ന് തങ്ങള്‍ക്കേകിയ ‎തില്‍ അവര്‍ സന്തുഷ്ടരാണ്. തങ്ങളുടെ പിന്നിലുള്ളവരും ‎തങ്ങളോടൊപ്പം വന്നെത്തിയിട്ടില്ലാത്തവരുമായ ‎വിശ്വാസികളുടെ കാര്യത്തിലുമവര്‍ സംതൃപ്തരാണ്. ‎അവര്‍ക്ക് ഒന്നും പേടിക്കാനോ ദുഃഖിക്കാനോ ഇല്ലെന്ന് ‎അവരറിയുന്നതിനാലാണിത്. ‎
\end{malayalam}}
\flushright{\begin{Arabic}
\quranayah[3][171]
\end{Arabic}}
\flushleft{\begin{malayalam}
അല്ലാഹുവിന്റെ അനുഗ്രഹവും ഔദാര്യവും കാരണം ‎അവര്‍ ആഹ്ളാദഭരിതരാണ്. സത്യവിശ്വാസികള്‍ക്കുള്ള ‎പ്രതിഫലം അല്ലാഹു തീരേ പാഴാക്കുകയില്ല; തീര്‍ച്ച. ‎
\end{malayalam}}
\flushright{\begin{Arabic}
\quranayah[3][172]
\end{Arabic}}
\flushleft{\begin{malayalam}
പോരാട്ടത്തില്‍ പരിക്കുപറ്റിയ ശേഷവും ‎അല്ലാഹുവിന്റെയും അവന്റെ ദൂതന്റെയും ‎വിളിക്ക്ഉത്തരം നല്‍കിയവരുണ്ട്. അവരിലെ, ‎സല്‍ക്കര്‍മങ്ങള്‍ പ്രവര്‍ത്തിക്കുകയും സൂക്ഷ്മത ‎പാലിക്കുകയും ചെയ്തവര്‍ക്ക് അതിമഹത്തായ ‎പ്രതിഫലമുണ്ട്. ‎
\end{malayalam}}
\flushright{\begin{Arabic}
\quranayah[3][173]
\end{Arabic}}
\flushleft{\begin{malayalam}
‎"നിങ്ങള്‍ക്കെതിരെ ജനം സംഘടിച്ചിരിക്കുന്നു. അതിനാല്‍ ‎നിങ്ങളവരെ പേടിക്കണം” എന്ന് ജനങ്ങള്‍ അവരോടു ‎പറഞ്ഞപ്പോള്‍ അതവരുടെ വിശ്വാസം ‎വര്‍ധിപ്പിക്കുകയാണുണ്ടായത്. അവര്‍ പറഞ്ഞു: ‎‎"ഞങ്ങള്‍ക്ക് അല്ലാഹു മതി. ഭരമേേല്‍പിക്കാന്‍ ഏറ്റം ‎പറ്റിയവന്‍ അവനാണ്.” ‎
\end{malayalam}}
\flushright{\begin{Arabic}
\quranayah[3][174]
\end{Arabic}}
\flushleft{\begin{malayalam}
അല്ലാഹുവിന്റെ അനുഗ്രഹത്താലും ഔദാര്യത്താലും ‎ബുദ്ധിമുട്ടൊന്നുമുണ്ടാവാതെ അവര്‍ മടങ്ങി. ‎അല്ലാഹുവിന്റെ പ്രീതിയെ അനുധാവനം ചെയ്തു ‎മുന്നേറി. അതിമഹത്തായ ഔദാര്യത്തിനുടമയാണ് ‎അല്ലാഹു. ‎
\end{malayalam}}
\flushright{\begin{Arabic}
\quranayah[3][175]
\end{Arabic}}
\flushleft{\begin{malayalam}
അത് പിശാചു തന്നെ. അവന്‍ തന്റെ മിത്രങ്ങളെ ‎ക്കാണിച്ച് നിങ്ങളെ ഭയപ്പെടുത്തുകയാണ്. അതിനാല്‍ ‎നിങ്ങളവരെ പേടിക്കരുത്. എന്നെ മാത്രം ഭയപ്പെടുക; ‎നിങ്ങള്‍ സത്യവിശ്വാസികളെങ്കില്‍! ‎
\end{malayalam}}
\flushright{\begin{Arabic}
\quranayah[3][176]
\end{Arabic}}
\flushleft{\begin{malayalam}
സത്യനിഷേധത്തില്‍ ധൃതിയില്‍ മുന്നേറുന്നവര്‍ നിന്നെ ‎ദുഃഖിപ്പിക്കാതിരിക്കട്ടെ. അല്ലാഹുവിന് ഒരുപദ്രവവും ‎വരുത്താന്‍ അവര്‍ക്കാവില്ല. പരലോകത്ത് അവര്‍ക്കൊരു ‎വിഹിതവും നല്‍കാതിരിക്കാന്‍ അല്ലാഹു ഉദ്ദേശിക്കുന്നു. ‎കൊടിയ ശിക്ഷയാണ് അവര്‍ക്കുണ്ടാവുക. ‎
\end{malayalam}}
\flushright{\begin{Arabic}
\quranayah[3][177]
\end{Arabic}}
\flushleft{\begin{malayalam}
സത്യവിശ്വാസം വിറ്റ് പകരം സത്യനിഷേധം വാങ്ങിയവര്‍ ‎അല്ലാഹുവിന് ഒരു ദോഷവും വരുത്തുന്നില്ല. അവര്‍ക്ക് ‎നോവുറ്റ ശിക്ഷയുണ്ട്. ‎
\end{malayalam}}
\flushright{\begin{Arabic}
\quranayah[3][178]
\end{Arabic}}
\flushleft{\begin{malayalam}
സത്യനിഷേധികള്‍ക്ക് നാം സമയം നീട്ടിക്കൊടുക്കുന്നത് ‎തങ്ങള്‍ക്ക് ഗുണകരമാണെന്ന് അവരൊരിക്കലും ‎കരുതേണ്ടതില്ല. അവര്‍ തങ്ങളുടെ കുറ്റം ‎പെരുപ്പിച്ചുകൊണ്ടിരിക്കാന്‍ മാത്രമാണ് നാമവര്‍ക്ക് ‎സമയം നീട്ടിക്കൊടുക്കുന്നത്. ഏറ്റം നിന്ദ്യമായ ‎ശിക്ഷയാണ് അവര്‍ക്കുണ്ടാവുക. ‎
\end{malayalam}}
\flushright{\begin{Arabic}
\quranayah[3][179]
\end{Arabic}}
\flushleft{\begin{malayalam}
സത്യവിശ്വാസികളെ അവര്‍ ഇന്നുള്ള അവസ്ഥയില്‍ ‎നിലകൊള്ളാന്‍ അല്ലാഹു അനുവദിക്കുകയില്ല; ‎നല്ലതില്‍നിന്ന് തിയ്യതിനെ വേര്‍തിരിച്ചെടുക്കാതെ. ‎അഭൌതിക കാര്യങ്ങള്‍ അല്ലാഹു നിങ്ങള്‍ക്ക് ‎വെളിപ്പെടുത്തിത്തരില്ല. എന്നാല്‍ അല്ലാഹു അവന്റെ ‎ദൂതന്മാരില്‍നിന്ന് അവനിച്ഛിക്കുന്നവരെ ‎തെരഞ്ഞെടുക്കുന്നു. അതിനാല്‍ നിങ്ങള്‍ അല്ലാഹുവിലും ‎അവന്റെ ദൂതന്മാരിലും വിശ്വസിക്കുക. ‎വിശ്വസിക്കുകയും സൂക്ഷ്മത പാലിക്കുകയുമാണെങ്കില്‍ ‎നിങ്ങള്‍ക്ക് മഹത്തായ പ്രതിഫലമുണ്ട്. ‎
\end{malayalam}}
\flushright{\begin{Arabic}
\quranayah[3][180]
\end{Arabic}}
\flushleft{\begin{malayalam}
അല്ലാഹു തന്റെ അനുഗ്രഹമായി നല്‍കിയ സമ്പത്തില്‍ ‎പിശുക്കുകാണിക്കുന്നവര്‍ തങ്ങള്‍ക്കത് ഗുണമാണെന്ന് ‎ഒരിക്കലും കരുതരുത്. അതവര്‍ക്ക് ഹാനികരമാണ്. ‎ഉയിര്‍ത്തെഴുന്നേല്‍പുനാളില്‍ അവര്‍ പിശുക്കു ‎കാണിച്ചുണ്ടാക്കിയ ധനത്താല്‍ അവരുടെ കണ്ഠങ്ങളില്‍ ‎വളയമണിയിക്കപ്പെടും. ആകാശഭൂമികളുടെ അന്തിമമായ ‎അവകാശം അല്ലാഹുവിനാണ്. നിങ്ങള്‍ ചെയ്യുന്നതെല്ലാം ‎നന്നായറിയുന്നവനാണവന്‍. ‎
\end{malayalam}}
\flushright{\begin{Arabic}
\quranayah[3][181]
\end{Arabic}}
\flushleft{\begin{malayalam}
അല്ലാഹു ദരിദ്രനും തങ്ങള്‍ ധനികരുമാണെന്ന് ‎പറഞ്ഞവരുടെ വാക്ക് അല്ലാഹു കേട്ടിരിക്കുന്നു. അവര്‍ ‎അങ്ങനെ പറഞ്ഞതും അന്യായമായി പ്രവാചകന്മാരെ ‎കൊന്നതും നാം രേഖപ്പെടുത്തിവെക്കുന്നുണ്ട്. ‎നാമവരോട് പറയും: "കത്തിയെരിയും നരകത്തീ ‎അനുഭവിച്ചുകൊള്ളുക. ‎
\end{malayalam}}
\flushright{\begin{Arabic}
\quranayah[3][182]
\end{Arabic}}
\flushleft{\begin{malayalam}
‎"ഇത് നിങ്ങളുടെ കൈകള്‍ നേരത്തെ ചെയ്തുവെച്ചതാണ്. ‎തീര്‍ച്ചയായും അല്ലാഹു തന്റെ അടിമകളോട് അനീതി ‎കാണിക്കുന്നവനല്ലല്ലോ.” ‎
\end{malayalam}}
\flushright{\begin{Arabic}
\quranayah[3][183]
\end{Arabic}}
\flushleft{\begin{malayalam}
ഞങ്ങളുടെ മുന്നില്‍വച്ച് ഒരു ബലിനടത്തി അതിനെ തീ ‎വന്നു തിന്നുംവരെ ഒരു ദൈവദൂതനിലും ‎വിശ്വസിക്കേണ്ടതില്ലെന്ന് അല്ലാഹു ഞങ്ങളോട് കരാര്‍ ‎ചെയ്തിരിക്കുന്നുവെന്ന് വാദിക്കുന്നവരോട് പറയുക: ‎വ്യക്തമായ തെളിവുകളോടെയും ‎നിങ്ങളിപ്പറഞ്ഞതൊക്കെ ചെയ്തുകാണിച്ചും ‎ദൈവദൂതന്മാര്‍ നിങ്ങളുടെ അടുത്ത് വന്നിരുന്നുവല്ലോ. ‎എന്നിട്ടും നിങ്ങളവരെ കൊന്നതെന്തിന്? നിങ്ങള്‍ ‎സത്യവാദികളെങ്കില്‍! ‎
\end{malayalam}}
\flushright{\begin{Arabic}
\quranayah[3][184]
\end{Arabic}}
\flushleft{\begin{malayalam}
അതിനാല്‍ നിന്നെ അവര്‍ തള്ളിപ്പറയുന്നുവെങ്കില്‍ ‎നിനക്കുമുമ്പും നിരവധി ദൈവദൂതന്മാരെ അവര്‍ ‎തള്ളിപ്പറഞ്ഞിട്ടുണ്ട്. അവരൊക്കെയും വ്യക്തമായ ‎തെളിവുകളും ഏടുകളും പ്രകാശം പരത്തുന്ന ‎വേദപുസ്തകവുമായി വന്നവരായിരുന്നു. ‎
\end{malayalam}}
\flushright{\begin{Arabic}
\quranayah[3][185]
\end{Arabic}}
\flushleft{\begin{malayalam}
എല്ലാ മനുഷ്യരും മരണം രുചിക്കുന്നവരാണ്. നിങ്ങളുടെ ‎കര്‍മഫലമെല്ലാം ഉയിര്‍ത്തെഴുന്നേല്‍പുനാളില്‍ ‎പൂര്‍ണമായും നിങ്ങള്‍ക്കു നല്‍കും. അപ്പോള്‍ ‎നരകത്തീയില്‍ നിന്നകറ്റപ്പെടുകയും സ്വര്‍ഗത്തിലേക്ക് ‎നയിക്കപ്പെടുകയും ചെയ്യുന്നവനാണ് വിജയംവരിച്ചവന്‍. ‎ഐഹികജീവിതം ചതിക്കുന്ന ചരക്കല്ലാതൊന്നുമല്ല. ‎
\end{malayalam}}
\flushright{\begin{Arabic}
\quranayah[3][186]
\end{Arabic}}
\flushleft{\begin{malayalam}
തീര്‍ച്ചയായും നിങ്ങളുടെ സമ്പത്തിലും ശരീരത്തിലും ‎നിങ്ങള്‍ പരീക്ഷണ വിധേയരാകും. നിങ്ങള്‍ക്കുമുമ്പെ ‎വേദം ലഭിച്ചവരില്‍ നിന്നും ബഹുദൈവ വിശ്വാസികളില്‍ ‎നിന്നും നിങ്ങള്‍ ധാരാളം ചീത്തവാക്കുകള്‍ ‎കേള്‍ക്കേണ്ടിവരും. അപ്പോഴൊക്കെ നിങ്ങള്‍ ‎ക്ഷമപാലിക്കുകയും സൂക്ഷ്മത ‎പുലര്‍ത്തുകയുമാണെങ്കില്‍ തീര്‍ച്ചയായും അത് ‎നിശ്ചയദാര്‍ഢ്യമുള്ള കാര്യം തന്നെ. ‎
\end{malayalam}}
\flushright{\begin{Arabic}
\quranayah[3][187]
\end{Arabic}}
\flushleft{\begin{malayalam}
ഓര്‍ക്കുക: വേദം കിട്ടിയവരോട് അവരത് ജനങ്ങള്‍ക്ക് ‎വിവരിച്ചുകൊടുക്കണമെന്നും അത് ‎ഒളിപ്പിച്ചുവെക്കരുതെന്നും അല്ലാഹു ഉറപ്പ് ‎വാങ്ങിയിരുന്നു. എന്നിട്ടും അവരത് തങ്ങളുടെ ‎പിറകിലേക്ക് വലിച്ചെറിഞ്ഞു. നിസ്സാരമായ വിലയ്ക്ക് ‎അത് വില്‍ക്കുകയും ചെയ്തു. അവര്‍ പകരം വാങ്ങുന്നത് ‎വളരെ ചീത്തതന്നെ. ‎
\end{malayalam}}
\flushright{\begin{Arabic}
\quranayah[3][188]
\end{Arabic}}
\flushleft{\begin{malayalam}
സ്വന്തം ചെയ്തികളില്‍ ഊറ്റം കൊള്ളുകയും, ചെയ്യാത്ത ‎കാര്യങ്ങളുടെ പേരില്‍ പ്രശംസ ആഗ്രഹിക്കുകയും ‎ചെയ്യുന്നവര്‍ ശിക്ഷയില്‍നിന്നൊഴിവാകുമെന്ന് നീ ‎കരുതരുത്. അവര്‍ക്കാണ് നോവേറിയ ശിക്ഷയുള്ളത്. ‎
\end{malayalam}}
\flushright{\begin{Arabic}
\quranayah[3][189]
\end{Arabic}}
\flushleft{\begin{malayalam}
ആകാശഭൂമികളുടെ ആധിപത്യം അല്ലാഹുവിനാണ്. ‎അല്ലാഹു എല്ലാ കാര്യത്തിനും കഴിവുറ്റവന്‍ തന്നെ. ‎
\end{malayalam}}
\flushright{\begin{Arabic}
\quranayah[3][190]
\end{Arabic}}
\flushleft{\begin{malayalam}
ആകാശഭൂമികളുടെ സൃഷ്ടിയിലും രാപ്പകലുകള്‍ ‎മാറിമാറി വരുന്നതിലും ചിന്താശേഷിയുള്ളവര്‍ക്ക് ‎ധാരാളം ദൃഷ്ടാന്തങ്ങളുണ്ട്. ‎
\end{malayalam}}
\flushright{\begin{Arabic}
\quranayah[3][191]
\end{Arabic}}
\flushleft{\begin{malayalam}
നിന്നും ഇരുന്നും കിടന്നും അല്ലാഹുവെ ‎സ്മരിക്കുന്നവരാണവര്‍; ആകാശഭൂമികളുടെ ‎സൃഷ്ടിയെപ്പറ്റി ചിന്തിക്കുന്നവരും. അവര്‍ സ്വയം പറയും: ‎‎"ഞങ്ങളുടെ നാഥാ! നീ ഇതൊന്നും വെറുതെ സൃഷ്ടിച്ചതല്ല. ‎നീയെത്ര പരിശുദ്ധന്‍! അതിനാല്‍ നീ ഞങ്ങളെ ‎നരകത്തീയില്‍നിന്ന് കാത്തുരക്ഷിക്കേണമേ. ‎
\end{malayalam}}
\flushright{\begin{Arabic}
\quranayah[3][192]
\end{Arabic}}
\flushleft{\begin{malayalam}
‎"ഞങ്ങളുടെ നാഥാ, നീ ആരെയെങ്കിലും ‎നരകത്തിലേക്കയച്ചാല്‍ അവനെ നീ നിന്ദിച്ചതു തന്നെ. ‎അതിക്രമികള്‍ക്ക് തുണയായി ആരുമുണ്ടാവുകയില്ല. ‎
\end{malayalam}}
\flushright{\begin{Arabic}
\quranayah[3][193]
\end{Arabic}}
\flushleft{\begin{malayalam}
‎"ഞങ്ങളുടെ നാഥാ! സത്യവിശ്വാസത്തിലേക്കു ക്ഷണിക്കുന്ന ‎ഒരു വിളിയാളന്‍ “നിങ്ങള്‍ നിങ്ങളുടെ നാഥനില്‍ ‎വിശ്വസിക്കുവിന്‍” എന്നു വിളംബരംചെയ്യുന്നത്് ‎ഞങ്ങള്‍ കേട്ടു. അങ്ങനെ ഞങ്ങള്‍ വിശ്വസിച്ചു. ഞങ്ങളുടെ ‎നാഥാ! അതിനാല്‍ ഞങ്ങളുടെ പാപങ്ങള്‍ നീ ‎പൊറുത്തുതരേണമേ. ഞങ്ങളുടെ തിന്മകളെ ‎മായ്ച്ചുകളയുകയും സല്‍ക്കര്‍മികളായി ഞങ്ങളെ നീ ‎മരിപ്പിക്കുകയും ചെയ്യേണമേ! ‎
\end{malayalam}}
\flushright{\begin{Arabic}
\quranayah[3][194]
\end{Arabic}}
\flushleft{\begin{malayalam}
‎"ഞങ്ങളുടെ നാഥാ; നിന്റെ ദൂതന്മാരിലൂടെ നീ ഞങ്ങള്‍ക്ക് ‎വാഗ്ദാനം ചെയ്തതൊക്കെയും ഞങ്ങള്‍ക്കു നല്‍കേണമേ. ‎ഉയിര്‍ത്തെഴുന്നേല്‍പുനാളില്‍ ഞങ്ങളെ നീ നിന്ദിക്കരുതേ. ‎നിശ്ചയമായും നീ വാഗ്ദാനംലംഘിക്കുകയില്ല.” ‎
\end{malayalam}}
\flushright{\begin{Arabic}
\quranayah[3][195]
\end{Arabic}}
\flushleft{\begin{malayalam}
അപ്പോള്‍ അവരുടെ നാഥന്‍ അവര്‍ക്കുത്തരമേകി: ‎‎"പുരുഷനായാലും സ്ത്രീയായാലും നിങ്ങളിലാരുടെയും ‎പ്രവര്‍ത്തനത്തെ ഞാന്‍ പാഴാക്കുകയില്ല. നിങ്ങളിലൊരു ‎വിഭാഗം മറുവിഭാഗത്തില്‍ നിന്നുണ്ടായവരാണ്. ‎അതിനാല്‍ അല്ലാഹുവിന്റെ മാര്‍ഗത്തില്‍ തങ്ങളുടെ നാട് ‎വെടിഞ്ഞവര്‍; സ്വന്തം വീടുകളില്‍നിന്ന് ‎പുറന്തള്ളപ്പെട്ടവര്‍; ദൈവമാര്‍ഗത്തില്‍ ‎പീഡിപ്പിക്കപ്പെട്ടവര്‍; യുദ്ധത്തിലേര്‍പ്പെടുകയും ‎വധിക്കപ്പെടുകയും ചെയ്തവര്‍- എല്ലാവരുടെയും ‎തിന്മകളെ നാം മായ്ച്ചില്ലാതാക്കും; തീര്‍ച്ച. ‎താഴ്ഭാഗത്തൂടെ ആറുകളൊഴുകുന്ന ‎സ്വര്‍ഗീയാരാമങ്ങളില്‍ നാമവരെ പ്രവേശിപ്പിക്കും. ‎ഇതൊക്കെയും അല്ലാഹുവിങ്കല്‍ നിന്നുള്ള പ്രതിഫലമാണ്. ‎അല്ലാഹുവിന്റെയടുത്ത് മാത്രമാണ് ഉത്തമമായ ‎പ്രതിഫലമുള്ളത്.” ‎
\end{malayalam}}
\flushright{\begin{Arabic}
\quranayah[3][196]
\end{Arabic}}
\flushleft{\begin{malayalam}
നാടെങ്ങുമുള്ള സത്യനിഷേധികളുടെ വിളയാട്ടം നിന്നെ ‎വഞ്ചിക്കാതിരിക്കട്ടെ. ‎
\end{malayalam}}
\flushright{\begin{Arabic}
\quranayah[3][197]
\end{Arabic}}
\flushleft{\begin{malayalam}
അത് നന്നെ തുച്ഛമായ സുഖോത്സവം മാത്രം. പിന്നെ ‎അവര്‍ ചെന്നെത്തുന്ന താവളം നരകമാണ്. അതെത്ര ചീത്ത ‎സങ്കേതം! ‎
\end{malayalam}}
\flushright{\begin{Arabic}
\quranayah[3][198]
\end{Arabic}}
\flushleft{\begin{malayalam}
എന്നാല്‍ തങ്ങളുടെ നാഥനോട് ഭക്തിപുലര്‍ത്തിയവര്‍ക്ക് ‎താഴ്ഭാഗത്തൂടെ ആറുകളൊഴുകുന്ന ‎സ്വര്‍ഗീയാരാമങ്ങളുണ്ട്. അവരവിടെ ‎സ്ഥിരവാസികളായിരിക്കും. അല്ലാഹുവിങ്കല്‍നിന്നുള്ള ‎സല്‍ക്കാരമാണത്. അല്ലാഹുവിങ്കലുള്ളതാണ് ‎സജ്ജനങ്ങള്‍ക്ക് ഏറ്റം ഉത്തമം. ‎
\end{malayalam}}
\flushright{\begin{Arabic}
\quranayah[3][199]
\end{Arabic}}
\flushleft{\begin{malayalam}
വേദക്കാരിലൊരു വിഭാഗമുണ്ട്. അല്ലാഹുവിലും ‎നിങ്ങള്‍ക്കവതീര്‍ണമായ വേദത്തിലും ‎അവര്‍ക്കവതീര്‍ണമായ വേദത്തിലും ‎വിശ്വസിക്കുന്നവരാണവര്‍. അല്ലാഹുവോട് ‎ഭയഭക്തിയുള്ളവരുമാണ്. നിസ്സാര വിലയ്ക്ക് അവര്‍ ‎അല്ലാഹുവിന്റെ വചനങ്ങള്‍ വില്‍ക്കുകയില്ല. അവര്‍ക്കു ‎തന്നെയാണ് അവരുടെ നാഥന്റെ അടുക്കല്‍ മഹത്തായ ‎പ്രതിഫലമുള്ളത്. തീര്‍ച്ചയായും അല്ലാഹു അതിവേഗം ‎കണക്കുനോക്കുന്നവനാണ്. ‎
\end{malayalam}}
\flushright{\begin{Arabic}
\quranayah[3][200]
\end{Arabic}}
\flushleft{\begin{malayalam}
വിശ്വസിച്ചവരേ, നിങ്ങള്‍ ക്ഷമിക്കുക. ‎അസത്യവാദികള്‍ക്കെതിരെ സ്ഥൈര്യമുള്ളവരാവുക. ‎സത്യസേവനത്തിന് സന്നദ്ധരാവുക. അല്ലാഹുവോട് ‎ഭക്തിയുള്ളവരാവുക. നിങ്ങള്‍ വിജയിച്ചേക്കാം. ‎
\end{malayalam}}
\chapter{\textmalayalam{നിസാഅ് ( സ്ത്രീകള്‍ )}}
\begin{Arabic}
\Huge{\centerline{\basmalah}}\end{Arabic}
\flushright{\begin{Arabic}
\quranayah[4][1]
\end{Arabic}}
\flushleft{\begin{malayalam}
ജനങ്ങളേ, നിങ്ങളുടെ നാഥനോട് ഭക്തിയുള്ളവരാവുക. ഒരൊറ്റ സത്തയില്നിധന്ന് നിങ്ങളെ ‎സൃഷ്ടിച്ചവനാണവന്‍. അതില്നികന്നുതന്നെ അതിന്റെ ഇണയെ സൃഷ്ടിച്ചു. അവ രണ്ടില്‍ ‎നിന്നുമായി ധാരാളം പുരുഷന്മാരെയും സ്ത്രീകളെയും അവന്‍ വ്യാപിപ്പിച്ചു. ഏതൊരു ‎അല്ലാഹുവിന്റെ പേരിലാണോ നിങ്ങള്‍ അന്യോന്യം അവകാശങ്ങള്‍ ചോദിക്കുന്നത് ‎അവനെ സൂക്ഷിക്കുക; കുടുംബബന്ധങ്ങളെയും. തീര്ച്ച്യായും അല്ലാഹു നിങ്ങളെ സദാ ‎ശ്രദ്ധിച്ചുകൊണ്ടിരിക്കുന്നവനാണ്. ‎
\end{malayalam}}
\flushright{\begin{Arabic}
\quranayah[4][2]
\end{Arabic}}
\flushleft{\begin{malayalam}
അനാഥകളുടെ സ്വത്ത് നിങ്ങള്‍അവര്‍ക്കുതന്നെ വിട്ടുകൊടുക്കുക. നല്ല സമ്പത്തിനെ ചീത്തയാക്കി മാറ്റരുത്. അവരുടെ സ്വത്തും നിങ്ങളുടെ സ്വത്തും കൂട്ടിക്കലര്‍ത്തി തിന്നരുത്. സംശയം വേണ്ട; കൊടും പാപമാണത്.
\end{malayalam}}
\flushright{\begin{Arabic}
\quranayah[4][3]
\end{Arabic}}
\flushleft{\begin{malayalam}
അനാഥകളുടെ കാര്യത്തില്‍ നീതി പാലിക്കാനാവില്ലെന്ന് നിങ്ങളാശങ്കിക്കുന്നുവെങ്കില്‍ നിങ്ങള്‍ക്കിഷ്ടപ്പെട്ട മറ്റു സ്ത്രീകളില്‍നിന്ന് രണ്ടോ മൂന്നോ നാലോ പേരെ വിവാഹം ചെയ്യുക. എന്നാല്‍ അവര്‍ക്കിടയില്‍ നീതി പാലിക്കാ നാവില്ലെന്ന് ആശങ്കിക്കുന്നുവെങ്കില്‍ ഒരൊറ്റ സ്ത്രീയെ മാത്രമേ വിവാഹം ചെയ്യാവൂ. അല്ലെങ്കില്‍ നിങ്ങളുടെ അധീനതയിലുള്ളവരെ ഭാര്യമാരാക്കുക. നിങ്ങള്‍ പരിധി ലംഘിക്കുന്നവരാവാതിരിക്കാന്‍ അതാണ് ഏറ്റം നല്ലത്.
\end{malayalam}}
\flushright{\begin{Arabic}
\quranayah[4][4]
\end{Arabic}}
\flushleft{\begin{malayalam}
സ്ത്രീകള്‍ക്ക് അവരുടെ വിവാഹമൂല്യം തികഞ്ഞ തൃപ്തിയോടെ നല്‍കുക. അതില്‍ നിന്നെന്തെങ്കിലും അവര്‍ നല്ല മനസ്സോടെ വിട്ടുതരികയാണെങ്കില്‍ നിങ്ങള്‍ക്കത് സ്വീകരിച്ചനുഭവിക്കാം.
\end{malayalam}}
\flushright{\begin{Arabic}
\quranayah[4][5]
\end{Arabic}}
\flushleft{\begin{malayalam}
അല്ലാഹു നിങ്ങളുടെ നിലനില്‍പ്പിന് ആധാരമായി നിശ്ചയിച്ച സമ്പത്ത് കാര്യവിചാരമില്ലാത്തവര്‍ക്ക് നിങ്ങള്‍ കൈവിട്ടുകൊടുക്കരുത്. എന്നാല്‍ അതില്‍നിന്ന് അവര്‍ക്ക് നിങ്ങള്‍ ഉണ്ണാനും ഉടുക്കാനും കൊടുക്കുക. അവരോട് നല്ല വാക്കു പറയുകയും ചെയ്യുക.
\end{malayalam}}
\flushright{\begin{Arabic}
\quranayah[4][6]
\end{Arabic}}
\flushleft{\begin{malayalam}
വിവാഹ പ്രായമാകുംവരെ അനാഥകളെ, അവര്‍ പക്വത പ്രാപിച്ചോ എന്ന് നിങ്ങള്‍ പരീക്ഷിച്ചുകൊണ്ടിരിക്കുക. അങ്ങനെ അവര്‍ കാര്യപ്രാപ്തി കൈവരിച്ചതായി കണ്ടാല്‍ അവരുടെ സ്വത്ത് അവര്‍ക്കു വിട്ടുകൊടുക്കുക. അവര്‍ വളര്‍ന്നുവലുതാവുകയാണല്ലോ എന്ന് കരുതി അവരുടെ ധനം ധൂര്‍ത്തടിച്ച് ധൃതിയില്‍ തിന്നുതീര്‍ക്കരുത്. സ്വത്ത് കൈകാര്യം ചെയ്യുന്നവന്‍ സമ്പന്നനാണെങ്കില്‍ അനാഥകളുടെ സ്വത്തില്‍നിന്ന് ഒന്നും എടുക്കാതെ മാന്യത കാണിക്കണം. ദരിദ്രനാണെങ്കില്‍ ന്യായമായതെടുത്ത് ആഹരിക്കാവുന്നതാണ്. സ്വത്ത് അവരെ തിരിച്ചേല്‍പിക്കുമ്പോള്‍ നിങ്ങളതിന് സാക്ഷിനിര്‍ത്തണം. കണക്കുനോക്കാന്‍ അല്ലാഹുതന്നെ മതി.
\end{malayalam}}
\flushright{\begin{Arabic}
\quranayah[4][7]
\end{Arabic}}
\flushleft{\begin{malayalam}
മാതാപിതാക്കളും ഉറ്റബന്ധുക്കളും വിട്ടേച്ചുപോയ സ്വത്തില്‍ പുരുഷന്മാര്‍ക്ക് വിഹിതമുണ്ട്. മാതാപിതാക്കളും ഉറ്റബന്ധുക്കളും വിട്ടേച്ചുപോയ സ്വത്തില്‍ സ്ത്രീകള്‍ക്കും വിഹിതമുണ്ട്. സ്വത്ത് കുറവായാലും കൂടുതലായാലും ശരി. ഈ വിഹിതം അല്ലാഹു നിശ്ചയിച്ചതാണ്.
\end{malayalam}}
\flushright{\begin{Arabic}
\quranayah[4][8]
\end{Arabic}}
\flushleft{\begin{malayalam}
ഓഹരിവെക്കുമ്പോള്‍ ബന്ധുക്കളും അനാഥരും ദരിദ്രരും അവിടെ വന്നിട്ടുണ്ടെങ്കില്‍ അതില്‍നിന്ന് അവര്‍ക്കും എന്തെങ്കിലും കൊടുക്കുക. അവരോട് നല്ല വാക്ക് പറയുകയും ചെയ്യുക.
\end{malayalam}}
\flushright{\begin{Arabic}
\quranayah[4][9]
\end{Arabic}}
\flushleft{\begin{malayalam}
തങ്ങള്‍ക്കു പിറകെ ദുര്‍ബലരായ മക്കളെ വിട്ടേച്ചുപോകുന്നവര്‍ അവരെയോര്‍ത്ത് ആശങ്കിക്കുന്നതുപോലെ മറ്റുള്ളവരുടെ കാര്യത്തിലും അവര്‍ ആശങ്കയുള്ളവരാകട്ടെ. അങ്ങനെ അവര്‍ അല്ലാഹുവെ സൂക്ഷിക്കുകയും നല്ല വാക്ക് പറയുകയും ചെയ്യട്ടെ.
\end{malayalam}}
\flushright{\begin{Arabic}
\quranayah[4][10]
\end{Arabic}}
\flushleft{\begin{malayalam}
അനാഥകളുടെ ധനം അന്യായമായി ആഹരിക്കുന്നവര്‍ അവരുടെ വയറുകളില്‍ തിന്നുനിറക്കുന്നത് തീയാണ്. സംശയം വേണ്ട; അവര്‍ നരകത്തീയില്‍ കത്തിയെരിയും.
\end{malayalam}}
\flushright{\begin{Arabic}
\quranayah[4][11]
\end{Arabic}}
\flushleft{\begin{malayalam}
നിങ്ങളുടെ മക്കളുടെ കാര്യത്തില്‍ അല്ലാഹു നിങ്ങളെ ഉപദേശിക്കുന്നു: പുരുഷന്ന് രണ്ടു സ്ത്രീയുടെ വിഹിതത്തിന് തുല്യമായതുണ്ട്. അഥവാ, രണ്ടിലേറെ പെണ്‍മക്കള്‍ മാത്രമാണുള്ളതെങ്കില്‍ മരിച്ചയാള്‍ വിട്ടേച്ചുപോയ സ്വത്തിന്റെ മൂന്നില്‍ രണ്ട് ഭാഗമാണ് അവര്‍ക്കുണ്ടാവുക. ഒരു മകള്‍ മാത്രമാണെങ്കില്‍ അവള്‍ക്ക് പാതി ലഭിക്കും. മരിച്ചയാള്‍ക്ക് മക്കളുണ്ടെങ്കില്‍ മാതാപിതാക്കളിലോരോരുത്തര്‍ക്കും അയാള്‍ വിട്ടേച്ചുപോയ സ്വത്തിന്റെ ആറിലൊന്നു വീതമാണുണ്ടാവുക. അഥവാ, അയാള്‍ക്ക് മക്കളില്ലാതെ മാതാപിതാക്കള്‍ അനന്തരാവകാശികളാവുകയാണെങ്കില്‍ മാതാവിന് മൂന്നിലൊന്നുണ്ടായിരിക്കും. അയാള്‍ക്ക് സഹോദരങ്ങളുണ്ടെങ്കില്‍ മാതാവിന് ആറിലൊന്നാണുണ്ടാവുക. ഇതെല്ലാം മരണമടഞ്ഞയാളുടെ വസ്വിയ്യത്തും കടവും കഴിച്ചുള്ളവയുടെ കാര്യത്തിലാണ്. മാതാപിതാക്കളാണോ മക്കളാണോ നിങ്ങള്‍ക്ക് കൂടുതലുപകരിക്കുകയെന്ന് നിങ്ങള്‍ക്കറിയില്ല. ഈ ഓഹരി നിര്‍ണയം അല്ലാഹുവില്‍ നിന്നുള്ളതാണ്. അല്ലാഹു എല്ലാം അറിയുന്നവനും തികഞ്ഞ യുക്തിമാനുമത്രെ.
\end{malayalam}}
\flushright{\begin{Arabic}
\quranayah[4][12]
\end{Arabic}}
\flushleft{\begin{malayalam}
നിങ്ങളുടെ ഭാര്യമാര്‍ മക്കളില്ലാതെയാണ് മരിക്കുന്നതെങ്കില്‍ അവര്‍ വിട്ടേച്ചുപോയ സ്വത്തിന്റെ പാതി നിങ്ങള്‍ക്കുള്ളതാണ്. അഥവാ, അവര്‍ക്ക് മക്കളുണ്ടെങ്കില്‍ അവര്‍ വിട്ടേച്ചുപോയതിന്റെ നാലിലൊന്നാണ് നിങ്ങള്‍ക്കുണ്ടാവുക. ഇത് അവര്‍ ചെയ്യുന്ന വസ്വിയ്യത്തും കടമുണ്ടെങ്കിലതും കഴിച്ചുള്ളതില്‍ നിന്നാണ്. നിങ്ങള്‍ക്ക് മക്കളില്ലെങ്കില്‍ നിങ്ങള്‍ വിട്ടേച്ചുപോകുന്ന സ്വത്തിന്റെ നാലിലൊന്ന് ഭാര്യമാര്‍ക്കുള്ളതാണ്. അഥവാ, നിങ്ങള്‍ക്ക് മക്കളുണ്ടെങ്കില്‍ നിങ്ങള്‍ വിട്ടേച്ചുപോയതിന്റെ എട്ടിലൊന്നാണ് അവര്‍ക്കുണ്ടാവുക. നിങ്ങള്‍ നല്‍കുന്ന വസ്വിയ്യത്തും കടമുണ്ടെങ്കിലതും കഴിച്ച ശേഷമാണിത്. അനന്തരമെടുക്കപ്പെടുന്ന പുരുഷന്നോ സ്ത്രീക്കോ പിതാവും മക്കളും മാതാപിതാക്കളൊത്ത സഹോദരങ്ങളും ഇല്ലാതിരിക്കുകയും മാതാവൊത്ത സഹോദരനോ സഹോദരിയോ ഉണ്ടാവുകയുമാണെങ്കില്‍ അവരിലോരോരുത്തര്‍ക്കും ആറിലൊന്ന് വീതം ലഭിക്കുന്നതാണ്. അഥവാ, അവര്‍ ഒന്നില്‍ കൂടുതല്‍ പേരുണ്ടെങ്കില്‍ മൂന്നിലൊന്നില്‍ അവര്‍ സമാവകാശികളായിരിക്കും. ദ്രോഹകരമല്ലാത്ത വസ്വിയ്യത്തോ കടമോ ഉണ്ടെങ്കില്‍ അവ കഴിച്ചാണിത്. ഇതൊക്കെയും അല്ലാഹുവില്‍നിന്നുള്ള ഉപദേശമാണ്. അല്ലാഹു എല്ലാം അറിയുന്നവനും ഏറെ ക്ഷമിക്കുന്നവനുമത്രെ.
\end{malayalam}}
\flushright{\begin{Arabic}
\quranayah[4][13]
\end{Arabic}}
\flushleft{\begin{malayalam}
ഇവയെല്ലാം അല്ലാഹു നിശ്ചയിച്ച നിയമപരിധികളാണ്. അല്ലാഹുവെയും അവന്റെ ദൂതനെയും അനുസരിക്കുന്നവനെ അല്ലാഹു താഴ്ഭാഗത്തൂടെ ആറുകളൊഴുകുന്ന സ്വര്‍ഗീയാരാമങ്ങളില്‍ പ്രവേശിപ്പിക്കും. അവരതില്‍ സ്ഥിരവാസികളായിരിക്കും. അതുതന്നെയാണ് അതിമഹത്തായ വിജയം.
\end{malayalam}}
\flushright{\begin{Arabic}
\quranayah[4][14]
\end{Arabic}}
\flushleft{\begin{malayalam}
എന്നാല്‍, അല്ലാഹുവെയും അവന്റെ ദൂതനെയും ധിക്കരിക്കുകയും അവന്റെ പരിധികള്‍ ലംഘിക്കുകയും ചെയ്യുന്നവനെ അല്ലാഹു നരകത്തീയിലേക്കാണ് തള്ളിവിടുക. അവനതില്‍ സ്ഥിരവാസിയായിരിക്കും. വളരെ നിന്ദ്യമായ ശിക്ഷയാണ് അവന്നുണ്ടാവുക.
\end{malayalam}}
\flushright{\begin{Arabic}
\quranayah[4][15]
\end{Arabic}}
\flushleft{\begin{malayalam}
നിങ്ങളുടെ സ്ത്രീകളില്‍ അവിഹിതവൃത്തിയിലേര്‍പ്പെട്ടവര്‍ക്കെതിരെ നിങ്ങളില്‍നിന്ന് നാലുപേരെ സാക്ഷികളായി കൊണ്ടുവരിക. അവര്‍ സാക്ഷ്യം വഹിച്ചാല്‍ ആ സ്ത്രീകളെ വീടുകളില്‍ തടഞ്ഞുവെക്കുക; അവരെ മരണം പിടികൂടുകയോ അല്ലാഹു അവര്‍ക്ക് എന്തെങ്കിലും വഴി തുറന്നുകൊടുക്കുകയോ ചെയ്യുംവരെ.
\end{malayalam}}
\flushright{\begin{Arabic}
\quranayah[4][16]
\end{Arabic}}
\flushleft{\begin{malayalam}
നിങ്ങളില്‍നിന്ന് ഈ ഹീനവൃത്തിയിലേര്‍പ്പെടുന്ന ഇരുവരെയും നിങ്ങള്‍ പീഡിപ്പിക്കുക. അവരിരുവരും പശ്ചാത്തപിക്കുകയും സ്വയം നന്നാവുകയും ചെയ്താല്‍ നിങ്ങളവരെ വെറുതെ വിട്ടേക്കുക. അല്ലാഹു പശ്ചാത്താപം സ്വീകരിക്കുന്നവനും ദയാപരനുമാകുന്നു.
\end{malayalam}}
\flushright{\begin{Arabic}
\quranayah[4][17]
\end{Arabic}}
\flushleft{\begin{malayalam}
അറിയുക: അറിവില്ലായ്മ കാരണം തെറ്റ് ചെയ്യുകയും ഒട്ടും വൈകാതെ അനുതപിക്കുകയും ചെയ്യുന്നവര്‍ക്കുള്ളതാണ് പശ്ചാത്താപം. അവരുടെ പശ്ചാത്താപം അല്ലാഹു സ്വീകരിക്കും. അല്ലാഹു എല്ലാം അറിയുന്നവനും യുക്തിമാനുമാണ്.
\end{malayalam}}
\flushright{\begin{Arabic}
\quranayah[4][18]
\end{Arabic}}
\flushleft{\begin{malayalam}
തെറ്റുകള്‍ ചെയ്തുകൊണ്ടിരിക്കുകയും മരണമടുക്കുമ്പോള്‍ “ഞാനിതാ പശ്ചാത്തപിച്ചിരിക്കുന്നു” എന്നു പറയുകയും ചെയ്യുന്നവര്‍ക്കുള്ളതല്ല പശ്ചാത്താപം. സത്യനിഷേധികളായി മരണമടയുന്നവര്‍ക്കുള്ളതുമല്ല. അവര്‍ക്കു നാം ഒരുക്കിവെച്ചത് നോവുറ്റ ശിക്ഷയാണ്.
\end{malayalam}}
\flushright{\begin{Arabic}
\quranayah[4][19]
\end{Arabic}}
\flushleft{\begin{malayalam}
വിശ്വസിച്ചവരേ, സ്ത്രീകളെ ബലപ്രയോഗത്തിലൂടെ അനന്തരമെടുക്കാന്‍ നിങ്ങള്‍ക്കനുവാദമില്ല. നിങ്ങള്‍ അവര്‍ക്ക് നല്‍കിയ വിവാഹമൂല്യത്തില്‍നിന്ന് ഒരുഭാഗം തട്ടിയെടുക്കാനായി നിങ്ങളവരെ പീഡിപ്പിക്കരുത്- അവര്‍ പ്രകടമായ ദുര്‍നടപ്പില്‍ ഏര്‍പ്പെട്ടാലല്ലാതെ. അവരോട് മാന്യമായി സഹവസിക്കുക. അഥവാ, നിങ്ങളവരെ വെറുക്കുന്നുവെങ്കില്‍ അറിയുക: നിങ്ങള്‍ വെറുക്കുന്ന പലതിലും അല്ലാഹു ധാരാളം നന്മ ഉള്‍ക്കൊള്ളിച്ചിട്ടുണ്ടാവാം.
\end{malayalam}}
\flushright{\begin{Arabic}
\quranayah[4][20]
\end{Arabic}}
\flushleft{\begin{malayalam}
നിങ്ങള്‍ ഒരു ഭാര്യയുടെ സ്ഥാനത്ത് മറ്റൊരു ഭാര്യയെ സ്വീകരിക്കാന്‍ തന്നെയാണ് ഉദ്ദേശിക്കുന്നതെങ്കില്‍ ആദ്യഭാര്യക്ക് സമ്പത്തിന്റെ ഒരു കൂമ്പാരം തന്നെ കൊടുത്തിട്ടുണ്ടെങ്കിലും അതില്‍നിന്ന് ഒന്നുംതന്നെ തിരിച്ചുവാങ്ങരുത്. കള്ളം കെട്ടിച്ചമച്ചും പ്രകടമായ അനീതി കാണിച്ചും നിങ്ങളത് തിരിച്ചെടുക്കുകയോ?
\end{malayalam}}
\flushright{\begin{Arabic}
\quranayah[4][21]
\end{Arabic}}
\flushleft{\begin{malayalam}
നിങ്ങളെങ്ങനെ അവളില്‍നിന്നത് തിരിച്ചുവാങ്ങും? നിങ്ങള്‍ പരസ്പരം ലയിച്ചുചേര്‍ന്ന് ജീവിക്കുകയും നിങ്ങളില്‍നിന്ന് അവര്‍ കരുത്തുറ്റ കരാര്‍ വാങ്ങുകയും ചെയ്തിരിക്കെ!
\end{malayalam}}
\flushright{\begin{Arabic}
\quranayah[4][22]
\end{Arabic}}
\flushleft{\begin{malayalam}
നിങ്ങളുടെ പിതാക്കള്‍ വിവാഹം ചെയ്തിരുന്ന സ്ത്രീകളെ നിങ്ങള്‍ വിവാഹം കഴിക്കരുത് -മുമ്പ് നടന്നുകഴിഞ്ഞതല്ലാതെ- തീര്‍ച്ചയായും അത് മ്ളേഛമാണ്; വെറുക്കപ്പെട്ടതും ദുര്‍മാര്‍ഗവുമാണ്.
\end{malayalam}}
\flushright{\begin{Arabic}
\quranayah[4][23]
\end{Arabic}}
\flushleft{\begin{malayalam}
നിങ്ങളുടെ മാതാക്കള്‍, പുത്രിമാര്‍, സഹോദരിമാര്‍, പിതൃസഹോദരിമാര്‍, മാതൃസഹോദരിമാര്‍, സഹോദരപുത്രിമാര്‍, സഹോദരീ പുത്രിമാര്‍, നിങ്ങളെ മുലയൂട്ടിയ പോറ്റമ്മമാര്‍, മുലകുടി ബന്ധത്തിലെ സഹോദരിമാര്‍, നിങ്ങളുടെ ഭാര്യാമാതാക്കള്‍ എന്നിവരെ വിവാഹം ചെയ്യല്‍ നിങ്ങള്‍ക്ക് നിഷിദ്ധമാക്കിയിരിക്കുന്നു. നിങ്ങള്‍ ശാരീരികബന്ധത്തിലേര്‍പ്പെട്ട നിങ്ങളുടെ ഭാര്യമാരുടെ, നിങ്ങളുടെ സംരക്ഷണത്തിലുള്ള വളര്‍ത്തുപുത്രിമാരെയും നിങ്ങള്‍ക്ക് വിലക്കിയിരിക്കുന്നു. അഥവാ നിങ്ങളവരുമായി ശാരീരികബന്ധത്തിലേര്‍പ്പെട്ടിട്ടില്ലെങ്കില്‍ നിങ്ങള്‍ക്കതില്‍ തെറ്റില്ല. നിങ്ങളുടെ ബീജത്തില്‍ ജനിച്ച പുത്രന്മാരുടെ ഭാര്യമാരെയും നിങ്ങള്‍ക്ക് നിഷിദ്ധമാക്കിയിരിക്കുന്നു. രണ്ടു സഹോദരിമാരെ ഒരുമിച്ചു ഭാര്യമാരാക്കുന്നതും വിലക്കപ്പെട്ടതുതന്നെ- നേരത്തെ സംഭവിച്ചതൊഴികെ. അല്ലാഹു ഏറെ പൊറുക്കുന്നവനും ദയാപരനുമാകുന്നു.
\end{malayalam}}
\flushright{\begin{Arabic}
\quranayah[4][24]
\end{Arabic}}
\flushleft{\begin{malayalam}
ഭര്‍ത്തൃമതികളായ സ്ത്രീകളും നിങ്ങള്‍ക്കു നിഷിദ്ധമാണ്. എന്നാല്‍ യുദ്ധത്തടവുകാരായി നിങ്ങളുടെ അധീനതയില്‍ വന്നവര്‍ ഇതില്‍നിന്നൊഴിവാണ്. ഇതെല്ലാം നിങ്ങള്‍ക്കുള്ള ദൈവിക നിയമമാണ്. ഇവരല്ലാത്ത സ്ത്രീകളെയെല്ലാം വിവാഹമൂല്യം നല്‍കി നിങ്ങള്‍ക്ക് കല്യാണം കഴിക്കാവുന്നതാണ്. നിങ്ങള്‍ വിവാഹജീവിതം ആഗ്രഹിക്കുന്നുവരാകണം. അവിഹിതവേഴ്ച കാംക്ഷിക്കുന്നവരാകരുത്. അങ്ങനെ അവരുമായി ദാമ്പത്യസുഖമാസ്വദിച്ചാല്‍ നിര്‍ബന്ധമായും നിങ്ങളവര്‍ക്ക് വിവാഹമൂല്യം നല്‍കണം. വിവാഹമൂല്യം തീരുമാനിച്ചശേഷം നിങ്ങള്‍ പരസ്പരസമ്മതത്തോടെ വല്ല വിട്ടുവീഴ്ചയും ചെയ്യുന്നുവെങ്കില്‍ അതില്‍ തെറ്റില്ല. അല്ലാഹു എല്ലാം അറിയുന്നവനും യുക്തിമാനുമാണ്.
\end{malayalam}}
\flushright{\begin{Arabic}
\quranayah[4][25]
\end{Arabic}}
\flushleft{\begin{malayalam}
നിങ്ങളിലാര്‍ക്കെങ്കിലും വിശ്വാസിനികളായ സ്വതന്ത്ര സ്ത്രീകളെ വിവാഹം കഴിക്കാന്‍ കഴിയില്ലെങ്കില്‍ നിങ്ങളുടെ അധീനതയിലുള്ള വിശ്വാസിനികളായ അടിമസ്ത്രീകളെ വിവാഹം ചെയ്യാം. നിങ്ങളുടെ വിശ്വാസത്തെ സംബന്ധിച്ച് നന്നായറിയുക അല്ലാഹുവിനാണ്. നിങ്ങള്‍ ഒരേ വര്‍ഗത്തില്‍പെട്ടവരാണല്ലോ. അതിനാല്‍ അവരെ അവരുടെ രക്ഷിതാക്കളുടെ അനുവാദത്തോടെ നിങ്ങള്‍ വിവാഹം കഴിച്ചുകൊള്ളുക. അവര്‍ക്ക് ന്യായമായ വിവാഹമൂല്യം നല്‍കണം. അവര്‍ ചാരിത്രവതികളും ദുര്‍വൃത്തിയിലേര്‍പ്പെടാത്തവരും രഹസ്യവേഴ്ചക്കാരെ സ്വീകരിക്കാത്തവരുമായിരിക്കണം. അങ്ങനെ അവര്‍ ദാമ്പത്യവരുതിയില്‍ വന്നശേഷം അവര്‍ ദുര്‍വൃത്തിയിലേര്‍പ്പെടുകയാണെങ്കില്‍ സ്വതന്ത്ര സ്ത്രീകളുടെ പാതി ശിക്ഷയാണ് അവര്‍ക്കുണ്ടാവുക. വിവാഹം കഴിച്ചില്ലെങ്കില്‍ തെറ്റ് സംഭവിച്ചേക്കുമെന്ന് ഭയമുള്ളവര്‍ക്ക് വേണ്ടിയാണിത്. എന്നാല്‍ ക്ഷമയവലംബിക്കുന്നതാണ് നിങ്ങള്‍ക്ക് കൂടുതലുത്തമം. അല്ലാഹു ഏറെ പൊറുക്കുന്നവനും പരമദയാലുവുമാണ്.
\end{malayalam}}
\flushright{\begin{Arabic}
\quranayah[4][26]
\end{Arabic}}
\flushleft{\begin{malayalam}
നിങ്ങള്‍ക്ക് ദൈവിക നിയമങ്ങള്‍ വിവരിച്ചുതരാനും മുന്‍ഗാമികളുടെ മഹിതചര്യകള്‍ കാണിച്ചുതരാനും നിങ്ങളുടെ പശ്ചാത്താപം സ്വീകരിക്കാനും അല്ലാഹു ഉദ്ദേശിക്കുന്നു. അല്ലാഹു എല്ലാം അറിയുന്നവനും യുക്തിമാനുമാകുന്നു.
\end{malayalam}}
\flushright{\begin{Arabic}
\quranayah[4][27]
\end{Arabic}}
\flushleft{\begin{malayalam}
അല്ലാഹു നിങ്ങളുടെ പശ്ചാത്താപം സ്വീകരിക്കണമെന്നുദ്ദേശിക്കുന്നു. എന്നാല്‍ താന്തോന്നികളായി കഴിയുന്നവരാഗ്രഹിക്കുന്നത് നിങ്ങള്‍ നേര്‍വഴിയില്‍നിന്ന് ബഹുദൂരം അകന്നുപോകണമെന്നാണ്.
\end{malayalam}}
\flushright{\begin{Arabic}
\quranayah[4][28]
\end{Arabic}}
\flushleft{\begin{malayalam}
അല്ലാഹു നിങ്ങളുടെ ഭാരം കുറക്കാനുദ്ദേശിക്കുന്നു. ഏറെ ദുര്‍ബലനായാണല്ലോ മനുഷ്യന്‍ സൃഷ്ടിക്കപ്പെട്ടത്.
\end{malayalam}}
\flushright{\begin{Arabic}
\quranayah[4][29]
\end{Arabic}}
\flushleft{\begin{malayalam}
വിശ്വസിച്ചവരേ, നിങ്ങള്‍ നിങ്ങളുടെ ധനം അന്യോന്യം അന്യായമായി നേടിയെടുത്ത് തിന്നരുത്. പരസ്പരം പൊരുത്തത്തോടെ നടത്തുന്ന കച്ചവടത്തിലൂടെയല്ലാതെ. നിങ്ങള്‍ നിങ്ങളെത്തന്നെ കശാപ്പു ചെയ്യരുത്. അല്ലാഹു നിങ്ങളോട് ഏറെ കരുണയുള്ളവനാണ്; തീര്‍ച്ച.
\end{malayalam}}
\flushright{\begin{Arabic}
\quranayah[4][30]
\end{Arabic}}
\flushleft{\begin{malayalam}
അക്രമമായും അന്യായമായും അങ്ങനെ ചെയ്യുന്നവരെ നാം നരകത്തീയിലിട്ട് കരിക്കുക തന്നെ ചെയ്യും. അത് അല്ലാഹുവിന് ഏറെ എളുപ്പമാകുന്നു.
\end{malayalam}}
\flushright{\begin{Arabic}
\quranayah[4][31]
\end{Arabic}}
\flushleft{\begin{malayalam}
നിങ്ങളോട് വിലക്കിയ വന്‍പാപങ്ങള്‍ നിങ്ങള്‍ വര്‍ജിക്കുന്നുവെങ്കില്‍ നിങ്ങളുടെ ചെറിയ തെറ്റുകള്‍ നാം മായ്ച്ചുകളയും. മാന്യമായ ഇടങ്ങളില്‍ നിങ്ങളെ നാം പ്രവേശിപ്പിക്കും.
\end{malayalam}}
\flushright{\begin{Arabic}
\quranayah[4][32]
\end{Arabic}}
\flushleft{\begin{malayalam}
അല്ലാഹു നിങ്ങളില്‍ ചിലര്‍ക്ക് മറ്റു ചിലരെക്കാള്‍ ചില അനുഗ്രഹങ്ങള്‍ കൂടുതലായി നല്‍കിയിട്ടുണ്ട്. നിങ്ങള്‍ അതു കൊതിക്കാതിരിക്കുക. പുരുഷന്മാര്‍ക്ക് അവര്‍ സമ്പാദിച്ചതിനനുസരിച്ച വിഹിതമുണ്ട്. സ്ത്രീകള്‍ക്ക് അവര്‍ സമ്പാദിച്ചതിനൊത്ത വിഹിതവും. നിങ്ങള്‍ അല്ലാഹുവോട് അവന്റെ അനുഗ്രഹത്തിനായി പ്രാര്‍ഥിച്ചുകൊണ്ടിരിക്കുക. അല്ലാഹു എല്ലാ കാര്യങ്ങളും അറിയുന്നവനാണ്.
\end{malayalam}}
\flushright{\begin{Arabic}
\quranayah[4][33]
\end{Arabic}}
\flushleft{\begin{malayalam}
മാതാപിതാക്കളും ഉറ്റബന്ധുക്കളും വിട്ടേച്ചുപോയ സ്വത്തിനൊക്കെയും നാം അവകാശികളെ നിശ്ചയിച്ചിട്ടുണ്ട്. നിങ്ങളുടെ വലംകൈകള്‍ ബന്ധം സ്ഥാപിച്ച വര്‍ക്ക് അവരുടെ വിഹിതം നല്‍കുക. സംശയമില്ല; അല്ലാഹു എല്ലാ കാര്യങ്ങള്‍ക്കും സാക്ഷിയാണ്.
\end{malayalam}}
\flushright{\begin{Arabic}
\quranayah[4][34]
\end{Arabic}}
\flushleft{\begin{malayalam}
പുരുഷന്മാര്‍ സ്ത്രീകളുടെ നാഥന്മാരാണ്. അല്ലാഹു മനുഷ്യരിലൊരു വിഭാഗത്തിന് മറ്റുള്ളവരെക്കാള്‍ കഴിവു കൊടുത്തതിനാലും പുരുഷന്മാര്‍ അവരുടെ ധനം ചെലവഴിക്കുന്നതിനാലുമാണിത്. അതിനാല്‍ സച്ചരിതരായ സ്ത്രീകള്‍ അനുസരണശീലമുള്ളവരാണ്. പുരുഷന്മാരുടെ അഭാവത്തില്‍ അല്ലാഹു സംരക്ഷിക്കാനാവശ്യപ്പെട്ടതെല്ലാം കാത്തുസൂക്ഷിക്കുന്നവരുമാണ്. എന്നാല്‍ ഏതെങ്കിലും സ്ത്രീ അനുസരണക്കേട് കാണിക്കുമെന്ന് നിങ്ങളാശങ്കിക്കുന്നുവെങ്കില്‍ അവരെ ഗുണദോഷിക്കുക. കിടപ്പറകളില്‍ അവരുമായി അകന്നുനില്‍ക്കുക. അടിക്കുകയും ചെയ്യുക. അങ്ങനെ അവര്‍ നിങ്ങളെ അനുസരിക്കുന്നുവെങ്കില്‍ പിന്നെ നിങ്ങള്‍ അവര്‍ക്കെതിരായ നടപടികളൊന്നുമെടുക്കരുത്. അത്യുന്നതനും മഹാനുമാണ് അല്ലാഹു; തീര്‍ച്ച.
\end{malayalam}}
\flushright{\begin{Arabic}
\quranayah[4][35]
\end{Arabic}}
\flushleft{\begin{malayalam}
ദമ്പതികള്‍ക്കിടയില്‍ ഭിന്നിപ്പുണ്ടാകുമെന്ന് നിങ്ങള്‍ ഭയപ്പെടുന്നുവെങ്കില്‍ അവന്റെ ആള്‍ക്കാരില്‍നിന്ന് ഒരു മാധ്യസ്ഥനെ നിയോഗിക്കുക. അവളുടെ ആള്‍ക്കാരില്‍നിന്നൊരാളെയും. ഇരുവരും അനുരഞ്ജനമാണ് ആഗ്രഹിക്കുന്നതെങ്കില്‍ അല്ലാഹു അവര്‍ക്കിടയില്‍ യോജിപ്പുണ്ടാക്കുന്നതാണ്. അല്ലാഹു എല്ലാം അറിയുന്നവനും സൂക്ഷ്മജ്ഞനുമാണല്ലോ.
\end{malayalam}}
\flushright{\begin{Arabic}
\quranayah[4][36]
\end{Arabic}}
\flushleft{\begin{malayalam}
നിങ്ങള്‍ അല്ലാഹുവിന് വഴിപ്പെടുക. അവനില്‍ ഒന്നിനെയും പങ്കു ചേര്‍ക്കാതിരിക്കുക. മാതാപിതാക്കളോട് നന്നായി വര്‍ത്തിക്കുക. ബന്ധുക്കള്‍, അനാഥകള്‍, അഗതികള്‍, കുടുംബക്കാരായ അയല്‍ക്കാര്‍, അന്യരായ അയല്‍ക്കാര്‍, സഹവാസികള്‍, വഴിപോക്കര്‍, നിങ്ങളുടെ അധീനതയിലുള്ള അടിമകള്‍; എല്ലാവരോടും നല്ലനിലയില്‍ വര്‍ത്തിക്കുക. പൊങ്ങച്ചവും ദുരഹങ്കാരവുമുള്ള ആരെയും അല്ലാഹു ഒട്ടും ഇഷ്ടപ്പെടുന്നില്ല.
\end{malayalam}}
\flushright{\begin{Arabic}
\quranayah[4][37]
\end{Arabic}}
\flushleft{\begin{malayalam}
പിശുക്കുകാട്ടുകയും പിശുക്കുകാട്ടാന്‍ ജനങ്ങളെ പ്രേരിപ്പിക്കുകയും ചെയ്യുന്നവരാണവര്‍; അല്ലാഹു തന്റെ ഔദാര്യത്താല്‍ നല്‍കിയ അനുഗ്രഹങ്ങള്‍ മറച്ചുപിടിക്കുന്നവരും. ആ നന്ദികെട്ടവര്‍ക്ക് നന്നെ നിന്ദ്യമായ ശിക്ഷയാണ് നാം ഒരുക്കിവെച്ചിരിക്കുന്നത്.
\end{malayalam}}
\flushright{\begin{Arabic}
\quranayah[4][38]
\end{Arabic}}
\flushleft{\begin{malayalam}
ആളുകളെ കാണിക്കാനായി ധനം ചെലവഴിക്കുന്നവരാണവര്‍; അല്ലാഹുവിലോ അന്ത്യദിനത്തിലോ വിശ്വസിക്കാത്തവരും. പിശാച് ആരുടെയെങ്കിലും കൂട്ടാളിയാകുന്നുവെങ്കില്‍ അവന്‍ എത്ര ചീത്ത കൂട്ടുകാരന്‍.
\end{malayalam}}
\flushright{\begin{Arabic}
\quranayah[4][39]
\end{Arabic}}
\flushleft{\begin{malayalam}
അല്ലാഹുവിലും അന്ത്യദിനത്തിലും വിശ്വസിക്കുകയും അല്ലാഹു നല്‍കിയതില്‍നിന്ന് ചെലവഴിക്കുകയും ചെയ്യുന്നവര്‍ക്ക് എന്തു പറ്റാനാണ്? അല്ലാഹു അവരെപ്പറ്റി നന്നായറിയുന്നവനാണ്.
\end{malayalam}}
\flushright{\begin{Arabic}
\quranayah[4][40]
\end{Arabic}}
\flushleft{\begin{malayalam}
അല്ലാഹു ആരോടും അണുവോളം അനീതി കാണിക്കുകയില്ല. എന്നല്ല, നന്മയാണുള്ളതെങ്കില്‍ അതവന്‍ ഇരട്ടിയാക്കിക്കൊടുക്കും. തന്നില്‍ നിന്നുള്ള മഹത്തായ പ്രതിഫലം നല്‍കുകയും ചെയ്യും.
\end{malayalam}}
\flushright{\begin{Arabic}
\quranayah[4][41]
\end{Arabic}}
\flushleft{\begin{malayalam}
ഓരോ സമുദായത്തില്‍ നിന്നും ഓരോ സാക്ഷിയെ നാം കൊണ്ടുവരും. ഇക്കൂട്ടര്‍ക്കെതിരെ സാക്ഷിയായി നിന്നെയും കൊണ്ടുവരും. എന്തായിരിക്കും അപ്പോഴത്തെ അവസ്ഥ!
\end{malayalam}}
\flushright{\begin{Arabic}
\quranayah[4][42]
\end{Arabic}}
\flushleft{\begin{malayalam}
സത്യത്തെ നിഷേധിക്കുകയും ദൈവദൂതനെ ധിക്കരിക്കുകയും ചെയ്തവര്‍ അന്ന് കൊതിച്ചുപോകും: “തങ്ങളെ അകത്താക്കി ഭൂമിയൊന്ന് നിരപ്പായെങ്കില്‍ എത്ര നന്നായേനെ.” ഒരു വിവരവും അന്ന് അല്ലാഹുവില്‍നിന്ന് മറച്ചുവെക്കാനവര്‍ക്കാവില്ല.
\end{malayalam}}
\flushright{\begin{Arabic}
\quranayah[4][43]
\end{Arabic}}
\flushleft{\begin{malayalam}
വിശ്വസിച്ചവരേ, നിങ്ങള്‍ ലഹരി ബാധിതരായി നമസ്കാരത്തെ സമീപിക്കരുത്- നിങ്ങള്‍ പറയുന്നതെന്തെന്ന് നിങ്ങള്‍ക്ക് നല്ല ബോധമുണ്ടാകുംവരെ. ജനാബത്തുകാരനെങ്കില്‍ കുളിച്ചു ശുദ്ധി വരുത്തുന്നതുവരെയും- വഴിയാത്രക്കാരാണെങ്കിലല്ലാതെ. അഥവാ, നിങ്ങള്‍ രോഗികളാവുകയോ യാത്രയിലാവുകയോ ചെയ്തു; അല്ലെങ്കില്‍ നിങ്ങളിലൊരാള്‍ വിസര്‍ജനം കഴിഞ്ഞുവന്നു; അതുമല്ലെങ്കില്‍ സ്ത്രീകളുമായി സംസര്‍ഗം നടത്തി; എന്നിട്ട് വെള്ളം കിട്ടിയതുമില്ല; എങ്കില്‍ ശുദ്ധിയുള്ള മണ്ണ് ഉപയോഗിക്കുക. അതുകൊണ്ട് നിങ്ങളുടെ മുഖവും കൈകളും തടവുക. തീര്‍ച്ചയായും അല്ലാഹു ഏറെ മാപ്പേകുന്നവനും പൊറുക്കുന്നവനുമാണ്.
\end{malayalam}}
\flushright{\begin{Arabic}
\quranayah[4][44]
\end{Arabic}}
\flushleft{\begin{malayalam}
വേദപുസ്തകത്തില്‍നിന്ന് ഒരു ഭാഗം കിട്ടിയവരെ നീ കാണുന്നില്ലേ? അവര്‍ വഴികേട് വിലയ്ക്കു വാങ്ങുന്നു. നിങ്ങള്‍ വഴിതെറ്റിപ്പോകണമെന്നാഗ്രഹിക്കുകയും ചെയ്യുന്നു.
\end{malayalam}}
\flushright{\begin{Arabic}
\quranayah[4][45]
\end{Arabic}}
\flushleft{\begin{malayalam}
നിങ്ങളുടെ എതിരാളികളെപ്പറ്റി നന്നായറിയുന്നവന്‍ അല്ലാഹുവാണ്. രക്ഷകനായി നിങ്ങള്‍ക്ക് അല്ലാഹു മതി. തുണയായും അല്ലാഹുതന്നെ മതി.
\end{malayalam}}
\flushright{\begin{Arabic}
\quranayah[4][46]
\end{Arabic}}
\flushleft{\begin{malayalam}
ആ എതിരാളികള്‍ ജൂതന്മാരില്‍ പെട്ടവരാണ്. അവര്‍ വാക്കുകളെ സന്ദര്‍ഭത്തില്‍ നിന്നടര്‍ത്തിയെടുത്ത് ഉപയോഗിക്കുന്നു. തങ്ങളുടെ നാവു കോട്ടിയും സത്യമതത്തെ കടന്നാക്രമിച്ചും “സമിഅ്നാ വ അസ്വൈനാ” എന്നും “ഇസ്മഅ് ഗൈറ മുസ്മഅ്” എന്നും “റാഇനാ” എന്നും അവര്‍ പറയുന്നു. “സമിഅ്നാ വ അത്വഅ്നാ” എന്നും “ഇസ്മഅ്” എന്നും “ഉന്‍ളുര്‍നാ” എന്നുമാണ് അവര്‍ പറഞ്ഞിരുന്നതെങ്കില്‍ അതവര്‍ക്ക് കൂടുതല്‍ നന്നായേനെ. ഏറ്റം ശരിയായതും അതുതന്നെ. പക്ഷേ, അവരുടെ സത്യനിഷേധം കാരണമായി അല്ലാഹു അവരെ ശപിച്ചിരിക്കുന്നു. അതിനാലവര്‍ വിശ്വസിക്കുകയില്ല; ഇത്തിരിയല്ലാതെ.
\end{malayalam}}
\flushright{\begin{Arabic}
\quranayah[4][47]
\end{Arabic}}
\flushleft{\begin{malayalam}
വേദക്കാരേ, നിങ്ങളുടെ വശമുള്ള വേദത്തെ ശരിവെച്ചുകൊണ്ട്, നാം ഇറക്കിയ ഈ വേദത്തില്‍ വിശ്വസിക്കുക. നാം ചില മുഖങ്ങളെ വികൃതമാക്കി പിറകോട്ട് തിരിക്കുകയോ സാബത്തുകാരെ ശപിച്ചപോലെ ശപിക്കുകയോ ചെയ്യുംമുമ്പെ നിങ്ങള്‍ വിശ്വസിക്കുവിന്‍. അല്ലാഹുവിന്റെ വിധി നടപ്പിലാവുക തന്നെ ചെയ്യും.
\end{malayalam}}
\flushright{\begin{Arabic}
\quranayah[4][48]
\end{Arabic}}
\flushleft{\begin{malayalam}
അല്ലാഹു, തന്നില്‍ പങ്കുചേര്‍ക്കുന്നത് പൊറുക്കില്ല. അതല്ലാത്ത പാപങ്ങളൊക്കെയും അവനിച്ഛിക്കുന്നവര്‍ക്ക് അവന്‍ പൊറുത്തുകൊടുക്കും. അല്ലാഹുവിന് പങ്കാളികളെ സങ്കല്‍പിക്കുന്നവന്‍ കൊടിയ കുറ്റമാണ് ചെയ്യുന്നത്; തീര്‍ച്ച.
\end{malayalam}}
\flushright{\begin{Arabic}
\quranayah[4][49]
\end{Arabic}}
\flushleft{\begin{malayalam}
വിശുദ്ധരെന്ന് സ്വയം അവകാശപ്പെടുന്നവരെ നീ കണ്ടില്ലേ? എന്നാല്‍ അല്ലാഹു അവനിച്ഛിക്കുന്നവരെ ശുദ്ധീകരിക്കുന്നു. അവരോട് ഒട്ടും അനീതി കാണിക്കുകയില്ല.
\end{malayalam}}
\flushright{\begin{Arabic}
\quranayah[4][50]
\end{Arabic}}
\flushleft{\begin{malayalam}
അവര്‍ അല്ലാഹുവിന്റെ പേരില്‍ കള്ളം കെട്ടിച്ചമയ്ക്കുന്നതെങ്ങനെയെന്ന് നോക്കൂ? പ്രകടമായ പാപമായിട്ട് അതു തന്നെ മതി.
\end{malayalam}}
\flushright{\begin{Arabic}
\quranayah[4][51]
\end{Arabic}}
\flushleft{\begin{malayalam}
വേദവിജ്ഞാനത്തില്‍നിന്നൊരു വിഹിതം ലഭിച്ചവരെ നീ കണ്ടില്ലേ? അവര്‍ ഗൂഢവിദ്യകളിലും പൈശാചിക ശക്തികളിലും വിശ്വസിക്കുന്നു. “ഇവര്‍ സത്യവിശ്വാസികളെക്കാള്‍ നേര്‍വഴിയിലാണെ”ന്ന് സത്യനിഷേധികളെ സംബന്ധിച്ച് പറയുകയും ചെയ്യുന്നു.
\end{malayalam}}
\flushright{\begin{Arabic}
\quranayah[4][52]
\end{Arabic}}
\flushleft{\begin{malayalam}
അറിയുക: അല്ലാഹു ശപിച്ചവരാണവര്‍. അല്ലാഹു ശപിച്ചവനെ സഹായിക്കുന്ന ആരെയും നിനക്ക് കണ്ടെത്താനാവില്ല.
\end{malayalam}}
\flushright{\begin{Arabic}
\quranayah[4][53]
\end{Arabic}}
\flushleft{\begin{malayalam}
അതല്ല; അവര്‍ക്ക് അധികാരത്തിലെന്തെങ്കിലും പങ്കുണ്ടോ? ഉണ്ടായിരുന്നെങ്കില്‍ അവരതില്‍ നിന്ന് ഒന്നും ജനങ്ങള്‍ക്ക് നല്‍കുമായിരുന്നില്ല.
\end{malayalam}}
\flushright{\begin{Arabic}
\quranayah[4][54]
\end{Arabic}}
\flushleft{\begin{malayalam}
അതല്ല; അല്ലാഹു തന്റെ ഔദാര്യത്തില്‍നിന്ന് നല്‍കിയതിന്റെ പേരില്‍ അവര്‍ ജനങ്ങളോട് അസൂയപ്പെടുകയാണോ? എന്നാല്‍ ഇബ്റാഹീം കുടുംബത്തിന് നാം വേദവും തത്ത്വജ്ഞാനവും നല്‍കിയിട്ടുണ്ട്. അവര്‍ക്കു നാം അതിമഹത്തായ ആധിപത്യവും നല്‍കി.
\end{malayalam}}
\flushright{\begin{Arabic}
\quranayah[4][55]
\end{Arabic}}
\flushleft{\begin{malayalam}
അവരില്‍ ആ സന്ദേശത്തില്‍ വിശ്വസിച്ചവരുണ്ട്. അതില്‍നിന്ന് പിന്തിരിഞ്ഞവരുമുണ്ട്. അവര്‍ക്ക് കത്തിക്കാളും നരകത്തീതന്നെമതി.
\end{malayalam}}
\flushright{\begin{Arabic}
\quranayah[4][56]
\end{Arabic}}
\flushleft{\begin{malayalam}
നമ്മുടെ പ്രമാണങ്ങളെ തള്ളിക്കളഞ്ഞവരെ നാം നരകത്തീയിലെറിയും; തീര്‍ച്ച. അവരുടെ തൊലി വെന്തുരുകുംതോറും അവര്‍ക്കു പുതിയ തൊലി നാം മാറ്റിക്കൊടുത്തുകൊണ്ടിരിക്കും. തുടര്‍ന്നും അവര്‍ നമ്മുടെ ശിക്ഷ അനുഭവിക്കാന്‍. സംശയമില്ല; അല്ലാഹു പ്രതാപിയും യുക്തിമാനും തന്നെ.
\end{malayalam}}
\flushright{\begin{Arabic}
\quranayah[4][57]
\end{Arabic}}
\flushleft{\begin{malayalam}
എന്നാല്‍ വിശ്വസിക്കുകയും സല്‍ക്കര്‍മങ്ങള്‍ പ്രവര്‍ത്തിക്കുകയും ചെയ്തവരെ നാം താഴ്ഭാഗത്തൂടെ ആറുകളൊഴുകുന്ന സ്വര്‍ഗീയാരാമങ്ങളില്‍ പ്രവേശിപ്പിക്കും. അവരതില്‍ സ്ഥിരവാസികളായിരിക്കും. അവര്‍ക്കവിടെ പരിശുദ്ധരായ ഇണകളുണ്ട്. അവരെ നാം ഇടതിങ്ങിയ പച്ചിലത്തണലില്‍ പ്രവേശിപ്പിക്കും.
\end{malayalam}}
\flushright{\begin{Arabic}
\quranayah[4][58]
\end{Arabic}}
\flushleft{\begin{malayalam}
അല്ലാഹു നിങ്ങളോടിതാ കല്‍പിക്കുന്നു: നിങ്ങളെ വിശ്വസിച്ചേല്‍പിച്ച വസ്തുക്കള്‍ അവയുടെ അവകാശികളെ തിരിച്ചേല്‍പിക്കുക. ജനങ്ങള്‍ക്കിടയില്‍ തീര്‍പ്പ് കല്‍പിക്കുകയാണെങ്കില്‍ നീതിപൂര്‍വം വിധി നടത്തുക. എത്ര നല്ല ഉപദേശമാണ് അല്ലാഹു നിങ്ങള്‍ക്കു നല്‍കുന്നത്. അല്ലാഹു എല്ലാം കേള്‍ക്കുന്നവനും കാണുന്നവനുമാണ്.
\end{malayalam}}
\flushright{\begin{Arabic}
\quranayah[4][59]
\end{Arabic}}
\flushleft{\begin{malayalam}
വിശ്വസിച്ചവരേ, അല്ലാഹുവെ അനുസരിക്കുക. ദൈവദൂതനെയും നിങ്ങളില്‍നിന്നുള്ള കൈകാര്യ കര്‍ത്താക്കളെയും അനുസരിക്കുക. ഏതെങ്കിലും കാര്യത്തില്‍ നിങ്ങള്‍ തമ്മില്‍ തര്‍ക്കമുണ്ടായാല്‍ അത് അല്ലാഹുവിലേക്കും അവന്റെ ദൂതനിലേക്കും മടക്കുക. നിങ്ങള്‍ അല്ലാഹുവിലും അന്ത്യദിനത്തിലും വിശ്വസിക്കുന്നവരെങ്കില്‍ ഇതാണ് ഏറ്റം നല്ലത്. മെച്ചപ്പെട്ട ഒടുക്കമുണ്ടാവുന്നതും ഇതിനുതന്നെ.
\end{malayalam}}
\flushright{\begin{Arabic}
\quranayah[4][60]
\end{Arabic}}
\flushleft{\begin{malayalam}
നിനക്ക് ഇറക്കിത്തന്നതിലും നിനക്കുമുമ്പ് ഇറക്കിക്കിട്ടിയതിലും തങ്ങള്‍ വിശ്വസിച്ചിരിക്കുന്നുവെന്ന് വാദിക്കുന്നവരെ നീ കണ്ടില്ലേ? അല്ലാഹുവിന്റേതല്ലാത്ത വിധികള്‍ നല്‍കുന്നവരുടെ അടുത്തേക്ക് തീര്‍പ്പു തേടിപ്പോകാനാണ് അവരുദ്ദേശിക്കുന്നത്. സത്യത്തില്‍ അവരെ തള്ളിക്കളയാനാണ് ഇവരോട് കല്‍പിച്ചിരിക്കുന്നത്. പിശാച് അവരെ നേര്‍വഴിയില്‍നിന്ന് തെറ്റിച്ച് സത്യത്തില്‍ നിന്ന് ഏറെ ദൂരെയാക്കാനാണ്ആഗ്രഹിക്കുന്നത്.
\end{malayalam}}
\flushright{\begin{Arabic}
\quranayah[4][61]
\end{Arabic}}
\flushleft{\begin{malayalam}
അല്ലാഹു ഇറക്കിത്തന്നതിലേക്കും അവന്റെ ദൂതനിലേക്കും വരികയെന്ന് പറഞ്ഞാല്‍ ആ കപടവിശ്വാസികള്‍ നിന്നില്‍നിന്നും പിന്തിരിഞ്ഞുപോകുന്നത് നിനക്കുകാണാം.
\end{malayalam}}
\flushright{\begin{Arabic}
\quranayah[4][62]
\end{Arabic}}
\flushleft{\begin{malayalam}
എന്നാല്‍ സ്വന്തം കരങ്ങള്‍ വരുത്തിവെച്ച വിനകള്‍ അവരെ ബാധിക്കുമ്പോഴത്തെ അവസ്ഥ എന്തായിരിക്കും? അപ്പോഴവര്‍ നിന്റെ അടുത്തുവന്ന് അല്ലാഹുവിന്റെ പേരില്‍ ആണയിട്ടുപറയും: "ഞങ്ങള്‍ നന്മയും അനുരഞ്ജനവുമല്ലാതൊന്നും ഉദ്ദേശിച്ചിട്ടില്ല.”
\end{malayalam}}
\flushright{\begin{Arabic}
\quranayah[4][63]
\end{Arabic}}
\flushleft{\begin{malayalam}
എന്നാല്‍, അവരുടെ മനസ്സുകളിലുള്ളത് അല്ലാഹു അറിയുന്നുണ്ട്. അതിനാല്‍ അവരെ വിട്ടേക്കുക. അവര്‍ക്ക് സദുപദേശം നല്‍കുക. അവരോട് ഉള്ളില്‍ത്തട്ടുന്ന വാക്ക് പറയുകയും ചെയ്യുക.
\end{malayalam}}
\flushright{\begin{Arabic}
\quranayah[4][64]
\end{Arabic}}
\flushleft{\begin{malayalam}
അല്ലാഹുവിന്റെ കല്‍പനപ്രകാരം അനുസരിക്കപ്പെടാന്‍വേണ്ടിയല്ലാതെ ഒരു ദൂതനെയും നാം അയച്ചിട്ടില്ല. അവര്‍ തങ്ങളോടുതന്നെ അതിക്രമം കാണിച്ചുകൊണ്ട് നിന്റെ അടുത്തുവന്നു. എന്നിട്ടവര്‍ അല്ലാഹുവോട് മാപ്പിരന്നു, ദൈവദൂതന്‍ അവര്‍ക്കായി പാപമോചനം തേടുകയും ചെയ്തു. എങ്കില്‍, അല്ലാഹുവെ അവര്‍ക്ക് ഏറെ മാപ്പരുളുന്നവനും കരുണാമയനുമായി കാണാമായിരുന്നു.
\end{malayalam}}
\flushright{\begin{Arabic}
\quranayah[4][65]
\end{Arabic}}
\flushleft{\begin{malayalam}
എന്നാല്‍ അങ്ങനെയല്ല; നിന്റെ നാഥന്‍ തന്നെ സത്യം! അവര്‍ക്കിടയിലെ തര്‍ക്കങ്ങളില്‍ നിന്നെയവര്‍ വിധികര്‍ത്താവാക്കുകയും നീ നല്‍കുന്ന വിധിതീര്‍പ്പില്‍ അവരൊട്ടും അലോസരമനുഭവിക്കാതിരിക്കുകയും അതിനെ പൂര്‍ണസമ്മതത്തോടെ സ്വീകരിക്കുകയും ചെയ്യുന്നില്ലെങ്കില്‍ അവര്‍ യഥാര്‍ഥ സത്യവിശ്വാസികളാവുകയില്ല; തീര്‍ച്ച.
\end{malayalam}}
\flushright{\begin{Arabic}
\quranayah[4][66]
\end{Arabic}}
\flushleft{\begin{malayalam}
ദൈവമാര്‍ഗത്തില്‍ ജീവന്‍ അര്‍പ്പിക്കണമെന്നോ വീട് വിട്ടുപോകണമെന്നോ നാം ആജ്ഞാപിച്ചിരുന്നുവെങ്കില്‍ അവരില്‍ ചുരുക്കം ചിലരൊഴികെ ആരും അത് നടപ്പാക്കുമായിരുന്നില്ല. എന്നാല്‍ ഉപദേശിച്ചതനുസരിച്ച് പ്രവര്‍ത്തിച്ചിരുന്നുവെങ്കില്‍ അതവര്‍ക്ക് ഏറെ ഗുണകരമായേനെ. കൂടുതല്‍ സ്ഥൈര്യം നല്‍കുകയും ചെയ്യുമായിരുന്നു.
\end{malayalam}}
\flushright{\begin{Arabic}
\quranayah[4][67]
\end{Arabic}}
\flushleft{\begin{malayalam}
അതോടൊപ്പം നാമവര്‍ക്ക് നമ്മുടെ ഭാഗത്തുനിന്നുള്ള അതിമഹത്തായ പ്രതിഫലം നല്‍കുമായിരുന്നു.
\end{malayalam}}
\flushright{\begin{Arabic}
\quranayah[4][68]
\end{Arabic}}
\flushleft{\begin{malayalam}
നാം അവരെ നേര്‍വഴിയില്‍ നയിക്കുകയും ചെയ്യുമായിരുന്നു.
\end{malayalam}}
\flushright{\begin{Arabic}
\quranayah[4][69]
\end{Arabic}}
\flushleft{\begin{malayalam}
അല്ലാഹുവെയും അവന്റെ ദൂതനെയും അനുസരിക്കുന്നവര്‍ അല്ലാഹു അനുഗ്രഹിച്ച പ്രവാചകന്മാര്‍, സത്യസന്ധര്‍, രക്തസാക്ഷികള്‍, സച്ചരിതര്‍ എന്നിവരോടൊപ്പമായിരിക്കും. അവരെത്ര നല്ല കൂട്ടുകാര്‍.
\end{malayalam}}
\flushright{\begin{Arabic}
\quranayah[4][70]
\end{Arabic}}
\flushleft{\begin{malayalam}
അല്ലാഹുവിങ്കല്‍ നിന്നുള്ള അനുഗ്രഹം തന്നെയാണത്. എല്ലാം അറിയുന്നവനായി അല്ലാഹു തന്നെ മതി.
\end{malayalam}}
\flushright{\begin{Arabic}
\quranayah[4][71]
\end{Arabic}}
\flushleft{\begin{malayalam}
വിശ്വസിച്ചവരേ, നിങ്ങള്‍ ജാഗ്രത പാലിക്കുക. അങ്ങനെ നിങ്ങള്‍ ചെറുസംഘങ്ങളായോ ഒന്നിച്ച് ഒറ്റസംഘമായോ യുദ്ധത്തിന് പുറപ്പെടുക.
\end{malayalam}}
\flushright{\begin{Arabic}
\quranayah[4][72]
\end{Arabic}}
\flushleft{\begin{malayalam}
എന്നാല്‍ അറച്ചുനില്‍ക്കുന്ന ചിലരും നിങ്ങളിലുണ്ട്. അങ്ങനെ നിങ്ങള്‍ക്ക് വല്ല വിപത്തും വന്നുപെട്ടാല്‍ അവന്‍ പറയും: "അല്ലാഹു എന്നെ അനുഗ്രഹിച്ചിരിക്കുന്നു. അല്ലെങ്കില്‍ ഞാനും അവരോടൊപ്പമുണ്ടാകുമായിരുന്നല്ലോ.”
\end{malayalam}}
\flushright{\begin{Arabic}
\quranayah[4][73]
\end{Arabic}}
\flushleft{\begin{malayalam}
എന്നാല്‍ നിങ്ങള്‍ക്ക് അല്ലാഹുവിങ്കല്‍ നിന്ന് വല്ല അനുഗ്രഹവും കിട്ടിയാലോ; അവനും നിങ്ങളും തമ്മില്‍ ഒട്ടും അടുപ്പം ഉണ്ടായിട്ടില്ലാത്തപോലെ അവന്‍ പറയും: "ഞാനും അവരുടെ കൂടെ ഉണ്ടായിരുന്നെങ്കില്‍ എനിക്കു വലിയ നേട്ടം കിട്ടിയേനെ.”
\end{malayalam}}
\flushright{\begin{Arabic}
\quranayah[4][74]
\end{Arabic}}
\flushleft{\begin{malayalam}
പരലോകത്തിനു വേണ്ടി ഈ ലോകജീവിതത്തെ വിറ്റവര്‍ അല്ലാഹുവിന്റെ മാര്‍ഗത്തില്‍ പടപൊരുതട്ടെ. അല്ലാഹുവിന്റെ മാര്‍ഗത്തില്‍ പടവെട്ടി വധിക്കപ്പെട്ടവന്നും വിജയം വരിച്ചവന്നും നാം അതിമഹത്തായ പ്രതിഫലം നല്‍കുന്നതാണ്.
\end{malayalam}}
\flushright{\begin{Arabic}
\quranayah[4][75]
\end{Arabic}}
\flushleft{\begin{malayalam}
നിങ്ങളെന്തുകൊണ്ട് ദൈവമാര്‍ഗത്തില്‍ യുദ്ധം ചെയ്യുന്നില്ല? മര്‍ദ്ദിതരായ പുരുഷന്മാര്‍ക്കും സ്ത്രീകള്‍ക്കും കുട്ടികള്‍ക്കും വേണ്ടിയും? അവരോ ഇങ്ങനെ പ്രാര്‍ഥിച്ചുകൊണ്ടിരിക്കുന്നവരാണ്: "ഞങ്ങളുടെ നാഥാ; മര്‍ദ്ദകരായ ജനം വിലസുന്ന ഈ നാട്ടില്‍ നിന്ന് ഞങ്ങളെ നീ മോചിപ്പിക്കേണമേ. നിന്റെ പക്കല്‍ നിന്ന് ഞങ്ങള്‍ക്ക് നീ ഒരു രക്ഷകനെ നിശ്ചയിച്ചുതരേണമേ. നിന്റെ ഭാഗത്തു നിന്ന് ഞങ്ങള്‍ക്ക് ഒരു സഹായിയെ നല്‍കേണമേ.”
\end{malayalam}}
\flushright{\begin{Arabic}
\quranayah[4][76]
\end{Arabic}}
\flushleft{\begin{malayalam}
സത്യവിശ്വാസികള്‍ അല്ലാഹുവിന്റെ മാര്‍ഗത്തില്‍ സമരം ചെയ്യുന്നു. സത്യനിഷേധികള്‍ ദൈവേതര ശക്തികളുടെ മാര്‍ഗത്തിലാണ് പടവെട്ടുന്നത്. അതിനാല്‍ നിങ്ങള്‍ പിശാചിന്റെ കൂട്ടാളികളോട് പടവെട്ടുക. പിശാചിന്റെ തന്ത്രം നന്നെ ദുര്‍ബലം തന്നെ; തീര്‍ച്ച.
\end{malayalam}}
\flushright{\begin{Arabic}
\quranayah[4][77]
\end{Arabic}}
\flushleft{\begin{malayalam}
"നിങ്ങള്‍ നിങ്ങളുടെ കൈകളെ നിയന്ത്രിച്ചു നിര്‍ത്തുക; നമസ്കാരം നിഷ്ഠയോടെ നിര്‍വഹിക്കുക; സകാത്ത് നല്‍കുകയും ചെയ്യുക; എന്ന കല്‍പന ലഭിച്ചവരെ നീ കണ്ടില്ലേ? പിന്നെ അവര്‍ക്ക് യുദ്ധം നിര്‍ബന്ധമാക്കിയപ്പോള്‍ അവരിലൊരുവിഭാഗം ജനങ്ങളെ പേടിക്കുന്നു; അല്ലാഹുവെപേടിക്കും പോലെയോ അതിനേക്കാള്‍ കൂടുതലോ ആയി. അവരിങ്ങനെ ആവലാതിപ്പെടുകയും ചെയ്യുന്നു: "ഞങ്ങളുടെ നാഥാ, നീ എന്തിനാണ് ഞങ്ങള്‍ക്ക് യുദ്ധം നിര്‍ബന്ധമാക്കിയത്. അടുത്ത ഒരവധിവരെയെങ്കിലും ഞങ്ങള്‍ക്ക് അവസരം തന്നുകൂടായിരുന്നോ?” അവരോടു പറയുക: "ഐഹിക ജീവിതവിഭവം നന്നെ നിസ്സാരമാണ്. ദൈവഭക്തര്‍ക്ക് പരലോകമാണ് കൂടുതലുത്തമം. അവിടെ നിങ്ങളോട് തീരേ അനീതി ഉണ്ടാവുകയില്ല.
\end{malayalam}}
\flushright{\begin{Arabic}
\quranayah[4][78]
\end{Arabic}}
\flushleft{\begin{malayalam}
"നിങ്ങള്‍ എവിടെയായിരുന്നാലും മരണം നിങ്ങളെ പിടികൂടും. നിങ്ങള്‍ ഭദ്രമായി കെട്ടിപ്പൊക്കിയ കോട്ടകള്‍ക്കകത്തായാലും.” വല്ല നന്മയും വന്നുകിട്ടിയാല്‍ അവര്‍ പറയും: "ഇത് ദൈവത്തിങ്കല്‍ നിന്നുള്ളതാണ്.” വല്ല വിപത്തും ബാധിച്ചാല്‍ അവര്‍ പറയും: "നീയാണിതിന് കാരണക്കാരന്‍.” പറയുക: "എല്ലാം അല്ലാഹുവിങ്കല്‍ നിന്നു തന്നെ. ഇവര്‍ക്കെന്തുപറ്റി? ഇവരൊരു കാര്യവും മനസ്സിലാക്കുന്നില്ലല്ലോ.”
\end{malayalam}}
\flushright{\begin{Arabic}
\quranayah[4][79]
\end{Arabic}}
\flushleft{\begin{malayalam}
നിനക്കു വന്നെത്തുന്ന നന്മയൊക്കെയും അല്ലാഹുവില്‍ നിന്നുള്ളതാണ്. നിന്നെ ബാധിക്കുന്ന വിപത്തുകളെല്ലാം നിന്നില്‍ നിന്നുള്ളതും. ജനങ്ങള്‍ക്കുള്ള ദൂതനായാണ് നിന്നെ നാം അയച്ചത്. അതിനു സാക്ഷിയായി അല്ലാഹു മതി.
\end{malayalam}}
\flushright{\begin{Arabic}
\quranayah[4][80]
\end{Arabic}}
\flushleft{\begin{malayalam}
ദൈവദൂതനെ അനുസരിക്കുന്നവന്‍ ഫലത്തില്‍ അല്ലാഹുവെയാണ് അനുസരിക്കുന്നത്. ആരെങ്കിലും പിന്തിരിഞ്ഞു പോകുന്നുവെങ്കില്‍ സാരമാക്കേണ്ടതില്ല. നിന്നെ നാം അവരുടെ മേല്‍നോട്ടക്കാരനായിട്ടൊന്നുമല്ലല്ലോ നിയോഗിച്ചത്.
\end{malayalam}}
\flushright{\begin{Arabic}
\quranayah[4][81]
\end{Arabic}}
\flushleft{\begin{malayalam}
തങ്ങള്‍ അനുസരണമുള്ളവരാണെന്ന് അവര്‍ പറയും. എന്നാല്‍ നിന്റെ അടുത്തുനിന്ന് പോയാല്‍ അവരില്‍ ചിലര്‍ തങ്ങള്‍ പറയുന്നതിന് വിരുദ്ധമായി രാത്രിയില്‍ ഒത്തുകൂടി നിനക്കെതിരെ ഗൂഢാലോചന നടത്തുന്നു. രാത്രിയിലെ അവരുടെ ഈ ചെയ്തികളൊക്കെയും അല്ലാഹു രേഖപ്പെടുത്തുന്നുണ്ട്. അതിനാല്‍ നീ അവരെ അവഗണിക്കുക. എല്ലാം അല്ലാഹുവില്‍ ഭരമേല്‍പിക്കുക. ഭരമേല്‍പിക്കാന്‍ അല്ലാഹു തന്നെ മതി.
\end{malayalam}}
\flushright{\begin{Arabic}
\quranayah[4][82]
\end{Arabic}}
\flushleft{\begin{malayalam}
അവര്‍ ഖുര്‍ആനെ സംബന്ധിച്ച് ചിന്തിക്കുന്നില്ലേ? അല്ലാഹു അല്ലാത്ത ആരില്‍ നിന്നെങ്കിലുമായിരുന്നെങ്കില്‍ അവരതില്‍ ധാരാളം പൊരുത്തക്കേടുകള്‍ കണ്ടെത്തുമായിരുന്നു.
\end{malayalam}}
\flushright{\begin{Arabic}
\quranayah[4][83]
\end{Arabic}}
\flushleft{\begin{malayalam}
സമാധാനത്തിന്റെയോ ഭയത്തിന്റെയോ വല്ല വാര്‍ത്തയും വന്നുകിട്ടിയാല്‍ അവരത് കൊട്ടിഘോഷിക്കും. മറിച്ച് അവരത് ദൈവദൂതന്നും അവരിലെത്തന്നെ ഉത്തരവാദപ്പെട്ടവര്‍ക്കും എത്തിച്ചിരുന്നെങ്കില്‍ ഉറപ്പായും അവരിലെ നിരീക്ഷണപാടവമുള്ളവര്‍ അതിന്റെ സത്യാവസ്ഥ മനസ്സിലാക്കുമായിരുന്നു. അല്ലാഹുവിന്റെ അനുഗ്രഹവും കാരുണ്യവും ഇല്ലായിരുന്നുവെങ്കില്‍, നിങ്ങളെല്ലാവരും പിശാചിന്റെ പിറകെ പോകുമായിരുന്നു, ഏതാനും ചിലരൊഴികെ.
\end{malayalam}}
\flushright{\begin{Arabic}
\quranayah[4][84]
\end{Arabic}}
\flushleft{\begin{malayalam}
അതിനാല്‍ നീ ദൈവമാര്‍ഗത്തില്‍ സമരം ചെയ്യുക. നിന്റെ സ്വന്തം കാര്യത്തിലല്ലാതെ ആരുടെമേലും നിനക്കൊരു ബാധ്യതയുമില്ല. സത്യവിശ്വാസികളെ സമരത്തിന് പ്രേരിപ്പിക്കുക. സത്യനിഷേധികളുടെ കടന്നാക്രമണ കഴിവിനെ അല്ലാഹു തടഞ്ഞുനിര്‍ത്തിയേക്കാം. അല്ലാഹു ഏറെ കരുത്തുറ്റവനാണ്. കൊടിയ ശിക്ഷ കൊടുക്കുന്നവനും.
\end{malayalam}}
\flushright{\begin{Arabic}
\quranayah[4][85]
\end{Arabic}}
\flushleft{\begin{malayalam}
നല്ലത് ശിപാര്‍ശ ചെയ്യുന്നവന് അതിലൊരു പങ്കു ലഭിക്കും. തിന്മ ശിപാര്‍ശ ചെയ്യുന്നവന് അതിലൊരു വിഹിതവുമുണ്ടാകും. അല്ലാഹു എല്ലാ കാര്യങ്ങളുടെയും മേല്‍നോട്ടം വഹിക്കുന്നവനത്രേ.
\end{malayalam}}
\flushright{\begin{Arabic}
\quranayah[4][86]
\end{Arabic}}
\flushleft{\begin{malayalam}
നിങ്ങളെ ആരെങ്കിലും അഭിവാദ്യം ചെയ്താല്‍ നിങ്ങള്‍ അതിലും നന്നായി പ്രത്യഭിവാദ്യം ചെയ്യുക. കുറഞ്ഞപക്ഷം അവ്വിധമെങ്കിലും തിരിച്ചുനല്‍കുക. അല്ലാഹു എല്ലാ കാര്യങ്ങളുടെയും കണക്ക് കൃത്യമായി നോക്കുന്നവനാണ്.
\end{malayalam}}
\flushright{\begin{Arabic}
\quranayah[4][87]
\end{Arabic}}
\flushleft{\begin{malayalam}
അല്ലാഹു അല്ലാതെ ദൈവമില്ല. ഉയിര്‍ത്തെഴുന്നേല്‍പുനാളില്‍ അവന്‍ നിങ്ങളെ ഒരുമിച്ചുകൂട്ടും. അതിലൊട്ടും സംശയമില്ല. അല്ലാഹുവെക്കാള്‍ വസ്തുനിഷ്ഠമായി വിവരം തരുന്ന ആരുണ്ട്?
\end{malayalam}}
\flushright{\begin{Arabic}
\quranayah[4][88]
\end{Arabic}}
\flushleft{\begin{malayalam}
കപടവിശ്വാസികളുടെ കാര്യത്തില്‍ നിങ്ങളെന്തുകൊണ്ട് രണ്ടു തട്ടുകളിലായി? അവര്‍ സമ്പാദിച്ച തിന്മ കാരണം അല്ലാഹു അവരെ കറക്കിയിട്ടിരിക്കുകയാണ്. അല്ലാഹു ദുര്‍മാര്‍ഗത്തിലാക്കിയവനെ നേര്‍വഴിയിലാക്കാനാണോ നിങ്ങള്‍ ശ്രമിക്കുന്നത്? എന്നാല്‍ അല്ലാഹു വഴികേടിലാക്കിയവനെ നേര്‍വഴിയിലാക്കാന്‍ ഒരു വഴിയും നിനക്ക് കണ്ടെത്താവില്ല.
\end{malayalam}}
\flushright{\begin{Arabic}
\quranayah[4][89]
\end{Arabic}}
\flushleft{\begin{malayalam}
അവര്‍ അവിശ്വസിച്ചപോലെ നിങ്ങളും അവിശ്വസിച്ച് എല്ലാവരും ഒരുപോലെ ആകണമെന്നാണ് അവരാഗ്രഹിക്കുന്നത്. അതിനാല്‍ അവര്‍ അല്ലാഹുവിന്റെ മാര്‍ഗത്തില്‍ നാടുവിട്ടു വരുംവരെ അവരില്‍നിന്നാരെയും നിങ്ങള്‍ ആത്മമിത്രങ്ങളാക്കരുത്. അവരതിന് വിസമ്മതിക്കുകയാണെങ്കില്‍ അവരെ കണ്ടേടത്തുവെച്ച് പിടികൂടുകയും വധിക്കുകയും ചെയ്യുക. അവരില്‍ നിന്നാരെയും നിങ്ങള്‍ ആത്മമിത്രമോ സഹായിയോ ആക്കരുത്.
\end{malayalam}}
\flushright{\begin{Arabic}
\quranayah[4][90]
\end{Arabic}}
\flushleft{\begin{malayalam}
എന്നാല്‍ നിങ്ങളുമായി സഖ്യത്തിലുള്ള ജനതയോടൊപ്പം ചേരുന്ന കപടവിശ്വാസികള്‍ ഇതില്‍ നിന്നൊഴിവാണ്. നിങ്ങളോടു യുദ്ധം ചെയ്യാനോ സ്വന്തം ജനത്തോടേറ്റുമുട്ടാനോ ഇഷ്ടപ്പെടാതെ മനഃക്ളേശത്തോടെ നിങ്ങളെ സമീപിക്കുന്നവരും അവരില്‍പ്പെടുകയില്ല. അല്ലാഹു ഇച്ഛിച്ചിരുന്നെങ്കില്‍ അവന്‍ നിങ്ങള്‍ക്കെതിരില്‍ അവര്‍ക്ക് കരുത്തുനല്‍കുകയും അങ്ങനെ അവര്‍ നിങ്ങളോട് യുദ്ധത്തിലേര്‍പ്പെടുകയും ചെയ്യുമായിരുന്നു. അവര്‍ നിങ്ങളോട് യുദ്ധത്തിലേര്‍പ്പെടാതെ മാറിനില്‍ക്കുകയും നിങ്ങളുടെ മുന്നില്‍ സമാധാന നിര്‍ദേശം സമര്‍പ്പിക്കുകയും ചെയ്തിട്ടുണ്ടെങ്കില്‍ പിന്നെ, അവര്‍ക്കെതിരെ ഒരു നടപടിക്കും അല്ലാഹു നിങ്ങള്‍ക്ക് അനുമതി നല്‍കുന്നില്ല.
\end{malayalam}}
\flushright{\begin{Arabic}
\quranayah[4][91]
\end{Arabic}}
\flushleft{\begin{malayalam}
വേറൊരു വിഭാഗം കപടവിശ്വാസികളെ നിങ്ങള്‍ക്കു കാണാം. അവര്‍ നിങ്ങളില്‍നിന്നും സ്വന്തം ജനതയില്‍നിന്നും സുരക്ഷിതരായി കഴിയാനാഗ്രഹിക്കുന്നു. എന്നാല്‍ കുഴപ്പത്തിനവസരം കിട്ടുമ്പോഴൊക്കെ, അതിലേക്കവര്‍ തലകുത്തിമറിയുന്നു. അതിനാല്‍ നിങ്ങള്‍ക്കെതിരെ തിരിയുന്നതില്‍നിന്ന് ഒഴിഞ്ഞുനില്‍ക്കുകയും നിങ്ങള്‍ക്കു മുന്നില്‍ സമാധാനനിര്‍ദേശം സമര്‍പ്പിക്കുകയും തങ്ങളുടെ കൈകള്‍ അടക്കിവെക്കുകയും ചെയ്യുന്നില്ലെങ്കില്‍ നിങ്ങളവരെ കണ്ടേടത്തുവെച്ച് പിടികൂടി കൊന്നുകളയുക. അവര്‍ക്കെതിരെ നിങ്ങള്‍ക്കു നാം വ്യക്തമായ ന്യായം നല്‍കിയിരിക്കുന്നു.
\end{malayalam}}
\flushright{\begin{Arabic}
\quranayah[4][92]
\end{Arabic}}
\flushleft{\begin{malayalam}
ഒരു വിശ്വാസിയും മറ്റൊരു വിശ്വാസിയെ വധിക്കാവതല്ല. അബദ്ധത്തില്‍ സംഭവിക്കുന്നതൊഴികെ. ആരെങ്കിലും അബദ്ധത്തില്‍ ഒരു വിശ്വാസിയെ വധിച്ചാല്‍ പ്രായശ്ചിത്തമായി വിശ്വാസിയായ ഒരടിമയെ മോചിപ്പിക്കുകയും കൊല്ലപ്പെട്ടവന്റെ അവകാശികള്‍ക്ക് നഷ്ടപരിഹാരം നല്‍കുകയും വേണം. അവര്‍ ഔദാര്യത്തോടെ വിട്ടുവീഴ്ച ചെയ്താലൊഴികെ. വധിക്കപ്പെട്ട സത്യവിശ്വാസി നിങ്ങളുടെ ശത്രുസമൂഹത്തില്‍പ്പെട്ടവനാണെങ്കില്‍ വിശ്വാസിയായ ഒരടിമയെ മോചിപ്പിക്കുക. എന്നാല്‍ കൊല്ലപ്പെട്ടവന്‍ നിങ്ങളുമായി സഖ്യത്തിലുള്ളവരില്‍പ്പെട്ടവനാണെങ്കില്‍ അയാളുടെ അവകാശികള്‍ക്ക് നഷ്ടപരിഹാരം നല്‍കുകയും വിശ്വാസിയായ ഒരടിമയെ മോചിപ്പിക്കുകയും വേണം. ആര്‍ക്കെങ്കിലും അതിനു സാധ്യമല്ലെങ്കില്‍ അവന്‍ തുടര്‍ച്ചയായി രണ്ടു മാസം നോമ്പനുഷ്ഠിക്കേണ്ടതാണ്. അല്ലാഹു നിശ്ചയിച്ച പ്രായശ്ചിത്തമാണിത്. അല്ലാഹു എല്ലാം അറിയുന്നവനും യുക്തിമാനുമാണ്.
\end{malayalam}}
\flushright{\begin{Arabic}
\quranayah[4][93]
\end{Arabic}}
\flushleft{\begin{malayalam}
എന്നാല്‍ ബോധപൂര്‍വം ഒരു വിശ്വാസിയെ കൊന്നവനുള്ള പ്രതിഫലം നരകമാണ്. അവനവിടെ സ്ഥിരവാസിയായിരിക്കും. അല്ലാഹുവിന്റെ കോപവും ശാപവും അവനില്‍ പതിച്ചുകഴിഞ്ഞിരിക്കുന്നു. കൊടിയ ശിക്ഷയാണ് അല്ലാഹു അവന്നായി ഒരുക്കിവെച്ചിരിക്കുന്നത്.
\end{malayalam}}
\flushright{\begin{Arabic}
\quranayah[4][94]
\end{Arabic}}
\flushleft{\begin{malayalam}
വിശ്വസിച്ചവരേ, നിങ്ങള്‍ അല്ലാഹുവിന്റെ മാര്‍ഗത്തില്‍ യുദ്ധത്തിനിറങ്ങിയാല്‍ ശത്രുക്കളെയും മിത്രങ്ങളെയും വേര്‍തിരിച്ചറിയണം. ആരെങ്കിലും നിങ്ങള്‍ക്ക് സലാം ചൊല്ലിയാല്‍ ഐഹികനേട്ടമാഗ്രഹിച്ച് “നീ വിശ്വാസിയല്ലെ”ന്ന് അയാളോടു പറയരുത്. അല്ലാഹുവിങ്കല്‍ സമരാര്‍ജിത സമ്പത്ത് ധാരാളമുണ്ട്. നേരത്തെ നിങ്ങളും അവരിപ്പോഴുള്ള അതേ അവസ്ഥ യിലായിരുന്നല്ലോ. പിന്നെ അല്ലാഹു നിങ്ങളോട് ഔദാര്യം കാണിച്ചു. അതിനാല്‍ കാര്യങ്ങള്‍ വ്യക്തമായി മനസ്സിലാക്കുക. അല്ലാഹു നിങ്ങള്‍ ചെയ്യുന്നതെല്ലാം സൂക്ഷ്മമായി അറിയുന്നവനാണ്.
\end{malayalam}}
\flushright{\begin{Arabic}
\quranayah[4][95]
\end{Arabic}}
\flushleft{\begin{malayalam}
ന്യായമായ കാരണമില്ലാതെ വീട്ടിലിരിക്കുന്ന വിശ്വാസികളും, തങ്ങളുടെ സമ്പത്തും ശരീരവുമുപയോഗിച്ച് ദൈവമാര്‍ഗത്തില്‍ സമരം ചെയ്യുന്നവരും ഒരുപോലെയല്ല. സമ്പത്തുകൊണ്ടും ശരീരം കൊണ്ടും അല്ലാഹുവിന്റെ മാര്‍ഗത്തില്‍ സമരം ചെയ്യുന്നവരെ അല്ലാഹു വെറുതെയിരിക്കുന്നവരെക്കാള്‍ ഏറെ ഉയര്‍ന്ന പദവിയിലാക്കിയിരിക്കുന്നു. എല്ലാവര്‍ക്കും അല്ലാഹു മെച്ചപ്പെട്ട പ്രതിഫലം വാഗ്ദാനം ചെയ്തിട്ടുണ്ട്. എന്നാല്‍ അല്ലാഹു പോരാളികള്‍ക്ക് മഹത്തായ പ്രതിഫലത്താല്‍ ചടഞ്ഞിരിക്കുന്നവരെക്കാള്‍ ശ്രേഷ്ഠത നല്‍കിയിരിക്കുന്നു.
\end{malayalam}}
\flushright{\begin{Arabic}
\quranayah[4][96]
\end{Arabic}}
\flushleft{\begin{malayalam}
അല്ലാഹുവിങ്കല്‍ നിന്നുള്ള ഉന്നത പദവികളും പാപമോചനവും എല്ലാവിധ അനുഗ്രഹങ്ങളും അവര്‍ക്കുണ്ട്. അല്ലാഹു ഏറെ പൊറുക്കുന്നവനും ദയാപരനുമാണ്.
\end{malayalam}}
\flushright{\begin{Arabic}
\quranayah[4][97]
\end{Arabic}}
\flushleft{\begin{malayalam}
സ്വന്തത്തോട് അതിക്രമം പ്രവര്‍ത്തിച്ചവരെ മരിപ്പിക്കുമ്പോള്‍ മലക്കുകള്‍ അവരോട് ചോദിക്കും: "നിങ്ങള്‍ ഏതവസ്ഥയിലാണുണ്ടായിരുന്നത്?” അവര്‍ പറയും: "ഭൂമിയില്‍ ഞങ്ങള്‍ അടിച്ചമര്‍ത്തപ്പെട്ടവരായിരുന്നു.” മലക്കുകള്‍ ചോദിക്കും: "അല്ലാഹുവിന്റെ ഭൂമി വിശാലമായിരുന്നില്ലേ? നിങ്ങള്‍ക്ക് നാടുവിട്ടെവിടെയെങ്കിലും രക്ഷപ്പെടാമായിരുന്നില്ലേ?” അവരുടെ താവളം നരകമാണ്. അതെത്ര ചീത്ത സങ്കേതം!
\end{malayalam}}
\flushright{\begin{Arabic}
\quranayah[4][98]
\end{Arabic}}
\flushleft{\begin{malayalam}
എന്നാല്‍ യഥാര്‍ഥത്തില്‍ തന്നെ എന്തെങ്കിലും തന്ത്രമോ രക്ഷാമാര്‍ഗമോ കണ്ടെത്താനാവാതെ അടിച്ചമര്‍ത്തപ്പെട്ടവരായി കഴിയുന്ന പുരുഷന്മാരും സ്ത്രീകളും കുട്ടികളും ഇതില്‍ നിന്നൊഴിവാണ്.
\end{malayalam}}
\flushright{\begin{Arabic}
\quranayah[4][99]
\end{Arabic}}
\flushleft{\begin{malayalam}
അത്തരക്കാര്‍ക്ക് അല്ലാഹു മാപ്പേകിയേക്കാം. അല്ലാഹു ഏറെ മാപ്പേകുന്നവനും പൊറുക്കുന്നവനുമാണല്ലോ.
\end{malayalam}}
\flushright{\begin{Arabic}
\quranayah[4][100]
\end{Arabic}}
\flushleft{\begin{malayalam}
അല്ലാഹുവിന്റെ മാര്‍ഗത്തില്‍ നാടുവെടിയുന്നവന് ഭൂമിയില്‍ ധാരാളം അഭയസ്ഥാനങ്ങളും വിശാലമായ ജീവിത സൌകര്യങ്ങളും കണ്ടെത്താം. വീടുവെടിഞ്ഞ് അല്ലാഹുവിലേക്കും അവന്റെ ദൂതനിലേക്കും അഭയം തേടി പുറപ്പെട്ടവന്‍ വഴിയില്‍വെച്ച് മരണപ്പെടുകയാണെങ്കില്‍ ഉറപ്പായും അവന് അല്ലാഹുവിങ്കല്‍ പ്രതിഫലമുണ്ട്. അല്ലാഹു ഏറെ പൊറുക്കുന്നവനും പരമദയാലുവുമാണ്.
\end{malayalam}}
\flushright{\begin{Arabic}
\quranayah[4][101]
\end{Arabic}}
\flushleft{\begin{malayalam}
നിങ്ങള്‍ ഭൂമിയില്‍ യാത്ര ചെയ്യുമ്പോള്‍ സത്യനിഷേധികള്‍ നിങ്ങളെ അപകടപ്പെടുത്തുമെന്ന് ഭയപ്പെടുന്നുവെങ്കില്‍ നമസ്കാരം ചുരുക്കി നിര്‍വഹിക്കാം. അതില്‍ നിങ്ങള്‍ക്കു കുറ്റമില്ല. സത്യനിഷേധികള്‍ നിങ്ങളുടെ പ്രത്യക്ഷ ശത്രുക്കള്‍ തന്നെ; തീര്‍ച്ച.
\end{malayalam}}
\flushright{\begin{Arabic}
\quranayah[4][102]
\end{Arabic}}
\flushleft{\begin{malayalam}
നീ അവര്‍ക്കിടയിലുണ്ടാവുകയും അവര്‍ക്ക് നമസ്കാരത്തിന് നേതൃത്വം നല്‍കുകയുമാണെങ്കില്‍ അവരിലൊരുകൂട്ടര്‍ നിന്നോടൊപ്പം നില്‍ക്കട്ടെ. അവര്‍ തങ്ങളുടെ ആയുധങ്ങള്‍ എടുക്കുകയും ചെയ്യട്ടെ. അവര്‍ സാഷ്ടാംഗം ചെയ്തുകഴിഞ്ഞാല്‍ പിറകോട്ട് മാറിനില്‍ക്കുകയും നമസ്കരിച്ചിട്ടില്ലാത്ത വിഭാഗം വന്ന് നിന്റെ കൂടെ നമസ്കരിക്കുകയും വേണം. അവരും ജാഗ്രത പുലര്‍ത്തുകയും ആയുധമണിയുകയും ചെയ്യട്ടെ. നിങ്ങള്‍ ആയുധങ്ങളുടെയും സാധനസാമഗ്രികളുടെയും കാര്യത്തില്‍ അല്‍പം അശ്രദ്ധരായാല്‍ നിങ്ങളുടെ മേല്‍ ചാടിവീണ് ഒരൊറ്റ ആഞ്ഞടി നടത്താന്‍ തക്കം പാര്‍ത്തിരിക്കുകയാണ് സത്യനിഷേധികള്‍. മഴ കാരണം ക്ളേശമുണ്ടാവുകയോ രോഗികളാവുകയോ ചെയ്താല്‍ ആയുധം താഴെ വെക്കുന്നതില്‍ നിങ്ങള്‍ക്കു കുറ്റമില്ല. അപ്പോഴും നിങ്ങള്‍ അതീവ ജാഗ്രത പുലര്‍ത്തണം. സംശയമില്ല; അല്ലാഹു സത്യനിഷേധികള്‍ക്ക് നിന്ദ്യമായ ശിക്ഷ ഒരുക്കിവെച്ചിട്ടുണ്ട്.
\end{malayalam}}
\flushright{\begin{Arabic}
\quranayah[4][103]
\end{Arabic}}
\flushleft{\begin{malayalam}
അങ്ങനെ നിങ്ങള്‍ നമസ്കാരം നിര്‍വഹിച്ചുകഴിഞ്ഞാല്‍ പിന്നെ, നിന്നും ഇരുന്നും കിടന്നും അല്ലാഹുവെ ഓര്‍ത്തുകൊണ്ടിരിക്കുക. നിങ്ങള്‍ നിര്‍ഭയാവസ്ഥയിലായാല്‍ നമസ്കാരം തികവോടെ നിര്‍വഹിക്കുക. നമസ്കാരം സത്യവിശ്വാസികള്‍ക്ക് സമയബന്ധിതമായി നിശ്ചയിക്കപ്പെട്ട നിര്‍ബന്ധ ബാധ്യതയാണ്.
\end{malayalam}}
\flushright{\begin{Arabic}
\quranayah[4][104]
\end{Arabic}}
\flushleft{\begin{malayalam}
ശത്രുജനതയെ തേടിപ്പിടിക്കുന്നതില്‍ നിങ്ങള്‍ ഭീരുത്വം കാണിക്കരുത്. നിങ്ങള്‍ വേദന അനുഭവിക്കുന്നുവെങ്കില്‍ നിങ്ങള്‍ വേദന അനുഭവിക്കുന്നപോലെ അവരും വേദന അനുഭവിക്കുന്നുണ്ട്. അതോടൊപ്പം അവര്‍ക്ക് പ്രതീക്ഷിക്കാനില്ലാത്തത് അല്ലാഹുവിങ്കല്‍ നിന്ന് നിങ്ങള്‍ പ്രതീക്ഷിക്കുന്നുമുണ്ട്. അല്ലാഹു എല്ലാം അറിയുന്നവനും യുക്തിമാനുമാണ്.
\end{malayalam}}
\flushright{\begin{Arabic}
\quranayah[4][105]
\end{Arabic}}
\flushleft{\begin{malayalam}
നാം നിനക്ക് സത്യസന്ദേശവുമായി ഈ വേദപുസ്തകം ഇറക്കിത്തന്നിരിക്കുന്നു. അല്ലാഹു കാണിച്ചുതന്നതനുസരിച്ച് ജനങ്ങള്‍ക്കിടയില്‍ വിധി കല്‍പിക്കാന്‍ വേണ്ടിയാണിത്. നീ വഞ്ചകര്‍ക്കുവേണ്ടി വാദിക്കുന്നവനാകരുത്.
\end{malayalam}}
\flushright{\begin{Arabic}
\quranayah[4][106]
\end{Arabic}}
\flushleft{\begin{malayalam}
അല്ലാഹുവോട് പാപമോചനം തേടുക. അല്ലാഹു ഏറെ പൊറുക്കുന്നവനും ദയാപരനും തന്നെ; തീര്‍ച്ച.
\end{malayalam}}
\flushright{\begin{Arabic}
\quranayah[4][107]
\end{Arabic}}
\flushleft{\begin{malayalam}
ആത്മവഞ്ചന നടത്തുന്നവര്‍ക്കുവേണ്ടി നീ വാദിക്കരുത്. കൊടുംവഞ്ചകനും പെരുംപാപിയുമായ ആരെയും അല്ലാഹു ഇഷ്ടപ്പെടുന്നില്ല.
\end{malayalam}}
\flushright{\begin{Arabic}
\quranayah[4][108]
\end{Arabic}}
\flushleft{\begin{malayalam}
അവര്‍ ജനങ്ങളില്‍നിന്ന് മറച്ചുപിടിക്കുന്നു. എന്നാല്‍ അല്ലാഹുവില്‍നിന്ന് മറച്ചുവെക്കാനവര്‍ക്കാവില്ല. അല്ലാഹുവിന് ഇഷ്ടപ്പെടാത്ത സംസാരത്തിലൂടെ രാത്രിയിലവര്‍ ഗൂഢാലോചന നടത്തിക്കൊണ്ടിരിക്കുമ്പോഴും അവന്‍ അവരോടൊപ്പമുണ്ട്. അവര്‍ ചെയ്യുന്നതൊക്കെ സൂക്ഷ്മമായി അറിയുന്നവനാണ് അല്ലാഹു.
\end{malayalam}}
\flushright{\begin{Arabic}
\quranayah[4][109]
\end{Arabic}}
\flushleft{\begin{malayalam}
ഐഹികജീവിതത്തില്‍ അവര്‍ക്കുവേണ്ടി വാദിക്കാന്‍ നിങ്ങളുണ്ട്. എന്നാല്‍ ഉയിര്‍ത്തെഴുന്നേല്‍പുനാളില്‍ അവര്‍ക്കുവേണ്ടി അല്ലാഹുവോട് തര്‍ക്കിക്കാന്‍ ആരാണുണ്ടാവുക? ആരാണ് അവിടെ അവരുടെ വക്കാലത്ത് ഏറ്റെടുക്കുക?
\end{malayalam}}
\flushright{\begin{Arabic}
\quranayah[4][110]
\end{Arabic}}
\flushleft{\begin{malayalam}
തെറ്റ് ചെയ്യുകയോ തന്നോടുതന്നെ അതിക്രമം കാണിക്കുകയോ ചെയ്തശേഷം അല്ലാഹുവോട് പാപമോചനം തേടുന്നവന്‍, ഏറെ പൊറുക്കുന്നവനും ദയാപരനുമായി അല്ലാഹുവെ കണ്ടെത്തുന്നതാണ്.
\end{malayalam}}
\flushright{\begin{Arabic}
\quranayah[4][111]
\end{Arabic}}
\flushleft{\begin{malayalam}
എന്നാല്‍ തെറ്റുകള്‍ ഒരുക്കൂട്ടിവെക്കുന്നവന്‍ സ്വന്തം നാശത്തിനിടവരുത്തുന്ന സംഗതികളാണ് ശേഖരിച്ചുവെക്കുന്നത്. അല്ലാഹു സര്‍വജ്ഞനും യുക്തിജ്ഞനുമാകുന്നു.
\end{malayalam}}
\flushright{\begin{Arabic}
\quranayah[4][112]
\end{Arabic}}
\flushleft{\begin{malayalam}
ആരെങ്കിലും വല്ല തെറ്റോ കുറ്റമോ ചെയ്തശേഷം അത് നിരപരാധിയുടെ പേരില്‍ ചാര്‍ത്തുന്നുവെങ്കില്‍ ഉറപ്പായും കടുത്ത കള്ളാരോപണവും പ്രകടമായ പാപവുമാണവന്‍ പേറുന്നത്.
\end{malayalam}}
\flushright{\begin{Arabic}
\quranayah[4][113]
\end{Arabic}}
\flushleft{\begin{malayalam}
നിന്റെമേല്‍ അല്ലാഹുവിന്റെ അനുഗ്രഹവും കാരുണ്യവും ഇല്ലായിരുന്നുവെങ്കില്‍ അവരിലൊരു വിഭാഗം നിന്നെ വഴിതെറ്റിക്കുമായിരുന്നു. യഥാര്‍ഥത്തില്‍ അവര്‍ ആരെയും വഴിപിഴപ്പിക്കുന്നില്ല; തങ്ങളെത്തന്നെയല്ലാതെ. നിനക്കൊരു ദ്രോഹവും വരുത്താനവര്‍ക്കാവില്ല. അല്ലാഹു നിനക്ക് വേദപുസ്തകവും തത്ത്വജ്ഞാനവും ഇറക്കിത്തന്നു. നിനക്കറിയാത്തത് നിന്നെ പഠിപ്പിക്കുകയും ചെയ്തു. അല്ലാഹു നിനക്കേകിയ അനുഗ്രഹം അതിമഹത്തരംതന്നെ.
\end{malayalam}}
\flushright{\begin{Arabic}
\quranayah[4][114]
\end{Arabic}}
\flushleft{\begin{malayalam}
അവരുടെ ഗൂഢാലോചനകളിലേറെയും ഒരു നന്മയുമില്ലാത്തവയാണ്. എന്നാല്‍ ദാനധര്‍മത്തിനും സല്‍ക്കാര്യത്തിനും ജനങ്ങള്‍ക്കിടയില്‍ ഒത്തുതീര്‍പ്പുണ്ടാക്കാനും കല്‍പിക്കുന്നവരുടേത് ഇതില്‍പെടുകയില്ല. ആരെങ്കിലും ദൈവപ്രീതി പ്രതീക്ഷിച്ച് അങ്ങനെ ചെയ്യുന്നുവെങ്കില്‍ നാമവന് അളവറ്റ പ്രതിഫലം നല്‍കും.
\end{malayalam}}
\flushright{\begin{Arabic}
\quranayah[4][115]
\end{Arabic}}
\flushleft{\begin{malayalam}
നേര്‍മാര്‍ഗം വ്യക്തമായ ശേഷം ദൈവദൂതനെ എതിര്‍ക്കുകയും സത്യവിശ്വാസികളുടേതല്ലാത്ത പാത പിന്തുടരുകയും ചെയ്യുന്നവനെ നാം അവന്‍ പ്രവേശിച്ച വഴിയിലൂടെ തന്നെ തിരിച്ചുവിടും. അവസാനം നരകത്തീയിലേക്ക് തള്ളുകയും ചെയ്യും. അതെത്ര ചീത്ത താവളം.
\end{malayalam}}
\flushright{\begin{Arabic}
\quranayah[4][116]
\end{Arabic}}
\flushleft{\begin{malayalam}
തന്നില്‍ ആരെയും പങ്കുചേര്‍ക്കുന്നത് അല്ലാഹു പൊറുക്കുകയില്ല. അതൊഴിച്ചുള്ളവയൊക്കെ താനിച്ഛിക്കുന്നവര്‍ക്ക് അവന്‍ പൊറുത്തുകൊടുക്കും. അല്ലാഹുവില്‍ പങ്കുചേര്‍ക്കുന്നവന്‍ വഴികേടില്‍ ഒരുപാട് ദൂരം പിന്നിട്ടിരിക്കുന്നു.
\end{malayalam}}
\flushright{\begin{Arabic}
\quranayah[4][117]
\end{Arabic}}
\flushleft{\begin{malayalam}
അവര്‍ അല്ലാഹുവെ വിട്ട് ചില ദേവതകളെ വിളിച്ചുപ്രാര്‍ഥിക്കുന്നു. സത്യത്തില്‍ അവര്‍ സഹായാര്‍ഥന നടത്തുന്നത് ധിക്കാരിയായ പിശാചിനോടല്ലാതാരോടുമല്ല.
\end{malayalam}}
\flushright{\begin{Arabic}
\quranayah[4][118]
\end{Arabic}}
\flushleft{\begin{malayalam}
അവനെ അല്ലാഹു ശപിച്ചിരിക്കുന്നു. അവന്‍ അല്ലാഹുവോട്പറഞ്ഞിട്ടുണ്ടായിരുന്നു: "നിന്റെ ദാസന്മാരില്‍ ഒരു വിഭാഗത്തെ ഞാന്‍ എന്റേതാക്കി മാറ്റും.
\end{malayalam}}
\flushright{\begin{Arabic}
\quranayah[4][119]
\end{Arabic}}
\flushleft{\begin{malayalam}
"അവരെ ഞാന്‍ വഴിപിഴപ്പിക്കും. വ്യാമോഹങ്ങള്‍ക്കടിപ്പെടുത്തും. ഞാന്‍ കല്‍പിക്കുന്നതിനനുസരിച്ച് അവര്‍ കാലികളുടെ കാത് കീറിമുറിക്കും. അവര്‍ അല്ലാഹുവിന്റ സൃഷ്ടിയെ കോലംകെടുത്തും.” അല്ലാഹുവെ വിട്ട് പിശാചിനെ രക്ഷകനാക്കുന്നവന്‍ പ്രകടമായ നഷ്ടത്തിലകപ്പെട്ടതു തന്നെ; തീര്‍ച്ച.
\end{malayalam}}
\flushright{\begin{Arabic}
\quranayah[4][120]
\end{Arabic}}
\flushleft{\begin{malayalam}
പിശാച് അവര്‍ക്ക് വാഗ്ദാനം നല്‍കും. അങ്ങനെ അവരെ വ്യാമോഹിപ്പിക്കും. പിശാച് അവര്‍ക്ക് നല്‍കുന്ന വാഗ്ദാനം കൊടുംചതിയല്ലാതൊന്നുമല്ല.
\end{malayalam}}
\flushright{\begin{Arabic}
\quranayah[4][121]
\end{Arabic}}
\flushleft{\begin{malayalam}
അക്കൂട്ടരുടെ താവളം നരകമാണ്. അതില്‍നിന്നൊരു രക്ഷാമാര്‍ഗവും കണ്ടെത്താന്‍ അവര്‍ക്കാവില്ല.
\end{malayalam}}
\flushright{\begin{Arabic}
\quranayah[4][122]
\end{Arabic}}
\flushleft{\begin{malayalam}
എന്നാല്‍ സത്യവിശ്വാസം സ്വീകരിക്കുകയും സല്‍ക്കര്‍മങ്ങള്‍ പ്രവര്‍ത്തിക്കുകയും ചെയ്തവരെ താഴ്ഭാഗത്തൂടെ ആറുകളൊഴുകുന്ന ആരാമങ്ങളില്‍ നാം പ്രവേശിപ്പിക്കും. അവരതില്‍ സ്ഥിരവാസികളായിരിക്കും. അല്ലാഹുവിന്റെ സത്യമായ വാഗ്ദാനമാണിത്. അല്ലാഹുവെക്കാള്‍ സത്യനിഷ്ഠമായി വാഗ്ദാനം നല്‍കുന്ന ആരുണ്ട്?
\end{malayalam}}
\flushright{\begin{Arabic}
\quranayah[4][123]
\end{Arabic}}
\flushleft{\begin{malayalam}
കാര്യം നടക്കുന്നത് നിങ്ങളുടെ വ്യാമോഹങ്ങള്‍ക്കനുസരിച്ചല്ല. വേദക്കാരുടെ വ്യാമോഹങ്ങള്‍ക്കൊത്തുമല്ല. തിന്മ ചെയ്യുന്നതാരായാലും അതിന്റെ ഫലം അവന് ലഭിക്കും. അല്ലാഹുവെക്കൂടാതെ ഒരു രക്ഷകനെയും സഹായിയെയും അവന് കണ്ടെത്താനാവില്ല.
\end{malayalam}}
\flushright{\begin{Arabic}
\quranayah[4][124]
\end{Arabic}}
\flushleft{\begin{malayalam}
ആണായാലും പെണ്ണായാലും സത്യവിശ്വാസിയായി സല്‍ക്കര്‍മങ്ങള്‍ ചെയ്യുന്നവര്‍ സ്വര്‍ഗത്തില്‍ പ്രവേശിക്കും. അവരോടൊട്ടും അനീതിയുണ്ടാവില്ല.
\end{malayalam}}
\flushright{\begin{Arabic}
\quranayah[4][125]
\end{Arabic}}
\flushleft{\begin{malayalam}
സല്‍ക്കര്‍മിയായി സ്വന്തത്തെ അല്ലാഹുവിന് സമര്‍പ്പിക്കുകയും നേര്‍മാര്‍ഗത്തിലുറച്ചുനിന്ന് ഇബ്റാഹീമിന്റെ പാത പിന്തുടരുകയും ചെയ്തവനേക്കാള്‍ ഉത്തമമായ ജീവിതരീതി സ്വീകരിച്ച ആരുണ്ട്? ഇബ്റാഹീമിനെ അല്ലാഹു തന്റെ സുഹൃത്തായി സ്വീകരിച്ചിരിക്കുന്നു.
\end{malayalam}}
\flushright{\begin{Arabic}
\quranayah[4][126]
\end{Arabic}}
\flushleft{\begin{malayalam}
ആകാശഭൂമികളിലുള്ളതൊക്കെയും അല്ലാഹുവിന്റേതാണ്. എല്ലാ കാര്യങ്ങളെപ്പറ്റിയും സൂക്ഷ്മമായറിയുന്നവനാണ് അല്ലാഹു.
\end{malayalam}}
\flushright{\begin{Arabic}
\quranayah[4][127]
\end{Arabic}}
\flushleft{\begin{malayalam}
സ്ത്രീകളുടെ കാര്യത്തില്‍ അവര്‍ നിന്നോട് വിധി തേടുന്നു. പറയുക: അവരുടെ കാര്യത്തില്‍ അല്ലാഹു നിങ്ങള്‍ക്ക് വിധി നല്‍കുന്നു. ഈ വേദപുസ്തകത്തില്‍ നേരത്തെ നിങ്ങളെ ഓതിക്കേള്‍പ്പിച്ച വിധികള്‍ ഓര്‍മിപ്പിക്കുകയും ചെയ്യുന്നു. നിശ്ചയിക്കപ്പെട്ട അവകാശം നല്‍കാതെ നിങ്ങള്‍ അനാഥസ്ത്രീകളെ വിവാഹം കഴിക്കാനാഗ്രഹിക്കുന്നതിനെ സംബന്ധിച്ചും ദുര്‍ബലരായ കുട്ടികളെക്കുറിച്ചുമുള്ള വിധിയും അനാഥകളോട് നിങ്ങള്‍ നീതിയോടെ വര്‍ത്തിക്കണമെന്ന കല്‍പനയും അതുള്‍ക്കൊള്ളുന്നു. നിങ്ങള്‍ ചെയ്യുന്ന നല്ല കാര്യങ്ങളെക്കുറിച്ചെല്ലാം നന്നായറിയുന്നവനാണ് അല്ലാഹു.
\end{malayalam}}
\flushright{\begin{Arabic}
\quranayah[4][128]
\end{Arabic}}
\flushleft{\begin{malayalam}
ഏതെങ്കിലും സ്ത്രീ തന്റെ ഭര്‍ത്താവില്‍ നിന്ന് പിണക്കമോ അവഗണനയോ ഭയപ്പെട്ടാല്‍ അവരന്യോന്യം ഒത്തുതീര്‍പ്പുണ്ടാക്കുന്നതില്‍ കുറ്റമില്ല. എന്നല്ല; ഒത്തുതീര്‍പ്പാണ് ഉത്തമം. മനുഷ്യമനസ്സ് എപ്പോഴും സങ്കുചിതമായിരിക്കും. നിങ്ങള്‍ നല്ലനിലയില്‍ കഴിയുകയും സൂക്ഷ്മത പുലര്‍ത്തുകയുമാണെങ്കില്‍, ഓര്‍ക്കുക: അല്ലാഹു നിങ്ങള്‍ ചെയ്യുന്നവയൊക്കെയും സൂക്ഷ്മമായി അറിയുന്നവനാണ്.
\end{malayalam}}
\flushright{\begin{Arabic}
\quranayah[4][129]
\end{Arabic}}
\flushleft{\begin{malayalam}
നിങ്ങളെത്ര ആഗ്രഹിച്ചാലും ഭാര്യമാര്‍ക്കിടയില്‍ തുല്യനീതി പാലിക്കാനാവില്ല. അതിനാല്‍ നിങ്ങള്‍ ഒരുവളിലേക്ക് പൂര്‍ണമായി ചാഞ്ഞ് മറ്റവളെ കെട്ടിയിടപ്പെട്ടവളായി കയ്യൊഴിക്കരുത്. നിങ്ങള്‍ ഭാര്യമാരോട് നന്നായി വര്‍ത്തിക്കുക. സൂക്ഷ്മത പാലിക്കുകയും ചെയ്യുക. എങ്കില്‍ അല്ലാഹു ഏറെ പൊറുക്കുന്നവനും പരമകാരുണികനുമാകുന്നു.
\end{malayalam}}
\flushright{\begin{Arabic}
\quranayah[4][130]
\end{Arabic}}
\flushleft{\begin{malayalam}
അഥവാ, അവരിരുവരും വേര്‍പിരിയുകയാണെങ്കില്‍ അല്ലാഹു തന്റെ അതിരില്ലാത്ത അനുഗ്രഹത്താല്‍ ഇരുവരെയും സ്വന്തം കാലില്‍ നില്‍ക്കാന്‍ കെല്‍പുറ്റവരാക്കും. അല്ലാഹു ഏറെ വിശാലതയുള്ളവനും യുക്തിമാനുമാകുന്നു.
\end{malayalam}}
\flushright{\begin{Arabic}
\quranayah[4][131]
\end{Arabic}}
\flushleft{\begin{malayalam}
ആകാശ ഭൂമികളിലുള്ളതെല്ലാം അല്ലാഹുവിന്റേതാണ്. അല്ലാഹുവെ സൂക്ഷിച്ചു ജീവിക്കണമെന്ന് നിങ്ങള്‍ക്കുമുമ്പെ വേദം നല്‍കപ്പെട്ടവരോടും നിങ്ങളോടും നാം ഉപദേശിച്ചിട്ടുണ്ട്. എന്നിട്ടും നിങ്ങള്‍ വിശ്വസിക്കുന്നില്ലെങ്കില്‍ വേണ്ട. എന്തെന്നാല്‍ ആകാശഭൂമികളിലുള്ളതൊക്കെയും അല്ലാഹുവിന്റേതാണ്. അല്ലാഹു അന്യാശ്രയമാവശ്യമില്ലാത്തവനാണ്. സ്തുത്യര്‍ഹനും.
\end{malayalam}}
\flushright{\begin{Arabic}
\quranayah[4][132]
\end{Arabic}}
\flushleft{\begin{malayalam}
ആകാശങ്ങളിലും ഭൂമിയിലുമുള്ളതെല്ലാം അല്ലാഹുവിന്റേതാണ്. കാര്യനിര്‍വഹണത്തിനു അല്ലാഹുതന്നെ മതി.
\end{malayalam}}
\flushright{\begin{Arabic}
\quranayah[4][133]
\end{Arabic}}
\flushleft{\begin{malayalam}
ജനങ്ങളേ, അല്ലാഹു ഇഛിക്കുകയാണെങ്കില്‍ അവന്‍ നിങ്ങളെ ഇല്ലാതാക്കും. പകരം മറ്റൊരു കൂട്ടരെ കൊണ്ടുവരും. അല്ലാഹു ഇതിനൊക്കെ കഴിവുറ്റവനാണ്.
\end{malayalam}}
\flushright{\begin{Arabic}
\quranayah[4][134]
\end{Arabic}}
\flushleft{\begin{malayalam}
ഇഹലോകത്തിലെ പ്രതിഫലമാഗ്രഹിക്കുന്നവര്‍ ഓര്‍ക്കുക: ഇഹലോകത്തെ പ്രതിഫലവും പരലോകത്തെ പ്രതിഫലവും അല്ലാഹുവിന്റെ അടുക്കലാണ്. അല്ലാഹു എല്ലാം കേള്‍ക്കുന്നവനും കാണുന്നവനുമാണ്.
\end{malayalam}}
\flushright{\begin{Arabic}
\quranayah[4][135]
\end{Arabic}}
\flushleft{\begin{malayalam}
വിശ്വസിച്ചവരേ, നിങ്ങള്‍ നീതി നടത്തി അല്ലാഹുവിനുവേണ്ടി സാക്ഷ്യം വഹിക്കുന്നവരാവുക. അത് നിങ്ങള്‍ക്കോ നിങ്ങളുടെ മാതാപിതാക്കള്‍ക്കോ അടുത്ത ബന്ധുക്കള്‍ക്കോ എതിരായിരുന്നാലും. കക്ഷി ധനികനോ ദരിദ്രനോ എന്നു നോക്കേണ്ടതില്ല. ഇരുകൂട്ടരോടും കൂടുതല്‍ അടുപ്പമുള്ളവന്‍ അല്ലാഹുവാണ്. അതിനാല്‍ നിങ്ങള്‍ സ്വന്തം ഇഷ്ടാനിഷ്ടങ്ങളുടെ പേരില്‍ നീതി നടത്താതിരിക്കരുത്. വസ്തുതകള്‍ വളച്ചൊടിക്കുകയോ സത്യത്തില്‍നിന്ന് തെന്നിമാറുകയോ ചെയ്യുകയാണെങ്കില്‍ അറിയുക. തീര്‍ച്ചയായും നിങ്ങള്‍ ചെയ്യുന്നതിനെപ്പറ്റിയെല്ലാം സൂക്ഷ്മമായി അറിയുന്നവനാണ് അല്ലാഹു.
\end{malayalam}}
\flushright{\begin{Arabic}
\quranayah[4][136]
\end{Arabic}}
\flushleft{\begin{malayalam}
വിശ്വസിച്ചവരേ, അല്ലാഹു, അവന്റെ ദൂതന്‍, തന്റെ ദൂതന് അവനവതരിപ്പിച്ച വേദപുസ്തകം, അതിനുമുമ്പ് അവനവതരിപ്പിച്ച വേദപുസ്തകം; എല്ലാറ്റിലും നിങ്ങള്‍ വിശ്വസിക്കുക. അല്ലാഹുവിലും അവന്റെ മലക്കുകളിലും വേദങ്ങളിലും ദൂതന്മാരിലും അന്ത്യദിനത്തിലും വിശ്വസിക്കാത്തവര്‍ ഉറപ്പായും ദുര്‍മാര്‍ഗത്തില്‍ ഏറെദൂരം പിന്നിട്ടിരിക്കുന്നു.
\end{malayalam}}
\flushright{\begin{Arabic}
\quranayah[4][137]
\end{Arabic}}
\flushleft{\begin{malayalam}
വിശ്വസിക്കുക, പിന്നെ അവിശ്വസിക്കുക, വീണ്ടും വിശ്വസിക്കുക, പിന്നെയും അവിശ്വസിക്കുക, പിന്നെ അവിശ്വാസം വര്‍ധിപ്പിച്ചുകൊണ്ടിരിക്കുക; ഇങ്ങനെ ചെയ്തവര്‍ക്ക് അല്ലാഹു ഒരിക്കലും മാപ്പേകുകയില്ല. അവരെ അവന്‍ നേര്‍വഴിയിലാക്കുകയുമില്ല.
\end{malayalam}}
\flushright{\begin{Arabic}
\quranayah[4][138]
\end{Arabic}}
\flushleft{\begin{malayalam}
കപടവിശ്വാസികള്‍ക്ക് നോവേറിയ ശിക്ഷയുണ്ടെന്ന് അവരെ “സുവാര്‍ത്ത” അറിയിക്കുക.
\end{malayalam}}
\flushright{\begin{Arabic}
\quranayah[4][139]
\end{Arabic}}
\flushleft{\begin{malayalam}
സത്യവിശ്വാസികളെ വെടിഞ്ഞ് സത്യനിഷേധികളെ ആത്മമിത്രങ്ങളായി സ്വീകരിക്കുന്നവരാണവര്‍. സത്യനിഷേധികളുടെ അടുത്തുചെന്ന് അന്തസ്സ് നേടിയെടുക്കാമെന്ന് അവര്‍ കരുതുന്നുവോ? എന്നാല്‍ അറിയുക: അന്തസ്സൊക്കെയും അല്ലാഹുവിന്റെ അധീനതയിലാണ്.
\end{malayalam}}
\flushright{\begin{Arabic}
\quranayah[4][140]
\end{Arabic}}
\flushleft{\begin{malayalam}
അല്ലാഹുവിന്റെ വചനങ്ങളെ നിഷേധിക്കുന്നതും നിന്ദിക്കുന്നതും നിങ്ങള്‍ കേള്‍ക്കുകയാണെങ്കില്‍ അങ്ങനെ ചെയ്യുന്നവര്‍ മറ്റു വര്‍ത്തമാനങ്ങളില്‍ ഏര്‍പ്പെടും വരെ അവരോടൊപ്പം ഇരിക്കരുതെന്ന് ഈ വേദപുസ്തകത്തില്‍ നാം നിങ്ങളോടു നിര്‍ദേശിച്ചതാണല്ലോ. അങ്ങനെ ചെയ്താല്‍ നിങ്ങളും അവരെപ്പോലെയാകും. അല്ലാഹു കപടവിശ്വാസികളെയും സത്യനിഷേധികളെയും ഒന്നാകെ നരകത്തില്‍ ഒരുക്കൂട്ടുക തന്നെ ചെയ്യും; തീര്‍ച്ച.
\end{malayalam}}
\flushright{\begin{Arabic}
\quranayah[4][141]
\end{Arabic}}
\flushleft{\begin{malayalam}
ആ കപടവിശ്വാസികള്‍ നിങ്ങളെ സദാ ഉറ്റുനോക്കിക്കൊണ്ടിരിക്കുകയാണ്. നിങ്ങള്‍ക്ക് അല്ലാഹുവില്‍നിന്ന് വല്ല വിജയവുമുണ്ടാവുകയാണെങ്കില്‍ അവര്‍ ചോദിക്കും: "ഞങ്ങളും നിങ്ങളോടൊപ്പമായിരുന്നില്ലേ?” അഥവാ സത്യനിഷേധികള്‍ക്കാണ് നേട്ടമുണ്ടാകുന്നതെങ്കില്‍ അവര്‍ പറയും: "നിങ്ങളെ ജയിച്ചടക്കാന്‍ ഞങ്ങള്‍ക്ക് അവസരം കൈവന്നിരുന്നില്ലേ. എന്നിട്ടും വിശ്വാസികളില്‍നിന്ന് നിങ്ങളെ ഞങ്ങള്‍ രക്ഷിച്ചില്ലേ?” എന്നാല്‍ ഉയിര്‍ത്തെഴുന്നേല്‍പുനാളില്‍ അല്ലാഹു നിങ്ങള്‍ക്കിടയില്‍ തീര്‍പ്പുകല്‍പിക്കും. വിശ്വാസികള്‍ക്കെതിരില്‍ സത്യനിഷേധികള്‍ക്ക് അനുകൂലമായ ഒരു പഴുതും അല്ലാഹു ഒരിക്കലും ഉണ്ടാക്കുകയില്ല.
\end{malayalam}}
\flushright{\begin{Arabic}
\quranayah[4][142]
\end{Arabic}}
\flushleft{\begin{malayalam}
തീര്‍ച്ചയായും ഈ കപടവിശ്വാസികള്‍ അല്ലാഹുവെ വഞ്ചിക്കാന്‍ നോക്കുകയാണ്. യഥാര്‍ഥത്തില്‍ അല്ലാഹു അവരെ സ്വയം വഞ്ചിതരാക്കുകയാണ്. അവര്‍ നമസ്കാരത്തിനു നില്‍ക്കുന്നതുപോലും അലസന്മാരായാണ്. ആളുകളെ കാണിക്കാന്‍ വേണ്ടിയും. അവര്‍ വളരെ കുറച്ചേ അല്ലാഹുവെ ഓര്‍ക്കുന്നുള്ളൂ.
\end{malayalam}}
\flushright{\begin{Arabic}
\quranayah[4][143]
\end{Arabic}}
\flushleft{\begin{malayalam}
ഇവരോടോ അവരോടോ ചേരാതെ രണ്ടുകൂട്ടര്‍ക്കുമിടയില്‍ ചാഞ്ചാടിക്കൊണ്ടിരിക്കുകയാണവര്‍. അല്ലാഹു ആരെ വഴികേടിലാക്കിയോ അവന് വിജയമാര്‍ഗം കണ്ടെത്താന്‍ നിനക്കാവില്ല.
\end{malayalam}}
\flushright{\begin{Arabic}
\quranayah[4][144]
\end{Arabic}}
\flushleft{\begin{malayalam}
വിശ്വസിച്ചവരേ, സത്യവിശ്വാസികളെ ഒഴിവാക്കി സത്യനിഷേധികളെ നിങ്ങള്‍ ഉറ്റചങ്ങാതിമാരാക്കരുത്. നിങ്ങള്‍ക്കെതിരെ നടപടിയെടുക്കാന്‍ അല്ലാഹുവിന് വ്യക്തമായ ന്യായമുണ്ടാക്കിക്കൊടുക്കാന്‍ നിങ്ങളാഗ്രഹിക്കുന്നുവോ?
\end{malayalam}}
\flushright{\begin{Arabic}
\quranayah[4][145]
\end{Arabic}}
\flushleft{\begin{malayalam}
കപടവിശ്വാസികള്‍ നരകത്തിന്റെ അടിത്തട്ടിലാണ്; തീര്‍ച്ച. അവര്‍ക്ക് ഒരു സഹായിയെയും കണ്ടെത്താന്‍ നിനക്കാവില്ല.
\end{malayalam}}
\flushright{\begin{Arabic}
\quranayah[4][146]
\end{Arabic}}
\flushleft{\begin{malayalam}
പശ്ചാത്തപിക്കുകയും സ്വന്തത്തെ സംസ്കരിക്കുകയും അല്ലാഹുവെ മുറുകെപ്പിടിക്കുകയും സ്വന്തത്തെ അല്ലാഹുവിന് മാത്രമായി സമര്‍പ്പിക്കുകയും ചെയ്തവരൊഴികെ. അവര്‍ സത്യവിശ്വാസികളോടൊപ്പമാണ്. തീര്‍ച്ചയായും അല്ലാഹു സത്യവിശ്വാസികള്‍ക്ക് അതിമഹത്തായ പ്രതിഫലം നല്‍കും.
\end{malayalam}}
\flushright{\begin{Arabic}
\quranayah[4][147]
\end{Arabic}}
\flushleft{\begin{malayalam}
നിങ്ങള്‍ നന്ദി കാണിക്കുകയും വിശ്വസിക്കുകയുമാണെങ്കില്‍ പിന്നെ നിങ്ങളെ ശിക്ഷിച്ചിട്ട് അല്ലാഹുവിന് എന്തുകിട്ടാനാണ്? അല്ലാഹു അളവറ്റ നന്ദിയുള്ളവനാണ്. എല്ലാം നന്നായറിയുന്നവനും.
\end{malayalam}}
\flushright{\begin{Arabic}
\quranayah[4][148]
\end{Arabic}}
\flushleft{\begin{malayalam}
ചീത്ത വാക്ക് പരസ്യപ്പെടുത്തുന്നത് അല്ലാഹുവിനിഷ്ടമില്ല.Bഅനീതിക്കിരയായവനൊഴികെ. അല്ലാഹു എല്ലാം കേള്‍ക്കുന്നവനും അറിയുന്നവനുമാകുന്നു.
\end{malayalam}}
\flushright{\begin{Arabic}
\quranayah[4][149]
\end{Arabic}}
\flushleft{\begin{malayalam}
നിങ്ങള്‍ പരസ്യമായും രഹസ്യമായും നന്മ ചെയ്യുകയും തെറ്റുകള്‍ പൊറുത്തുകൊടുക്കുകയും ചെയ്യുന്നുവെങ്കില്‍ അറിയുക: അല്ലാഹു ഏറെ പൊറുക്കുന്നവനാണ്. എല്ലാറ്റിനും കഴിവുറ്റവനും.
\end{malayalam}}
\flushright{\begin{Arabic}
\quranayah[4][150]
\end{Arabic}}
\flushleft{\begin{malayalam}
അല്ലാഹുവെയും അവന്റെ ദൂതന്മാരെയും തള്ളിപ്പറയുന്നവരും, അല്ലാഹുവിനും അവന്റെ ദൂതന്മാര്‍ക്കുമിടയില്‍ വിവേചനം കല്‍പിക്കാനാഗ്രഹിക്കുന്നവരും, “ഞങ്ങള്‍ ചിലരെ വിശ്വസിക്കുകയും ചിലരെ നിഷേധിക്കുകയും ചെയ്യുന്നു” വെന്ന് പറയുന്നവരും, വിശ്വാസത്തിനും നിഷേധത്തിനുമിടയില്‍ മറ്റൊരു മാര്‍ഗം സ്വീകരിക്കാനാഗ്രഹിക്കുന്നവരുമുണ്ടല്ലോ-
\end{malayalam}}
\flushright{\begin{Arabic}
\quranayah[4][151]
\end{Arabic}}
\flushleft{\begin{malayalam}
അറിയുക: അവര്‍ തന്നെയാണ് യഥാര്‍ഥ സത്യനിഷേധികള്‍. അത്തരം സത്യനിഷേധികള്‍ക്കു നാം നിന്ദ്യമായ ശിക്ഷ ഒരുക്കിവെച്ചിട്ടുണ്ട്.
\end{malayalam}}
\flushright{\begin{Arabic}
\quranayah[4][152]
\end{Arabic}}
\flushleft{\begin{malayalam}
എന്നാല്‍ അല്ലാഹുവിലും അവന്റെ ദൂതന്മാരിലും വിശ്വസിക്കുകയും അവരില്‍ ആര്‍ക്കിടയിലും ഒരുവിധ വിവേചനവും കാണിക്കാതിരിക്കുകയും ചെയ്തവര്‍ക്ക് അല്ലാഹു അതിനൊത്ത പ്രതിഫലം നല്‍കും. അല്ലാഹു ഏറെ പൊറുക്കുന്നവനാണ്; പരമദയാലുവും.
\end{malayalam}}
\flushright{\begin{Arabic}
\quranayah[4][153]
\end{Arabic}}
\flushleft{\begin{malayalam}
വേദക്കാര്‍ നിന്നോടാവശ്യപ്പെടുന്നു: അവര്‍ക്ക് വാനലോകത്തുനിന്ന് നീയൊരു ഗ്രന്ഥം ഇറക്കിക്കൊടുക്കണമെന്ന്. ഇതിനെക്കാള്‍ ഗുരുതരമായ ഒരാവശ്യം അവര്‍ മൂസായോട് ഉന്നയിച്ചിട്ടുണ്ട്. അവര്‍ പറഞ്ഞു: "നീ ഞങ്ങള്‍ക്ക് അല്ലാഹുവെ നേരില്‍ കാണിച്ചുതരണം.” അവരുടെ ഈ ധിക്കാരം കാരണം പെട്ടെന്നൊരു ഇടിനാദം അവരെ പിടികൂടി. പിന്നെ, വ്യക്തമായ തെളിവുകള്‍ വന്നെത്തിയിട്ടും അവര്‍ പശുക്കുട്ടിയെ ഉണ്ടാക്കി ദൈവമാക്കി. അതും നാം പൊറുത്തുകൊടുത്തു. തുടര്‍ന്ന് മൂസാക്കു നാം വ്യക്തമായ ന്യായപ്രമാണം നല്‍കുകയും ചെയ്തു.
\end{malayalam}}
\flushright{\begin{Arabic}
\quranayah[4][154]
\end{Arabic}}
\flushleft{\begin{malayalam}
അവരോട് ഉറപ്പ് വാങ്ങാനായി സീനാ പര്‍വതത്തെ നാം അവര്‍ക്കുമീതെ ഉയര്‍ത്തിക്കാട്ടി. നഗരകവാടം കടക്കുന്നത് പ്രണാമമര്‍പ്പിച്ചുകൊണ്ടാവണമെന്ന് നാമവരോട് കല്‍പിച്ചു. സാബത്ത് നാളില്‍ അതിക്രമം കാട്ടരുതെന്നും. അക്കാര്യത്തില്‍ നാം അവരോട് സുദൃഢമായ കരാര്‍ വാങ്ങുകയും ചെയ്തു.
\end{malayalam}}
\flushright{\begin{Arabic}
\quranayah[4][155]
\end{Arabic}}
\flushleft{\begin{malayalam}
എന്നിട്ടും അവര്‍ കരാര്‍ ലംഘിച്ചു. ദൈവിക വചനങ്ങളെ ധിക്കരിച്ചു. അന്യായമായി പ്രവാചകന്മാരെ കൊന്നു. തങ്ങളുടെ ഹൃദയങ്ങള്‍ മൂടിക്കുള്ളില്‍ ഭദ്രമാണെന്ന് വീമ്പുപറഞ്ഞു. അങ്ങനെ അവരുടെ നിഷേധഫലമായി അല്ലാഹു അവരുടെ മനസ്സുകള്‍ അടച്ചുപൂട്ടി മുദ്രവെച്ചു. അതിനാല്‍ അവര്‍ വളരെ കുറച്ചേ വിശ്വസിക്കുന്നുള്ളൂ.
\end{malayalam}}
\flushright{\begin{Arabic}
\quranayah[4][156]
\end{Arabic}}
\flushleft{\begin{malayalam}
അവരുടെ സത്യനിഷേധം കാരണമായും മര്‍യമിന്റെ പേരില്‍ ഗുരുതരമായ അപവാദം പറഞ്ഞതിനാലുമാണത്.
\end{malayalam}}
\flushright{\begin{Arabic}
\quranayah[4][157]
\end{Arabic}}
\flushleft{\begin{malayalam}
ദൈവദൂതനായ, മര്‍യമിന്റെ മകന്‍ മസീഹ് ഈസായെ ഞങ്ങള്‍ കൊന്നിരിക്കുന്നുവെന്ന് വാദിച്ചതിനാലും. സത്യത്തിലവര്‍ അദ്ദേഹത്തെ കൊന്നിട്ടില്ല. ക്രൂശിച്ചിട്ടുമില്ല. അവര്‍ ആശയക്കുഴപ്പത്തിലാവുകയാണുണ്ടായത്. അദ്ദേഹത്തിന്റെ കാര്യത്തില്‍ ഭിന്നാഭിപ്രായമുള്ളവര്‍ അതേപ്പറ്റി സംശയത്തില്‍ തന്നെയാണ്. കേവലം ഊഹാപോഹത്തെ പിന്തുടരുന്നതല്ലാതെ അവര്‍ക്ക് അതേപ്പറ്റി ഒന്നുമറിയില്ല. അവരദ്ദേഹത്തെ കൊന്നിട്ടില്ല; ഉറപ്പ്.
\end{malayalam}}
\flushright{\begin{Arabic}
\quranayah[4][158]
\end{Arabic}}
\flushleft{\begin{malayalam}
എന്നാല്‍ അല്ലാഹു അദ്ദേഹത്തെ തന്നിലേക്കുയര്‍ത്തുകയാണുണ്ടായത്. അല്ലാഹു പ്രതാപിയും യുക്തിമാനും തന്നെ.
\end{malayalam}}
\flushright{\begin{Arabic}
\quranayah[4][159]
\end{Arabic}}
\flushleft{\begin{malayalam}
ഈസായുടെ മരണത്തിനു മുമ്പെ അദ്ദേഹത്തില്‍ വിശ്വസിക്കാത്തവരായി വേദക്കാരിലാരുമുണ്ടാവില്ല. ഉയിര്‍ത്തെഴുന്നേല്‍പു നാളിലോ ഉറപ്പായും അദ്ദേഹം അവര്‍ക്കെതിരെ സാക്ഷിയാവുകയും ചെയ്യും.
\end{malayalam}}
\flushright{\begin{Arabic}
\quranayah[4][160]
\end{Arabic}}
\flushleft{\begin{malayalam}
ജൂതമതക്കാരില്‍ നിന്നുണ്ടായ അതിക്രമം കാരണം അവര്‍ക്കനുവദിച്ചിരുന്ന പല നല്ല വസ്തുക്കളും നാമവര്‍ക്ക് നിഷിദ്ധമാക്കി. അല്ലാഹുവിന്റെ മാര്‍ഗത്തില്‍ അവര്‍ ഒട്ടേറെ തടസ്സങ്ങള്‍ സൃഷ്ടിച്ചതിനാലുമാണിത്.
\end{malayalam}}
\flushright{\begin{Arabic}
\quranayah[4][161]
\end{Arabic}}
\flushleft{\begin{malayalam}
പലിശ അവര്‍ക്ക് വിരോധിക്കപ്പെട്ടതായിരുന്നിട്ടും അവരതനുഭവിച്ചു. അവര്‍ അവിഹിതമായി ജനങ്ങളുടെ സ്വത്ത് കവര്‍ന്നെടുത്ത് ആഹരിച്ചു. അവരിലെ സത്യനിഷേധികള്‍ക്കു നാം നോവേറിയ ശിക്ഷ ഒരുക്കിവെച്ചിട്ടുണ്ട്.
\end{malayalam}}
\flushright{\begin{Arabic}
\quranayah[4][162]
\end{Arabic}}
\flushleft{\begin{malayalam}
എന്നാല്‍ അവരിലെ അഗാധജ്ഞാനമുള്ളവരും സത്യവിശ്വാസികളും നിനക്ക് ഇറക്കിത്തന്നതിലും നിനക്ക് മുമ്പെ ഇറക്കിക്കൊടുത്തതിലും വിശ്വസിക്കുന്നു. നമസ്കാരം നിഷ്ഠയോടെ നിര്‍വഹിക്കുന്നവരാണവര്‍. സകാത്ത് നല്‍കുന്നവരും അല്ലാഹുവിലും അന്ത്യദിനത്തിലും വിശ്വസിക്കുന്നവരുമാണ്. അവര്‍ക്ക് നാം മഹത്തായ പ്രതിഫലം നല്‍കും.
\end{malayalam}}
\flushright{\begin{Arabic}
\quranayah[4][163]
\end{Arabic}}
\flushleft{\begin{malayalam}
നൂഹിനും തുടര്‍ന്നുവന്ന പ്രവാചകന്മാര്‍ക്കും നാം ബോധനം നല്‍കിയപോലെത്തന്നെ നിനക്കും നാം ബോധനം നല്‍കിയിരിക്കുന്നു. ഇബ്റാഹീം, ഇസ്മാഈല്‍, ഇസ്ഹാഖ്, യഅ്ഖൂബ്, യഅ്ഖൂബ് സന്തതികള്‍, ഈസാ, അയ്യൂബ്, യൂനുസ്, ഹാറൂന്‍, സുലൈമാന്‍ എന്നിവര്‍ക്കും നാം ബോധനം നല്‍കിയിരിക്കുന്നു. ദാവൂദിന് സങ്കീര്‍ത്തനവും നല്‍കി.
\end{malayalam}}
\flushright{\begin{Arabic}
\quranayah[4][164]
\end{Arabic}}
\flushleft{\begin{malayalam}
ഇതിനുമുമ്പ് നിനക്കു നാം പറഞ്ഞുതന്നതും ഇനിയും നിന്നെ അറിയിച്ചിട്ടില്ലാത്തതുമായ മറ്റു ദൈവദൂതന്മാര്‍ക്കും നാം ബോധനം നല്‍കിയിട്ടുണ്ട്. മൂസായോട് അല്ലാഹു നേരിട്ടുതന്നെ സംസാരിച്ചു.
\end{malayalam}}
\flushright{\begin{Arabic}
\quranayah[4][165]
\end{Arabic}}
\flushleft{\begin{malayalam}
ഇവരൊക്കെയും ശുഭവാര്‍ത്ത അറിയിക്കുന്നവരും മുന്നറിയിപ്പു നല്‍കുന്നവരുമായ ദൈവദൂതന്മാരായിരുന്നു. അവരുടെ നിയോഗശേഷം ജനങ്ങള്‍ക്ക് അല്ലാഹുവിനെതിരെ ഒരു ന്യായവും പറയാനില്ലാതിരിക്കാനാണിത്. അല്ലാഹു പ്രതാപിയും യുക്തിമാനുമാണ്.
\end{malayalam}}
\flushright{\begin{Arabic}
\quranayah[4][166]
\end{Arabic}}
\flushleft{\begin{malayalam}
അല്ലാഹു നിനക്ക് ഇറക്കിത്തന്നത് തന്റെ ജ്ഞാനത്തില്‍ നിന്നുതന്നെയാണെന്നതിന് അവന്‍ തന്നെ സാക്ഷ്യം വഹിക്കുന്നുണ്ട്; മലക്കുകളും അതിനു സാക്ഷ്യം വഹിക്കുന്നു. സാക്ഷിയായി അല്ലാഹുതന്നെ ധാരാളമാണെങ്കിലും!
\end{malayalam}}
\flushright{\begin{Arabic}
\quranayah[4][167]
\end{Arabic}}
\flushleft{\begin{malayalam}
സത്യത്തെ തള്ളിപ്പറയുകയും അല്ലാഹുവിന്റെ മാര്‍ഗത്തില്‍ തടസ്സം സൃഷ്ടിക്കുകയും ചെയ്യുന്നവര്‍ വഴികേടില്‍ ഏറെദൂരം പിന്നിട്ടിരിക്കുന്നു.
\end{malayalam}}
\flushright{\begin{Arabic}
\quranayah[4][168]
\end{Arabic}}
\flushleft{\begin{malayalam}
സത്യത്തെ നിഷേധിക്കുകയും അതിക്രമം കാണിക്കുകയും ചെയ്തവര്‍ക്ക് അല്ലാഹു ഒരിക്കലും പൊറുത്തുകൊടുക്കുകയില്ല. ഒരു മാര്‍ഗവും അവര്‍ക്ക് കാണിച്ചുകൊടുക്കുകയുമില്ല;
\end{malayalam}}
\flushright{\begin{Arabic}
\quranayah[4][169]
\end{Arabic}}
\flushleft{\begin{malayalam}
നരകത്തിന്റെ മാര്‍ഗമല്ലാതെ. അവരതില്‍ സ്ഥിരവാസികളായിരിക്കും. അല്ലാഹുവിന് അത് വളരെ എളുപ്പമുള്ള കാര്യമാണ്.
\end{malayalam}}
\flushright{\begin{Arabic}
\quranayah[4][170]
\end{Arabic}}
\flushleft{\begin{malayalam}
ജനങ്ങളേ, നിങ്ങളുടെ നാഥങ്കല്‍ നിന്നുള്ള സത്യസന്ദേശവുമായി ദൈവദൂതനിതാ നിങ്ങളുടെ അടുത്ത് വന്നിരിക്കുന്നു. അതിനാല്‍ നിങ്ങള്‍ വിശ്വസിക്കുക. അതാണ് നിങ്ങള്‍ക്കുത്തമം. അഥവാ, നിങ്ങള്‍ നിഷേധിക്കുകയാണെങ്കില്‍ അറിയുക: ആകാശ ഭൂമികളിലുള്ളതെല്ലാം അല്ലാഹുവിന്റേതാണ്. അല്ലാഹു എല്ലാം അറിയുന്നവനാണ്. യുക്തിമാനും.
\end{malayalam}}
\flushright{\begin{Arabic}
\quranayah[4][171]
\end{Arabic}}
\flushleft{\begin{malayalam}
വേദക്കാരേ, നിങ്ങള്‍ നിങ്ങളുടെ മതകാര്യത്തില്‍ അതിരുകവിയരുത്. അല്ലാഹുവിന്റെ പേരില്‍ സത്യമല്ലാത്തതൊന്നും പറയരുത്. മര്‍യമിന്റെ മകന്‍ മസീഹ് ഈസാ, അല്ലാഹുവിന്റെ ദൂതനും മര്‍യമിലേക്ക് അവനിട്ടുകൊടുത്ത തന്റെ വചനവും അവങ്കല്‍നിന്നുള്ള ഒരാത്മാവും മാത്രമാണ്. അതിനാല്‍ നിങ്ങള്‍ അല്ലാഹുവിലും അവന്റെ ദൂതന്മാരിലും വിശ്വസിക്കുക. “ത്രിത്വം” പറയരുത്. നിങ്ങളതവസാനിപ്പിക്കുക. അതാണ് നിങ്ങള്‍ക്കുത്തമം. അല്ലാഹു ഏകനായ ദൈവം മാത്രമാണ്. തനിക്ക് ഒരു പുത്രനുണ്ടാവുകയെന്നതില്‍നിന്ന് അവനെത്ര പരിശുദ്ധന്‍. ആകാശഭൂമികളിലുള്ളതെല്ലാം അവന്റേതാണ്. കാര്യനിര്‍വഹണത്തിന് അല്ലാഹുതന്നെ മതി.
\end{malayalam}}
\flushright{\begin{Arabic}
\quranayah[4][172]
\end{Arabic}}
\flushleft{\begin{malayalam}
അല്ലാഹുവിന്റെ അടിമയായിരിക്കുന്നതില്‍ മസീഹ് ഒട്ടും വൈമനസ്യം കാണിച്ചിട്ടില്ല. ദിവ്യസാമീപ്യം സിദ്ധിച്ച മലക്കുകളും അങ്ങനെത്തന്നെ. അല്ലാഹുവിന് വഴിപ്പെടുന്നതില്‍ വൈമനസ്യം കാണിക്കുകയും അഹന്ത നടിക്കുകയും ചെയ്യുന്നവരെയൊക്കെ അവന്‍ തന്റെ അടുത്തേക്ക് ഒരുമിച്ചുകൂട്ടുന്നതാണ്.
\end{malayalam}}
\flushright{\begin{Arabic}
\quranayah[4][173]
\end{Arabic}}
\flushleft{\begin{malayalam}
എന്നാല്‍ സത്യവിശ്വാസം സ്വീകരിക്കുകയും സല്‍ക്കര്‍മങ്ങള്‍ പ്രവര്‍ത്തിക്കുകയും ചെയ്തവരുടെ പ്രതിഫലം അവര്‍ക്ക് അവന്‍ പൂര്‍ണമായി നല്‍കും. അതോടൊപ്പം അവന്റെ ഔദാര്യത്താല്‍ കൂടുതലായും കൊടുക്കും. എന്നാല്‍ അല്ലാഹുവിന് വഴിപ്പെടാന്‍ വിമുഖത കാണിക്കുകയും അഹന്ത നടിക്കുകയും ചെയ്തവര്‍ക്ക് അവന്‍ നോവേറിയ ശിക്ഷ നല്‍കും. അല്ലാഹുവെക്കൂടാതെ അവര്‍ക്കൊരു ആത്മ മിത്രത്തെയോ സഹായിയെയോ കണ്ടെത്താനാവില്ല.
\end{malayalam}}
\flushright{\begin{Arabic}
\quranayah[4][174]
\end{Arabic}}
\flushleft{\begin{malayalam}
മനുഷ്യരേ, നിങ്ങളുടെ നാഥങ്കല്‍ നിന്നുള്ള ന്യായപ്രമാണം നിങ്ങള്‍ക്കിതാ വന്നെത്തിയിരിക്കുന്നു. എല്ലാം വ്യക്തമായി തെളിയിച്ചുകാണിക്കുന്ന പ്രകാശം നാമിതാ നിങ്ങള്‍ക്ക് ഇറക്കിത്തന്നിരിക്കുന്നു.
\end{malayalam}}
\flushright{\begin{Arabic}
\quranayah[4][175]
\end{Arabic}}
\flushleft{\begin{malayalam}
അതിനാല്‍ അല്ലാഹുവില്‍ വിശ്വസിക്കുകയും അവന്റെ സംരക്ഷണം തേടുകയും ചെയ്തവരെ തന്റെ കാരുണ്യത്തിലും അനുഗ്രഹത്തിലും അവന്‍ പ്രവേശിപ്പിക്കുന്നതാണ്. അവരെ തന്നിലേക്ക് നേര്‍വഴിയിലൂടെ നയിക്കുകയും ചെയ്യും.
\end{malayalam}}
\flushright{\begin{Arabic}
\quranayah[4][176]
\end{Arabic}}
\flushleft{\begin{malayalam}
അവര്‍ നിന്നോട് വിധിതേടുന്നു. പറയുക: കലാലത്തി ന്റെ കാര്യത്തില്‍ അല്ലാഹു നിങ്ങള്‍ക്കിതാ വിധി നല്‍കുന്നു: ഒരാള്‍ മരണപ്പെട്ടു. അയാള്‍ക്ക് മക്കളില്ല. ഒരു സഹോദരിയുണ്ട്. എങ്കില്‍ അയാള്‍ വിട്ടേച്ചുപോയ സ്വത്തില്‍ പാതി അവള്‍ക്കുള്ളതാണ്. അവള്‍ക്ക് മക്കളില്ലെങ്കില്‍ അവളുടെ അനന്തര സ്വത്ത് അയാള്‍ക്കുള്ളതുമാണ്. രണ്ടു സഹോദരിമാരാണുള്ളതെങ്കില്‍ സഹോദരന്‍ വിട്ടേച്ചുപോയ സ്വത്തിന്റെ മൂന്നില്‍ രണ്ട് അവര്‍ക്കായിരിക്കും. സഹോദരന്മാരും സഹോദരിമാരുമാണുള്ളതെങ്കില്‍ ആണിന് രണ്ടു പെണ്ണിന്റെ ഓഹരിയാണുണ്ടാവുക. നിങ്ങള്‍ക്കു പിഴവുപറ്റാതിരിക്കാനാണ് അല്ലാഹു ഇതൊക്കെ ഇവ്വിധം വിവരിച്ചുതരുന്നത്. അല്ലാഹു എല്ലാ കാര്യങ്ങളെക്കുറിച്ചും നന്നായറിയുന്നവനാണ്.
\end{malayalam}}
\chapter{\textmalayalam{മാഇദ ( ഭക്ഷണ തളിക )}}
\begin{Arabic}
\Huge{\centerline{\basmalah}}\end{Arabic}
\flushright{\begin{Arabic}
\quranayah[5][1]
\end{Arabic}}
\flushleft{\begin{malayalam}
വിശ്വസിച്ചവരേ, കരാറുകള്‍ പാലിക്കുക. നാല്‍ക്കാലികളില്‍പെട്ട മൃഗങ്ങള്‍ നിങ്ങള്‍ക്ക് അനുവദിക്കപ്പെട്ടിരിക്കുന്നു; വഴിയെ വിവരിക്കുന്നവ ഒഴികെ. എന്നാല്‍ ഇഹ്റാമിലായിരിക്കെ വേട്ടയാടുന്നത് അനുവദനീയമായി ഗണിക്കരുത്. അല്ലാഹു അവനിച്ഛിക്കുന്നത് വിധിക്കുന്നു.
\end{malayalam}}
\flushright{\begin{Arabic}
\quranayah[5][2]
\end{Arabic}}
\flushleft{\begin{malayalam}
വിശ്വസിച്ചവരേ, അല്ലാഹുവിന്റെ ചിഹ്നങ്ങളെ നിങ്ങള്‍ അനാദരിക്കരുത്. പവിത്രമാസം; ബലിമൃഗങ്ങള്‍; അവയെ തിരിച്ചറിയാനുള്ള കഴുത്തിലെ വടങ്ങള്‍; തങ്ങളുടെ നാഥന്റെ അനുഗ്രഹവും പ്രീതിയും തേടി പുണ്യഗേഹലേക്ക് പോകുന്നവര്‍- ഇവയെയും നിങ്ങള്‍ അനാദരിക്കരുത്. ഇഹ്റാമില്‍ നിന്നൊഴിവായാല്‍ നിങ്ങള്‍ക്ക് വേട്ടയിലേര്‍പ്പെടാവുന്നതാണ്. മസ്ജിദുല്‍ ഹറാമില്‍ പ്രവേശിക്കുന്നതില്‍ നിന്ന് നിങ്ങളെ വിലക്കിയവരോടുള്ള വെറുപ്പ് അവര്‍ക്കെതിരെ അതിക്രമം പ്രവര്‍ത്തിക്കാന്‍ നിങ്ങളെ പ്രേരിപ്പിക്കാതിരിക്കട്ടെ. പുണ്യത്തിലും ദൈവഭക്തിയിലും പരസ്പരം സഹായികളാവുക. പാപത്തിലും പരാക്രമത്തിലും പരസ്പരം സഹായികളാകരുത്. നിങ്ങള്‍ അല്ലാഹുവെ സൂക്ഷിക്കുക. അല്ലാഹു കഠിനമായി ശിക്ഷിക്കുന്നവനാണ്.
\end{malayalam}}
\flushright{\begin{Arabic}
\quranayah[5][3]
\end{Arabic}}
\flushleft{\begin{malayalam}
ശവം, രക്തം, പന്നിയിറച്ചി, അല്ലാഹുവല്ലാത്തവരുടെ പേരില്‍ അറുക്കപ്പെട്ടത്, ശ്വാസംമുട്ടിച്ചത്തത്, തല്ലിക്കൊന്നത്, വീണുചത്തത്, തമ്മില്‍കുത്തിച്ചത്തത്, വന്യമൃഗം കടിച്ചു തിന്നിട്ടത്- ചാവും മുമ്പെ നിങ്ങള്‍ അറുത്തത് ഒഴികെ- പ്രതിഷ്ഠകള്‍ക്ക് ബലിയറുത്തത്; ഇതൊക്കെയും നിങ്ങള്‍ക്ക് നിഷിദ്ധമാണ്. അമ്പുകള്‍കൊണ്ട് ഭാഗ്യപരീക്ഷണം നടത്തലും നിഷിദ്ധം തന്നെ. ഇതെല്ലാം മ്ളേഛമാണ്. സത്യനിഷേധികള്‍ നിങ്ങളുടെ ദീനിനെ നേരിടുന്നതില്‍ ഇന്ന് നിരാശരായിരിക്കുന്നു. അതിനാല്‍ നിങ്ങളവരെ പേടിക്കേണ്ടതില്ല. എന്നെ മാത്രം ഭയപ്പെടുക. ഇന്ന് നിങ്ങളുടെ ജീവിതവ്യവസ്ഥ ഞാന്‍ നിങ്ങള്‍ക്കു തികവുറ്റതാക്കി തന്നിരിക്കുന്നു. എന്റെ അനുഗ്രഹം നിങ്ങള്‍ക്ക് പൂര്‍ത്തീകരിച്ചു തന്നിരിക്കുന്നു. ഇസ്ലാമിനെ നിങ്ങള്‍ക്കുള്ള ജീവിതവ്യവസ്ഥയായി തൃപ്തിപ്പെടുകയും ചെയ്തിരിക്കുന്നു. ആരെങ്കിലും പട്ടിണി കാരണം നിഷിദ്ധം തിന്നാന്‍ നിര്‍ബന്ധിതനായാല്‍, അവന്‍ തെറ്റുചെയ്യാന്‍ തല്‍പരനല്ലെങ്കില്‍, അറിയുക: ഉറപ്പായും അല്ലാഹു ഏറെ പൊറുക്കുന്നവനും പരമകാരുണികനുമാകുന്നു.
\end{malayalam}}
\flushright{\begin{Arabic}
\quranayah[5][4]
\end{Arabic}}
\flushleft{\begin{malayalam}
അവര്‍ നിന്നോടു ചോദിക്കുന്നു: എന്തൊക്കെയാണ് തങ്ങള്‍ക്ക് തിന്നാന്‍ പാടുള്ളതെന്ന്. പറയുക: നിങ്ങള്‍ക്ക് നല്ല വസ്തുക്കളൊക്കെയും തിന്നാന്‍ അനുവാദമുണ്ട്. അല്ലാഹു നിങ്ങള്‍ക്കേകിയ അറിവുപയോഗിച്ച് നിങ്ങള്‍ പരിശീലിപ്പിച്ച വേട്ടമൃഗം നിങ്ങള്‍ക്കായി പിടിച്ചുകൊണ്ടുവന്നു തരുന്നതും നിങ്ങള്‍ക്ക് തിന്നാം. എന്നാല്‍ ആ ഉരുവിന്റെ മേല്‍ നിങ്ങള്‍ അല്ലാഹുവിന്റെ നാമം ഉരുവിടണം. അല്ലാഹുവെ സൂക്ഷിക്കുക. സംശയം വേണ്ട; അല്ലാഹു അതിവേഗം കണക്കു നോക്കുന്നവനാണ്.
\end{malayalam}}
\flushright{\begin{Arabic}
\quranayah[5][5]
\end{Arabic}}
\flushleft{\begin{malayalam}
ഇന്ന് എല്ലാ നല്ല വസ്തുക്കളും നിങ്ങള്‍ക്ക് അനുവദനീയമാക്കിയിരിക്കുന്നു. വേദക്കാരുടെ ആഹാരം നിങ്ങള്‍ക്കും നിങ്ങളുടെ ആഹാരം അവര്‍ക്കും അനുവദനീയമാണ്. സത്യവിശ്വാസിനികളില്‍ നിന്നുള്ള ചാരിത്രവതികളും നിങ്ങള്‍ക്കുമുമ്പേ വേദം നല്‍കപ്പെട്ടവരില്‍ നിന്നുള്ള ചാരിത്രവതികളും നിങ്ങള്‍ക്ക് അനുവദനീയരാണ്. നിങ്ങള്‍ അവര്‍ക്ക് വിവാഹമൂല്യം നല്‍കി കല്യാണം കഴിക്കണമെന്നുമാത്രം. അതോടൊപ്പം അവര്‍ പരസ്യമായി വ്യഭിചാരത്തിലേര്‍പ്പെടുന്നവരോ രഹസ്യവേഴ്ചക്കാരെ സ്വീകരിക്കുന്നവരോ ആവരുത്. സത്യവിശ്വാസത്തെ നിഷേധിക്കുന്നവന്റെ പ്രവര്‍ത്തനങ്ങളൊക്കെയും പാഴായിരിക്കുന്നു. പരലോകത്ത് അവന്‍ പാപ്പരായിരിക്കും.
\end{malayalam}}
\flushright{\begin{Arabic}
\quranayah[5][6]
\end{Arabic}}
\flushleft{\begin{malayalam}
വിശ്വസിച്ചവരേ, നിങ്ങള്‍ നമസ്കാരത്തിനൊരുങ്ങിയാല്‍ നിങ്ങളുടെ മുഖങ്ങളും മുട്ടുവരെ ഇരുകരങ്ങളും കഴുകുക. തല തടവുകയും ഞെരിയാണിവരെ കാലുകള്‍ കഴുകുകയും ചെയ്യുക. നിങ്ങള്‍ വലിയ അശുദ്ധിയുള്ളവരാണെങ്കില്‍ കുളിച്ചു ശുദ്ധിയാവുക. രോഗികളോ യാത്രക്കാരോ ആണെങ്കിലും നിങ്ങളിലാരെങ്കിലും വിസര്‍ജിച്ചുവരികയോ സ്ത്രീസംസര്‍ഗം നടത്തുകയോ ചെയ്തിട്ട് വെള്ളം കിട്ടാതിരിക്കുകയാണെങ്കിലും ശുദ്ധിവരുത്താന്‍ മാലിന്യമില്ലാത്ത മണ്ണ് ഉപയോഗിക്കുക. അതില്‍ കയ്യടിച്ച് മുഖവും കൈകളും തടവുക. നിങ്ങളെ പ്രയാസപ്പെടുത്താന്‍ അല്ലാഹു ഉദ്ദേശിക്കുന്നില്ല. എന്നാല്‍ നിങ്ങളെ ശുദ്ധീകരിക്കാനും നിങ്ങള്‍ക്ക് അവന്റെ അനുഗ്രഹം പൂര്‍ത്തീകരിച്ചു തരാനും അവനുദ്ദേശിക്കുന്നു. നിങ്ങള്‍ നന്ദിയുള്ളവരാകാന്‍.
\end{malayalam}}
\flushright{\begin{Arabic}
\quranayah[5][7]
\end{Arabic}}
\flushleft{\begin{malayalam}
അല്ലാഹു നിങ്ങള്‍ക്കേകിയ അനുഗ്രഹങ്ങള്‍ ഓര്‍ക്കുക. അവന്‍ നിങ്ങളോട് കരുത്തുറ്റ കരാര്‍ വാങ്ങിയ കാര്യവും. അഥവാ, “ഞങ്ങള്‍ കേള്‍ക്കുകയും അനുസരിക്കുകയും ചെയ്തിരിക്കുന്നു” വെന്ന് നിങ്ങള്‍ പറഞ്ഞ കാര്യം. നിങ്ങള്‍ അല്ലാഹുവെ സൂക്ഷിക്കുക. തീര്‍ച്ചയായും അല്ലാഹു മനസ്സുകളിലുള്ളതൊക്കെയും നന്നായറിയുന്നവനാണ്.
\end{malayalam}}
\flushright{\begin{Arabic}
\quranayah[5][8]
\end{Arabic}}
\flushleft{\begin{malayalam}
വിശ്വസിച്ചവരേ, നിങ്ങള്‍ അല്ലാഹുവിനുവേണ്ടി നേരാംവിധം നിലകൊള്ളുന്നവരാവുക. നീതിക്ക് സാക്ഷ്യം വഹിക്കുന്നവരും. ഒരു ജനതയോടുള്ള വിരോധം നീതി നടത്താതിരിക്കാന്‍ നിങ്ങളെ പ്രേരിപ്പിക്കാതിരിക്കട്ടെ. നീതി പാലിക്കുക. അതാണ് ദൈവഭക്തിക്ക് ഏറ്റം പറ്റിയത്. നിങ്ങള്‍ അല്ലാഹുവെ സൂക്ഷിക്കുക. ഉറപ്പായും അല്ലാഹു നിങ്ങള്‍ ചെയ്യുന്നതെല്ലാം സൂക്ഷ്മമായി അറിയുന്നവനാണ്.
\end{malayalam}}
\flushright{\begin{Arabic}
\quranayah[5][9]
\end{Arabic}}
\flushleft{\begin{malayalam}
സത്യവിശ്വാസം സ്വീകരിക്കുകയും സല്‍ക്കര്‍മങ്ങള്‍ പ്രവര്‍ത്തിക്കുകയും ചെയ്തവര്‍ക്ക് പാപമോചനവും മഹത്തായ പ്രതിഫലവുമുണ്ടെന്ന് അല്ലാഹു വാഗ്ദാനം ചെയ്തിരിക്കുന്നു.
\end{malayalam}}
\flushright{\begin{Arabic}
\quranayah[5][10]
\end{Arabic}}
\flushleft{\begin{malayalam}
എന്നാല്‍ സത്യത്തെ നിഷേധിക്കുകയും നമ്മുടെ വചനങ്ങളെ തള്ളിക്കളയുകയും ചെയ്തവരോ, അവരാണ് നരകാവകാശികള്‍.
\end{malayalam}}
\flushright{\begin{Arabic}
\quranayah[5][11]
\end{Arabic}}
\flushleft{\begin{malayalam}
വിശ്വസിച്ചവരേ, അല്ലാഹു നിങ്ങള്‍ക്കേകിയ അനുഗ്രഹം ഓര്‍ത്തുനോക്കൂ: ഒരുകൂട്ടര്‍ നിങ്ങള്‍ക്ക് നേരെ കൈയോങ്ങാന്‍ ഒരുമ്പെടുകയായിരുന്നു. അപ്പോള്‍ അല്ലാഹു നിങ്ങളില്‍ നിന്ന് അവരുടെ കൈകളെ തടഞ്ഞുനിര്‍ത്തി. അതിനാല്‍ അല്ലാഹുവോട് ഭക്തിയുള്ളവരാവുക. സത്യവിശ്വാസികള്‍ അല്ലാഹുവില്‍ മാത്രം സര്‍വസ്വം സമര്‍പ്പിക്കട്ടെ.
\end{malayalam}}
\flushright{\begin{Arabic}
\quranayah[5][12]
\end{Arabic}}
\flushleft{\begin{malayalam}
അല്ലാഹു ഇസ്രയേല്‍ മക്കളോട് കരാര്‍ വാങ്ങിയിരുന്നു. അവരില്‍ പന്ത്രണ്ടുപേരെ മുഖ്യന്മാരായി നാം നിയോഗിക്കുകയും ചെയ്തു. അല്ലാഹു അവരോടു പറഞ്ഞു: "തീര്‍ച്ചയായും ഞാന്‍ നിങ്ങളോടൊപ്പമുണ്ട്. നിങ്ങള്‍ നമസ്കാരം നിഷ്ഠയോടെ നിര്‍വഹിക്കുക. സകാത്ത് നല്‍കുക. എന്റെ ദൂതന്മാരില്‍ വിശ്വസിക്കുക. അവരെ സഹായിക്കുക. അല്ലാഹുവിന് ശ്രേഷ്ഠമായ കടം കൊടുക്കുകയും ചെയ്യുക. എങ്കില്‍ ഞാന്‍ നിങ്ങളുടെ തിന്മകള്‍ മായ്ച്ചുകളയും; താഴ്ഭാഗത്തൂടെ ആറുകളൊഴുകുന്ന സ്വര്‍ഗീയാരാമങ്ങളില്‍ നിങ്ങളെ പ്രവേശിപ്പിക്കും; തീര്‍ച്ച. എന്നാല്‍ അതിനുശേഷം നിങ്ങളാരെങ്കിലും നിഷേധികളാവുകയാണെങ്കില്‍ അവന്‍ നേര്‍വഴിയില്‍നിന്ന് തെറ്റിപ്പോയതുതന്നെ.
\end{malayalam}}
\flushright{\begin{Arabic}
\quranayah[5][13]
\end{Arabic}}
\flushleft{\begin{malayalam}
പിന്നീട് അവരുടെ കരാര്‍ ലംഘനം കാരണമായി നാമവരെ ശപിച്ചു. അവരുടെ ഹൃദയങ്ങളെ കഠിനമാക്കുകയും ചെയ്തു. അവര്‍ വേദവാക്യങ്ങള്‍ വളച്ചൊടിക്കുന്നു. നാം നല്‍കിയ ഉദ്ബോധനങ്ങളില്‍ വലിയൊരു ഭാഗം മറക്കുകയും ചെയ്തു. അവരില്‍ അല്‍പം ചിലരൊഴിച്ച് ബാക്കിയുള്ളവരൊക്കെ ചെയ്തുകൊണ്ടിരിക്കുന്ന വഞ്ചന നീ കണ്ടുകൊണ്ടേയിരിക്കും. അതിനാല്‍ നീ അവര്‍ക്ക് മാപ്പേകുക. അവരോടു വിട്ടുവീഴ്ച കാണിക്കുക. നന്മ ചെയ്യുന്നവരെ അല്ലാഹു ഇഷ്ടപ്പെടും; തീര്‍ച്ച.
\end{malayalam}}
\flushright{\begin{Arabic}
\quranayah[5][14]
\end{Arabic}}
\flushleft{\begin{malayalam}
ഞങ്ങള്‍ ക്രിസ്ത്യാനികളാണ് എന്ന് അവകാശപ്പെടുന്നവരില്‍ നിന്നും നാം കരാര്‍ വാങ്ങിയിരുന്നു. എന്നാല്‍ അവരും തങ്ങള്‍ക്കു ലഭിച്ച ഉദ്ബോധനങ്ങളില്‍ വലിയൊരുഭാഗം മറന്നുകളഞ്ഞു. അതിനാല്‍ അവര്‍ക്കിടയില്‍ നാം ഉയിര്‍ത്തെഴുന്നേല്‍പുനാള്‍ വരെ പരസ്പര വൈരവും വെറുപ്പും വളര്‍ത്തി. അവര്‍ ചെയ്തുകൊണ്ടിരുന്നതിനെപ്പറ്റിയെല്ലാം അല്ലാഹു പിന്നീടവരെ അറിയിക്കുന്നതാണ്.
\end{malayalam}}
\flushright{\begin{Arabic}
\quranayah[5][15]
\end{Arabic}}
\flushleft{\begin{malayalam}
വേദക്കാരേ, വേദഗ്രന്ഥത്തില്‍നിന്ന് നിങ്ങള്‍ മറച്ചുവെച്ചിരുന്ന ഒത്തിരി കാര്യങ്ങള്‍ വെളിപ്പെടുത്തിക്കൊണ്ട്, നമ്മുടെ ദൂതനിതാ നിങ്ങളുടെ അടുത്ത് വന്നിരിക്കുന്നു. ഒട്ടു വളരെ കാര്യങ്ങളില്‍ അദ്ദേഹം വിട്ടുവീഴ്ച കാണിച്ചിരിക്കുന്നു. നിങ്ങള്‍ക്കിതാ അല്ലാഹുവില്‍ നിന്നുള്ള വെളിച്ചവും തെളിവുറ്റ വേദവും വന്നെത്തിയിരിക്കുന്നു.
\end{malayalam}}
\flushright{\begin{Arabic}
\quranayah[5][16]
\end{Arabic}}
\flushleft{\begin{malayalam}
തന്റെ തൃപ്തി തേടിയവരെ അല്ലാഹു വേദംവഴി സമാധാനത്തിന്റെ പാതയിലേക്കു നയിക്കുന്നു. തന്റെ ഹിതത്താല്‍, അവരെ ഇരുളില്‍നിന്ന് വെളിച്ചത്തിലേക്കു കൊണ്ടുവരുന്നു. നേരായ വഴിയിലൂടെ നയിക്കുകയും ചെയ്യുന്നു.
\end{malayalam}}
\flushright{\begin{Arabic}
\quranayah[5][17]
\end{Arabic}}
\flushleft{\begin{malayalam}
മര്‍യമിന്റെ മകന്‍ മസീഹ് തന്നെയാണ് ദൈവമെന്ന് പറഞ്ഞവര്‍ തീര്‍ച്ചയായും സത്യനിഷേധികളായിരിക്കുന്നു. ചോദിക്കുക: അല്ലാഹു മര്‍യമിന്റെ മകന്‍ മസീഹിനെയും അയാളുടെ മാതാവിനെയും ഭൂമിയിലുള്ളവരെയൊക്കെയും നശിപ്പിക്കാന്‍ തീരുമാനിച്ചാല്‍ അവന്റെ തീരുമാനത്തില്‍ മാറ്റം വരുത്താന്‍ ആര്‍ക്കാണ് കഴിയുക? ആകാശഭൂമികളുടെയും അവയ്ക്കിടയിലുള്ളവയുടെയുമെല്ലാം ആധിപത്യം അല്ലാഹുവിനാണ്. അവനിച്ഛിക്കുന്നതെല്ലാം അവന്‍ സൃഷ്ടിക്കുന്നു. അല്ലാഹു എല്ലാ കാര്യങ്ങള്‍ക്കും കഴിവുറ്റവനാണ്.
\end{malayalam}}
\flushright{\begin{Arabic}
\quranayah[5][18]
\end{Arabic}}
\flushleft{\begin{malayalam}
യഹൂദരും ക്രിസ്ത്യാനികളും വാദിക്കുന്നു, തങ്ങള്‍ ദൈവത്തിന്റെ മക്കളും അവനു പ്രിയപ്പെട്ടവരുമാണെന്ന്. അവരോടു ചോദിക്കുക: എങ്കില്‍ പിന്നെ നിങ്ങളുടെ പാപങ്ങളുടെ പേരില്‍ അവന്‍ നിങ്ങളെ ശിക്ഷിക്കുന്നതെന്തുകൊണ്ട്? എന്നാല്‍ ഓര്‍ക്കുക; നിങ്ങളും അവന്റെ സൃഷ്ടികളില്‍പെട്ട മനുഷ്യര്‍ മാത്രമാണ്. അവനിച്ഛിക്കുന്നവര്‍ക്ക് അവന്‍ മാപ്പേകുന്നു. അവനുദ്ദേശിക്കുന്നവരെ അവന്‍ ശിക്ഷിക്കുന്നു. വിണ്ണിന്റെയും മണ്ണിന്റെയും അവയ്ക്കിടയിലുള്ളവയുടെയുമെല്ലാം ഉടമ അല്ലാഹുവാണ്. എല്ലാറ്റിന്റെയും മടക്കവും അവനിലേക്കുതന്നെ.
\end{malayalam}}
\flushright{\begin{Arabic}
\quranayah[5][19]
\end{Arabic}}
\flushleft{\begin{malayalam}
വേദക്കാരേ, ദൈവദൂതന്മാരുടെ വരവ് നിലച്ചുപോയ വേളയില്‍ നമ്മുടെ ദൂതനിതാ കാര്യങ്ങള്‍ വിശദീകരിച്ചുതരുന്നവനായി നിങ്ങളുടെ അടുത്ത് വന്നിരിക്കുന്നു. “ഞങ്ങളുടെ അടുത്ത് ശുഭവാര്‍ത്ത അറിയിക്കുന്നവനോ മുന്നറിയിപ്പുകാരനോ വന്നിട്ടില്ലല്ലോ” എന്ന് നിങ്ങള്‍ പറയാതിരിക്കാനാണിത്. തീര്‍ച്ചയായും നിങ്ങള്‍ക്ക് സന്തോഷവാര്‍ത്ത അറിയിക്കുകയും മുന്നറിയിപ്പ് നല്‍കുകയും ചെയ്യുന്ന ദൂതനിതാ വന്നെത്തിയിരിക്കുന്നു. അല്ലാഹു എല്ലാ കാര്യങ്ങള്‍ക്കും കഴിവുറ്റവന്‍ തന്നെ.
\end{malayalam}}
\flushright{\begin{Arabic}
\quranayah[5][20]
\end{Arabic}}
\flushleft{\begin{malayalam}
മൂസാ തന്റെ ജനത്തോടു പറഞ്ഞ സന്ദര്‍ഭം: "എന്റെ ജനമേ, അല്ലാഹു നിങ്ങള്‍ക്കേകിയ അനുഗ്രഹങ്ങള്‍ ഓര്‍ക്കുക: അവന്‍ നിങ്ങളില്‍ പ്രവാചകന്മാരെ നിയോഗിച്ചു. നിങ്ങളെ രാജാക്കന്മാരാക്കി. ലോകരില്‍ മറ്റാര്‍ക്കും നല്‍കാത്ത പലതും അവന്‍ നിങ്ങള്‍ക്കു നല്‍കി.
\end{malayalam}}
\flushright{\begin{Arabic}
\quranayah[5][21]
\end{Arabic}}
\flushleft{\begin{malayalam}
"എന്റെ ജനമേ, അല്ലാഹു നിങ്ങള്‍ക്കായി നിശ്ചയിച്ച പുണ്യഭൂമിയില്‍ പ്രവേശിക്കുക. പിറകോട്ട് തിരിച്ചുപോകരുത്. അങ്ങനെ ചെയ്താല്‍ നിങ്ങള്‍ പരാജിതരായിത്തീരും.”
\end{malayalam}}
\flushright{\begin{Arabic}
\quranayah[5][22]
\end{Arabic}}
\flushleft{\begin{malayalam}
അവര്‍ പറഞ്ഞു: "ഹേ, മൂസാ, മഹാ മല്ലന്മാരായ ജനമാണ് അവിടെയുള്ളത്. അവര്‍ പുറത്തുപോകാതെ ഞങ്ങളവിടെ പ്രവേശിക്കുകയില്ല. അവര്‍ അവിടം വിട്ടൊഴിഞ്ഞാല്‍ ഞങ്ങളങ്ങോട്ടുപോകാം.”
\end{malayalam}}
\flushright{\begin{Arabic}
\quranayah[5][23]
\end{Arabic}}
\flushleft{\begin{malayalam}
ദൈവഭയമുള്ളവരും ദിവ്യാനുഗ്രഹം ലഭിച്ചവരുമായ രണ്ടുപേര്‍ മുന്നോട്ടുവന്നു. അവര്‍ പറഞ്ഞു: "പട്ടണവാതിലിലൂടെ നിങ്ങളവിടെ കടന്നുചെല്ലുക. അങ്ങനെ പ്രവേശിച്ചാല്‍ തീര്‍ച്ചയായും നിങ്ങളാണ് വിജയികളാവുക. നിങ്ങള്‍ വിശ്വാസികളെങ്കില്‍ അല്ലാഹുവില്‍ ഭരമേല്‍പിക്കുക.”
\end{malayalam}}
\flushright{\begin{Arabic}
\quranayah[5][24]
\end{Arabic}}
\flushleft{\begin{malayalam}
എന്നാല്‍ അവര്‍ ഇതുതന്നെ പറയുകയാണുണ്ടായത്: "മൂസാ, അവരവിടെ ഉള്ളേടത്തോളം കാലം ഞങ്ങളങ്ങോട്ട് പോവുകയില്ല. അതിനാല്‍ താനും തന്റെ ദൈവവും പോയി യുദ്ധം ചെയ്തുകൊള്ളുക. ഞങ്ങള്‍ ഇവിടെ ഇരുന്നുകൊള്ളാം.”
\end{malayalam}}
\flushright{\begin{Arabic}
\quranayah[5][25]
\end{Arabic}}
\flushleft{\begin{malayalam}
മൂസാ പ്രാര്‍ഥിച്ചു: "എന്റെ നാഥാ, എന്റെയും എന്റെ സഹോദരന്റെയും മേലല്ലാതെ എനിക്കു നിയന്ത്രണമില്ല. അതിനാല്‍ ധിക്കാരികളായ ഈ ജനത്തില്‍നിന്ന് നീ ഞങ്ങളെ വേര്‍പെടുത്തേണമേ.”
\end{malayalam}}
\flushright{\begin{Arabic}
\quranayah[5][26]
\end{Arabic}}
\flushleft{\begin{malayalam}
അല്ലാഹു മൂസായെ അറിയിച്ചു: "തീര്‍ച്ചയായും നാല്‍പതു കൊല്ലത്തേക്ക് ആ പ്രദേശം അവര്‍ക്ക് വിലക്കപ്പെട്ടിരിക്കുന്നു. അക്കാലമത്രയും അവര്‍ ഭൂമിയില്‍ അലഞ്ഞുതിരിയും. ധിക്കാരികളായ ഈ ജനത്തിന്റെ പേരില്‍ നീ ദുഃഖിക്കേണ്ടതില്ല.”
\end{malayalam}}
\flushright{\begin{Arabic}
\quranayah[5][27]
\end{Arabic}}
\flushleft{\begin{malayalam}
നീ അവര്‍ക്ക് ആദമിന്റെ രണ്ടു പുത്രന്മാരുടെ കഥ വസ്തുനിഷ്ഠമായി വിവരിച്ചുകൊടുക്കുക. അവരിരുവരും ബലി നടത്തിയപ്പോള്‍ ഒരാളുടെ ബലി സ്വീകാര്യമായി. അപരന്റേത് സ്വീകരിക്കപ്പെട്ടില്ല. അതിനാല്‍ അവന്‍ പറഞ്ഞു: "ഞാന്‍ നിന്നെ കൊല്ലുക തന്നെ ചെയ്യും.” അപരന്‍ പറഞ്ഞു: "ഭക്തന്മാരുടെ ബലിയേ അല്ലാഹു സ്വീകരിക്കുകയുള്ളൂ.
\end{malayalam}}
\flushright{\begin{Arabic}
\quranayah[5][28]
\end{Arabic}}
\flushleft{\begin{malayalam}
"എന്നെ കൊല്ലാന്‍ നീ എന്റെ നേരെ കൈനീട്ടിയാലും നിന്നെ കൊല്ലാന്‍ ഞാന്‍ നിന്റെ നേരെ കൈനീട്ടുകയില്ല. തീര്‍ച്ചയായും ഞാന്‍ പ്രപഞ്ചനാഥനായ അല്ലാഹുവെ ഭയപ്പെടുന്നു.
\end{malayalam}}
\flushright{\begin{Arabic}
\quranayah[5][29]
\end{Arabic}}
\flushleft{\begin{malayalam}
"എന്റെ പാപവും നിന്റെ പാപവും നീ തന്നെ പേറണമെന്ന് ഞാന്‍ ആഗ്രഹിക്കുന്നു. അങ്ങനെ നീ നരകാവകാശിയായിത്തീരണമെന്നും. അക്രമികള്‍ക്കുള്ള പ്രതിഫലം അതാണല്ലോ.”
\end{malayalam}}
\flushright{\begin{Arabic}
\quranayah[5][30]
\end{Arabic}}
\flushleft{\begin{malayalam}
എന്നിട്ടും അവന്റെ മനസ്സ് തന്റെ സഹോദരനെ വധിക്കാന്‍ തയ്യാറായി. അങ്ങനെ അവന്‍ അയാളെ കൊന്നു. അതിനാല്‍ അവന്‍ നഷ്ടം പറ്റിയവരുടെ കൂട്ടത്തിലായി.
\end{malayalam}}
\flushright{\begin{Arabic}
\quranayah[5][31]
\end{Arabic}}
\flushleft{\begin{malayalam}
പിന്നീട് അവന് തന്റെ സഹോദരന്റെ മൃതദേഹം മറവുചെയ്യേണ്ടതെങ്ങനെയെന്ന് കാണിച്ചുകൊടുക്കാനായി ഒരു കാക്കയെ അല്ലാഹു അയച്ചു. അത് മണ്ണില്‍ ഒരു കുഴിയുണ്ടാക്കുകയായിരുന്നു. ഇതുകണ്ട് അയാള്‍ വിലപിച്ചു: "കഷ്ടം! എന്റെ സഹോദരന്റെ മൃതദേഹം മറമാടുന്ന കാര്യത്തില്‍ ഈ കാക്കയെപ്പോലെയാകാന്‍ പോലും എനിക്കു കഴിഞ്ഞില്ലല്ലോ.” അങ്ങനെ അവന്‍ കൊടും ഖേദത്തിലകപ്പെട്ടു.
\end{malayalam}}
\flushright{\begin{Arabic}
\quranayah[5][32]
\end{Arabic}}
\flushleft{\begin{malayalam}
അക്കാരണത്താല്‍ ഇസ്രയേല്‍ സന്തതികളോടു നാം കല്‍പിച്ചു: "ആരെയെങ്കിലും കൊന്നതിനോ ഭൂമിയില്‍ കുഴപ്പമുണ്ടാക്കിയതിനോ അല്ലാതെ വല്ലവനും ഒരാളെ വധിച്ചാല്‍ അവന്‍ മുഴുവന്‍ മനുഷ്യരെയും വധിച്ചവനെപ്പോലെയാണ്. ഒരാളുടെ ജീവന്‍ രക്ഷിച്ചാല്‍ മുഴുവന്‍ മനുഷ്യരുടെയും ജീവന്‍ രക്ഷിച്ചവനെപ്പോലെയും.” നമ്മുടെ ദൂതന്മാര്‍ വ്യക്തമായ തെളിവുകളുമായി അവരുടെ അടുത്ത് വന്നിട്ടുണ്ടായിരുന്നു. എന്നിട്ട് പിന്നെയും അവരിലേറെപേരും ഭൂമിയില്‍ അതിക്രമം പ്രവര്‍ത്തിക്കുന്നവരാണ്.
\end{malayalam}}
\flushright{\begin{Arabic}
\quranayah[5][33]
\end{Arabic}}
\flushleft{\begin{malayalam}
അല്ലാഹുവോടും അവന്റെ ദൂതനോടും യുദ്ധത്തിലേര്‍പ്പെടുകയും ഭൂമിയില്‍ കുഴപ്പം കുത്തിപ്പൊക്കാന്‍ ശ്രമിക്കുകയും ചെയ്യുന്നവര്‍ക്കുള്ള ശിക്ഷ വധമോ കുരിശിലേറ്റലോ കൈകാലുകള്‍ എതിര്‍ദിശകളില്‍ മുറിച്ചുകളയലോ നാടുകടത്തലോ ആണ്. ഇത് അവര്‍ക്ക് ഈ ലോകത്തുള്ള മാനക്കേടാണ്. പരലോകത്തോ ഇതേക്കാള്‍ കടുത്ത ശിക്ഷയാണുണ്ടാവുക.
\end{malayalam}}
\flushright{\begin{Arabic}
\quranayah[5][34]
\end{Arabic}}
\flushleft{\begin{malayalam}
എന്നാല്‍ നിങ്ങള്‍ അവരെ പിടികൂടി നടപടിയെടുക്കാന്‍ തുടങ്ങുംമുമ്പെ അവര്‍ പശ്ചാത്തപിക്കുകയാണെങ്കില്‍ അവര്‍ക്ക് ഈ ശിക്ഷ ബാധകമല്ല. നിങ്ങളറിയുക: അല്ലാഹു ഏറെ പൊറുക്കുന്നവനും പരമദയാലുവുമാണ്.
\end{malayalam}}
\flushright{\begin{Arabic}
\quranayah[5][35]
\end{Arabic}}
\flushleft{\begin{malayalam}
വിശ്വസിച്ചവരേ, നിങ്ങള്‍ അല്ലാഹുവോട് ഭക്തിയുള്ളവരാവുക. അവനിലേക്ക് അടുക്കാനുള്ള വഴിതേടുക. അവന്റെ മാര്‍ഗത്തില്‍ പരമാവധി ത്യാഗപരിശ്രമങ്ങളനുഷ്ഠിക്കുക. നിങ്ങള്‍ വിജയം വരിച്ചേക്കാം.
\end{malayalam}}
\flushright{\begin{Arabic}
\quranayah[5][36]
\end{Arabic}}
\flushleft{\begin{malayalam}
ഭൂമിയിലുള്ളതൊക്കെയും അത്രതന്നെ വേറെയും സത്യനിഷേധികളുടെ വശമുണ്ടാവുകയും, ഉയിര്‍ത്തെഴുന്നേല്‍പുനാളിലെ ശിക്ഷയില്‍ നിന്നൊഴിവാകാന്‍ അതൊക്കെയും അവര്‍ പിഴയായി ഒടുക്കാനൊരുങ്ങുകയും ചെയ്താലും അവരില്‍ നിന്ന് അതൊന്നും സ്വീകരിക്കുകയില്ല. അവര്‍ക്ക് നോവേറിയ ശിക്ഷയാണുണ്ടാവുക.
\end{malayalam}}
\flushright{\begin{Arabic}
\quranayah[5][37]
\end{Arabic}}
\flushleft{\begin{malayalam}
നരകത്തില്‍ നിന്ന് പുറത്തുകടക്കാന്‍ അവര്‍ കൊതിക്കും. പക്ഷേ അതില്‍നിന്നു പുറത്തുകടക്കാനാവില്ല. സ്ഥിരമായ ശിക്ഷയാണ് അവര്‍ക്കുണ്ടാവുക.
\end{malayalam}}
\flushright{\begin{Arabic}
\quranayah[5][38]
\end{Arabic}}
\flushleft{\begin{malayalam}
കക്കുന്നവരുടെ - ആണായാലും പെണ്ണായാലും - കൈകള്‍ മുറിച്ചുകളയുക. അവര്‍ പ്രവര്‍ത്തിച്ചതിനുള്ള പ്രതിഫലമാണത്; അല്ലാഹുവില്‍ നിന്നുള്ള മാതൃകാപരമായ ശിക്ഷയും. അല്ലാഹു പ്രതാപവാനും യുക്തിമാനുമാകുന്നു.
\end{malayalam}}
\flushright{\begin{Arabic}
\quranayah[5][39]
\end{Arabic}}
\flushleft{\begin{malayalam}
എന്നാല്‍ അതിക്രമം ചെയ്തശേഷം ആരെങ്കിലും പശ്ചാത്തപിക്കുകയും നന്നാവുകയും ചെയ്താല്‍ അല്ലാഹു അവന്റെ പശ്ചാത്താപം സ്വീകരിക്കുന്നതാണ്. അല്ലാഹു ഏറെ പൊറുക്കുന്നവനും ദയാപരനുമാകുന്നു.
\end{malayalam}}
\flushright{\begin{Arabic}
\quranayah[5][40]
\end{Arabic}}
\flushleft{\begin{malayalam}
നിനക്കറിഞ്ഞുകൂടേ, ആകാശഭൂമികളുടെ ആധിപത്യം അല്ലാഹുവിനാണെന്ന്? അവനിച്ഛിക്കുന്നവരെ അവന്‍ ശിക്ഷിക്കുന്നു. അവനുദ്ദേശിക്കുന്നവര്‍ക്ക് അവന്‍ പൊറുത്തുകൊടുക്കുകയും ചെയ്യുന്നു. അല്ലാഹു എല്ലാ കാര്യങ്ങള്‍ക്കും കഴിവുറ്റവനാണ്.
\end{malayalam}}
\flushright{\begin{Arabic}
\quranayah[5][41]
\end{Arabic}}
\flushleft{\begin{malayalam}
പ്രവാചകരേ, സത്യനിഷേധത്തില്‍ കുതിച്ചു മുന്നേറുന്നവര്‍ നിന്നെ ദുഃഖിപ്പിക്കാതിരിക്കട്ടെ. അവര്‍ “ഞങ്ങള്‍ വിശ്വസിച്ചിരിക്കുന്നു”വെന്ന് വായകൊണ്ട് വാദിക്കുന്നവരാണ്. എന്നാല്‍ അവരുടെ ഹൃദയങ്ങള്‍ വിശ്വസിച്ചിട്ടില്ല. യഹൂദരില്‍പെട്ടവരോ, അവര്‍ കള്ളത്തിന് കാതോര്‍ക്കുന്നവരാണ്. നിന്റെ അടുത്തുവരാത്ത മറ്റുള്ളവരുടെ വാക്കുകള്‍ക്ക് കാതുകൂര്‍പ്പിക്കുന്നവരും. വേദവാക്യങ്ങളെ അവയുടെ സ്ഥാനങ്ങളില്‍ നിന്ന് അവര്‍ മാറ്റിമറിക്കുന്നു. അവര്‍ പറയുന്നു: "നിങ്ങള്‍ക്ക് ഈ നിയമമാണ് നല്‍കുന്നതെങ്കില്‍ അതു സ്വീകരിക്കുക. അതല്ല നല്‍കുന്നതെങ്കില്‍ നിരസിക്കുക.” അല്ലാഹു ആരെയെങ്കിലും നാശത്തിലകപ്പെടുത്താനുദ്ദേശിച്ചാല്‍ അല്ലാഹുവില്‍ നിന്ന് അയാള്‍ക്ക് യാതൊന്നും നേടിക്കൊടുക്കാന്‍ നിനക്കാവില്ല. അത്തരക്കാരുടെ മനസ്സുകളെ നന്നാക്കാന്‍ അല്ലാഹു ഉദ്ദേശിച്ചിട്ടില്ല. അവര്‍ക്ക് ഇഹലോകത്ത് മാനക്കേടാണുണ്ടാവുക. പരലോകത്തോ കൊടിയ ശിക്ഷയും.
\end{malayalam}}
\flushright{\begin{Arabic}
\quranayah[5][42]
\end{Arabic}}
\flushleft{\begin{malayalam}
അവര്‍ കള്ളത്തിന് കാതോര്‍ക്കുന്നവരാണ്. നിഷിദ്ധധനം ധാരാളമായി തിന്നുന്നവരും. അവര്‍ നിന്റെ അടുത്തുവരികയാണെങ്കില്‍ നീ അവര്‍ക്കിടയില്‍ തീര്‍പ്പുകല്‍പിക്കുകയോ അവരെ അവഗണിക്കുകയോ ചെയ്യുക. അവരെ അവഗണിച്ചാല്‍ നിനക്കൊരു ദ്രോഹവും വരുത്താന്‍ അവര്‍ക്കാവില്ല. എന്നാല്‍ നീ അവര്‍ക്കിടയില്‍ തീര്‍പ്പുകല്‍പിക്കുകയാണെങ്കില്‍ നീതിപൂര്‍വം വിധിക്കുക. സംശയമില്ല; നീതി നടത്തുന്നവരെ അല്ലാഹു സ്നേഹിക്കുന്നു.
\end{malayalam}}
\flushright{\begin{Arabic}
\quranayah[5][43]
\end{Arabic}}
\flushleft{\begin{malayalam}
എന്നാല്‍ അവരെങ്ങനെ നിന്നെ വിധികര്‍ത്താവാക്കും? അവരുടെ വശം തൌറാത്തുണ്ട്; അതില്‍ ദൈവിക നിയമങ്ങളുമുണ്ട്. എന്നിട്ടും അതില്‍നിന്ന് സ്വയം പിന്തിരിഞ്ഞവരാണവര്‍. അവര്‍ വിശ്വാസികളേ അല്ല.
\end{malayalam}}
\flushright{\begin{Arabic}
\quranayah[5][44]
\end{Arabic}}
\flushleft{\begin{malayalam}
നാം തന്നെയാണ് തൌറാത്ത് ഇറക്കിയത്. അതില്‍ വെളിച്ചവും നേര്‍വഴിയുമുണ്ട്. അല്ലാഹുവിന് അടിപ്പെട്ടുജീവിച്ച പ്രവാചകന്മാര്‍ യഹൂദര്‍ക്ക് അതനുസരിച്ച് വിധി നടത്തിയിരുന്നു. പുണ്യപുരുഷന്മാരും പണ്ഡിതന്മാരും അതുതന്നെ ചെയ്തു. കാരണം, അവരെയായിരുന്നു വേദപുസ്തകത്തിന്റെ സംരക്ഷണം ഏല്‍പിച്ചിരുന്നത്. അവരതിന് സാക്ഷികളുമായിരുന്നു. അതിനാല്‍ നിങ്ങള്‍ ജനങ്ങളെ പേടിക്കരുത്. എന്നെ മാത്രം ഭയപ്പെടുക. എന്റെ വചനങ്ങള്‍ നിസ്സാര വിലയ്ക്ക് വില്‍ക്കരുത്. ആര്‍ അല്ലാഹു അവതരിപ്പിച്ച നിയമമനുസരിച്ച് വിധി നടത്തുന്നില്ലയോ, അവര്‍ തന്നെയാണ് അവിശ്വാസികള്‍.
\end{malayalam}}
\flushright{\begin{Arabic}
\quranayah[5][45]
\end{Arabic}}
\flushleft{\begin{malayalam}
നാം അവര്‍ക്ക് അതില്‍ ഇവ്വിധം നിയമം നല്‍കിയിരിക്കുന്നു; ജീവനു ജീവന്‍, കണ്ണിനു കണ്ണ്, മൂക്കിനു മൂക്ക്, ചെവിക്കു ചെവി, പല്ലിനു പല്ല്, എല്ലാ പരിക്കുകള്‍ക്കും തത്തുല്യമായ പ്രതിക്രിയ. എന്നാല്‍ ആരെങ്കിലും മാപ്പ് നല്‍കുകയാണെങ്കില്‍ അത് അവന്നുള്ള പ്രായശ്ചിത്തമാകുന്നു. ആര്‍ അല്ലാഹു അവതരിപ്പിച്ച നിയമമനുസരിച്ച് വിധിക്കുന്നില്ലയോ, അവര്‍ തന്നെയാണ് അതിക്രമികള്‍.
\end{malayalam}}
\flushright{\begin{Arabic}
\quranayah[5][46]
\end{Arabic}}
\flushleft{\begin{malayalam}
ആ പ്രവാചകന്മാര്‍ക്കുശേഷം നാം മര്‍യമിന്റെ മകന്‍ ഈസായെ നിയോഗിച്ചു. അദ്ദേഹം തൌറാത്തില്‍ നിന്ന് തന്റെ മുന്നിലുള്ളവയെ ശരിവെക്കുന്നവനായിരുന്നു. നാം അദ്ദേഹത്തിന് വെളിച്ചവും നേര്‍വഴിയുമുള്ള ഇഞ്ചീല്‍ നല്‍കി. അത് തൌറാത്തില്‍ നിന്ന് അന്നുള്ളവയെ ശരിവെക്കുന്നതായിരുന്നു. ഭക്തന്മാര്‍ക്ക് നേര്‍വഴി കാണിക്കുന്നതും സദുപദേശം നല്‍കുന്നതും.
\end{malayalam}}
\flushright{\begin{Arabic}
\quranayah[5][47]
\end{Arabic}}
\flushleft{\begin{malayalam}
ഇഞ്ചീലിന്റെ അനുയായികള്‍ അല്ലാഹു അതിലവതരിപ്പിച്ച നിയമമനുസരിച്ച് വിധി നടത്തട്ടെ. ആര്‍ അല്ലാഹു അവതരിപ്പിച്ച നിയമമനുസരിച്ച് വിധി നടത്തുന്നില്ലയോ, അവര്‍ തന്നെയാകുന്നു അധര്‍മികള്‍.
\end{malayalam}}
\flushright{\begin{Arabic}
\quranayah[5][48]
\end{Arabic}}
\flushleft{\begin{malayalam}
പ്രവാചകരേ, നിനക്ക് നാമിതാ ഈ വേദപുസ്തകം സത്യസന്ദേശവുമായി അവതരിപ്പിച്ചുതന്നിരിക്കുന്നു. അത് മുന്‍വേദഗ്രന്ഥത്തില്‍ നിന്ന് അതിന്റെ മുന്നിലുള്ളവയെ ശരിവെക്കുന്നതാണ്. അതിനെ ഭദ്രമായി കാത്തുരക്ഷിക്കുന്നതും. അതിനാല്‍ അല്ലാഹു അവതരിപ്പിച്ചുതന്ന നിയമമനുസരിച്ച് നീ അവര്‍ക്കിടയില്‍ വിധി കല്‍പിക്കുക. നിനക്കു വന്നെത്തിയ സത്യത്തെ നിരാകരിച്ച് അവരുടെ തന്നിഷ്ടങ്ങളെ പിന്‍പറ്റരുത്. നിങ്ങളില്‍ ഓരോ വിഭാഗത്തിനും നാം ഓരോ നിയമവ്യവസ്ഥയും കര്‍മരീതിയും നിശ്ചയിച്ചു തന്നിട്ടുണ്ട്. അല്ലാഹു ഇച്ഛിച്ചിരുന്നെങ്കില്‍ നിങ്ങളെ ഒന്നാകെ ഒരൊറ്റ സമുദായമാക്കുമായിരുന്നു. അങ്ങനെ ചെയ്യാത്തത് നിങ്ങള്‍ക്ക് അവന്‍ നല്‍കിയതില്‍ നിങ്ങളെ പരീക്ഷിക്കാനാണ്. അതിനാല്‍ മഹത്കൃത്യങ്ങളില്‍ മത്സരിച്ചു മുന്നേറുക. നിങ്ങളുടെയൊക്കെ മടക്കം അല്ലാഹുവിങ്കലേക്കാണ്. നിങ്ങള്‍ ഭിന്നിച്ചുകൊണ്ടിരുന്ന കാര്യങ്ങളുടെയെല്ലാം നിജസ്ഥിതി അപ്പോള്‍ അവന്‍ നിങ്ങളെ അറിയിക്കുന്നതാണ്.
\end{malayalam}}
\flushright{\begin{Arabic}
\quranayah[5][49]
\end{Arabic}}
\flushleft{\begin{malayalam}
അല്ലാഹു ഇറക്കിത്തന്ന നിയമമനുസരിച്ച് നീ അവര്‍ക്കിടയില്‍ വിധി കല്‍പിക്കുക. നീ അവരുടെ തോന്നിവാസങ്ങളെ പിന്‍പറ്റരുത്. അല്ലാഹു നിനക്ക് ഇറക്കിത്തന്ന ഏതെങ്കിലും നിയമങ്ങളില്‍ നിന്ന് അവര്‍ നിന്നെ തെറ്റിച്ചുകളയുന്നതിനെക്കുറിച്ച് ജാഗ്രത പുലര്‍ത്തുക. അഥവാ, അവര്‍ പിന്തിരിഞ്ഞുപോകുന്നുവെങ്കില്‍ അറിയുക: അവരുടെ ചില തെറ്റുകള്‍ കാരണമായി അവരെ ആപത്തിലകപ്പെടുത്തണമെന്നാണ് അല്ലാഹു ഉദ്ദേശിക്കുന്നത്. തീര്‍ച്ചയായും ജനങ്ങളിലേറെ പേരും കടുത്ത ധിക്കാരികളാണ്.
\end{malayalam}}
\flushright{\begin{Arabic}
\quranayah[5][50]
\end{Arabic}}
\flushleft{\begin{malayalam}
അനിസ്ലാമിക വ്യവസ്ഥയുടെ വിധിയാണോ അവരാഗ്രഹിക്കുന്നത്. അടിയുറച്ച സത്യവിശ്വാസികള്‍ക്ക് അല്ലാഹുവെക്കാള്‍ നല്ല വിധികര്‍ത്താവായി ആരുണ്ട്?
\end{malayalam}}
\flushright{\begin{Arabic}
\quranayah[5][51]
\end{Arabic}}
\flushleft{\begin{malayalam}
വിശ്വസിച്ചവരേ, ജൂതന്മാരെയും ക്രിസ്ത്യാനികളെയും നിങ്ങള്‍ ആത്മമിത്രങ്ങളാക്കരുത്. അവരന്യോന്യം ആത്മമിത്രങ്ങളാണ്. നിങ്ങളിലാരെങ്കിലും അവരെ ആത്മമിത്രങ്ങളാക്കുന്നുവെങ്കില്‍ അവനും അവരില്‍പെട്ടവനായിത്തീരും. അക്രമികളായ ആളുകളെ അല്ലാഹു നേര്‍വഴിയിലാക്കുകയില്ല; തീര്‍ച്ച.
\end{malayalam}}
\flushright{\begin{Arabic}
\quranayah[5][52]
\end{Arabic}}
\flushleft{\begin{malayalam}
എന്നാല്‍ ദീനംപിടിച്ച മനസ്സുള്ളവര്‍ അവരുമായി കൂട്ടുകൂടുന്നതിന് തിടുക്കം കൂട്ടുന്നതായി കാണാം. തങ്ങള്‍ക്കു വല്ല വിപത്തും വന്നുപെട്ടേക്കുമോയെന്ന് ആശങ്കയുണ്ടെന്നാണ് അതിനവര്‍ കാരണം പറയുക. എന്നാല്‍ അല്ലാഹു നിങ്ങള്‍ക്ക് നിര്‍ണായക വിജയം നല്‍കിയേക്കാം. അല്ലെങ്കില്‍ അവന്റെ ഭാഗത്തുനിന്ന് മറ്റുവല്ല നടപടിയും ഉണ്ടായേക്കാം. അപ്പോള്‍ അവര്‍ തങ്ങളുടെ മനസ്സ് മറച്ചുവെക്കുന്നതിനെ സംബന്ധിച്ച് ഖേദിക്കുന്നവരായിത്തീരും.
\end{malayalam}}
\flushright{\begin{Arabic}
\quranayah[5][53]
\end{Arabic}}
\flushleft{\begin{malayalam}
അന്നേരം സത്യവിശ്വാസികള്‍ ചോദിക്കും: “ഞങ്ങള്‍ നിങ്ങളോടൊപ്പം തന്നെയാണെ”ന്ന് അല്ലാഹുവിന്റെ പേരില്‍ ആണയിട്ടുപറഞ്ഞിരുന്ന അക്കൂട്ടര്‍ തന്നെയാണോ ഇവര്‍? അവരുടെ പ്രവര്‍ത്തനങ്ങള്‍ പാഴായിരിക്കുന്നു. അങ്ങനെ അവര്‍ പരാജിതരാവുകയും ചെയ്തിരിക്കുന്നു.
\end{malayalam}}
\flushright{\begin{Arabic}
\quranayah[5][54]
\end{Arabic}}
\flushleft{\begin{malayalam}
വിശ്വസിച്ചവരേ, നിങ്ങളിലാരെങ്കിലും തന്റെ മതം ഉപേക്ഷിച്ച് പോവുന്നുവെങ്കില്‍ അല്ലാഹു മറ്റൊരു ജനവിഭാഗത്തെ പകരം കൊണ്ടുവരും. അല്ലാഹു ഇഷ്ടപ്പെടുകയും അല്ലാഹുവെ ഇഷ്ടപ്പെടുകയും ചെയ്യുന്ന ഒരു വിഭാഗത്തെ. അവര്‍ വിശ്വാസികളോട് വിനയവും സത്യനിഷേധികളോട് പ്രതാപവും കാണിക്കുന്നവരായിരിക്കും. ദൈവമാര്‍ഗത്തില്‍ സമരം നടത്തുന്നവരും ഒരാളുടെയും ആക്ഷേപത്തെ ഭയപ്പെടാത്തവരുമായിരിക്കും. അത് അല്ലാഹുവിന്റെ അനുഗ്രഹമാണ്. അവനിച്ഛിക്കുന്നവര്‍ക്ക് അവനതു നല്‍കുന്നു. അല്ലാഹു വിപുലമായ ഔദാര്യമുടയവനാണ്. എല്ലാം അറിയുന്നവനും.
\end{malayalam}}
\flushright{\begin{Arabic}
\quranayah[5][55]
\end{Arabic}}
\flushleft{\begin{malayalam}
നിങ്ങളുടെ ആത്മമിത്രങ്ങള്‍ അല്ലാഹുവും അവന്റെ ദൂതനുമാണ്. നമസ്കാരം നിഷ്ഠയോടെ നിര്‍വഹിക്കുകയും സകാത്ത് നല്‍കുകയും അല്ലാഹുവെ മാത്രം നമിക്കുകയും ചെയ്യുന്ന സത്യവിശ്വാസികളും.
\end{malayalam}}
\flushright{\begin{Arabic}
\quranayah[5][56]
\end{Arabic}}
\flushleft{\begin{malayalam}
അല്ലാഹുവെയും അവന്റെ ദൂതനെയും സത്യവിശ്വാസികളെയും ആത്മമിത്രങ്ങളാക്കുന്നവര്‍ അറിയട്ടെ: അല്ലാഹുവിന്റെ കക്ഷി തന്നെയാണ് വിജയം വരിക്കുന്നവര്‍.
\end{malayalam}}
\flushright{\begin{Arabic}
\quranayah[5][57]
\end{Arabic}}
\flushleft{\begin{malayalam}
വിശ്വസിച്ചവരേ, നിങ്ങളുടെ മതത്തെ പരിഹസിക്കുകയും കളിയാക്കുകയും ചെയ്തുപോന്ന, നിങ്ങള്‍ക്കുമുമ്പെ വേദം നല്‍കപ്പെട്ടവരെയും സത്യനിഷേധികളെയും ഉറ്റമിത്രങ്ങളാക്കരുത്. നിങ്ങള്‍ സത്യവിശ്വാസികളെങ്കില്‍ അല്ലാഹുവോട് ഭയഭക്തിയുള്ളവരാവുക.
\end{malayalam}}
\flushright{\begin{Arabic}
\quranayah[5][58]
\end{Arabic}}
\flushleft{\begin{malayalam}
നിങ്ങള്‍ നമസ്കാരത്തിന് വിളിച്ചാല്‍ അവരതിനെ പരിഹാസവും കളിയുമാക്കുന്നു. അവര്‍ ആലോചിച്ചറിയാത്ത ജനമായതിനാലാണത്.
\end{malayalam}}
\flushright{\begin{Arabic}
\quranayah[5][59]
\end{Arabic}}
\flushleft{\begin{malayalam}
ചോദിക്കുക: വേദക്കാരേ, നിങ്ങള്‍ ഞങ്ങളോടു ശത്രുത പുലര്‍ത്താന്‍ വല്ല കാരണവുമുണ്ടോ, അല്ലാഹുവിലും ഞങ്ങള്‍ക്ക് ഇറക്കിക്കിട്ടിയതിലും ഞങ്ങള്‍ക്കുമുമ്പ് ഇറക്കപ്പെട്ടതിലും ഞങ്ങള്‍ വിശ്വസിക്കുന്നുവെന്നതല്ലാതെ? നിങ്ങളിലേറെപ്പേരും ധിക്കാരികളാണെന്നതും?
\end{malayalam}}
\flushright{\begin{Arabic}
\quranayah[5][60]
\end{Arabic}}
\flushleft{\begin{malayalam}
ചോദിക്കുക: അല്ലാഹുവിങ്കല്‍ അതിനെക്കാള്‍ ഹീനമായ പ്രതിഫലമുള്ളവരെപ്പറ്റി ഞാന്‍ നിങ്ങള്‍ക്ക് പറഞ്ഞുതരട്ടെയോ? അല്ലാഹു ശപിച്ചവര്‍; അല്ലാഹു കോപിച്ചവര്‍; അല്ലാഹു കുരങ്ങന്മാരും പന്നികളുമാക്കിയവര്‍; ദൈവേതര ശക്തികള്‍ക്ക് അടിപ്പെട്ടവര്‍- ഇവരൊക്കെയാണ് ഏറ്റം നീചമായ സ്ഥാനക്കാര്‍. നേര്‍വഴിയില്‍നിന്ന് തീര്‍ത്തും തെറ്റിപ്പോയവരും അവര്‍ തന്നെ.
\end{malayalam}}
\flushright{\begin{Arabic}
\quranayah[5][61]
\end{Arabic}}
\flushleft{\begin{malayalam}
നിങ്ങളുടെ അടുത്ത് വരുമ്പോള്‍ “ഞങ്ങള്‍ വിശ്വസിച്ചിരിക്കുന്നു”വെന്ന് അവര്‍ പറയുന്നു. എന്നാല്‍ ഉറപ്പായും അവര്‍ വരുന്നത് സത്യനിഷേധവുമായാണ്. തിരിച്ചുപോവുന്നതും സത്യനിഷേധവുമായിത്തന്നെ. അവര്‍ മറച്ചുവെക്കുന്നവയെക്കുറിച്ചൊ ക്കെയും നന്നായറിയുന്നവനാണ് അല്ലാഹു.
\end{malayalam}}
\flushright{\begin{Arabic}
\quranayah[5][62]
\end{Arabic}}
\flushleft{\begin{malayalam}
അവരില്‍ ഒട്ടേറെയാളുകള്‍ പാപവൃത്തികളിലും അതിക്രമങ്ങളിലും ആവേശത്തോടെ മുന്നേറുന്നതും നിഷിദ്ധ ധനം തിന്നുതിമര്‍ക്കുന്നതും നിനക്കു കാണാം. അവര്‍ ചെയ്തുകൊണ്ടിരിക്കുന്നത് നന്നെ നീചം തന്നെ.
\end{malayalam}}
\flushright{\begin{Arabic}
\quranayah[5][63]
\end{Arabic}}
\flushleft{\begin{malayalam}
അവരുടെ പാപഭാഷണങ്ങളെയും നിഷിദ്ധ ഭോജനത്തെയും പുണ്യവാളന്മാരും പണ്ഡിതന്മാരും തടയാത്തതെന്ത്? അവര്‍ ചെയ്തുകൊണ്ടിരിക്കുന്നത് വളരെ ചീത്ത തന്നെ.
\end{malayalam}}
\flushright{\begin{Arabic}
\quranayah[5][64]
\end{Arabic}}
\flushleft{\begin{malayalam}
ദൈവത്തിന്റെ കൈകള്‍ കെട്ടിപ്പൂട്ടിയിരിക്കുകയാണെന്ന് ജൂതന്മാര്‍ പറയുന്നു. കെട്ടിപ്പൂട്ടിയത് അവരുടെ കൈകള്‍ തന്നെയാണ്. അങ്ങനെ പറഞ്ഞത് കാരണം അവര്‍ അഭിശപ്തരായിരിക്കുന്നു. എന്നാല്‍ അല്ലാഹുവിന്റെ ഹസ്തങ്ങള്‍ തുറന്നുവെച്ചവയാണ്. അവനിച്ഛിക്കും പോലെ അവന്‍ ചെലവഴിക്കുന്നു. നിനക്ക് നിന്റെ നാഥനില്‍നിന്ന് അവതരിച്ചുകിട്ടിയ സന്ദേശം അവരില്‍ അധിക പേരുടെയും ധിക്കാരവും സത്യനിഷേധവും വര്‍ധിപ്പിക്കുക തന്നെ ചെയ്യും. അവര്‍ക്കിടയില്‍ ഉയിര്‍ത്തെഴുന്നേല്‍പുനാള്‍ വരെ നാം പകയും വിദ്വേഷവും ഉളവാക്കിയിരിക്കുന്നു. അവര്‍ യുദ്ധത്തീ ആളിക്കത്തിക്കുമ്പോഴെല്ലാം അല്ലാഹു അത് ഊതിക്കെടുത്തുന്നു. അവര്‍ ഭൂമിയില്‍ കുഴപ്പമുണ്ടാക്കാനാണ് ശ്രമിക്കുന്നത്. കുഴപ്പക്കാരെ അല്ലാഹു ഇഷ്ടപ്പെടുന്നില്ല.
\end{malayalam}}
\flushright{\begin{Arabic}
\quranayah[5][65]
\end{Arabic}}
\flushleft{\begin{malayalam}
വേദക്കാര്‍ വിശ്വസിക്കുകയും ഭക്തി പുലര്‍ത്തുകയും ചെയ്തിരുന്നുവെങ്കില്‍ ഉറപ്പായും അവരുടെ തിന്മകള്‍ നാം മായ്ച്ചുകളയുകയും അവരെ അനുഗൃഹീതമായ സ്വര്‍ഗീയാരാമങ്ങളില്‍ പ്രവേശിപ്പിക്കുകയും ചെയ്യുമായിരുന്നു.
\end{malayalam}}
\flushright{\begin{Arabic}
\quranayah[5][66]
\end{Arabic}}
\flushleft{\begin{malayalam}
തൌറാത്തും ഇഞ്ചീലും, തങ്ങളുടെ നാഥനില്‍ നിന്ന് ഇറക്കിക്കിട്ടിയ മറ്റു സന്ദേശങ്ങളും യഥാവിധി പ്രയോഗത്തില്‍ വരുത്തിയിരുന്നുവെങ്കില്‍ അവര്‍ക്ക് മുകള്‍ഭാഗത്തുനിന്നും കാല്‍ച്ചുവട്ടില്‍നിന്നും ആഹാരം കിട്ടുമായിരുന്നു. അവരില്‍ നേര്‍വഴി കൈക്കൊണ്ട ചിലരുണ്ട്. എന്നാല്‍ ഏറെ പേരുടെയും ചെയ്തികള്‍ തീര്‍ത്തും നീചമാണ്.
\end{malayalam}}
\flushright{\begin{Arabic}
\quranayah[5][67]
\end{Arabic}}
\flushleft{\begin{malayalam}
ദൈവദൂതരേ, നിന്റെ നാഥനില്‍നിന്ന് നിനക്ക് ഇറക്കിക്കിട്ടിയത് ജനങ്ങള്‍ക്കെത്തിച്ചുകൊടുക്കുക. അങ്ങനെ ചെയ്യുന്നില്ലെങ്കില്‍ നീ അവന്‍ ഏല്‍പിച്ച ദൌത്യം നിറവേറ്റാത്തവനായിത്തീരും. ജനങ്ങളില്‍നിന്ന് അല്ലാഹു നിന്നെ രക്ഷിക്കും. സത്യനിഷേധികളായ ജനത്തെ അല്ലാഹു നേര്‍വഴിയിലാക്കുകയില്ല.
\end{malayalam}}
\flushright{\begin{Arabic}
\quranayah[5][68]
\end{Arabic}}
\flushleft{\begin{malayalam}
പറയുക: വേദവാഹകരേ, തൌറാത്തും ഇഞ്ചീലും നിങ്ങളുടെ നാഥനില്‍നിന്ന് നിങ്ങള്‍ക്ക് അവതരിച്ചുകിട്ടിയ സന്ദേശങ്ങളും യഥാവിധി നിലനിര്‍ത്തുംവരെ നിങ്ങളുടെ നിലപാടുകള്‍ക്ക് ഒരടിസ്ഥാനവും ഉണ്ടാവുകയില്ല. എന്നാല്‍ നിനക്ക് നിന്റെ നാഥനില്‍നിന്ന് അവതരിച്ചുകിട്ടിയ സന്ദേശം അവരില്‍ ഏറെ പേരുടെയും ധിക്കാരവും സത്യനിഷേധവും വര്‍ധിപ്പിക്കുകതന്നെ ചെയ്യും. അതിനാല്‍ നീ സത്യനിഷേധികളായ ജനത്തെയോര്‍ത്ത് ദുഃഖിക്കേണ്ടതില്ല.
\end{malayalam}}
\flushright{\begin{Arabic}
\quranayah[5][69]
\end{Arabic}}
\flushleft{\begin{malayalam}
സത്യവിശ്വാസികളോ യഹൂദരോ സാബികളോ ക്രിസ്ത്യാനികളോ ആരാവട്ടെ; അല്ലാഹുവിലും അന്ത്യദിനത്തിലും വിശ്വസിക്കുകയും സല്‍ക്കര്‍മങ്ങള്‍ പ്രവര്‍ത്തിക്കുകയും ചെയ്യുന്നവര്‍ ഒന്നും പേടിക്കേണ്ടതില്ല. അവര്‍ ദുഃഖിക്കേണ്ടിവരികയുമില്ല.
\end{malayalam}}
\flushright{\begin{Arabic}
\quranayah[5][70]
\end{Arabic}}
\flushleft{\begin{malayalam}
ഇസ്രയേല്‍ മക്കളോട് നാം കരാര്‍ വാങ്ങിയിട്ടുണ്ട്. അവരിലേക്ക് നാം ദൂതന്മാരെ നിയോഗിക്കുകയും ചെയ്തു. എന്നാല്‍ ഓരോ ദൈവദൂതനും അവരുടെ മനസ്സിനിണങ്ങാത്ത സന്ദേശങ്ങളുമായി അവരുടെ അടുത്ത് ചെന്നപ്പോഴെല്ലാം അവര്‍ ആ ദൈവദൂതന്മാരില്‍ ചിലരെ തള്ളിപ്പറയുകയും മറ്റുചിലരെ കൊല്ലുകയുമാണുണ്ടാത്.
\end{malayalam}}
\flushright{\begin{Arabic}
\quranayah[5][71]
\end{Arabic}}
\flushleft{\begin{malayalam}
ഇതിനാല്‍ ഒരു കുഴപ്പവുമുണ്ടാവില്ലെന്ന് അവര്‍ കണക്കുകൂട്ടി. അങ്ങനെ അവര്‍ അന്ധരും ബധിരരുമായിത്തീര്‍ന്നു. പിന്നീട് അല്ലാഹു അവരുടെ പശ്ചാത്താപം സ്വീകരിച്ചു. എന്നാല്‍ പിന്നെയും അവരിലേറെപ്പേരും അന്ധരും ബധിരരുമാവുകയാണുണ്ടായത്. അവര്‍ ചെയ്യുന്നതെല്ലാം സൂക്ഷ്മമായി കണ്ടറിയുന്നവനാണ് അല്ലാഹു.
\end{malayalam}}
\flushright{\begin{Arabic}
\quranayah[5][72]
\end{Arabic}}
\flushleft{\begin{malayalam}
മര്‍യമിന്റെ മകന്‍ മസീഹ് ദൈവം തന്നെയെന്ന് വാദിച്ചവര്‍ ഉറപ്പായും സത്യനിഷേധികളായിരിക്കുന്നു. യഥാര്‍ഥത്തില്‍ മസീഹ് പറഞ്ഞതിതാണ്: "ഇസ്രയേല്‍ മക്കളേ, എന്റെയും നിങ്ങളുടെയും നാഥനായ അല്ലാഹുവെ മാത്രം ആരാധിക്കുക. അല്ലാഹുവില്‍ ആരെയെങ്കിലും പങ്കുചേര്‍ക്കുന്നവന് അല്ലാഹു സ്വര്‍ഗം നിഷിദ്ധമാക്കും; തീര്‍ച്ച. അവന്റെ വാസസ്ഥലം നരകമാണ്. അക്രമികള്‍ക്ക് സഹായികളുണ്ടാവില്ല.”
\end{malayalam}}
\flushright{\begin{Arabic}
\quranayah[5][73]
\end{Arabic}}
\flushleft{\begin{malayalam}
ദൈവം മൂവരില്‍ ഒരുവനാണെന്ന് വാദിച്ചവര്‍ തീര്‍ച്ചയായും സത്യനിഷേധികള്‍ തന്നെ. കാരണം, ഏകനായ അല്ലാഹുവല്ലാതെ ദൈവമില്ല. തങ്ങളുടെ വിടുവാദങ്ങളില്‍ നിന്ന് അവര്‍ വിരമിക്കുന്നില്ലെങ്കില്‍ അവരിലെ സത്യനിഷേധികളെ നോവേറിയ ശിക്ഷ ബാധിക്കുകതന്നെ ചെയ്യും.
\end{malayalam}}
\flushright{\begin{Arabic}
\quranayah[5][74]
\end{Arabic}}
\flushleft{\begin{malayalam}
ഇനിയും അവര്‍ അല്ലാഹുവിലേക്ക് പശ്ചാത്തപിച്ചുമടങ്ങുകയും അവനോട് മാപ്പിരക്കുകയും ചെയ്യുന്നില്ലേ? അല്ലാഹു ഏറെ പൊറുക്കുന്നവനും പരമദയാലുവുമല്ലോ.
\end{malayalam}}
\flushright{\begin{Arabic}
\quranayah[5][75]
\end{Arabic}}
\flushleft{\begin{malayalam}
മര്‍യമിന്റെ മകന്‍ മസീഹ് ഒരു ദൈവദൂതന്‍ മാത്രമാണ്. അദ്ദേഹത്തിനു മുമ്പും നിരവധി ദൈവദൂതന്മാര്‍ ഉണ്ടായിട്ടുണ്ട്. അദ്ദേഹത്തിന്റെ മാതാവ് സത്യവതിയായിരുന്നു. ഇരുവരും ആഹാരം കഴിക്കുന്നവരുമായിരുന്നു. നോക്കൂ: നാം അവര്‍ക്ക് എങ്ങനെയൊക്കെ തെളിവുകള്‍ വിവരിച്ചുകൊടുക്കുന്നുവെന്ന്. ചിന്തിച്ചുനോക്കൂ; എന്നിട്ടും അവരെങ്ങനെയാണ് തെന്നിമാറിപ്പോകുന്നത്.
\end{malayalam}}
\flushright{\begin{Arabic}
\quranayah[5][76]
\end{Arabic}}
\flushleft{\begin{malayalam}
ചോദിക്കുക: നിങ്ങള്‍ക്ക് ഉപദ്രവമോ ഉപകാരമോ ചെയ്യാനാവാത്ത വസ്തുക്കളെയാണോ അല്ലാഹുവെക്കൂടാതെ നിങ്ങള്‍ ആരാധിക്കുന്നത്? എന്നാല്‍ അല്ലാഹു എല്ലാം കേള്‍ക്കുന്നവനും അറിയുന്നവനുമാകുന്നു.
\end{malayalam}}
\flushright{\begin{Arabic}
\quranayah[5][77]
\end{Arabic}}
\flushleft{\begin{malayalam}
പറയുക: വേദക്കാരേ, നിങ്ങള്‍ നിങ്ങളുടെ മതകാര്യങ്ങളില്‍ അന്യായമായി അതിരുകവിയാതിരിക്കുക. നേരത്തെ പിഴച്ചുപോവുകയും വളരെ പേരെ പിഴപ്പിക്കുകയും നേര്‍വഴിയില്‍നിന്ന് തെന്നിമാറുകയും ചെയ്ത ജനത്തിന്റെ തന്നിഷ്ടങ്ങളെ നിങ്ങള്‍ പിന്‍പറ്റരുത്.
\end{malayalam}}
\flushright{\begin{Arabic}
\quranayah[5][78]
\end{Arabic}}
\flushleft{\begin{malayalam}
ഇസ്രയേല്‍ മക്കളിലെ സത്യനിഷേധികളെ ദാവൂദും മര്‍യമിന്റെ മകന്‍ ഈസായും ശപിച്ചിരിക്കുന്നു. അവര്‍ അനുസരണക്കേട് കാണിക്കുകയും അതിക്രമം പ്രവര്‍ത്തിക്കുകയും ചെയ്തതിനാലാണത്.
\end{malayalam}}
\flushright{\begin{Arabic}
\quranayah[5][79]
\end{Arabic}}
\flushleft{\begin{malayalam}
അവര്‍ ചെയ്തുകൊണ്ടിരുന്ന ദുര്‍വൃത്തികളെ അവരന്യോന്യം വിലക്കിയിരുന്നില്ല. അവര്‍ ചെയ്തുകൊണ്ടിരുന്നത് തീര്‍ത്തും നീചമാണ്.
\end{malayalam}}
\flushright{\begin{Arabic}
\quranayah[5][80]
\end{Arabic}}
\flushleft{\begin{malayalam}
അവരിലേറെപേരും സത്യനിഷേധികളുമായി ഉറ്റചങ്ങാത്തം പുലര്‍ത്തുന്നത് നിനക്കു കാണാം. അവര്‍ തങ്ങള്‍ക്കായി തയ്യാര്‍ ചെയ്തുവച്ചത് വളരെ ചീത്ത തന്നെ. അല്ലാഹു അവരോട് കോപിച്ചിരിക്കുന്നു. അവര്‍ എക്കാലവും ശിക്ഷയനുഭവിക്കുന്നവരായിരിക്കും.
\end{malayalam}}
\flushright{\begin{Arabic}
\quranayah[5][81]
\end{Arabic}}
\flushleft{\begin{malayalam}
അല്ലാഹുവിലും പ്രവാചകനിലും അദ്ദേഹത്തിന് ഇറക്കിക്കിട്ടിയതിലും വിശ്വസിച്ചിരുന്നുവെങ്കില്‍ അവരൊരിക്കലും സത്യനിഷേധികളുമായി ഇവ്വിധം ചങ്ങാത്തം സ്ഥാപിക്കുമായിരുന്നില്ല. എന്നാല്‍ അവരിലേറെ പേരും ധിക്കാരികളാകുന്നു.
\end{malayalam}}
\flushright{\begin{Arabic}
\quranayah[5][82]
\end{Arabic}}
\flushleft{\begin{malayalam}
മനുഷ്യരില്‍ സത്യവിശ്വാസികളോട് ഏറ്റവും കൂടുതല്‍ ശത്രുതയുള്ളവര്‍ യഹൂദരും ബഹുദൈവാരാധകരുമാണെന്ന് നിശ്ചയമായും നിനക്ക് കാണാം; ഞങ്ങള്‍ ക്രിസ്ത്യാനികളാണ് എന്നു പറഞ്ഞവരാണ് വിശ്വാസികളോട് കൂടുതല്‍ സ്നേഹമുള്ളവരെന്നും. അവരില്‍ പണ്ഡിതന്മാരും ലോകപരിത്യാഗികളായ പുരോഹിതന്മാരുമുണ്ടെന്നതും അവര്‍ അഹന്ത നടിക്കുന്നില്ലെന്നതുമാണിതിനു കാരണം.
\end{malayalam}}
\flushright{\begin{Arabic}
\quranayah[5][83]
\end{Arabic}}
\flushleft{\begin{malayalam}
സത്യം മനസ്സിലായതിനാല്‍, ദൈവദൂതന് അവതീര്‍ണമായ വചനങ്ങള്‍ കേള്‍ക്കുമ്പോള്‍ അവരുടെ കണ്ണുകളില്‍ നിന്ന് കണ്ണീരൊഴുകുന്നത് നിനക്കു കാണാം. അവരിങ്ങനെ പ്രാര്‍ഥിക്കുന്നു: "ഞങ്ങളുടെ നാഥാ! ഞങ്ങള്‍ വിശ്വസിച്ചിരിക്കുന്നു. അതിനാല്‍ ഞങ്ങളെയും നീ സത്യസാക്ഷികളുടെ കൂട്ടത്തില്‍ പെടുത്തേണമേ.
\end{malayalam}}
\flushright{\begin{Arabic}
\quranayah[5][84]
\end{Arabic}}
\flushleft{\begin{malayalam}
"ഞങ്ങളുടെ നാഥന്‍ ഞങ്ങളെ സച്ചരിതരിലുള്‍പ്പെടുത്തണമെന്ന് ഞങ്ങളാഗ്രഹിച്ചുകൊണ്ടിരിക്കെ ഞങ്ങളുടെ നാഥനിലും ഞങ്ങള്‍ക്കു വന്നെത്തിയ സത്യത്തിലും ഞങ്ങളെന്തിനു വിശ്വസിക്കാതിരിക്കണം?”
\end{malayalam}}
\flushright{\begin{Arabic}
\quranayah[5][85]
\end{Arabic}}
\flushleft{\begin{malayalam}
അവരിങ്ങനെ പ്രാര്‍ഥിച്ചതിനാല്‍ അല്ലാഹു അവര്‍ക്ക് താഴ്ഭാഗത്തൂടെ അരുവികളൊഴുകുന്ന സ്വര്‍ഗീയാരാമങ്ങള്‍ പ്രതിഫലമായി നല്‍കും. അവരതില്‍ സ്ഥിരവാസികളായിരിക്കും. സല്‍ക്കര്‍മികള്‍ക്കുള്ള പ്രതിഫലമാണിത്.
\end{malayalam}}
\flushright{\begin{Arabic}
\quranayah[5][86]
\end{Arabic}}
\flushleft{\begin{malayalam}
സത്യത്തെ നിഷേധിക്കുകയും നമ്മുടെ വചനങ്ങളെ തള്ളിപ്പറയുകയും ചെയ്തവര്‍ തന്നെയാണ് നരകാവകാശികള്‍.
\end{malayalam}}
\flushright{\begin{Arabic}
\quranayah[5][87]
\end{Arabic}}
\flushleft{\begin{malayalam}
വിശ്വസിച്ചവരേ, അല്ലാഹു നിങ്ങള്‍ക്ക് അനുവദിച്ചുതന്ന വിശിഷ്ട വസ്തുക്കളെ നിങ്ങള്‍ നിഷിദ്ധമാക്കരുത്. നിങ്ങള്‍ അതിരുകവിയരുത്. അതിരുകവിയുന്നവരെ അല്ലാഹു ഒട്ടും ഇഷ്ടപ്പെടുന്നില്ല.
\end{malayalam}}
\flushright{\begin{Arabic}
\quranayah[5][88]
\end{Arabic}}
\flushleft{\begin{malayalam}
അല്ലാഹു നിങ്ങള്‍ക്കു നല്‍കിയവയില്‍ നിന്ന് അനുവദനീയമായവയും നല്ലതും നിങ്ങള്‍ തിന്നുകൊള്ളുക. നിങ്ങള്‍ വിശ്വസിക്കുന്ന അല്ലാഹുവുണ്ടല്ലോ, അവനോട് നിങ്ങള്‍ ഭക്തി പുലര്‍ത്തുക.
\end{malayalam}}
\flushright{\begin{Arabic}
\quranayah[5][89]
\end{Arabic}}
\flushleft{\begin{malayalam}
ഓര്‍ക്കാതെ ചെയ്തുപോകുന്ന ശപഥങ്ങളുടെ പേരില്‍ അല്ലാഹു നിങ്ങളെ പിടികൂടുകയില്ല. എന്നാല്‍ നിങ്ങള്‍ കരുതിക്കൂട്ടി ചെയ്യുന്ന ശപഥങ്ങളുടെ പേരില്‍ അവന്‍ നിങ്ങളെ പിടികൂടും. അപ്പോള്‍ ശപഥ ലംഘനത്തിനുള്ള പ്രായശ്ചിത്തം ഇതാകുന്നു: പത്ത് അഗതികള്‍ക്ക്, നിങ്ങള്‍ നിങ്ങളുടെ കുടുംബത്തെ തീറ്റിപ്പോറ്റുന്ന സാമാന്യനിലവാരത്തിലുള്ള ആഹാരം നല്‍കുക. അല്ലെങ്കില്‍ അവര്‍ക്ക് വസ്ത്രം നല്‍കുക. അതുമല്ലെങ്കില്‍ ഒരടിമയെ മോചിപ്പിക്കുക. ഇതിനൊന്നും സാധിക്കാത്തവര്‍ മൂന്നുദിവസം നോമ്പെടുക്കട്ടെ. ഇതാണ് സത്യം ചെയ്ത ശേഷം അത് ലംഘിച്ചാലുള്ള പ്രായശ്ചിത്തം. നിങ്ങളുടെ ശപഥങ്ങള്‍ നിങ്ങള്‍ പാലിക്കുക. അവ്വിധം അല്ലാഹു തന്റെ വചനങ്ങള്‍ നിങ്ങള്‍ക്ക് വിവരിച്ചുതരുന്നു. നിങ്ങള്‍ നന്ദിയുള്ളവരാകാന്‍.
\end{malayalam}}
\flushright{\begin{Arabic}
\quranayah[5][90]
\end{Arabic}}
\flushleft{\begin{malayalam}
വിശ്വസിച്ചവരേ, മദ്യവും ചൂതും പ്രതിഷ്ഠകളും ഭാഗ്യപരീക്ഷണത്തിനുള്ള അമ്പുകളും പൈശാചികവൃത്തികളില്‍പെട്ട മാലിന്യങ്ങളാണ്. അതിനാല്‍ നിങ്ങള്‍ അവയൊക്കെ ഒഴിവാക്കുക. നിങ്ങള്‍ വിജയിച്ചേക്കാം.
\end{malayalam}}
\flushright{\begin{Arabic}
\quranayah[5][91]
\end{Arabic}}
\flushleft{\begin{malayalam}
മദ്യത്തിലൂടെയും ചൂതാട്ടത്തിലൂടെയും നിങ്ങള്‍ക്കിടയില്‍ വെറുപ്പും വിദ്വേഷവും വളര്‍ത്താനും, അല്ലാഹുവെ ഓര്‍ക്കുന്നതില്‍നിന്നും നമസ്കാരത്തില്‍ നിന്നും നിങ്ങളെ തടയാനുമാണ് പിശാച് ആഗ്രഹിക്കുന്നത്. അതിനാല്‍ നിങ്ങള്‍ ആ തിന്മകളില്‍നിന്ന് വിരമിക്കാനൊരുക്കമുണ്ടോ?
\end{malayalam}}
\flushright{\begin{Arabic}
\quranayah[5][92]
\end{Arabic}}
\flushleft{\begin{malayalam}
അല്ലാഹുവിനെയും അവന്റെ ദൂതനെയും അനുസരിക്കുക. ജാഗ്രത പുലര്‍ത്തുകയും ചെയ്യുക. അഥവാ നിങ്ങള്‍ പിന്തിരിയുകയാണെങ്കില്‍ അറിയുക: നമ്മുടെ ദൂതന്റെ കടമ ദിവ്യസന്ദേശം വ്യക്തമായി എത്തിച്ചുതരല്‍ മാത്രമാണ്.
\end{malayalam}}
\flushright{\begin{Arabic}
\quranayah[5][93]
\end{Arabic}}
\flushleft{\begin{malayalam}
സത്യവിശ്വാസം സ്വീകരിക്കുകയും സല്‍ക്കര്‍മങ്ങള്‍ പ്രവര്‍ത്തിക്കുകയും ചെയ്തവര്‍ക്ക് അവര്‍ നേരത്തെ നിഷിദ്ധം ഭക്ഷിച്ചതിന്റെ പേരില്‍ കുറ്റമില്ല. എന്നാല്‍ അവര്‍ ഭക്തി പുലര്‍ത്തുകയും സത്യവിശ്വാസം സ്വീകരിക്കുകയും സല്‍ക്കര്‍മങ്ങള്‍ പ്രവര്‍ത്തിക്കുകയും പിന്നെയും സൂക്ഷ്മത പാലിക്കുകയും സത്യവിശ്വാസികളാവുകയും വീണ്ടും തെറ്റ് പറ്റാതിരിക്കാന്‍ ശ്രദ്ധ പുലര്‍ത്തുകയും നല്ലനിലയില്‍ വര്‍ത്തിക്കുകയും വേണം. തീര്‍ച്ചയായും നന്മ ചെയ്യുന്നവരെ അല്ലാഹു ഇഷ്ടപ്പെടുന്നു.
\end{malayalam}}
\flushright{\begin{Arabic}
\quranayah[5][94]
\end{Arabic}}
\flushleft{\begin{malayalam}
വിശ്വസിച്ചവരേ, നിങ്ങളുടെ കൈകള്‍ക്കും കുന്തങ്ങള്‍ക്കും വേഗം പിടികൂടാവുന്ന ചില വേട്ട ജന്തുക്കളെക്കൊണ്ട് അല്ലാഹു നിങ്ങളെ പരീക്ഷിക്കുക തന്നെ ചെയ്യും. കാണാതെതന്നെ അല്ലാഹുവെ ഭയപ്പെടുന്നവരാരെന്ന് തിരിച്ചറിയാനാണിത്. ആരെങ്കിലും അതിനുശേഷം അതിക്രമം കാണിച്ചാല്‍ അയാള്‍ക്ക് നോവേറിയ ശിക്ഷയുണ്ട്.
\end{malayalam}}
\flushright{\begin{Arabic}
\quranayah[5][95]
\end{Arabic}}
\flushleft{\begin{malayalam}
വിശ്വാസികളേ, നിങ്ങള്‍ ഇഹ്റാമിലായിരിക്കെ വേട്ടമൃഗത്തെ കൊല്ലരുത്. ആരെങ്കിലും ബോധപൂര്‍വം അങ്ങനെ ചെയ്താല്‍ പരിഹാരമായി, അയാള്‍ കൊന്നതിനു തുല്യമായ ഒരു കാലിയെ ബലി നല്‍കണം. നിങ്ങളിലെ നീതിമാന്മാരായ രണ്ടുപേരാണ് അത് തീരുമാനിക്കേണ്ടത്. ആ ബലിമൃഗത്തെ കഅ്ബയിലെത്തിക്കുകയും വേണം. അതല്ലെങ്കില്‍ പ്രായശ്ചിത്തം ചെയ്യണം. ഏതാനും അഗതികള്‍ക്ക് അന്നം നല്‍കലാണത്. അല്ലെങ്കില്‍ അതിനു തുല്യമായി നോമ്പനുഷ്ഠിക്കലാണ്. താന്‍ ചെയ്തതിന്റെ ഭവിഷ്യത്ത് സ്വയം തന്നെ അനുഭവിക്കാനാണിത്. നേരത്തെ കഴിഞ്ഞുപോയതെല്ലാം അല്ലാഹു മാപ്പാക്കിയിരിക്കുന്നു. എന്നാല്‍ ഇനി ആരെങ്കിലും അതാവര്‍ത്തിച്ചാല്‍ അല്ലാഹു അവന്റെ മേല്‍ ശിക്ഷാനടപടി സ്വീകരിക്കും. അല്ലാഹു പ്രതാപിയും പകരം ചെയ്യാന്‍ പോന്നവനുമാണ്.
\end{malayalam}}
\flushright{\begin{Arabic}
\quranayah[5][96]
\end{Arabic}}
\flushleft{\begin{malayalam}
കടലിലെ വേട്ടയും അതിലെ ആഹാരവും നിങ്ങള്‍ക്ക് അനുവദനീയമാണ്. അത് നിങ്ങള്‍ക്കും യാത്രാസംഘങ്ങള്‍ക്കുമുള്ള ഭക്ഷണമാണ്. എന്നാല്‍ ഇഹ്റാമിലായിരിക്കെ കരയിലെ വേട്ട നിങ്ങള്‍ക്കു നിഷിദ്ധമാക്കിയിരിക്കുന്നു. നിങ്ങള്‍ ആരിലേക്കാണോ ഒരുമിച്ചു കൂട്ടപ്പെടുക, ആ അല്ലാഹുവെ സൂക്ഷിക്കുക.
\end{malayalam}}
\flushright{\begin{Arabic}
\quranayah[5][97]
\end{Arabic}}
\flushleft{\begin{malayalam}
ആദരണീയ മന്ദിരമായ കഅ്ബയെ അല്ലാഹു മനുഷ്യരാശിയുടെ നിലനില്‍പിനുള്ള ആധാരമാക്കിയിരിക്കുന്നു. ആദരണീയ മാസം,ബലിമൃഗം, അവയുടെ കഴുത്തിലെ അടയാളപ്പട്ടകള്‍ എന്നിവയെയും. നിശ്ചയമായും ആകാശഭൂമികളിലുള്ളതെല്ലാം അല്ലാഹു അറിയുന്നുവെന്നും അവന്‍ എല്ലാ കാര്യങ്ങളെക്കുറിച്ചും സൂക്ഷ്മ ജ്ഞാനമുള്ളവനാണെന്നും നിങ്ങള്‍ അറിയാനാണിത്.
\end{malayalam}}
\flushright{\begin{Arabic}
\quranayah[5][98]
\end{Arabic}}
\flushleft{\begin{malayalam}
അറിയുക: അല്ലാഹു കഠിനമായി ശിക്ഷിക്കുന്നവനാണ്. അതോടൊപ്പം അവന്‍ ഏറെ പൊറുക്കുന്നവനും ദയാപരനുമാകുന്നു.
\end{malayalam}}
\flushright{\begin{Arabic}
\quranayah[5][99]
\end{Arabic}}
\flushleft{\begin{malayalam}
സന്ദേശം എത്തിച്ചുതരുന്ന ബാധ്യത മാത്രമേ ദൈവദൂതന്നുള്ളൂ. നിങ്ങള്‍ വെളിപ്പെടുത്തുന്നതും മറച്ചുവെക്കുന്നതുമെല്ലാം അല്ലാഹു അറിയുന്നുണ്ട്.
\end{malayalam}}
\flushright{\begin{Arabic}
\quranayah[5][100]
\end{Arabic}}
\flushleft{\begin{malayalam}
പറയുക: നല്ലതും തിയ്യതും തുല്യമല്ല. തിയ്യതിന്റെ ആധിക്യം നിന്നെ എത്രതന്നെ അത്ഭുതപ്പെടുത്തിയാലും ശരി! അതിനാല്‍ ബുദ്ധിമാന്മാരേ, നിങ്ങള്‍ അല്ലാഹുവെ സൂക്ഷിക്കുക. നിങ്ങള്‍ക്കു വിജയംവരിക്കാം.
\end{malayalam}}
\flushright{\begin{Arabic}
\quranayah[5][101]
\end{Arabic}}
\flushleft{\begin{malayalam}
വിശ്വസിച്ചവരേ, ചില കാര്യങ്ങളെക്കുറിച്ച് നിങ്ങള്‍ ചോദിക്കാതിരിക്കുക. അവ വെളിപ്പെടുത്തിത്തരുന്നത് നിങ്ങള്‍ക്ക് പ്രയാസകരമായിരിക്കും. ഖുര്‍ആന്‍ അവതരിച്ചുകൊണ്ടിരിക്കുന്ന സമയത്ത് നിങ്ങള്‍ അവയെ സംബന്ധിച്ച് ചോദിച്ചാല്‍ അവന്‍ നിങ്ങള്‍ക്കവ വെളിപ്പെടുത്തിത്തരും. കഴിഞ്ഞ കാര്യത്തില്‍ അല്ലാഹു നിങ്ങള്‍ക്ക് മാപ്പേകിയിരിക്കുന്നു. അല്ലാഹു ഏറെ പൊറുക്കുന്നവനും കനിവുറ്റവനുമാണ്.
\end{malayalam}}
\flushright{\begin{Arabic}
\quranayah[5][102]
\end{Arabic}}
\flushleft{\begin{malayalam}
നിങ്ങള്‍ക്കുമുമ്പ് ഒരു വിഭാഗം ഇത്തരം ചോദ്യങ്ങള്‍ ചോദിക്കുകയുണ്ടായി. എന്നിട്ടോ, ഉത്തരം കിട്ടിയപ്പോള്‍ അവര്‍ അവയെ നിഷേധിക്കുന്നവരായിത്തീര്‍ന്നു.
\end{malayalam}}
\flushright{\begin{Arabic}
\quranayah[5][103]
\end{Arabic}}
\flushleft{\begin{malayalam}
ബഹീറ, സാഇബ, വസ്വീല, ഹാം എന്നിങ്ങനെയൊന്നും അല്ലാഹു നിശ്ചയിച്ചിട്ടില്ല. എന്നാല്‍, സത്യനിഷേധികള്‍ അല്ലാഹുവിന്റെ പേരില്‍ കള്ളം കെട്ടിച്ചമക്കുകയായിരുന്നു. അവരിലേറെ പേരും ചിന്തിച്ചു മനസ്സിലാക്കുന്നില്ല.
\end{malayalam}}
\flushright{\begin{Arabic}
\quranayah[5][104]
\end{Arabic}}
\flushleft{\begin{malayalam}
അല്ലാഹു ഇറക്കിത്തന്നതിലേക്കും അവന്റെ ദൂതനിലേക്കും വരാന്‍ ആവശ്യപ്പെടുമ്പോള്‍ അവര്‍ പറയുന്നു: "ഞങ്ങളുടെ പൂര്‍വപിതാക്കള്‍ നടന്നതായി ഞങ്ങള്‍ കാണുന്ന പാതതന്നെ ഞങ്ങള്‍ക്കു മതി.” അവരുടെ പിതാക്കന്മാര്‍ ഒന്നുമറിയാത്തവരും നേര്‍വഴി പ്രാപിക്കാത്തവരുമാണെങ്കിലോ?
\end{malayalam}}
\flushright{\begin{Arabic}
\quranayah[5][105]
\end{Arabic}}
\flushleft{\begin{malayalam}
വിശ്വസിച്ചവരേ, നിങ്ങള്‍ നിങ്ങളുടെ കാര്യം ശ്രദ്ധിക്കുക. നിങ്ങള്‍ നേര്‍വഴി പ്രാപിച്ചവരാണെങ്കില്‍ വഴിപിഴച്ചവര്‍ നിങ്ങള്‍ക്കൊരു ദ്രോഹവും വരുത്തുകയില്ല. അല്ലാഹുവിങ്കലേക്കാണ് നിങ്ങളുടെയൊക്കെ മടക്കം. അപ്പോള്‍ നിങ്ങള്‍ ചെയ്തുകൊണ്ടിരുന്നതിനെപ്പറ്റിയെല്ലാം അവന്‍ നിങ്ങളെ വിവരമറിയിക്കും.
\end{malayalam}}
\flushright{\begin{Arabic}
\quranayah[5][106]
\end{Arabic}}
\flushleft{\begin{malayalam}
വിശ്വസിച്ചവരേ, നിങ്ങളിലാര്‍ക്കെങ്കിലും മരണമടുക്കുകയും വസിയ്യത്ത് ചെയ്യുകയുമാണെങ്കില്‍ നിങ്ങളില്‍നിന്നുള്ള നീതിമാന്മാരായ രണ്ടാളുകള്‍ അതിനു സാക്ഷ്യം വഹിക്കണം. നിങ്ങള്‍ യാത്രയിലായിരിക്കെയാണ് മരണവിപത്ത് നിങ്ങളെ ബാധിക്കുന്നതെങ്കില്‍ അപ്പോള്‍ അന്യരായ രണ്ടാളുകളെ സാക്ഷികളാക്കാവുന്നതാണ്. പിന്നീട് നിങ്ങള്‍ക്ക് അവരില്‍ സംശയമുണ്ടാവുകയാണെങ്കില്‍ അവരിരുവരെയും നമസ്കാരശേഷം തടഞ്ഞുവെക്കണം. അപ്പോള്‍ അവര്‍ അല്ലാഹുവിന്റെ പേരില്‍ ഇങ്ങനെ സത്യം ചെയ്യട്ടെ: "ഞങ്ങളുടെ അടുത്ത ബന്ധുക്കള്‍ക്കുതന്നെ എതിരായാല്‍ പോലും ഞങ്ങള്‍ സത്യത്തെ വിറ്റു വില വാങ്ങുകയില്ല. അല്ലാഹുവിനുവേണ്ടിയുള്ള സാക്ഷ്യത്തെ ഒളിപ്പിച്ചുവെക്കുകയുമില്ല. അങ്ങനെ ചെയ്താല്‍ തീര്‍ച്ചയായും ഞങ്ങള്‍ പാപികളായിത്തീരും.”
\end{malayalam}}
\flushright{\begin{Arabic}
\quranayah[5][107]
\end{Arabic}}
\flushleft{\begin{malayalam}
അഥവാ, അവരിരുവരും തങ്ങളെ സ്വയം തെറ്റിലകപ്പെടുത്തിയിരിക്കുന്നുവെന്ന് വ്യക്തമായാല്‍ കുറ്റം ചെയ്തത് ആര്‍ക്കെതിരിലാണോ അയാളോട് ഏറ്റം അടുത്ത ബന്ധമുള്ള രണ്ടുപേര്‍ അവരുടെ സ്ഥാനത്ത് സാക്ഷികളായി നില്‍ക്കണം. എന്നിട്ട് അവരിരുവരും അല്ലാഹുവിന്റെ പേരില്‍ ഇങ്ങനെ സത്യം ചെയ്തുപറയണം: "ഉറപ്പായും ഞങ്ങളുടെ സാക്ഷ്യമാണ് ഇവരുടെ സാക്ഷ്യത്തെക്കാള്‍ സത്യസന്ധമായിട്ടുള്ളത്. ഞങ്ങള്‍ ഒരനീതിയും ചെയ്തിട്ടില്ല. അങ്ങനെ ചെയ്താല്‍ തീര്‍ച്ചയായും ഞങ്ങള്‍ അതിക്രമികളായിത്തീരും.”
\end{malayalam}}
\flushright{\begin{Arabic}
\quranayah[5][108]
\end{Arabic}}
\flushleft{\begin{malayalam}
ജനം യഥാവിധി സാക്ഷ്യം നിര്‍വഹിക്കാന്‍ ഏറ്റം പറ്റിയ മാര്‍ഗം ഇതാണ്. അല്ലെങ്കില്‍ തങ്ങളുടെ സത്യത്തിനുശേഷം മറ്റുള്ളവരുടെ സത്യത്താല്‍ തങ്ങള്‍ ഖണ്ഡിക്കപ്പെടുമെന്ന് അവര്‍ ഭയപ്പെടുകയെങ്കിലും ചെയ്യും. നിങ്ങള്‍ അല്ലാഹുവെ സൂക്ഷിക്കുക. അവന്റെ കല്‍പനകള്‍ കേട്ടനുസരിക്കുക. അധാര്‍മികരെ അല്ലാഹു നേര്‍വഴിയിലാക്കുകയില്ല.
\end{malayalam}}
\flushright{\begin{Arabic}
\quranayah[5][109]
\end{Arabic}}
\flushleft{\begin{malayalam}
അല്ലാഹു തന്റെ ദൂതന്മാരെ ഒരുമിച്ചുകൂട്ടും. നിങ്ങള്‍ക്ക് എന്ത് ഉത്തരമാണ് കിട്ടിയതെന്ന് അവരോട് ചോദിക്കും. ആ ദിനം അവര്‍ പറയും: "ഞങ്ങള്‍ക്കൊന്നുമറിഞ്ഞുകൂടാ. അദൃശ്യ കാര്യങ്ങളൊക്കെയും നന്നായറിയുന്നവന്‍ നീ മാത്രം.”
\end{malayalam}}
\flushright{\begin{Arabic}
\quranayah[5][110]
\end{Arabic}}
\flushleft{\begin{malayalam}
അല്ലാഹു പറഞ്ഞ സന്ദര്‍ഭം: മര്‍യമിന്റെ മകന്‍ ഈസാ, നിനക്കും നിന്റെ മാതാവിനും നാം നല്‍കിയ അനുഗ്രഹം ഓര്‍ക്കുക: ഞാന്‍ പരിശുദ്ധാത്മാവിനാല്‍ നിന്നെ കരുത്തനാക്കി. തൊട്ടിലില്‍ വെച്ചും പ്രായമായ ശേഷവും നീ ജനങ്ങളോടു സംസാരിച്ചു. നാം നിനക്ക് വേദവും തത്ത്വജ്ഞാനവും തൌറാത്തും ഇഞ്ചീലും പഠിപ്പിച്ചുതന്നു. നീ എന്റെ അനുമതിയോടെ കളിമണ്ണുകൊണ്ട് പക്ഷിയുടെ രൂപമുണ്ടാക്കി. പിന്നെ അതിലൂതി. എന്റെ ഹിതത്താല്‍ അത് പക്ഷിയായി. ജന്മനാ കുരുടനായവനെയും വെള്ളപ്പാണ്ടുകാരനെയും എന്റെ ഹിതത്താല്‍ നീ സുഖപ്പെടുത്തി; എന്റെ അനുമതിയോടെ നീ മരണപ്പെട്ടവരെ ജീവിതത്തിലേക്ക് തിരിച്ചുകൊണ്ടുവന്നു. പിന്നീട് നീ വ്യക്തമായ തെളിവുകളുമായി ഇസ്രയേല്‍ മക്കളുടെ അടുത്ത് ചെന്നു. അപ്പോള്‍ അവരിലെ സത്യനിഷേധികള്‍, “ഈ തെളിവുകളെല്ലാം തെളിഞ്ഞ ആഭിചാരം മാത്രമാണെ”ന്ന് തള്ളിപ്പറയുകയും ചെയ്തു. പിന്നെ അവരില്‍ നിന്ന് ഞാന്‍ നിന്നെ രക്ഷിച്ചു.
\end{malayalam}}
\flushright{\begin{Arabic}
\quranayah[5][111]
\end{Arabic}}
\flushleft{\begin{malayalam}
എന്നിലും എന്റെ ദൂതനിലും വിശ്വസിക്കണമെന്ന് ഞാന്‍ ഹവാരികള്‍ക്ക് നിര്‍ദേശം നല്‍കി. അവര്‍ പറഞ്ഞു: "ഞങ്ങള്‍ വിശ്വസിച്ചിരിക്കുന്നു. ഞങ്ങള്‍ മുസ്ലിംകളാണെന്ന് നീ സാക്ഷ്യം വഹിക്കുക.”
\end{malayalam}}
\flushright{\begin{Arabic}
\quranayah[5][112]
\end{Arabic}}
\flushleft{\begin{malayalam}
ഓര്‍ക്കുക: ഹവാരികള്‍ പറഞ്ഞ സന്ദര്‍ഭം: "മര്‍യമിന്റെ മകന്‍ ഈസാ, മാനത്തുനിന്ന് ഒരു ഭക്ഷണത്തളിക ഞങ്ങള്‍ക്ക് ഇറക്കിത്തരാന്‍ നിന്റെ നാഥന് കഴിയുമോ?” അദ്ദേഹം പറഞ്ഞു: "നിങ്ങള്‍ വിശ്വാസികളാണെങ്കില്‍ അല്ലാഹുവെ സൂക്ഷിക്കുക.”
\end{malayalam}}
\flushright{\begin{Arabic}
\quranayah[5][113]
\end{Arabic}}
\flushleft{\begin{malayalam}
അവര്‍ പറഞ്ഞു: "ഞങ്ങള്‍ക്ക് അതില്‍നിന്ന് ആഹരിക്കണം. അങ്ങനെ ഞങ്ങള്‍ക്ക് മനസ്സമാധാനമുണ്ടാകണം. താങ്കള്‍ ഞങ്ങളോടു പറഞ്ഞത് സത്യമാണെന്ന് ഞങ്ങള്‍ക്ക് ബോധ്യമാകണം. ഞങ്ങള്‍ ഇതിനെല്ലാം നേരില്‍ സാക്ഷികളാവുകയും വേണം. ഇതിനൊക്കെയാണ് ഞങ്ങളിതാവശ്യപ്പെടുന്നത്.”
\end{malayalam}}
\flushright{\begin{Arabic}
\quranayah[5][114]
\end{Arabic}}
\flushleft{\begin{malayalam}
മര്‍യമിന്റെ മകന്‍ ഈസാ പ്രാര്‍ഥിച്ചു: "ഞങ്ങളുടെ നാഥനായ അല്ലാഹുവേ, മാനത്തുനിന്ന് ഞങ്ങള്‍ക്ക് ഒരു ഭക്ഷണത്തളിക ഇറക്കിത്തരേണമേ! അതു ഞങ്ങളുടെ, ആദ്യക്കാര്‍ക്കും അവസാനക്കാര്‍ക്കും ഒരാഘോഷവും നിന്നില്‍ നിന്നുള്ള ഒരു ദൃഷ്ടാന്തവുമായിരിക്കട്ടെ. ഞങ്ങള്‍ക്കു നീ അന്നം നല്‍കുക. അന്നം നല്‍കുന്നവരില്‍ അത്യുത്തമന്‍ നീയല്ലോ.”
\end{malayalam}}
\flushright{\begin{Arabic}
\quranayah[5][115]
\end{Arabic}}
\flushleft{\begin{malayalam}
അല്ലാഹു അറിയിച്ചു: "ഞാന്‍ നിങ്ങള്‍ക്ക് അതിറക്കിത്തരാം. എന്നാല്‍ അതിനുശേഷം നിങ്ങളിലാരെങ്കിലും സത്യനിഷേധികളായാല്‍ ലോകരിലൊരാള്‍ക്കും നല്‍കാത്ത വിധമുള്ള ശിക്ഷ നാമവന് ബാധകമാക്കും.”
\end{malayalam}}
\flushright{\begin{Arabic}
\quranayah[5][116]
\end{Arabic}}
\flushleft{\begin{malayalam}
ഓര്‍ക്കുക: അല്ലാഹു ചോദിക്കുന്ന സന്ദര്‍ഭം: "മര്‍യമിന്റെ മകന്‍ ഈസാ! “അല്ലാഹുവെക്കൂടാതെ എന്നെയും എന്റെ മാതാവിനെയും ആരാധ്യരാക്കുവിന്‍” എന്ന് നീയാണോ ജനങ്ങളോട് പറഞ്ഞത്?” അപ്പോള്‍ അദ്ദേഹം പറയും: "നീ എത്ര പരിശുദ്ധന്‍! എനിക്കു പറയാന്‍ പാടില്ലാത്ത ഒരു കാര്യം ഞാന്‍ പറയാവതല്ലല്ലോ. ഞാന്‍ അങ്ങനെ പറഞ്ഞിരുന്നെങ്കില്‍ ഉറപ്പായും നീ അതറിഞ്ഞിരിക്കും. എന്റെ മനസ്സിലുള്ളത് നീ അറിയും. എന്നാല്‍ നിന്റെ ഉള്ളിലുള്ളത് ഞാനറിയുകയില്ല. തീര്‍ച്ചയായും നീ തന്നെയാണ് കണ്ണുകൊണ്ട് കാണാന്‍ കഴിയാത്തതുപോലും നന്നായറിയുന്നവന്‍.
\end{malayalam}}
\flushright{\begin{Arabic}
\quranayah[5][117]
\end{Arabic}}
\flushleft{\begin{malayalam}
"നീ എന്നോട് കല്‍പിച്ചതല്ലാത്തതൊന്നും ഞാനവരോടു പറഞ്ഞിട്ടില്ല. അഥവാ, “എന്റെ നാഥനും നിങ്ങളുടെ നാഥനുമായ അല്ലാഹുവെ മാത്രം വഴിപ്പെട്ട് ജീവിക്കണ”മെന്നാണ് ഞാന്‍ പറഞ്ഞത്. ഞാന്‍ അവരിലുണ്ടായിരുന്ന കാലത്തോളം അവരുടെ എല്ലാ കാര്യങ്ങള്‍ക്കും സാക്ഷിയായിരുന്നു ഞാന്‍. പിന്നെ നീ എന്നെ തിരിച്ചുവിളിച്ചപ്പോള്‍ അവരുടെ നിരീക്ഷകന്‍ നീ തന്നെ ആയിരുന്നുവല്ലോ. നീ സകല സംഗതികള്‍ക്കും സാക്ഷിയാകുന്നു.
\end{malayalam}}
\flushright{\begin{Arabic}
\quranayah[5][118]
\end{Arabic}}
\flushleft{\begin{malayalam}
"നീ അവരെ ശിക്ഷിക്കുന്നുവെങ്കില്‍ തീര്‍ച്ചയായും അവര്‍ നിന്റെ അടിമകള്‍ തന്നെയല്ലോ. നീ അവര്‍ക്ക് മാപ്പേകുന്നുവെങ്കിലോ, നീ തന്നെയാണല്ലോ പ്രതാപവാനും യുക്തിമാനും.”
\end{malayalam}}
\flushright{\begin{Arabic}
\quranayah[5][119]
\end{Arabic}}
\flushleft{\begin{malayalam}
അല്ലാഹു അറിയിക്കും: സത്യസന്ധന്മാര്‍ക്ക് തങ്ങളുടെ സത്യം ഉപകരിക്കും ദിനമാണിത്. അവര്‍ക്ക് താഴ്ഭാഗത്തൂടെ അരുവികളൊഴുകുന്ന സ്വര്‍ഗീയാരാമങ്ങളുണ്ട്. അവരവിടെ സ്ഥിരവാസികളായിരിക്കും. അവരെക്കുറിച്ച് അല്ലാഹു സംതൃപ്തനായിരിക്കുന്നു. അവര്‍ അവനെപ്പറ്റിയും സംതൃപ്തരാണ്. അതത്രെ അതിമഹത്തായ വിജയം!
\end{malayalam}}
\flushright{\begin{Arabic}
\quranayah[5][120]
\end{Arabic}}
\flushleft{\begin{malayalam}
ആകാശഭൂമികളുടെയും അവയിലുള്ളവയുടെയും ആധിപത്യം അല്ലാഹുവിനു മാത്രമാണ്. അവന്‍ എല്ലാ കാര്യങ്ങള്‍ക്കും കഴിവുറ്റവനാണ്.
\end{malayalam}}
\chapter{\textmalayalam{അന്‍ആം ( കാലികള്‍ )}}
\begin{Arabic}
\Huge{\centerline{\basmalah}}\end{Arabic}
\flushright{\begin{Arabic}
\quranayah[6][1]
\end{Arabic}}
\flushleft{\begin{malayalam}
സമസ്ത സ്തുതിയും അല്ലാഹുവിന്. അവനാണ് ആകാശഭൂമികളെ സൃഷ്ടിച്ചവന്‍. ഇരുട്ടുകളും വെളിച്ചവും ഉണ്ടാക്കിയവനും. എന്നിട്ടും സത്യനിഷേധികളിതാ തങ്ങളുടെ നാഥന്ന് തുല്യരെ കല്‍പിക്കുന്നു.
\end{malayalam}}
\flushright{\begin{Arabic}
\quranayah[6][2]
\end{Arabic}}
\flushleft{\begin{malayalam}
അവനാണ് കളിമണ്ണില്‍നിന്ന് നിങ്ങളെ സൃഷ്ടിച്ചത്. എന്നിട്ട് അവന്‍ ഒരവധി നിശ്ചയിച്ചു. അവന്റെ അടുക്കല്‍ നിര്‍ണിതമായ മറ്റൊരവധി കൂടിയുണ്ട്. എന്നിട്ടും നിങ്ങള്‍ സംശയിച്ചുകൊണ്ടിരിക്കുകയാണ്.
\end{malayalam}}
\flushright{\begin{Arabic}
\quranayah[6][3]
\end{Arabic}}
\flushleft{\begin{malayalam}
അവന്‍ തന്നെയാണ് ആകാശ ഭൂമികളിലെ സാക്ഷാല്‍ ദൈവം. നിങ്ങളുടെ രഹസ്യവും പരസ്യവുമെല്ലാം അവനറിയുന്നു. നിങ്ങള്‍ പ്രവര്‍ത്തിച്ചുകൊണ്ടിരിക്കുന്നതെന്തെന്നും അവന് നന്നായറിയാം.
\end{malayalam}}
\flushright{\begin{Arabic}
\quranayah[6][4]
\end{Arabic}}
\flushleft{\begin{malayalam}
തങ്ങളുടെ നാഥനില്‍നിന്ന് എന്തു തെളിവ് വന്നെത്തിയാലും അതിനെ അവഗണിക്കുകയാണവര്‍.
\end{malayalam}}
\flushright{\begin{Arabic}
\quranayah[6][5]
\end{Arabic}}
\flushleft{\begin{malayalam}
അങ്ങനെ അവര്‍ക്കിപ്പോള്‍ വന്നെത്തിയ ഈ സത്യത്തെയും അവര്‍ തള്ളിപ്പറഞ്ഞിരിക്കുന്നു. എന്നാല്‍ ഏതൊന്നിനെയാണോ അവര്‍ പരിഹസിച്ചുകൊണ്ടിരുന്നത് അതിന്റെ യഥാര്‍ഥ വിവരം വഴിയെ അവര്‍ക്ക് വന്നെത്തും; തീര്‍ച്ച.
\end{malayalam}}
\flushright{\begin{Arabic}
\quranayah[6][6]
\end{Arabic}}
\flushleft{\begin{malayalam}
അവരറിഞ്ഞിട്ടില്ലേ? അവര്‍ക്കുമുമ്പ് എത്ര തലമുറകളെയാണ് നാം നശിപ്പിച്ചത്. നിങ്ങള്‍ക്കു നാം ചെയ്തുതന്നിട്ടില്ലാത്ത സൌകര്യം ഭൂമിയില്‍ നാമവര്‍ക്ക് ചെയ്തുകൊടുത്തിരുന്നു. അവര്‍ക്കു നാം മാനത്തുനിന്ന് ധാരാളമായി മഴ വര്‍ഷിച്ചു. അവരുടെ താഴ്ഭാഗത്തൂടെ പുഴകളൊഴുക്കുകയും ചെയ്തു. പിന്നെ അവരുടെ പാപങ്ങളുടെ ഫലമായി നാമവരെ നശിപ്പിച്ചു. അവര്‍ക്കുപിറകെ മറ്റു തലമുറകളെ വളര്‍ത്തിക്കൊണ്ടുവരികയും ചെയ്തു.
\end{malayalam}}
\flushright{\begin{Arabic}
\quranayah[6][7]
\end{Arabic}}
\flushleft{\begin{malayalam}
നിനക്കു നാം കടലാസിലെഴുതിയ ഗ്രന്ഥം ഇറക്കിത്തന്നുവെന്ന് വെക്കുക. അങ്ങനെ അവരത് തങ്ങളുടെ കൈകള്‍ കൊണ്ട് തൊട്ടുനോക്കി. എന്നാലും സത്യനിഷേധികള്‍ പറയും: "ഇത് വ്യക്തമായ മായാജാലമല്ലാതൊന്നുമല്ല.”
\end{malayalam}}
\flushright{\begin{Arabic}
\quranayah[6][8]
\end{Arabic}}
\flushleft{\begin{malayalam}
അവര്‍ ചോദിക്കുന്നു: "ഈ പ്രവാചകന് ഒരു മലക്കിനെ ഇറക്കിക്കൊടുക്കാത്തതെന്ത്?” നാം ഒരു മലക്കിനെ ഇറക്കിക്കൊടുത്തിരുന്നുവെങ്കില്‍ കാര്യം ഇതിനുമുമ്പേ തീരുമാനിക്കപ്പെടുമായിരുന്നു. പിന്നീട് അവര്‍ക്കൊട്ടും അവസരം കിട്ടുമായിരുന്നില്ല.
\end{malayalam}}
\flushright{\begin{Arabic}
\quranayah[6][9]
\end{Arabic}}
\flushleft{\begin{malayalam}
നാം മലക്കിനെ നിയോഗിക്കുകയാണെങ്കില്‍ തന്നെ മനുഷ്യരൂപത്തിലാണയക്കുക. അങ്ങനെ അവരിപ്പോഴുള്ള ആശയക്കുഴപ്പം അപ്പോഴും നാമവരിലുണ്ടാക്കുമായിരുന്നു.
\end{malayalam}}
\flushright{\begin{Arabic}
\quranayah[6][10]
\end{Arabic}}
\flushleft{\begin{malayalam}
നിനക്കുമുമ്പും നിരവധി ദൈവദൂതന്മാര്‍ പരിഹസിക്കപ്പെട്ടിട്ടുണ്ട്. എന്നിട്ടോ, അവരെ പരിഹസിച്ചിരുന്നവര്‍ക്ക് അവര്‍ പരിഹസിച്ചിരുന്നതുതന്നെ വന്നുഭവിച്ചു.
\end{malayalam}}
\flushright{\begin{Arabic}
\quranayah[6][11]
\end{Arabic}}
\flushleft{\begin{malayalam}
പറയുക: നിങ്ങള്‍ ഭൂമിയിലൂടെ സഞ്ചരിക്കുക; എന്നിട്ട് സത്യനിഷേധികളുടെ അന്ത്യം എവ്വിധമായിരുന്നുവെന്ന് നോക്കിക്കാണുക.
\end{malayalam}}
\flushright{\begin{Arabic}
\quranayah[6][12]
\end{Arabic}}
\flushleft{\begin{malayalam}
ചോദിക്കുക: ആകാശഭൂമികളിലുള്ളതെല്ലാം ആരുടേതാണ്? പറയുക: എല്ലാം അല്ലാഹുവിന്റേതുമാത്രം. കാരുണ്യത്തെ അവന്‍ സ്വന്തം ബാധ്യതയായി നിശ്ചയിച്ചിരിക്കുന്നു. ഉയിര്‍ത്തെഴുന്നേല്‍പുനാളില്‍ അവന്‍ നിങ്ങളെയൊക്കെ ഒരുമിച്ചുകൂട്ടുക തന്നെ ചെയ്യും. അതിലൊട്ടും സംശയമില്ല. എന്നാല്‍ സ്വന്തത്തെ നഷ്ടത്തിലകപ്പെടുത്തിയവരത് വിശ്വസിക്കുകയില്ല.
\end{malayalam}}
\flushright{\begin{Arabic}
\quranayah[6][13]
\end{Arabic}}
\flushleft{\begin{malayalam}
രാവിലും പകലിലും നിലനില്‍ക്കുന്നവയെല്ലാം അവന്റേതാണ്.അവന്‍ എല്ലാം കേള്‍ക്കുന്നവനും അറിയുന്നവനുമല്ലോ.
\end{malayalam}}
\flushright{\begin{Arabic}
\quranayah[6][14]
\end{Arabic}}
\flushleft{\begin{malayalam}
ചോദിക്കുക: അല്ലാഹുവെയല്ലാതെ മറ്റാരെയെങ്കിലും ഞാന്‍ രക്ഷകനായി സ്വീകരിക്കുകയോ? അവനാണ് ആകാശഭൂമികളുടെ സ്രഷ്ടാവ്. അവന്‍ അന്നം നല്‍കുന്നു. എന്നാല്‍ ആരും അവന്ന് അന്നം നല്‍കുന്നുമില്ല. പറയുക: "അല്ലാഹുവിനെ അനുസരിച്ച് ജീവിക്കുന്നവരില്‍ ഒന്നാമനാകാനാണ് എന്നോട് കല്‍പിച്ചിരിക്കുന്നത്. ഒരിക്കലും ബഹുദൈവ വിശ്വാസികളില്‍ പെട്ടുപോകാതിരിക്കാനും.”
\end{malayalam}}
\flushright{\begin{Arabic}
\quranayah[6][15]
\end{Arabic}}
\flushleft{\begin{malayalam}
പറയുക: ഞാനെന്റെ നാഥനെ ധിക്കരിച്ചാല്‍ ഭയങ്കരമായൊരു നാളിന്റെ ശിക്ഷയനുഭവിക്കേണ്ടിവരുമെന്ന് ഞാന്‍ ഭയപ്പെടുന്നു.
\end{malayalam}}
\flushright{\begin{Arabic}
\quranayah[6][16]
\end{Arabic}}
\flushleft{\begin{malayalam}
അന്ന് ആ ശിക്ഷയില്‍ നിന്ന് ഒഴിവാകുന്നവനെ ഉറപ്പായും അല്ലാഹു അനുഗ്രഹിച്ചിരിക്കുന്നു. അതുതന്നെയാണ് വ്യക്തമായ വിജയം.
\end{malayalam}}
\flushright{\begin{Arabic}
\quranayah[6][17]
\end{Arabic}}
\flushleft{\begin{malayalam}
അല്ലാഹു നിനക്കു വല്ല വിപത്തും വരുത്തുകയാണെങ്കില്‍ അതൊഴിവാക്കാന്‍ അവന്നല്ലാതെ ആര്‍ക്കും സാധ്യമല്ല. അവന്‍ നിനക്കു വല്ല നന്മയുമാണ് വരുത്തുന്നതെങ്കിലോ? അറിയുക: അവന്‍ എല്ലാ കാര്യങ്ങള്‍ക്കും കഴിവുറ്റവനാണ്.
\end{malayalam}}
\flushright{\begin{Arabic}
\quranayah[6][18]
\end{Arabic}}
\flushleft{\begin{malayalam}
അല്ലാഹു തന്റെ അടിമകളുടെമേല്‍ പരമാധികാരമുള്ളവനാണ്. അവന്‍ യുക്തിമാനാണ്. സൂക്ഷ്മജ്ഞനും.
\end{malayalam}}
\flushright{\begin{Arabic}
\quranayah[6][19]
\end{Arabic}}
\flushleft{\begin{malayalam}
ചോദിക്കുക: ഏതു സാക്ഷ്യമാണ് ഏറെ മഹത്തരം? പറയുക: അല്ലാഹുവാണ് എനിക്കും നിങ്ങള്‍ക്കുമിടയില്‍ സാക്ഷി. ഈ ഖുര്‍ആന്‍ എനിക്കു ബോധനമായി ലഭിച്ചത് നിങ്ങള്‍ക്കും ഇത് ചെന്നെത്തുന്ന മറ്റെല്ലാവര്‍ക്കും ഇതുവഴി മുന്നറിയിപ്പു നല്‍കാനാണ്. അല്ലാഹുവോടൊപ്പം വേറെ ദൈവങ്ങളുണ്ടെന്ന് നിങ്ങള്‍ക്ക് സാക്ഷ്യം വഹിക്കാനാകുമോ? പറയുക: ഞാനതിന് സാക്ഷ്യം വഹിക്കുകയില്ല. പറയുക: അവന്‍ ഒരേയൊരു ദൈവം മാത്രം. നിങ്ങള്‍ അവന്ന് പങ്കാളികളെ സങ്കല്‍പിക്കുന്നതുമായി എനിക്കൊരു ബന്ധവുമില്ല.
\end{malayalam}}
\flushright{\begin{Arabic}
\quranayah[6][20]
\end{Arabic}}
\flushleft{\begin{malayalam}
നാം വേദം നല്‍കിയവരോ, സ്വന്തം മക്കളെ അറിയുംപോലെ അവര്‍ക്ക് ഇതറിയാം. എന്നാല്‍ സ്വയം നഷ്ടം വരുത്തിവെച്ചവര്‍ വിശ്വസിക്കുകയില്ല.
\end{malayalam}}
\flushright{\begin{Arabic}
\quranayah[6][21]
\end{Arabic}}
\flushleft{\begin{malayalam}
അല്ലാഹുവിന്റെ പേരില്‍ കള്ളം കെട്ടിച്ചമക്കുകയോ അവന്റെ വചനങ്ങളെ തള്ളിപ്പറയുകയോ ചെയ്യുന്നവനെക്കാള്‍ അക്രമിയായി ആരുണ്ട്? അക്രമികള്‍ വിജയിക്കില്ല; ഉറപ്പ്.
\end{malayalam}}
\flushright{\begin{Arabic}
\quranayah[6][22]
\end{Arabic}}
\flushleft{\begin{malayalam}
നാം അവരെയൊക്കെയും ഒരുമിച്ചുകൂട്ടും ദിനം; ബഹുദൈവ വിശ്വാസികളോടു നാം ചോദിക്കും: നിങ്ങളുടെ ദൈവങ്ങളെന്ന് നിങ്ങള്‍ വാദിച്ചിരുന്ന ആ പങ്കാളികള്‍ ഇപ്പോള്‍ എവിടെ?
\end{malayalam}}
\flushright{\begin{Arabic}
\quranayah[6][23]
\end{Arabic}}
\flushleft{\begin{malayalam}
അപ്പോള്‍ അവര്‍ക്കൊരു കുഴപ്പവും ഉണ്ടാക്കാനാവില്ല; “ഞങ്ങളുടെ നാഥനായ അല്ലാഹുവാണ് സത്യം! ഞങ്ങള്‍ ബഹുദൈവ വിശ്വാസികളായിരുന്നില്ല” എന്നു പറയാനല്ലാതെ.
\end{malayalam}}
\flushright{\begin{Arabic}
\quranayah[6][24]
\end{Arabic}}
\flushleft{\begin{malayalam}
നോക്കൂ, അവര്‍ തങ്ങളെക്കുറിച്ചുതന്നെ കള്ളം പറയുന്നതെങ്ങനെയെന്ന്! അവര്‍ കെട്ടിച്ചമച്ചതൊക്കെയും അവരെ വിട്ട് അപ്രത്യക്ഷമായിരിക്കുന്നു.
\end{malayalam}}
\flushright{\begin{Arabic}
\quranayah[6][25]
\end{Arabic}}
\flushleft{\begin{malayalam}
നീ പറയുന്നത് ശ്രദ്ധിച്ചുകേള്‍ക്കുന്നവരും അവരിലുണ്ട്. എങ്കിലും നാം അവരുടെ മനസ്സുകള്‍ക്ക് മറയിട്ടിരിക്കുന്നു. അതിനാല്‍ അവരത് മനസ്സിലാക്കുന്നില്ല. അവരുടെ കാതുകള്‍ക്ക് നാം അടപ്പിട്ടിരിക്കുന്നു. എന്തൊക്കെ തെളിവുകള്‍ കണ്ടാലും അവര്‍ വിശ്വസിക്കുകയില്ല. എത്രത്തോളമെന്നാല്‍ അവര്‍ നിന്റെയടുത്ത് നിന്നോട് തര്‍ക്കിക്കാന്‍ വന്നാല്‍ അവരിലെ സത്യനിഷേധികള്‍ പറയും: "ഇത് പൂര്‍വികരുടെ കെട്ടുകഥകളല്ലാതൊന്നുമല്ല.”
\end{malayalam}}
\flushright{\begin{Arabic}
\quranayah[6][26]
\end{Arabic}}
\flushleft{\begin{malayalam}
അവര്‍ വേദവാക്യങ്ങളില്‍ നിന്ന് മറ്റുള്ളവരെ തടയുന്നു. സ്വയം അവയില്‍നിന്ന് അകന്നുനില്‍ക്കുകയും ചെയ്യുന്നു. യഥാര്‍ഥത്തിലവര്‍ തങ്ങള്‍ക്കുതന്നെയാണ് വിപത്തു വരുത്തുന്നത്. അവര്‍ അതേക്കുറിച്ച് ബോധവാന്മാരല്ലെന്നു മാത്രം.
\end{malayalam}}
\flushright{\begin{Arabic}
\quranayah[6][27]
\end{Arabic}}
\flushleft{\begin{malayalam}
അവരെ നരകത്തിനു മുന്നില്‍ കൊണ്ടുവന്ന് നിര്‍ത്തുന്നത് നീ കണ്ടിരുന്നെങ്കില്‍! അപ്പോഴവര്‍ കേണുകൊണ്ടിരിക്കും: "ഞങ്ങള്‍ ഭൂമിയിലേക്ക് തിരിച്ചയക്കപ്പെടുകയും അങ്ങനെ ഞങ്ങള്‍ ഞങ്ങളുടെ നാഥന്റെ തെളിവുകളെ തള്ളിക്കളയാതെ സത്യവിശ്വാസികളായിത്തീരുകയും ചെയ്തിരുന്നെങ്കില്‍!”
\end{malayalam}}
\flushright{\begin{Arabic}
\quranayah[6][28]
\end{Arabic}}
\flushleft{\begin{malayalam}
എന്നാല്‍, അവര്‍ നേരത്തെ മറച്ചുവെച്ചിരുന്നത് അവര്‍ക്കിപ്പോള്‍ വെളിപ്പെട്ടിരിക്കുകയാണ്. അവരെ ഭൂമിയിലേക്കു മടക്കിയയച്ചാലും വിലക്കപ്പെട്ട കാര്യങ്ങളിലേക്കുതന്നെ അവര്‍ തിരിച്ചുചെല്ലും. അവര്‍ കള്ളം പറയുന്നവരാണ്; തീര്‍ച്ച.
\end{malayalam}}
\flushright{\begin{Arabic}
\quranayah[6][29]
\end{Arabic}}
\flushleft{\begin{malayalam}
അവര്‍ പറഞ്ഞുകൊണ്ടിരിക്കുന്നു: "നമ്മുടെ ഈ ഐഹിക ജീവിതമല്ലാതെ വേറൊരു ജീവിതമില്ല. നാമൊരിക്കലും ഉയിര്‍ത്തെഴുന്നേല്‍ക്കുകയില്ല.”
\end{malayalam}}
\flushright{\begin{Arabic}
\quranayah[6][30]
\end{Arabic}}
\flushleft{\begin{malayalam}
അവര്‍ തങ്ങളുടെ നാഥന്റെ മുമ്പില്‍ നിര്‍ത്തപ്പെടുന്ന രംഗം നീ കണ്ടിരുന്നെങ്കില്‍! അപ്പോള്‍ അവന്‍ അവരോടു ചോദിക്കും: "ഇത് യഥാര്‍ഥം തന്നെയല്ലേ?” അവര്‍ പറയും: "അതെ; ഞങ്ങളുടെ നാഥന്‍ തന്നെ സത്യം!” അവന്‍ പറയും: "എങ്കില്‍ നിങ്ങള്‍ സത്യത്തെ തള്ളിപ്പറഞ്ഞതിനുള്ള ശിക്ഷ അനുഭവിച്ചുകൊള്ളുക.”
\end{malayalam}}
\flushright{\begin{Arabic}
\quranayah[6][31]
\end{Arabic}}
\flushleft{\begin{malayalam}
അല്ലാഹുവുമായി കണ്ടുമുട്ടുമെന്ന കാര്യം കള്ളമാക്കിത്തള്ളിയവര്‍ തീര്‍ച്ചയായും തുലഞ്ഞതുതന്നെ. അങ്ങനെ പെട്ടെന്ന് അവര്‍ക്ക് ആ സമയം വന്നെത്തുമ്പോള്‍ അവര്‍ വിലപിക്കും: "കഷ്ടം! ഐഹികജീവിതത്തില്‍ എന്തൊരു വീഴ്ചയാണ് നാം കാണിച്ചത്.” അപ്പോഴവര്‍ തങ്ങളുടെ പാപഭാരം സ്വന്തം മുതുകുകളില്‍ വഹിക്കുന്നവരായിരിക്കും. അവര്‍ പേറുന്ന ഭാരം എത്ര ചീത്ത.
\end{malayalam}}
\flushright{\begin{Arabic}
\quranayah[6][32]
\end{Arabic}}
\flushleft{\begin{malayalam}
ഐഹികജീവിതമെന്നത് കളിതമാശയല്ലാതൊന്നുമല്ല. ഭക്തിപുലര്‍ത്തുന്നവര്‍ക്ക് ഉത്തമം പരലോകമാണ്. നിങ്ങള്‍ ആലോചിച്ചറിയുന്നില്ലേ?
\end{malayalam}}
\flushright{\begin{Arabic}
\quranayah[6][33]
\end{Arabic}}
\flushleft{\begin{malayalam}
തീര്‍ച്ചയായും അവര്‍ പറഞ്ഞുകൊണ്ടിരിക്കുന്നത് നിന്നെ ദുഃഖിപ്പിക്കുന്നുണ്ടെന്ന് നാമറിയുന്നു. യഥാര്‍ഥത്തില്‍ അവര്‍ തള്ളിപ്പറയുന്നത് നിന്നെയല്ല. മറിച്ച് ആ അക്രമികള്‍ തള്ളിപ്പറഞ്ഞുകൊണ്ടിരിക്കുന്നത് അല്ലാഹുവിന്റെ വചനങ്ങളെയാണ്.
\end{malayalam}}
\flushright{\begin{Arabic}
\quranayah[6][34]
\end{Arabic}}
\flushleft{\begin{malayalam}
നിനക്കുമുമ്പും നിരവധി ദൈവദൂതന്മാരെ അവരുടെ ജനം തള്ളിപ്പറഞ്ഞിട്ടുണ്ട്. എന്നാല്‍ നമ്മുടെ സഹായം വന്നെത്തുംവരെ തങ്ങളെ തള്ളിപ്പറഞ്ഞതും പീഡിപ്പിച്ചതുമൊക്കെ അവര്‍ ക്ഷമിക്കുകയായിരുന്നു. അല്ലാഹുവിന്റെ വചനങ്ങളെ മാറ്റിമറിക്കാന്‍ പോരുന്ന ആരുമില്ല. ദൈവദൂതന്മാരുടെ കഥകളില്‍ ചിലതൊക്കെ നിനക്കു വന്നുകിട്ടിയിട്ടുണ്ടല്ലോ.
\end{malayalam}}
\flushright{\begin{Arabic}
\quranayah[6][35]
\end{Arabic}}
\flushleft{\begin{malayalam}
എന്നിട്ടും ഈ ജനത്തിന്റെ അവഗണന നിനക്ക് അസഹ്യമാകുന്നുവെങ്കില്‍ ഭൂമിയില്‍ ഒരു തുരങ്കമുണ്ടാക്കിയോ ആകാശത്തേക്ക് കോണിവെച്ചോ അവര്‍ക്ക് എന്തെങ്കിലും ദൃഷ്ടാന്തം എത്തിച്ചുകൊടുക്കാന്‍ നിനക്ക് കഴിയുമെങ്കില്‍ അങ്ങനെ ചെയ്തുകൊള്ളുക. അല്ലാഹു ഇച്ഛിച്ചിരുന്നെങ്കില്‍ അവരെയൊക്കെ അവന്‍ സന്മാര്‍ഗത്തില്‍ ഒരുമിച്ചുകൂട്ടുമായിരുന്നു. അതിനാല്‍ നീ ഒരിക്കലും അവിവേകികളുടെ കൂട്ടത്തില്‍ പെട്ടുപോകരുത്.
\end{malayalam}}
\flushright{\begin{Arabic}
\quranayah[6][36]
\end{Arabic}}
\flushleft{\begin{malayalam}
കേട്ടുമനസ്സിലാക്കുന്നവരേ ഉത്തരമേകുകയുള്ളൂ. മരിച്ചവരെ അല്ലാഹു ഉയിര്‍ത്തെഴുന്നേല്‍പിക്കും. എന്നിട്ട് അവരെ തന്റെഅരികിലേക്ക് തിരിച്ചുകൊണ്ടുപോകും
\end{malayalam}}
\flushright{\begin{Arabic}
\quranayah[6][37]
\end{Arabic}}
\flushleft{\begin{malayalam}
അവര്‍ ചോദിക്കുന്നു: "ഈ പ്രവാചകന് തന്റെ നാഥനില്‍ നിന്ന് ഒരു ദൃഷ്ടാന്തവും അവതരിക്കാത്തതെന്ത്?” പറയുക: ദൃഷ്ടാന്തം ഇറക്കാന്‍ കഴിവുറ്റവന്‍ തന്നെയാണ് അല്ലാഹു; എന്നാല്‍ അവരിലേറെ പേരും അതറിയുന്നില്ല.
\end{malayalam}}
\flushright{\begin{Arabic}
\quranayah[6][38]
\end{Arabic}}
\flushleft{\begin{malayalam}
ഭൂമിയില്‍ ചരിക്കുന്ന ഏത് ജീവിയും ഇരുചിറകുകളില്‍ പറക്കുന്ന ഏതു പറവയും നിങ്ങളെപ്പോലുള്ള ചില സമൂഹങ്ങളാണ്. മൂലപ്രമാണത്തില്‍ നാമൊന്നും വിട്ടുകളഞ്ഞിട്ടില്ല. പിന്നീട് അവരെല്ലാം തങ്ങളുടെ നാഥങ്കല്‍ ഒരുമിച്ചുചേര്‍ക്കപ്പെടും.
\end{malayalam}}
\flushright{\begin{Arabic}
\quranayah[6][39]
\end{Arabic}}
\flushleft{\begin{malayalam}
നമ്മുടെ തെളിവുകളെ തള്ളിക്കളഞ്ഞവര്‍ ഇരുളിലകപ്പെട്ട ബധിരരും മൂകരുമാകുന്നു. അല്ലാഹു അവനിച്ഛിക്കുന്നവരെ ദുര്‍മാര്‍ഗത്തിലാക്കുന്നു. അവനിച്ഛിക്കുന്നവരെ നേര്‍വഴിയിലുമാക്കുന്നു.
\end{malayalam}}
\flushright{\begin{Arabic}
\quranayah[6][40]
\end{Arabic}}
\flushleft{\begin{malayalam}
പറയുക: അല്ലാഹുവിന്റെ ശിക്ഷ നിങ്ങളെ ബാധിക്കുകയോ അല്ലെങ്കില്‍ അന്ത്യനാള്‍ വന്നെത്തുകയോ ചെയ്താല്‍ അല്ലാഹു അല്ലാത്ത ആരെയെങ്കിലും നിങ്ങള്‍ വിളിച്ചു പ്രാര്‍ഥിക്കുമോ? നിങ്ങള്‍ സത്യസന്ധരെങ്കില്‍ ചിന്തിച്ച് പറയൂ!
\end{malayalam}}
\flushright{\begin{Arabic}
\quranayah[6][41]
\end{Arabic}}
\flushleft{\begin{malayalam}
ഇല്ല. ഉറപ്പായും അപ്പോള്‍ അവനെ മാത്രമേ നിങ്ങള്‍ വിളിച്ചു പ്രാര്‍ഥിക്കുകയുള്ളൂ. അങ്ങനെ അവനിച്ഛിക്കുന്നുവെങ്കില്‍ നിങ്ങള്‍ കേണുകൊണ്ടിരിക്കുന്നത് ഏതൊരു വിപത്തിന്റെ പേരിലാണോ അതിനെ അവന്‍ തട്ടിമാറ്റിയേക്കും. അന്നേരം അല്ലാഹുവില്‍ പങ്കുചേര്‍ക്കുന്നവയെ നിങ്ങള്‍ മറക്കുകയും ചെയ്യും.
\end{malayalam}}
\flushright{\begin{Arabic}
\quranayah[6][42]
\end{Arabic}}
\flushleft{\begin{malayalam}
നിനക്കുമുമ്പും നിരവധി സമുദായങ്ങളിലേക്ക് നാം ദൂതന്മാരെ നിയോഗിച്ചിട്ടുണ്ട്. പിന്നെ ആ സമുദായങ്ങളെ നാം പീഡനങ്ങളാലും പ്രയാസങ്ങളാലും പിടികൂടി. അവര്‍ വിനീതരാകാന്‍.
\end{malayalam}}
\flushright{\begin{Arabic}
\quranayah[6][43]
\end{Arabic}}
\flushleft{\begin{malayalam}
അങ്ങനെ നമ്മുടെ ദുരിതം അവരെ ബാധിച്ചപ്പോള്‍ അവര്‍ വിനീതരാവാതിരുന്നതെന്ത്? എന്നല്ല, അവരുടെ ഹൃദയങ്ങള്‍ കൂടുതല്‍ കടുത്തുപോവുകയാണുണ്ടായത്. അവര്‍ ചെയ്തുകൊണ്ടിരിക്കുന്നതെല്ലാം വളരെ നല്ല കാര്യങ്ങളാണെന്ന് പിശാച് അവരെ തോന്നിപ്പിക്കുകയും ചെയ്തു.
\end{malayalam}}
\flushright{\begin{Arabic}
\quranayah[6][44]
\end{Arabic}}
\flushleft{\begin{malayalam}
അവര്‍ക്കു നാം നല്‍കിയ ഉദ്ബോധനം അവര്‍ മറന്നപ്പോള്‍ സകല സൌഭാഗ്യങ്ങളുടെയും കവാടങ്ങള്‍ നാമവര്‍ക്ക് തുറന്നുകൊടുത്തു. അങ്ങനെ തങ്ങള്‍ക്കു നല്‍കപ്പെട്ടവയില്‍ അവര്‍ അതിരറ്റു സന്തോഷിച്ചു കൊണ്ടിരിക്കെ പൊടുന്നനെ നാമവരെ പിടികൂടി. അപ്പോഴതാ അവര്‍ നിരാശരായിത്തീരുന്നു.
\end{malayalam}}
\flushright{\begin{Arabic}
\quranayah[6][45]
\end{Arabic}}
\flushleft{\begin{malayalam}
അക്രമികളായ ആ ജനത അങ്ങനെ നിശ്ശേഷം നശിപ്പിക്കപ്പെട്ടു. സര്‍വലോകസംരക്ഷകനായ അല്ലാഹുവിന് സ്തുതി.
\end{malayalam}}
\flushright{\begin{Arabic}
\quranayah[6][46]
\end{Arabic}}
\flushleft{\begin{malayalam}
ചോദിക്കുക: നിങ്ങള്‍ ചിന്തിച്ചുനോക്കിയിട്ടുണ്ടോ? അല്ലാഹു നിങ്ങളുടെ കേള്‍വിയും കാഴ്ചയും നഷ്ടപ്പെടുത്തുകയും നിങ്ങളുടെ ഹൃദയങ്ങള്‍ക്ക് മുദ്രവെക്കുകയും ചെയ്താല്‍ അല്ലാഹു അല്ലാതെ ഏതു ദൈവമാണ് നിങ്ങള്‍ക്കവ വീണ്ടെടുത്ത് തരിക? നോക്കൂ, നാം എങ്ങനെയൊക്കെയാണ്അവര്‍ക്ക് തെളിവുകള്‍ വിവരിച്ചുകൊടുക്കുന്നതെന്ന്. എന്നിട്ടും അവര്‍ പിന്തിരിഞ്ഞുപോവുകയാണ്!
\end{malayalam}}
\flushright{\begin{Arabic}
\quranayah[6][47]
\end{Arabic}}
\flushleft{\begin{malayalam}
ചോദിക്കുക: നിങ്ങള്‍ ചിന്തിച്ചിട്ടുണ്ടോ? പെട്ടെന്നോ പ്രത്യക്ഷത്തിലോ വല്ല ദൈവശിക്ഷയും നിങ്ങള്‍ക്കു വന്നെത്തിയാല്‍ എന്തായിരിക്കും സ്ഥിതി? അക്രമികളായ ജനതയല്ലാതെ നശിപ്പിക്കപ്പെടുമോ?
\end{malayalam}}
\flushright{\begin{Arabic}
\quranayah[6][48]
\end{Arabic}}
\flushleft{\begin{malayalam}
ശുഭവാര്‍ത്ത അറിയിക്കുന്നവരും താക്കീത് നല്‍കുന്നവരുമായല്ലാതെ നാം ദൂതന്മാരെ അയക്കാറില്ല. അതിനാല്‍ സത്യവിശ്വാസം സ്വീകരിക്കുകയും കര്‍മങ്ങള്‍ കുറ്റമറ്റതാക്കുകയും ചെയ്യുന്നവര്‍ക്ക് ഒന്നും പേടിക്കാനില്ല. അവര്‍ ദുഃഖിക്കേണ്ടിവരികയുമില്ല.
\end{malayalam}}
\flushright{\begin{Arabic}
\quranayah[6][49]
\end{Arabic}}
\flushleft{\begin{malayalam}
എന്നാല്‍ നമ്മുടെ തെളിവുകളെ തള്ളിപ്പറഞ്ഞവരെ, തങ്ങളുടെ ധിക്കാരം കാരണമായി ശിക്ഷ ബാധിക്കുക തന്നെ ചെയ്യും.
\end{malayalam}}
\flushright{\begin{Arabic}
\quranayah[6][50]
\end{Arabic}}
\flushleft{\begin{malayalam}
പറയുക: എന്റെ വശം അല്ലാഹുവിന്റെ ഖജനാവുകളുണ്ടെന്ന് ഞാന്‍ നിങ്ങളോട് അവകാശപ്പെടുന്നില്ല. അഭൌതിക കാര്യങ്ങള്‍ ഞാന്‍ അറിയുന്നുമില്ല. ഞാനൊരു മലക്കാണെന്നും നിങ്ങളോടു പറയുന്നില്ല. എനിക്കു അല്ലാഹുവില്‍ നിന്ന് ബോധനമായി ലഭിക്കുന്നവയല്ലാതൊന്നും ഞാന്‍ പിന്‍പറ്റുന്നില്ല. ചോദിക്കുക: കുരുടനും കാഴ്ചയുള്ളവനും ഒരുപോലെയാണോ? നിങ്ങള്‍ ചിന്തിക്കുന്നില്ലേ?
\end{malayalam}}
\flushright{\begin{Arabic}
\quranayah[6][51]
\end{Arabic}}
\flushleft{\begin{malayalam}
തങ്ങളുടെ നാഥന്റെ സന്നിധിയില്‍ ഒരുനാള്‍ ഒരുമിച്ചുകൂട്ടപ്പെടുമെന്ന് ഭയപ്പെടുന്നവര്‍ക്ക് ഇതു വഴി നീ മുന്നറിയിപ്പു നല്‍കുക: അവനെക്കൂടാതെ ഒരു രക്ഷകനും ശിപാര്‍ശകനും അവര്‍ക്കില്ലെന്ന്. അവര്‍ ഭക്തരായേക്കാം.
\end{malayalam}}
\flushright{\begin{Arabic}
\quranayah[6][52]
\end{Arabic}}
\flushleft{\begin{malayalam}
തങ്ങളുടെ നാഥന്റെ പ്രീതി പ്രതീക്ഷിച്ച് രാവിലെയും വൈകുന്നേരവും അവനോടു പ്രാര്‍ഥിച്ചുകൊണ്ടിരിക്കുന്നവരെ നീ ആട്ടിയകറ്റരുത്. അവരുടെ കണക്കില്‍പെട്ട ഒന്നിന്റെയും ബാധ്യത നിനക്കില്ല. നിന്റെ കണക്കിലുള്ള ഒന്നിന്റെയും ബാധ്യത അവര്‍ക്കുമില്ല. എന്നിട്ടും അവരെ ആട്ടിയകറ്റിയാല്‍ നീ അക്രമികളില്‍ പെട്ടുപോകും.
\end{malayalam}}
\flushright{\begin{Arabic}
\quranayah[6][53]
\end{Arabic}}
\flushleft{\begin{malayalam}
അവ്വിധം അവരില്‍ ചിലരെ നാം മറ്റുചിലരാല്‍ പരീക്ഷണത്തിലകപ്പെടുത്തിയിരിക്കുന്നു. "ഞങ്ങളുടെ ഇടയില്‍നിന്ന് ഇവരെയാണോ അല്ലാഹു അനുഗ്രഹിച്ചത്” എന്ന് അവര്‍ പറയാനാണിത്. നന്ദിയുള്ളവരെ നന്നായറിയുന്നവന്‍ അല്ലാഹുവല്ലയോ?
\end{malayalam}}
\flushright{\begin{Arabic}
\quranayah[6][54]
\end{Arabic}}
\flushleft{\begin{malayalam}
നമ്മുടെ വചനങ്ങളില്‍ വിശ്വസിക്കുന്നവര്‍ നിന്നെ സമീപിച്ചാല്‍ നീ പറയണം: നിങ്ങള്‍ക്കു സമാധാനം. നിങ്ങളുടെ നാഥന്‍ കാരുണ്യത്തെ തന്റെ ബാധ്യതയാക്കിയിരിക്കുന്നു. അതിനാല്‍ നിങ്ങളിലാരെങ്കിലും അറിവില്ലായ്മ കാരണം വല്ല തെറ്റും ചെയ്യുകയും പിന്നീട് പശ്ചാത്തപിച്ച് കര്‍മങ്ങള്‍ നന്നാക്കുകയുമാണെങ്കില്‍, അറിയുക: തീര്‍ച്ചയായും അല്ലാഹു ഏറെ പൊറുക്കുന്നവനും ദയാപരനുമാണ്.
\end{malayalam}}
\flushright{\begin{Arabic}
\quranayah[6][55]
\end{Arabic}}
\flushleft{\begin{malayalam}
ഇവ്വിധം നാം തെളിവുകള്‍ വിവരിച്ചുതരുന്നു. പാപികളുടെ വഴി വ്യക്തമായി വേര്‍തിരിഞ്ഞു കാണാനാണിത്.
\end{malayalam}}
\flushright{\begin{Arabic}
\quranayah[6][56]
\end{Arabic}}
\flushleft{\begin{malayalam}
പറയുക: “അല്ലാഹുവെക്കൂടാതെ നിങ്ങള്‍ വിളിച്ചു പ്രാര്‍ഥിക്കുന്നവയെ പൂജിക്കുന്നത് എനിക്കു തീര്‍ത്തും വിലക്കപ്പെട്ടിരിക്കുന്നു.” പറയുക: “നിങ്ങളുടെ തന്നിഷ്ടങ്ങളെ ഞാന്‍ പിന്‍പറ്റുകയില്ല. അങ്ങനെ ചെയ്താല്‍ ഞാന്‍ വഴിപിഴച്ചവനാകും. ഞാനൊരിക്കലും നേര്‍വഴി പ്രാപിച്ചവരില്‍ പെടുകയുമില്ല.”
\end{malayalam}}
\flushright{\begin{Arabic}
\quranayah[6][57]
\end{Arabic}}
\flushleft{\begin{malayalam}
പറയുക: “ഉറപ്പായും ഞാനെന്റെ നാഥനില്‍ നിന്നുള്ള വ്യക്തമായ പ്രമാണം മുറുകെപ്പിടിക്കുന്നവനാണ്. നിങ്ങളോ അതിനെ തള്ളിപ്പറഞ്ഞവരും. നിങ്ങള്‍ തിരക്കുകൂട്ടിക്കൊണ്ടിരിക്കുന്ന അക്കാര്യം എന്റെ വശമില്ല. വിധിത്തീര്‍പ്പിനുള്ള സമസ്താധികാരവും അല്ലാഹുവിനു മാത്രമാണ്. അവന്‍ സത്യാവസ്ഥ വിവരിച്ചുതരും. തീരുമാനമെടുക്കുന്നവരില്‍ അത്യുത്തമന്‍ അവനത്രെ.”
\end{malayalam}}
\flushright{\begin{Arabic}
\quranayah[6][58]
\end{Arabic}}
\flushleft{\begin{malayalam}
പറയുക: നിങ്ങള്‍ ധൃതികൂട്ടിക്കൊണ്ടിരിക്കുന്ന അക്കാര്യം എന്റെ വശമുണ്ടായിരുന്നെങ്കില്‍ എനിക്കും നിങ്ങള്‍ക്കുമിടയില്‍ പെട്ടെന്ന് കാര്യം തീരുമാനിക്കപ്പെടുമായിരുന്നു. അക്രമികളെക്കുറിച്ച് നന്നായറിയുന്നവനാണ് അല്ലാഹു.
\end{malayalam}}
\flushright{\begin{Arabic}
\quranayah[6][59]
\end{Arabic}}
\flushleft{\begin{malayalam}
അഭൌതിക കാര്യങ്ങളുടെ താക്കോലുകള്‍ അല്ലാഹുവിന്റെ വശമാണ്. അവനല്ലാതെ അതറിയുകയില്ല. കരയിലും കടലിലുമുള്ളതെല്ലാം അവനറിയുന്നു. അവനറിയാതെ ഒരിലപോലും പൊഴിയുന്നില്ല. ഭൂമിയുടെ ഉള്‍ഭാഗത്ത് ഒരു ധാന്യമണിയോ പച്ചയും ഉണങ്ങിയതുമായ ഏതെങ്കിലും വസ്തുവോ ഒന്നും തന്നെ വ്യക്തമായ മൂലപ്രമാണത്തില്‍ രേഖപ്പെടുത്താത്തതായി ഇല്ല.
\end{malayalam}}
\flushright{\begin{Arabic}
\quranayah[6][60]
\end{Arabic}}
\flushleft{\begin{malayalam}
രാത്രിയില്‍ നിങ്ങളുടെ ജീവനെ പിടിച്ചുവെക്കുന്നത് അവനാണ്. പകലില്‍ നിങ്ങള്‍ ചെയ്യുന്നതെല്ലാം അവനറിയുകയും ചെയ്യുന്നു. പിന്നീട് നിശ്ചിത ജീവിതാവധി പൂര്‍ത്തീകരിക്കാനായി അവന്‍ നിങ്ങളെ പകലില്‍ എഴുന്നേല്‍പിക്കുന്നു. അതിനുശേഷം അവങ്കലേക്കുതന്നെയാണ് നിങ്ങള്‍ തിരിച്ചുചെല്ലുന്നത്. അപ്പോള്‍, അവന്‍ നിങ്ങള്‍ പ്രവര്‍ത്തിച്ചുകൊണ്ടിരുന്നതിനെപ്പറ്റിയെല്ലാം നിങ്ങളെ വിവരമറിയിക്കും.
\end{malayalam}}
\flushright{\begin{Arabic}
\quranayah[6][61]
\end{Arabic}}
\flushleft{\begin{malayalam}
അല്ലാഹു തന്റെ ദാസന്മാരുടെമേല്‍ പൂര്‍ണാധികാരമുള്ളവനാണ്. നിങ്ങളുടെമേല്‍ അവന്‍ കാവല്‍ക്കാരെ നിയോഗിക്കുന്നു. അങ്ങനെ നിങ്ങളിലാര്‍ക്കെങ്കിലും മരണസമയമായാല്‍ നമ്മുടെ ദൂതന്മാര്‍ അയാളുടെ ആയുസ്സവസാനിപ്പിക്കുന്നു. അതിലവര്‍ ഒരു വീഴ്ചയും വരുത്തുകയില്ല.
\end{malayalam}}
\flushright{\begin{Arabic}
\quranayah[6][62]
\end{Arabic}}
\flushleft{\begin{malayalam}
പിന്നെ അവരെ തങ്ങളുടെ സാക്ഷാല്‍ യജമാനനായ അല്ലാഹുവിന്റെ സന്നിധിയിലേക്ക് മടക്കിയയക്കും. അറിയുക: വിധിത്തീര്‍പ്പിനുള്ള അധികാരം അല്ലാഹുവിനാണ്. അവന്‍ അതിവേഗം വിചാരണ ചെയ്യുന്നവനാകുന്നു.
\end{malayalam}}
\flushright{\begin{Arabic}
\quranayah[6][63]
\end{Arabic}}
\flushleft{\begin{malayalam}
ചോദിക്കുക: “ഈ വിപത്തില്‍ നിന്ന് ഞങ്ങളെ രക്ഷിച്ചാല്‍ ഉറപ്പായും ഞങ്ങള്‍ നന്ദിയുള്ളവരാകു”മെന്ന് നിങ്ങള്‍ വിനയത്തോടും സ്വകാര്യമായും പ്രാര്‍ഥിക്കുമ്പോള്‍ ആരാണ് കരയുടെയും കടലിന്റെയും കൂരിരുളില്‍ നിന്ന് നിങ്ങളെ രക്ഷിക്കുന്നത്?
\end{malayalam}}
\flushright{\begin{Arabic}
\quranayah[6][64]
\end{Arabic}}
\flushleft{\begin{malayalam}
പറയുക: അല്ലാഹുവാണ് അവയില്‍ നിന്നും മറ്റെല്ലാ വിപത്തുകളില്‍ നിന്നും നിങ്ങളെ രക്ഷിക്കുന്നത്. എന്നിട്ടും നിങ്ങളവന് പങ്കുകാരെ സങ്കല്‍പിക്കുകയാണല്ലോ.
\end{malayalam}}
\flushright{\begin{Arabic}
\quranayah[6][65]
\end{Arabic}}
\flushleft{\begin{malayalam}
പറയുക: നിങ്ങളുടെ മുകള്‍ഭാഗത്തുനിന്നോ കാല്‍ച്ചുവട്ടില്‍ നിന്നോ നിങ്ങളുടെമേല്‍ ശിക്ഷ വരുത്താന്‍ കഴിവുറ്റവനാണവന്‍. അല്ലെങ്കില്‍ നിങ്ങളെ പല കക്ഷികളാക്കി ആശയക്കുഴപ്പത്തിലകപ്പെടുത്തി പരസ്പരം പീഡനമേല്‍പിക്കാനും അവനു കഴിയും. നോക്കൂ, അവര്‍ കാര്യം മനസ്സിലാക്കാനായി എവ്വിധമാണ് നാം തെളിവുകള്‍ വിവരിച്ചുകൊടുക്കുന്നത്.
\end{malayalam}}
\flushright{\begin{Arabic}
\quranayah[6][66]
\end{Arabic}}
\flushleft{\begin{malayalam}
നിന്റെ ജനത അതിനെ തള്ളിക്കളഞ്ഞിരിക്കുന്നു. അതാകട്ടെ സത്യവുമാണ്. പറയുക: “ഞാന്‍ നിങ്ങളുടെ കൈകാര്യകര്‍ത്താവൊന്നുമല്ല.”
\end{malayalam}}
\flushright{\begin{Arabic}
\quranayah[6][67]
\end{Arabic}}
\flushleft{\begin{malayalam}
ഓരോ വാര്‍ത്തക്കും അത് പുലരുന്ന സന്ദര്‍ഭമുണ്ട്. അത് പിന്നീട് നിങ്ങള്‍ അറിയുക തന്നെ ചെയ്യും.
\end{malayalam}}
\flushright{\begin{Arabic}
\quranayah[6][68]
\end{Arabic}}
\flushleft{\begin{malayalam}
നമ്മുടെ വചനങ്ങളെ ആരെങ്കിലും പരിഹസിക്കുന്നതു നിന്റെ ശ്രദ്ധയില്‍പ്പെട്ടാല്‍ അവര്‍ മറ്റു വല്ല സംസാരത്തിലും വ്യാപൃതമാവും വരെ നീ അവരില്‍നിന്ന് അകന്നു നില്‍ക്കുക. വല്ലപ്പോഴും പിശാച് നിന്നെ മറപ്പിച്ചാല്‍ ഓര്‍മ വന്ന ശേഷം നീ ആ അതിക്രമികളോടൊപ്പമിരിക്കരുത്.
\end{malayalam}}
\flushright{\begin{Arabic}
\quranayah[6][69]
\end{Arabic}}
\flushleft{\begin{malayalam}
അവരുടെ വിചാരണയില്‍ വരുന്ന ഒന്നിന്റെയും ബാധ്യത ഭക്തന്മാര്‍ക്കില്ല. എന്നാല്‍ അവരെ ഓര്‍മിപ്പിക്കേണ്ട ഉത്തരവാദിത്വമുണ്ട്. അതുവഴി അവര്‍ ഭക്തി പുലര്‍ത്തുന്നവരായേക്കാം.
\end{malayalam}}
\flushright{\begin{Arabic}
\quranayah[6][70]
\end{Arabic}}
\flushleft{\begin{malayalam}
തങ്ങളുടെ മതത്തെ കളിയും തമാശയുമാക്കുകയും ലൌകികജീവിതത്തില്‍ വഞ്ചിതരാവുകയും ചെയ്തവരെ വിട്ടേക്കുക. അതോടൊപ്പം ഈ ഖുര്‍ആനുപയോഗിച്ച് അവരെ ഉദ്ബോധിപ്പിക്കുക. ആരും തങ്ങള്‍ ചെയ്തുകൂട്ടിയതിന്റെ പേരില്‍ നാശത്തിലകപ്പെടാതിരിക്കാനാണിത്. ആര്‍ക്കും അല്ലാഹുവെക്കൂടാതെ ഒരു രക്ഷകനോ ശിപാര്‍ശകനോ ഇല്ല. എന്തു തന്നെ പ്രായശ്ചിത്തം നല്‍കിയാലും അവരില്‍ നിന്ന് അത് സ്വീകരിക്കപ്പെടുന്നതല്ല. തങ്ങള്‍ ചെയ്തുകൂട്ടിയതിനാല്‍ നാശത്തിലകപ്പെട്ടവരാണവര്‍. തങ്ങളുടെ സത്യനിഷേധം കാരണമായി, ചുട്ടുപൊള്ളുന്ന കുടിനീരാണ് അവര്‍ക്കുണ്ടാവുക. നോവേറിയ ശിക്ഷയും.
\end{malayalam}}
\flushright{\begin{Arabic}
\quranayah[6][71]
\end{Arabic}}
\flushleft{\begin{malayalam}
ചോദിക്കുക: അല്ലാഹുവെക്കൂടാതെ, ഞങ്ങള്‍ക്കു ഗുണമോ ദോഷമോ വരുത്താനാവാത്തവയെ ഞങ്ങള്‍ വിളിച്ചു പ്രാര്‍ഥിക്കുകയോ? അങ്ങനെ, അല്ലാഹു ഞങ്ങളെ നേര്‍വഴിയിലാക്കിയ ശേഷം വീണ്ടും പിറകോട്ട് തിരിച്ചുപോവുകയോ? പിശാചിനാല്‍ വഴിപിഴച്ച് ഭൂമിയില്‍ പരിഭ്രാന്തനായി അലയുന്നവനെപ്പോലെ ആവുകയോ? അവന് ചില കൂട്ടുകാരുണ്ട്. അവര്‍ “ഇങ്ങോട്ടുവരൂ” എന്നു പറഞ്ഞ് നേര്‍വഴിയിലേക്ക് അവനെ ക്ഷണിക്കുന്നു. പറയുക: തീര്‍ച്ചയായും അല്ലാഹുവിന്റെ മാര്‍ഗദര്‍ശനമാണ് യഥാര്‍ഥ വഴികാട്ടി. പ്രപഞ്ചനാഥന്ന് വഴിപ്പെടാന്‍ ഞങ്ങളോട് കല്‍പിച്ചിരിക്കുന്നു.
\end{malayalam}}
\flushright{\begin{Arabic}
\quranayah[6][72]
\end{Arabic}}
\flushleft{\begin{malayalam}
നമസ്കാരം നിഷ്ഠയോടെ നിര്‍വഹിക്കാനും അല്ലാഹുവോട് ഭക്തിപുലര്‍ത്താനും ഞങ്ങളോട് ആജ്ഞാപിച്ചിരിക്കുന്നു. അവന്റെ സന്നിധിയിലാണ് നിങ്ങളെയെല്ലാം ഒരുമിച്ചുകൂട്ടുക.
\end{malayalam}}
\flushright{\begin{Arabic}
\quranayah[6][73]
\end{Arabic}}
\flushleft{\begin{malayalam}
അവനാണ് ആകാശഭൂമികളെ യാഥാര്‍ഥ്യ നിഷ്ഠമായി സൃഷ്ടിച്ചവന്‍. അവന്‍ “ഉണ്ടാവുക” എന്നു പറയുംനാള്‍ അതു സംഭവിക്കുക തന്നെ ചെയ്യും. അവന്റെ വചനം സത്യമാകുന്നു. കാഹളത്തില്‍ ഊതുംനാള്‍ സര്‍വാധിപത്യം അവനുമാത്രമായിരിക്കും. ദൃശ്യവും അദൃശ്യവും നന്നായറിയുന്നവനാണവന്‍. അവന്‍ യുക്തിമാനും സൂക്ഷ്മജ്ഞനുമാണ്.
\end{malayalam}}
\flushright{\begin{Arabic}
\quranayah[6][74]
\end{Arabic}}
\flushleft{\begin{malayalam}
ഓര്‍ക്കുക: ഇബ്റാഹീം തന്റെ പിതാവ് ആസറിനോടു പറഞ്ഞ സന്ദര്‍ഭം: "വിഗ്രഹങ്ങളെയാണോ താങ്കള്‍ ദൈവങ്ങളാക്കിയിരിക്കുന്നത്? തീര്‍ച്ചയായും താങ്കളും താങ്കളുടെ ജനതയും വ്യക്തമായ വഴികേടിലാണെന്ന് ഞാന്‍ മനസ്സിലാക്കുന്നു.”
\end{malayalam}}
\flushright{\begin{Arabic}
\quranayah[6][75]
\end{Arabic}}
\flushleft{\begin{malayalam}
അവ്വിധം തന്നെയാണ് ഇബ്റാഹീമിനു നാം ആകാശഭൂമികളിലെ നമ്മുടെ ആധിപത്യവ്യവസ്ഥ കാണിച്ചുകൊടുത്തത്. അദ്ദേഹം അടിയുറച്ച സത്യവിശ്വാസിയാകാന്‍.
\end{malayalam}}
\flushright{\begin{Arabic}
\quranayah[6][76]
\end{Arabic}}
\flushleft{\begin{malayalam}
അങ്ങനെ രാവ് അദ്ദേഹത്തെ ആവരണം ചെയ്തപ്പോള്‍ അദ്ദേഹം ഒരു നക്ഷത്രത്തെ കണ്ടു. അപ്പോള്‍ പറഞ്ഞു: "ഇതാണെന്റെ ദൈവം.” പിന്നെ അതസ്തമിച്ചപ്പോള്‍ അദ്ദേഹം പറഞ്ഞു: "അസ്തമിച്ചുപോകുന്നവയെ ഞാന്‍ ഇഷ്ടപ്പെടുന്നില്ല.”
\end{malayalam}}
\flushright{\begin{Arabic}
\quranayah[6][77]
\end{Arabic}}
\flushleft{\begin{malayalam}
പിന്നീട് ചന്ദ്രന്‍ ഉദിച്ചുയരുന്നത് കണ്ടപ്പോള്‍ അദ്ദേഹം പറഞ്ഞു: "ഇതാ; ഇതാണെന്റെ ദൈവം.” അതും അസ്തമിച്ചപ്പോള്‍ അദ്ദേഹം പറഞ്ഞു: "എന്റെ ദൈവം എനിക്ക് നേര്‍വഴി കാണിച്ചു തരുന്നില്ലെങ്കില്‍ തീര്‍ച്ചയായും ഞാന്‍ വഴി പിഴച്ചവരില്‍ പെട്ടുപോകും.”
\end{malayalam}}
\flushright{\begin{Arabic}
\quranayah[6][78]
\end{Arabic}}
\flushleft{\begin{malayalam}
പിന്നീട് സൂര്യന്‍ ഉദിച്ചുവരുന്നതുകണ്ടപ്പോള്‍ അദ്ദേഹം പറഞ്ഞു: "ഇതാണെന്റെ ദൈവം! ഇത് മറ്റെല്ലാറ്റിനെക്കാളും വലുതാണ്.” അങ്ങനെ അതും അസ്തമിച്ചപ്പോള്‍ അദ്ദേഹം പ്രഖ്യാപിച്ചു: "എന്റെ ജനമേ, നിങ്ങള്‍ അല്ലാഹുവില്‍ പങ്കുചേര്‍ക്കുന്നതില്‍ നിന്നൊക്കെയും ഞാനിതാ മുക്തനായിരിക്കുന്നു;
\end{malayalam}}
\flushright{\begin{Arabic}
\quranayah[6][79]
\end{Arabic}}
\flushleft{\begin{malayalam}
"തീര്‍ച്ചയായും ഞാന്‍ നേര്‍വഴിയിലുറച്ചുനിന്നുകൊണ്ട് എന്റെ മുഖം ആകാശഭൂമികളെ സൃഷ്ടിച്ചവനിലേക്ക് തിരിച്ചിരിക്കുന്നു. ഞാനൊരിക്കലും ബഹുദൈവ വിശ്വാസികളില്‍ പെട്ടവനല്ല; തീര്‍ച്ച.”
\end{malayalam}}
\flushright{\begin{Arabic}
\quranayah[6][80]
\end{Arabic}}
\flushleft{\begin{malayalam}
തന്റെ ജനം അദ്ദേഹത്തോട് തര്‍ക്കത്തിലേര്‍പ്പെട്ടു. അപ്പോള്‍ അദ്ദേഹം ചോദിച്ചു: "അല്ലാഹുവിന്റെ കാര്യത്തിലാണോ നിങ്ങളെന്നോടു തര്‍ക്കിക്കുന്നത്? അവനെന്നെ നേര്‍വഴിയിലാക്കിയിരിക്കുന്നു. നിങ്ങള്‍ അവന്റെ പങ്കാളികളാക്കുന്ന ഒന്നിനെയും ഞാന്‍ പേടിക്കുന്നില്ല. എന്റെ നാഥന്‍ ഇച്ഛിക്കുന്നതല്ലാതെ ഒന്നും ഇവിടെ സംഭവിക്കുകയില്ല. എന്റെ നാഥന്റെ അറിവ് എല്ലാറ്റിനെയും ഉള്‍ക്കൊള്ളുന്നു. എന്നിട്ടും നിങ്ങള്‍ ചിന്തിച്ചു മനസ്സിലാക്കുന്നില്ലേ?
\end{malayalam}}
\flushright{\begin{Arabic}
\quranayah[6][81]
\end{Arabic}}
\flushleft{\begin{malayalam}
"നിങ്ങള്‍ അല്ലാഹുവിന് പങ്കാളികളാക്കുന്നവയെ ഞാനെങ്ങനെ പേടിക്കും? നിങ്ങളാകട്ടെ, അല്ലാഹു നിങ്ങള്‍ക്കൊരു തെളിവും തന്നിട്ടില്ലാത്തവയെ അവനില്‍ പങ്കാളികളാക്കുന്നതിനെക്കുറിച്ച് ഭയപ്പെടുന്നുമില്ല. നാം ഇരുവിഭാഗങ്ങളില്‍ ആരാണ് നിര്‍ഭയരായിരിക്കാന്‍ കൂടുതല്‍ അര്‍ഹര്‍? നിങ്ങള്‍ക്ക് തിരിച്ചറിവുണ്ടെങ്കില്‍ പറയൂ.”
\end{malayalam}}
\flushright{\begin{Arabic}
\quranayah[6][82]
\end{Arabic}}
\flushleft{\begin{malayalam}
വിശ്വസിക്കുകയും തങ്ങളുടെ വിശ്വാസത്തെ വികലധാരണകളാല്‍ വികൃതമാക്കാതിരിക്കുകയും ചെയ്തവര്‍ക്ക് ഒന്നും പേടിക്കേണ്ടതില്ല. നേര്‍വഴി പ്രാപിച്ചവരും അവര്‍ തന്നെ.
\end{malayalam}}
\flushright{\begin{Arabic}
\quranayah[6][83]
\end{Arabic}}
\flushleft{\begin{malayalam}
ഇബ്റാഹീമിന് തന്റെ ജനതക്കെതിരെ നാം നല്‍കിയ ന്യായം അതായിരുന്നു: നാമിച്ഛിക്കുന്നവര്‍ക്കു നാം പദവികള്‍ ഉയര്‍ത്തിക്കൊടുക്കുന്നു. നിന്റെ നാഥന്‍ യുക്തിമാനും അഭിജ്ഞനും തന്നെ; തീര്‍ച്ച.
\end{malayalam}}
\flushright{\begin{Arabic}
\quranayah[6][84]
\end{Arabic}}
\flushleft{\begin{malayalam}
അദ്ദേഹത്തിനു നാം ഇസ്ഹാഖിനെയും യഅ്ഖൂബിനെയും സമ്മാനിച്ചു. അവരെയൊക്കെ നാം നേര്‍വഴിയിലാക്കി. അതിനുമുമ്പ് നൂഹിനു നാം സത്യമാര്‍ഗം കാണിച്ചുകൊടുത്തിരുന്നു. അദ്ദേഹത്തിന്റെ സന്താനങ്ങളില്‍പ്പെട്ട ദാവൂദിനെയും സുലൈമാനെയും അയ്യൂബിനെയും യൂസുഫിനെയും മൂസായെയും ഹാറൂനെയും നാം നേര്‍വഴിയിലാക്കി. അവ്വിധം നാം സല്‍ക്കര്‍മികള്‍ക്ക് പ്രതിഫലം നല്‍കുന്നു.
\end{malayalam}}
\flushright{\begin{Arabic}
\quranayah[6][85]
\end{Arabic}}
\flushleft{\begin{malayalam}
സകരിയ്യാ, യഹ്യാ, ഈസാ, ഇല്‍യാസ് എന്നിവര്‍ക്കും നാം സന്മാര്‍ഗമരുളി. അവരൊക്കെയും സച്ചരിതരായിരുന്നു.
\end{malayalam}}
\flushright{\begin{Arabic}
\quranayah[6][86]
\end{Arabic}}
\flushleft{\begin{malayalam}
അവ്വിധം ഇസ്മാഈല്‍, അല്‍യസഅ്, യൂനുസ്, ലൂത്ത്വ് എന്നിവര്‍ക്കും നാം സന്മാര്‍ഗമേകി. അവരെയെല്ലാം നാം ലോകത്തുള്ള മറ്റാരെക്കാളും ശ്രേഷ്ഠരാക്കിയിരിക്കുന്നു.
\end{malayalam}}
\flushright{\begin{Arabic}
\quranayah[6][87]
\end{Arabic}}
\flushleft{\begin{malayalam}
അവ്വിധം അവരുടെ പിതാക്കളില്‍ നിന്നും മക്കളില്‍ നിന്നും സഹോദരങ്ങളില്‍ നിന്നും ചിലരെ നാം മഹാന്മാരാക്കിയിട്ടുണ്ട്. അവരെ നാം പ്രത്യേകം തെരഞ്ഞെടുക്കുകയും നേര്‍വഴിയില്‍ നയിക്കുകയും ചെയ്തു.
\end{malayalam}}
\flushright{\begin{Arabic}
\quranayah[6][88]
\end{Arabic}}
\flushleft{\begin{malayalam}
അതാണ് അല്ലാഹുവിന്റെ സന്മാര്‍ഗം. തന്റെ ദാസന്മാരില്‍ താനിച്ഛിക്കുന്നവരെ അവന്‍ നേര്‍വഴിയിലാക്കുന്നു. അവര്‍ അല്ലാഹുവില്‍ പങ്കുകാരെ സങ്കല്‍പിച്ചിരുന്നുവെങ്കില്‍ അവര്‍ക്ക് തങ്ങളുടെ പ്രവര്‍ത്തികളൊക്കെ പാഴായിപ്പോകുമായിരുന്നു.
\end{malayalam}}
\flushright{\begin{Arabic}
\quranayah[6][89]
\end{Arabic}}
\flushleft{\begin{malayalam}
നാം വേദവും വിജ്ഞാനവും പ്രവാചകത്വവും നല്‍കിയവരാണവര്‍. ഇപ്പോളിവര്‍ അതിനെ തള്ളിപ്പറയുന്നുവെങ്കില്‍ ഇവര്‍ അറിഞ്ഞിരിക്കട്ടെ: അതിനെ തള്ളിക്കളയാത്ത മറ്റൊരു ജനതയെയാണ് നാം അത് ഏല്‍പിച്ചുകൊടുത്തിട്ടുള്ളത്.
\end{malayalam}}
\flushright{\begin{Arabic}
\quranayah[6][90]
\end{Arabic}}
\flushleft{\begin{malayalam}
അവരെതന്നെയാണ് അല്ലാഹു നേര്‍വഴിയിലാക്കിയത്. അതിനാല്‍ അവരുടെ സത്യപാത നീയും പിന്തുടരുക. പറയുക: “ഇതിന്റെ പേരിലൊരു പ്രതിഫലവും ഞാന്‍ നിങ്ങളോടാവശ്യപ്പെടുന്നില്ല. ഇത് ലോകര്‍ക്കാകമാനമുള്ള ഉദ്ബോധനമല്ലാതൊന്നുമല്ല.”
\end{malayalam}}
\flushright{\begin{Arabic}
\quranayah[6][91]
\end{Arabic}}
\flushleft{\begin{malayalam}
“അല്ലാഹു ഒരാള്‍ക്കും ഒന്നും ഇറക്കിക്കൊടുത്തിട്ടില്ലെ”ന്ന് അവര്‍ വാദിച്ചത് അല്ലാഹുവിന്റെ മഹത്വം യഥാവിധി വിലയിരുത്തിക്കൊണ്ടല്ല. ചോദിക്കുക: ജനങ്ങള്‍ക്ക് വഴികാട്ടിയും വെളിച്ചവുമായി മൂസാ കൊണ്ടുവന്ന വേദപുസ്തകം ആരാണ് ഇറക്കിത്തന്നത്? നിങ്ങളതിനെ കേവലം കടലാസുതുണ്ടുകളാക്കി. അങ്ങനെ ചിലത് വെളിപ്പെടുത്തുകയും മറ്റു പലതും മറച്ചുവെക്കുകയും ചെയ്യുന്നു. നിങ്ങള്‍ക്കും നിങ്ങളുടെ പൂര്‍വപിതാക്കള്‍ക്കും അറിവില്ലാതിരുന്ന പലതും അതിലൂടെ നിങ്ങളെ പഠിപ്പിച്ചിട്ടുണ്ട്. പറയുക: "അല്ലാഹുവാണ് അതിറക്കിത്തന്നത്.” എന്നിട്ട് അവരെ തങ്ങളുടെ വിടുവായത്തങ്ങളില്‍ തന്നെ വിഹരിക്കാന്‍ വിട്ടേക്കുക.
\end{malayalam}}
\flushright{\begin{Arabic}
\quranayah[6][92]
\end{Arabic}}
\flushleft{\begin{malayalam}
നാം ഇറക്കിയ അനുഗൃഹീത ഗ്രന്ഥം ഇതാ? ഇതിനു മുമ്പുള്ളവയെ ശരിവെക്കുന്നതാണിത്. മാതൃനഗര ത്തിലും പരിസരങ്ങളിലുമുള്ളവര്‍ക്ക് മുന്നറിയിപ്പ് നല്‍കാനുള്ളതും. പരലോകത്തില്‍ വിശ്വസിക്കുന്നവരെല്ലാം ഈ വേദത്തിലും വിശ്വസിക്കുന്നു. അവര്‍ തങ്ങളുടെ നമസ്കാരം നിഷ്ഠയോടെ നിര്‍വഹിക്കുന്നു.
\end{malayalam}}
\flushright{\begin{Arabic}
\quranayah[6][93]
\end{Arabic}}
\flushleft{\begin{malayalam}
അല്ലാഹുവിന്റെ പേരില്‍ കള്ളം കെട്ടിയുണ്ടാക്കുകയോ; ഒരു ദിവ്യസന്ദേശവും ലഭിക്കാതെ, തനിക്ക് ദിവ്യബോധനം ലഭിച്ചിരിക്കുന്നുവെന്ന് വാദിക്കുകയോ, അല്ലാഹു അവതരിപ്പിച്ചതുപോലുള്ളത് താനും അവതരിപ്പിക്കുമെന്ന് വീമ്പ് പറയുകയോ ചെയ്തവനെക്കാള്‍ വലിയ അക്രമി ആരുണ്ട്? ആ അക്രമികള്‍ മരണവെപ്രാളത്തിലകപ്പെടുമ്പോള്‍ മലക്കുകള്‍ കൈനീട്ടിക്കൊണ്ട് ഇങ്ങനെ പറയുന്നു: "നിങ്ങള്‍ നിങ്ങളുടെ ആത്മാക്കളെ പുറത്തേക്ക് തള്ളുക; നിങ്ങള്‍ അല്ലാഹുവിന്റെ പേരില്‍ സത്യവിരുദ്ധമായത് പ്രചരിപ്പിച്ചു. അവന്റെ പ്രമാണങ്ങളെ അഹങ്കാരത്തോടെ തള്ളിക്കളഞ്ഞു. അതിനാല്‍ നിങ്ങള്‍ക്കു നന്നെ നിന്ദ്യമായ ശിക്ഷയുണ്ട്.” ഇതൊക്കെയും നിനക്ക് കാണാന്‍ കഴിഞ്ഞിരുന്നെങ്കില്‍!
\end{malayalam}}
\flushright{\begin{Arabic}
\quranayah[6][94]
\end{Arabic}}
\flushleft{\begin{malayalam}
അവരോട് പറയും: നിങ്ങളെ നാം ആദ്യതവണ സൃഷ്ടിച്ചപോലെ നിങ്ങളിതാ നമ്മുടെ അടുക്കല്‍ ഒറ്റയൊറ്റയായി വന്നെത്തിയിരിക്കുന്നു. നാം നിങ്ങള്‍ക്ക് അധീനപ്പെടുത്തിത്തന്നിരുന്നതെല്ലാം പിന്നില്‍ വിട്ടേച്ചുകൊണ്ടാണ് നിങ്ങള്‍ വന്നിരിക്കുന്നത്. നിങ്ങളുടെ കാര്യത്തില്‍ അല്ലാഹുവിന്റെ പങ്കുകാരെന്ന് നിങ്ങള്‍ അവകാശപ്പെട്ടിരുന്ന ശിപാര്‍ശകരെയൊന്നും ഇപ്പോള്‍ നാം നിങ്ങളോടൊപ്പം കാണുന്നില്ലല്ലോ. നിങ്ങള്‍ക്കിടയിലെ ബന്ധങ്ങളൊക്കെ അറ്റുപോയിരിക്കുന്നു. നിങ്ങളുടെ അവകാശവാദങ്ങളെല്ലാം നിങ്ങള്‍ക്ക് കൈമോശം വന്നിരിക്കുന്നു.
\end{malayalam}}
\flushright{\begin{Arabic}
\quranayah[6][95]
\end{Arabic}}
\flushleft{\begin{malayalam}
ധാന്യമണികളെയും പഴക്കുരുകളെയും പിളര്‍ക്കുന്നവന്‍ അല്ലാഹുവാണ്. ജീവനില്ലാത്തതില്‍ നിന്ന് ജീവനുള്ളതിനെ ഉല്‍പാദിപ്പിക്കുന്നതും ജീവനുള്ളതില്‍ നിന്ന് ജീവനില്ലാത്തതിനെ പുറത്തെടുക്കുന്നതും അവനാണ്. ഇതൊക്കെ ചെയ്യുന്നവനാണ് അല്ലാഹു. എന്നിട്ടും നിങ്ങളെങ്ങോട്ടാണ് വഴിതെറ്റിപ്പോകുന്നത്?
\end{malayalam}}
\flushright{\begin{Arabic}
\quranayah[6][96]
\end{Arabic}}
\flushleft{\begin{malayalam}
പ്രഭാതത്തെ വിടര്‍ത്തുന്നതവനാണ്. രാവിനെ അവന്‍ വിശ്രമവേളയാക്കി; സൂര്യചന്ദ്രന്മാരെ സമയനിര്‍ണയത്തിനുള്ള അടിസ്ഥാനവും. പ്രതാപിയും എല്ലാം അറിയുന്നവനുമായ അല്ലാഹുവിന്റെ ക്രമീകരണമാണിതെല്ലാം.
\end{malayalam}}
\flushright{\begin{Arabic}
\quranayah[6][97]
\end{Arabic}}
\flushleft{\begin{malayalam}
കരയിലെയും കടലിലെയും കൂരിരുളില്‍ നിങ്ങള്‍ക്ക് വഴി കാണാന്‍ നക്ഷത്രങ്ങളെ സൃഷ്ടിച്ചതും അവന്‍ തന്നെ. കാര്യമറിയാന്‍ കഴിയുന്നവര്‍ക്ക് നാമിതാ തെളിവുകള്‍ വിശദീകരിച്ചുതരുന്നു.
\end{malayalam}}
\flushright{\begin{Arabic}
\quranayah[6][98]
\end{Arabic}}
\flushleft{\begin{malayalam}
ഒരേയൊരു സത്തയില്‍ നിന്ന് നിങ്ങളെയൊക്കെ സൃഷ്ടിച്ചുണ്ടാക്കിയതും അവനാണ്. പിന്നെ നിങ്ങള്‍ക്കാവശ്യമായ വാസസ്ഥലവും ഏല്‍പിക്കപ്പെടുന്ന ഇടവുമുണ്ട്. ഈ തെളിവുകളൊക്കെയും നാം വിവരിച്ചുതരുന്നത് കാര്യം മനസ്സിലാക്കുന്ന ജനത്തിനുവേണ്ടിയാണ്.
\end{malayalam}}
\flushright{\begin{Arabic}
\quranayah[6][99]
\end{Arabic}}
\flushleft{\begin{malayalam}
അവന്‍ തന്നെയാണ് മാനത്തുനിന്ന് വെള്ളം വീഴ്ത്തുന്നത്. അങ്ങനെ അതുവഴി നാം സകല വസ്തുക്കളുടെയും മുളകള്‍ കിളിര്‍പ്പിച്ചു. പിന്നെ നാം അവയില്‍ നിന്ന് പച്ചപ്പുള്ള ചെടികള്‍ വളര്‍ത്തി. അവയില്‍ നിന്ന് ഇടതൂര്‍ന്ന ധാന്യക്കതിരുകളും. നാം ഈന്തപ്പനയുടെ കൂമ്പോളകളില്‍ തൂങ്ങിക്കിടക്കുന്ന കുലകള്‍ ഉല്‍പാദിപ്പിച്ചു. മുന്തിരിത്തോട്ടങ്ങളും ഒലീവും റുമ്മാനും ഉണ്ടാക്കി. ഒരു പോലെയുള്ളതും എന്നാല്‍ വ്യത്യസ്തങ്ങളുമായവ. അവ കായ്ക്കുമ്പോള്‍ അവയില്‍ കനികളുണ്ടാകുന്നതും അവ പാകമാകുന്നതും നന്നായി നിരീക്ഷിക്കുക. വിശ്വസിക്കുന്ന ജനത്തിന് ഇതിലെല്ലാം തെളിവുകളുണ്ട്.
\end{malayalam}}
\flushright{\begin{Arabic}
\quranayah[6][100]
\end{Arabic}}
\flushleft{\begin{malayalam}
എന്നിട്ടും അവര്‍ ജിന്നുകളെ അല്ലാഹുവിന്റെ പങ്കാളികളാക്കുന്നു. എന്നാല്‍ അവനാണ് ജിന്നുകളെ സൃഷ്ടിച്ചത്. ഒരു വിവരവുമില്ലാതെ അവരവന് പുത്രന്മാരെയും പുത്രിമാരെയും സങ്കല്‍പിക്കുന്നു. അവനാകട്ടെ അവരുടെ വിവരണങ്ങള്‍ക്കെല്ലാം അതീതനും പരിശുദ്ധനുമത്രെ.
\end{malayalam}}
\flushright{\begin{Arabic}
\quranayah[6][101]
\end{Arabic}}
\flushleft{\begin{malayalam}
ആകാശഭൂമികളെ മുന്‍മാതൃകകളില്ലാതെ സൃഷ്ടിച്ചവനാണ് അല്ലാഹു. അവന്നെങ്ങനെ മക്കളുണ്ടാകും? അവന്ന് ഇണപോലും ഇല്ലല്ലോ. അവന്‍ സകല വസ്തുക്കളെയും സൃഷ്ടിച്ചു. അവന്‍ എല്ലാ കാര്യങ്ങളും അറിയുന്നവനാണ്.
\end{malayalam}}
\flushright{\begin{Arabic}
\quranayah[6][102]
\end{Arabic}}
\flushleft{\begin{malayalam}
അവനാണ് അല്ലാഹു; നിങ്ങളുടെ നാഥന്‍. അവനല്ലാതെ ദൈവമില്ല. സകല വസ്തുക്കളെയും സൃഷ്ടിച്ചവനാണവന്‍. അതിനാല്‍ നിങ്ങള്‍ അവനുമാത്രം വഴിപ്പെടുക. അവന്‍ എല്ലാ കാര്യങ്ങളുടെയും കൈകാര്യകര്‍ത്താവാണ്.
\end{malayalam}}
\flushright{\begin{Arabic}
\quranayah[6][103]
\end{Arabic}}
\flushleft{\begin{malayalam}
കണ്ണുകള്‍ക്ക് അവനെ കാണാനാവില്ല. എന്നാല്‍ അവന്‍ കണ്ണുകളെ കാണുന്നു. അവന്‍ സൂക്ഷ്മജ്ഞനാണ്. എല്ലാം അറിയുന്നവനും.
\end{malayalam}}
\flushright{\begin{Arabic}
\quranayah[6][104]
\end{Arabic}}
\flushleft{\begin{malayalam}
നിങ്ങളുടെ നാഥനില്‍ നിന്ന് നിങ്ങള്‍ക്കിതാ ഉള്‍ക്കാഴ്ചതരുന്ന തെളിവുകള്‍ വന്നെത്തിയിരിക്കുന്നു. ആരെങ്കിലും അത് കണ്ടറിയുന്നുവെങ്കില്‍ അതിന്റെ ഗുണം അവന്നുതന്നെയാണ്. ആരെങ്കിലും അന്ധത നടിച്ചാല്‍ അതിന്റെ ദോഷവും അവന്നു തന്നെ. ഞാന്‍ നിങ്ങളുടെ സംരക്ഷണച്ചുമതല ഏറ്റെടുത്തവനൊന്നുമല്ല.
\end{malayalam}}
\flushright{\begin{Arabic}
\quranayah[6][105]
\end{Arabic}}
\flushleft{\begin{malayalam}
അവ്വിധം വിവിധ രൂപേണ നാം നമ്മുടെ വചനങ്ങള്‍ വിശദീകരിച്ചുതരുന്നു. നീ ആരില്‍ നിന്നൊക്കെയോ പഠിച്ചുവന്നതാണെന്ന് സത്യനിഷേധികളെക്കൊണ്ട് പറയിക്കാനാണിത്. കാര്യം മനസ്സിലാക്കുന്നവര്‍ക്ക് വസ്തുത വ്യക്തമാക്കിക്കൊടുക്കാനും.
\end{malayalam}}
\flushright{\begin{Arabic}
\quranayah[6][106]
\end{Arabic}}
\flushleft{\begin{malayalam}
നിനക്കു നിന്റെ നാഥനില്‍ നിന്ന് ബോധനമായി ലഭിച്ചത് പിന്‍പറ്റുക. അവനല്ലാതെ ദൈവമില്ല. ഈ ബഹുദൈവവിശ്വാസികളെ അവഗണിക്കുക.
\end{malayalam}}
\flushright{\begin{Arabic}
\quranayah[6][107]
\end{Arabic}}
\flushleft{\begin{malayalam}
അല്ലാഹു ഇച്ഛിച്ചിരുന്നെങ്കില്‍ അവരവന് പങ്കാളികളെ സങ്കല്‍പിക്കുമായിരുന്നില്ല. നിന്നെ നാം അവരുടെ രക്ഷാകര്‍ത്തൃത്വം ഏല്‍പിച്ചിട്ടില്ല. നീ അവരുടെ ചുമതലകള്‍ ഏല്‍പിക്കപ്പെട്ടവനുമല്ല.
\end{malayalam}}
\flushright{\begin{Arabic}
\quranayah[6][108]
\end{Arabic}}
\flushleft{\begin{malayalam}
അല്ലാഹുവെക്കൂടാതെ അവര്‍ വിളിച്ചു പ്രാര്‍ഥിക്കുന്നവയെ നിങ്ങള്‍ ശകാരിക്കരുത്. അങ്ങനെ ചെയ്താല്‍ അവര്‍ തങ്ങളുടെ അറിവില്ലായ്മയാല്‍ അല്ലാഹുവെയും അന്യായമായി ശകാരിക്കും. അവ്വിധം ഓരോ വിഭാഗത്തിനും അവരുടെ ചെയ്തികളെ നാം ചേതോഹരങ്ങളായി തോന്നിപ്പിച്ചിരിക്കുന്നു. പിന്നീട് തങ്ങളുടെ നാഥന്റെ അടുത്തേക്കാണ് അവരുടെ മടക്കം. അപ്പോള്‍, അവര്‍ ചെയ്തുകൊണ്ടിരുന്നതിനെപ്പറ്റിയെല്ലാം അവന്‍ അവരെ വിവരമറിയിക്കും.
\end{malayalam}}
\flushright{\begin{Arabic}
\quranayah[6][109]
\end{Arabic}}
\flushleft{\begin{malayalam}
അവര്‍ അല്ലാഹുവിന്റെ പേരില്‍ സത്യം ചെയ്തു തറപ്പിച്ചു പറയുന്നു, തങ്ങള്‍ക്ക് വല്ല ദൃഷ്ടാന്തവും വന്നെത്തിയാല്‍ അതില്‍ വിശ്വസിക്കുക തന്നെ ചെയ്യുമെന്ന്. പറയുക: "ദൃഷ്ടാന്തങ്ങള്‍ അല്ലാഹുവിന്റെ അധീനതയിലാണ്.” ദൃഷ്ടാന്തങ്ങള്‍ വന്നു കിട്ടിയാലും അവര്‍ വിശ്വസിക്കുകയില്ലെന്ന് നിങ്ങളെ എങ്ങനെ ധരിപ്പിക്കാനാണ്?
\end{malayalam}}
\flushright{\begin{Arabic}
\quranayah[6][110]
\end{Arabic}}
\flushleft{\begin{malayalam}
അവരുടെ ഹൃദയങ്ങളെയും കണ്ണുകളെയും നാം മാറ്റിമറിച്ചുകൊണ്ടിരിക്കുകയാണ്; ആദ്യതവണ അവരിതില്‍ വിശ്വസിക്കാതിരുന്നപോലെത്തന്നെ. തങ്ങളുടെ അതിക്രമങ്ങളില്‍ തന്നെ വിഹരിക്കാന്‍ നാമവരെ വിടുകയും ചെയ്യുന്നു.
\end{malayalam}}
\flushright{\begin{Arabic}
\quranayah[6][111]
\end{Arabic}}
\flushleft{\begin{malayalam}
നാം മലക്കുകളെത്തന്നെ അവരിലേക്കിറക്കിയാലും മരിച്ചവര്‍ അവരോടു സംസാരിച്ചാലും സകല വസ്തുക്കളെയും നാം അവരുടെ മുന്നില്‍ ഒരുമിച്ചുകൂട്ടിയാലും അവര്‍ വിശ്വസിക്കുകയില്ല; ദൈവേച്ഛയുണ്ടെങ്കിലല്ലാതെ. എന്നിട്ടും അവരിലേറെപ്പേരും വിവരക്കേട് പറയുകയാണ്.
\end{malayalam}}
\flushright{\begin{Arabic}
\quranayah[6][112]
\end{Arabic}}
\flushleft{\begin{malayalam}
അവ്വിധം നാം ഓരോ പ്രവാചകന്നും മനുഷ്യരിലും ജിന്നുകളിലുംപെട്ട പിശാചുക്കളെ ശത്രുക്കളാക്കിവെച്ചിട്ടുണ്ട്. അവര്‍ അന്യോന്യം വഞ്ചിക്കുന്ന മോഹനവാക്കുകള്‍ വാരിവിതറുന്നു. നിന്റെ നാഥന്‍ ഇച്ഛിച്ചിരുന്നെങ്കില്‍ അവരങ്ങനെ ചെയ്യുമായിരുന്നില്ല. അതിനാല്‍ നീ അവരെയും അവരുടെ പൊയ്മൊഴികളെയും അവഗണിക്കുക.
\end{malayalam}}
\flushright{\begin{Arabic}
\quranayah[6][113]
\end{Arabic}}
\flushleft{\begin{malayalam}
പരലോകത്തില്‍ വിശ്വസിക്കാത്തവരുടെ മനസ്സുകള്‍ ആ വഞ്ചനയിലേക്ക് ചായാനാണിത്. അവരതില്‍ തൃപ്തരാകാനും. അവര്‍ ചെയ്തുകൂട്ടുന്നതൊക്കെ ഇവരും കാട്ടിക്കൂട്ടാന്‍ വേണ്ടിയും.
\end{malayalam}}
\flushright{\begin{Arabic}
\quranayah[6][114]
\end{Arabic}}
\flushleft{\begin{malayalam}
“കാര്യം ഇതായിരിക്കെ ഞാന്‍ അല്ലാഹു അല്ലാത്ത മറ്റൊരു വിധി കര്‍ത്താവിനെ തേടുകയോ? അവനോ, വിശദവിവരങ്ങളടങ്ങിയ വേദപുസ്തകം നിങ്ങള്‍ക്ക് ഇറക്കിത്തന്നവനാണ്.” നാം നേരത്തെ വേദം നല്‍കിയവര്‍ക്കറിയാം, ഇത് നിന്റെ നാഥനില്‍ നിന്ന് സത്യവുമായി അവതീര്‍ണമായതാണെന്ന്. അതിനാല്‍ നീ ഒരിക്കലും സംശയാലുക്കളില്‍ പെട്ടുപോകരുത്.
\end{malayalam}}
\flushright{\begin{Arabic}
\quranayah[6][115]
\end{Arabic}}
\flushleft{\begin{malayalam}
നിന്റെ നാഥന്റെ വചനം സത്യത്താലും നീതിയാലും സമഗ്രമായിരിക്കുന്നു. അവന്റെ വചനങ്ങളില്‍ ഭേദഗതി വരുത്തുന്ന ആരുമില്ല. അവന്‍ എല്ലാം കേള്‍ക്കുന്നവനും അറിയുന്നവനുമാണ്.
\end{malayalam}}
\flushright{\begin{Arabic}
\quranayah[6][116]
\end{Arabic}}
\flushleft{\begin{malayalam}
ഭൂമുഖത്തുള്ള ഭൂരിപക്ഷംപേരും പറയുന്നത് നീ അനുസരിക്കുകയാണെങ്കില്‍ അവര്‍ നിന്നെ അല്ലാഹുവിന്റെ മാര്‍ഗത്തില്‍ നിന്ന് തെറ്റിച്ചുകളയും. കേവലം ഊഹങ്ങളെ മാത്രമാണ് അവര്‍ പിന്‍പറ്റുന്നത്. അവര്‍ അനുമാനങ്ങളില്‍ ആടിയുലയുകയാണ്.
\end{malayalam}}
\flushright{\begin{Arabic}
\quranayah[6][117]
\end{Arabic}}
\flushleft{\begin{malayalam}
തന്റെ വഴിയില്‍ നിന്ന് തെറ്റിപ്പോകുന്നവര്‍ ആരൊക്കെയെന്ന് നിന്റെ നാഥന് നന്നായറിയാം. നേര്‍വഴി പ്രാപിച്ചവരെപ്പറ്റി സൂക്ഷ്മമായറിയുന്നവനും അവന്‍ തന്നെ.
\end{malayalam}}
\flushright{\begin{Arabic}
\quranayah[6][118]
\end{Arabic}}
\flushleft{\begin{malayalam}
അതിനാല്‍ നിങ്ങള്‍ അല്ലാഹുവിന്റെ നാമത്തില്‍ അറുത്തവയില്‍ നിന്നും തിന്നുക. നിങ്ങള്‍ അവന്റെ വചനങ്ങളില്‍ വിശ്വസിക്കുന്നവരെങ്കില്‍!
\end{malayalam}}
\flushright{\begin{Arabic}
\quranayah[6][119]
\end{Arabic}}
\flushleft{\begin{malayalam}
ദൈവനാമത്തില്‍ അറുത്തതില്‍ നിന്ന് നിങ്ങളെന്തിനു തിന്നാതിരിക്കണം? നിങ്ങള്‍ക്കു നിഷിദ്ധമാക്കിയത് ഏതൊക്കെയെന്ന് അല്ലാഹു വിവരിച്ചുതന്നിട്ടുണ്ടല്ലോ. നിങ്ങളവ തിന്നാന്‍ നിര്‍ബന്ധിതമാകുമ്പോളൊഴികെ. പലരും ഒരു വിവരവുമില്ലാതെ തോന്നിയപോലെ ആളുകളെ വഴിപിഴപ്പിച്ചുകൊണ്ടിരിക്കുകയാണ്. സംശയമില്ല; നിന്റെ നാഥന്‍ അതിക്രമികളെപ്പറ്റി നന്നായറിയുന്നവനാണ്.
\end{malayalam}}
\flushright{\begin{Arabic}
\quranayah[6][120]
\end{Arabic}}
\flushleft{\begin{malayalam}
പരസ്യവും രഹസ്യവുമായ പാപങ്ങള്‍ വര്‍ജിക്കുക. കുറ്റം സമ്പാദിച്ചുവെക്കുന്നവര്‍ക്ക് അവര്‍ പ്രവര്‍ത്തിക്കുന്നതിനനുസരിച്ച ശിക്ഷ ലഭിക്കും.
\end{malayalam}}
\flushright{\begin{Arabic}
\quranayah[6][121]
\end{Arabic}}
\flushleft{\begin{malayalam}
അല്ലാഹുവിന്റെ നാമത്തില്‍ അറുക്കാത്ത മൃഗങ്ങളുടെ മാംസം നിങ്ങള്‍ തിന്നരുത്. അതു അധര്‍മമാണ്; തീര്‍ച്ച. നിങ്ങളോട് തര്‍ക്കിക്കാനായി പിശാചുക്കള്‍ തങ്ങളുടെ കൂട്ടാളികള്‍ക്ക് ചില ദുര്‍ബോധനങ്ങള്‍ നല്‍കിക്കൊണ്ടിരിക്കും. നിങ്ങള്‍ അവരെ അനുസരിക്കുകയാണെങ്കില്‍ തീര്‍ച്ചയായും നിങ്ങളും ദൈവത്തില്‍ പങ്കുചേര്‍ത്തവരായിത്തീരും.
\end{malayalam}}
\flushright{\begin{Arabic}
\quranayah[6][122]
\end{Arabic}}
\flushleft{\begin{malayalam}
ഒരുവനു നാം ജീവനില്ലാത്ത അവസ്ഥയില്‍ ജീവന്‍ നല്‍കി. വെളിച്ചമേകുകയും ചെയ്തു. അതുമായി ജനങ്ങള്‍ക്കിടയിലൂടെ നടന്നുകൊണ്ടിരിക്കുന്ന അയാള്‍, പുറത്തു കടക്കാനാവാതെ കൂരിരുട്ടില്‍പെട്ടവനെപ്പോലെയാണോ? അവ്വിധം സത്യനിഷേധികള്‍ക്ക് തങ്ങളുടെ ചെയ്തികള്‍ ചേതോഹരമായിത്തോന്നി.
\end{malayalam}}
\flushright{\begin{Arabic}
\quranayah[6][123]
\end{Arabic}}
\flushleft{\begin{malayalam}
അപ്രകാരം തന്നെ എല്ലാ നാട്ടിലും കുതന്ത്രങ്ങള്‍ കുത്തിപ്പൊക്കാന്‍ അവിടങ്ങളിലെ തെമ്മാടികളുടെ തലവന്മാരെ നാം ചുമതലപ്പെടുത്തിയിട്ടുണ്ട്. യഥാര്‍ഥത്തില്‍ അവര്‍ കുതന്ത്രം പ്രയോഗിക്കുന്നത് തങ്ങള്‍ക്കെതിരെ തന്നെയാണ്. എന്നാല്‍ അതേക്കുറിച്ച് അവരൊട്ടും ബോധവാന്മാരല്ല.
\end{malayalam}}
\flushright{\begin{Arabic}
\quranayah[6][124]
\end{Arabic}}
\flushleft{\begin{malayalam}
അവര്‍ക്ക് വല്ല പ്രമാണവും വന്നെത്തിയാല്‍ അവര്‍ പറയും: "ദൈവദൂതന്മാര്‍ക്ക് കിട്ടിയതുപോലുള്ളത് ഞങ്ങള്‍ക്കും ലഭിക്കുംവരെ ഞങ്ങള്‍ വിശ്വസിക്കുകയില്ല.” എന്നാല്‍ അല്ലാഹുവിന് നന്നായറിയാം; തന്റെ സന്ദേശം എവിടെ ഏല്‍പിക്കണമെന്ന്. അധര്‍മം പ്രവര്‍ത്തിക്കുന്നവര്‍ക്ക് അല്ലാഹുവിങ്കല്‍ നിന്ദ്യതയാണുണ്ടാവുക. കഠിനശിക്ഷയും. അവര്‍ കാട്ടിക്കൂട്ടിയ കുതന്ത്രങ്ങള്‍ കാരണമാണത്.
\end{malayalam}}
\flushright{\begin{Arabic}
\quranayah[6][125]
\end{Arabic}}
\flushleft{\begin{malayalam}
അല്ലാഹു ആരെയെങ്കിലും നേര്‍വഴിയിലാക്കാന്‍ ഉദ്ദേശിക്കുന്നുവെങ്കില്‍ അയാളുടെ മനസ്സിനെ അവന്‍ ഇസ്ലാമിനായി തുറന്നുകൊടുക്കുന്നു. ആരെയെങ്കിലും ദുര്‍മാര്‍ഗത്തിലാക്കാനാണ് അവനുദ്ദേശിക്കുന്നതെങ്കില്‍ അയാളുടെ ഹൃദയത്തെ ഇടുങ്ങിയതും സങ്കുചിതവുമാക്കുന്നു. അപ്പോള്‍ താന്‍ ആകാശത്തേക്ക് കയറിപ്പോകുംപോലെ അവനു തോന്നുന്നു. വിശ്വസിക്കാത്തവര്‍ക്ക് അല്ലാഹു ഇവ്വിധം നീചമായ ശിക്ഷ നല്‍കും.
\end{malayalam}}
\flushright{\begin{Arabic}
\quranayah[6][126]
\end{Arabic}}
\flushleft{\begin{malayalam}
ഇതാണ് നിന്റെ നാഥന്റെ നേര്‍വഴി. ആലോചിച്ചറിയുന്ന ജനത്തിന് നാമിതാ തെളിവുകള്‍ വിശദീകരിച്ചിരിക്കുന്നു.
\end{malayalam}}
\flushright{\begin{Arabic}
\quranayah[6][127]
\end{Arabic}}
\flushleft{\begin{malayalam}
അവര്‍ക്ക് അവരുടെ നാഥന്റെ അടുത്ത് ശാന്തിമന്ദിരമുണ്ട്. അവനാണ് അവരുടെ രക്ഷാധികാരി. അവര്‍ പ്രവര്‍ത്തിച്ചതിനുള്ള പ്രതിഫലമാണത്.
\end{malayalam}}
\flushright{\begin{Arabic}
\quranayah[6][128]
\end{Arabic}}
\flushleft{\begin{malayalam}
അല്ലാഹു അവരെയെല്ലാം ഒരുമിച്ചു ചേര്‍ക്കുംദിനം അവന്‍ പറയും: "ജിന്ന്സമൂഹമേ; മനുഷ്യരില്‍ വളരെ പേരെ നിങ്ങള്‍ വഴിപിഴപ്പിച്ചിട്ടുണ്ട്.” അപ്പോള്‍ അവരുടെ ആത്മമിത്രങ്ങളായിരുന്ന മനുഷ്യര്‍ പറയും: "ഞങ്ങളുടെ നാഥാ! ഞങ്ങള്‍ പരസ്പരം സുഖാസ്വാദനങ്ങള്‍ക്ക് ഉപയോഗപ്പെടുത്തിയിട്ടുണ്ട്. ഇപ്പോള്‍ നീ ഞങ്ങള്‍ക്ക് അനുവദിച്ച അവധിയില്‍ ഞങ്ങളെത്തിയിരിക്കുന്നു”. അല്ലാഹു അറിയിക്കും: ശരി, ഇനി നരകത്തീയാണ് നിങ്ങളുടെ താമസസ്ഥലം. നിങ്ങളവിടെ സ്ഥിരവാസികളായിരിക്കും. അല്ലാഹു ഇച്ഛിച്ച സമയമൊഴികെ. നിന്റെ നാഥന്‍ യുക്തിമാനും എല്ലാം അറിയുന്നവനും തന്നെ; തീര്‍ച്ച.
\end{malayalam}}
\flushright{\begin{Arabic}
\quranayah[6][129]
\end{Arabic}}
\flushleft{\begin{malayalam}
ഇവ്വിധം ആ അക്രമികളെ നാം അന്യോന്യം കൂട്ടാളികളാക്കും. അവര്‍ സമ്പാദിച്ചുകൊണ്ടിരുന്നതിന്റെ ഫലമാണത്.
\end{malayalam}}
\flushright{\begin{Arabic}
\quranayah[6][130]
\end{Arabic}}
\flushleft{\begin{malayalam}
"ജിന്നുകളുടെയും മനുഷ്യരുടെയും സമൂഹമേ, എന്റെ പ്രമാണങ്ങള്‍ വിവരിച്ചുതരികയും ഈ ദിനത്തെ നേരിടേണ്ടിവരുമെന്ന് മുന്നറിയിപ്പ് നല്‍കുകയും ചെയ്യുന്ന, നിങ്ങളില്‍ നിന്നുതന്നെയുള്ള ദൈവദൂതന്മാര്‍ നിങ്ങളുടെ അടുത്ത് വന്നിരുന്നില്ലേ?” അവര്‍ പറയും: "അതെ; ഞങ്ങളിതാ ഞങ്ങള്‍ക്കെതിരെ തന്നെ സാക്ഷ്യം വഹിക്കുന്നു.” ഐഹികജീവിതം അവരെ വഞ്ചനയിലകപ്പെടുത്തി. തങ്ങള്‍ സത്യനിഷേധികളായിരുന്നുവെന്ന് അന്നേരം അവര്‍ തങ്ങള്‍ക്കെതിരെ തന്നെ സാക്ഷ്യം വഹിക്കുന്നു.
\end{malayalam}}
\flushright{\begin{Arabic}
\quranayah[6][131]
\end{Arabic}}
\flushleft{\begin{malayalam}
ഒരു പ്രദേശത്തുകാര്‍ സന്മാര്‍ഗത്തെപ്പറ്റി ഒന്നുമറിയാതെ കഴിയുമ്പോള്‍ നിന്റെ നാഥന്‍ അന്യായമായി അവരെ നശിപ്പിക്കുകയില്ലെന്നതിന് തെളിവാണ് ഇവരുടെ ഈ സാക്ഷ്യം.
\end{malayalam}}
\flushright{\begin{Arabic}
\quranayah[6][132]
\end{Arabic}}
\flushleft{\begin{malayalam}
ഓരോരുത്തര്‍ക്കും തങ്ങള്‍ പ്രവര്‍ത്തിച്ചതിനനുസരിച്ച പദവിയുണ്ട്. അവര്‍ ചെയ്തുകൊണ്ടിരിക്കുന്നതിനെക്കുറിച്ച് നിന്റെ നാഥന്‍ ഒട്ടും അശ്രദ്ധനല്ല.
\end{malayalam}}
\flushright{\begin{Arabic}
\quranayah[6][133]
\end{Arabic}}
\flushleft{\begin{malayalam}
നിന്റെ നാഥന്‍ സ്വയംപര്യാപ്തനാണ്. ഏറെ ദയാപരനും. അവനിച്ഛിക്കുന്നുവെങ്കില്‍ നിങ്ങളെ നീക്കംചെയ്യുകയും നിങ്ങള്‍ക്കുശേഷം താനിച്ഛിക്കുന്നവരെ പകരം കൊണ്ടുവരികയും ചെയ്യും. മറ്റൊരു ജനതയുടെ വംശപരമ്പരയില്‍നിന്ന് നിങ്ങളെ അവന്‍ ഉയര്‍ത്തിക്കൊണ്ടുവന്നപോലെ.
\end{malayalam}}
\flushright{\begin{Arabic}
\quranayah[6][134]
\end{Arabic}}
\flushleft{\begin{malayalam}
നിങ്ങളോട് വാഗ്ദാനം ചെയ്യുന്നത് സംഭവിക്കുക തന്നെ ചെയ്യും; തീര്‍ച്ച. അല്ലാഹുവിനെ തോല്‍പിക്കാന്‍ നിങ്ങള്‍ക്കാവില്ല.
\end{malayalam}}
\flushright{\begin{Arabic}
\quranayah[6][135]
\end{Arabic}}
\flushleft{\begin{malayalam}
പറയുക: എന്റെ ജനമേ, നിങ്ങള്‍ നിങ്ങളുടെ നിലപാടനുസരിച്ച് പ്രവര്‍ത്തിച്ചുകൊള്ളുക; ഞാനും പ്രവര്‍ത്തിക്കാം. ഈ ലോകത്തിന്റെ ഒടുക്കം ആര്‍ക്കനുകൂലമായിരിക്കുമെന്ന് വഴിയെ നിങ്ങളറിയുക തന്നെ ചെയ്യും. ഒന്നു തീര്‍ച്ച; അക്രമികള്‍ വിജയിക്കുകയില്ല.
\end{malayalam}}
\flushright{\begin{Arabic}
\quranayah[6][136]
\end{Arabic}}
\flushleft{\begin{malayalam}
അല്ലാഹുതന്നെ സൃഷ്ടിച്ചുണ്ടാക്കിയ വിളകളില്‍നിന്നും കാലികളില്‍നിന്നും ഒരു വിഹിതം അവരവന് നിശ്ചയിച്ചുകൊടുത്തിരിക്കുന്നു. എന്നിട്ടവര്‍ കെട്ടിച്ചമച്ച് പറയുന്നു: "ഇത് അല്ലാഹുവിനുള്ളതാണ്. ഇത് തങ്ങള്‍ പങ്കാളികളാക്കിവെച്ച ദൈവങ്ങള്‍ക്കും.” അതോടൊപ്പം അവരുടെ പങ്കാളികള്‍ക്കുള്ളതൊന്നും അല്ലാഹുവിലേക്കെത്തിച്ചേരുകയില്ല. അല്ലാഹുവിനുള്ളതോ അവരുടെ പങ്കാളികള്‍ക്കെത്തിച്ചേരുകയും ചെയ്യും. അവരുടെ തീരുമാനം എത്ര ചീത്ത!
\end{malayalam}}
\flushright{\begin{Arabic}
\quranayah[6][137]
\end{Arabic}}
\flushleft{\begin{malayalam}
അതുപോലെത്തന്നെ ധാരാളം ബഹുദൈവവിശ്വാസികള്‍ക്ക് തങ്ങളുടെ മക്കളെ കൊല്ലുന്നത് അവരുടെ പങ്കാളികള്‍ ഭൂഷണമായി തോന്നിപ്പിച്ചിരിക്കുന്നു. അവരെ വിപത്തില്‍പെടുത്തലും അവര്‍ക്ക് തങ്ങളുടെ മതം തിരിച്ചറിയാതാകലുമാണ് അതുകൊണ്ടുണ്ടാവുന്നത്. അല്ലാഹു ഇച്ഛിച്ചിരുന്നെങ്കില്‍ അവരങ്ങനെ ചെയ്യുമായിരുന്നില്ല. അവരെയും അവര്‍ കെട്ടിച്ചമച്ചുണ്ടാക്കുന്നവയെയും അവരുടെ പാട്ടിന് വിട്ടേക്കുക.
\end{malayalam}}
\flushright{\begin{Arabic}
\quranayah[6][138]
\end{Arabic}}
\flushleft{\begin{malayalam}
അവര്‍ പറഞ്ഞു: "ഇവ വിലക്കപ്പെട്ട കാലികളും വിളകളുമാകുന്നു. ഞങ്ങളുദ്ദേശിക്കുന്നവരല്ലാതെ, അവ തിന്നാന്‍ പാടില്ല.” അവര്‍ സ്വയം കെട്ടിച്ചമച്ച വാദമാണിത്. അവര്‍ സവാരി ചെയ്യാനും ചരക്കു ചുമക്കാനും പുറം ഉപയോഗിക്കുന്നത് നിഷിദ്ധമാക്കിയ മറ്റു മൃഗങ്ങളുണ്ട്. അവര്‍ അല്ലാഹുവിന്റെ നാമം ഉച്ചരിക്കാത്ത മൃഗങ്ങളുമുണ്ട്. ഇതൊക്കെയും അവര്‍ അല്ലാഹുവിന്റെ പേരില്‍ കെട്ടിച്ചമച്ചുണ്ടാക്കിയവയാണ്. അവര്‍ ഇവ്വിധം കെട്ടിച്ചമച്ചുണ്ടാക്കുന്നതിന് അല്ലാഹു അവര്‍ക്ക് വൈകാതെ മതിയായ പ്രതിഫലം നല്‍കും.
\end{malayalam}}
\flushright{\begin{Arabic}
\quranayah[6][139]
\end{Arabic}}
\flushleft{\begin{malayalam}
അവര്‍ പറയുന്നു: "ഈ കാലികളുടെ വയറുകളിലുള്ളത് ഞങ്ങളിലെ ആണുങ്ങള്‍ക്ക് മാത്രമുള്ളതാണ്. ഞങ്ങളുടെ ഭാര്യമാര്‍ക്ക് അത് നിഷിദ്ധമാണ്.” എന്നാല്‍ അത് ശവമാണെങ്കില്‍ അവരെല്ലാം അതില്‍ പങ്കാളികളാകും. തീര്‍ച്ചയായും അവരുടെ ഈ കെട്ടിച്ചമക്കലുകള്‍ക്ക് അല്ലാഹു അനുയോജ്യമായ പ്രതിഫലം വൈകാതെ നല്‍കും. സംശയമില്ല; അവന്‍ യുക്തിമാനും എല്ലാം അറിയുന്നവനുമാണ്.
\end{malayalam}}
\flushright{\begin{Arabic}
\quranayah[6][140]
\end{Arabic}}
\flushleft{\begin{malayalam}
ഒരു വിവരവുമില്ലാതെ, തികഞ്ഞ അവിവേകം കാരണം സ്വന്തം മക്കളെ കൊല്ലുന്നവരും അല്ലാഹു അവര്‍ക്കേകിയ അന്നം അല്ലാഹുവിന്റെ പേരില്‍ കള്ളം കെട്ടിച്ചമച്ച് സ്വയം നിഷിദ്ധമാക്കുന്നവരും നഷ്ടത്തില്‍പ്പെട്ടതുതന്നെ. സംശയമില്ല അവര്‍ വഴികേടിലായിരിക്കുന്നു. അവര്‍ നേര്‍വഴി പ്രാപിച്ചതുമില്ല.
\end{malayalam}}
\flushright{\begin{Arabic}
\quranayah[6][141]
\end{Arabic}}
\flushleft{\begin{malayalam}
പന്തലില്‍ പടര്‍ത്തുന്നതും അല്ലാത്തതുമായ ഉദ്യാനങ്ങള്‍; ഈത്തപ്പനകള്‍; പലതരം കായ്കനികളുള്ള കൃഷികള്‍; പരസ്പരം സമാനത തോന്നുന്നതും എന്നാല്‍ വ്യത്യസ്തങ്ങളുമായ ഒലീവും റുമ്മാനും എല്ലാം സൃഷ്ടിച്ചുണ്ടാക്കിയത് അല്ലാഹുവാണ്. അവ കായ്ക്കുമ്പോള്‍ പഴങ്ങള്‍ തിന്നുകൊള്ളുക. വിളവെടുപ്പുകാലത്ത് അതിന്റെ ബാധ്യത അഥവാ സകാത്ത് കൊടുത്തുതീര്‍ക്കുക. എന്നാല്‍ അമിതവ്യയം അരുത്. അതിരുകവിയുന്നവരെ അല്ലാഹു ഇഷ്ടപ്പെടുന്നില്ല.
\end{malayalam}}
\flushright{\begin{Arabic}
\quranayah[6][142]
\end{Arabic}}
\flushleft{\begin{malayalam}
കന്നുകാലികളില്‍ ഭാരം ചുമക്കുന്നവയെയും അറുത്തുതിന്നാനുള്ളവയെയും അവന്‍ സൃഷ്ടിച്ചിരിക്കുന്നു. അല്ലാഹു നിങ്ങള്‍ക്കേകിയ വിഭവങ്ങളില്‍നിന്ന് ആഹരിച്ചുകൊള്ളുക. പിശാചിന്റെ കാല്‍പ്പാടുകള്‍ പിന്‍പറ്റരുത്. സംശയംവേണ്ട; അവന്‍ നിങ്ങളുടെ പ്രത്യക്ഷ ശത്രുവാണ്.
\end{malayalam}}
\flushright{\begin{Arabic}
\quranayah[6][143]
\end{Arabic}}
\flushleft{\begin{malayalam}
അല്ലാഹു എട്ടു ഇണകളെ സൃഷ്ടിച്ചു. ചെമ്മരിയാടു വര്‍ഗത്തില്‍ നിന്ന് രണ്ടും കോലാടു വര്‍ഗത്തില്‍ നിന്ന് രണ്ടും. ചോദിക്കുക: അല്ലാഹു അവയില്‍ ആണ്‍വര്‍ഗത്തെയാണോ നിഷിദ്ധമാക്കിയത്; അതോ പെണ്‍വര്‍ഗത്തെയോ? അതുമല്ലെങ്കില്‍ ഇരുതരം പെണ്ണാടുകളുടെയും ഗര്‍ഭാശയങ്ങളിലുള്ള കുട്ടികളെയോ? അറിവിന്റെ അടിസ്ഥാനത്തില്‍ എനിക്കു പറഞ്ഞുതരിക; നിങ്ങള്‍ സത്യസന്ധരെങ്കില്‍.
\end{malayalam}}
\flushright{\begin{Arabic}
\quranayah[6][144]
\end{Arabic}}
\flushleft{\begin{malayalam}
ഇവ്വിധം ഒട്ടകവര്‍ഗത്തില്‍ നിന്ന് രണ്ട് ഇണകളും പശുവര്‍ഗത്തില്‍ നിന്ന് രണ്ട് ഇണകളും ഇതാ. ചോദിക്കുക: അല്ലാഹു ഇരുവിഭാഗത്തിലെയും ആണ്‍വര്‍ഗത്തെയാണോ നിഷിദ്ധമാക്കിയത്; അതോ പെണ്‍വര്‍ഗത്തെയോ? അതുമല്ലെങ്കില്‍ ഇരുതരം പെണ്‍വര്‍ഗങ്ങളുടെയും ഗര്‍ഭാശയങ്ങളിലുള്ള കുട്ടികളെയോ? അതല്ല; അല്ലാഹു ഇതൊക്കെയും നിങ്ങളെ ഉപദേശിക്കുമ്പോള്‍ നിങ്ങളതിന് സാക്ഷികളായി ഉണ്ടായിരുന്നോ? ഒരു വിവരവുമില്ലാതെ ജനങ്ങളെ വഴിപിഴപ്പിക്കാന്‍ അല്ലാഹുവിന്റെ പേരില്‍ കള്ളം കെട്ടിച്ചമച്ചവനെക്കാള്‍ കൊടിയ അതിക്രമി ആരുണ്ട്? അതിക്രമികളെ അല്ലാഹു നേര്‍വഴിയിലാക്കുകയില്ല; തീര്‍ച്ച.
\end{malayalam}}
\flushright{\begin{Arabic}
\quranayah[6][145]
\end{Arabic}}
\flushleft{\begin{malayalam}
പറയുക: എനിക്കു ബോധനമായി ലഭിച്ചവയില്‍, ഭക്ഷിക്കുന്നവന് തിന്നാന്‍ പാടില്ലാത്തതായി ഒന്നും ഞാന്‍ കാണുന്നില്ല; ശവവും ഒഴുക്കപ്പെട്ട രക്തവും പന്നിമാംസവും ഒഴികെ. അവയൊക്കെ മ്ളേച്ഛ വസ്തുക്കളാണ്. അല്ലാഹു അല്ലാത്തവരുടെ പേരില്‍ അറുക്കപ്പെട്ട് അധാര്‍മികമായതും വിലക്കപ്പെട്ടതു തന്നെ. അഥവാ, ആരെങ്കിലും നിര്‍ബന്ധിതമായും ധിക്കാരം ഉദ്ദേശിക്കാതെയും അത്യാവശ്യ പരിധി ലംഘിക്കാതെയുമാണെങ്കില്‍ വിലക്കപ്പെട്ടവ തിന്നുന്നതിനു വിരോധമില്ല. നിന്റെ നാഥന്‍ ഏറെ പൊറുക്കുന്നവനും ദയാപരനും തന്നെ; തീര്‍ച്ച.
\end{malayalam}}
\flushright{\begin{Arabic}
\quranayah[6][146]
\end{Arabic}}
\flushleft{\begin{malayalam}
നഖമുള്ളവയെയെല്ലാം ജൂതന്മാര്‍ക്കു നാം നിഷിദ്ധമാക്കി. ആടുമാടുകളുടെ കൊഴുപ്പും നാമവര്‍ക്ക് വിലക്കിയിരുന്നു; അവയുടെ മുതുകിലും കുടലിലും പറ്റിപ്പിടിച്ചതോ എല്ലുമായി ഒട്ടിച്ചേര്‍ന്നതോ ഒഴികെ. അവരുടെ ധിക്കാരത്തിന് നാമവര്‍ക്കു നല്‍കിയ ശിക്ഷയാണത്. നാം പറയുന്നത് സത്യം തന്നെ; സംശയമില്ല.
\end{malayalam}}
\flushright{\begin{Arabic}
\quranayah[6][147]
\end{Arabic}}
\flushleft{\begin{malayalam}
അഥവാ അവര്‍ നിന്നെ തള്ളിപ്പറയുകയാണെങ്കില്‍ നീ അവരോടു പറയുക: നിങ്ങളുടെ നാഥന്‍ അതിരുകളില്ലാത്ത കാരുണ്യത്തിനുടമയാകുന്നു. എന്നാല്‍ കുറ്റവാളികളായ ജനത്തില്‍നിന്ന് അവന്റെ ശിക്ഷ തട്ടിമാറ്റപ്പെടുന്നതുമല്ല.
\end{malayalam}}
\flushright{\begin{Arabic}
\quranayah[6][148]
\end{Arabic}}
\flushleft{\begin{malayalam}
ആ ബഹുദൈവ വാദികള്‍ പറയും: "അല്ലാഹു ഇച്ഛിച്ചിരുന്നെങ്കില്‍ ഞങ്ങളോ ഞങ്ങളുടെ പൂര്‍വപിതാക്കളോ ബഹുദൈവവിശ്വാസികളാകുമായിരുന്നില്ല. ഞങ്ങള്‍ ഒന്നും നിഷിദ്ധമാക്കുമായിരുന്നുമില്ല.” അപ്രകാരം തന്നെ അവര്‍ക്ക് മുമ്പുള്ളവരും നമ്മുടെ ശിക്ഷ അനുഭവിക്കുവോളം സത്യത്തെ തള്ളിപ്പറഞ്ഞു. പറയുക: "നിങ്ങളുടെ വശം ഞങ്ങള്‍ക്ക് വെളിപ്പെടുത്തിത്തരാവുന്ന വല്ല വിവരവുമുണ്ടോ? ഊഹത്തെ മാത്രമാണ് നിങ്ങള്‍ പിന്‍പറ്റുന്നത്. നിങ്ങള്‍ കേവലം അനുമാനങ്ങളാവിഷ്കരിക്കുകയാണ്.”
\end{malayalam}}
\flushright{\begin{Arabic}
\quranayah[6][149]
\end{Arabic}}
\flushleft{\begin{malayalam}
പറയുക: തികവുറ്റ തെളിവുള്ളത് അല്ലാഹുവിനാണ്. അവനിച്ഛിച്ചിരുന്നെങ്കില്‍ നിങ്ങളെയെല്ലാം അവന്‍ നേര്‍വഴിയിലാക്കുമായിരുന്നു.
\end{malayalam}}
\flushright{\begin{Arabic}
\quranayah[6][150]
\end{Arabic}}
\flushleft{\begin{malayalam}
പറയുക: അല്ലാഹു ഈ വസ്തുക്കളൊക്കെ വിലക്കിയിരിക്കുന്നുവെന്ന് നിങ്ങള്‍ക്കായി സാക്ഷ്യം വഹിക്കുന്നവരെയെല്ലാം ഇങ്ങ് കൊണ്ടുവരിക. അഥവാ, അവരങ്ങനെ സാക്ഷ്യം വഹിക്കുകയാണെങ്കില്‍ നീ അവരോടൊപ്പം സാക്ഷിയാവരുത്. നമ്മുടെ തെളിവുകളെ കള്ളമാക്കി തള്ളിയവരുടെയും പരലോകത്തില്‍ വിശ്വസിക്കാത്തവരുടെയും തങ്ങളുടെ നാഥന്ന് തുല്യരെ സങ്കല്‍പിച്ചവരുടെയും തന്നിഷ്ടങ്ങളെ പിന്‍പറ്റരുത്.
\end{malayalam}}
\flushright{\begin{Arabic}
\quranayah[6][151]
\end{Arabic}}
\flushleft{\begin{malayalam}
പറയുക: വരുവിന്‍; നിങ്ങളുടെ നാഥന്‍ നിങ്ങള്‍ക്ക് നിഷിദ്ധമാക്കിയതെന്തൊക്കെയെന്ന് ഞാന്‍ പറഞ്ഞുതരാം: നിങ്ങള്‍ ഒന്നിനെയും അവനില്‍ പങ്കാളികളാക്കരുത്; മാതാപിതാക്കളോട് നല്ല നിലയില്‍ വര്‍ത്തിക്കണം; ദാരിദ്യ്രം കാരണം നിങ്ങള്‍ നിങ്ങളുടെ കുട്ടികളെ കൊല്ലരുത്; നിങ്ങള്‍ക്കും അവര്‍ക്കും അന്നം തരുന്നത് നാമാണ്. തെളിഞ്ഞതും മറഞ്ഞതുമായ നീചവൃത്തികളോടടുക്കരുത്; അല്ലാഹു ആദരണീയമാക്കിയ ജീവനെ അന്യായമായി ഹനിക്കരുത്. നിങ്ങള്‍ ചിന്തിച്ചറിയാന്‍ അല്ലാഹു നിങ്ങള്‍ക്കു നല്‍കിയ നിര്‍ദേശങ്ങളാണിവയെല്ലാം.
\end{malayalam}}
\flushright{\begin{Arabic}
\quranayah[6][152]
\end{Arabic}}
\flushleft{\begin{malayalam}
ഏറ്റം ഉത്തമമായ രീതിയിലല്ലാതെ നിങ്ങള്‍ അനാഥയുടെ ധനത്തോടടുക്കരുത്; അവനു കാര്യബോധമുണ്ടാകുംവരെ. അളവു- തൂക്കങ്ങളില്‍ നീതിപൂര്‍വം തികവു വരുത്തുക. നാം ആര്‍ക്കും അയാളുടെ കഴിവിന്നതീതമായ ബാധ്യത ചുമത്തുന്നില്ല. നിങ്ങള്‍ സംസാരിക്കുകയാണെങ്കില്‍ നീതിപാലിക്കുക; അത് അടുത്ത കുടുംബക്കാരന്റെ കാര്യത്തിലായാലും. അല്ലാഹുവോടുള്ള കരാര്‍ പൂര്‍ത്തീകരിക്കുക. നിങ്ങള്‍ കാര്യബോധമുള്ളവരാകാന്‍ അല്ലാഹു നിങ്ങള്‍ക്കു നല്‍കുന്ന ഉപദേശമാണിത്.
\end{malayalam}}
\flushright{\begin{Arabic}
\quranayah[6][153]
\end{Arabic}}
\flushleft{\begin{malayalam}
സംശയം വേണ്ട; ഇതു തന്നെയാണ് എന്റെ നേര്‍വഴി. അതിനാല്‍ നിങ്ങളിത് പിന്തുടരുക. മറ്റു മാര്‍ഗങ്ങള്‍ അവലംബിക്കരുത്. അവയൊക്കെ അല്ലാഹുവിന്റെ വഴിയില്‍നിന്ന് നിങ്ങളെ തെറ്റിച്ചുകളയും. നിങ്ങള്‍ ഭക്തിയുള്ളവരാകാന്‍ അല്ലാഹു നിങ്ങള്‍ക്കു നല്‍കുന്ന ഉപദേശമാണിത്.
\end{malayalam}}
\flushright{\begin{Arabic}
\quranayah[6][154]
\end{Arabic}}
\flushleft{\begin{malayalam}
നാം മൂസാക്കു വേദപുസ്തകം നല്‍കി. നന്മ ചെയ്തവര്‍ക്കുള്ള അനുഗ്രഹത്തിന്റെ പൂര്‍ത്തീകരണമായാണത്. എല്ലാ കാര്യങ്ങളുടെയും വിശദീകരണവും മാര്‍ഗദര്‍ശനവും കാരുണ്യവുമായാണത്. അവര്‍ തങ്ങളുടെ നാഥനുമായി കണ്ടുമുട്ടുമെന്ന് വിശ്വസിക്കുന്നവരാകാന്‍.
\end{malayalam}}
\flushright{\begin{Arabic}
\quranayah[6][155]
\end{Arabic}}
\flushleft{\begin{malayalam}
നാം ഇറക്കിയ അനുഗൃഹീതമായ വേദപുസ്തകമാണിത്. അതിനാല്‍ നിങ്ങളിതിനെ പിന്‍പറ്റുക. ഭക്തരാവുകയും ചെയ്യുക. നിങ്ങള്‍ കാരുണ്യത്തിനര്‍ഹരായേക്കാം.
\end{malayalam}}
\flushright{\begin{Arabic}
\quranayah[6][156]
\end{Arabic}}
\flushleft{\begin{malayalam}
"ഞങ്ങള്‍ക്കു മുമ്പുള്ള രണ്ടു വിഭാഗക്കാര്‍ക്കു മാത്രമേ വേദപുസ്തകം ലഭിച്ചിരുന്നുള്ളൂ. ഞങ്ങളാകട്ടെ, അവര്‍ പഠിക്കുകയും പഠിപ്പിക്കുകയും ചെയ്തിരുന്നതിനെപ്പറ്റി തീര്‍ത്തും അശ്രദ്ധരുമായിരുന്നു”വെന്ന് നിങ്ങള്‍ പറയാതിരിക്കാനാണ് നാമിതവതരിപ്പിച്ചത്.
\end{malayalam}}
\flushright{\begin{Arabic}
\quranayah[6][157]
\end{Arabic}}
\flushleft{\begin{malayalam}
സംശയം വേണ്ട; ഇതു തന്നെയാണ് എന്റെ നേര്‍വഴി. അതിനാല്‍ നിങ്ങളിത് പിന്തുടരുക. മറ്റു മാര്‍ഗങ്ങള്‍ അവലംബിക്കരുത്. അവയൊക്കെ അല്ലാഹുവിന്റെ വഴിയില്‍നിന്ന് നിങ്ങളെ തെറ്റിച്ചുകളയും. നിങ്ങള്‍ ഭക്തിയുള്ളവരാകാന്‍ അല്ലാഹു നിങ്ങള്‍ക്കു നല്‍കുന്ന ഉപദേശമാണിത്.
\end{malayalam}}
\flushright{\begin{Arabic}
\quranayah[6][158]
\end{Arabic}}
\flushleft{\begin{malayalam}
നാം മൂസാക്കു വേദപുസ്തകം നല്‍കി. നന്മ ചെയ്തവര്‍ക്കുള്ള അനുഗ്രഹത്തിന്റെ പൂര്‍ത്തീകരണമായാണത്. എല്ലാ കാര്യങ്ങളുടെയും വിശദീകരണവും മാര്‍ഗദര്‍ശനവും കാരുണ്യവുമായാണത്. അവര്‍ തങ്ങളുടെ നാഥനുമായി കണ്ടുമുട്ടുമെന്ന് വിശ്വസിക്കുന്നവരാകാന്‍.
\end{malayalam}}
\flushright{\begin{Arabic}
\quranayah[6][159]
\end{Arabic}}
\flushleft{\begin{malayalam}
നാം ഇറക്കിയ അനുഗൃഹീതമായ വേദപുസ്തകമാണിത്. അതിനാല്‍ നിങ്ങളിതിനെ പിന്‍പറ്റുക. ഭക്തരാവുകയും ചെയ്യുക. നിങ്ങള്‍ കാരുണ്യത്തിനര്‍ഹരായേക്കാം.
\end{malayalam}}
\flushright{\begin{Arabic}
\quranayah[6][160]
\end{Arabic}}
\flushleft{\begin{malayalam}
"ഞങ്ങള്‍ക്കു മുമ്പുള്ള രണ്ടു വിഭാഗക്കാര്‍ക്കു മാത്രമേ വേദപുസ്തകം ലഭിച്ചിരുന്നുള്ളൂ. ഞങ്ങളാകട്ടെ, അവര്‍ പഠിക്കുകയും പഠിപ്പിക്കുകയും ചെയ്തിരുന്നതിനെപ്പറ്റി തീര്‍ത്തും അശ്രദ്ധരുമായിരുന്നു”വെന്ന് നിങ്ങള്‍ പറയാതിരിക്കാനാണ് നാമിതവതരിപ്പിച്ചത്.
\end{malayalam}}
\flushright{\begin{Arabic}
\quranayah[6][161]
\end{Arabic}}
\flushleft{\begin{malayalam}
പറയുക: ഉറപ്പായും എന്റെ നാഥന്‍ എന്നെ നേര്‍വഴിയിലേക്ക് നയിച്ചിരിക്കുന്നു. വളവുതിരിവുകളേതുമില്ലാത്ത മതത്തിലേക്ക്. ഇബ്റാഹീം നിലകൊണ്ട വക്രതയില്ലാത്ത മാര്‍ഗത്തിലേക്ക്. അദ്ദേഹം ബഹുദൈവവാദികളില്‍ പെട്ടവനായിരുന്നില്ല.
\end{malayalam}}
\flushright{\begin{Arabic}
\quranayah[6][162]
\end{Arabic}}
\flushleft{\begin{malayalam}
പറയുക: "നിശ്ചയമായും എന്റെ നമസ്കാരവും ആരാധനാകര്‍മങ്ങളും ജീവിതവും മരണവുമെല്ലാം പ്രപഞ്ചനാഥനായ അല്ലാഹുവിനുള്ളതാണ്.
\end{malayalam}}
\flushright{\begin{Arabic}
\quranayah[6][163]
\end{Arabic}}
\flushleft{\begin{malayalam}
"അവന് പങ്കാളികളാരുമില്ല. അവ്വിധമാണ് എന്നോട് കല്‍പിച്ചിരിക്കുന്നത്. അവനെ അനുസരിക്കുന്നവരില്‍ ഒന്നാമനാണ് ഞാന്‍.”
\end{malayalam}}
\flushright{\begin{Arabic}
\quranayah[6][164]
\end{Arabic}}
\flushleft{\begin{malayalam}
ചോദിക്കുക: “ഞാന്‍ അല്ലാഹുവല്ലാത്ത മറ്റൊരു രക്ഷകനെ തേടുകയോ; അവന്‍ എല്ലാറ്റിന്റെയും നാഥനായിരിക്കെ.” ഏതൊരാളും ചെയ്തുകൂട്ടുന്നതിന്റെ ഉത്തരവാദിത്തം അയാള്‍ക്കു മാത്രമായിരിക്കും. ഭാരം ചുമക്കുന്ന ആരും മറ്റൊരാളുടെ ഭാരം വഹിക്കുകയില്ല. പിന്നീട് നിങ്ങളുടെയൊക്കെ മടക്കം നിങ്ങളുടെ നാഥങ്കലേക്കു തന്നെയാണ്. നിങ്ങള്‍ ഭിന്നാഭിപ്രായം പുലര്‍ത്തിയ കാര്യങ്ങളുടെ നിജസ്ഥിതി അപ്പോള്‍ അവന്‍അവിടെവെച്ച് നിങ്ങളെ അറിയിക്കും.
\end{malayalam}}
\flushright{\begin{Arabic}
\quranayah[6][165]
\end{Arabic}}
\flushleft{\begin{malayalam}
നിങ്ങളെ ഭൂമിയില്‍ പ്രതിനിധികളാക്കിയത് അവനാണ്. നിങ്ങളില്‍ ചിലരെ മറ്റു ചിലരെക്കാള്‍ ഉന്നത പദവികളിലേക്ക് ഉയര്‍ത്തിയതും അവന്‍ തന്നെ. നിങ്ങള്‍ക്ക് അവന്‍ നല്‍കിയ കഴിവില്‍ നിങ്ങളെ പരീക്ഷിക്കാനാണിത്. സംശയമില്ല; നിന്റെ നാഥന്‍ വേഗം ശിക്ഷാ നടപടി സ്വീകരിക്കുന്നവനാണ്. ഒപ്പം ഏറെ പൊറുക്കുന്നവനും ദയാപരനും തന്നെ.
\end{malayalam}}
\chapter{\textmalayalam{അഅ്റാഫ് ( ഉന്നതസ്ഥലങ്ങള്‍‍ )}}
\begin{Arabic}
\Huge{\centerline{\basmalah}}\end{Arabic}
\flushright{\begin{Arabic}
\quranayah[7][1]
\end{Arabic}}
\flushleft{\begin{malayalam}
അലിഫ് - ലാം - മീം - സ്വാദ്.
\end{malayalam}}
\flushright{\begin{Arabic}
\quranayah[7][2]
\end{Arabic}}
\flushleft{\begin{malayalam}
നിനക്കിറക്കിയ വേദമാണിത്. ഇതേക്കുറിച്ച് നിന്റെ മനസ്സ് ഒട്ടും അശാന്തമാവേണ്ടതില്ല. മുന്നറിയിപ്പ് നല്‍കാനുള്ളതാണിത്. വിശ്വാസികള്‍ക്ക് ഉദ്ബോധനമേകാനും.
\end{malayalam}}
\flushright{\begin{Arabic}
\quranayah[7][3]
\end{Arabic}}
\flushleft{\begin{malayalam}
നിങ്ങളുടെ നാഥനില്‍നിന്ന് നിങ്ങള്‍ക്കിറക്കിയതിനെ പിന്‍പറ്റുക. അവനെ കൂടാതെ മറ്റു രക്ഷകരെ പിന്തുടരരുത്. നിങ്ങള്‍ വളരെ കുറച്ചേ ആലോചിച്ചറിയുന്നുള്ളൂ.
\end{malayalam}}
\flushright{\begin{Arabic}
\quranayah[7][4]
\end{Arabic}}
\flushleft{\begin{malayalam}
എത്രയെത്ര നാടുകളെയാണ് നാം നശിപ്പിച്ചത്. അങ്ങനെ നമ്മുടെ ശിക്ഷ രാത്രിയിലവരില്‍ വന്നെത്തി. അല്ലെങ്കില്‍ അവര്‍ ഉച്ചയുറക്കിലായിരിക്കെ.
\end{malayalam}}
\flushright{\begin{Arabic}
\quranayah[7][5]
\end{Arabic}}
\flushleft{\begin{malayalam}
നമ്മുടെ ശിക്ഷ വന്നെത്തിയപ്പോള്‍ അവരുടെ വിലാപം ഇതു മാത്രമായിരുന്നു: “ഞങ്ങള്‍ അക്രമികളായിപ്പോയല്ലോ.”
\end{malayalam}}
\flushright{\begin{Arabic}
\quranayah[7][6]
\end{Arabic}}
\flushleft{\begin{malayalam}
ദൈവദൂതന്മാര്‍ ആഗതരായ ജനതയെ തീര്‍ച്ചയായും നാം ചോദ്യം ചെയ്യും; ദൈവദൂതന്മാരെയും നാം ചോദ്യം ചെയ്യും; ഉറപ്പ്.
\end{malayalam}}
\flushright{\begin{Arabic}
\quranayah[7][7]
\end{Arabic}}
\flushleft{\begin{malayalam}
പിന്നെ നാംതന്നെ കൃത്യമായ അറിവോടെ കഴിഞ്ഞതൊക്കെയും അവര്‍ക്കു വിവരിച്ചുകൊടുക്കും. നമ്മുടെ സാന്നിധ്യം എവിടെയും ഉണ്ടാവാതിരുന്നിട്ടില്ല.
\end{malayalam}}
\flushright{\begin{Arabic}
\quranayah[7][8]
\end{Arabic}}
\flushleft{\begin{malayalam}
അന്നാളിലെ തൂക്കം സത്യമായിരിക്കും. അപ്പോള്‍ ആരുടെ തുലാസുകള്‍ കനം തൂങ്ങുന്നുവോ അവര്‍ തന്നെയായിരിക്കും വിജയികള്‍.
\end{malayalam}}
\flushright{\begin{Arabic}
\quranayah[7][9]
\end{Arabic}}
\flushleft{\begin{malayalam}
ആരുടെ തുലാസിന്‍തട്ട് കനം കുറഞ്ഞതാവുന്നുവോ അവര്‍ തന്നെയാണ് സ്വയം നഷ്ടത്തിലകപ്പെട്ടവര്‍. അവര്‍, നമ്മുടെ പ്രമാണങ്ങളെ ധിക്കരിച്ചുകൊണ്ടിരുന്നതിനാലാണത്.
\end{malayalam}}
\flushright{\begin{Arabic}
\quranayah[7][10]
\end{Arabic}}
\flushleft{\begin{malayalam}
നിങ്ങള്‍ക്കു നാം ഭൂമിയില്‍ സൌകര്യമൊരുക്കിത്തന്നു. ജീവിത വിഭവങ്ങള്‍ തയ്യാറാക്കിത്തരികയും ചെയ്തു. എന്നിട്ടും നന്നെക്കുറച്ചേ നിങ്ങള്‍ നന്ദി കാണിക്കുന്നുള്ളൂ.
\end{malayalam}}
\flushright{\begin{Arabic}
\quranayah[7][11]
\end{Arabic}}
\flushleft{\begin{malayalam}
തീര്‍ച്ചയായും നാം നിങ്ങളെ സൃഷ്ടിച്ചു. പിന്നെ നിങ്ങള്‍ക്ക് രൂപമേകി. തുടര്‍ന്ന് നാം മലക്കുകളോട് പറഞ്ഞു: "ആദമിനെ പ്രണമിക്കുക.” അവര്‍ പ്രണമിച്ചു. ഇബ്ലീസൊഴികെ. അവന്‍ പ്രണമിച്ചവരില്‍ പെട്ടില്ല.
\end{malayalam}}
\flushright{\begin{Arabic}
\quranayah[7][12]
\end{Arabic}}
\flushleft{\begin{malayalam}
അല്ലാഹു ചോദിച്ചു: "ഞാന്‍ നിന്നോട് കല്‍പിച്ചപ്പോള്‍ പ്രണാമമര്‍പ്പിക്കുന്നതില്‍ നിന്ന് നിന്നെ തടഞ്ഞതെന്ത്?” അവന്‍ പറഞ്ഞു: "ഞാനാണ് അവനേക്കാള്‍ മെച്ചം. നീയെന്നെ സൃഷ്ടിച്ചത് തീയില്‍ നിന്നാണ്. അവനെ മണ്ണില്‍ നിന്നും.”
\end{malayalam}}
\flushright{\begin{Arabic}
\quranayah[7][13]
\end{Arabic}}
\flushleft{\begin{malayalam}
അല്ലാഹു കല്‍പിച്ചു: "എങ്കില്‍ നീ ഇവിടെ നിന്നിറങ്ങിപ്പോകൂ. നിനക്കിവിടെ അഹങ്കരിക്കാന്‍ അര്‍ഹതയില്ല. പുറത്തുപോ. സംശയമില്ല; നീ നിന്ദ്യരില്‍പെട്ടവന്‍ തന്നെ.”
\end{malayalam}}
\flushright{\begin{Arabic}
\quranayah[7][14]
\end{Arabic}}
\flushleft{\begin{malayalam}
ഇബ്ലീസ് പറഞ്ഞു: "എല്ലാവരും ഉയിര്‍ത്തെഴുന്നേല്‍ക്കുന്ന ദിവസംവരെ എനിക്കു കാലാവധി നല്‍കിയാലും.”
\end{malayalam}}
\flushright{\begin{Arabic}
\quranayah[7][15]
\end{Arabic}}
\flushleft{\begin{malayalam}
അല്ലാഹു പറഞ്ഞു: "ശരി, സംശയംവേണ്ട നിനക്ക് അവധി അനുവദിച്ചിരിക്കുന്നു.”
\end{malayalam}}
\flushright{\begin{Arabic}
\quranayah[7][16]
\end{Arabic}}
\flushleft{\begin{malayalam}
ഇബ്ലീസ് പറഞ്ഞു: "നീ എന്നെ വഴിപിഴപ്പിച്ചതിനാല്‍ നിന്റെ നേര്‍വഴിയില്‍ ഞാന്‍ അവര്‍ക്കായി തക്കം പാര്‍ത്തിരിക്കും.
\end{malayalam}}
\flushright{\begin{Arabic}
\quranayah[7][17]
\end{Arabic}}
\flushleft{\begin{malayalam}
"പിന്നെ അവരുടെ മുന്നിലൂടെയും പിന്നിലൂടെയും വലത്തുനിന്നും ഇടത്തുനിന്നും ഞാനവരുടെ അടുത്ത് ചെല്ലും. ഉറപ്പായും അവരിലേറെ പേരെയും നന്ദിയുള്ളവരായി നിനക്കു കാണാനാവില്ല.”
\end{malayalam}}
\flushright{\begin{Arabic}
\quranayah[7][18]
\end{Arabic}}
\flushleft{\begin{malayalam}
അല്ലാഹു കല്‍പിച്ചു: "നിന്ദ്യനും ആട്ടിയിറക്കപ്പെട്ടവനുമായി നീ ഇവിടെനിന്ന് പുറത്തുപോവുക. മനുഷ്യരില്‍ നിന്ന് ആരെങ്കിലും നിന്നെ പിന്തുടര്‍ന്നാല്‍ നിങ്ങളെയൊക്കെ ഞാന്‍ നരകത്തീയിലിട്ട് നിറക്കും.”
\end{malayalam}}
\flushright{\begin{Arabic}
\quranayah[7][19]
\end{Arabic}}
\flushleft{\begin{malayalam}
"ആദം, നീയും നിന്റെ ഇണയും ഈ സ്വര്‍ഗത്തില്‍ താമസിക്കുക. നിങ്ങള്‍ക്കിരുവര്‍ക്കും ഇഷ്ടമുള്ളിടത്തുനിന്ന് തിന്നാം. എന്നാല്‍ ഈ മരത്തോട് അടുക്കരുത്; നിങ്ങള്‍ അക്രമികളില്‍ പെട്ടുപോകും.”
\end{malayalam}}
\flushright{\begin{Arabic}
\quranayah[7][20]
\end{Arabic}}
\flushleft{\begin{malayalam}
പിന്നെ, പിശാച് ഇരുവരോടും ദുര്‍മന്ത്രണം നടത്തി; അവരില്‍ ഒളിഞ്ഞിരിക്കുന്ന നഗ്നസ്ഥാനങ്ങള്‍ അവര്‍ക്ക് വെളിപ്പെടുത്താന്‍. അവന്‍ പറഞ്ഞു: "നിങ്ങളുടെ നാഥന്‍ ഈ മരം നിങ്ങള്‍ക്ക് വിലക്കിയത് നിങ്ങള്‍ മലക്കുകളായിമാറുകയോ ഇവിടെ നിത്യവാസികളായിത്തീരുകയോ ചെയ്യുമെന്നതിനാല്‍ മാത്രമാണ്.”
\end{malayalam}}
\flushright{\begin{Arabic}
\quranayah[7][21]
\end{Arabic}}
\flushleft{\begin{malayalam}
ഒപ്പം അവന്‍ അവരോട് ആണയിട്ടു പറഞ്ഞു: "ഞാന്‍ നിങ്ങളുടെ ഗുണകാംക്ഷി മാത്രമാണ്.”
\end{malayalam}}
\flushright{\begin{Arabic}
\quranayah[7][22]
\end{Arabic}}
\flushleft{\begin{malayalam}
അങ്ങനെ അവരിരുവരെയും അവന്‍ വഞ്ചനയിലൂടെ വശപ്പെടുത്തി. ഇരുവരും ആ മരത്തിന്റെ രുചി ആസ്വദിച്ചു. അതോടെ തങ്ങളുടെ നഗ്നത ഇരുവര്‍ക്കും വെളിപ്പെട്ടു. ആ തോട്ടത്തിലെ ഇലകള്‍ ചേര്‍ത്തുവെച്ച് അവര്‍ തങ്ങളുടെ ശരീരം മറയ്ക്കാന്‍ തുടങ്ങി. അവരുടെ നാഥന്‍ ഇരുവരെയും വിളിച്ചുചോദിച്ചു: "ആ മരം നിങ്ങള്‍ക്കു ഞാന്‍ വിലക്കിയിരുന്നില്ലേ? പിശാച് നിങ്ങളുടെ പ്രത്യക്ഷ ശത്രുവാണെന്ന് നിങ്ങളോട് പറഞ്ഞിരുന്നില്ലേ?”
\end{malayalam}}
\flushright{\begin{Arabic}
\quranayah[7][23]
\end{Arabic}}
\flushleft{\begin{malayalam}
ഇരുവരും പറഞ്ഞു: "ഞങ്ങളുടെ നാഥാ! ഞങ്ങള്‍ ഞങ്ങളോടു തന്നെ അക്രമം കാണിച്ചിരിക്കുന്നു. നീ മാപ്പേകുകയും ദയ കാണിക്കുകയും ചെയ്തില്ലെങ്കില്‍ ഉറപ്പായും ഞങ്ങള്‍ നഷ്ടം പറ്റിയവരായിത്തീരും.”
\end{malayalam}}
\flushright{\begin{Arabic}
\quranayah[7][24]
\end{Arabic}}
\flushleft{\begin{malayalam}
അല്ലാഹു കല്‍പിച്ചു: "ഇറങ്ങിപ്പോകൂ. നിങ്ങളന്യോന്യം ശത്രുക്കളായിരിക്കും. ഭൂമിയില്‍ നിങ്ങള്‍ക്ക് താമസസൌകര്യമുണ്ട്. നിശ്ചിതകാലംവരെ ജീവിത വിഭവങ്ങളും.”
\end{malayalam}}
\flushright{\begin{Arabic}
\quranayah[7][25]
\end{Arabic}}
\flushleft{\begin{malayalam}
അവന്‍ പറഞ്ഞു: "നിങ്ങള്‍ അവിടെത്തന്നെ ജീവിക്കും. അവിടെത്തന്നെ മരിക്കും. അവിടെ നിന്നുതന്നെ നിങ്ങളെ പുറത്തുകൊണ്ടുവരികയും ചെയ്യും.”
\end{malayalam}}
\flushright{\begin{Arabic}
\quranayah[7][26]
\end{Arabic}}
\flushleft{\begin{malayalam}
ആദം സന്തതികളേ, നിങ്ങള്‍ക്കു നാം നിങ്ങളുടെ ഗുഹ്യസ്ഥാനം മറയ്ക്കാനും ശരീരം അലങ്കരിക്കാനും പറ്റിയ വസ്ത്രങ്ങളുല്‍പാദിപ്പിച്ചു തന്നിരിക്കുന്നു. എന്നാല്‍ ഭക്തിയുടെ വസ്ത്രമാണ് ഏറ്റം ഉത്തമം. അല്ലാഹുവിന്റെ ദൃഷ്ടാന്തങ്ങളിലൊന്നാണിത്. അവര്‍ മനസ്സിലാക്കി പാഠമുള്‍ക്കൊള്ളാന്‍.
\end{malayalam}}
\flushright{\begin{Arabic}
\quranayah[7][27]
\end{Arabic}}
\flushleft{\begin{malayalam}
ആദം സന്തതികളേ, പിശാച് നിങ്ങളുടെ മാതാപിതാക്കളെ സ്വര്‍ഗത്തില്‍ നിന്ന് പുറത്താക്കിയപോലെ അവന്‍ നിങ്ങളെ നാശത്തില്‍ പെടുത്താതിരിക്കട്ടെ. അവരിരുവര്‍ക്കും തങ്ങളുടെ ഗുഹ്യസ്ഥാനങ്ങള്‍ കാണിച്ചുകൊടുക്കാനായി അവന്‍ അവരുടെ വസ്ത്രം അഴിച്ചുമാറ്റുകയായിരുന്നു. അവനും അവന്റെ കൂട്ടുകാരും നിങ്ങളെ കണ്ടുകൊണ്ടിരിക്കും. എന്നാല്‍ നിങ്ങള്‍ക്ക് അവരെ കാണാനാവില്ല. പിശാചുക്കളെ നാം അവിശ്വാസികളുടെ രക്ഷാധികാരികളാക്കിയിരിക്കുന്നു.
\end{malayalam}}
\flushright{\begin{Arabic}
\quranayah[7][28]
\end{Arabic}}
\flushleft{\begin{malayalam}
വല്ല മ്ളേച്ഛവൃത്തിയും ചെയ്താല്‍ അവര്‍ പറയുന്നു: "ഞങ്ങളുടെ പിതാക്കന്മാര്‍ അങ്ങനെ ചെയ്യുന്നത് ഞങ്ങള്‍ കണ്ടിട്ടുണ്ട്. അല്ലാഹു ഞങ്ങളോട് കല്‍പിച്ചതും അതാണ്.” പറയുക: മ്ളേച്ഛവൃത്തികള്‍ ചെയ്യാന്‍ അല്ലാഹു കല്‍പിക്കുകയില്ല. നിങ്ങള്‍ അറിവില്ലാത്ത കാര്യങ്ങള്‍ അല്ലാഹുവിന്റെ പേരില്‍ പറഞ്ഞുണ്ടാക്കുകയാണോ?
\end{malayalam}}
\flushright{\begin{Arabic}
\quranayah[7][29]
\end{Arabic}}
\flushleft{\begin{malayalam}
പറയുക: എന്റെ നാഥന്‍ നീതിയാണ് നിര്‍ദേശിച്ചത്. എല്ലാ ആരാധനകളിലും നിങ്ങളുടെ മുഖം അവന് നേരെ നിറുത്തണമെന്ന് അവന്‍ കല്‍പിച്ചിരിക്കുന്നു. വിധേയത്വം അവനോടു മാത്രമാക്കി അവനോടു പ്രാര്‍ഥിക്കണമെന്നും. ആദ്യത്തില്‍ നിങ്ങളെ എങ്ങനെ സൃഷ്ടിച്ചുവോ അവ്വിധം തന്നെ നിങ്ങള്‍ തിരിച്ചുചെല്ലും.
\end{malayalam}}
\flushright{\begin{Arabic}
\quranayah[7][30]
\end{Arabic}}
\flushleft{\begin{malayalam}
ഒരു വിഭാഗത്തെ അവന്‍ നേര്‍വഴിയിലാക്കി. മറ്റൊരു വിഭാഗം, അല്ലാഹുവെ വിട്ട് പിശാചുക്കളെ രക്ഷാധികാരികളാക്കിയതിനാല്‍ ദുര്‍മാര്‍ഗത്തിനാണ് അര്‍ഹരായത്. എന്നിട്ടും അവര്‍ വിചാരിക്കുന്നു; തങ്ങള്‍ നേര്‍വഴിയിലാണെന്ന്.
\end{malayalam}}
\flushright{\begin{Arabic}
\quranayah[7][31]
\end{Arabic}}
\flushleft{\begin{malayalam}
ആദം സന്തതികളേ, എല്ലാ ആരാധനകളിലും നിങ്ങള്‍ നിങ്ങളുടെ അലങ്കാരങ്ങളണിയുക. തിന്നുകയും കുടിക്കുകയും ചെയ്യുക. എന്നാല്‍ അമിതമാവരുത്. അമിതവ്യയം ചെയ്യുന്നവരെ അല്ലാഹു ഇഷ്ടപ്പെടുന്നില്ല.
\end{malayalam}}
\flushright{\begin{Arabic}
\quranayah[7][32]
\end{Arabic}}
\flushleft{\begin{malayalam}
ചോദിക്കുക: അല്ലാഹു തന്റെ ദാസന്മാര്‍ക്കായുണ്ടാക്കിയ അലങ്കാരങ്ങളും ഉത്തമമായ ആഹാരപദാര്‍ഥങ്ങളും നിഷിദ്ധമാക്കിയതാരാണ്? പറയുക: അവ ഐഹിക ജീവിതത്തില്‍ സത്യവിശ്വാസികള്‍ക്കുള്ളതാണ്. ഉയിര്‍ത്തെഴുന്നേല്‍പു നാളിലോ അവര്‍ക്കു മാത്രവും. കാര്യം ഗ്രഹിക്കുന്നവര്‍ക്കായി നാം ഇവ്വിധം തെളിവുകള്‍ വിശദീകരിക്കുന്നു.
\end{malayalam}}
\flushright{\begin{Arabic}
\quranayah[7][33]
\end{Arabic}}
\flushleft{\begin{malayalam}
പറയുക: രഹസ്യവും പരസ്യവുമായ നീചവൃത്തികള്‍, കുറ്റകൃത്യം, അന്യായമായ അതിക്രമം, അല്ലാഹു ഒരു തെളിവും ഇറക്കിത്തരാത്ത വസ്തുക്കളെ അവനില്‍ പങ്കുചേര്‍ക്കല്‍, നിങ്ങള്‍ക്കറിയാത്ത കാര്യങ്ങള്‍ അല്ലാഹുവിന്റെ പേരില്‍ പറഞ്ഞുണ്ടാക്കല്‍- ഇതൊക്കെയാണ് എന്റെ നാഥന്‍ നിഷിദ്ധമാക്കിയത്.
\end{malayalam}}
\flushright{\begin{Arabic}
\quranayah[7][34]
\end{Arabic}}
\flushleft{\begin{malayalam}
ഓരോ സമുദായത്തിനും നിശ്ചിതമായ കാലാവധിയുണ്ട്. അങ്ങനെ അവരുടെ അവധി വന്നെത്തിയാല്‍ പിന്നെ അവര്‍ക്കൊരു നിമിഷംപോലും മുന്നോട്ടോ പിന്നോട്ടോ നീങ്ങാനാവില്ല.
\end{malayalam}}
\flushright{\begin{Arabic}
\quranayah[7][35]
\end{Arabic}}
\flushleft{\begin{malayalam}
ആദം സന്തതികളേ, നിങ്ങളുടെ അടുത്ത് എന്റെ പ്രമാണങ്ങള്‍ വിവരിച്ചുതരാനായി നിങ്ങളില്‍ നിന്നുതന്നെയുള്ള ദൂതന്മാര്‍ വരും. അപ്പോള്‍ ഭക്തിപുലര്‍ത്തുകയും തങ്ങളുടെ നടപടികള്‍ നന്നാക്കിത്തീര്‍ക്കുകയും ചെയ്യുന്നവര്‍ പേടിക്കേണ്ടതില്ല. അവര്‍ ദുഃഖിക്കേണ്ടിവരികയുമില്ല.
\end{malayalam}}
\flushright{\begin{Arabic}
\quranayah[7][36]
\end{Arabic}}
\flushleft{\begin{malayalam}
എന്നാല്‍ നമ്മുടെ വചനങ്ങളെ കള്ളമാക്കിത്തള്ളുകയും അവയുടെ നേരെ അഹന്ത നടിക്കുകയും ചെയ്യുന്നവരാണ് നരകാവകാശികള്‍. അവരതില്‍ സ്ഥിരവാസികളായിരിക്കും.
\end{malayalam}}
\flushright{\begin{Arabic}
\quranayah[7][37]
\end{Arabic}}
\flushleft{\begin{malayalam}
അല്ലാഹുവിന്റെ പേരില്‍ കള്ളം കെട്ടിച്ചമക്കുകയോ അവന്റെ വചനങ്ങളെ കള്ളമാക്കിത്തള്ളുകയോ ചെയ്തവനെക്കാള്‍ കൊടിയ അക്രമി ആരുണ്ട്? അവര്‍ ദൈവത്തിന്റെ വിധിത്തീര്‍പ്പനുസരിച്ചുള്ള തങ്ങളുടെ വിഹിതം ഏറ്റുവാങ്ങേണ്ടിവരിക തന്നെ ചെയ്യും. അങ്ങനെ അവരെ മരിപ്പിക്കാനായി നമ്മുടെ ദൂതന്മാര്‍ അവരുടെ അടുത്ത് ചെല്ലുമ്പോള്‍ ചോദിക്കും: "അല്ലാഹുവെ വിട്ട് നിങ്ങള്‍ വിളിച്ചു പ്രാര്‍ഥിച്ചുകൊണ്ടിരുന്നവര്‍ ഇപ്പോഴെവിടെ?” അവര്‍ പറയും: "അവരൊക്കെയും ഞങ്ങളെ കൈവിട്ടിരിക്കുന്നു.” അങ്ങനെ, തങ്ങള്‍ സത്യനിഷേധികളായിരുന്നുവെന്ന് അവര്‍ തന്നെ തങ്ങള്‍ക്കെതിരെ സാക്ഷ്യം വഹിക്കും.
\end{malayalam}}
\flushright{\begin{Arabic}
\quranayah[7][38]
\end{Arabic}}
\flushleft{\begin{malayalam}
അല്ലാഹു പറയും: നിങ്ങള്‍ക്കുമുമ്പെ കഴിഞ്ഞുപോയ ജിന്നുകളിലും മനുഷ്യരിലും പെട്ട സമൂഹങ്ങളോടൊപ്പം നിങ്ങളും നരകത്തീയില്‍ പ്രവേശിക്കുക. ഓരോ സംഘവും അതില്‍ പ്രവേശിക്കുമ്പോള്‍ തങ്ങളുടെ മുന്‍ഗാമികളായ സംഘത്തെ ശപിച്ചുകൊണ്ടിരിക്കും. അങ്ങനെ അവരൊക്കെയും അവിടെ ഒരുമിച്ചുകൂടിയാല്‍ അവരിലെ പിന്‍ഗാമികള്‍ തങ്ങളുടെ മുന്‍ഗാമികളെക്കുറിച്ച് പറയും: "ഞങ്ങളുടെ നാഥാ! ഇവരാണ് ഞങ്ങളെ വഴി പിഴപ്പിച്ചത്. അതിനാല്‍ ഇവര്‍ക്കു നീ ഇരട്ടി നരകശിക്ഷ നല്‍കേണമേ.” അല്ലാഹു അരുളും: "എല്ലാവര്‍ക്കും ഇരട്ടി ശിക്ഷയുണ്ട്. പക്ഷേ; നിങ്ങളറിയുന്നില്ലെന്നുമാത്രം.”
\end{malayalam}}
\flushright{\begin{Arabic}
\quranayah[7][39]
\end{Arabic}}
\flushleft{\begin{malayalam}
അവരിലെ മുന്‍ഗാമികള്‍ തങ്ങളുടെ പിന്‍ഗാമികളോടു പറയും: "അപ്പോള്‍ നിങ്ങള്‍ക്ക് ഞങ്ങളെക്കാള്‍ ഒരു ശ്രേഷ്ഠതയുമില്ല. അതിനാല്‍ നിങ്ങള്‍ ശേഖരിച്ചുവെച്ചിരുന്നതിന്റെ ഫലമായുള്ള ശിക്ഷ നിങ്ങളനുഭവിച്ചുകൊള്ളുക.”
\end{malayalam}}
\flushright{\begin{Arabic}
\quranayah[7][40]
\end{Arabic}}
\flushleft{\begin{malayalam}
നമ്മുടെ വചനങ്ങളെ കള്ളമാക്കിത്തള്ളുകയും അവയുടെ നേരെ അഹന്ത നടിക്കുകയും ചെയ്തവര്‍ക്കുവേണ്ടി ഒരിക്കലും ആകാശത്തിന്റെ കവാടങ്ങള്‍ തുറന്നുകൊടുക്കുകയില്ല. ഒട്ടകം സൂചിക്കുഴയിലൂടെ കടന്നുപോകുവോളം അവര്‍ സ്വര്‍ഗത്തില്‍ പ്രവേശിക്കുകയുമില്ല. അവ്വിധമാണ് നാം കുറ്റവാളികള്‍ക്ക് പ്രതിഫലം നല്‍കുക.
\end{malayalam}}
\flushright{\begin{Arabic}
\quranayah[7][41]
\end{Arabic}}
\flushleft{\begin{malayalam}
അവര്‍ക്ക് നരകത്തീയാലുള്ള മെത്തകളാണുണ്ടാവുക. അവര്‍ക്കുമീതെ തീ കൊണ്ടുള്ള പുതപ്പുകളുമുണ്ടാകും. അവ്വിധമാണ് നാം അക്രമികള്‍ക്ക് പ്രതിഫലം നല്‍കുക.
\end{malayalam}}
\flushright{\begin{Arabic}
\quranayah[7][42]
\end{Arabic}}
\flushleft{\begin{malayalam}
എന്നാല്‍, സത്യവിശ്വാസം സ്വീകരിക്കുകയും സല്‍ക്കര്‍മങ്ങള്‍ പ്രവര്‍ത്തിക്കുകയും ചെയ്തവരോ, - ആരെയും അവരുടെ കഴിവിന്നതീതമായ ബാധ്യത നാം ഏല്‍പിക്കുന്നില്ല - അവരാണ് സ്വര്‍ഗാവകാശികള്‍. അതിലവര്‍ നിത്യവാസികളായിരിക്കും.
\end{malayalam}}
\flushright{\begin{Arabic}
\quranayah[7][43]
\end{Arabic}}
\flushleft{\begin{malayalam}
അവരുടെ മനസ്സുകളിലെ പകയെ നാം തുടച്ചുമാറ്റും. അവരുടെ താഴ്ഭാഗത്തൂടെ അരുവികളൊഴുകിക്കൊണ്ടിരിക്കും. അപ്പോള്‍ അവരിങ്ങനെ പറയും: "ഞങ്ങളെ ഇവിടേക്ക് നയിച്ച അല്ലാഹുവിന് സ്തുതി. അല്ലാഹു ഞങ്ങളെ നേര്‍വഴിയിലാക്കിയില്ലായിരുന്നെങ്കില്‍ ഞങ്ങളൊരിക്കലും സന്മാര്‍ഗം പ്രാപിക്കുമായിരുന്നില്ല. ഞങ്ങളുടെ നാഥന്റെ ദൂതന്മാര്‍ സത്യസന്ദേശവുമായി എത്തിയവരായിരുന്നു.” അപ്പോള്‍ അവരോടിങ്ങനെ വിളിച്ചുപറയും: "ഇതാ നിങ്ങള്‍ക്കുള്ള സ്വര്‍ഗം. നിങ്ങള്‍ പ്രവര്‍ത്തിച്ചിരുന്നതിന്റെ ഫലമായി നിങ്ങളിതിന്റെ അവകാശികളായിത്തീര്‍ന്നിരിക്കുന്നു.”
\end{malayalam}}
\flushright{\begin{Arabic}
\quranayah[7][44]
\end{Arabic}}
\flushleft{\begin{malayalam}
സ്വര്‍ഗാവകാശികള്‍ നരകാവകാശികളോട് വിളിച്ചുചോദിക്കും: "ഞങ്ങളുടെ നാഥന്‍ ഞങ്ങളോട് ചെയ്ത വാഗ്ദാനങ്ങളൊക്കെയും സത്യമായിപ്പുലര്‍ന്നത് ഞങ്ങള്‍ കണ്ടറിഞ്ഞിരിക്കുന്നു. നിങ്ങളുടെ നാഥന്‍ നിങ്ങളോടു ചെയ്ത വാഗ്ദാനങ്ങള്‍ യാഥാര്‍ഥ്യമായി പുലര്‍ന്നത് നിങ്ങള്‍ നേരില്‍ കണ്ട് മനസ്സിലാക്കിയോ?” അവര്‍ പറയും: "അതെ.” അപ്പോള്‍ ഒരു വിളിയാളന്‍ അവര്‍ക്കിടയില്‍ വിളിച്ചറിയിക്കും: "അല്ലാഹുവിന്റെ ശാപം അക്രമികള്‍ക്കാണ്; സംശയമില്ല.”
\end{malayalam}}
\flushright{\begin{Arabic}
\quranayah[7][45]
\end{Arabic}}
\flushleft{\begin{malayalam}
അവര്‍ അല്ലാഹുവിന്റെ മാര്‍ഗത്തില്‍ നിന്ന് ആളുകളെ തടയുകയും അതിനെ വക്രമാക്കാന്‍ ശ്രമിക്കുകയും ചെയ്യുന്നവരാണ്; പരലോകത്തെ നിഷേധിക്കുന്നവരും.
\end{malayalam}}
\flushright{\begin{Arabic}
\quranayah[7][46]
\end{Arabic}}
\flushleft{\begin{malayalam}
ഈ രണ്ടു വിഭാഗത്തിനുമിടയില്‍ ഒരു മതിലുണ്ടായിരിക്കും. അതിന്റെ മുകളില്‍ ചില മനുഷ്യരുണ്ടാവും. അവരോരോരുത്തരെയും തങ്ങളുടെ അടയാളങ്ങളിലൂടെ തിരിച്ചറിയും. സ്വര്‍ഗസ്ഥരോട് അവര്‍ വിളിച്ചുപറയും: "നിങ്ങള്‍ക്ക് സമാധാനം.” ഇക്കൂട്ടര്‍ ഇനിയും സ്വര്‍ഗത്തില്‍ പ്രവേശിച്ചിട്ടില്ലാത്തവരാണ്. അതോടൊപ്പം അതാഗ്രഹിച്ചുകൊണ്ടിരിക്കുന്നവരും.
\end{malayalam}}
\flushright{\begin{Arabic}
\quranayah[7][47]
\end{Arabic}}
\flushleft{\begin{malayalam}
അവരുടെ കണ്ണുകള്‍ നരകാവകാശികളുടെ നേരെ തിരിഞ്ഞാല്‍ അവര്‍ പറയും: "ഞങ്ങളുടെ നാഥാ! ഞങ്ങളെ നീ അക്രമികളുടെ കൂട്ടത്തില്‍ പെടുത്തരുതേ!”
\end{malayalam}}
\flushright{\begin{Arabic}
\quranayah[7][48]
\end{Arabic}}
\flushleft{\begin{malayalam}
മതില്‍പ്പുറത്തുള്ളവര്‍ അടയാളങ്ങളിലൂടെ തങ്ങള്‍ക്ക് തിരിച്ചറിയാവുന്ന ചില നരകക്കാരെ വിളിച്ച് പറയും: "നിങ്ങളുടെ ആള്‍ബലവും നിങ്ങള്‍ അഹങ്കരിച്ചുനടന്നതും നിങ്ങള്‍ക്കെന്ത് നേട്ടമാണുണ്ടാക്കിയത്?
\end{malayalam}}
\flushright{\begin{Arabic}
\quranayah[7][49]
\end{Arabic}}
\flushleft{\begin{malayalam}
"ഇക്കൂട്ടരെപ്പറ്റിയല്ലേ അല്ലാഹു അവര്‍ക്കൊരനുഗ്രഹവും നല്‍കുകയില്ലെന്ന് നിങ്ങള്‍ ആണയിട്ട് പറഞ്ഞിരുന്നത്? എന്നിട്ട് അവരോടാണല്ലോ “നിങ്ങള്‍ സ്വര്‍ഗത്തില്‍ കടന്നുകൊള്ളുക. നിങ്ങളൊന്നും പേടിക്കേണ്ടതില്ല. നിങ്ങള്‍ ദുഃഖിക്കേണ്ടിവരികയുമില്ല” എന്നു പറഞ്ഞത്.
\end{malayalam}}
\flushright{\begin{Arabic}
\quranayah[7][50]
\end{Arabic}}
\flushleft{\begin{malayalam}
നരകത്തിലെത്തിയവര്‍ സ്വര്‍ഗത്തിലെത്തിയവരോട് വിളിച്ചുകേഴും: "ഞങ്ങള്‍ക്ക് ഇത്തിരി വെള്ളം ഒഴിച്ചുതരേണമേ, അല്ലെങ്കില്‍ അല്ലാഹു നിങ്ങള്‍ക്കു തന്ന അന്നത്തില്‍നിന്ന് അല്‍പം തരേണമേ.” അവര്‍ പറയും: "സത്യനിഷേധികള്‍ക്ക് അല്ലാഹു ഇവ രണ്ടും പൂര്‍ണമായും വിലക്കിയിരിക്കുന്നു.”
\end{malayalam}}
\flushright{\begin{Arabic}
\quranayah[7][51]
\end{Arabic}}
\flushleft{\begin{malayalam}
അവര്‍ തങ്ങളുടെ ജീവിതക്രമത്തെ കളിതമാശയാക്കിയവരാണ്. ഐഹികജീവിതം കണ്ട് വഞ്ചിതരായവരും. അതിനാല്‍ ഇന്ന് നാം അവരെ മറന്നിരിക്കുന്നു. അവര്‍ ഈ ദിനത്തെ കണ്ടുമുട്ടുമെന്ന കാര്യം മറന്നിരുന്നപോലെത്തന്നെ. നമ്മുടെ വചനങ്ങളെ അവര്‍ തള്ളിക്കളഞ്ഞിരുന്ന പോലെയും.
\end{malayalam}}
\flushright{\begin{Arabic}
\quranayah[7][52]
\end{Arabic}}
\flushleft{\begin{malayalam}
യഥാര്‍ഥ അറിവിന്റെ അടിസ്ഥാനത്തില്‍ വസ്തുതകള്‍ വിശദീകരിച്ച ഗ്രന്ഥം നാം അവര്‍ക്കെത്തിച്ചുകൊടുത്തു. വിശ്വസിക്കുന്ന ജനത്തിന് മാര്‍ഗദര്‍ശനവും അനുഗ്രഹവുമാണത്.
\end{malayalam}}
\flushright{\begin{Arabic}
\quranayah[7][53]
\end{Arabic}}
\flushleft{\begin{malayalam}
ഈ വേദപുസ്തകത്തിലുള്ളത് പുലരുന്നതല്ലാതെ മറ്റെന്താണ് അവര്‍ പ്രതീക്ഷിച്ചുകൊണ്ടിരിക്കുന്നത്? അത് പുലരുംനാളില്‍ നേരത്തെ അതിനെ മറന്നിരുന്നവര്‍ പറയും: "നമ്മുടെ നാഥന്റെ ദൂതന്മാര്‍ സത്യസന്ദേശവുമായി വന്നവരായിരുന്നു. ഇനി ശിപാര്‍ശകരായി വല്ലവരെയും നമുക്ക് കിട്ടുമോ? അതല്ലെങ്കില്‍ ഞങ്ങളെയൊന്ന് തിരിച്ചയക്കുമോ? എങ്കില്‍ ഞങ്ങള്‍ നേരത്തെ പ്രവര്‍ത്തിച്ചിരുന്നതില്‍നിന്ന് വ്യത്യസ്തമായി നല്ല കാര്യങ്ങള്‍ ചെയ്യുമായിരുന്നു.” അവര്‍ സ്വയം നഷ്ടം വരുത്തിവെച്ചവരാണ്. അവര്‍ കെട്ടിച്ചമച്ചിരുന്നതൊക്കെയും അവരെ വിട്ടകന്നിരിക്കുന്നു.
\end{malayalam}}
\flushright{\begin{Arabic}
\quranayah[7][54]
\end{Arabic}}
\flushleft{\begin{malayalam}
നിങ്ങളുടെ നാഥന്‍ അല്ലാഹുവാണ്. ആറ് നാളുകളിലായി ആകാശഭൂമികളെ സൃഷ്ടിച്ചവനാണവന്‍. പിന്നെ അവന്‍ തന്റെ സിംഹാസനത്തിലുപവിഷ്ടനായി. രാവിനെക്കൊണ്ട് അവന്‍ പകലിനെ പൊതിയുന്നു. പകലാണെങ്കില്‍ രാവിനെത്തേടി കുതിക്കുന്നു. സൂര്യ- ചന്ദ്ര-നക്ഷത്രങ്ങളെയെല്ലാം തന്റെ കല്‍പനക്ക് വിധേയമാംവിധം അവന്‍ സൃഷ്ടിച്ചു. അറിയുക: സൃഷ്ടിക്കാനും കല്‍പിക്കാനും അവന്നു മാത്രമാണ് അധികാരം. സര്‍വലോക സംരക്ഷകനായ അല്ലാഹു ഏറെ മഹത്വമുള്ളവനാണ്.
\end{malayalam}}
\flushright{\begin{Arabic}
\quranayah[7][55]
\end{Arabic}}
\flushleft{\begin{malayalam}
നിങ്ങള്‍ വിനയത്തോടെയും രഹസ്യമായും നിങ്ങളുടെ നാഥനോടു പ്രാര്‍ഥിക്കുക. പരിധി ലംഘിക്കുന്നവരെ അവനിഷ്ടമില്ല; തീര്‍ച്ച.
\end{malayalam}}
\flushright{\begin{Arabic}
\quranayah[7][56]
\end{Arabic}}
\flushleft{\begin{malayalam}
ഭൂമിയില്‍ നന്മ വരുത്തിയശേഷം നിങ്ങളതില്‍ നാശമുണ്ടാക്കരുത്. നിങ്ങള്‍ പേടിയോടെയും പ്രതീക്ഷയോടെയും പടച്ചവനോടു മാത്രം പ്രാര്‍ഥിക്കുക. അല്ലാഹുവിന്റെ അനുഗ്രഹം നന്മ ചെയ്യുന്നവരോടു ചേര്‍ന്നാണ്.
\end{malayalam}}
\flushright{\begin{Arabic}
\quranayah[7][57]
\end{Arabic}}
\flushleft{\begin{malayalam}
തന്റെ അനുഗ്രഹത്തിന്റെ മുന്നോടിയായി സുവാര്‍ത്ത അറിയിക്കുന്ന കാറ്റുകളയക്കുന്നതും അവന്‍ തന്നെ. അങ്ങനെ കാറ്റ് കനത്ത കാര്‍മേഘത്തെ വഹിച്ചുകഴിഞ്ഞാല്‍ നാം ആ കാറ്റിനെ ഉണര്‍വറ്റുകിടക്കുന്ന ഏതെങ്കിലും നാട്ടിലേക്ക് നയിക്കുന്നു. അങ്ങനെ അതുവഴി നാം അവിടെ മഴ പെയ്യിക്കുന്നു. അതിലൂടെ എല്ലായിനം പഴങ്ങളും ഉല്‍പാദിപ്പിക്കുന്നു. അവ്വിധം നാം മരിച്ചവരെ ഉയിര്‍ത്തെഴുന്നേല്‍പിക്കും. നിങ്ങള്‍ കാര്യബോധമുള്ളവരായേക്കാം.
\end{malayalam}}
\flushright{\begin{Arabic}
\quranayah[7][58]
\end{Arabic}}
\flushleft{\begin{malayalam}
നല്ല പ്രദേശത്തെ സസ്യങ്ങള്‍ അതിന്റെ നാഥന്റെ അനുമതിയോടെ കിളുര്‍ത്തുവരുന്നു. എന്നാല്‍ ചീത്തമണ്ണില്‍ വളരെക്കുറച്ചല്ലാതെ സസ്യങ്ങള്‍ മുളച്ചുവരില്ല. ഇവ്വിധം നന്ദിയുള്ള ജനത്തിന് നാം പ്രമാണങ്ങള്‍ പലവിധം വിവരിച്ചുകൊടുക്കുന്നു.
\end{malayalam}}
\flushright{\begin{Arabic}
\quranayah[7][59]
\end{Arabic}}
\flushleft{\begin{malayalam}
നൂഹിനെ നാം അദ്ദേഹത്തിന്റെ ജനതയിലേക്കയച്ചു. അപ്പോള്‍ അദ്ദേഹം പറഞ്ഞു: "എന്റെ ജനമേ, നിങ്ങള്‍ അല്ലാഹുവിന് വഴിപ്പെടുക. അവനല്ലാതെ നിങ്ങള്‍ക്കു ദൈവമില്ല. ഒരു ഭീകര നാളിലെ കൊടിയ ശിക്ഷ നിങ്ങളെ ബാധിക്കുമോയെന്ന് ഞാന്‍ ഭയപ്പെടുന്നു.”
\end{malayalam}}
\flushright{\begin{Arabic}
\quranayah[7][60]
\end{Arabic}}
\flushleft{\begin{malayalam}
അദ്ദേഹത്തിന്റെ ജനതയിലെ പ്രമാണിമാര്‍ പറഞ്ഞു: "നീ വ്യക്തമായ വഴികേടിലകപ്പെട്ടതായി ഞങ്ങള്‍ കാണുന്നു.”
\end{malayalam}}
\flushright{\begin{Arabic}
\quranayah[7][61]
\end{Arabic}}
\flushleft{\begin{malayalam}
അദ്ദേഹം പറഞ്ഞു: "എന്റെ ജനമേ, എന്നില്‍ വഴികേടൊന്നുമില്ല. ഞാന്‍ പ്രപഞ്ചനാഥന്റെ ദൂതനാകുന്നു.
\end{malayalam}}
\flushright{\begin{Arabic}
\quranayah[7][62]
\end{Arabic}}
\flushleft{\begin{malayalam}
"ഞാനെന്റെ നാഥന്റെ സന്ദേശങ്ങള്‍ നിങ്ങള്‍ക്കെത്തിച്ചുതരുന്നു. നിങ്ങളുടെ നന്മമാത്രം കൊതിക്കുന്നു. നിങ്ങള്‍ക്കറിഞ്ഞുകൂടാത്ത പലതും അല്ലാഹുവില്‍ നിന്ന് ഞാനറിയുന്നു.
\end{malayalam}}
\flushright{\begin{Arabic}
\quranayah[7][63]
\end{Arabic}}
\flushleft{\begin{malayalam}
"നിങ്ങള്‍ക്ക് മുന്നറിയിപ്പു നല്‍കാനും നിങ്ങള്‍ ഭക്തിയുള്ളവരാകാനും നിങ്ങള്‍ക്ക് കാരുണ്യം കിട്ടാനുമായി നിങ്ങളുടെ നാഥനില്‍ നിന്നുള്ള ഉദ്ബോധനം നിങ്ങളില്‍ നിന്നു തന്നെയുള്ള ഒരാളിലൂടെ വന്നെത്തിയതില്‍ നിങ്ങള്‍ അത്ഭുതപ്പെടുകയോ?”
\end{malayalam}}
\flushright{\begin{Arabic}
\quranayah[7][64]
\end{Arabic}}
\flushleft{\begin{malayalam}
എന്നിട്ടും അവരദ്ദേഹത്തെ തള്ളിപ്പറഞ്ഞു. അപ്പോള്‍ നാം അദ്ദേഹത്തെയും കൂടെയുള്ളവരെയും കപ്പലില്‍ രക്ഷപ്പെടുത്തി. നമ്മുടെ പ്രമാണങ്ങളെ കള്ളമാക്കി തള്ളിപ്പറഞ്ഞവരെ മുക്കിക്കൊല്ലുകയും ചെയ്തു. തീര്‍ച്ചയായും അവര്‍ ഉള്‍ക്കാഴ്ചയില്ലാത്ത ജനമായിരുന്നു.
\end{malayalam}}
\flushright{\begin{Arabic}
\quranayah[7][65]
\end{Arabic}}
\flushleft{\begin{malayalam}
ആദ്സമുദായത്തിലേക്ക് നാം അവരുടെ സഹോദരനായ ഹൂദിനെ അയച്ചു. അദ്ദേഹം പറഞ്ഞു: "എന്റെ ജനമേ, നിങ്ങള്‍ അല്ലാഹുവിന് വഴിപ്പെടുക. അവനല്ലാതെ നിങ്ങള്‍ക്ക് ദൈവമില്ല. നിങ്ങള്‍ സൂക്ഷ്മതയുള്ളവരാവുന്നില്ലേ?”
\end{malayalam}}
\flushright{\begin{Arabic}
\quranayah[7][66]
\end{Arabic}}
\flushleft{\begin{malayalam}
അദ്ദേഹത്തിന്റെ ജനതയിലെ സത്യനിഷേധികളായ പ്രമാണിമാര്‍ പറഞ്ഞു: "നീ വിഡ്ഢിത്തത്തിലകപ്പെട്ടതായി ഞങ്ങള്‍ കാണുന്നു. നീ കള്ളം പറയുന്നവന്‍ തന്നെയാണെന്ന് ഞങ്ങള്‍ കരുതുന്നു.”
\end{malayalam}}
\flushright{\begin{Arabic}
\quranayah[7][67]
\end{Arabic}}
\flushleft{\begin{malayalam}
അദ്ദേഹം പറഞ്ഞു: "എന്റെ ജനമേ, ഒരു വിഡ്ഢിത്തവും എനിക്കില്ല. എന്നാല്‍ ഓര്‍ക്കുക: ഞാന്‍ പ്രപഞ്ചനാഥന്റെ ദൂതനാണ്.
\end{malayalam}}
\flushright{\begin{Arabic}
\quranayah[7][68]
\end{Arabic}}
\flushleft{\begin{malayalam}
"ഞാനെന്റെ നാഥന്റെ സന്ദേശങ്ങള്‍ നിങ്ങള്‍ക്കെത്തിച്ചുതരുന്നു. നിങ്ങള്‍ക്ക് വിശ്വസിക്കാവുന്ന നിങ്ങളുടെ ഗുണകാംക്ഷിയാണ് ഞാന്‍.
\end{malayalam}}
\flushright{\begin{Arabic}
\quranayah[7][69]
\end{Arabic}}
\flushleft{\begin{malayalam}
"നിങ്ങള്‍ക്കു മുന്നറിയിപ്പു നല്‍കാന്‍ നിങ്ങളുടെ നാഥനില്‍ നിന്നുള്ള ഉദ്ബോധനം നിങ്ങളില്‍ നിന്നു തന്നെയുള്ള ഒരാളിലൂടെ വന്നെത്തിയതില്‍ നിങ്ങള്‍ അദ്ഭുതപ്പെടുകയോ? നൂഹിന്റെ ജനതക്കുശേഷം അവന്‍ നിങ്ങളെ പ്രതിനിധികളാക്കിയതോര്‍ക്കുക. നിങ്ങള്‍ക്ക് പ്രകൃത്യാ തന്നെ കായികശേഷി പോഷിപ്പിച്ചുതന്നതും. അല്ലാഹുവിന്റെ അനുഗ്രഹങ്ങള്‍ അനുസ്മരിക്കുക. നിങ്ങള്‍ വിജയം വരിച്ചേക്കാം.”
\end{malayalam}}
\flushright{\begin{Arabic}
\quranayah[7][70]
\end{Arabic}}
\flushleft{\begin{malayalam}
അവര്‍ പറഞ്ഞു: "ഞങ്ങള്‍ ഏകദൈവത്തെ മാത്രം ആരാധിക്കാനും ഞങ്ങളുടെ പൂര്‍വപിതാക്കള്‍ പൂജിച്ചിരുന്നവയെയൊക്കെ വെടിയാനും വേണ്ടിയാണോ നീ ഞങ്ങളുടെ അടുത്തേക്ക് വന്നത്? എങ്കില്‍, നീ ഞങ്ങളെ ഭീഷണിപ്പെടുത്തിക്കൊണ്ടിരിക്കുന്ന ആ ശിക്ഷയിങ്ങ് കൊണ്ടുവരിക; നീ സത്യവാനാണെങ്കില്‍.”
\end{malayalam}}
\flushright{\begin{Arabic}
\quranayah[7][71]
\end{Arabic}}
\flushleft{\begin{malayalam}
ഹൂദ് പറഞ്ഞു: "നിങ്ങളുടെ നാഥനില്‍ നിന്നുള്ള ശാപകോപങ്ങളെല്ലാം നിങ്ങളില്‍ വന്നുപതിച്ചിരിക്കുന്നു. അല്ലാഹു ഒരു തെളിവും തരാതിരിക്കെ, നിങ്ങളും നിങ്ങളുടെ പൂര്‍വപിതാക്കളും കെട്ടിച്ചമച്ച കുറേ ദൈവങ്ങളുടെ പേരും പറഞ്ഞ് എന്നോട് തര്‍ക്കിക്കുകയാണോ? എങ്കില്‍ നിങ്ങള്‍ കാത്തിരിക്കുക. ഉറപ്പായും ഞാനും നിങ്ങളോടൊപ്പം കാത്തിരിക്കാം.”
\end{malayalam}}
\flushright{\begin{Arabic}
\quranayah[7][72]
\end{Arabic}}
\flushleft{\begin{malayalam}
അങ്ങനെ നാം നമ്മുടെ കാരുണ്യത്താല്‍ ഹൂദിനേയും അദ്ദേഹത്തിന്റെ കൂടെയുള്ളവരെയും രക്ഷപ്പെടുത്തി. നമ്മുടെ പ്രമാണങ്ങളെ കള്ളമാക്കിത്തള്ളിയവരെ മുരടോടെ മുറിച്ചുമാറ്റുകയും ചെയ്തു. അവര്‍ സത്യവിശ്വാസികളായിരുന്നില്ല.
\end{malayalam}}
\flushright{\begin{Arabic}
\quranayah[7][73]
\end{Arabic}}
\flushleft{\begin{malayalam}
സമൂദ്സമുദായത്തിലേക്ക് നാം അവരുടെ സഹോദരന്‍ സ്വാലിഹിനെ അയച്ചു. അദ്ദേഹം പറഞ്ഞു: "എന്റെ ജനമേ, നിങ്ങള്‍ അല്ലാഹുവിനു വഴിപ്പെടുക. അവനല്ലാതെ നിങ്ങള്‍ക്ക് ദൈവമില്ല. നിങ്ങളുടെ നാഥനില്‍ നിന്നുള്ള വ്യക്തമായ തെളിവ് നിങ്ങള്‍ക്ക് വന്നെത്തിയിട്ടുണ്ട്. അല്ലാഹുവിന്റെ ഈ ഒട്ടകം നിങ്ങള്‍ക്കുള്ള ദൃഷ്ടാന്തമാണ്. അതിനാല്‍ അതിനെ വിട്ടേക്കുക. അത് അല്ലാഹുവിന്റെ ഭൂമിയില്‍ തിന്നുനടക്കട്ടെ. നിങ്ങളതിന് ഒരു ദ്രോഹവും വരുത്തരുത്. അങ്ങനെ ചെയ്താല്‍ നോവേറിയ ശിക്ഷ നിങ്ങളെ പിടികൂടും.
\end{malayalam}}
\flushright{\begin{Arabic}
\quranayah[7][74]
\end{Arabic}}
\flushleft{\begin{malayalam}
"ആദ് സമുദായത്തിനു ശേഷം അവന്‍ നിങ്ങളെ തന്റെ പ്രതിനിധികളാക്കിയതും ഭൂമിയില്‍ താമസ സൌകര്യമൊരുക്കിത്തന്നതും ഓര്‍ക്കുക. നിങ്ങള്‍ അതിലെ സമതലങ്ങളില്‍ കൊട്ടാരങ്ങള്‍ ഉണ്ടാക്കുന്നു. മല തുരന്നു വീടുണ്ടാക്കുന്നു. അതിനാല്‍ നിങ്ങള്‍ അല്ലാഹുവിന്റെ അനുഗ്രഹങ്ങള്‍ ഓര്‍ക്കുക. കുഴപ്പക്കാരായി ഭൂമിയില്‍ നാശമുണ്ടാക്കരുത്.”
\end{malayalam}}
\flushright{\begin{Arabic}
\quranayah[7][75]
\end{Arabic}}
\flushleft{\begin{malayalam}
അദ്ദേഹത്തിന്റെ ജനതയിലെ അഹങ്കാരികളായ പ്രമാണിമാര്‍ അവരിലെ ദുര്‍ബലരോട് അഥവാ അവരിലെ വിശ്വസിച്ചവരോട് ചോദിച്ചു: "സ്വാലിഹ് തന്റെ നാഥന്‍ നിയോഗിച്ച ദൂതന്‍ തന്നെയാണെന്ന് നിങ്ങള്‍ക്കുറപ്പുണ്ടോ?” അവര്‍ അറിയിച്ചു: "അദ്ദേഹം ഞങ്ങളിലേക്കു വന്നത് ഏതൊരു സന്ദേശവുമായാണോ അതില്‍ വിശ്വസിച്ചവരാണ് ഞങ്ങള്‍; സംശയമില്ല.”
\end{malayalam}}
\flushright{\begin{Arabic}
\quranayah[7][76]
\end{Arabic}}
\flushleft{\begin{malayalam}
ആ അഹങ്കാരികള്‍ പറഞ്ഞു: "നിങ്ങള്‍ വിശ്വസിക്കുന്നതെന്തോ അതിനെ നിഷേധിക്കുന്നവരാണ് ഞങ്ങള്‍.”
\end{malayalam}}
\flushright{\begin{Arabic}
\quranayah[7][77]
\end{Arabic}}
\flushleft{\begin{malayalam}
അങ്ങനെ അവര്‍ ആ ഒട്ടകത്തെ അറുത്തു. തങ്ങളുടെ നാഥന്റെ കല്‍പനയെ ധിക്കരിച്ചു. അവര്‍ പറഞ്ഞു: "സ്വാലിഹേ, നീ ഞങ്ങളെ ഭീഷണിപ്പെടുത്തിക്കൊണ്ടിരിക്കുന്ന ആ ശിക്ഷയിങ്ങ് കൊണ്ടുവരിക. നീ ദൈവദൂതനെങ്കില്‍!”
\end{malayalam}}
\flushright{\begin{Arabic}
\quranayah[7][78]
\end{Arabic}}
\flushleft{\begin{malayalam}
പെട്ടെന്നൊരു പ്രകമ്പനം അവരെ പിടികൂടി. അങ്ങനെ പ്രഭാതത്തില്‍ അവര്‍ തങ്ങളുടെ വീടുകളില്‍ മരിച്ചുവീണവരായി കാണപ്പെട്ടു.
\end{malayalam}}
\flushright{\begin{Arabic}
\quranayah[7][79]
\end{Arabic}}
\flushleft{\begin{malayalam}
സ്വാലിഹ് അവരെ വിട്ടുപോയി. അദ്ദേഹം ഇങ്ങനെ പറഞ്ഞു: "എന്റെ ജനമേ, ഞാനെന്റെ നാഥന്റെ സന്ദേശം നിങ്ങള്‍ക്കെത്തിച്ചു തന്നു. നിങ്ങള്‍ക്കു നന്മ വരട്ടെയെന്നാഗ്രഹിച്ചു. പക്ഷേ, ഗുണകാംക്ഷികളെ നിങ്ങള്‍ ഇഷ്ടപ്പെടുന്നില്ല.”
\end{malayalam}}
\flushright{\begin{Arabic}
\quranayah[7][80]
\end{Arabic}}
\flushleft{\begin{malayalam}
ലൂത്തിനെയും നാം നിയോഗിച്ചു. അദ്ദേഹം തന്റെ ജനത്തോട് പറഞ്ഞതോര്‍ക്കുക: "നിങ്ങള്‍ക്കു മുമ്പ് ലോകരിലാരും ചെയ്തിട്ടില്ലാത്ത നീചവൃത്തി യിലാണോ നിങ്ങളേര്‍പ്പെട്ടിരിക്കുന്നത്?
\end{malayalam}}
\flushright{\begin{Arabic}
\quranayah[7][81]
\end{Arabic}}
\flushleft{\begin{malayalam}
"നിങ്ങള്‍ സ്ത്രീകളെ ഒഴിവാക്കി ഭോഗേച്ഛയോടെ പുരുഷന്മാരെ സമീപിക്കുന്നു. അല്ല; നിങ്ങള്‍ കൊടിയ അതിക്രമികള്‍ തന്നെ.”
\end{malayalam}}
\flushright{\begin{Arabic}
\quranayah[7][82]
\end{Arabic}}
\flushleft{\begin{malayalam}
എന്നാല്‍ അദ്ദേഹത്തിന്റെ ജനത്തിന്റെ മറുപടി ഇത്രമാത്രമായിരുന്നു: "ഇവരെ നിങ്ങളുടെ നാട്ടില്‍ നിന്ന് പുറത്താക്കുക. ഇവര്‍ വല്ലാത്ത വിശുദ്ധന്മാര്‍ തന്നെ!”
\end{malayalam}}
\flushright{\begin{Arabic}
\quranayah[7][83]
\end{Arabic}}
\flushleft{\begin{malayalam}
അപ്പോള്‍ ലൂത്തിനെയും കുടുംബത്തേയും നാം രക്ഷപ്പെടുത്തി. അദ്ദേഹത്തിന്റെ ഭാര്യയെ ഒഴികെ. അവള്‍ പിന്മാറിനിന്നവരില്‍ പെട്ടവളായിരുന്നു.
\end{malayalam}}
\flushright{\begin{Arabic}
\quranayah[7][84]
\end{Arabic}}
\flushleft{\begin{malayalam}
നാം ആ ജനതക്കുമേല്‍ പേമാരി പെയ്യിച്ചു. നോക്കൂ: എവ്വിധമായിരുന്നു ആ പാപികളുടെ പരിണതിയെന്ന്!
\end{malayalam}}
\flushright{\begin{Arabic}
\quranayah[7][85]
\end{Arabic}}
\flushleft{\begin{malayalam}
മദ്യന്‍ ജനതയിലേക്ക് അവരുടെ സഹോദരന്‍ ശുഐബിനെ നാം നിയോഗിച്ചു. അദ്ദേഹം പറഞ്ഞു: "എന്റെ ജനമേ, നിങ്ങള്‍ അല്ലാഹുവിന് വഴിപ്പെടുക. നിങ്ങള്‍ക്ക് അവനല്ലാതെ ദൈവമില്ല. നിങ്ങള്‍ക്ക് നിങ്ങളുടെ നാഥനില്‍ നിന്ന് വ്യക്തമായ തെളിവ് വന്നെത്തിയിട്ടുണ്ട്. അതിനാല്‍ നിങ്ങള്‍ അളത്തത്തിലും തൂക്കത്തിലും കൃത്യത പാലിക്കുക. ജനങ്ങള്‍ക്ക് അവരുടെ സാധനങ്ങളില്‍ കുറവ് വരുത്തരുത്. ഭൂമിയെ യഥാവിധി ചിട്ടപ്പെടുത്തിവെച്ചിരിക്കെ നിങ്ങളതില്‍ നാശമുണ്ടാക്കരുത്. നിങ്ങള്‍ സത്യവിശ്വാസികളെങ്കില്‍ അതാണ് നിങ്ങള്‍ക്കുത്തമം.”
\end{malayalam}}
\flushright{\begin{Arabic}
\quranayah[7][86]
\end{Arabic}}
\flushleft{\begin{malayalam}
ജനങ്ങളെ ഭീഷണിപ്പെടുത്തുന്നവരായും അല്ലാഹുവിന്റെ മാര്‍ഗത്തില്‍നിന്ന് വിശ്വാസികളെ തടയുന്നവരായും ആ മാര്‍ഗ്ഗത്തെ വക്രമാക്കാന്‍ ശ്രമിക്കുന്നവരായും പാതവക്കിലൊക്കെയും നിങ്ങള്‍ ഇരിക്കരുത്. നിങ്ങള്‍ എണ്ണത്തില്‍ കുറവായിരുന്ന കാലത്തെക്കുറിച്ച് ഒന്നോര്‍ത്തുനോക്കൂ. പിന്നീട് അല്ലാഹു നിങ്ങളെ പെരുപ്പിച്ചു. നോക്കൂ; നാശകാരികളുടെ അന്ത്യം എവ്വിധമായിരുന്നുവെന്ന്.
\end{malayalam}}
\flushright{\begin{Arabic}
\quranayah[7][87]
\end{Arabic}}
\flushleft{\begin{malayalam}
ഏതൊരു സന്ദേശവുമായാണോ ഞാന്‍ നിയോഗിതനായിരിക്കുന്നത് അതില്‍ നിങ്ങളിലൊരു വിഭാഗം വിശ്വസിക്കുകയും മറ്റൊരു വിഭാഗം അവിശ്വസിക്കുകയുമാണെങ്കില്‍ അല്ലാഹു നമുക്കിടയില്‍ തീര്‍പ്പ് കല്‍പിക്കുംവരെ ക്ഷമിക്കുക. തീരുമാനമെടുക്കുന്നവരില്‍ അത്യുത്തമന്‍ അവന്‍ തന്നെ.
\end{malayalam}}
\flushright{\begin{Arabic}
\quranayah[7][88]
\end{Arabic}}
\flushleft{\begin{malayalam}
അദ്ദേഹത്തിന്റെ ജനതയിലെ അഹങ്കാരികളായ പ്രമാണിമാര്‍ പറഞ്ഞു: "ശുഐബേ, നിന്നെയും നിന്നോടൊപ്പമുള്ള വിശ്വാസികളെയും ഞങ്ങള്‍ ഞങ്ങളുടെ നാട്ടില്‍നിന്ന് പുറത്താക്കും; ഉറപ്പ്. അല്ലെങ്കില്‍ നിങ്ങള്‍ ഞങ്ങളുടെ മതത്തിലേക്ക് തിരിച്ചുവരിക തന്നെ വേണം.” അദ്ദേഹം ചോദിച്ചു: "ഞങ്ങള്‍ക്കത് ഇഷ്ടമില്ലെങ്കിലും?
\end{malayalam}}
\flushright{\begin{Arabic}
\quranayah[7][89]
\end{Arabic}}
\flushleft{\begin{malayalam}
"അല്ലാഹു ഞങ്ങളെ നിങ്ങളുടെ മതത്തില്‍ നിന്ന് രക്ഷപ്പെടുത്തി.അതിലേക്കു തന്നെ തിരിച്ചു വരികയാണെങ്കില്‍ തീര്‍ച്ചയായും ഞങ്ങള്‍ അല്ലാഹുവിന്റെ പേരില്‍ കള്ളം കെട്ടിച്ചമച്ചവരായിത്തീരും. ഞങ്ങള്‍ക്ക് ഇനി ഒരിക്കലും അതിലേക്കു തിരിച്ചുവരാനാവില്ല; ഞങ്ങളുടെ നാഥനായ അല്ലാഹു ഉദ്ദേശിച്ചാലല്ലാതെ. ഞങ്ങളുടെ നാഥനായ അല്ലാഹു സകല സംഗതികളെ സംബന്ധിച്ചും വിപുലമായ അറിവുള്ളവനാണ്. അല്ലാഹുവിലാണ് ഞങ്ങള്‍ ഭരമേല്‍പിച്ചിരിക്കുന്നത്. നാഥാ! ഞങ്ങള്‍ക്കും ഞങ്ങളുടെ ജനത്തിനുമിടയില്‍ നീ ന്യായമായ തീരുമാനമെടുക്കേണമേ. തീരുമാനമെടുക്കുന്നവരില്‍ ഏറ്റം ഉത്തമന്‍ നീയാണല്ലോ.”
\end{malayalam}}
\flushright{\begin{Arabic}
\quranayah[7][90]
\end{Arabic}}
\flushleft{\begin{malayalam}
അദ്ദേഹത്തിന്റെ ജനതയിലെ സത്യനിഷേധികളായ പ്രമാണിമാര്‍ പറഞ്ഞു: "നിങ്ങള്‍ ശുഐബിനെ പിന്‍പറ്റിയാല്‍ ഉറപ്പായും നിങ്ങള്‍ നഷ്ടം പറ്റിയവരായിത്തീരും.”
\end{malayalam}}
\flushright{\begin{Arabic}
\quranayah[7][91]
\end{Arabic}}
\flushleft{\begin{malayalam}
അപ്പോള്‍ ഘോരമായ പ്രകമ്പനം അവരെ പിടികൂടി. അങ്ങനെ പ്രഭാതത്തില്‍ അവര്‍ തങ്ങളുടെ വീടുകളില്‍ ചേതനയറ്റ് കമിഴ്ന്നു വീണുകിടക്കുന്നവരായിത്തീര്‍ന്നു.
\end{malayalam}}
\flushright{\begin{Arabic}
\quranayah[7][92]
\end{Arabic}}
\flushleft{\begin{malayalam}
ശുഐബിനെ തള്ളിപ്പറഞ്ഞവരുടെ അവസ്ഥ അവരവിടെ പാര്‍ത്തിട്ടുപോലുമില്ലാത്തവിധം ആയിത്തീര്‍ന്നു. ശുഐബിനെ തള്ളിപ്പറഞ്ഞവര്‍ തന്നെയാണ് നഷ്ടം പറ്റിയവര്‍.
\end{malayalam}}
\flushright{\begin{Arabic}
\quranayah[7][93]
\end{Arabic}}
\flushleft{\begin{malayalam}
ശുഐബ് അവരെ വിട്ടുപോയി. അന്നേരം അദ്ദേഹം പറഞ്ഞു: "എന്റെ ജനമേ, ഞാനെന്റെ നാഥന്റെ സന്ദേശങ്ങള്‍ നിങ്ങള്‍ക്കെത്തിച്ചുതന്നു. നിങ്ങളോടു ഗുണകാംക്ഷ പുലര്‍ത്തി. അതിനാല്‍ സത്യനിഷേധികളായ ജനത്തിന്റെ പേരില്‍ എനിക്കെങ്ങനെ ഖേദമുണ്ടാകും?”
\end{malayalam}}
\flushright{\begin{Arabic}
\quranayah[7][94]
\end{Arabic}}
\flushleft{\begin{malayalam}
നാം പ്രവാചകനെ നിയോഗിച്ച ഒരു നാട്ടിലെയും നിവാസികളെ പ്രയാസവും പ്രതിസന്ധിയും കൊണ്ട് പിടികൂടാതിരുന്നിട്ടില്ല. അവര്‍ വിനീതരാവാന്‍ വേണ്ടിയാണത്.
\end{malayalam}}
\flushright{\begin{Arabic}
\quranayah[7][95]
\end{Arabic}}
\flushleft{\begin{malayalam}
പിന്നീട് നാം അവരുടെ ദുഃസ്ഥിതി സുസ്ഥിതിയാക്കി മാറ്റി. അവര്‍ അഭിവൃദ്ധിപ്പെടുവോളം. അങ്ങനെ അവര്‍ പറഞ്ഞു: "ഞങ്ങളുടെ പൂര്‍വപിതാക്കള്‍ക്കും ദുരിതവും സന്തോഷവുമൊക്കെ ഉണ്ടായിട്ടുണ്ടല്ലോ.” അപ്പോള്‍ പെട്ടെന്ന് നാം അവരെ പിടികൂടി. അവര്‍ക്ക് അതേക്കുറിച്ച് ബോധമുണ്ടായിരുന്നില്ല.
\end{malayalam}}
\flushright{\begin{Arabic}
\quranayah[7][96]
\end{Arabic}}
\flushleft{\begin{malayalam}
അന്നാട്ടുകാര്‍ വിശ്വസിക്കുകയും ഭക്തരാവുകയും ചെയ്തിരുന്നെങ്കില്‍ നാമവര്‍ക്ക് വിണ്ണില്‍നിന്നും മണ്ണില്‍നിന്നും അനുഗ്രഹങ്ങളുടെ കവാടങ്ങള്‍ തുറന്നുകൊടുക്കുമായിരുന്നു. എന്നാല്‍ അവര്‍ നിഷേധിച്ചുതള്ളുകയാണുണ്ടായത്. അതിനാല്‍ അവര്‍ സമ്പാദിച്ചുവെച്ചതിന്റെ ഫലമായി നാം അവരെ പിടികൂടി.
\end{malayalam}}
\flushright{\begin{Arabic}
\quranayah[7][97]
\end{Arabic}}
\flushleft{\begin{malayalam}
എന്നാല്‍ അന്നാട്ടുകാര്‍, രാത്രിയില്‍ അവര്‍ ഉറക്കിലായിരിക്കെ നമ്മുടെ ശിക്ഷ വന്നെത്തുന്നതിനെപ്പറ്റി നിര്‍ഭയരായിപ്പോയോ?
\end{malayalam}}
\flushright{\begin{Arabic}
\quranayah[7][98]
\end{Arabic}}
\flushleft{\begin{malayalam}
അല്ലെങ്കില്‍, അവിടത്തെയാളുകള്‍ പകല്‍വേളയില്‍ വിനോദ വൃത്തികളിലായിരിക്കെ, നമ്മുടെ ശിക്ഷ വന്നെത്തുന്നതിനെപ്പറ്റി നിര്‍ഭയരായിരിക്കയാണോ?
\end{malayalam}}
\flushright{\begin{Arabic}
\quranayah[7][99]
\end{Arabic}}
\flushleft{\begin{malayalam}
അങ്ങനെ അവര്‍ അല്ലാഹുവിന്റെ തന്ത്രത്തെപ്പറ്റിത്തന്നെ നിര്‍ഭയരായിരിക്കയാണോ? എന്നാല്‍ അറിയുക: നശിച്ച ജനമല്ലാതെ അല്ലാഹുവിന്റെ തന്ത്രത്തെപ്പറ്റി നിര്‍ഭയരാവുകയില്ല.
\end{malayalam}}
\flushright{\begin{Arabic}
\quranayah[7][100]
\end{Arabic}}
\flushleft{\begin{malayalam}
നേരത്തെ ഭൂമിയില്‍ വസിച്ചിരുന്നവര്‍ക്കുശേഷം അതില്‍ അനന്തരാവകാശികളായി വന്നവര്‍ മനസ്സിലാക്കുന്നില്ലയോ, നാം ഇച്ഛിക്കുന്നുവെങ്കില്‍ അവരെയും തങ്ങളുടെ പാപങ്ങളുടെ പേരില്‍ നമ്മുടെ ശിക്ഷ ബാധിക്കുമെന്ന്. നാം അവരുടെ മനസ്സുകള്‍ അടച്ചുപൂട്ടി മുദ്രവെക്കും. അതോടെ അവരൊന്നും കേട്ടു മനസ്സിലാക്കാത്തവരായിത്തീരും.
\end{malayalam}}
\flushright{\begin{Arabic}
\quranayah[7][101]
\end{Arabic}}
\flushleft{\begin{malayalam}
ആ നാടുകളെ സംബന്ധിച്ച ചില വിവരങ്ങള്‍ നാം നിനക്ക് പറഞ്ഞു തരികയാണ്: അവരിലേക്കുള്ള ദൈവദൂതന്മാര്‍ വ്യക്തമായ തെളിവുകളുമായി അവരുടെ അടുത്തു വന്നു. എന്നിട്ടും അവര്‍ നേരത്തെ നിഷേധിച്ചു തള്ളിയതില്‍ വിശ്വസിക്കാന്‍ തയ്യാറായില്ല. ഇവ്വിധം നാം സത്യനിഷേധികളുടെ മനസ്സുകള്‍ക്ക് മുദ്രവെക്കും.
\end{malayalam}}
\flushright{\begin{Arabic}
\quranayah[7][102]
\end{Arabic}}
\flushleft{\begin{malayalam}
അവരിലേറെ പേരെയും കരാര്‍ പാലിക്കുന്നവരായി നാം കണ്ടില്ല. അവരിലേറെ പേരെയും അധര്‍മികളായാണ് നാം കണ്ടത്.
\end{malayalam}}
\flushright{\begin{Arabic}
\quranayah[7][103]
\end{Arabic}}
\flushleft{\begin{malayalam}
പിന്നീട് അവരുടെയൊക്കെ ശേഷം മൂസായെ നാം നമ്മുടെ തെളിവുകളുമായി ഫറവോന്റെയും അവന്റെ പ്രമാണിമാരുടെയും അടുത്തേക്കയച്ചു. അവരും നമ്മുടെ തെളിവുകളോട് അനീതി ചെയ്തു. നോക്കൂ! ആ നാശകാരികളുടെ ഒടുക്കം എവ്വിധമായിരുന്നുവെന്ന്.
\end{malayalam}}
\flushright{\begin{Arabic}
\quranayah[7][104]
\end{Arabic}}
\flushleft{\begin{malayalam}
മൂസാ പറഞ്ഞു: "ഫിര്‍ഔന്‍, ഉറപ്പായും ഞാന്‍ പ്രപഞ്ചനാഥനില്‍ നിന്നുള്ള ദൂതനാണ്.
\end{malayalam}}
\flushright{\begin{Arabic}
\quranayah[7][105]
\end{Arabic}}
\flushleft{\begin{malayalam}
"അല്ലാഹുവിന്റെ പേരില്‍ സത്യമല്ലാത്തതൊന്നും പറയാതിരിക്കാന്‍ ഞാന്‍ ബാധ്യസ്ഥനാണ്. നിങ്ങളുടെ നാഥനില്‍ നിന്നുള്ള വ്യക്തമായ തെളിവുമായാണ് ഞാന്‍ നിങ്ങളുടെ അടുത്തു വന്നിരിക്കുന്നത്. അതിനാല്‍ ഇസ്രയേല്‍ മക്കളെ എന്നോടൊപ്പം അയക്കുക.”
\end{malayalam}}
\flushright{\begin{Arabic}
\quranayah[7][106]
\end{Arabic}}
\flushleft{\begin{malayalam}
ഫറവോന്‍ പറഞ്ഞു: "നീ തെളിവുമായാണ് വന്നതെങ്കില്‍ അതിങ്ങു കൊണ്ടുവാ; നീ സത്യവാനെങ്കില്‍!”
\end{malayalam}}
\flushright{\begin{Arabic}
\quranayah[7][107]
\end{Arabic}}
\flushleft{\begin{malayalam}
അപ്പോള്‍ മൂസാ തന്റെ വടി നിലത്തിട്ടു. ഉടനെ അത് പൂര്‍ണാര്‍ഥത്തില്‍ ഒരു പാമ്പായി മാറി.
\end{malayalam}}
\flushright{\begin{Arabic}
\quranayah[7][108]
\end{Arabic}}
\flushleft{\begin{malayalam}
അദ്ദേഹം തന്റെ കൈ പുറത്തെടുത്തു. അപ്പോഴത് കാണുന്നവര്‍ക്കൊക്കെ വെളുത്തു തിളങ്ങുന്നതായിത്തീര്‍ന്നു.
\end{malayalam}}
\flushright{\begin{Arabic}
\quranayah[7][109]
\end{Arabic}}
\flushleft{\begin{malayalam}
ഫറവോന്റെ ജനതയിലെ പ്രമാണിമാര്‍ പറഞ്ഞു: "സംശയമില്ല; ഇവനൊരു പഠിച്ച മായാജാലക്കാരന്‍ തന്നെ.”
\end{malayalam}}
\flushright{\begin{Arabic}
\quranayah[7][110]
\end{Arabic}}
\flushleft{\begin{malayalam}
"നിങ്ങളെ നിങ്ങളുടെ നാട്ടില്‍ നിന്നു പുറത്താക്കാനാണ് ഇവനാഗ്രഹിക്കുന്നത്. അതിനാല്‍ നിങ്ങള്‍ക്കെന്താണ് നിര്‍ദേശിക്കാനുള്ളത്?”
\end{malayalam}}
\flushright{\begin{Arabic}
\quranayah[7][111]
\end{Arabic}}
\flushleft{\begin{malayalam}
അവര്‍ പറഞ്ഞു: "ഇവനും ഇവന്റെ സഹോദരനും ഇവിടെ നില്‍ക്കട്ടെ. എന്നിട്ടു വിളിച്ചുകൂട്ടുവാന്‍ കഴിയുന്നവരെയെല്ലാം നഗരങ്ങളിലേക്കയക്കുക.
\end{malayalam}}
\flushright{\begin{Arabic}
\quranayah[7][112]
\end{Arabic}}
\flushleft{\begin{malayalam}
"അവര്‍ വിവരമുള്ള മായാജാലക്കാരെയൊക്കെ അങ്ങയുടെ സന്നിധിയില്‍ കൊണ്ടുവരട്ടെ.”
\end{malayalam}}
\flushright{\begin{Arabic}
\quranayah[7][113]
\end{Arabic}}
\flushleft{\begin{malayalam}
ജാലവിദ്യക്കാര്‍ ഫറവോന്റെ അടുത്തു വന്നു. അവര്‍ പറഞ്ഞു: " ഞങ്ങള്‍ വിജയിക്കുകയാണെങ്കില്‍ ഞങ്ങള്‍ക്ക് മികച്ച പ്രതിഫലമുണ്ടാകുമെന്ന് ഉറപ്പാണല്ലോ.”
\end{malayalam}}
\flushright{\begin{Arabic}
\quranayah[7][114]
\end{Arabic}}
\flushleft{\begin{malayalam}
ഫറവോന്‍ പറഞ്ഞു: "അതെ ഉറപ്പായും. അതോടൊപ്പം നിങ്ങള്‍ നമ്മുടെ സ്വന്തക്കാരായി മാറുകയും ചെയ്യും.”
\end{malayalam}}
\flushright{\begin{Arabic}
\quranayah[7][115]
\end{Arabic}}
\flushleft{\begin{malayalam}
മായാജാലക്കാര്‍ പറഞ്ഞു: "മൂസാ, ഒന്നുകില്‍ നീ ആദ്യം വടിയെറിയുക. അല്ലെങ്കില്‍ ഞങ്ങളെറിയാം.”
\end{malayalam}}
\flushright{\begin{Arabic}
\quranayah[7][116]
\end{Arabic}}
\flushleft{\begin{malayalam}
മൂസാ പറഞ്ഞു: "നിങ്ങള്‍ തന്നെ എറിഞ്ഞുകൊള്ളുക.” അവര്‍ വടിയെറിഞ്ഞു. അവര്‍ ആളുകളുടെ കണ്ണുകളെ മാരണം ചെയ്യുകയും അവരില്‍ ഭീതിയുണര്‍ത്തുകയും ചെയ്തു. ഗംഭീരമായ മായാജാലമാണ് അവര്‍ കാണിച്ചത്.
\end{malayalam}}
\flushright{\begin{Arabic}
\quranayah[7][117]
\end{Arabic}}
\flushleft{\begin{malayalam}
മൂസായോട് നാം നിര്‍ദേശിച്ചു: "നീ നിന്റെ വടിയെറിയുക.” അതൊരു പാമ്പായി അവരുടെ മായാജാലത്തെ മുഴുവന്‍ വിഴുങ്ങാന്‍ തുടങ്ങി.
\end{malayalam}}
\flushright{\begin{Arabic}
\quranayah[7][118]
\end{Arabic}}
\flushleft{\begin{malayalam}
അങ്ങനെ സത്യം സ്ഥാപിതമായി. അവര്‍ ചെയ്തുകൊണ്ടിരുന്നതെല്ലാം പാഴാവുകയും ചെയ്തു.
\end{malayalam}}
\flushright{\begin{Arabic}
\quranayah[7][119]
\end{Arabic}}
\flushleft{\begin{malayalam}
അവ്വിധം അവിടെ വെച്ചവര്‍ പരാജിതരായി. നന്നെ നിന്ദ്യരായിത്തീരുകയും ചെയ്തു.
\end{malayalam}}
\flushright{\begin{Arabic}
\quranayah[7][120]
\end{Arabic}}
\flushleft{\begin{malayalam}
അതോടെ ആ മായാജാലക്കാര്‍ സാഷ്ടാംഗംചെയ്തു വീണു.
\end{malayalam}}
\flushright{\begin{Arabic}
\quranayah[7][121]
\end{Arabic}}
\flushleft{\begin{malayalam}
അവര്‍ പറഞ്ഞു: "ഞങ്ങളിതാ പ്രപഞ്ചനാഥനില്‍ വിശ്വസിച്ചിരിക്കുന്നു.
\end{malayalam}}
\flushright{\begin{Arabic}
\quranayah[7][122]
\end{Arabic}}
\flushleft{\begin{malayalam}
"മൂസായുടെയും ഹാറൂന്റെയും നാഥനില്‍.”
\end{malayalam}}
\flushright{\begin{Arabic}
\quranayah[7][123]
\end{Arabic}}
\flushleft{\begin{malayalam}
ഫറവോന്‍ പറഞ്ഞു: "ഞാന്‍ അനുവാദം തരുംമുമ്പെ നിങ്ങളവനില്‍ വിശ്വസിക്കുകയോ? സംശയമില്ല; ഇതൊരു കൊടുംവഞ്ചന തന്നെ. ഇന്നാട്ടുകാരെ ഇവിടെ നിന്ന് പുറത്താക്കാനായി നിങ്ങളിവിടെ വെച്ചു നടത്തിയ ഗൂഢതന്ത്രമാണിത്. അതിനാല്‍ ഇതിന്റെ തിക്ത ഫലം നിങ്ങളിതാ അറിയാന്‍ പോകുന്നു.
\end{malayalam}}
\flushright{\begin{Arabic}
\quranayah[7][124]
\end{Arabic}}
\flushleft{\begin{malayalam}
"ഞാന്‍ നിങ്ങളുടെ കൈകാലുകള്‍ ഒന്നിനൊന്ന് വിപരീതമായി വെട്ടിമുറിക്കുക തന്നെ ചെയ്യും. പിന്നെ നിങ്ങളെയൊക്കെ ഞാന്‍ കുരിശിലേറ്റും; തീര്‍ച്ച.”
\end{malayalam}}
\flushright{\begin{Arabic}
\quranayah[7][125]
\end{Arabic}}
\flushleft{\begin{malayalam}
അവര്‍ പറഞ്ഞു: "ഉറപ്പായും ഞങ്ങളുടെ നാഥങ്കലേക്കാണ് ഞങ്ങള്‍ക്ക് മടങ്ങിച്ചെല്ലാനുള്ളത്.
\end{malayalam}}
\flushright{\begin{Arabic}
\quranayah[7][126]
\end{Arabic}}
\flushleft{\begin{malayalam}
"ഞങ്ങളുടെ നാഥന്റെ തെളിവുകള്‍ ഞങ്ങള്‍ക്ക് വന്നെത്തിയപ്പോള്‍ ഞങ്ങളതില്‍ വിശ്വസിച്ചു. അതിന്റെ പേരില്‍ മാത്രമാണല്ലോ താങ്കള്‍ പ്രതികാരത്തിനൊരുങ്ങുന്നത്. ഞങ്ങളുടെ നാഥാ; ഞങ്ങള്‍ക്കു നീ ക്ഷമ നല്‍കേണമേ! ഞങ്ങളെ നീ മുസ്ലിംകളായി മരിപ്പിക്കേണമേ!”
\end{malayalam}}
\flushright{\begin{Arabic}
\quranayah[7][127]
\end{Arabic}}
\flushleft{\begin{malayalam}
ഫറവോന്റെ ജനതയിലെ പ്രമാണിമാര്‍ പറഞ്ഞു: "നാട്ടില്‍ കുഴപ്പമുണ്ടാക്കാനും അങ്ങയെയും അങ്ങയുടെ ദൈവങ്ങളെയും തള്ളിപ്പറയാനും അങ്ങ് മൂസായെയും അവന്റെ ആള്‍ക്കാരെയും സ്വതന്ത്രമായി വിടുകയാണോ?” ഫറവോന്‍ പറഞ്ഞു: "നാം അവരുടെ ആണ്‍കുട്ടികളെ കൊന്നൊടുക്കും. സ്ത്രീകളെ മാത്രം ജീവിക്കാന്‍ വിടും. തീര്‍ച്ചയായും നാം അവരുടെ മേല്‍ മേധാവിത്വമുള്ളവരായിരിക്കും.”
\end{malayalam}}
\flushright{\begin{Arabic}
\quranayah[7][128]
\end{Arabic}}
\flushleft{\begin{malayalam}
മൂസാ തന്റെ ജനതയോടു പറഞ്ഞു: "നിങ്ങള്‍ അല്ലാഹുവോട് സഹായം തേടുക. എല്ലാം ക്ഷമിക്കുക. ഭൂമി അല്ലാഹുവിന്റേതാണ്. തന്റെ ദാസന്മാരില്‍ താനിച്ഛിക്കുന്നവരെ അവനതിന്റെ അവകാശികളാക്കും. അന്തിമ വിജയം ഭക്തന്മാര്‍ക്കാണ്.”
\end{malayalam}}
\flushright{\begin{Arabic}
\quranayah[7][129]
\end{Arabic}}
\flushleft{\begin{malayalam}
അവര്‍ പറഞ്ഞു: "താങ്കള്‍ ഞങ്ങളുടെ അടുത്ത് വരുന്നതിനുമുമ്പ് ഞങ്ങള്‍ പീഡിപ്പിക്കപ്പെട്ടുകൊണ്ടിരുന്നു. താങ്കള്‍ വന്നശേഷവും ഞങ്ങള്‍ പീഡിപ്പിക്കപ്പെടുകയാണല്ലോ.” മൂസാ പറഞ്ഞു: "നിങ്ങളുടെ നാഥന്‍ നിങ്ങളുടെ എതിരാളിയെ നശിപ്പിച്ചേക്കാം. അങ്ങനെ നിങ്ങളെ അവന്‍ ഭൂമിയില്‍ പ്രതിനിധികളാക്കുകയും ചെയ്തേക്കാം. അപ്പോള്‍ നിങ്ങള്‍ എങ്ങനെ പ്രവര്‍ത്തിക്കുന്നുവെന്ന് അവന്‍ നോക്കും.”
\end{malayalam}}
\flushright{\begin{Arabic}
\quranayah[7][130]
\end{Arabic}}
\flushleft{\begin{malayalam}
ഫറവോന്റെ ആള്‍ക്കാരെ കൊല്ലങ്ങളോളം നാം ക്ഷാമത്തിലും വിളക്കമ്മിയിലുമകപ്പെടുത്തി. അവര്‍ ബോധവാന്മാരാകുമോയെന്ന് നോക്കാന്‍.
\end{malayalam}}
\flushright{\begin{Arabic}
\quranayah[7][131]
\end{Arabic}}
\flushleft{\begin{malayalam}
അങ്ങനെ, വല്ല നന്മയും വന്നാല്‍ അവര്‍ പറയും: "ഇതു നാം അര്‍ഹിക്കുന്നതുതന്നെ.” വല്ല വിപത്തും ബാധിച്ചാല്‍ അതിനെ മൂസായുടെയും കൂടെയുള്ളവരുടെയും ദുശ്ശകുനമായി കാണുകയും ചെയ്യും. അറിയുക: അവരുടെ ശകുനം അല്ലാഹുവിന്റെ അടുക്കല്‍ തന്നെയാണ്. പക്ഷേ, അവരിലേറെപേരും അറിയുന്നില്ല.
\end{malayalam}}
\flushright{\begin{Arabic}
\quranayah[7][132]
\end{Arabic}}
\flushleft{\begin{malayalam}
അവര്‍ പറഞ്ഞു: "ഞങ്ങളെ മായാജാലത്തിലകപ്പെടുത്താനായി എന്ത് വിദ്യ കൊണ്ടുവന്നാലും ഞങ്ങള്‍ നിന്നില്‍ വിശ്വസിക്കുകയില്ല.”
\end{malayalam}}
\flushright{\begin{Arabic}
\quranayah[7][133]
\end{Arabic}}
\flushleft{\begin{malayalam}
അപ്പോള്‍ നാം അവരുടെ നേരെ വെള്ളപ്പൊക്കം, വെട്ടുകിളി, കീടങ്ങള്‍, തവളകള്‍, രക്തം എന്നീ വ്യക്തമായ ദൃഷ്ടാന്തങ്ങളയച്ചു. എന്നിട്ടും അവര്‍ അഹങ്കരിക്കുകയാണുണ്ടായത്. കുറ്റവാളികളായ ജനമായിരുന്നു അവര്‍.
\end{malayalam}}
\flushright{\begin{Arabic}
\quranayah[7][134]
\end{Arabic}}
\flushleft{\begin{malayalam}
അവര്‍ക്ക് വിപത്ത് വന്നുഭവിച്ചപ്പോള്‍ അവര്‍ പറഞ്ഞു: "മൂസാ, നിന്റെ നാഥന്‍ നിനക്കു നല്‍കിയ ഉറപ്പനുസരിച്ച് നീ ഞങ്ങള്‍ക്കുവേണ്ടി അവനോട് പ്രാര്‍ഥിക്കുക. അങ്ങനെ ഞങ്ങളില്‍ നിന്ന് ഈ വിപത്തുകള്‍ നീക്കിത്തന്നാല്‍ ഉറപ്പായും ഞങ്ങള്‍ നിന്നില്‍ വിശ്വസിക്കും. നിന്റെ കൂടെ ഇസ്രയേല്‍ മക്കളെ അയക്കുകയും ചെയ്യും.”
\end{malayalam}}
\flushright{\begin{Arabic}
\quranayah[7][135]
\end{Arabic}}
\flushleft{\begin{malayalam}
എന്നാല്‍, അവരെത്തേണ്ട നിശ്ചിത അവധിവരെ നാം അവരില്‍ നിന്ന് എല്ലാ വിപത്തുകളും ഒഴിവാക്കി. അപ്പോള്‍ അവരെല്ലാം ആ വാക്ക് ലംഘിക്കുകയാണുണ്ടായത്.
\end{malayalam}}
\flushright{\begin{Arabic}
\quranayah[7][136]
\end{Arabic}}
\flushleft{\begin{malayalam}
അതിനാല്‍ നാം അവരോട് പ്രതികാരം ചെയ്തു. നാം അവരെ കടലില്‍ മുക്കിക്കൊന്നു. അവര്‍ നമ്മുടെ വചനങ്ങളെ തള്ളിക്കളയുകയും അവയെ അപ്പാടെ അവഗണിക്കുകയും ചെയ്തതിനാലാണിത്.
\end{malayalam}}
\flushright{\begin{Arabic}
\quranayah[7][137]
\end{Arabic}}
\flushleft{\begin{malayalam}
മര്‍ദിച്ചൊതുക്കപ്പെട്ടിരുന്ന ആ ജനതയെ, നാം അനുഗ്രഹിച്ച കിഴക്കും പടിഞ്ഞാറുമുള്ള പ്രദേശങ്ങളുടെ അവകാശികളാക്കി. അങ്ങനെ ഇസ്രയേല്‍ മക്കളോടുള്ള നിന്റെ നാഥന്റെ ശുഭവാഗ്ദാനം പൂര്‍ത്തിയായി. അവര്‍ ക്ഷമ പാലിച്ചതിനാലാണിത്. ഫറവോനും അവന്റെ ജനതയും നിര്‍മിച്ചുകൊണ്ടിരുന്നതും കെട്ടിപ്പൊക്കിയിരുന്നതുമായ എല്ലാം നാം തകര്‍ത്ത് തരിപ്പണമാക്കുകയും ചെയ്തു.
\end{malayalam}}
\flushright{\begin{Arabic}
\quranayah[7][138]
\end{Arabic}}
\flushleft{\begin{malayalam}
ഇസ്രയേല്‍ മക്കളെ നാം കടല്‍ കടത്തിക്കൊടുത്തു. അവര്‍ വിഗ്രഹപൂജകരായ ഒരു ജനതയുടെ അടുത്തെത്തി. അവര്‍ പറഞ്ഞു: "മൂസാ, ഇവര്‍ക്ക് ഒരുപാട് ദൈവങ്ങളുള്ളതുപോലെ ഒരു ദൈവത്തെ ഞങ്ങള്‍ക്കും ഉണ്ടാക്കിത്തരിക.” മൂസാ പറഞ്ഞു: "നിങ്ങളൊരു വിവരംകെട്ട ജനം തന്നെ.”
\end{malayalam}}
\flushright{\begin{Arabic}
\quranayah[7][139]
\end{Arabic}}
\flushleft{\begin{malayalam}
ഇക്കൂട്ടര്‍ അവലംബങ്ങളാക്കിയവയെല്ലാം നശിക്കാനുള്ളതാണ്. അവര്‍ ചെയ്തുപോരുന്നതോ നിഷ്ഫലവും.
\end{malayalam}}
\flushright{\begin{Arabic}
\quranayah[7][140]
\end{Arabic}}
\flushleft{\begin{malayalam}
മൂസാ പറഞ്ഞു: "അല്ലാഹു അല്ലാത്ത വേറെ ദൈവത്തെ ഞാന്‍ നിങ്ങള്‍ക്കായി തേടുകയോ? ലോകരിലാരെക്കാളും നിങ്ങളെ ശ്രേഷ്ഠരാക്കിയത് അവനായിരിക്കെ.”
\end{malayalam}}
\flushright{\begin{Arabic}
\quranayah[7][141]
\end{Arabic}}
\flushleft{\begin{malayalam}
ഫറവോന്റെ ആള്‍ക്കാരില്‍ നിന്ന് നാം നിങ്ങളെ രക്ഷിച്ചതോര്‍ക്കുക: അവര്‍ നിങ്ങളെ പീഡനങ്ങളേല്‍പിക്കുകയായിരുന്നു. നിങ്ങളുടെ ആണ്‍കുട്ടികളെ അവര്‍ അറുകൊല നടത്തി. സ്ത്രീകളെ മാത്രം ജീവിക്കാന്‍ വിട്ടു. അതില്‍ നിങ്ങള്‍ക്ക് നിങ്ങളുടെ നാഥനില്‍ നിന്നുള്ള കടുത്ത പരീക്ഷണമുണ്ടായിരുന്നു.
\end{malayalam}}
\flushright{\begin{Arabic}
\quranayah[7][142]
\end{Arabic}}
\flushleft{\begin{malayalam}
മൂസാക്ക് നാം മുപ്പത് രാവുകള്‍ നിശ്ചയിച്ചുകൊടുത്തു. പിന്നീട് പത്തുകൂടി ചേര്‍ത്ത് അത് പൂര്‍ത്തിയാക്കി. അങ്ങനെ തന്റെ നാഥന്‍ നിശ്ചയിച്ച നാല്‍പത് നാള്‍ തികഞ്ഞു. മൂസാ തന്റെ സഹോദരന്‍ ഹാറൂനോട് പറഞ്ഞു: "എനിക്കു പിറകെ നീ എന്റെ ജനത്തിന് എന്റെ പ്രതിനിധിയാവണം. നല്ല നിലയില്‍ വര്‍ത്തിക്കണം. കുഴപ്പക്കാരുടെ വഴിയെ പോകരുത്.”
\end{malayalam}}
\flushright{\begin{Arabic}
\quranayah[7][143]
\end{Arabic}}
\flushleft{\begin{malayalam}
നാം നിശ്ചയിച്ച സമയത്ത് മൂസാ വന്നു. തന്റെ നാഥന്‍ അദ്ദേഹത്തോട് സംസാരിച്ചു. അപ്പോള്‍ മൂസാ പറഞ്ഞു: "എന്റെ നാഥാ, നിന്നെ എനിക്കൊന്നു കാണിച്ചുതരൂ! ഞാന്‍ നിന്നെയൊന്നു നോക്കിക്കാണട്ടെ.” അല്ലാഹു പറഞ്ഞു: "നിനക്ക് എന്നെ കാണാനാവില്ല. എന്നാലും നീ ആ മലയിലേക്ക് നോക്കുക. അത് സ്വസ്ഥാനത്ത് ഉറച്ചുനിന്നാല്‍ നീയെന്നെ കാണും.” അങ്ങനെ അദ്ദേഹത്തിന്റെ നാഥന്‍ പര്‍വതത്തിന് പ്രത്യക്ഷമായപ്പോള്‍ അവനതിനെ പൊടിയാക്കി. മൂസാ ബോധംകെട്ടു വീഴുകയും ചെയ്തു. പിന്നീട് ബോധമുണര്‍ന്നപ്പോള്‍ അദ്ദേഹം പറഞ്ഞു: "നീയെത്ര പരിശുദ്ധന്‍. ഞാനിതാ നിന്നിലേക്ക് പശ്ചാത്തപിച്ചു മടങ്ങുന്നു. ഞാന്‍ സത്യവിശ്വാസികളില്‍ ഒന്നാമനാകുന്നു.”
\end{malayalam}}
\flushright{\begin{Arabic}
\quranayah[7][144]
\end{Arabic}}
\flushleft{\begin{malayalam}
അല്ലാഹു പറഞ്ഞു: "മൂസാ, ഞാനെന്റെ സന്ദേശങ്ങളാലും സംഭാഷണങ്ങളാലും മറ്റെല്ലാ മനുഷ്യരെക്കാളും പ്രാധാന്യം നല്‍കി നിന്നെ തെരഞ്ഞെടുത്തിരിക്കുന്നു. അതിനാല്‍ ഞാന്‍ നിനക്കു തന്നതൊക്കെ മുറുകെപ്പിടിക്കുക. നന്ദിയുള്ളവനായിത്തീരുകയും ചെയ്യുക.”
\end{malayalam}}
\flushright{\begin{Arabic}
\quranayah[7][145]
\end{Arabic}}
\flushleft{\begin{malayalam}
സകലസംഗതികളെയും സംബന്ധിച്ച സദുപദേശങ്ങളും എല്ലാ കാര്യങ്ങളുടെയും വിശദാംശങ്ങളും നാം മൂസാക്ക് ഫലകങ്ങളില്‍ രേഖപ്പെടുത്തിക്കൊടുത്തു. എന്നിട്ടിങ്ങനെ പറഞ്ഞു: "അവയെ മുറുകെപ്പിടിക്കുക. അവയിലെ ഏറ്റം നല്ല കാര്യങ്ങള്‍ ഉള്‍ക്കൊള്ളാന്‍ നിന്റെ ജനതയോട് കല്‍പിക്കുകയും ചെയ്യുക. അധര്‍മകാരികളുടെ താമസസ്ഥലം വൈകാതെ തന്നെ ഞാന്‍ നിങ്ങള്‍ക്ക് കാണിച്ചുതരുന്നതാണ്.”
\end{malayalam}}
\flushright{\begin{Arabic}
\quranayah[7][146]
\end{Arabic}}
\flushleft{\begin{malayalam}
ഭൂമിയില്‍ അന്യായമായി അഹങ്കരിച്ചു നടക്കുന്നവരെ ഞാനെന്റെ വചനങ്ങളില്‍ നിന്ന് തെറ്റിച്ചുകളയും. എന്തു തെളിവു കണ്ടാലും അവരതില്‍ വിശ്വസിക്കുകയില്ല. നേര്‍വഴി കണ്ടാലും അവര്‍ ആ മാര്‍ഗം സ്വീകരിക്കുകയില്ല. ദുര്‍മാര്‍ഗം കണ്ടാലോ അവര്‍ ആ പാത പിന്തുടരുകയും ചെയ്യും. നമ്മുടെ വചനങ്ങളെ കള്ളമാക്കിത്തള്ളുകയും അവയെ അപ്പാടെ അവഗണിക്കുകയും ചെയ്തതിനാലാണിത്.
\end{malayalam}}
\flushright{\begin{Arabic}
\quranayah[7][147]
\end{Arabic}}
\flushleft{\begin{malayalam}
നമ്മുടെ വചനങ്ങളെയും പരലോകത്തെ അഭിമുഖീകരിക്കുമെന്നതിനെയും തള്ളിപ്പറയുന്നവരുടെ എല്ലാ പ്രവര്‍ത്തനങ്ങളും പാഴായിരിക്കുന്നു. അവര്‍ പ്രവര്‍ത്തിച്ചിരുന്നതിന്റെ ഫലമല്ലാതെ അവര്‍ക്ക് കിട്ടുമോ?
\end{malayalam}}
\flushright{\begin{Arabic}
\quranayah[7][148]
\end{Arabic}}
\flushleft{\begin{malayalam}
മൂസാ പോയശേഷം അദ്ദേഹത്തിന്റെ ജനത തങ്ങളുടെ ആഭരണങ്ങള്‍ കൊണ്ട്, മുക്രയിടുന്ന ഒരു പശുക്കിടാവിന്റെ രൂപമുണ്ടാക്കി. അത് അവരോട് സംസാരിക്കുന്നില്ലെന്നും അതവരെ നേര്‍വഴിയില്‍ നയിക്കുന്നില്ലെന്നും അവര്‍ മനസ്സിലാക്കുന്നില്ലേ? എന്നിട്ടും അവരതിനെ ദൈവമാക്കി. അവര്‍ കടുത്ത അക്രമികള്‍ തന്നെ.
\end{malayalam}}
\flushright{\begin{Arabic}
\quranayah[7][149]
\end{Arabic}}
\flushleft{\begin{malayalam}
പിന്നീട് അവര്‍ക്ക് കുറ്റവിചാരമുണ്ടാവുകയും തങ്ങള്‍ പിഴച്ചുപോയതായി അവര്‍ കണ്ടറിയുകയും ചെയ്തു. അപ്പോള്‍ അവര്‍ പറഞ്ഞു: "ഞങ്ങളുടെ നാഥന്‍ ഞങ്ങളോടു കരുണ കാണിക്കുകയും ഞങ്ങള്‍ക്ക് പൊറുത്തുതരികയും ചെയ്തില്ലെങ്കില്‍ തീര്‍ച്ചയായും ഞങ്ങള്‍ നഷ്ടപ്പെട്ടവരില്‍ പെട്ടുപോകും.”
\end{malayalam}}
\flushright{\begin{Arabic}
\quranayah[7][150]
\end{Arabic}}
\flushleft{\begin{malayalam}
മൂസാ കുപിതനും ദുഃഖിതനുമായി തന്റെ ജനതയിലേക്ക് തിരിച്ചുവന്നു. അദ്ദേഹം പറഞ്ഞു: "എനിക്ക് പിറകെ എന്റെ പ്രതിനിധികളായി നിങ്ങള്‍ ചെയ്തതെല്ലാം വളരെ ചീത്ത തന്നെ. നിങ്ങളുടെ നാഥന്റെ വിധി വരാന്‍ കാത്തിരിക്കാതെ നിങ്ങള്‍ ധൃതി കാണിച്ചോ?” അദ്ദേഹം ഫലകം നിലത്തെറിഞ്ഞു. സഹോദരന്റെ തല തന്റെ നേരെ പിടിച്ചുവലിച്ചു. സഹോദരന്‍ പറഞ്ഞു: "എന്റെ മാതാവിന്റെ മകനേ, ഈ ജനം എന്നെ കഴിവുകെട്ടവനായിക്കണ്ടു. അവരെന്നെ കൊല്ലുമെന്നേടത്തോളമെത്തി. അതിനാല്‍ എതിരാളികള്‍ക്ക് എന്നെ നോക്കിച്ചിരിക്കാന്‍ ഇടവരുത്താതിരിക്കുക. അക്രമികളായ ജനത്തിന്റെ കൂട്ടത്തില്‍ എന്നെ പെടുത്താതിരിക്കുക.”
\end{malayalam}}
\flushright{\begin{Arabic}
\quranayah[7][151]
\end{Arabic}}
\flushleft{\begin{malayalam}
മൂസാ പറഞ്ഞു: "എന്റെ നാഥാ, എനിക്കും എന്റെ സഹോദരന്നും നീ പൊറുത്തുതരേണമേ. ഞങ്ങളെ നീ നിന്റെ അനുഗ്രഹത്തിന് അര്‍ഹരാക്കേണമേ. നീ പരമകാരുണികനല്ലോ.”
\end{malayalam}}
\flushright{\begin{Arabic}
\quranayah[7][152]
\end{Arabic}}
\flushleft{\begin{malayalam}
പശുക്കിടാവിനെ ദൈവമാക്കിയവരെ അവരുടെ നാഥന്റെ കോപം ബാധിക്കുക തന്നെ ചെയ്യും. ഐഹികജീവിതത്തില്‍ അവര്‍ക്ക് നിന്ദ്യതയാണുണ്ടാവുക. കള്ളം കെട്ടിച്ചമക്കുന്നവര്‍ക്ക് നാം ഇവ്വിധമാണ് പ്രതിഫലം നല്‍കുക.
\end{malayalam}}
\flushright{\begin{Arabic}
\quranayah[7][153]
\end{Arabic}}
\flushleft{\begin{malayalam}
ദുര്‍വൃത്തികള്‍ ചെയ്തശേഷം പശ്ചാത്തപിക്കുകയും സത്യവിശ്വാസം സ്വീകരിക്കുകയും ചെയ്തവരോട് നിന്റെ നാഥന്‍ ഏറെ പൊറുക്കുന്നവനും കരുണ കാണിക്കുന്നവനുമാണ്.
\end{malayalam}}
\flushright{\begin{Arabic}
\quranayah[7][154]
\end{Arabic}}
\flushleft{\begin{malayalam}
മൂസായുടെ കോപം ശമിച്ചപ്പോള്‍ അദ്ദേഹം ആ ഫലകങ്ങളെടുത്തു. തങ്ങളുടെ നാഥനെ ഭയപ്പെടുന്നവര്‍ക്ക് ആ രേഖാഫലകത്തില്‍ മാര്‍ഗദര്‍ശനവും അനുഗ്രഹവുമാണുണ്ടായിരുന്നത്.
\end{malayalam}}
\flushright{\begin{Arabic}
\quranayah[7][155]
\end{Arabic}}
\flushleft{\begin{malayalam}
നാം നിശ്ചയിച്ച സമയത്ത് തന്റെ കൂടെ ഹാജരാകാന്‍ മൂസാ അദ്ദേഹത്തിന്റെ ജനതയില്‍നിന്ന് എഴുപതുപേരെ തെരഞ്ഞെടുത്തു. പെട്ടെന്ന് ഞെട്ടലുണ്ടാക്കുന്ന പ്രകമ്പനം അവരെ പിടികൂടി. അപ്പോള്‍ മൂസാ പറഞ്ഞു: "എന്റെ നാഥാ, നീ ഇച്ഛിച്ചിരുന്നെങ്കില്‍ നേരത്തെതന്നെ അവരെയും എന്നെയും നിനക്ക് നശിപ്പിക്കാമായിരുന്നു. ഞങ്ങളിലെ ഏതാനും വിഡ്ഢികള്‍ പ്രവര്‍ത്തിച്ച പാപത്തിന്റെ പേരില്‍ നീ ഞങ്ങളെയൊക്കെ നശിപ്പിക്കുകയാണോ? നിന്റെ ഒരു പരീക്ഷണമല്ലാതൊന്നുമല്ലിത്. അതുവഴി നീ ഇച്ഛിച്ചവരെ നീ വഴികേടിലാക്കുന്നു. നീ ഇച്ഛിച്ചവരെ നേര്‍വഴിയിലാക്കുകയും ചെയ്യുന്നു. നീയാണ് ഞങ്ങളുടെ രക്ഷകന്‍. അതിനാല്‍ ഞങ്ങള്‍ക്കു നീ പൊറുത്തുതരേണമേ. ഞങ്ങളോട് കരുണ കാണിക്കേണമേ. പൊറുക്കുന്നവരില്‍ അത്യുത്തമന്‍ നീയാണല്ലോ.
\end{malayalam}}
\flushright{\begin{Arabic}
\quranayah[7][156]
\end{Arabic}}
\flushleft{\begin{malayalam}
"ഞങ്ങള്‍ക്കു നീ ഈ ലോകത്തും പരലോകത്തും നന്മ വിധിക്കേണമേ. തീര്‍ച്ചയായും ഞങ്ങള്‍ നിന്നിലേക്ക് പശ്ചാത്തപിച്ചു മടങ്ങിയിരിക്കുന്നു.” അല്ലാഹു അറിയിച്ചു: "എന്റെ ശിക്ഷ ഞാനുദ്ദേശിക്കുന്നവരെ ബാധിക്കും. എന്നാല്‍ എന്റെ കാരുണ്യം എല്ലാ വസ്തുക്കളെയും ചൂഴ്ന്നുനില്‍ക്കുന്നു. സൂക്ഷ്മത പാലിക്കുകയും സകാത്ത് നല്‍കുകയും നമ്മുടെ പ്രമാണങ്ങളില്‍ വിശ്വസിക്കുകയും ചെയ്യുന്നവര്‍ക്ക് നാമത് രേഖപ്പെടുത്തുന്നു.”
\end{malayalam}}
\flushright{\begin{Arabic}
\quranayah[7][157]
\end{Arabic}}
\flushleft{\begin{malayalam}
തങ്ങളുടെ വശമുള്ള തൌറാത്തിലും ഇഞ്ചീലിലും രേഖപ്പെടുത്തിയതായി അവര്‍ കാണുന്ന നിരക്ഷരനായ പ്രവാചകനുണ്ടല്ലോ അവര്‍ ആ ദൈവദൂതനെ പിന്‍പറ്റുന്നവരാണ്. അവരോട് അദ്ദേഹം നന്മ കല്‍പിക്കുകയും തിന്മ വിലക്കുകയും ചെയ്യുന്നു. ഉത്തമ വസ്തുക്കള്‍ അവര്‍ക്ക് അനുവദനീയമാക്കുകയും ചീത്ത വസ്തുക്കള്‍ നിഷിദ്ധമാക്കുകയും ചെയ്യുന്നു. അവരെ ഞെരിച്ചുകൊണ്ടിരിക്കുന്ന ഭാരങ്ങള്‍ ഇറക്കിവെക്കുന്നു. അവരെ കുരുക്കിയിട്ട വിലങ്ങുകള്‍ അഴിച്ചുമാറ്റുന്നു. അതിനാല്‍ അദ്ദേഹത്തില്‍ വിശ്വസിക്കുകയും അദ്ദേഹത്തെ ശക്തിപ്പെടുത്തുകയും സഹായിക്കുകയും അദ്ദേഹത്തിന് അവതീര്‍ണമായ പ്രകാശത്തെ പിന്തുടരുകയും ചെയ്യുന്നവരാരോ, അവരാണ് വിജയം വരിച്ചവര്‍.
\end{malayalam}}
\flushright{\begin{Arabic}
\quranayah[7][158]
\end{Arabic}}
\flushleft{\begin{malayalam}
പറയുക: മനുഷ്യരേ, ഞാന്‍ നിങ്ങളെല്ലാവരിലേക്കുമുള്ള, ആകാശഭൂമികളുടെ അധിപനായ അല്ലാഹുവിന്റെ ദൂതനാണ്. അവനല്ലാതെ ദൈവമില്ല. അവന്‍ ജീവിപ്പിക്കുകയും മരിപ്പിക്കുകയും ചെയ്യുന്നു. അതിനാല്‍ നിങ്ങള്‍ അല്ലാഹുവില്‍ വിശ്വസിക്കുക. അവന്റെ ദൂതനിലും. അഥവാ നിരക്ഷരനായ പ്രവാചകനില്‍. അദ്ദേഹം അല്ലാഹുവിലും അവന്റെ വചനങ്ങളിലും വിശ്വസിക്കുന്നു. നിങ്ങള്‍ അദ്ദേഹത്തെ പിന്‍പറ്റുക. നിങ്ങള്‍ നേര്‍വഴിയിലായേക്കാം.
\end{malayalam}}
\flushright{\begin{Arabic}
\quranayah[7][159]
\end{Arabic}}
\flushleft{\begin{malayalam}
മൂസായുടെ ജനതയില്‍തന്നെ സത്യമനുസരിച്ച് നേര്‍വഴി കാട്ടുകയും അതിനനുസരിച്ച് നീതി നടത്തുകയും ചെയ്യുന്ന ഒരു സമുദായമുണ്ട്.
\end{malayalam}}
\flushright{\begin{Arabic}
\quranayah[7][160]
\end{Arabic}}
\flushleft{\begin{malayalam}
അവരെ നാം പന്ത്രണ്ട് ഗോത്രങ്ങളായി അഥവാ സമൂഹങ്ങളായി വിഭജിച്ചു. മൂസായോട് അദ്ദേഹത്തിന്റെ ജനത കുടിവെള്ളം ചോദിച്ചു. അപ്പോള്‍ നാം അദ്ദേഹത്തോട് നിര്‍ദേശിച്ചു: "നീ നിന്റെ വടികൊണ്ട് പാറക്കല്ലില്‍ അടിക്കുക.” അങ്ങനെ അതില്‍നിന്ന് പന്ത്രണ്ട് ഉറവകള്‍ പൊട്ടിയൊഴുകി. ഓരോ വിഭാഗത്തിനും തങ്ങള്‍ വെള്ളം കുടിക്കേണ്ട സ്ഥലമേതെന്ന് മനസ്സിലാക്കാന്‍ സാധിച്ചു. അവര്‍ക്കു നാം കാര്‍മേഘംകൊണ്ട് തണലേകി. മന്നായും സല്‍വായും ഇറക്കിക്കൊടുത്തു. നാം നിര്‍ദേശിച്ചു: "നാം നിങ്ങള്‍ക്കു നല്‍കിയ ഉത്തമ പദാര്‍ഥങ്ങളില്‍നിന്ന് തിന്നുകൊള്ളുക.” തങ്ങളുടെ ചെയ്തികളിലൂടെ അവര്‍ നമുക്കൊരു ദ്രോഹവും വരുത്തിയിട്ടില്ല. അവരെത്തന്നെയാണ് അവര്‍ ദ്രോഹിച്ചുകൊണ്ടിരുന്നത്.
\end{malayalam}}
\flushright{\begin{Arabic}
\quranayah[7][161]
\end{Arabic}}
\flushleft{\begin{malayalam}
അവരോടിങ്ങനെ പറഞ്ഞതോര്‍ക്കുക: "നിങ്ങള്‍ ഈ പട്ടണത്തില്‍ പാര്‍ക്കുകയും ഇവിടെ നിങ്ങള്‍ക്കിഷ്ടമുള്ളേടത്തുനിന്ന് തിന്നുകയും ചെയ്യുക. നിങ്ങള്‍ പാപമോചനത്തിനായി പ്രാര്‍ഥിക്കുക. പട്ടണ കവാടത്തിലൂടെ പ്രണമിക്കുന്നവരായി പ്രവേശിക്കുകയും ചെയ്യുക. എങ്കില്‍ നിങ്ങള്‍ക്ക് നിങ്ങളുടെ പാപങ്ങള്‍ നാം പൊറുത്തുതരും. നല്ലവര്‍ക്ക് നാം കൂടുതല്‍ നല്‍കും.
\end{malayalam}}
\flushright{\begin{Arabic}
\quranayah[7][162]
\end{Arabic}}
\flushleft{\begin{malayalam}
എന്നാല്‍ അവരോട് പറഞ്ഞതിനെ അവരിലെ അക്രമികള്‍ മാറ്റിമറിച്ചു. അങ്ങനെ അവര്‍ അക്രമം കാണിച്ചു. തദ്ഫലമായി നാം അവരുടെമേല്‍ ഉപരിലോകത്തുനിന്ന് ശിക്ഷ അയച്ചു.
\end{malayalam}}
\flushright{\begin{Arabic}
\quranayah[7][163]
\end{Arabic}}
\flushleft{\begin{malayalam}
സമുദ്ര തീരത്ത് സ്ഥിതിചെയ്തിരുന്ന ആ പട്ടണത്തെപ്പറ്റി നീ ഇവരോടൊന്നു ചോദിച്ചുനോക്കൂ. സാബത്ത് ദിനാചരണത്തില്‍ അവര്‍ അതിക്രമം കാണിച്ച കാര്യം. സാബത്ത് ദിനത്തില്‍ അവര്‍ക്കാവശ്യമായ മത്സ്യങ്ങള്‍ ജലപ്പരപ്പില്‍ അവരുടെയടുത്ത് കൂട്ടമായി വന്നതും സാബത്ത് ആചരിക്കേണ്ടാത്ത ദിനങ്ങളില്‍ അവ അവരുടെ അടുത്ത് വരാതിരുന്നതുമായ കാര്യം. അവര്‍ അധര്‍മം പ്രവര്‍ത്തിച്ചിരുന്നതിനാല്‍ നാം അവരെ അവ്വിധം പരീക്ഷിക്കുകയായിരുന്നു.
\end{malayalam}}
\flushright{\begin{Arabic}
\quranayah[7][164]
\end{Arabic}}
\flushleft{\begin{malayalam}
അവരില്‍ ഒരു വിഭാഗം ഇങ്ങനെ പറഞ്ഞതോര്‍ക്കുക: "അല്ലാഹു നിശ്ശേഷം നശിപ്പിക്കുകയോ അല്ലെങ്കില്‍ കഠിനമായി ശിക്ഷിക്കുകയോ ചെയ്യാനിരിക്കുന്ന ജനത്തെ നിങ്ങളെന്തിന് ഉപദേശിക്കുന്നു?” അവര്‍ മറുപടി പറഞ്ഞു: "നിങ്ങളുടെ നാഥന്റെ അടുത്ത് ഞങ്ങള്‍ ബാധ്യത നിറവേറ്റിയിട്ടുണ്ടെന്ന് ന്യായം ബോധിപ്പിക്കാന്‍. ഒരുവേള അവര്‍ സൂക്ഷ്മതയുള്ളവരായേക്കാമല്ലോ.”
\end{malayalam}}
\flushright{\begin{Arabic}
\quranayah[7][165]
\end{Arabic}}
\flushleft{\begin{malayalam}
അങ്ങനെ അവരെ ഓര്‍മിപ്പിച്ചിരുന്ന ഉപദേശം അവര്‍ പൂര്‍ണമായും മറന്നപ്പോള്‍, തിന്മകള്‍ തടഞ്ഞിരുന്നവരെ നാം രക്ഷപ്പെടുത്തി. അതിക്രമം കാണിച്ചവരെയെല്ലാം അവരുടെ പാപവൃത്തികളുടെ പേരില്‍ കൊടും ശിക്ഷയാല്‍ പിടികൂടുകയും ചെയ്തു.
\end{malayalam}}
\flushright{\begin{Arabic}
\quranayah[7][166]
\end{Arabic}}
\flushleft{\begin{malayalam}
അവരോട് വിലക്കിയിരുന്ന കാര്യങ്ങളിലെല്ലാം അവര്‍ ധിക്കാരം കാണിച്ചപ്പോള്‍ നാം അവരോട് പറഞ്ഞു: "നിങ്ങള്‍ നിന്ദ്യരായ കുരങ്ങന്മാരായിത്തീരട്ടെ.”
\end{malayalam}}
\flushright{\begin{Arabic}
\quranayah[7][167]
\end{Arabic}}
\flushleft{\begin{malayalam}
നിന്റെ നാഥന്‍ പ്രഖ്യാപിച്ചതോര്‍ക്കുക: അവരെ ക്രൂരമായി ശിക്ഷിച്ചുകൊണ്ടിരിക്കുന്നവരെ അവരുടെനേരെ അന്ത്യനാള്‍ വരെയും അവന്‍ നിയോഗിച്ചുകൊണ്ടിരിക്കും. നിന്റെ നാഥന്‍ വളരെ വേഗം ശിക്ഷ നടപ്പാക്കുന്നവനാണ്. ഒപ്പം ഏറെ പൊറുക്കുന്നവനും കരുണാമയനും.
\end{malayalam}}
\flushright{\begin{Arabic}
\quranayah[7][168]
\end{Arabic}}
\flushleft{\begin{malayalam}
ഭൂമിയില്‍ അവരെ നാം പല സമൂഹങ്ങളായി വിഭജിച്ചിരിക്കുന്നു. അവരില്‍ സജ്ജനങ്ങളുണ്ട്. നേരെമറിച്ചുള്ളവരുമുണ്ട്. നാം അവരെ ഗുണദോഷങ്ങളാല്‍ പരീക്ഷിച്ചുകൊണ്ടിരുന്നു. ഒരുവേള അവര്‍ തിരിച്ചുവന്നെങ്കിലോ.
\end{malayalam}}
\flushright{\begin{Arabic}
\quranayah[7][169]
\end{Arabic}}
\flushleft{\begin{malayalam}
പിന്നീട് അവര്‍ക്കുപിറകെ അവരുടെ പിന്‍ഗാമികളായി ഒരു വിഭാഗം വന്നു. അവര്‍ വേദഗ്രന്ഥം അനന്തരമെടുത്തു. ഈ അധമലോകത്തിന്റെ വിഭവങ്ങളാണ് അവര്‍ ശേഖരിച്ചുകൊണ്ടിരിക്കുന്നത്. “ഞങ്ങള്‍ക്ക് ഇതൊക്കെ പൊറുത്തുകിട്ടു”മെന്ന് അവര്‍ പറയുകയും ചെയ്യുന്നു. അത്തരത്തിലുള്ള ഐഹികവിഭവങ്ങള്‍ വീണ്ടും വന്നുകിട്ടിയാല്‍ അതവര്‍ വാരിപ്പുണരും. അല്ലാഹുവിന്റെ പേരില്‍ സത്യമല്ലാതൊന്നും പറയുകയില്ലെന്ന് വേദഗ്രന്ഥത്തിലൂടെ അവരോട് കരാര്‍ വാങ്ങുകയും അതിലുള്ളത് അവര്‍ പഠിച്ചറിയുകയും ചെയ്തിട്ടില്ലേ? പരലോക ഭവനമാണ് ഭക്തി പുലര്‍ത്തുന്നവര്‍ക്ക് ഉത്തമം. നിങ്ങള്‍ ആലോചിച്ചറിയുന്നില്ലേ?
\end{malayalam}}
\flushright{\begin{Arabic}
\quranayah[7][170]
\end{Arabic}}
\flushleft{\begin{malayalam}
വേദഗ്രന്ഥം മുറുകെപ്പിടിക്കുകയും നമസ്കാരം നിഷ്ഠയോടെ നിര്‍വഹിക്കുകയും ചെയ്യുന്ന സല്‍ക്കര്‍മികള്‍ക്കുള്ള പ്രതിഫലം നാം നഷ്ടപ്പെടുത്തുകയില്ല; തീര്‍ച്ച.
\end{malayalam}}
\flushright{\begin{Arabic}
\quranayah[7][171]
\end{Arabic}}
\flushleft{\begin{malayalam}
ഓര്‍ക്കുക: നാം അവര്‍ക്കുമീതെ മലയെ കുടയായി പിടിച്ചു. അത് തങ്ങളുടെ മേല്‍ വീഴുമെന്ന് അവര്‍ കരുതി. അപ്പോള്‍ നാം പറഞ്ഞു: "നാം നിങ്ങള്‍ക്കു നല്‍കിയത് മുറുകെപ്പിടിക്കുക. അതിലുള്ളത് ഓര്‍ക്കുകയും ചെയ്യുക. നിങ്ങള്‍ സൂക്ഷ്മതയുള്ളവരായേക്കാം.”
\end{malayalam}}
\flushright{\begin{Arabic}
\quranayah[7][172]
\end{Arabic}}
\flushleft{\begin{malayalam}
നിന്റെ നാഥന്‍ ആദം സന്തതികളുടെ മുതുകുകളില്‍ നിന്ന് അവരുടെ സന്താന പരമ്പരകളെ പുറത്തെടുക്കുകയും അവരുടെമേല്‍ അവരെത്തന്നെ സാക്ഷിയാക്കുകയും ചെയ്ത സന്ദര്‍ഭം. അവന്‍ ചോദിച്ചു: "നിങ്ങളുടെ നാഥന്‍ ഞാനല്ലയോ?” അവര്‍ പറഞ്ഞു: "അതെ; ഞങ്ങളതിന് സാക്ഷ്യം വഹിച്ചിരിക്കുന്നു.” ഉയിര്‍ത്തെഴുന്നേല്‍പുനാളില്‍ “ഞങ്ങള്‍ ഇതേക്കുറിച്ച് അശ്രദ്ധരായിരുന്നു”വെന്ന് നിങ്ങള്‍ പറയാതിരിക്കാനാണിത്.
\end{malayalam}}
\flushright{\begin{Arabic}
\quranayah[7][173]
\end{Arabic}}
\flushleft{\begin{malayalam}
അല്ലെങ്കില്‍ നിങ്ങളിങ്ങനെയും പറയാതിരിക്കാനാണ്: "വളരെ മുമ്പുതന്നെ ഞങ്ങളുടെ പൂര്‍വ പിതാക്കള്‍ അല്ലാഹുവില്‍ പങ്കുചേര്‍ത്തിരുന്നു. അവര്‍ക്കു ശേഷം വന്ന അവരുടെ പിന്മുറക്കാര്‍ മാത്രമാണ് ഞങ്ങള്‍. എന്നിട്ടും ആ ദുരാചാരികള്‍ പ്രവര്‍ത്തിച്ചതിന്റെ പേരില്‍ ഞങ്ങളെ ശിക്ഷിക്കുകയോ?”
\end{malayalam}}
\flushright{\begin{Arabic}
\quranayah[7][174]
\end{Arabic}}
\flushleft{\begin{malayalam}
ഇവ്വിധം നാം തെളിവുകള്‍ വിശദമായി വിവരിച്ചുതരുന്നു. ഒരുവേള അവര്‍ തിരിച്ചുവന്നെങ്കിലോ.
\end{malayalam}}
\flushright{\begin{Arabic}
\quranayah[7][175]
\end{Arabic}}
\flushleft{\begin{malayalam}
ആ ഒരുവന്റെ വിവരം നീ അവരെ വായിച്ചു കേള്‍പ്പിക്കുക. നാം അയാള്‍ക്ക് നമ്മുടെ വചനങ്ങള്‍ നല്‍കി. എന്നിട്ടും അയാള്‍ അതില്‍നിന്നൊഴിഞ്ഞുമാറി. അപ്പോള്‍ പിശാച് അവന്റെ പിറകെകൂടി. അങ്ങനെ അവന്‍ വഴികേടിലായി.
\end{malayalam}}
\flushright{\begin{Arabic}
\quranayah[7][176]
\end{Arabic}}
\flushleft{\begin{malayalam}
നാം ഇച്ഛിച്ചിരുന്നെങ്കില്‍ ആ വചനങ്ങളിലൂടെത്തന്നെ നാമവനെ ഉന്നതിയിലേക്ക് നയിക്കുമായിരുന്നു. പക്ഷേ, അയാള്‍ ഭൂമിയോട് ഒട്ടിച്ചേര്‍ന്ന് തന്നിഷ്ടത്തെ പിന്‍പറ്റുകയാണുണ്ടായത്. അതിനാല്‍ അയാളുടെ ഉപമ ഒരു നായയുടേതാണ്. നീ അതിനെ ദ്രോഹിച്ചാല്‍ അത് നാക്ക് തൂക്കിയിടും. നീ അതിനെ വെറുതെ വിട്ടാലും അത് നാവ് നീട്ടിയിടും. നമ്മുടെ വചനങ്ങളെ കള്ളമാക്കിയ ജനത്തിന്റെ ഉദാഹരണവും ഇതുതന്നെ. അതിനാല്‍ അവര്‍ക്ക് ഇക്കഥയൊന്ന് വിശദീകരിച്ചുകൊടുക്കുക. ഒരുവേള അവര്‍ ചിന്തിച്ചെങ്കിലോ.
\end{malayalam}}
\flushright{\begin{Arabic}
\quranayah[7][177]
\end{Arabic}}
\flushleft{\begin{malayalam}
നമ്മുടെ വചനങ്ങളെ തള്ളിപ്പറയുകയും തങ്ങള്‍ക്കുതന്നെ ദ്രോഹം വരുത്തിവെക്കുകയും ചെയ്യുന്നവരുടെ ഉപമ വളരെ ചീത്ത തന്നെ.
\end{malayalam}}
\flushright{\begin{Arabic}
\quranayah[7][178]
\end{Arabic}}
\flushleft{\begin{malayalam}
അല്ലാഹു നന്മയിലേക്കു നയിക്കുന്നവര്‍ മാത്രമാണ് നേര്‍വഴി പ്രാപിച്ചവര്‍. അവന്‍ ദുര്‍മാര്‍ഗത്തിലാക്കുന്നവര്‍ നഷ്ടം പറ്റിയവരാണ്.
\end{malayalam}}
\flushright{\begin{Arabic}
\quranayah[7][179]
\end{Arabic}}
\flushleft{\begin{malayalam}
ജിന്നുകളിലും മനുഷ്യരിലും ധാരാളം പേരെ നാം നരകത്തിനുവേണ്ടി സൃഷ്ടിച്ചിട്ടുണ്ട്. അവര്‍ക്ക് ഹൃദയങ്ങളുണ്ട്; അതുപയോഗിച്ച് അവര്‍ പഠിക്കുന്നില്ല. കണ്ണുകളുണ്ട്; അതുകൊണ്ട് കണ്ടറിയുന്നില്ല. കാതുകളുണ്ട്; അതുപയോഗിച്ച് കേട്ടു മനസ്സിലാക്കുന്നില്ല. അവര്‍ നാല്‍ക്കാലികളെപ്പോലെയാണ്. എന്നല്ല, അവരാണ് പിഴച്ചവര്‍. അവര്‍ തന്നെയാണ് ഒരു ശ്രദ്ധയുമില്ലാത്തവര്‍.
\end{malayalam}}
\flushright{\begin{Arabic}
\quranayah[7][180]
\end{Arabic}}
\flushleft{\begin{malayalam}
അല്ലാഹുവിന് അത്യുല്‍കൃഷ്ടമായ അനേകം നാമങ്ങളുണ്ട്. ആ നാമങ്ങളില്‍ തന്നെ നിങ്ങളവനെ വിളിച്ചു പ്രാര്‍ഥിക്കുക. അവന്റെ നാമങ്ങളില്‍ കൃത്രിമം കാണിക്കുന്നവരെ അവഗണിക്കുക. സംശയം വേണ്ട. അവര്‍ പ്രവര്‍ത്തിച്ചുകൊണ്ടിരിക്കുന്നതിന്റെ ഫലം അവര്‍ക്ക് കിട്ടുക തന്നെ ചെയ്യും.
\end{malayalam}}
\flushright{\begin{Arabic}
\quranayah[7][181]
\end{Arabic}}
\flushleft{\begin{malayalam}
നമ്മുടെ സൃഷ്ടികളില്‍ ജനത്തെ സത്യപാതയില്‍ നയിക്കുകയും സത്യനിഷ്ഠയോടെ നീതി നടത്തുകയും ചെയ്യുന്ന ഒരു വിഭാഗമുണ്ട്.
\end{malayalam}}
\flushright{\begin{Arabic}
\quranayah[7][182]
\end{Arabic}}
\flushleft{\begin{malayalam}
എന്നാല്‍ നമ്മുടെ വചനങ്ങളെ തള്ളിക്കളയുന്നവരെ അവരറിയാതെ നാം ക്രമേണ പിടികൂടും.
\end{malayalam}}
\flushright{\begin{Arabic}
\quranayah[7][183]
\end{Arabic}}
\flushleft{\begin{malayalam}
നാം അവര്‍ക്ക് വീണ്ടും വീണ്ടും അവസരം കൊടുത്തുകൊണ്ടിരിക്കുകയാണ്. അറിയുക: തീര്‍ച്ചയായും നമ്മുടെ തന്ത്രം ഭദ്രം തന്നെ.
\end{malayalam}}
\flushright{\begin{Arabic}
\quranayah[7][184]
\end{Arabic}}
\flushleft{\begin{malayalam}
ഇക്കൂട്ടര്‍ ആലോചിച്ചറിഞ്ഞിട്ടില്ലേ; തങ്ങളുടെ കൂട്ടുകാരന് ഭ്രാന്തൊന്നുമില്ലെന്ന്. അദ്ദേഹം തെളിഞ്ഞ മുന്നറിയിപ്പുകാരന്‍ മാത്രമാണ്.
\end{malayalam}}
\flushright{\begin{Arabic}
\quranayah[7][185]
\end{Arabic}}
\flushleft{\begin{malayalam}
ആകാശഭൂമികളുടെ ഭരണ സംവിധാനത്തെക്കുറിച്ച് അവര്‍ അല്‍പവും ആലോചിച്ചുനോക്കിയിട്ടില്ലേ? അല്ലാഹു സൃഷ്ടിച്ച ഒന്നിനെക്കുറിച്ചും അവര്‍ മനസ്സിലാക്കിയിട്ടില്ലേ? അവരുടെ ജീവിതാവധി അടുത്തെത്തിയിരിക്കാമെന്നതിനെപ്പറ്റിയും? ഇനി ഈ ഖുര്‍ആനിനുശേഷം അതല്ലാത്ത ഏതൊരു സന്ദേശത്തിലാണ് അവര്‍ വിശ്വസിക്കാന്‍ പോകുന്നത്?
\end{malayalam}}
\flushright{\begin{Arabic}
\quranayah[7][186]
\end{Arabic}}
\flushleft{\begin{malayalam}
അല്ലാഹു വഴികേടിലാക്കുന്നവരെ നേര്‍വഴിയിലാക്കുന്ന ആരുമില്ല. അവനവരെ തങ്ങളുടെ അതിക്രമത്തില്‍ അന്ധമായി വിഹരിക്കാന്‍ വിട്ടിരിക്കയാണ്.
\end{malayalam}}
\flushright{\begin{Arabic}
\quranayah[7][187]
\end{Arabic}}
\flushleft{\begin{malayalam}
ആ അന്ത്യനിമിഷത്തെപ്പറ്റി അവര്‍ നിന്നോട് ചോദിക്കുന്നു: അതെപ്പോഴാണ് വന്നെത്തുകയെന്ന്. പറയുക: അതേക്കുറിച്ച അറിവ് എന്റെ നാഥന്റെ വശം മാത്രമേയുള്ളൂ. യഥാസമയം അവനാണത് വെളിപ്പെടുത്തുക. ആകാശഭൂമികളില്‍ അതുണ്ടാക്കുന്ന ആഘാതം വളരെ കടുത്തതായിരിക്കും. തീര്‍ത്തും യാദൃഛികമായാണ് അത് നിങ്ങളില്‍ വന്നെത്തുക. നീ അതേക്കുറിച്ച് ചുഴിഞ്ഞ് അന്വേഷിച്ചറിഞ്ഞവനാണെന്നപോലെ അവര്‍ നിന്നോട് ചോദിക്കുന്നു. പറയുക: അതേക്കുറിച്ച അറിവ് അല്ലാഹുവിങ്കല്‍ മാത്രമേയുള്ളൂ. എങ്കിലും ഏറെപ്പേരും ഇതൊന്നുമറിയുന്നില്ല.
\end{malayalam}}
\flushright{\begin{Arabic}
\quranayah[7][188]
\end{Arabic}}
\flushleft{\begin{malayalam}
പറയുക: "ഞാന്‍ എനിക്കുതന്നെ ഗുണമോ ദോഷമോ വരുത്താന്‍ കഴിയാത്തവനാണ്. അല്ലാഹു ഇച്ഛിച്ചതുമാത്രം നടക്കുന്നു. എനിക്ക് അഭൌതിക കാര്യങ്ങള്‍ അറിയുമായിരുന്നെങ്കില്‍ നിശ്ചയമായും ഞാന്‍ എനിക്കുതന്നെ അളവറ്റ നേട്ടങ്ങള്‍ കൈവരുത്തുമായിരുന്നു. ദോഷങ്ങള്‍ എന്നെ ഒട്ടും ബാധിക്കുമായിരുന്നുമില്ല. എന്നാല്‍ ഞാനൊരു മുന്നറിയിപ്പുകാരന്‍ മാത്രമാണ്. വിശ്വസിക്കുന്ന ജനത്തിന് ശുഭവാര്‍ത്ത അറിയിക്കുന്നവനും.”
\end{malayalam}}
\flushright{\begin{Arabic}
\quranayah[7][189]
\end{Arabic}}
\flushleft{\begin{malayalam}
ഒരൊറ്റ സത്തയില്‍ നിന്ന് നിങ്ങളെ സൃഷ്ടിച്ചവനാണവന്‍. അതില്‍ നിന്നുതന്നെ അതിന്റെ ഇണയേയും സൃഷ്ടിച്ചു. ആ ഇണയോടൊത്ത് സംതൃപ്തി നേടാന്‍. അവന്‍ അവളെ പുണര്‍ന്നു. അങ്ങനെ അവള്‍ ഗര്‍ഭത്തിന്റെ ലഘുവായ ഭാരം വഹിച്ചു. അവള്‍ അതും ചുമന്നു നടന്നു. പിന്നീട് അതവള്‍ക്ക് ഭാരമായപ്പോള്‍ അവരിരുവരും തങ്ങളുടെ നാഥനായ അല്ലാഹുവോട് പ്രാര്‍ഥിച്ചു: "ഞങ്ങള്‍ക്ക് നീ നല്ലൊരു കുഞ്ഞിനെ തരികയാണെങ്കില്‍ തീര്‍ച്ചയായും ഞങ്ങളെന്നും നന്ദിയുള്ളവരായിരിക്കും.”
\end{malayalam}}
\flushright{\begin{Arabic}
\quranayah[7][190]
\end{Arabic}}
\flushleft{\begin{malayalam}
അങ്ങനെ അല്ലാഹു അവര്‍ക്ക് നല്ലൊരു കുഞ്ഞിനെ കൊടുത്തു. അപ്പോള്‍ അവനവര്‍ക്ക് നല്‍കിയതില്‍ അവര്‍ അല്ലാഹുവിന് പങ്കുകാരെ സങ്കല്‍പിച്ചു. എന്നാല്‍ അവര്‍ സങ്കല്‍പിക്കുന്ന പങ്കാളികളില്‍നിന്നെല്ലാം അതീതനും ഉന്നതനുമാണ് അല്ലാഹു.
\end{malayalam}}
\flushright{\begin{Arabic}
\quranayah[7][191]
\end{Arabic}}
\flushleft{\begin{malayalam}
ഒന്നും പടച്ചുണ്ടാക്കാത്തവരെയാണോ അവര്‍ അവനില്‍ പങ്കാളികളാക്കുന്നത്? അവര്‍ തന്നെയും അല്ലാഹുവാല്‍ സൃഷ്ടിക്കപ്പെട്ടവരാണ്.
\end{malayalam}}
\flushright{\begin{Arabic}
\quranayah[7][192]
\end{Arabic}}
\flushleft{\begin{malayalam}
ഇവര്‍ക്കൊരു സഹായവും ചെയ്യാന്‍ അവര്‍ക്കാവില്ല. എന്തിനേറെ തങ്ങളെത്തന്നെ സഹായിക്കാന്‍ അവര്‍ക്കു സാധ്യമല്ല.
\end{malayalam}}
\flushright{\begin{Arabic}
\quranayah[7][193]
\end{Arabic}}
\flushleft{\begin{malayalam}
നിങ്ങള്‍ ഇവരെ നേര്‍വഴിയിലേക്ക് ക്ഷണിച്ചാല്‍ ഉറപ്പായും ഇവര്‍ നിങ്ങളെ പിന്‍പറ്റുകയില്ല. നിങ്ങളിവരെ ക്ഷണിക്കുന്നതും വെറുതെ മൌനമവലംബിക്കുന്നതും നിങ്ങളെ സംബന്ധിച്ചേടത്തോളം സമമാണ്.
\end{malayalam}}
\flushright{\begin{Arabic}
\quranayah[7][194]
\end{Arabic}}
\flushleft{\begin{malayalam}
അല്ലാഹുവെ വിട്ട് നിങ്ങള്‍ വിളിച്ചു പ്രാര്‍ഥിക്കുന്നവര്‍, നിങ്ങളെപ്പോലുള്ള അടിമകള്‍ മാത്രമാണ്. നിങ്ങള്‍ അവരോട് പ്രാര്‍ഥിച്ചുനോക്കൂ. നിങ്ങള്‍ക്കവര്‍ ഉത്തരം നല്‍കട്ടെ; നിങ്ങള്‍ സത്യവാദികളെങ്കില്‍!
\end{malayalam}}
\flushright{\begin{Arabic}
\quranayah[7][195]
\end{Arabic}}
\flushleft{\begin{malayalam}
അവര്‍ക്ക് കാലുകളുണ്ടോ നടക്കാന്‍? കൈകളുണ്ടോ പിടിക്കാന്‍? കണ്ണുകളുണ്ടോ കാണാന്‍? കാതുകളുണ്ടോ കേള്‍ക്കാന്‍? പറയുക: നിങ്ങള്‍ നിങ്ങളുടെ പങ്കാളികളെ വിളിക്കൂ; എന്നിട്ട് എനിക്കെതിരെ തന്ത്രങ്ങള്‍ പ്രയോഗിക്കൂ. എനിക്കൊട്ടും അവധി അനുവദിക്കേണ്ടതില്ല.
\end{malayalam}}
\flushright{\begin{Arabic}
\quranayah[7][196]
\end{Arabic}}
\flushleft{\begin{malayalam}
ഈ വേദഗ്രന്ഥമിറക്കിയ അല്ലാഹുവാണ് എന്റെ രക്ഷകന്‍. അവന്‍ സജ്ജനങ്ങളെ സംരക്ഷിക്കുന്നു.
\end{malayalam}}
\flushright{\begin{Arabic}
\quranayah[7][197]
\end{Arabic}}
\flushleft{\begin{malayalam}
അവനെക്കൂടാതെ നിങ്ങള്‍ വിളിച്ചു പ്രാര്‍ഥിക്കുന്നവര്‍ക്കൊന്നും നിങ്ങളെ സഹായിക്കാന്‍ സാധ്യമല്ല. തങ്ങളെത്തന്നെ സഹായിക്കാന്‍ അവര്‍ക്കാവില്ല.
\end{malayalam}}
\flushright{\begin{Arabic}
\quranayah[7][198]
\end{Arabic}}
\flushleft{\begin{malayalam}
നിങ്ങള്‍ അവരെ നേര്‍വഴിയിലേക്ക് ക്ഷണിക്കുകയാണെങ്കില്‍ അതവര്‍ കേള്‍ക്കുക പോലുമില്ല. അവര്‍ നിന്റെ നേരെ നോക്കുന്നതായി നിനക്കു കാണാം. ഫലത്തിലോ അവരൊന്നും കാണുന്നില്ല.
\end{malayalam}}
\flushright{\begin{Arabic}
\quranayah[7][199]
\end{Arabic}}
\flushleft{\begin{malayalam}
നീ വിട്ടുവീഴ്ച കാണിക്കുക. നല്ലതു കല്‍പിക്കുക. അവിവേകികളെ അവഗണിക്കുക.
\end{malayalam}}
\flushright{\begin{Arabic}
\quranayah[7][200]
\end{Arabic}}
\flushleft{\begin{malayalam}
പിശാചില്‍ നിന്നുള്ള വല്ല ദുര്‍ബോധനവും നിന്നെ ബാധിക്കുകയാണെങ്കില്‍ നീ അല്ലാഹുവില്‍ ശരണം തേടുക. തീര്‍ച്ചയായും അവന്‍ എല്ലാം കേള്‍ക്കുന്നവനും അറിയുന്നവനുമാണ്.
\end{malayalam}}
\flushright{\begin{Arabic}
\quranayah[7][201]
\end{Arabic}}
\flushleft{\begin{malayalam}
ദൈവഭക്തരെ പിശാചില്‍ നിന്നുള്ള വല്ല ദുര്‍ബോധനവും ബാധിച്ചാല്‍ പെട്ടെന്നുതന്നെ അവര്‍ അതേക്കുറിച്ച് ബോധവാന്മാരായിത്തീരുന്നു. അപ്പോഴവര്‍ തികഞ്ഞ ഉള്‍ക്കാഴ്ചയുള്ളവരായി മാറും.
\end{malayalam}}
\flushright{\begin{Arabic}
\quranayah[7][202]
\end{Arabic}}
\flushleft{\begin{malayalam}
എന്നാല്‍ പിശാചുക്കളുടെ സഹോദരന്മാരെ അവര്‍ ദുര്‍മാര്‍ഗത്തിലൂടെ വിഹരിക്കാന്‍ വിടുന്നു. പിന്നെ അവരതിലൊട്ടും കുറവ് വരുത്തുകയില്ല.
\end{malayalam}}
\flushright{\begin{Arabic}
\quranayah[7][203]
\end{Arabic}}
\flushleft{\begin{malayalam}
നീ ഈ ജനത്തിന് ഒരു ദിവ്യാത്ഭുതവും കാണിച്ചുകൊടുത്തില്ലെങ്കില്‍ അവര്‍ പറയും: "നിനക്ക് സ്വയം തന്നെ ഒരു ദൃഷ്ടാന്തം തെരഞ്ഞെടുത്തുകൂടേ?” പറയുക: "എനിക്കെന്റെ നാഥനില്‍ നിന്ന് ബോധനമായി കിട്ടുന്ന സന്ദേശം പിന്‍പറ്റുക മാത്രമാണ് ഞാന്‍ ചെയ്യുന്നത്. ഇത് നിങ്ങളുടെ നാഥങ്കല്‍ നിന്നുള്ള വ്യക്തമായ വചനങ്ങളാണ്. ഒപ്പം വിശ്വസിക്കുന്ന ജനത്തിന് മാര്‍ഗദര്‍ശനവും അനുഗ്രഹവുമാണ്.”
\end{malayalam}}
\flushright{\begin{Arabic}
\quranayah[7][204]
\end{Arabic}}
\flushleft{\begin{malayalam}
ഖുര്‍ആന്‍ പാരായണം ചെയ്യുമ്പോള്‍ നിങ്ങളത് ശ്രദ്ധയോടെ കേള്‍ക്കുകയും മൌനം പാലിക്കുകയും ചെയ്യുക. നിങ്ങള്‍ക്ക് കാരുണ്യം കിട്ടിയേക്കാം.
\end{malayalam}}
\flushright{\begin{Arabic}
\quranayah[7][205]
\end{Arabic}}
\flushleft{\begin{malayalam}
നീ നിന്റെ നാഥനെ രാവിലെയും വൈകുന്നേരവും മനസ്സില്‍ സ്മരിക്കുക. അത് വിനയത്തോടെയും ഭയത്തോടെയുമാവണം. വാക്കുകള്‍ ഉറക്കെയാവാതെയും. നീ അതില്‍ അശ്രദ്ധ കാണിക്കുന്നവനാകരുത്.
\end{malayalam}}
\flushright{\begin{Arabic}
\quranayah[7][206]
\end{Arabic}}
\flushleft{\begin{malayalam}
നിന്റെ നാഥന്റെ അടുത്തുള്ളവര്‍ അവനെ വണങ്ങുന്ന കാര്യത്തില്‍ ഒരിക്കലും അഹങ്കരിക്കാറില്ല. അവര്‍ അവന്റെ വിശുദ്ധിയെ വാഴ്ത്തുന്നു. അവന് സാഷ്ടാംഗം പ്രണമിക്കുകയും ചെയ്യുന്നു.
\end{malayalam}}
\chapter{\textmalayalam{അന്‍ഫാല്‍ ( യുദ്ധമുതല്‍‍ )}}
\begin{Arabic}
\Huge{\centerline{\basmalah}}\end{Arabic}
\flushright{\begin{Arabic}
\quranayah[8][1]
\end{Arabic}}
\flushleft{\begin{malayalam}
യുദ്ധമുതലുകളെക്കുറിച്ച് അവര്‍ നിന്നോട് ചോദിക്കുന്നു. പറയുക: യുദ്ധമുതലുകള്‍ ദൈവത്തിനും ‎അവന്റെ ദൂതന്നുമുള്ളതാണ്. അതിനാല്‍ നിങ്ങള്‍ ദൈവഭക്തരാവുക. നിങ്ങള്‍ പരസ്പര ബന്ധം ‎മെച്ചപ്പെടുത്തുക. അല്ലാഹുവിനെയും അവന്റെ ദൂതനെയും അനുസരിക്കുക. നിങ്ങള്‍ ‎സത്യവിശ്വാസികളെങ്കില്‍! ‎
\end{malayalam}}
\flushright{\begin{Arabic}
\quranayah[8][2]
\end{Arabic}}
\flushleft{\begin{malayalam}
അല്ലാഹുവിന്റെ പേര്‍ കേള്ക്കു മ്പോള്‍ ഹൃദയം ഭയചകിതമാകുന്നവര്‍ മാത്രമാണ് യഥാര്ഥ് ‎വിശ്വാസികള്‍. അവന്റെ വചനങ്ങള്‍ വായിച്ചുകേട്ടാല്‍ അവരുടെ വിശ്വാസം വര്ധിാക്കും. അവര്‍ ‎എല്ലാം തങ്ങളുടെ നാഥനില്‍ സമര്പ്പി ക്കും. ‎
\end{malayalam}}
\flushright{\begin{Arabic}
\quranayah[8][3]
\end{Arabic}}
\flushleft{\begin{malayalam}
അവര്‍ നമസ്കാരം നിഷ്ഠയോടെ നിര്വ്ഹിക്കുന്നവരാണ്. നാം നല്കിസയതില്നിിന്ന് ‎ചെലവഴിക്കുന്നവരും. ‎
\end{malayalam}}
\flushright{\begin{Arabic}
\quranayah[8][4]
\end{Arabic}}
\flushleft{\begin{malayalam}
അവരാണ് യഥാര്ഥവ വിശ്വാസികള്‍. അവര്ക്ക് തങ്ങളുടെ നാഥന്റെയടുത്ത് ഉന്നത സ്ഥാനമുണ്ട്. ‎പാപമോചനവും ഉദാരമായ ഉപജീവനവുമുണ്ട്. ‎
\end{malayalam}}
\flushright{\begin{Arabic}
\quranayah[8][5]
\end{Arabic}}
\flushleft{\begin{malayalam}
ന്യായമായ കാരണത്താല്‍ നിന്റെ നാഥന്‍ നിന്നെ നിന്റെ വീട്ടില്‍ നിന്ന് ‎പുറത്തിറക്കിക്കൊണ്ടുപോയ പോലെയാണിത്. വിശ്വാസികളിലൊരു വിഭാഗം അതിഷ്ടപ്പെട്ടിരുന്നില്ല. ‎
\end{malayalam}}
\flushright{\begin{Arabic}
\quranayah[8][6]
\end{Arabic}}
\flushleft{\begin{malayalam}
സത്യം നന്നായി ബോധ്യമായിട്ടും അവര്‍ നിന്നോടു തര്ക്കി ക്കുകയായിരുന്നു. നോക്കിനില്ക്കെന ‎മരണത്തിലേക്ക് നയിക്കപ്പെടുന്നതുപോലെയായിരുന്നു അവരുടെ അവസ്ഥ. ‎
\end{malayalam}}
\flushright{\begin{Arabic}
\quranayah[8][7]
\end{Arabic}}
\flushleft{\begin{malayalam}
രണ്ടു സംഘങ്ങളില്‍ ഒന്നിനെ നിങ്ങള്ക്ക്അ കീഴ്പ്പെടുത്തിത്തരാമെന്ന് അല്ലാഹു നിങ്ങളോട് വാഗ്ദാനം ‎ചെയ്ത സന്ദര്ഭം‍. ആയുധമില്ലാത്ത സംഘത്തെ നിങ്ങള്ക്കു കിട്ടണമെന്നായിരുന്നു നിങ്ങളാഗ്രഹിച്ചത്. ‎എന്നാല്‍ അല്ലാഹു ഉദ്ദേശിച്ചത് തന്റെ കല്പ്നകള്‍ വഴി സത്യത്തെ സത്യമായി സ്ഥാപിക്കാനും ‎സത്യനിഷേധികളുടെ മുരട് മുറിച്ചുകളയാനുമാണ്. ‎
\end{malayalam}}
\flushright{\begin{Arabic}
\quranayah[8][8]
\end{Arabic}}
\flushleft{\begin{malayalam}
സത്യം സ്ഥാപിക്കാനും അസത്യത്തെ തൂത്തെറിയാനുമായിരുന്നു അത്. പാപികള്‍ അത് എത്രയേറെ ‎വെറുക്കുന്നുവെങ്കിലും! ‎
\end{malayalam}}
\flushright{\begin{Arabic}
\quranayah[8][9]
\end{Arabic}}
\flushleft{\begin{malayalam}
നിങ്ങള്‍ നിങ്ങളുടെ നാഥനോട് സഹായം തേടിയ സന്ദര്ഭം്. അപ്പോള്‍ അവന്‍ നിങ്ങള്ക്കുധ മറുപടി ‎നല്കിള, “ആയിരം മലക്കുകളെ തുടരെത്തുടരെ നിയോഗിച്ച് ഞാന്‍ നിങ്ങളെ സഹായിക്കാ”മെന്ന്. ‎
\end{malayalam}}
\flushright{\begin{Arabic}
\quranayah[8][10]
\end{Arabic}}
\flushleft{\begin{malayalam}
അല്ലാഹു ഇതു പറഞ്ഞത് നിങ്ങള്ക്കൊ്രു ശുഭവാര്ത്തെയായി ട്ടാണ്. അതിലൂടെ നിങ്ങള്ക്ക്ി ‎മനസ്സമാധാനം കിട്ടാനും. യഥാര്ഥക സഹായം അല്ലാഹുവില്‍ നിന്നു മാത്രമാണ്. അല്ലാഹു ‎പ്രതാപിയും യുക്തിമാനും തന്നെ. ‎
\end{malayalam}}
\flushright{\begin{Arabic}
\quranayah[8][11]
\end{Arabic}}
\flushleft{\begin{malayalam}
അല്ലാഹു തന്നില്നിതന്നുള്ള നിര്ഭഥയത്വം നല്കി മയക്കമേകുകയും മാനത്തുനിന്ന് മഴ വര്ഷികപ്പിച്ചു ‎തരികയും ചെയ്ത സന്ദര്ഭം. നിങ്ങളെ ശുദ്ധീകരിക്കാനും നിങ്ങളില്നി ന്ന് പൈശാചികമായ മ്ളേഛത ‎നീക്കിക്കളയാനുമായിരുന്നു അത്. ഒപ്പം നിങ്ങളുടെ മനസ്സുകളെ ഭദ്രമാക്കാനും കാലുകള്‍ ‎ഉറപ്പിച്ചുനിര്ത്താരനും. ‎
\end{malayalam}}
\flushright{\begin{Arabic}
\quranayah[8][12]
\end{Arabic}}
\flushleft{\begin{malayalam}
നിന്റെ നാഥന്‍ മലക്കുകള്ക്ക് ബോധനം നല്കിളയ സന്ദര്ഭംക: ഞാന്‍ നിങ്ങളോടൊപ്പമുണ്ട്. ‎അതിനാല്‍ സത്യവിശ്വാസികളെ നിങ്ങള്‍ ഉറപ്പിച്ചുനിര്ത്തുനക. സത്യനിഷേധികളുടെ മനസ്സുകളില്‍ ‎ഞാന്‍ ഭീതിയുളവാക്കും. അതിനാല്‍ അവരുടെ കഴുത്തുകള്ക്കു മീതെ വെട്ടുക. അവരുടെ എല്ലാ ‎വിരലുകളും വെട്ടിമാറ്റുക. ‎
\end{malayalam}}
\flushright{\begin{Arabic}
\quranayah[8][13]
\end{Arabic}}
\flushleft{\begin{malayalam}
അവര്‍ അല്ലാഹുവെയും അവന്റെ ദൂതനെയും ശത്രുതയോടെ എതിര്ത്ത തിനാലാണിത്. ആരെങ്കിലും ‎അല്ലാഹുവോടും അവന്റെ ദൂതനോടും ശത്രുത പുലര്ത്തു ന്നുവെങ്കില്‍ അറിയുക: അല്ലാഹു ‎കഠിനമായി ശിക്ഷിക്കുന്നവനാണ്. ‎
\end{malayalam}}
\flushright{\begin{Arabic}
\quranayah[8][14]
\end{Arabic}}
\flushleft{\begin{malayalam}
അതാണ് നിങ്ങള്ക്കുള്ള ശിക്ഷ. അതിനാല്‍ നിങ്ങളതനുഭവിച്ചുകൊള്ളുക. അറിയുക: ‎സത്യനിഷേധികള്ക്ക് കഠിനമായ നരകശിക്ഷയുമുണ്ട്. ‎
\end{malayalam}}
\flushright{\begin{Arabic}
\quranayah[8][15]
\end{Arabic}}
\flushleft{\begin{malayalam}
വിശ്വസിച്ചവരേ, സത്യനിഷേധികളുടെ സൈന്യവുമായി ഏറ്റുമുട്ടേണ്ടിവരുമ്പോള്‍ നിങ്ങള്‍ ‎പിന്തിരിഞ്ഞോടരുത്. ‎
\end{malayalam}}
\flushright{\begin{Arabic}
\quranayah[8][16]
\end{Arabic}}
\flushleft{\begin{malayalam}
യുദ്ധതന്ത്രമെന്ന നിലയില്‍ സ്ഥലം മാറുന്നതിനോ സ്വന്തം സംഘത്തോടൊപ്പം ചേരുന്നതിനോ അല്ലാതെ ‎ആരെങ്കിലും യുദ്ധരംഗത്തുനിന്ന് പിന്തിരിയുകയാണെങ്കില്‍ അവന്‍ അല്ലാഹുവിന്റെ ‎കോപത്തിനിരയാകും. അവന്‍ ചെന്നെത്തുന്നത് നരകത്തീയിലായിരിക്കും. അതെത്ര ചീത്ത സങ്കേതം! ‎
\end{malayalam}}
\flushright{\begin{Arabic}
\quranayah[8][17]
\end{Arabic}}
\flushleft{\begin{malayalam}
സത്യത്തില്‍ അവരെ വധിച്ചത് നിങ്ങളല്ല, അല്ലാഹുവാണ്. നീ എറിഞ്ഞപ്പോള്‍ യഥാര്ഥരത്തില്‍ നീയല്ല ‎എറിഞ്ഞത്. അല്ലാഹുവാണ്. മഹത്തായ ഒരു പരീക്ഷണത്തിലൂടെ അല്ലാഹു സത്യവിശ്വാസികളെ ‎വേര്തിതരിച്ചെടുക്കാനാണിത്. അല്ലാഹു എല്ലാം കേള്ക്കുലന്നവനും അറിയുന്നവനുമാണ്. ‎
\end{malayalam}}
\flushright{\begin{Arabic}
\quranayah[8][18]
\end{Arabic}}
\flushleft{\begin{malayalam}
അതാണ് നിങ്ങളോടുള്ള നിലപാട്. സംശയമില്ല; സത്യനിഷേധികളുടെ തന്ത്രത്തെ ‎ദുര്ബ ലമാക്കുന്നവനാണ് അല്ലാഹു. ‎
\end{malayalam}}
\flushright{\begin{Arabic}
\quranayah[8][19]
\end{Arabic}}
\flushleft{\begin{malayalam}
നിങ്ങള്‍ വിജയമാണ് ആഗ്രഹിച്ചിരുന്നതെങ്കില്‍ ആ വിജയമിതാ നിങ്ങള്ക്കുു വന്നെത്തിയിരിക്കുന്നു. ‎അഥവാ, നിങ്ങള്‍ അതിക്രമത്തില്‍ നിന്ന് വിരമിക്കുകയാണെങ്കില്‍ അതാണ് നിങ്ങള്ക്കു്ത്തമം. നിങ്ങള്‍ ‎അതാവര്ത്തിങക്കുകയാണെങ്കില്‍ നാമും അതാവര്ത്തി ക്കും. നിങ്ങളുടെ സംഘബലം എത്ര ‎വലുതായാലും അത് നിങ്ങള്ക്കൊണട്ടും ഉപകരിക്കുകയില്ല. അല്ലാഹു സത്യവിശ്വാസികള്ക്കൊണപ്പമാണ്; ‎തീര്ച്ചല. ‎
\end{malayalam}}
\flushright{\begin{Arabic}
\quranayah[8][20]
\end{Arabic}}
\flushleft{\begin{malayalam}
വിശ്വസിച്ചവരേ, നിങ്ങള്‍ അല്ലാഹുവെയും അവന്റെ ദൂതനെയും അനുസരിക്കുക. ‎അദ്ദേഹത്തില്നിചന്ന് സന്ദേശം കേട്ടുകൊണ്ടിരിക്കെ നിങ്ങള്‍ അദ്ദേഹത്തെ വിട്ട് പിന്തിരിഞ്ഞു ‎പോകരുത്. ‎
\end{malayalam}}
\flushright{\begin{Arabic}
\quranayah[8][21]
\end{Arabic}}
\flushleft{\begin{malayalam}
ഒന്നും കേള്ക്കാനതെ “ഞങ്ങള്‍ കേള്ക്കു ന്നുണ്ടെ”ന്ന് പറയുന്നവരെപ്പോലെയുമാവരുത് നിങ്ങള്‍. ‎
\end{malayalam}}
\flushright{\begin{Arabic}
\quranayah[8][22]
\end{Arabic}}
\flushleft{\begin{malayalam}
തീര്ച്ചകയായും അല്ലാഹുവിങ്കല്‍ ഏറ്റം നികൃഷ്ടജീവികള്‍ ഒന്നും ചിന്തിച്ചു മനസ്സിലാക്കാത്ത ഊമകളും ‎ബധിരരുമാണ്. ‎
\end{malayalam}}
\flushright{\begin{Arabic}
\quranayah[8][23]
\end{Arabic}}
\flushleft{\begin{malayalam}
അവരില്‍ എന്തെങ്കിലും നന്മയുള്ളതായി അല്ലാഹു മനസ്സിലാക്കിയിരുന്നെങ്കില്‍ അവന്‍ അവരെ ‎കാര്യം കേട്ടറിയുന്നവരാക്കുമായിരുന്നു. എന്നാല്‍, അവരില്‍ നന്മ ഒട്ടും ഇല്ലാത്തതിനാല്‍ അവന്‍ ‎കേള്പ്പി ച്ചാല്പ്പോ്ലും അവരത് അവഗണിച്ച് തിരിഞ്ഞുപോകുമായിരുന്നു. ‎
\end{malayalam}}
\flushright{\begin{Arabic}
\quranayah[8][24]
\end{Arabic}}
\flushleft{\begin{malayalam}
വിശ്വസിച്ചവരേ, നിങ്ങളെ ജീവസ്സുറ്റവരാക്കുന്ന ഒന്നിലേക്ക് വിളിക്കുമ്പോള്‍ അല്ലാഹുവിനും ‎അവന്റെ ദൂതന്നും നിങ്ങള്‍ ഉത്തരം നല്കുുക. മനുഷ്യന്നും അവന്റെ മനസ്സിനുമിടയില്‍ അല്ലാഹു ‎ഉണ്ട്. അവസാനം അവന്റെ അടുത്തേക്കാണ് നിങ്ങളെ ഒരുമിച്ചുകൂട്ടുക. ‎
\end{malayalam}}
\flushright{\begin{Arabic}
\quranayah[8][25]
\end{Arabic}}
\flushleft{\begin{malayalam}
വിപത്ത് വരുന്നത് കരുതിയിരിക്കുക: അതു ബാധിക്കുക നിങ്ങളിലെ അതിക്രമികളെ മാത്രമല്ല. ‎അറിയുക: കഠിനമായി ശിക്ഷിക്കുന്നവനാണ് അല്ലാഹു. ‎
\end{malayalam}}
\flushright{\begin{Arabic}
\quranayah[8][26]
\end{Arabic}}
\flushleft{\begin{malayalam}
ഓര്ക്കു ക: നിങ്ങള്‍ എണ്ണത്തില്‍ വളരെ കുറവായിരുന്ന കാലം! ഭൂമിയില്‍ നിങ്ങളന്ന് നന്നെ ‎ദുര്ബകലരായാണ് കരുതപ്പെട്ടിരുന്നത്. ആളുകള്‍ നിങ്ങളെ റാഞ്ചിയെടുത്തേക്കുമോയെന്നുപോലും ‎നിങ്ങള്‍ ഭയപ്പെട്ടിരുന്നു. പിന്നീട് അല്ലാഹു നിങ്ങള്ക്ക് അഭയമേകി. തന്റെ സഹായത്താല്‍ നിങ്ങളെ ‎പ്രബലരാക്കി. നിങ്ങള്ക്ക്ു ഉത്തമമായ ജീവിതവിഭവങ്ങള്‍ നല്കിഅ. നിങ്ങള്‍ നന്ദിയുള്ളവരാകാന്‍. ‎
\end{malayalam}}
\flushright{\begin{Arabic}
\quranayah[8][27]
\end{Arabic}}
\flushleft{\begin{malayalam}
വിശ്വസിച്ചവരേ, നിങ്ങള്‍ അല്ലാഹുവെയും അവന്റെ ദൂതനെയും ചതിക്കരുത്. നിങ്ങളെ ‎വിശ്വസിച്ചേല്പിരച്ച കാര്യങ്ങളില്‍ ബോധപൂര്വംറ വഞ്ചന കാണിക്കരുത്. ‎
\end{malayalam}}
\flushright{\begin{Arabic}
\quranayah[8][28]
\end{Arabic}}
\flushleft{\begin{malayalam}
അറിയുക: നിങ്ങളുടെ സമ്പത്തും സന്താനങ്ങളും പരീക്ഷണോപാധികള്‍ മാത്രമാണ്. ‎അല്ലാഹുവിങ്കലാണ് അതിമഹത്തായ പ്രതിഫലമുള്ളത്. ‎
\end{malayalam}}
\flushright{\begin{Arabic}
\quranayah[8][29]
\end{Arabic}}
\flushleft{\begin{malayalam}
വിശ്വസിച്ചവരേ, നിങ്ങള്‍ അല്ലാഹുവെ സൂക്ഷിക്കുക. എങ്കില്‍ അവന്‍ നിങ്ങള്ക്ക്് സത്യാസത്യങ്ങളെ ‎വേര്തിുരിച്ചറിയാനുള്ള കഴിവ് നല്കുംഫ. നിങ്ങളുടെ തിന്മകള്‍ മായ്ച്ചുകളയും. നിങ്ങള്ക്ക്് ‎മാപ്പേകുകയും ചെയ്യും. അല്ലാഹു അതിമഹത്തായ അനുഗ്രഹമുള്ളവനാണ്. ‎
\end{malayalam}}
\flushright{\begin{Arabic}
\quranayah[8][30]
\end{Arabic}}
\flushleft{\begin{malayalam}
നിന്നെ തടവിലാക്കാനോ കൊന്നുകളയാനോ നാടുകടത്താനോ സത്യനിഷേധികള്‍ നിനക്കെതിരെ തന്ത്രം ‎മെനഞ്ഞ സന്ദര്ഭം . അവര്‍ തന്ത്രം പ്രയോഗിക്കുന്നു. അല്ലാഹുവും തന്ത്രം പ്രയോഗിക്കുന്നു. തന്ത്രം ‎പ്രയോഗിക്കുന്നവരില്‍ മികവുറ്റവന്‍ അല്ലാഹു തന്നെ. ‎
\end{malayalam}}
\flushright{\begin{Arabic}
\quranayah[8][31]
\end{Arabic}}
\flushleft{\begin{malayalam}
നമ്മുടെ വചനങ്ങള്‍ ഓതിക്കേള്പ്പി ച്ചാല്‍ അവര്‍ പറയും: "ഇതൊക്കെ ഞങ്ങളെത്രയോ കേട്ടതാണ്. ‎ഞങ്ങളുദ്ദേശിക്കുകയാണെങ്കില്‍ ഇതുപോലെ ഞങ്ങളും പറഞ്ഞുതരുമായിരുന്നു. ഇത് പൂര്വിൊകരുടെ ‎പഴമ്പുരാണങ്ങളല്ലാതൊന്നുമല്ല.” ‎
\end{malayalam}}
\flushright{\begin{Arabic}
\quranayah[8][32]
\end{Arabic}}
\flushleft{\begin{malayalam}
അവര്‍ ഇങ്ങനെ പറഞ്ഞ സന്ദര്ഭ”വും ഓര്ക്കുക: "അല്ലാഹുവേ, ഇത് നിന്റെപക്കല്‍ നിന്നുള്ള ‎സത്യം തന്നെയാണെങ്കില്‍ നീ ഞങ്ങളുടെമേല്‍ മാനത്തുനിന്ന് കല്ല് വീഴ്ത്തുക. അല്ലെങ്കില്‍ ഞങ്ങള്ക്ക്ള ‎നോവേറിയ ശിക്ഷ വരുത്തുക.” ‎
\end{malayalam}}
\flushright{\begin{Arabic}
\quranayah[8][33]
\end{Arabic}}
\flushleft{\begin{malayalam}
എന്നാല്‍, നീ അവര്ക്കി ടയിലുണ്ടായിരിക്കെ അല്ലാഹു അവരെ ശിക്ഷിക്കുകയില്ല. അവര്‍ ‎പാപമോചനം തേടുമ്പോഴും അല്ലാഹു അവരെ ശിക്ഷിക്കുകയില്ല. ‎
\end{malayalam}}
\flushright{\begin{Arabic}
\quranayah[8][34]
\end{Arabic}}
\flushleft{\begin{malayalam}
എന്നാല്‍ ഇപ്പോള്‍ എന്തിന് അല്ലാഹു അവരെ ശിക്ഷിക്കാതിരിക്കണം? അവര്‍ മസ്ജിദുല്‍ ഹറാമില്‍ ‎നിന്ന് വിശ്വാസികളെ തടഞ്ഞുകൊണ്ടിരിക്കുന്നു. അവരാണെങ്കില്‍ അതിന്റെ ‎മേല്നോ്ട്ടത്തിനര്ഹളരല്ലതാനും. ദൈവഭക്തന്മാരല്ലാതെ അതിന്റെ കൈകാര്യകര്ത്താ ക്കളാകാവതല്ല. ‎എങ്കിലും അവരിലേറെപ്പേരും അതറിയുന്നില്ല. ‎
\end{malayalam}}
\flushright{\begin{Arabic}
\quranayah[8][35]
\end{Arabic}}
\flushleft{\begin{malayalam}
ആ ഭവനത്തിങ്കല്‍ അവരുടെ പ്രാര്ഥകന വെറും ചൂളംവിളിയും കൈകൊട്ടുമല്ലാതൊന്നുമല്ല. അതിനാല്‍ ‎നിങ്ങള്‍ സത്യനിഷേധം സ്വീകരിച്ചതിന്റെ ശിക്ഷ അനുഭവിച്ചുകൊള്ളുക. ‎
\end{malayalam}}
\flushright{\begin{Arabic}
\quranayah[8][36]
\end{Arabic}}
\flushleft{\begin{malayalam}
സത്യനിഷേധികള്‍ തങ്ങളുടെ ധനം ചെലവഴിക്കുന്നത് തീര്ച്ചളയായും അല്ലാഹുവിന്റെ മാര്ഗയത്തില്‍ ‎നിന്ന് ജനങ്ങളെ തടയാനാണ്. ഇനിയും അവരത് ചെലവഴിച്ചുകൊണ്ടേയിരിക്കും. അവസാനം ‎അതവരുടെ തന്നെ ഖേദത്തിനു കാരണമായിത്തീരും. അങ്ങനെയവര്‍ തീര്ത്തും പരാജിതരാവും. ‎ഒടുവില്‍ ഈ സത്യനിഷേധികളെ നരകത്തീയില്‍ ഒരുമിച്ചു കൂട്ടും. ‎
\end{malayalam}}
\flushright{\begin{Arabic}
\quranayah[8][37]
\end{Arabic}}
\flushleft{\begin{malayalam}
അല്ലാഹു നന്മയില്‍ നിന്ന് തിന്മയെ വേര്തി്രിച്ചെടുക്കും. പിന്നെ സകല തിന്മകളെയും ‎പരസ്പരം കൂട്ടിച്ചേര്ത്ത് കൂമ്പാരമാക്കും. അങ്ങനെയതിനെ നരകത്തീയില്‍ തള്ളിയിടും. സത്യത്തില്‍ ‎അക്കൂട്ടര്‍ തന്നെയാണ് എല്ലാം നഷ്ടപ്പെട്ടവര്‍. ‎
\end{malayalam}}
\flushright{\begin{Arabic}
\quranayah[8][38]
\end{Arabic}}
\flushleft{\begin{malayalam}
സത്യനിഷേധികളോടു പറയുക: ഇനിയെങ്കിലുമവര്‍ വിരമിക്കുകയാണെങ്കില്‍ മുമ്പ് കഴിഞ്ഞതൊക്കെ ‎അവര്ക്കു പൊറുത്തുകൊടുക്കും. അഥവാ, അവര്‍ പഴയത് ആവര്ത്തി ക്കുകയാണെങ്കില്‍ അവര്‍ ‎ഓര്ക്കഷട്ടെ; പൂര്വികകരുടെ കാര്യത്തില്‍ അല്ലാഹുവിന്റെ നടപടിക്രമം നടന്നു കഴിഞ്ഞതാണല്ലോ. ‎
\end{malayalam}}
\flushright{\begin{Arabic}
\quranayah[8][39]
\end{Arabic}}
\flushleft{\begin{malayalam}
കുഴപ്പം ഇല്ലാതാവുകയും വിധേയത്വം പൂര്ണ്മായും അല്ലാഹുവിനായിത്തീരുകയും ‎ചെയ്യുന്നതുവരെ നിങ്ങളവരോടു യുദ്ധം ചെയ്യുക. അവര്‍ വിരമിക്കുകയാണെങ്കിലോ, അവര്‍ ‎ചെയ്യുന്നതെല്ലാം നന്നായറിയുന്നവനാണ് അല്ലാഹു. ‎
\end{malayalam}}
\flushright{\begin{Arabic}
\quranayah[8][40]
\end{Arabic}}
\flushleft{\begin{malayalam}
അഥവാ അവര്‍ നിരാകരിക്കുകയാണെങ്കില്‍ അറിയുക: തീര്ച്ചമയായും നിങ്ങളുടെ രക്ഷകന്‍ ‎അല്ലാഹുവാണ്. അവന്‍ വളരെ നല്ല രക്ഷകനും സഹായിയുമാണ്. ‎
\end{malayalam}}
\flushright{\begin{Arabic}
\quranayah[8][41]
\end{Arabic}}
\flushleft{\begin{malayalam}
അറിയുക: നിങ്ങള്‍ നേടിയ യുദ്ധമുതല്‍ എന്തായാലും അതിന്റെ അഞ്ചിലൊന്ന് അല്ലാഹുവിനും ‎അവന്റെ ദൂതന്നും അടുത്ത ബന്ധുക്കള്ക്കും അനാഥകള്ക്കും അഗതികള്ക്കും ‎വഴിപോക്കര്ക്കുംമുള്ളതാണ്; അല്ലാഹുവിലും, ഇരുസംഘങ്ങള്‍ പരസ്പരം ഏറ്റുമുട്ടിയതിലൂടെ ‎സത്യാസത്യങ്ങള്‍ വ്യക്തമായി വേര്തിലരിഞ്ഞ നാളില്‍ നാം നമ്മുടെ ദാസന്ന് ഇറക്കിക്കൊടുത്തതിലും ‎വിശ്വസിച്ചവരാണ് നിങ്ങളെങ്കില്‍! അല്ലാഹു എല്ലാ കാര്യത്തിനും കഴിവുറ്റവനത്രെ. ‎
\end{malayalam}}
\flushright{\begin{Arabic}
\quranayah[8][42]
\end{Arabic}}
\flushleft{\begin{malayalam}
നിങ്ങള്‍ താഴ്വരയുടെ അടുത്ത ഭാഗത്തും അവര്‍ അകന്ന ഭാഗത്തും കച്ചവടസംഘം നിങ്ങള്ക്കു ‎താഴെയുമായ സന്ദര്ഭംങ. നിങ്ങള്‍ പരസ്പരം ഏറ്റുമുട്ടാന്‍ നേരത്തെ തീരുമാനിച്ചിരുന്നെങ്കില്‍ ‎നിങ്ങളതിനു വിരുദ്ധമായി പ്രവര്ത്തിങക്കുമായിരുന്നു. എന്നാല്‍ ഉറപ്പായും ഉണ്ടാകേണ്ട ഒരു കാര്യം ‎നടപ്പില്‍ വരുത്താനാണ് അല്ലാഹു ഇങ്ങനെ ചെയ്തത്. അഥവാ നശിക്കേണ്ടവന്‍ വ്യക്തമായ ‎തെളിവോടെ നശിക്കാനും ജീവിക്കേണ്ടവന്‍ വ്യക്തമായ തെളിവോടെ ജീവിക്കാനും വേണ്ടിയാണിത്. ‎അല്ലാഹു എല്ലാം കേള്ക്കുിന്നവനും അറിയുന്നവനും തന്നെ; തീര്ച്ചേ. ‎
\end{malayalam}}
\flushright{\begin{Arabic}
\quranayah[8][43]
\end{Arabic}}
\flushleft{\begin{malayalam}
അല്ലാഹു സ്വപ്നത്തിലൂടെ അവരെ വളരെ കുറച്ചുപേര്‍ മാത്രമായി നിനക്ക് കാണിച്ചുതന്ന ‎സന്ദര്ഭംസ. നിനക്ക് അവരെ എണ്ണക്കൂടുതലുള്ളതായി കാണിച്ചു തന്നിരുന്നെങ്കില്‍ ഉറപ്പായും ‎നിങ്ങള്ക്ക്പ ധൈര്യക്ഷയമുണ്ടാകുമായിരുന്നു. യുദ്ധത്തിന്റെ കാര്യത്തില്‍ നിങ്ങള്‍ ഭിന്നിക്കുകയും ‎ചെയ്യുമായിരുന്നു. എന്നാല്‍ അല്ലാഹു രക്ഷിച്ചു. തീര്ച്ചതയായും മനസ്സുകളിലുള്ളതെല്ലാം ‎അറിയുന്നവനാണ് അവന്‍. ‎
\end{malayalam}}
\flushright{\begin{Arabic}
\quranayah[8][44]
\end{Arabic}}
\flushleft{\begin{malayalam}
നിങ്ങള്‍ തമ്മില്‍ കണ്ടുമുട്ടിയപ്പോള്‍ നിങ്ങളുടെ കണ്ണില്‍ അവരെ കുറച്ചു കാണിച്ചതും അവരുടെ ‎കണ്ണില്‍ നിങ്ങളെ കുറച്ചു കാണിച്ചതും ഓര്ക്കുപക. സംഭവിക്കേണ്ട കാര്യം നടപ്പാക്കാന്‍ അല്ലാഹു ‎പ്രയോഗിച്ച തന്ത്രമായിരുന്നു അത്. കാര്യങ്ങളൊക്കെയും മടക്കപ്പെടുക അല്ലാഹുവിങ്കലേക്കാണ്. ‎
\end{malayalam}}
\flushright{\begin{Arabic}
\quranayah[8][45]
\end{Arabic}}
\flushleft{\begin{malayalam}
വിശ്വസിച്ചവരേ, നിങ്ങള്‍ ശത്രു സംഘവുമായി സന്ധിച്ചാല്‍ സ്ഥൈര്യത്തോടെ നിലകൊള്ളുക. ‎ദൈവത്തെ ധാരാളമായി സ്മരിക്കുക. നിങ്ങള്‍ വിജയം വരിച്ചേക്കാം. ‎
\end{malayalam}}
\flushright{\begin{Arabic}
\quranayah[8][46]
\end{Arabic}}
\flushleft{\begin{malayalam}
അല്ലാഹുവിനെയും അവന്റെ ദൂതനെയും അനുസരിക്കുക. നിങ്ങളന്യോന്യം കലഹിക്കരുത്. ‎അങ്ങനെ സംഭവിച്ചാല്‍ നിങ്ങള്‍ ദുര്ബംലരാകും. നിങ്ങളുടെ കാറ്റുപോകും. നിങ്ങള്‍ ക്ഷമിക്കൂ. ‎അല്ലാഹു ക്ഷമാശീലരോടൊപ്പമാണ്. ‎
\end{malayalam}}
\flushright{\begin{Arabic}
\quranayah[8][47]
\end{Arabic}}
\flushleft{\begin{malayalam}
അഹങ്കാരത്തോടെയും ജനങ്ങളെ കാണിക്കാനും അല്ലാഹുവിന്റെ മാര്ഗയത്തില്നിംന്ന് ജനത്തെ ‎തടയാനുമായി വീട് വിട്ടിറങ്ങിപ്പോന്നവരെപ്പോലെ നിങ്ങളാകരുത്. അവര്‍ ചെയ്യുന്നതൊക്കെയും ‎നന്നായി നിരീക്ഷിക്കുന്നവനാണ് അല്ലാഹു. ‎
\end{malayalam}}
\flushright{\begin{Arabic}
\quranayah[8][48]
\end{Arabic}}
\flushleft{\begin{malayalam}
ചെകുത്താന്‍ അവര്ക്ക് അവരുടെ ചെയ്തികള്‍ ചേതോഹരമായി തോന്നിപ്പിച്ച സന്ദര്ഭംം. അവന്‍ ‎പറഞ്ഞു: "ഇന്ന് നിങ്ങളെ ജയിക്കുന്നവരായി ജനങ്ങളിലാരുമില്ല. ഉറപ്പായും ഞാന്‍ നിങ്ങളുടെ ‎രക്ഷകനായിരിക്കും.” അങ്ങനെ ഇരുപക്ഷവും ഏറ്റുമുട്ടിയപ്പോള്‍ അവന്‍ പിന്മാറി. എന്നിട്ടിങ്ങനെ ‎പറയുകയും ചെയ്തു: "എനിക്ക് നിങ്ങളുമായി ഒരു ബന്ധവുമില്ല. നിങ്ങള്‍ കാണാത്തത് ഞാന്‍ ‎കാണുന്നുണ്ട്. ഞാന്‍ അല്ലാഹുവെ ഭയപ്പെടുന്നു. അല്ലാഹു കഠിനമായി ശിക്ഷിക്കുന്നവനാണല്ലോ.” ‎
\end{malayalam}}
\flushright{\begin{Arabic}
\quranayah[8][49]
\end{Arabic}}
\flushleft{\begin{malayalam}
കപടവിശ്വാസികളും ദീനംബാധിച്ച മനസ്സുള്ളവരും പറഞ്ഞുകൊണ്ടിരുന്ന സന്ദര്ഭംഅ: "ഇക്കൂട്ടരെ ‎അവരുടെ മതം വഞ്ചിച്ചിരിക്കുന്നു.” ആരെങ്കിലും അല്ലാഹുവില്‍ ഭരമേല്പി"ക്കുന്നുവെങ്കില്‍, ‎സംശയം വേണ്ട, അല്ലാഹു അജയ്യനും യുക്തിമാനുമാണ്. ‎
\end{malayalam}}
\flushright{\begin{Arabic}
\quranayah[8][50]
\end{Arabic}}
\flushleft{\begin{malayalam}
സത്യനിഷേധികളെ മരിപ്പിക്കുന്ന രംഗം നീ കണ്ടിരുന്നെങ്കില്‍! മലക്കുകള്‍ അവരുടെ മുഖത്തും ‎പിന്ഭാഷഗത്തും അടിക്കും. അവരോടിങ്ങനെ പറയുകയും ചെയ്യും: "കരിച്ചുകളയുന്ന ‎നരകത്തീയിന്റെ കൊടിയ ശിക്ഷ നിങ്ങള്‍ അനുഭവിച്ചുകൊള്ളുക.” ‎
\end{malayalam}}
\flushright{\begin{Arabic}
\quranayah[8][51]
\end{Arabic}}
\flushleft{\begin{malayalam}
നിങ്ങളുടെ കൈകള്‍ നേരത്തെ ചെയ്തുകൂട്ടിയതിന്റെ ഫലമാണിത്. അല്ലാഹു തന്റെ അടിമകളോട് ‎ഒട്ടും അനീതി കാണിക്കുന്നവനല്ല. ‎
\end{malayalam}}
\flushright{\begin{Arabic}
\quranayah[8][52]
\end{Arabic}}
\flushleft{\begin{malayalam}
ഇത് ഫറവോന്സംകഘത്തിനും അവരുടെ മുമ്പുള്ളവര്ക്കും സംഭവിച്ചപോലെത്തന്നെയാണ്. അവര്‍ ‎അല്ലാഹുവിന്റെ വചനങ്ങളെ തള്ളിപ്പറഞ്ഞു. അപ്പോള്‍ അവരുടെ പാപങ്ങളുടെ പേരില്‍ അല്ലാഹു ‎അവരെ പിടികൂടി. തീര്ച്ചഅയായും അല്ലാഹു സര്വോശക്തനാണ്. കഠിനമായി ശിക്ഷിക്കുന്നവനും. ‎
\end{malayalam}}
\flushright{\begin{Arabic}
\quranayah[8][53]
\end{Arabic}}
\flushleft{\begin{malayalam}
ഒരു ജനത തങ്ങളുടെ നിലപാട് സ്വയം മാറ്റുന്നതുവരെ അല്ലാഹു ആ ജനതയ്ക്കു ചെയ്തുകൊടുത്ത ‎അനുഗ്രഹത്തില്‍ ഒരു മാറ്റവും വരുത്തുകയില്ല. സംശയമില്ല; അല്ലാഹു എല്ലാം കേള്ക്കുനന്നവനും ‎അറിയുന്നവനുമാണ്. ‎
\end{malayalam}}
\flushright{\begin{Arabic}
\quranayah[8][54]
\end{Arabic}}
\flushleft{\begin{malayalam}
ഫറവോന്‍ സംഘത്തിനും അവര്ക്കു മുമ്പുള്ളവര്ക്കും സംഭവിച്ചതും ഇതുപോലെത്തന്നെയാണ്. ‎അവര്‍ തങ്ങളുടെ നാഥന്റെ പ്രമാണങ്ങളെ തള്ളിപ്പറഞ്ഞു. അപ്പോള്‍ അവരുടെ പാപങ്ങളുടെ ‎പേരില്‍ നാം അവരെ നശിപ്പിച്ചു. ഫറവോന്‍ സംഘത്തെ മുക്കിക്കൊന്നു. അവരൊക്കെയും ‎അക്രമികളായിരുന്നു. ‎
\end{malayalam}}
\flushright{\begin{Arabic}
\quranayah[8][55]
\end{Arabic}}
\flushleft{\begin{malayalam}
തീര്ച്ചളയായും അല്ലാഹുവിന്റെ അടുക്കല്‍ ഏറ്റം നികൃഷ്ടജീവികള്‍ സത്യനിഷേധികളാണ്. സത്യം ‎ബോധ്യപ്പെട്ടാലും വിശ്വസിക്കാത്തവരാണവര്‍. ‎
\end{malayalam}}
\flushright{\begin{Arabic}
\quranayah[8][56]
\end{Arabic}}
\flushleft{\begin{malayalam}
അവരിലൊരു വിഭാഗവുമായി നീ കരാറിലേര്പ്പെ ട്ടതാണല്ലോ. എന്നിട്ട് ഓരോ തവണയും അവര്‍ ‎തങ്ങളുടെ കരാര്‍ ലംഘിച്ചുകൊണ്ടിരുന്നു. അവരൊട്ടും സൂക്ഷ്മത പുലര്ത്തു ന്നവരല്ല. ‎
\end{malayalam}}
\flushright{\begin{Arabic}
\quranayah[8][57]
\end{Arabic}}
\flushleft{\begin{malayalam}
അതിനാല്‍ നീ യുദ്ധത്തില്‍ അവരുമായി സന്ധിച്ചാല്‍ അവരിലെ പിറകിലുള്ളവരെക്കൂടി ‎വിരട്ടിയോടിക്കുംവിധം അവരെ നേരിടുക. അവര്ക്ക്തൊരു പാഠമായെങ്കിലോ. ‎
\end{malayalam}}
\flushright{\begin{Arabic}
\quranayah[8][58]
\end{Arabic}}
\flushleft{\begin{malayalam}
ഉടമ്പടിയിലേര്പ്പെ ട്ട ഏതെങ്കിലും ജനത നിങ്ങളെ വഞ്ചിക്കുമെന്ന് നിങ്ങളാശങ്കിക്കുന്നുവെങ്കില്‍ ‎അവരുമായുള്ള കരാര്‍ പരസ്യമായി ദുര്ബലലപ്പെടുത്തുക. വഞ്ചകരെ അല്ലാഹു ഇഷ്ടപ്പെടുന്നില്ല; ‎തീര്ച്ചു. ‎
\end{malayalam}}
\flushright{\begin{Arabic}
\quranayah[8][59]
\end{Arabic}}
\flushleft{\begin{malayalam}
സത്യനിഷേധികള്‍ തങ്ങള്‍ ജയിച്ചു മുന്നേറുകയാണെന്ന് ധരിക്കരുത്. സംശയമില്ല; അവര്ക്കു നമ്മെ ‎തോല്പ്പി ക്കാനാവില്ല. ‎
\end{malayalam}}
\flushright{\begin{Arabic}
\quranayah[8][60]
\end{Arabic}}
\flushleft{\begin{malayalam}
അവരെ നേരിടാന്‍ നിങ്ങള്ക്കാലവുന്നത്ര ശക്തി സംഭരിക്കുക. കുതിരപ്പടയെ തയ്യാറാക്കി നിര്ത്തു ക. ‎അതിലൂടെ അല്ലാഹുവിന്റെയും നിങ്ങളുടെയും ശത്രുക്കളെ നിങ്ങള്ക്ക് ഭയപ്പെടുത്താം. ‎അവര്ക്കു പുറമെ നിങ്ങള്ക്ക് അറിയാത്തവരും എന്നാല്‍ അല്ലാഹുവിന് അറിയുന്നവരുമായ മറ്റു ‎ചിലരെയും. അല്ലാഹുവിന്റെ മാര്ഗ്ത്തില്‍ നിങ്ങള്‍ ചെലവഴിക്കുന്നതെന്തായാലും നിങ്ങള്ക്ക് ‎അതിന്റെ പ്രതിഫലം പൂര്ണ്മായി ലഭിക്കും. നിങ്ങളോടവന്‍ ഒട്ടും അനീതി കാണിക്കുകയില്ല. ‎
\end{malayalam}}
\flushright{\begin{Arabic}
\quranayah[8][61]
\end{Arabic}}
\flushleft{\begin{malayalam}
അഥവാ അവര്‍ സന്ധിക്കു സന്നദ്ധരായാല്‍ നീയും അതിനനുകൂലമായ നിലപാടെടുക്കുക. ‎അല്ലാഹുവില്‍ ഭരമേല്പി ക്കുകയും ചെയ്യുക. തീര്ച്ചായായും അവന്‍ തന്നെയാണ് എല്ലാം ‎കേള്ക്കു ന്നവനും അറിയുന്നവനും. ‎
\end{malayalam}}
\flushright{\begin{Arabic}
\quranayah[8][62]
\end{Arabic}}
\flushleft{\begin{malayalam}
ഇനി അവര്‍ നിന്നെ വഞ്ചിക്കാനാണ് ഉദ്ദേശിക്കുന്നതെങ്കില്‍ അറിയുക. തീര്ച്ചവയായും നിനക്ക് ‎അല്ലാഹു മതി. അവനാണ് തന്റെ സഹായത്താലും സത്യവിശ്വാസികളാലും നിനക്ക് ‎കരുത്തേകിയത്. ‎
\end{malayalam}}
\flushright{\begin{Arabic}
\quranayah[8][63]
\end{Arabic}}
\flushleft{\begin{malayalam}
സത്യവിശ്വാസികളുടെ മനസ്സുകള്ക്കി ടയില്‍ ഇണക്കമുണ്ടാക്കിയതും അവനാണ്. ‎ഭൂമിയിലുള്ളതൊക്കെ ചെലവഴിച്ചാലും അവരുടെ മനസ്സുകളെ കൂട്ടിയിണക്കാന്‍ നിനക്കു ‎കഴിയുമായിരുന്നില്ല. എന്നാല്‍ അല്ലാഹു അവരെ തമ്മിലിണക്കിച്ചേര്ത്തികരിക്കുന്നു. അവന്‍ ‎പ്രതാപിയും യുക്തിമാനും തന്നെ. ‎
\end{malayalam}}
\flushright{\begin{Arabic}
\quranayah[8][64]
\end{Arabic}}
\flushleft{\begin{malayalam}
നബിയേ, നിനക്കും നിന്നെ പിന്‍പറ്റിയ സത്യവിശ്വാസികള്‍ക്കും അല്ലാഹു തന്നെ മതി.
\end{malayalam}}
\flushright{\begin{Arabic}
\quranayah[8][65]
\end{Arabic}}
\flushleft{\begin{malayalam}
നബിയേ, നീ സത്യവിശ്വാസികളെ യുദ്ധത്തിന് പ്രേരിപ്പിക്കുക. നിങ്ങളില്‍ ക്ഷമാശീലരായ ‎ഇരുപതുപേരുണ്ടെങ്കില്‍ ഇരുനൂറുപേരെ ജയിക്കാം. നിങ്ങളില്‍ അത്തരം നൂറുപേരുണ്ടെങ്കില്‍ ‎സത്യനിഷേധികളിലെ ആയിരംപേരെ ജയിക്കാം. സത്യനിഷേധികള്‍ കാര്യബോധമില്ലാത്ത ‎ജനമായതിനാലാണിത്. ‎
\end{malayalam}}
\flushright{\begin{Arabic}
\quranayah[8][66]
\end{Arabic}}
\flushleft{\begin{malayalam}
എന്നാല്‍ ഇപ്പോള്‍ അല്ലാഹു നിങ്ങളുടെ ഭാരം ലഘൂകരിച്ചിരിക്കുന്നു. നിങ്ങള്ക്ക്ി ‎ദൌര്ബലല്യമുണ്ടെന്ന് അവന് നന്നായറിയാം. അതിനാല്‍ നിങ്ങളില്‍ ക്ഷമാലുക്കളായ ‎നൂറുപേരുണ്ടെങ്കില്‍ ഇരുനൂറുപേരെ ജയിക്കാം. നിങ്ങള്‍ ആയിരം പേരുണ്ടെങ്കില്‍ ‎ദൈവഹിതപ്രകാരം രണ്ടായിരം പേരെ ജയിക്കാം. അല്ലാഹു ക്ഷമാലുക്കളോടൊപ്പമാണ്. ‎
\end{malayalam}}
\flushright{\begin{Arabic}
\quranayah[8][67]
\end{Arabic}}
\flushleft{\begin{malayalam}
നാട്ടില്‍ എതിരാളികളെ കീഴ്പ്പെടുത്തി ശക്തി സ്ഥാപിക്കുംവരെ ഒരു പ്രവാചകന്നും തന്റെ കീഴില്‍ ‎യുദ്ധത്തടവുകാരുണ്ടാകാവതല്ല. നിങ്ങള്‍ ഐഹികനേട്ടം കൊതിക്കുന്നു. അല്ലാഹുവോ ‎പരലോകത്തെ ലക്ഷ്യമാക്കുന്നു. അല്ലാഹു പ്രതാപിയും യുക്തിമാനുംതന്നെ. ‎
\end{malayalam}}
\flushright{\begin{Arabic}
\quranayah[8][68]
\end{Arabic}}
\flushleft{\begin{malayalam}
അല്ലാഹുവില്നിഷന്നുള്ള വിധി നേരത്തെ രേഖപ്പെടുത്തിയിട്ടില്ലായിരുന്നെങ്കില്‍ നിങ്ങള്‍ ‎കൈപ്പറ്റിയതിന്റെ പേരില്‍ നിങ്ങളെ കടുത്ത ശിക്ഷ ബാധിക്കുമായിരുന്നു. ‎
\end{malayalam}}
\flushright{\begin{Arabic}
\quranayah[8][69]
\end{Arabic}}
\flushleft{\begin{malayalam}
എന്നാലും നിങ്ങള്‍ നേടിയ യുദ്ധമുതല്‍ അനുവദനീയവും നല്ലതുമെന്ന നിലയില്‍ ‎അനുഭവിച്ചുകൊള്ളുക. അല്ലാഹുവോട് ഭക്തി പുലര്ത്തുഷക. അല്ലാഹു ഏറെ പൊറുക്കുന്നവനും ‎ദയാപരനുമാണ്. ‎
\end{malayalam}}
\flushright{\begin{Arabic}
\quranayah[8][70]
\end{Arabic}}
\flushleft{\begin{malayalam}
നബിയേ, നിങ്ങളുടെ കൈവശമുള്ള യുദ്ധത്തടവുകാരോടു പറയുക: നിങ്ങളുടെ മനസ്സില്‍ വല്ല ‎നന്മയുമുള്ളതായി അല്ലാഹു അറിഞ്ഞാല്‍ നിങ്ങളില്നിുന്ന് വസൂല്‍ ചെയ്തതിനേക്കാള്‍ ഉത്തമമായത് ‎അവന്‍ നിങ്ങള്ക്ക്ല നല്കുംഞ. നിങ്ങള്ക്കംവന്‍ പൊറുത്തുതരികയും ചെയ്യും. അല്ലാഹു ഏറെ ‎പൊറുക്കുന്നവനും ദയാപരനുമാണ്. ‎
\end{malayalam}}
\flushright{\begin{Arabic}
\quranayah[8][71]
\end{Arabic}}
\flushleft{\begin{malayalam}
അഥവാ, നിന്നെ ചതിക്കാനാണ് അവരാഗ്രഹിക്കുന്നതെങ്കില്‍ അതിലൊട്ടും പുതുമയില്ല. അവര്‍ ‎നേരത്തെ തന്നെ അല്ലാഹുവോട് വഞ്ചന കാണിച്ചവരാണല്ലോ. അതിനാലാണ് അവന്‍ അവരെ ‎നിങ്ങള്ക്ക്ന അധീനപ്പെടുത്തിത്തന്നത്. അല്ലാഹു എല്ലാം അറിയുന്നവനും യുക്തിമാനുംതന്നെ. ‎
\end{malayalam}}
\flushright{\begin{Arabic}
\quranayah[8][72]
\end{Arabic}}
\flushleft{\begin{malayalam}
സത്യവിശ്വാസം സ്വീകരിക്കുകയും അതിന്റെ പേരില്‍ നാടുവിടേണ്ടിവരികയും തങ്ങളുടെ ‎ദേഹംകൊണ്ടും ധനംകൊണ്ടും അല്ലാഹുവിന്റെ മാര്ഗകത്തില്‍ സമരം നടത്തുകയും ചെയ്തവരും ‎അവര്ക്ക് അഭയം നല്കു്കയും അവരെ സഹായിക്കുകയും ചെയ്തവരും പരസ്പരം ‎ആത്മമിത്രങ്ങളാണ്. എന്നാല്‍ സത്യവിശ്വാസം സ്വീകരിക്കുകയും സ്വദേശം വെടിയാതിരിക്കുകയും ‎ചെയ്തവരുടെ സംരക്ഷണ ബാധ്യത നിങ്ങള്ക്കി ല്ല; അവര്‍ സ്വദേശം വെടിഞ്ഞ് വരും വരെ. ‎അഥവാ, മതകാര്യത്തില്‍ അവര്‍ സഹായം തേടിയാല്‍ അവരെ സഹായിക്കാന്‍ നിങ്ങള്‍ ‎ബാധ്യസ്ഥരാണ്. എന്നാല്‍ അത് നിങ്ങളുമായി കരാറിലേര്പ്പെ ട്ട ഏതെങ്കിലും ‎ജനതക്കെതിരെയാവരുത്. നിങ്ങള്‍ ചെയ്യുന്നതെല്ലാം കണ്ടറിയുന്നവനാണ് അല്ലാഹു. ‎
\end{malayalam}}
\flushright{\begin{Arabic}
\quranayah[8][73]
\end{Arabic}}
\flushleft{\begin{malayalam}
സത്യനിഷേധികളും പരസ്പരം ആത്മമിത്രങ്ങളാണ്. അതിനാല്‍ നിങ്ങളങ്ങനെ ചെയ്യുന്നില്ലെങ്കില്‍ ‎നാട്ടില്‍ കുഴപ്പവും വമ്പിച്ച നാശവുമുണ്ടാകും. ‎
\end{malayalam}}
\flushright{\begin{Arabic}
\quranayah[8][74]
\end{Arabic}}
\flushleft{\begin{malayalam}
വിശ്വസിക്കുകയും അതിന്റെ പേരില്‍ സ്വദേശം വെടിയുകയും ദൈവമാര്ഗനത്തില്‍ സമരം ‎നടത്തുകയും ചെയ്തവരാണ് യഥാര്ഥര സത്യവിശ്വാസികള്‍; അവര്ക്ക് അഭയമേകുകയും അവരെ ‎സഹായിക്കുകയും ചെയ്തവരും. അവര്ക്ക് പാപമോചനവും മാന്യമായ ജീവിതവിഭവങ്ങളുമുണ്ട്. ‎
\end{malayalam}}
\flushright{\begin{Arabic}
\quranayah[8][75]
\end{Arabic}}
\flushleft{\begin{malayalam}
പിന്നീട് സത്യവിശ്വാസം സ്വീകരിക്കുകയും സ്വദേശം വെടിഞ്ഞ് വരികയും നിങ്ങളോടൊത്ത് ‎ദൈവമാര്ഗ ത്തില്‍ സമരം നടത്തുകയും ചെയ്തവരും നിങ്ങളോടൊപ്പം തന്നെ. എങ്കിലും ദൈവിക ‎നിയമമനുസരിച്ച് രക്തബന്ധമുളളവര്‍ അന്യോന്യം കൂടുതല്‍ അടുത്തവരാണ്. അല്ലാഹു എല്ലാ ‎കാര്യങ്ങളെക്കുറിച്ചും നന്നായറിയുന്നവനാണ്. ‎
\end{malayalam}}
\chapter{\textmalayalam{തൌബ ( പശ്ചാത്താപം )}}
\begin{Arabic}
\Huge{\centerline{\basmalah}}\end{Arabic}
\flushright{\begin{Arabic}
\quranayah[9][1]
\end{Arabic}}
\flushleft{\begin{malayalam}
നിങ്ങളുമായി കരാറിലേര്പ്പെട്ടിരുന്ന ബഹുദൈവ വിശ്വാസികളോട് ‎അല്ലാഹുവിനും അവന്റെ ദൂതന്നും ഇനിമേല്‍ ബാധ്യതയൊന്നുമില്ലെന്ന ‎അറിയിപ്പാണിത്: ‎
\end{malayalam}}
\flushright{\begin{Arabic}
\quranayah[9][2]
\end{Arabic}}
\flushleft{\begin{malayalam}
‎"നാലു മാസം നിങ്ങള്‍ നാട്ടില്‍ സ്വൈരമായി സഞ്ചരിച്ചുകൊള്ളുക.” ‎അറിയുക: നിങ്ങള്ക്ക് അല്ലാഹുവെ തോല്പിരക്കാനാവില്ല. സത്യനിഷേധികളെ ‎അല്ലാഹു മാനം കെടുത്തുകതന്നെ ചെയ്യും. ‎
\end{malayalam}}
\flushright{\begin{Arabic}
\quranayah[9][3]
\end{Arabic}}
\flushleft{\begin{malayalam}
മഹത്തായ ഹജ്ജ് നാളില്‍ മുഴുവന്‍ മനുഷ്യര്ക്കു മായി അല്ലാഹുവും ‎അവന്റെ ദൂതനും നല്കു്ന്ന അറിയിപ്പാണിത്. ഇനിമുതല്‍ അല്ലാഹുവിനും ‎അവന്റെ ദുതന്നും ബഹുദൈവ വിശ്വാസികളോട് ഒരുവിധ ബാധ്യതയുമില്ല. ‎അതിനാല്‍ നിങ്ങള്‍ പശ്ചാത്തപിക്കുന്നുവെങ്കില്‍ അതാണ് നിങ്ങള്ക്ക് ഉത്തമം. ‎അഥവാ, നിങ്ങള്‍ പിന്തിരിയുകയാണെങ്കില്‍ അറിയുക: അല്ലാഹുവെ ‎തോല്പി്ക്കാന്‍ നിങ്ങള്ക്കാിവില്ല. സത്യനിഷേധികള്ക്ക് നോവേറിയ ‎ശിക്ഷയുണ്ടെന്ന് അവരെ നീ “സുവാര്ത്തു” അറിയിക്കുക. ‎
\end{malayalam}}
\flushright{\begin{Arabic}
\quranayah[9][4]
\end{Arabic}}
\flushleft{\begin{malayalam}
എന്നാല്‍ ബഹുദൈവ വിശ്വാസികളില്നിന്ന് നിങ്ങളുമായി ‎കരാറിലേര്പ്പെടുകയും പിന്നെ അത് പാലിക്കുന്നതില്‍ വീഴ്ച ‎വരുത്താതിരിക്കുകയും നിങ്ങള്ക്കെനതിരെ ആരെയും ‎സഹായിക്കാതിരിക്കുകയും ചെയ്തവര്ക്ക് ഇതു ബാധകമല്ല. അവരോടുള്ള ‎കരാര്‍ അവയുടെ കാലാവധിവരെ നിങ്ങള്‍ പാലിക്കുക. തീര്ച്ച്യായും ‎അല്ലാഹു സൂക്ഷ്മത പുലര്ത്തുരന്നവരെയാണ് ഇഷ്ടപ്പെടുന്നത്. ‎
\end{malayalam}}
\flushright{\begin{Arabic}
\quranayah[9][5]
\end{Arabic}}
\flushleft{\begin{malayalam}
അങ്ങനെ യുദ്ധം നിഷിദ്ധമായ മാസങ്ങള്‍ പിന്നിട്ടാല്‍ ആ ബഹുദൈവ ‎വിശ്വാസികളെ നിങ്ങള്‍ എവിടെ കണ്ടാലും കൊന്നുകളയുക. അവരെ ‎പിടികൂടുകയും ഉപരോധിക്കുകയും ചെയ്യുക. എല്ലാ മര്മളസ്ഥാനങ്ങളിലും ‎അവര്ക്കാ യി പതിയിരിക്കുക. അഥവാ, അവര്‍ പശ്ചാത്തപിക്കുകയും ‎നമസ്കാരം നിഷ്ഠയോടെ നിര്വുഹിക്കുകയും സകാത്ത് ‎നല്കുാകയുമാണെങ്കില്‍ അവരെ അവരുടെ പാട്ടിനുവിട്ടേക്കുക. സംശയം ‎വേണ്ട; അല്ലാഹു ഏറെ പൊറുക്കുന്നവനും പരമ ദയാലുവുമാണ്. ‎
\end{malayalam}}
\flushright{\begin{Arabic}
\quranayah[9][6]
\end{Arabic}}
\flushleft{\begin{malayalam}
ബഹുദൈവ വിശ്വാസികളിലാരെങ്കിലും നിന്റെയടുത്ത് അഭയം തേടിവന്നാല്‍ ‎അവന്ന് നീ അഭയം നല്കുക. അവന്‍ ദൈവവചനം കേട്ടറിയട്ടെ. പിന്നെ ‎അവനെ സുരക്ഷിതസ്ഥാനത്ത് എത്തിച്ചുകൊടുക്കുക. അവര്‍ അറിവില്ലാത്ത ‎ജനമായതിനാലാണ് ഇങ്ങനെയൊക്കെ ചെയ്യുന്നത്. ‎
\end{malayalam}}
\flushright{\begin{Arabic}
\quranayah[9][7]
\end{Arabic}}
\flushleft{\begin{malayalam}
ആ ബഹുദൈവ വിശ്വാസികള്ക്ക് അല്ലാഹുവിന്റെയും അവന്റെ ‎ദൂതന്റെയും അടുക്കല്‍ കരാര്‍ നിലനില്ക്കു ന്നതെങ്ങനെ? മസ്ജിദുല്‍ ‎ഹറാമിനടുത്തുവെച്ച് നിങ്ങളുമായി കരാര്‍ ചെയ്തവര്ക്കൊ ഴികെ. അവര്‍ ‎നിങ്ങളോട് നന്നായി വര്ത്തി ക്കുകയാണെങ്കില്‍ നിങ്ങള്‍ അവരോടും ‎നല്ലനിലയില്‍ വര്ത്തിിക്കുക. തീര്ച്ച്യായും അല്ലാഹു സൂക്ഷ്മത ‎പുലര്ത്തു ന്നവരെയാണ് ഇഷ്ടപ്പെടുന്നത്. ‎
\end{malayalam}}
\flushright{\begin{Arabic}
\quranayah[9][8]
\end{Arabic}}
\flushleft{\begin{malayalam}
അതെങ്ങനെ നിലനില്ക്കാനാണ്? അവര്ക്ക് നിങ്ങളെ കീഴ്പെടുത്താന്‍ ‎സാധിച്ചാല്‍ നിങ്ങളുമായുള്ള കുടുംബബന്ധമോ സന്ധിവ്യവസ്ഥകളോ ഒന്നും ‎അവര്‍ പരിഗണിക്കുകയില്ല. വാക്കുകള്‍ കൊണ്ട് അവര്‍ നിങ്ങളെ ‎തൃപ്തിപ്പെടുത്തും. എന്നാല്‍ അവരുടെ മനസ്സുകളത് നിരാകരിക്കും. ‎അവരിലേറെപ്പേരും അധാര്മി്കരാണ്. ‎
\end{malayalam}}
\flushright{\begin{Arabic}
\quranayah[9][9]
\end{Arabic}}
\flushleft{\begin{malayalam}
അവര്‍ തുച്ഛവിലയ്ക്ക് അല്ലാഹുവിന്റെ വചനങ്ങളെ വിറ്റു. ‎അല്ലാഹുവിന്റെ മാര്ഗ്ത്തില്നിുന്ന് ജനത്തെ തടയുകയും ചെയ്തു. അവര്‍ ‎ചെയ്തുകൊണ്ടിരിക്കുന്നത് വളരെ ചീത്ത തന്നെ. ‎
\end{malayalam}}
\flushright{\begin{Arabic}
\quranayah[9][10]
\end{Arabic}}
\flushleft{\begin{malayalam}
സത്യവിശ്വാസിയുടെ കാര്യത്തില്‍ രക്തബന്ധമോ സന്ധിവ്യവസ്ഥയോ അവര്‍ ‎പരിഗണിക്കാറില്ല. അവര്‍ തന്നെയാണ് അതിക്രമികള്‍. ‎
\end{malayalam}}
\flushright{\begin{Arabic}
\quranayah[9][11]
\end{Arabic}}
\flushleft{\begin{malayalam}
എന്നാല്‍ അവര്‍ പശ്ചാത്തപിക്കുകയും നമസ്കാരം നിഷ്ഠയോടെ ‎നിര്വലഹിക്കുകയും സകാത്ത് നല്കുശകയുമാണെങ്കില്‍ അവര്‍ നിങ്ങളുടെ ‎ആദര്ശലസഹോദരങ്ങളാണ്. കാര്യം മനസ്സിലാക്കുന്ന ജനത്തിനായി നാം ‎നമ്മുടെ പ്രമാണങ്ങള്‍ വിശദീകരിക്കുകയാണ്. ‎
\end{malayalam}}
\flushright{\begin{Arabic}
\quranayah[9][12]
\end{Arabic}}
\flushleft{\begin{malayalam}
അഥവാ, അവര്‍ കരാര്‍ ചെയ്തശേഷം തങ്ങളുടെ ശപഥങ്ങള്‍ ലംഘിക്കുകയും ‎നിങ്ങളുടെ മതത്തെ അവഹേളിക്കുകയുമാണെങ്കില്‍ സത്യനിഷേധത്തിന്റെ ‎തലതൊട്ടപ്പന്മാരോട് നിങ്ങള്‍ യുദ്ധം ചെയ്യുക. കാരണം അവരുടെ ‎പ്രതിജ്ഞകള്ക്ക്ോ ഒരര്ഥമവുമില്ല; തീര്ച്ചത. ഒരുവേള അവര്‍ വിരമിച്ചെങ്കിലോ. ‎
\end{malayalam}}
\flushright{\begin{Arabic}
\quranayah[9][13]
\end{Arabic}}
\flushleft{\begin{malayalam}
തങ്ങളുടെ കരാറുകള്‍ ലംഘിക്കുകയും ദൈവദൂതനെ നാടുകടത്താന്‍ ‎മുതിരുകയും ചെയ്ത ജനത്തോട് നിങ്ങള്‍ യുദ്ധം ചെയ്യുന്നില്ലെന്നോ? ‎അവരാണല്ലോ ആദ്യം യുദ്ധം ആരംഭിച്ചത്. എന്നിട്ടും നിങ്ങളവരെ ‎പേടിക്കുകയോ? എന്നാല്‍ ഭയപ്പെടാന്‍ കൂടുതല്‍ അര്ഹകന്‍ അല്ലാഹുവാണ്. ‎നിങ്ങള്‍ സത്യവിശ്വാസികളെങ്കില്‍! ‎
\end{malayalam}}
\flushright{\begin{Arabic}
\quranayah[9][14]
\end{Arabic}}
\flushleft{\begin{malayalam}
നിങ്ങള്‍ അവരോട് യുദ്ധം ചെയ്യുക. നിങ്ങളുടെ കൈകള്കൊുണ്ട് അല്ലാഹു ‎അവരെ ശിക്ഷിക്കും. അവരെ അവന്‍ നാണം കെടുത്തും. അവര്ക്കെ തിരെ ‎നിങ്ങളെ സഹായിക്കും. അങ്ങനെ സത്യവിശ്വാസികളുടെ മനസ്സുകള്ക്ക്ു ‎അവന്‍ സ്വസ്ഥത നല്കുംര. ‎
\end{malayalam}}
\flushright{\begin{Arabic}
\quranayah[9][15]
\end{Arabic}}
\flushleft{\begin{malayalam}
അവരുടെ മനസ്സുകളിലെ വെറുപ്പ് അവന്‍ തുടച്ചുനീക്കും. അല്ലാഹു ‎അവനിച്ഛിക്കുന്നവരുടെ പശ്ചാത്താപം സ്വീകരിക്കുന്നു. അല്ലാഹു എല്ലാം ‎അറിയുന്നവനും യുക്തിജ്ഞനുമാണ്. ‎
\end{malayalam}}
\flushright{\begin{Arabic}
\quranayah[9][16]
\end{Arabic}}
\flushleft{\begin{malayalam}
നിങ്ങളില്‍ അല്ലാഹുവിന്റെ മാര്ഗിത്തില്‍ സമരം നടത്തുകയും, ‎അല്ലാഹുവിനെയും അവന്റെ ദൂതനെയും സത്യവിശ്വാസികളെയുമല്ലാതെ ‎ആരെയും രഹസ്യകൂട്ടാളികളായി സ്വീകരിക്കാതിരിക്കുകയും ചെയ്യുന്നവര്‍ ‎ആരെന്ന് അല്ലാഹു വേര്തികരിച്ചെടുത്തിട്ടല്ലാതെ നിങ്ങളെ വെറുതെ ‎വിട്ടേക്കുമെന്ന് നിങ്ങള്‍ കരുതുന്നുണ്ടോ? നിങ്ങള്‍ ചെയ്യുന്നതൊക്കെയും ‎സൂക്ഷ്മമായി അറിയുന്നവനാണ് അല്ലാഹു. ‎
\end{malayalam}}
\flushright{\begin{Arabic}
\quranayah[9][17]
\end{Arabic}}
\flushleft{\begin{malayalam}
ബഹുദൈവ വിശ്വാസികള്‍ സത്യനിഷേധത്തിന് സ്വയം സാക്ഷ്യം ‎വഹിക്കുന്നവരായിരിക്കെ അവര്ക്ക്് അല്ലാഹുവിന്റെ പള്ളികള്‍ ‎പരിപാലിക്കാന്‍ ഒരവകാശവുമില്ല. അവരുടെ പ്രവര്ത്ത നങ്ങളൊക്കെയും ‎പാഴായിരിക്കുന്നു. നരകത്തീയിലവര്‍ നിത്യവാസികളായിരിക്കും. ‎
\end{malayalam}}
\flushright{\begin{Arabic}
\quranayah[9][18]
\end{Arabic}}
\flushleft{\begin{malayalam}
അല്ലാഹുവിന്റെ പള്ളികള്‍ പരിപാലിക്കേണ്ടത് അല്ലാഹുവിലും ‎അന്ത്യദിനത്തിലും വിശ്വസിക്കുകയും നമസ്കാരം നിഷ്ഠയോടെ ‎നിര്വയഹിക്കുകയും സകാത്ത് നല്കുകകയും അല്ലാഹുവെയല്ലാതെ ഒന്നിനെയും ‎ഭയപ്പെടാതിരിക്കുകയും ചെയ്യുന്നവര്‍ മാത്രമാണ്. അവര്‍ നേര്വകഴി ‎പ്രാപിച്ചവരായേക്കാം. ‎
\end{malayalam}}
\flushright{\begin{Arabic}
\quranayah[9][19]
\end{Arabic}}
\flushleft{\begin{malayalam}
തീര്ഥാചടകന് വെള്ളം കുടിക്കാന്‍ കൊടുക്കുന്നതിനെയും മസ്ജിദുല്‍ ഹറാം ‎പരിപാലിക്കുന്നതിനെയും അല്ലാഹുവിലും അന്ത്യദിനത്തിലും ‎വിശ്വസിക്കുകയും ദൈവമാര്ഗ്ത്തില്‍ സമരം നടത്തുകയും ചെയ്യുന്നവരുടെ ‎പ്രവര്ത്തകനങ്ങളെപ്പോലെയാക്കുകയാണോ നിങ്ങള്‍? അല്ലാഹുവിന്റെ ‎അടുക്കല്‍ അവ രണ്ടും ഒരേപോലെയല്ല. അല്ലാഹു അക്രമികളായ ജനത്തെ ‎നേര്വകഴിയിലാക്കുകയില്ല. ‎
\end{malayalam}}
\flushright{\begin{Arabic}
\quranayah[9][20]
\end{Arabic}}
\flushleft{\begin{malayalam}
സത്യവിശ്വാസം സ്വീകരിക്കുകയും സ്വന്തം നാട് വെടിയുകയും ‎അല്ലാഹുവിന്റെ മാര്ഗിത്തില്‍ ദേഹംകൊണ്ടും ധനംകൊണ്ടും സമരം ‎നടത്തുകയും ചെയ്യുന്നവര്‍ അല്ലാഹുവിങ്കല്‍ ഉന്നതസ്ഥാനീയരാണ്. വിജയം ‎വരിക്കുന്നവരും അവര്‍ തന്നെ. ‎
\end{malayalam}}
\flushright{\begin{Arabic}
\quranayah[9][21]
\end{Arabic}}
\flushleft{\begin{malayalam}
അവരുടെ നാഥന്‍ അവരെ തന്നില്‍ നിന്നുള്ള കാരുണ്യത്തെയും ‎തൃപ്തിയെയും സ്വര്ഗീിയാരാമങ്ങളെയും സംബന്ധിച്ച ശുഭവാര്ത്തന ‎അറിയിക്കുന്നു. അവര്ക്കങവിടെ അനശ്വരമായ സുഖാനുഭൂതികളുണ്ട്. ‎
\end{malayalam}}
\flushright{\begin{Arabic}
\quranayah[9][22]
\end{Arabic}}
\flushleft{\begin{malayalam}
അവരതില്‍ നിത്യവാസികളായിരിക്കും. തീര്ച്നുയായും അല്ലാഹുവിന്റെ ‎പക്കല്‍ മഹത്തായ പ്രതിഫലമാണുള്ളത്. ‎
\end{malayalam}}
\flushright{\begin{Arabic}
\quranayah[9][23]
\end{Arabic}}
\flushleft{\begin{malayalam}
വിശ്വസിച്ചവരേ, നിങ്ങള്‍ സ്വന്തം പിതാക്കളെയും സഹോദരങ്ങളെയും ‎നിങ്ങളുടെ രക്ഷാധികാരികളാക്കരുത്; അവര്‍ സത്യവിശ്വാസത്തെക്കാള്‍ ‎സത്യനിഷേധത്തെ സ്നേഹിക്കുന്നവരെങ്കില്‍! നിങ്ങളിലാരെങ്കിലും അവരെ ‎രക്ഷാധികാരികളാക്കുകയാണെങ്കില്‍ അവര്‍ തന്നെയാണ് അക്രമികള്‍. ‎
\end{malayalam}}
\flushright{\begin{Arabic}
\quranayah[9][24]
\end{Arabic}}
\flushleft{\begin{malayalam}
പറയുക: നിങ്ങളുടെ പിതാക്കളും പുത്രന്മാരും സഹോദരങ്ങളും ഇണകളും ‎കുടുംബക്കാരും, നിങ്ങള്‍ സമ്പാദിച്ചുണ്ടാക്കിയ സ്വത്തുക്കളും, നഷ്ടം ‎നേരിടുമോ എന്ന് നിങ്ങള്‍ ഭയപ്പെടുന്ന കച്ചവടവും, നിങ്ങള്ക്കേകറെ പ്രിയപ്പെട്ട ‎പാര്‍പ്പിടങ്ങളുമാണ് നിങ്ങള്ക്ക്പ അല്ലാഹുവെക്കാളും അവന്റെ ‎ദൂതനെക്കാളും അവന്റെ മാര്ഗ്ത്തിലെ അധ്വാനപരിശ്രമത്തെക്കാളും ‎പ്രിയപ്പെട്ടവയെങ്കില്‍ അല്ലാഹു തന്റെ കല്പ്ന നടപ്പില്‍ വരുത്തുന്നത് ‎കാത്തിരുന്നുകൊള്ളുക. കുറ്റവാളികളായ ജനത്തെ അല്ലാഹു ‎നേര്വിഴിയിലാക്കുകയില്ല. ‎
\end{malayalam}}
\flushright{\begin{Arabic}
\quranayah[9][25]
\end{Arabic}}
\flushleft{\begin{malayalam}
അല്ലാഹു നിങ്ങളെ നിരവധി സന്ദര്ഭ്ങ്ങളില്‍ സഹായിച്ചിട്ടുണ്ട്. ഹുനയ്ന്‍ ‎യുദ്ധദിനത്തിലും. അന്ന് നിങ്ങളുടെ എണ്ണപ്പെരുപ്പം നിങ്ങളെ ‎ദുരഭിമാനികളാക്കി. എന്നാല്‍ ആ സംഖ്യാധിക്യം നിങ്ങള്ക്കൊെട്ടും ‎നേട്ടമുണ്ടാക്കിയില്ല. ഭൂമി വളരെ വിശാലമായിരിക്കെ തന്നെ അത് പറ്റെ ‎ഇടുങ്ങിയതായി നിങ്ങള്ക്കുിതോന്നി. അങ്ങനെ നിങ്ങള്‍ പിന്തിരിഞ്ഞോടുകയും ‎ചെയ്തു. ‎
\end{malayalam}}
\flushright{\begin{Arabic}
\quranayah[9][26]
\end{Arabic}}
\flushleft{\begin{malayalam}
പിന്നീട് അല്ലാഹു തന്റെ ദൂതന്നും സത്യവിശ്വാസികള്ക്കും തന്നില്‍ നിന്നുള്ള ‎സമാധാനം സമ്മാനിച്ചു. നിങ്ങള്ക്ക്യ കാണാനാവാത്ത കുറേ പോരാളികളെ ‎ഇറക്കിത്തന്നു. സത്യനിഷേധികളെ അവന്‍ ശിക്ഷിക്കുകയും ചെയ്തു. ‎അതുതന്നെയാണ് സത്യനിഷേധികള്ക്കുഷള്ള പ്രതിഫലം. ‎
\end{malayalam}}
\flushright{\begin{Arabic}
\quranayah[9][27]
\end{Arabic}}
\flushleft{\begin{malayalam}
പിന്നെ അതിനുശേഷം അല്ലാഹു താനിച്ഛിക്കുന്നവരുടെ പശ്ചാത്താപം ‎സ്വീകരിക്കുന്നു. അല്ലാഹു ഏറെ പൊറുക്കുന്നവനും പരമ ദയാലുവുമാകുന്നു. ‎
\end{malayalam}}
\flushright{\begin{Arabic}
\quranayah[9][28]
\end{Arabic}}
\flushleft{\begin{malayalam}
വിശ്വസിച്ചവരേ, ബഹുദൈവ വിശ്വാസികള്‍ അവിശുദ്ധരാണ്. അതിനാല്‍ ‎ഇക്കൊല്ലത്തിനുശേഷം അവര്‍ മസ്ജിദുല്‍ ഹറാമിനെ സമീപിക്കരുത്. ‎ദാരിദ്യ്രം വന്നേക്കുമെന്ന് നിങ്ങള്‍ ഭയപ്പെടുന്നുവെങ്കില്‍ അറിയുക: അല്ലാഹു ‎ഇച്ഛിക്കുന്നുവെങ്കില്‍ തന്റെ അനുഗ്രഹത്താല്‍ നിങ്ങള്ക്ക്മ അവന്‍ സമൃദ്ധി ‎വരുത്തും. അല്ലാഹു എല്ലാം അറിയുന്നവനും യുക്തിമാനുമാണ്. ‎
\end{malayalam}}
\flushright{\begin{Arabic}
\quranayah[9][29]
\end{Arabic}}
\flushleft{\begin{malayalam}
വേദക്കാരില്‍ അല്ലാഹുവിലും അന്ത്യദിനത്തിലും വിശ്വസിക്കാത്തവരും ‎അല്ലാഹുവും അവന്റെ ദൂതനും വിലക്കിയത് നിഷിദ്ധമായി ‎ഗണിക്കാത്തവരും സത്യമതത്തെ ജീവിത വ്യവസ്ഥയായി ‎സ്വീകരിക്കാത്തവരുമായ ജനത്തോട് യുദ്ധം ചെയ്യുക. അവര്‍ വിധേയരായി ‎കയ്യോടെ ജിസ്യത നല്കുംടവരെ. ‎
\end{malayalam}}
\flushright{\begin{Arabic}
\quranayah[9][30]
\end{Arabic}}
\flushleft{\begin{malayalam}
യഹൂദര്‍ പറയുന്നു, ഉസൈര്‍ ദൈവപുത്രനാണെന്ന്. ക്രൈസ്തവര്‍ പറയുന്നു, ‎മിശിഹാ ദൈവപുത്രനാണെന്ന്. ഇതെല്ലാം അവരുടെ വാചകക്കസര്ത്തു കള്‍ ‎മാത്രമാണ്. നേരത്തെ സത്യത്തെ നിഷേധിച്ചവരെപ്പോലെത്തന്നെയാണ് ഇവരും ‎സംസാരിക്കുന്നത്. അല്ലാഹു അവരെ ശപിക്കട്ടെ. എങ്ങോട്ടാണ് അവര്‍ ‎വഴിവിട്ട് പോയിക്കൊണ്ടിരിക്കുന്നത്? ‎
\end{malayalam}}
\flushright{\begin{Arabic}
\quranayah[9][31]
\end{Arabic}}
\flushleft{\begin{malayalam}
അവര്‍ തങ്ങളുടെ പണ്ഡിതന്മാരെയും പുരോഹിതന്മാരെയും ‎അല്ലാഹുവിനു പുറമെ ദൈവങ്ങളാക്കി സ്വീകരിച്ചു. മര്യെമിന്റെ മകന്‍ ‎മസീഹിനെയും. എന്നാല്‍ ഇവരൊക്കെ ഒരേയൊരു ദൈവത്തിന് ‎വഴിപ്പെടാനല്ലാതെ കല്പിലക്കപ്പെട്ടിരുന്നില്ല. അവനല്ലാതെ ദൈവമില്ല. അവര്‍ ‎പങ്കുചേര്ക്കു ന്നവയില്‍ നിന്നൊക്കെ എത്രയോ വിശുദ്ധനാണ് അവന്‍. ‎
\end{malayalam}}
\flushright{\begin{Arabic}
\quranayah[9][32]
\end{Arabic}}
\flushleft{\begin{malayalam}
തങ്ങളുടെ വായകൊണ്ട് അല്ലാഹുവിന്റെ പ്രകാശത്തെ ഊതിക്കെടുത്താനാണ് ‎അവരുദ്ദേശിക്കുന്നത്. എന്നാല്‍ അല്ലാഹു തന്റെ പ്രകാശം ‎പൂര്ണ്തയിലെത്തിക്കാതിരിക്കില്ല. സത്യനിഷേധികള്ക്ക് അതെത്ര തന്നെ ‎അരോചകമാണെങ്കിലും! ‎
\end{malayalam}}
\flushright{\begin{Arabic}
\quranayah[9][33]
\end{Arabic}}
\flushleft{\begin{malayalam}
അവനാണ് തന്റെ ദൂതനെ സന്മാര്ഗിവും സത്യവ്യവസ്ഥയുമായി ‎നിയോഗിച്ചത്. അത് മറ്റെല്ലാ ജീവിത വ്യവസ്ഥകളെയും അതിജയിക്കാന്‍. ‎ബഹുദൈവ വിശ്വാസികള്ക്ക്മ അതെത്ര തന്നെ അനിഷ്ടകരമാണെങ്കിലും! ‎
\end{malayalam}}
\flushright{\begin{Arabic}
\quranayah[9][34]
\end{Arabic}}
\flushleft{\begin{malayalam}
വിശ്വസിച്ചവരേ, മതപണ്ഡിതന്മാരിലും പുരോഹിതന്മാരിലും ഏറെപ്പേരും ‎ജനങ്ങളുടെ ധനം അവിഹിതമായി അനുഭവിക്കുന്നവരാണ്. ജനങ്ങളെ ‎അല്ലാഹുവിന്റെ മാര്ഗതത്തില്‍ നിന്ന് തടയുന്നവരും. സ്വര്ണിവും വെള്ളിയും ‎ശേഖരിച്ചുവെക്കുകയും അവ അല്ലാഹുവിന്റെ മാര്ഗ്ത്തില്‍ ‎ചെലവഴിക്കാതിരിക്കുകയും ചെയ്യുന്നവരെ നോവേറിയ ശിക്ഷയെ ‎സംബന്ധിച്ച “സുവാര്ത്തു” അറിയിക്കുക. ‎
\end{malayalam}}
\flushright{\begin{Arabic}
\quranayah[9][35]
\end{Arabic}}
\flushleft{\begin{malayalam}
നരകത്തീയിലിട്ട് ചുട്ടുപഴുപ്പിച്ച് അവകൊണ്ട് അവരുടെ നെറ്റികളും ‎പാര്ശ്വയഭാഗങ്ങളും മുതുകുകളും ചൂടുവെക്കും ദിനം! അന്ന് അവരോടു ‎പറയും: "ഇതാണ് നിങ്ങള്‍ നിങ്ങള്ക്കാ യി സമ്പാദിച്ചുവെച്ചത്. അതിനാല്‍ ‎നിങ്ങള്‍ സമ്പാദിച്ചുവെച്ചതിന്റെ രുചി ആസ്വദിച്ചുകൊള്ളുക.” ‎
\end{malayalam}}
\flushright{\begin{Arabic}
\quranayah[9][36]
\end{Arabic}}
\flushleft{\begin{malayalam}
ആകാശഭൂമികളുടെ സൃഷ്ടി നടന്ന നാള്‍ തൊട്ട് അല്ലാഹുവിന്റെ അടുക്കല്‍ ‎ദൈവിക പ്രമാണമനുസരിച്ച് മാസങ്ങളുടെ എണ്ണം പന്ത്രണ്ടാണ്. അവയില്‍ ‎നാലെണ്ണം യുദ്ധം വിലക്കപ്പെട്ടവയാണ്. ഇതാണ് യഥാര്ഥ നിയമക്രമം. ‎അതിനാല്‍ ആ നാലുമാസം നിങ്ങള്‍ നിങ്ങളോടുതന്നെ അക്രമം ‎കാണിക്കാതിരിക്കുക. ബഹുദൈവ വിശ്വാസികള്‍ എവ്വിധം ഒറ്റക്കെട്ടായി ‎നിങ്ങളോട് യുദ്ധം ചെയ്യുന്നുവോ അവ്വിധം നിങ്ങളും ഒന്നായി അവരോട് ‎യുദ്ധം ചെയ്യുക. അറിയുക: അല്ലാഹു സൂക്ഷ്മതയുള്ളവരോടൊപ്പമാണ്. ‎
\end{malayalam}}
\flushright{\begin{Arabic}
\quranayah[9][37]
\end{Arabic}}
\flushleft{\begin{malayalam}
യുദ്ധം വിലക്കിയ മാസങ്ങളില്‍ മാറ്റം വരുത്തുന്നത് കടുത്ത ‎സത്യനിഷേധമാണ്. അതുവഴി ആ സത്യനിഷേധികള്‍ കൂടുതല്‍ വലിയ ‎വഴികേടിലകപ്പെടുന്നു. ചില കൊല്ലങ്ങളിലവര്‍ യുദ്ധം അനുവദനീയമാക്കുന്നു. ‎മറ്റുചില വര്ഷ്ങ്ങളിലത് നിഷിദ്ധമാക്കുകയും ചെയ്യുന്നു. അല്ലാഹു ‎നിഷിദ്ധമാക്കിയ മാസങ്ങളുടെ എണ്ണം ഒപ്പിക്കാനാണിത്. അങ്ങനെ അല്ലാഹു ‎വിലക്കിയതിനെ അവര്‍ അനുവദനീയമാക്കുന്നു. അവരുടെ ഈ ‎ദുഷ്ചെയ്തികള്‍ അവര്ക്ക് ആകര്ഷികമായി തോന്നുന്നു. സത്യനിഷേധികളായ ‎ജനത്തെ അല്ലാഹു നേര്വ ഴിയിലാക്കുകയില്ല. ‎
\end{malayalam}}
\flushright{\begin{Arabic}
\quranayah[9][38]
\end{Arabic}}
\flushleft{\begin{malayalam}
വിശ്വസിച്ചവരേ, നിങ്ങള്ക്കെകന്തുപറ്റി? അല്ലാഹുവിന്റെ മാര്ഗേത്തില്‍ ‎ഇറങ്ങിത്തിരിക്കുകയെന്നു പറയുമ്പോള്‍ നിങ്ങള്‍ ഭൂമിയോട് ‎അള്ളിപ്പിടിക്കുകയാണല്ലോ. പരലോകത്തെക്കാള്‍ ഐഹികജീവിതംകൊണ്ട് ‎നിങ്ങള്‍ തൃപ്തിപ്പെട്ടിരിക്കയാണോ? എന്നാല്‍ പരലോകത്തെ അപേക്ഷിച്ച് ‎ഐഹികജീവിത വിഭവം നന്നെ നിസ്സാരമാണ്. ‎
\end{malayalam}}
\flushright{\begin{Arabic}
\quranayah[9][39]
\end{Arabic}}
\flushleft{\begin{malayalam}
നിങ്ങള്‍ അല്ലാഹുവിന്റെ മാര്ഗയത്തില്‍ ഇറങ്ങിത്തിരിക്കുന്നില്ലെങ്കില്‍ ‎അല്ലാഹു നിങ്ങള്ക്കുന നോവേറിയ ശിക്ഷ നല്കുംത. നിങ്ങള്ക്കുുപകരം മറ്റൊരു ‎ജനതയെ കൊണ്ടുവരികയും ചെയ്യും. അല്ലാഹുവിന് ഒരു ദ്രോഹവും ‎വരുത്താന്‍ നിങ്ങള്ക്കാവവില്ല. അല്ലാഹു എല്ലാ കാര്യങ്ങള്ക്കും ‎കഴിവുറ്റവനാണ്. ‎
\end{malayalam}}
\flushright{\begin{Arabic}
\quranayah[9][40]
\end{Arabic}}
\flushleft{\begin{malayalam}
നിങ്ങള്‍ അദ്ദേഹത്തെ സഹായിക്കുന്നില്ലെങ്കില്‍ തീര്ച്ചുയായും അല്ലാഹു ‎അദ്ദേഹത്തെ സഹായിച്ചിട്ടുണ്ട്. സത്യനിഷേധികള്‍ അദ്ദേഹത്തെ ‎പുറത്താക്കിയ സന്ദര്ഭടത്തിലാണത്. അദ്ദേഹം രണ്ടാളുകളില്‍ ‎ഒരുവനാവുകയും ഇരുവരും ഗുഹയിലായിരിക്കുകയും ചെയ്തപ്പോള്‍. ‎അദ്ദേഹം തന്റെ കൂട്ടുകാരനോട് പറഞ്ഞു: "ദുഃഖിക്കാതിരിക്കുക; അല്ലാഹു ‎നമ്മോടൊപ്പമുണ്ട്.” അന്നേരം അല്ലാഹു തന്നില്‍ നിന്നുള്ള സമാധാനം ‎അദ്ദേഹത്തിന് സമ്മാനിച്ചു. നിങ്ങള്ക്കു കാണാനാവാത്ത പോരാളികളാല്‍ ‎അദ്ദേഹത്തിന് കരുത്തേകുകയും ചെയ്തു. ഒപ്പം സത്യനിഷേധികളുടെ ‎വചനത്തെ അവന്‍ പറ്റെ പതിതമാക്കി. അല്ലാഹുവിന്റെ വചനം തന്നെയാണ് ‎അത്യുന്നതം. അല്ലാഹു പ്രതാപിയും യുക്തിമാനും തന്നെ. ‎
\end{malayalam}}
\flushright{\begin{Arabic}
\quranayah[9][41]
\end{Arabic}}
\flushleft{\begin{malayalam}
നിങ്ങള്‍ സാധന സാമഗ്രികള്‍ കൂടിയവരായാലും കുറഞ്ഞവരായാലും ‎ഇറങ്ങിപ്പുറപ്പെടുക. നിങ്ങളുടെ ദേഹംകൊണ്ടും ധനംകൊണ്ടും ‎ദൈവമാര്ഗപത്തില്‍ സമരംചെയ്യുക. അതാണ് നിങ്ങള്ക്കു ത്തമം. നിങ്ങള്‍ ‎അറിയുന്നവരെങ്കില്‍! ‎
\end{malayalam}}
\flushright{\begin{Arabic}
\quranayah[9][42]
\end{Arabic}}
\flushleft{\begin{malayalam}
ലക്ഷ്യം തൊട്ടടുത്തതും യാത്ര പ്രയാസരഹിതവുമാണെങ്കില്‍ അവര്‍ നിന്നെ ‎അനുഗമിക്കുമായിരുന്നു. എന്നാല്‍ ലക്ഷ്യം വിദൂരവും വഴി ‎വിഷമകരവുമായി അവര്ക്ക് തോന്നി. അതിനാല്‍ അവര്‍ അല്ലാഹുവിന്റെ ‎പേരില്‍ സത്യം ചെയ്തു പറയും: "ഞങ്ങള്ക്ക്ന സാധിച്ചിരുന്നെങ്കില്‍ ഞങ്ങളും ‎നിങ്ങളോടൊപ്പം പുറപ്പെടുമായിരുന്നു.” സത്യത്തിലവര്‍ തങ്ങളെത്തന്നെയാണ് ‎നശിപ്പിക്കുന്നത്. അല്ലാഹുവിനറിയാം; അവര്‍ കള്ളം പറയുന്നവരാണെന്ന്. ‎
\end{malayalam}}
\flushright{\begin{Arabic}
\quranayah[9][43]
\end{Arabic}}
\flushleft{\begin{malayalam}
അല്ലാഹു നിനക്ക് മാപ്പേകിയിരിക്കുന്നു. അവരില്‍ സത്യം പറഞ്ഞവര്‍ ‎ആരെന്ന് നിനക്ക് വ്യക്തമാവുകയും കള്ളം പറഞ്ഞവരെ തിരിച്ചറിയുകയും ‎ചെയ്യുംവരെ നീ അവര്ക്ക്മ വിട്ടുനില്ക്കാുന്‍ അനുവാദം നല്കി്യതെന്തിന്? ‎
\end{malayalam}}
\flushright{\begin{Arabic}
\quranayah[9][44]
\end{Arabic}}
\flushleft{\begin{malayalam}
അല്ലാഹുവിലും അന്ത്യദിനത്തിലും വിശ്വസിക്കുന്നവരാരും തങ്ങളുടെ ‎ധനംകൊണ്ടും ദേഹംകൊണ്ടും ദൈവമാര്ഗ്ത്തില്‍ സമരം ചെയ്യുന്നതില്നിാന്ന് ‎വിട്ടുനില്ക്കാംന്‍ നിന്നോട് അനുവാദം ചോദിക്കുകയില്ല. സൂക്ഷ്മത ‎പാലിക്കുന്നവര്‍ ആരെന്ന് നന്നായറിയുന്നവനാണ് അല്ലാഹു. ‎
\end{malayalam}}
\flushright{\begin{Arabic}
\quranayah[9][45]
\end{Arabic}}
\flushleft{\begin{malayalam}
അല്ലാഹുവിലും അന്ത്യദിനത്തിലും വിശ്വസിക്കാത്തവരും മനസ്സില്‍ ‎സംശയമുള്ളവരും മാത്രമാണ് നിന്നോട് അനുവാദം ചോദിക്കുന്നത്. അവര്‍ ‎സംശയാലുക്കളായി അങ്ങുമിങ്ങും ആടിക്കളിക്കുന്നവരാണ്. ‎
\end{malayalam}}
\flushright{\begin{Arabic}
\quranayah[9][46]
\end{Arabic}}
\flushleft{\begin{malayalam}
അവര്‍ അല്ലാഹുവിന്റെ മാര്ഗിത്തില്‍ ഇറങ്ങി പുറപ്പെടാന്‍ ‎യഥാര്ഥുത്തില്താന്നെ ഉദ്ദേശിച്ചിരുന്നുവെങ്കില്‍ അതിനായി സജ്ജമാക്കേണ്ട ‎സാമഗ്രികളൊക്കെ ഒരുക്കിവെക്കുമായിരുന്നു. എന്നാല്‍ അവര്‍ ‎ഇറങ്ങിപ്പുറപ്പെടുന്നത് അല്ലാഹുവിന് അനിഷ്ടകരമായിരുന്നു. അതിനാല്‍ ‎അവനവരെ തടഞ്ഞുനിര്ത്തി . അവരോടിങ്ങനെ പറയുകയും ചെയ്തു: ‎‎"ഇവിടെ ചടഞ്ഞിരിക്കുന്നവരോടൊപ്പം നിങ്ങളും ഇരുന്നുകൊള്ളുക.” ‎
\end{malayalam}}
\flushright{\begin{Arabic}
\quranayah[9][47]
\end{Arabic}}
\flushleft{\begin{malayalam}
അവര്‍ നിങ്ങളുടെ കൂട്ടത്തില്‍ പുറപ്പെട്ടിരുന്നുവെങ്കില്‍ നിങ്ങള്ക്കനവര്‍ ‎കൂടുതല്‍ വിപത്തുകള്‍ വരുത്തിവെക്കുമായിരുന്നു. നിങ്ങള്ക്ക്വ നാശം ‎വരുത്താനായി അവര്‍ നിങ്ങള്ക്കിങടയില്‍ ഓടിനടക്കുമായിരുന്നു. നിങ്ങളുടെ ‎കൂട്ടത്തില്‍ അവര്ക്ക് ചെവികൊടുക്കുന്നവരുമുണ്ടല്ലോ. അക്രമികളെപ്പറ്റി ‎നന്നായറിയുന്നവനാണ് അല്ലാഹു. ‎
\end{malayalam}}
\flushright{\begin{Arabic}
\quranayah[9][48]
\end{Arabic}}
\flushleft{\begin{malayalam}
ഇതിനു മുമ്പും അവര്‍ കുഴപ്പമുണ്ടാക്കാന്‍ ശ്രമിച്ചിട്ടുണ്ട്. നിനക്കെതിരെ ‎തന്ത്രങ്ങള്‍ പ്രയോഗിക്കാന്‍ ശ്രമിച്ചിട്ടുമുണ്ട്. എന്നിട്ടും അവര്ക്ക്ത ‎അനിഷ്ടകരമായിരിക്കെത്തന്നെ സത്യം വന്നെത്തി. അല്ലാഹുവിന്റെ വിധി ‎പുലരുകയും ചെയ്തു. ‎
\end{malayalam}}
\flushright{\begin{Arabic}
\quranayah[9][49]
\end{Arabic}}
\flushleft{\begin{malayalam}
അവരില്‍ ഇങ്ങനെ പറയുന്നവരുണ്ട്: "എനിക്ക് ഇളവ് അനുവദിച്ചാലും. ‎എന്നെ കുഴപ്പത്തില്‍ പെടുത്താതിരുന്നാലും.” അറിയുക: കുഴപ്പത്തില്‍ ‎തന്നെയാണ് അവര്‍ വീണിരിക്കുന്നത്. തീര്ച്ച യായും നരകം സത്യനിഷേധികളെ ‎വലയം ചെയ്യും. ‎
\end{malayalam}}
\flushright{\begin{Arabic}
\quranayah[9][50]
\end{Arabic}}
\flushleft{\begin{malayalam}
നിനക്കു വല്ല നേട്ടവും കിട്ടിയാല്‍ അതവരെ ദുഃഖിതരാക്കും. നിനക്കു വല്ല ‎വിപത്തും വന്നാല്‍, അവര്‍ പറയും: "ഞങ്ങള്‍ നേരത്തെ തന്നെ ഞങ്ങളുടെ ‎കാര്യം കൈക്കലാക്കിയിരിക്കുന്നു.” അങ്ങനെ ആഹ്ളാദത്തോടെ അവര്‍ ‎പിന്തിരിഞ്ഞുപോവുകയും ചെയ്യും. ‎
\end{malayalam}}
\flushright{\begin{Arabic}
\quranayah[9][51]
\end{Arabic}}
\flushleft{\begin{malayalam}
പറയുക: അല്ലാഹു ഞങ്ങള്ക്ക്ര വിധിച്ചതല്ലാതൊന്നും ഞങ്ങളെ ‎ബാധിക്കുകയില്ല. അവനാണ് ഞങ്ങളുടെ രക്ഷകന്‍. സത്യവിശ്വാസികള്‍ ‎അല്ലാഹുവില്‍ ഭരമേല്പ്പി ച്ചുകൊള്ളട്ടെ. ‎
\end{malayalam}}
\flushright{\begin{Arabic}
\quranayah[9][52]
\end{Arabic}}
\flushleft{\begin{malayalam}
പറയുക: രണ്ടു നേട്ടങ്ങളില്‍ ഏതെങ്കിലുമൊന്നല്ലാതെ ഞങ്ങളുടെ കാര്യത്തില്‍ ‎മറ്റെന്തെങ്കിലും നിങ്ങള്‍ പ്രതീക്ഷിക്കുന്നുണ്ടോ? എന്നാല്‍ നിങ്ങളുടെ ‎കാര്യത്തില്‍ ഞങ്ങള്‍ പ്രതീക്ഷിക്കുന്നതിതാണ്: നേരിട്ടിടപെട്ടോ, ഞങ്ങളുടെ ‎കയ്യാലോ അല്ലാഹു നിങ്ങളെ ശിക്ഷിക്കും. അതിനാല്‍ നിങ്ങള്‍ ‎കാത്തിരുന്നുകൊള്ളുക. നിങ്ങളോടൊപ്പം ഞങ്ങളും കാത്തിരിക്കാം. ‎
\end{malayalam}}
\flushright{\begin{Arabic}
\quranayah[9][53]
\end{Arabic}}
\flushleft{\begin{malayalam}
പറയുക: നിങ്ങള്‍ സ്വമനസ്സാലോ പരപ്രേരണയാലോ ചെലവഴിച്ചുകൊള്ളുക. ‎എങ്ങനെയായാലും നിങ്ങളില്നി ന്നത് സ്വീകരിക്കുന്നതല്ല. കാരണം, നിങ്ങള്‍ ‎അധാര്മിാകരായ ജനതയാണെന്നതു തന്നെ. ‎
\end{malayalam}}
\flushright{\begin{Arabic}
\quranayah[9][54]
\end{Arabic}}
\flushleft{\begin{malayalam}
അവരുടെ പക്കല്നി ന്ന് അവരുടെ ദാനം സ്വീകരിക്കാതിരിക്കാന്‍ കാരണം ‎ഇതു മാത്രമാണ്: അവര്‍ അല്ലാഹുവിനെയും അവന്റെ ദൂതനെയും ‎തള്ളിപ്പറയുന്നു; മടിയന്മാരായല്ലാതെ അവര്‍ നമസ്കാരത്തിനെത്തുന്നില്ല. ‎വെറുപ്പോടെയല്ലാതെ ധനം ചെലവഴിക്കുന്നുമില്ല. ‎
\end{malayalam}}
\flushright{\begin{Arabic}
\quranayah[9][55]
\end{Arabic}}
\flushleft{\begin{malayalam}
അവരുടെ സമ്പത്തും സന്താനങ്ങളും നിന്നെ വിസ്മയിപ്പിക്കാതിരിക്കട്ടെ. ‎അവയിലൂടെ ഐഹികജീവിതത്തില്‍ തന്നെ അവരെ ശിക്ഷിക്കണമെന്നാണ് ‎അല്ലാഹു ഉദ്ദേശിക്കുന്നത്. സത്യനിഷേധികളായിരിക്കെ അവര്‍ ജീവന്‍ ‎വെടിയണമെന്നും. ‎
\end{malayalam}}
\flushright{\begin{Arabic}
\quranayah[9][56]
\end{Arabic}}
\flushleft{\begin{malayalam}
അവര്‍ അല്ലാഹുവിന്റെ പേരിലിങ്ങനെ സത്യം ചെയ്യുന്നു: "തീര്ച്ച്യായും ‎ഞങ്ങള്‍ നിങ്ങളില്പെേട്ടവര്‍ തന്നെയാണ്.” യഥാര്ഥരത്തില്‍ അവര്‍ ‎നിങ്ങളില്പെുട്ടവരല്ല. മറിച്ച്, നിങ്ങളെ പേടിച്ചുകഴിയുന്ന ജനമാണവര്‍. ‎
\end{malayalam}}
\flushright{\begin{Arabic}
\quranayah[9][57]
\end{Arabic}}
\flushleft{\begin{malayalam}
ഏതെങ്കിലും അഭയസ്ഥാനമോ ഗുഹകളോ ഒളിഞ്ഞിരിക്കാനുള്ള ഇടമോ ‎കണ്ടെത്തുകയാണെങ്കില്‍ അവര്‍ പിന്തിരിഞ്ഞ് അങ്ങോട്ട് ‎വിരണ്ടോടുമായിരുന്നു. ‎
\end{malayalam}}
\flushright{\begin{Arabic}
\quranayah[9][58]
\end{Arabic}}
\flushleft{\begin{malayalam}
ദാനധര്മതങ്ങളുടെ വിതരണ കാര്യത്തില്‍ നിന്നെ വിമര്ശിയക്കുന്നവര്‍ ‎അക്കൂട്ടത്തിലുണ്ട്. അതില്നിിന്ന് എന്തെങ്കിലും കിട്ടിയാല്‍ അവര്‍ ‎തൃപ്തരാകും. കിട്ടിയില്ലെങ്കിലോ കോപാകുലരാവും. ‎
\end{malayalam}}
\flushright{\begin{Arabic}
\quranayah[9][59]
\end{Arabic}}
\flushleft{\begin{malayalam}
അവര്‍ അല്ലാഹുവും അവന്റെ ദൂതനും നല്കിുയതില്‍ തൃപ്തിയടയുകയും ‎എന്നിട്ടിങ്ങനെ പറയുകയും ചെയ്തിരുന്നെങ്കില്‍ എത്ര നന്നായേനെ: ‎‎"ഞങ്ങള്ക്ക്ങ അല്ലാഹു മതി. അല്ലാഹുവിന്റെ അനുഗ്രഹത്തില്നിയന്ന് അവനും ‎അവന്റെ ദൂതനും ഞങ്ങള്ക്ക് ഇനിയും നല്കും്. ഞങ്ങള്‍ അല്ലാഹുവില്‍ ‎മാത്രം പ്രതീക്ഷയര്പ്പി്ച്ചവരാണ്.” ‎
\end{malayalam}}
\flushright{\begin{Arabic}
\quranayah[9][60]
\end{Arabic}}
\flushleft{\begin{malayalam}
സകാത്ത് ദരിദ്രര്ക്കും അഗതികള്‍ക്കും അതിന്റെ ജോലിക്കാര്ക്കുംന ‎മനസ്സിണങ്ങിയവര്ക്കും അടിമ മോചനത്തിനും കടംകൊണ്ട് വലഞ്ഞവര്ക്കും ‎ദൈവമാര്ഗ്ത്തില്‍ വിനിയോഗിക്കാനും വഴിപോക്കര്ക്കുംി മാത്രമുള്ളതാണ്. ‎അല്ലാഹുവിന്റെ നിര്ണതയമാണിത്. അല്ലാഹു എല്ലാം അറിയുന്നവനും ‎യുക്തിമാനുമാണ്. ‎
\end{malayalam}}
\flushright{\begin{Arabic}
\quranayah[9][61]
\end{Arabic}}
\flushleft{\begin{malayalam}
നബിയെ ദ്രോഹിക്കുന്ന ചിലരും അവരിലുണ്ട്. അദ്ദേഹം എല്ലാറ്റിനും ‎ചെവികൊടുക്കുന്നവനാണെന്ന് അവരാക്ഷേപിക്കുന്നു. പറയുക: അദ്ദേഹം ‎നിങ്ങള്ക്ക്ു ഗുണകരമായതിനെ ചെവിക്കൊള്ളുന്നവനാകുന്നു. അദ്ദേഹം ‎അല്ലാഹുവില്‍ വിശ്വസിക്കുന്നു. സത്യവിശ്വാസികളില്‍ വിശ്വാസമര്പ്പിതക്കുന്നു. ‎നിങ്ങളില്‍ സത്യവിശ്വാസം സ്വീകരിച്ചവര്ക്ക് അദ്ദേഹം മഹത്തായ ‎അനുഗ്രഹമാണ്. അല്ലാഹുവിന്റെ ദൂതനെ ദ്രോഹിക്കുന്നവര്ക്ക്ഹ നോവേറിയ ‎ശിക്ഷയുണ്ട്. ‎
\end{malayalam}}
\flushright{\begin{Arabic}
\quranayah[9][62]
\end{Arabic}}
\flushleft{\begin{malayalam}
നിങ്ങളെ പ്രീതിപ്പെടുത്താനായി നിങ്ങളോടവര്‍ അല്ലാഹുവിന്റെ പേരില്‍ ‎സത്യംചെയ്തു പറയുന്നു. എന്നാല്‍ അവര്‍ പ്രീതിപ്പെടുത്താന്‍ ഏറെ അര്ഹശര്‍ ‎അല്ലാഹുവും അവന്റെ ദൂതനുമാകുന്നു. അവര്‍ സത്യവിശ്വാസികളെങ്കില്‍! ‎
\end{malayalam}}
\flushright{\begin{Arabic}
\quranayah[9][63]
\end{Arabic}}
\flushleft{\begin{malayalam}
അവര്ക്കവറിയില്ലേ; ആരെങ്കിലും അല്ലാഹുവോടും അവന്റെ ദൂതനോടും ‎എതിരിടുന്നുവെങ്കില്‍ അവന്നുണ്ടാവുക നരകത്തീയാണെന്ന്. അവനവിടെ ‎നിത്യവാസിയായിരിക്കും. അത് അത്യന്തം അപമാനകരംതന്നെ. ‎
\end{malayalam}}
\flushright{\begin{Arabic}
\quranayah[9][64]
\end{Arabic}}
\flushleft{\begin{malayalam}
കപടവിശ്വാസികള്‍ ഭയപ്പെടുന്നു, തങ്ങളുടെ മനസ്സിലുള്ളത് അവരെ ‎അറിയിക്കുന്ന വല്ല അധ്യായവും അവരെപ്പറ്റി ‎അവതീര്ണകമായേക്കുമോയെന്ന്. പറയുക: നിങ്ങള്‍ പരിഹസിച്ചുകൊള്ളുക. ‎നിങ്ങള്‍ പേടിച്ചുകൊണ്ടിരിക്കുന്ന അക്കാര്യം അല്ലാഹു ‎പുറത്തുകൊണ്ടുവരിക തന്നെ ചെയ്യും. ‎
\end{malayalam}}
\flushright{\begin{Arabic}
\quranayah[9][65]
\end{Arabic}}
\flushleft{\begin{malayalam}
നീ അവരോട് അതേപ്പറ്റി ചോദിച്ചാല്‍ അവര്‍ പറയും: "ഞങ്ങള്‍ കളിയും ‎തമാശയും പറയുക മാത്രമായിരുന്നു.” ചോദിക്കുക: "അല്ലാഹുവെയും ‎അവന്റെ വചനങ്ങളെയും ദൂതനെയുമാണോ നിങ്ങള്‍ ‎പരിഹസിച്ചുകൊണ്ടിരുന്നത്?” ‎
\end{malayalam}}
\flushright{\begin{Arabic}
\quranayah[9][66]
\end{Arabic}}
\flushleft{\begin{malayalam}
ഇനി നിങ്ങള്‍ ഒഴികഴിവുകളൊന്നും പറയേണ്ട. തീര്ച്ചിയായും നിങ്ങള്‍ ‎സത്യവിശ്വാസം സ്വീകരിച്ചശേഷം അതിനെ തള്ളിപ്പറഞ്ഞിരിക്കുന്നു. ‎നിങ്ങളിലൊരു വിഭാഗത്തിന് നാം മാപ്പ് നല്കിതയാലും മറ്റൊരു വിഭാഗത്തെ ‎ശിക്ഷിക്കുക തന്നെ ചെയ്യും. കാരണം അവര്‍ കൊടുംകുറ്റവാളികളാണ്. ‎
\end{malayalam}}
\flushright{\begin{Arabic}
\quranayah[9][67]
\end{Arabic}}
\flushleft{\begin{malayalam}
കപടവിശ്വാസികളായ സ്ത്രീകളും പുരുഷന്മാരും ഒരേ തരക്കാര്‍ തന്നെ. ‎അവര്‍ തിന്മ കല്പിപക്കുന്നു. നന്മ വിലക്കുന്നു. അവര്‍ ധനം നല്ല ‎മാര്ഗശത്തില്‍ ചെലവഴിക്കാതെ തങ്ങളുടെ കൈകള്‍ മുറുക്കിപ്പിടിക്കുന്നു. ‎അവര്‍ അല്ലാഹുവെ മറന്നു. അതിനാല്‍ അവന്‍ അവരെയും മറന്നു. ‎സംശയമില്ല; കപടവിശ്വാസികള്‍ അധാര്മിലകര്‍ തന്നെ. ‎
\end{malayalam}}
\flushright{\begin{Arabic}
\quranayah[9][68]
\end{Arabic}}
\flushleft{\begin{malayalam}
കപടവിശ്വാസികളായ സ്ത്രീ പുരുഷന്മാര്ക്കുംക സത്യനിഷേധികള്ക്കും ‎അല്ലാഹു നരകത്തീ വാഗ്ദാനം ചെയ്തിരിക്കുന്നു. അവരതില്‍ ‎നിത്യവാസികളായിരിക്കും. അവര്ക്കനതുമതി. അല്ലാഹു അവരെ ‎ശപിച്ചിരിക്കുന്നു. അവര്ക്ക് നിത്യമായ ശിക്ഷയുണ്ട്. ‎
\end{malayalam}}
\flushright{\begin{Arabic}
\quranayah[9][69]
\end{Arabic}}
\flushleft{\begin{malayalam}
നിങ്ങളുടെ മുമ്പുണ്ടായിരുന്നവരെപ്പോലെത്തന്നെയാണ് നിങ്ങളും. എന്നാല്‍ ‎അവര്‍ നിങ്ങളേക്കാള്‍ കരുത്തന്മാരായിരുന്നു. കൂടുതല്‍ മുതലും ‎മക്കളുമുള്ളവരും. അങ്ങനെ തങ്ങളുടെ വിഹിതംകൊണ്ട് തന്നെ അവര്‍ ‎സുഖമാസ്വദിച്ചു. നിങ്ങളുടെ മുന്ഗാ്മികള്‍ തങ്ങളുടെ വിഹിതംകൊണ്ട് ‎സുഖമാസ്വദിച്ചപോലെ ഇപ്പോള്‍ നിങ്ങളും നിങ്ങളുടെ വിഹിതമുപയോഗിച്ച് ‎സുഖിച്ചു. അവര്‍ അധര്മളങ്ങളില്‍ ആണ്ടിറങ്ങിയപോലെ നിങ്ങളും ‎ആണ്ടിറങ്ങി. ഇഹത്തിലും പരത്തിലും അവരുടെ പ്രവര്ത്തിനങ്ങളൊക്കെയും ‎പാഴായിരിക്കുന്നു. അവര്‍ തന്നെയാണ് നഷ്ടം പറ്റിയവര്‍. ‎
\end{malayalam}}
\flushright{\begin{Arabic}
\quranayah[9][70]
\end{Arabic}}
\flushleft{\begin{malayalam}
ഇവരുടെ മുന്ഗാ്മികളുടെ വൃത്താന്തം ഇവര്ക്ക് വന്നെത്തിയിട്ടില്ലേ? ‎നൂഹിന്റെയും ആദിന്റെയും സമൂദിന്റെയും സമുദായങ്ങളുടെയും ‎ഇബ്റാഹീമിന്റെ ജനതയുടെയും മദ്യന്കാുരുടെയും കീഴ്മേല്‍ മറിക്കപ്പെട്ട ‎നാടുകളുടെയും കഥ! അവരിലേക്കുള്ള ദൈവദൂതന്മാര്‍ വ്യക്തമായ ‎തെളിവുകളുമായി അവരെ സമീപിച്ചു. അപ്പോള്‍ അല്ലാഹു അവരോട് ഒരു ‎ദ്രോഹവും കാണിച്ചില്ല. എന്നാല്‍ അവര്‍ തങ്ങളെത്തന്നെ ‎ദ്രോഹിക്കുകയായിരുന്നു. ‎
\end{malayalam}}
\flushright{\begin{Arabic}
\quranayah[9][71]
\end{Arabic}}
\flushleft{\begin{malayalam}
സത്യവിശ്വാസികളായ സ്ത്രീ പുരുഷന്മാര്‍ പരസ്പരം സഹായികളാണ്. ‎അവര്‍ നന്മ കല്പിരക്കുന്നു. തിന്മ തടയുന്നു. നമസ്കാരം നിഷ്ഠയോടെ ‎നിര്വിഹിക്കുന്നു. സകാത്ത് നല്കുരന്നു. അല്ലാഹുവിനെയും അവന്റെ ‎ദൂതനെയും അനുസരിക്കുന്നു. സംശയമില്ല; അല്ലാഹു അവരോട് കരുണ ‎കാണിക്കും. അല്ലാഹു പ്രതാപിയും യുക്തിമാനും തന്നെ; തീര്ച്ചല. ‎
\end{malayalam}}
\flushright{\begin{Arabic}
\quranayah[9][72]
\end{Arabic}}
\flushleft{\begin{malayalam}
സത്യവിശ്വാസികളായ സ്ത്രീ പുരുഷന്മാര്ക്ക്യ അല്ലാഹു താഴ്ഭാഗത്തൂടെ ‎അരുവികളൊഴുകുന്ന സ്വര്‍ഗീയാരാമങ്ങള്‍ വാഗ്ദാനം ചെയ്തിരിക്കുന്നു. ‎അവരവിടെ സ്ഥിരവാസികളായിരിക്കും. നിത്യവാസത്തിനുള്ള ആ ‎സ്വര്ഗീ യാരാമങ്ങളില്‍ അവര്ക്ക് ശ്രേഷ്ഠമായ പാര്പ്പി ടങ്ങളുണ്ട്; ‎സര്വോഗപരി അല്ലാഹുവിന്റെ നിറഞ്ഞ പ്രീതിയും. അതെത്ര മഹത്തരം! ‎ഉജ്ജ്വലമായ വിജയവും അതുതന്നെ. ‎
\end{malayalam}}
\flushright{\begin{Arabic}
\quranayah[9][73]
\end{Arabic}}
\flushleft{\begin{malayalam}
നബിയേ, സത്യനിഷേധികളോടും കപടവിശ്വാസികളോടും സമരം ചെയ്യുക. ‎അവരോട് പരുഷമായി പെരുമാറുക. അവരെത്തുക നരകത്തീയിലാണ്. ‎അതെത്ര ചീത്ത സങ്കേതം! ‎
\end{malayalam}}
\flushright{\begin{Arabic}
\quranayah[9][74]
\end{Arabic}}
\flushleft{\begin{malayalam}
തങ്ങള്‍ അങ്ങനെ പറഞ്ഞിട്ടേയില്ലെന്ന് അവര്‍ അല്ലാഹുവിന്റെ പേരില്‍ ‎ആണയിടുന്നു. എന്നാല്‍ ഉറപ്പായും അവര്‍ സത്യനിഷേധത്തിന്റെ വാക്ക് ‎ഉരുവിട്ടിരിക്കുന്നു. ഇസ്ലാം സ്വീകരിച്ചശേഷം അവര്‍ സത്യനിഷേധികളായി. ‎തങ്ങള്ക്കുട ചെയ്യാനാവാത്ത ചിലത് പ്രവര്ത്തി്ക്കാന്‍ മുതിരുകയും ചെയ്തു. ‎എന്നാല്‍ അവരുടെ ഈ ശത്രുതക്കൊക്കെയും കാരണം അല്ലാഹുവും ‎അവന്റെ ദൂതനും ദൈവാനുഗ്രഹത്താല്‍ അവര്ക്ക് സുഭിക്ഷത ‎നല്കിെയതുമാത്രമാണ്. ഇനിയെങ്കിലും അവര്‍ പശ്ചാത്തപിക്കുകയാണെങ്കില്‍ ‎അതാണവര്ക്ക്ു നല്ലത്. അഥവാ, അവര്‍ പിന്തിരിഞ്ഞുപോവുകയാണെങ്കില്‍ ‎അല്ലാഹു ഇഹത്തിലും പരത്തിലും അവര്ക്ക് നോവേറിയ ശിക്ഷ നല്കും . ‎ഇവിടെ ഭൂമിയിലും അവര്ക്ക് ഒരു രക്ഷകനോ സഹായിയോ ‎ഉണ്ടാവുകയില്ല. ‎
\end{malayalam}}
\flushright{\begin{Arabic}
\quranayah[9][75]
\end{Arabic}}
\flushleft{\begin{malayalam}
അല്ലാഹു തന്റെ ഔദാര്യത്താല്‍ തങ്ങള്ക്ക്യ സമ്പത്ത് നല്കുഭകയാണെങ്കില്‍ ‎തീര്ച്ച യായും തങ്ങള്‍ ദാനം ചെയ്യുമെന്നും സച്ചരിതരിലുള്പ്പെ ടുമെന്നും ‎അല്ലാഹുവോട് കരാര്‍ ചെയ്തവരും അവരിലുണ്ട്. ‎
\end{malayalam}}
\flushright{\begin{Arabic}
\quranayah[9][76]
\end{Arabic}}
\flushleft{\begin{malayalam}
അങ്ങനെ അല്ലാഹു അവര്ക്ക്് തന്റെ ഔദാര്യത്താല്‍ സമ്പത്ത് നല്കിയ. ‎എന്നാല്‍ അവരതില്‍ പിശുക്കു കാണിക്കുകയും പ്രതിജ്ഞ പാലിക്കാതെ ‎പിന്മാറുകയുമാണുണ്ടായത്. ‎
\end{malayalam}}
\flushright{\begin{Arabic}
\quranayah[9][77]
\end{Arabic}}
\flushleft{\begin{malayalam}
അതേ തുടര്ന്ന് അല്ലാഹു അവരുടെ മനസ്സുകളില്‍ കപടത കുടിയിരുത്തി. ‎അവര്‍ അവനെ കണ്ടുമുട്ടുന്ന ദിനം വരെയും അതായിരിക്കും അവരുടെ ‎അവസ്ഥ. അല്ലാഹുവോട് അവര്‍ ചെയ്ത പ്രതിജ്ഞ ലംഘിച്ചതിനാലും ‎അവര്‍ കള്ളം പറഞ്ഞുകൊണ്ടിരുന്നതിനാലുമാണിത്. ‎
\end{malayalam}}
\flushright{\begin{Arabic}
\quranayah[9][78]
\end{Arabic}}
\flushleft{\begin{malayalam}
അവരുടെ രഹസ്യവും ഗൂഢാലോചനകളുമെല്ലാം അല്ലാഹു ‎അറിയുന്നുണ്ടെന്ന് അവര്‍ മനസ്സിലാക്കിയിട്ടില്ലേ? തീര്ച്ചരയായും അഭൌതിക ‎കാര്യങ്ങള്‍ അറിയുന്നവനാണ് അല്ലാഹുവെന്നും? ‎
\end{malayalam}}
\flushright{\begin{Arabic}
\quranayah[9][79]
\end{Arabic}}
\flushleft{\begin{malayalam}
സ്വമനസ്സാലെ ദാനധര്മവങ്ങള്‍ ചെയ്യുന്ന സത്യവിശ്വാസികളെയും സ്വന്തം ‎അധ്വാനമല്ലാതൊന്നും ദൈവമാര്ഗ്ത്തിലര്പ്പി ക്കാനില്ലാത്തവരെയും ‎പഴിപറയുന്നവരാണവര്‍. അങ്ങനെ ആ വിശ്വാസികളെ അവര്‍ ‎പരിഹസിക്കുന്നു. അല്ലാഹു അവരെയും പരിഹാസ്യരാക്കിയിരിക്കുന്നു. ‎അവര്ക്ക് നോവേറിയ ശിക്ഷയുമുണ്ട്. ‎
\end{malayalam}}
\flushright{\begin{Arabic}
\quranayah[9][80]
\end{Arabic}}
\flushleft{\begin{malayalam}
നീ അവര്ക്കു്വേണ്ടി മാപ്പപേക്ഷിക്കുകയോ അപേക്ഷിക്കാതിരിക്കുകയോ ‎ചെയ്യുക. നീ അവര്ക്കുപവേണ്ടി എഴുപതു പ്രാവശ്യം പാപമോചനത്തിനു ‎പ്രാര്ഥി്ച്ചാലും അല്ലാഹു അവര്ക്ക്ണ പൊറുത്തുകൊടുക്കുകയില്ല. കാരണം ‎അല്ലാഹുവെയും അവന്റെ ദൂതനെയും തള്ളിപ്പറഞ്ഞവരാണവര്‍. ‎അധാര്മിവകരായ ആളുകളെ അല്ലാഹു നേര്വവഴിയിലാക്കുകയില്ല. ‎
\end{malayalam}}
\flushright{\begin{Arabic}
\quranayah[9][81]
\end{Arabic}}
\flushleft{\begin{malayalam}
ദൈവദൂതനെ ധിക്കരിച്ച് യുദ്ധത്തില്നി്ന്ന് പിന്മാറി വീട്ടിലിരുന്നതില്‍ ‎സന്തോഷിക്കുന്നവരാണവര്‍. തങ്ങളുടെ ധനംകൊണ്ടും ദേഹംകൊണ്ടും ‎ദൈവമാര്ഗ്ത്തില്‍ സമരം ചെയ്യുന്നത് അവര്ക്ക് അനിഷ്ടകരമായി. ‎അവരിങ്ങനെ പറയുകയും ചെയ്തു: "ഈ കൊടുംചൂടില്‍ നിങ്ങള്‍ ‎യുദ്ധത്തിനിറങ്ങിപ്പുറപ്പെടേണ്ട.” പറയുക: നരകത്തീ കൂടുതല്‍ ‎ചൂടേറിയതാണ്. അവര്‍ ബോധവാന്മാരായിരുന്നെങ്കില്‍ എത്ര നന്നായേനെ. ‎
\end{malayalam}}
\flushright{\begin{Arabic}
\quranayah[9][82]
\end{Arabic}}
\flushleft{\begin{malayalam}
അതിനാല്‍ അവര്‍ ഇത്തിരി ചിരിക്കുകയും പിന്നെ ഒത്തിരി കരയുകയും ‎ചെയ്യട്ടെ. അവരുടെ പ്രവര്ത്ത ന ഫലം അവ്വിധമാണ്. ‎
\end{malayalam}}
\flushright{\begin{Arabic}
\quranayah[9][83]
\end{Arabic}}
\flushleft{\begin{malayalam}
അല്ലാഹു നിന്നെ അവരിലൊരു കൂട്ടരുടെയടുത്ത് തിരിച്ചെത്തിക്കുകയും ‎പിന്നെ മറ്റൊരു യുദ്ധത്തിന് പോരാന്‍ അവര്‍ നിന്നോട് അനുവാദം ‎ചോദിക്കുകയും ചെയ്താല്‍ നീ പറയുക: "ഇനി നിങ്ങള്ക്കൊംരിക്കലും ‎എന്നോടൊത്ത് പുറപ്പെടാനാവില്ല. നിങ്ങള്‍ എന്റെ കൂടെ ശത്രുവോട് ‎പൊരുതുന്നതുമല്ല. തീര്ച്ചപയായും ആദ്യ തവണ യുദ്ധത്തില്‍ ‎നിന്നൊഴിഞ്ഞുനിന്നതില്‍ തൃപ്തിയടയുകയാണല്ലോ നിങ്ങള്‍ ചെയ്തത്. ‎അതിനാല്‍ യുദ്ധത്തില്‍ നിന്ന് വിട്ടൊഴിഞ്ഞു ചടഞ്ഞിരിക്കുന്നവരോടൊപ്പം ‎നിങ്ങളും ഇരുന്നുകൊള്ളുക.” ‎
\end{malayalam}}
\flushright{\begin{Arabic}
\quranayah[9][84]
\end{Arabic}}
\flushleft{\begin{malayalam}
അവരില്‍ നിന്ന് ആരു മരണമടഞ്ഞാലും അവനുവേണ്ടി നീ ഒരിക്കലും ‎നമസ്കരിക്കരുത്. അവന്റെ കുഴിമാടത്തിനടുത്ത് നില്ക്കരരുത്. തീര്ച്ചുയായും ‎അവര്‍ അല്ലാഹുവെയും അവന്റെ ദൂതനെയും തള്ളിപ്പറഞ്ഞവരാണ്. ‎അധാര്മികകരായി മരണമടഞ്ഞവരും. ‎
\end{malayalam}}
\flushright{\begin{Arabic}
\quranayah[9][85]
\end{Arabic}}
\flushleft{\begin{malayalam}
അവരുടെ സമ്പത്തും സന്താനങ്ങളും നിന്നെ വിസ്മയിപ്പിക്കാതിരിക്കട്ടെ. ‎അവയിലൂടെ അവരെ ഇഹലോകത്തുവെച്ചുതന്നെ ശിക്ഷിക്കണമെന്നാണ് ‎അല്ലാഹു ഉദ്ദേശിക്കുന്നത്. അവര്‍ സത്യനിഷേധികളായിരിക്കെത്തന്നെ ‎അവരുടെ ജീവന്‍ വെടിയണമെന്നും. ‎
\end{malayalam}}
\flushright{\begin{Arabic}
\quranayah[9][86]
\end{Arabic}}
\flushleft{\begin{malayalam}
‎"നിങ്ങള്‍ അല്ലാഹുവില്‍ വിശ്വസിക്കുകയും അവന്റെ ദൂതനോടൊപ്പം സമരം ‎നടത്തുകയും ചെയ്യുക” എന്ന ആഹ്വാനവുമായി വല്ല അധ്യായവും ‎അവതീര്ണയമായാല്‍ അവരിലെ സമ്പന്നര്‍ യുദ്ധത്തില്‍ നിന്നൊഴിവാകാന്‍ ‎നിന്നോട് സമ്മതം തേടും. അവര്‍ പറയും: "ഞങ്ങളെ വിട്ടേക്കൂ. ഞങ്ങള്‍ ‎വീട്ടിലിരിക്കുന്നവരോടൊപ്പം കഴിയാം.” ‎
\end{malayalam}}
\flushright{\begin{Arabic}
\quranayah[9][87]
\end{Arabic}}
\flushleft{\begin{malayalam}
യുദ്ധത്തില്നിുന്ന് മാറിനില്ക്കു ന്നവരോടൊപ്പം കഴിയാനാണ് അവരിഷ്ടപ്പെട്ടത്. ‎അവരുടെ മനസ്സുകള്ക്ക് മുദ്രവെക്കപ്പെട്ടിരിക്കുന്നു. അതിനാല്‍ അവരൊന്നും ‎മനസ്സിലാക്കുന്നില്ല. ‎
\end{malayalam}}
\flushright{\begin{Arabic}
\quranayah[9][88]
\end{Arabic}}
\flushleft{\begin{malayalam}
എന്നാല്‍ ദൈവദൂതനും കൂടെയുള്ള വിശ്വാസികളും തങ്ങളുടെ ‎ധനംകൊണ്ടും ദേഹംകൊണ്ടും സമരം ചെയ്തു. അവര്ക്കാ ണ് സകല ‎നന്മകളും. വിജയം വരിച്ചവരും അവര്‍ തന്നെ. ‎
\end{malayalam}}
\flushright{\begin{Arabic}
\quranayah[9][89]
\end{Arabic}}
\flushleft{\begin{malayalam}
അല്ലാഹു അവര്ക്ക് താഴ്ഭാഗത്തൂടെ അരുവികളൊഴുകുന്ന ‎സ്വര്ഗീ യാരാമങ്ങള്‍ ഒരുക്കിവെച്ചിരിക്കുന്നു. അവരതില്‍ ‎നിത്യവാസികളായിരിക്കും. അതിമഹത്തായ വിജയവും അതുതന്നെ. ‎
\end{malayalam}}
\flushright{\begin{Arabic}
\quranayah[9][90]
\end{Arabic}}
\flushleft{\begin{malayalam}
ഗ്രാമീണ അറബികളില്‍ ചിലരും യുദ്ധത്തില്നിംന്ന് ഒഴിഞ്ഞുനില്ക്കുളന്നതിന് ‎അനുവാദം തേടി വന്നല്ലോ. അല്ലാഹുവോടും അവന്റെ ദൂതനോടും ‎കള്ളംപറഞ്ഞുവന്നവര്‍ വീട്ടിലിരിക്കുകയും ചെയ്തു. അവരിലെ ‎സത്യനിഷേധികളെ അടുത്തുതന്നെ നോവേറിയ ശിക്ഷ ബാധിക്കും. ‎
\end{malayalam}}
\flushright{\begin{Arabic}
\quranayah[9][91]
\end{Arabic}}
\flushleft{\begin{malayalam}
ദുര്ബിലരും രോഗികളും ചെലവു ചെയ്യാന്‍ ഒന്നുമില്ലാത്തവരും യുദ്ധത്തില്‍ ‎നിന്ന് മാറിനില്ക്കുോന്നതില്‍ തെറ്റില്ല; അവര്‍ അല്ലാഹുവോടും അവന്റെ ‎ദൂതനോടും കൂറുപുലര്ത്തുകന്നവരാണെങ്കില്‍. ഇത്തരം സദ്വൃത്തരെ ‎കുറ്റപ്പെടുത്താന്‍ ന്യായമൊന്നുമില്ല. അല്ലാഹു ഏറെ പൊറുക്കുന്നവനും ‎പരമകാരുണികനുമാണ്. ‎
\end{malayalam}}
\flushright{\begin{Arabic}
\quranayah[9][92]
\end{Arabic}}
\flushleft{\begin{malayalam}
മറ്റൊരു വിഭാഗം തങ്ങളെ വാഹനങ്ങളില്‍ കൊണ്ടുപോകണമെന്ന ‎അപേക്ഷയുമായി നിന്റെ അടുത്തുവന്നു. നീ അവരോടു പറഞ്ഞു: ‎‎"നിങ്ങള്ക്കുന നല്കാനന്‍ ഞാന്‍ വാഹനമൊന്നും കാണുന്നില്ല.” ചെലവഴിക്കാന്‍ ‎ഒന്നും കണ്ടെത്താത്തതിന്റെ തീവ്രദുഃഖത്താല്‍ കണ്ണുകളില്‍ ‎വെള്ളംനിറച്ചുകൊണ്ട് അവര്‍ മടങ്ങിപ്പോയി. അവര്ക്കും കുറ്റമൊന്നുമില്ല. ‎
\end{malayalam}}
\flushright{\begin{Arabic}
\quranayah[9][93]
\end{Arabic}}
\flushleft{\begin{malayalam}
സമ്പന്നരായിരുന്നിട്ടും യുദ്ധത്തില്‍ നിന്നൊഴിയാന്‍ നിന്നോട് അനുവാദം ‎തേടുകയും പിന്തിരിഞ്ഞു നിന്നവരോടൊപ്പമാകുന്നതില്‍ തൃപ്തിയടയുകയും ‎ചെയ്തവരെ മാത്രമേ കുറ്റപ്പെടുത്താന്‍ വഴിയുള്ളൂ. അല്ലാഹു അവരുടെ ‎മനസ്സുകള്ക്ക് മുദ്രവെച്ചിരിക്കുന്നു. അതിനാല്‍ അവര്‍ ഒന്നും അറിയുന്നില്ല. ‎
\end{malayalam}}
\flushright{\begin{Arabic}
\quranayah[9][94]
\end{Arabic}}
\flushleft{\begin{malayalam}
യുദ്ധത്തില്നിരന്ന് നിങ്ങള്‍ അവരുടെ അടുത്ത് മടങ്ങിയെത്തിയാല്‍ അവര്‍ ‎നിങ്ങളോട് പല ഒഴികഴിവുകളും ബോധിപ്പിക്കും. പറയുക: "നിങ്ങള്‍ ‎ഒഴികഴിവൊന്നും ബോധിപ്പിക്കേണ്ട. നിങ്ങളെ ഞങ്ങളൊട്ടും ‎വിശ്വസിക്കുന്നില്ല. നിങ്ങളുടെ വിവരങ്ങള്‍ അല്ലാഹു ഞങ്ങളെ ‎ധരിപ്പിച്ചിട്ടുണ്ട്. നിങ്ങളുടെ പ്രവര്ത്തങനങ്ങളെല്ലാം അല്ലാഹുവും അവന്റെ ‎ദൂതനും കണ്ടറിയുന്നുണ്ട്. പിന്നീട് മറഞ്ഞതും തെളിഞ്ഞതും ‎അറിയുന്നവന്റെ അടുത്തേക്ക് നിങ്ങള്‍ മടക്കപ്പെടും. അപ്പോള്‍ നിങ്ങള്‍ ‎പ്രവര്ത്തി്ച്ചുകൊണ്ടിരിക്കുന്നതിനെപ്പറ്റി അവന്‍ നിങ്ങളെ ‎വിവരമറിയിക്കും.” ‎
\end{malayalam}}
\flushright{\begin{Arabic}
\quranayah[9][95]
\end{Arabic}}
\flushleft{\begin{malayalam}
നിങ്ങള്‍ അവരിലേക്ക് മടങ്ങിച്ചെല്ലുമ്പോള്‍ അവര്‍ നിങ്ങളോട് ‎അല്ലാഹുവിന്റെ പേരില്‍ ആണയിട്ടുകൊണ്ടിരിക്കും. നിങ്ങള്‍ അവരെ ‎ഒഴിവാക്കാന്‍ വേണ്ടിയാണത്. ഏതായാലും നിങ്ങള്‍ അവരെ വിട്ടേക്കുക. ‎അവര്‍ ഏറെ നീചന്മാരാണ്. അവരുടെ താവളം നരകമാണ്. അവര്‍ ‎പ്രവര്ത്തി്ച്ചുകൊണ്ടിരുന്നതിനുള്ള അര്ഹടമായ പ്രതിഫലം അതാണല്ലോ. ‎
\end{malayalam}}
\flushright{\begin{Arabic}
\quranayah[9][96]
\end{Arabic}}
\flushleft{\begin{malayalam}
നിങ്ങള്‍ അവരെ സംബന്ധിച്ച് സംതൃപ്തരാകാനാണ് അവര്‍ നിങ്ങളോട് ‎അല്ലാഹുവിന്റെ പേരില്‍ ആണയിടുന്നത്. അഥവാ, നിങ്ങളവരെ ‎തൃപ്തിപ്പെട്ടാലും അല്ലാഹു അധാര്മിണകരായ ജനത്തെ തൃപ്തിപ്പെടുകയില്ല. ‎
\end{malayalam}}
\flushright{\begin{Arabic}
\quranayah[9][97]
\end{Arabic}}
\flushleft{\begin{malayalam}
ഗ്രാമീണ അറബികള്‍ കടുത്ത സത്യനിഷേധവും കാപട്യവുമുള്ളവരത്രേ. ‎അല്ലാഹു തന്റെ ദൂതന്ന് ഇറക്കിക്കൊടുത്ത നിയമപരിധികള്‍ ‎അറിയാതിരിക്കാന്‍ കൂടുതല്‍ സാധ്യതയുള്ളതും അവര്ക്കായണ്. അല്ലാഹു ‎എല്ലാം അറിയുന്നവനും യുക്തിമാനുമാണ്. ‎
\end{malayalam}}
\flushright{\begin{Arabic}
\quranayah[9][98]
\end{Arabic}}
\flushleft{\begin{malayalam}
ധനം ചെലവഴിക്കുന്നത് നഷ്ടമായി കാണുന്നവരും നിങ്ങളെ കാലവിപത്ത് ‎ബാധിക്കുന്നത് കാത്തിരിക്കുന്നവരും ആ ഗ്രാമീണ അറബികളിലുണ്ട്. ‎എന്നാല്‍ കാലക്കേട് പിടികൂടാന്‍ പോകുന്നത് അവരെത്തന്നെയാണ്. അല്ലാഹു ‎എല്ലാം കേള്ക്കു ന്നവനും അറിയുന്നവനുമാണ്. ‎
\end{malayalam}}
\flushright{\begin{Arabic}
\quranayah[9][99]
\end{Arabic}}
\flushleft{\begin{malayalam}
ഗ്രാമീണ അറബികളില്‍ തന്നെ അല്ലാഹുവിലും അന്ത്യദിനത്തിലും ‎വിശ്വസിക്കുന്നവരുമുണ്ട്. അവര്‍ തങ്ങള്‍ ചെലവഴിക്കുന്നതിനെ ‎അല്ലാഹുവിന്റെ സാമീപ്യം സിദ്ധിക്കാനും പ്രവാചകന്റെ പ്രാര്ഥതന ‎ലഭിക്കാനുമുള്ള മാര്ഗകമായി കാണുന്നു. അറിയുക: തീര്ച്ചലയായും അതവര്ക്ക്ത ‎ദൈവസാമീപ്യം സമ്മാനിക്കും. അല്ലാഹു അവരെ തന്റെ അനുഗ്രഹത്തില്‍ ‎പ്രവേശിപ്പിക്കും. അല്ലാഹു ഏറെ പൊറുക്കുന്നവനും പരമദയാലുവുമാണ്. ‎
\end{malayalam}}
\flushright{\begin{Arabic}
\quranayah[9][100]
\end{Arabic}}
\flushleft{\begin{malayalam}
സത്യമാര്ഗപത്തില്‍ ആദ്യം മുന്നോട്ടു വന്ന മുഹാജിറുകളിലും ‎അന്സ്വാ്റുകളിലും സല്ക്കുര്മനങ്ങളിലൂടെ അവരെ പിന്തുടരുന്നവരിലും ‎അല്ലാഹു സംതൃപ്തനായിരിക്കുന്നു. അവര്‍ അവനിലും സംതൃപ്തരാണ്. ‎അവന്‍ അവര്ക്കാ യി താഴ്ഭാഗത്തിലൂടെ അരുവികളൊഴുകുന്ന ‎സ്വര്ഗീ യാരാമങ്ങള്‍ തയ്യാറാക്കിവെച്ചിട്ടുണ്ട്. അവരവിടെ ‎സ്ഥിരവാസികളായിരിക്കും. അതിമഹത്തായ വിജയവും അതു തന്നെ. ‎
\end{malayalam}}
\flushright{\begin{Arabic}
\quranayah[9][101]
\end{Arabic}}
\flushleft{\begin{malayalam}
നിങ്ങളുടെ ചുറ്റുമുള്ള ഗ്രാമീണ അറബികളിലും കപടവിശ്വാസികളുണ്ട്. ‎മദീനാ നിവാസികളിലുമുണ്ട്. അവര്‍ കാപട്യത്തിലാണ്ടുപോയിരിക്കുന്നു. ‎നിനക്ക് അവരെ അറിയില്ല. എന്നാല്‍ നാം അവരെ അറിയുന്നു. രണ്ടു ‎തവണ നാം അവരെ ശിക്ഷിക്കും. പിന്നീട് അവരെ ഭീകരമായ ശിക്ഷയിലേക്ക് ‎തള്ളുകയും ചെയ്യും. ‎
\end{malayalam}}
\flushright{\begin{Arabic}
\quranayah[9][102]
\end{Arabic}}
\flushleft{\begin{malayalam}
തങ്ങളുടെ തെറ്റുകള്‍ സ്വയം ഏറ്റുപറയുന്ന ചിലരുണ്ട്. അവര്‍ ‎സല്ക്കടര്മിങ്ങളും ദുഷ്കര്മ്ങ്ങളും കൂട്ടിക്കലര്ത്തി യിരിക്കുന്നു. അല്ലാഹു ‎അവരുടെ പശ്ചാത്താപം സ്വീകരിച്ചേക്കാം. അല്ലാഹു ഏറെ ‎പൊറുക്കുന്നവനും പരമദയാലുവുമാണ്. ‎
\end{malayalam}}
\flushright{\begin{Arabic}
\quranayah[9][103]
\end{Arabic}}
\flushleft{\begin{malayalam}
നീ അവരുടെ സ്വത്തില്നിവന്ന് സകാത്ത് വസൂല്‍ ചെയ്യുക. അതവരെ ‎ശുദ്ധീകരിക്കുകയും സംസ്കരിക്കുകയും ചെയ്യും. നീ അവര്ക്കു്വേണ്ടി ‎പ്രാര്ഥിിക്കുക. നിശ്ചയമായും നിന്റെ പ്രാര്ഥുന അവര്ക്ക്ള ശാന്തിയേകും. ‎അല്ലാഹു എല്ലാം കേള്ക്കുുന്നവനും അറിയുന്നവനുമാണ്. ‎
\end{malayalam}}
\flushright{\begin{Arabic}
\quranayah[9][104]
\end{Arabic}}
\flushleft{\begin{malayalam}
അവര്ക്കിറിഞ്ഞുകൂടെ, അല്ലാഹു തന്റെ ദാസന്മാരുടെ പശ്ചാത്താപം ‎സ്വീകരിക്കുന്നവനും ദാനധര്മ്ങ്ങള്‍ ഏറ്റുവാങ്ങുന്നവനുമാണെന്ന്? ‎തീര്ച്ചകയായും അല്ലാഹു ധാരാളമായി പശ്ചാത്താപം സ്വീകരിക്കുന്നവനും ‎പരമദയാലുവുമെന്നും. ‎
\end{malayalam}}
\flushright{\begin{Arabic}
\quranayah[9][105]
\end{Arabic}}
\flushleft{\begin{malayalam}
പറയുക: നിങ്ങള്‍ പ്രവര്ത്തിതച്ചുകൊണ്ടിരിക്കുക. അല്ലാഹുവും അവന്റെ ‎ദൂതനും സത്യവിശ്വാസികളുമൊക്കെ നിങ്ങളുടെ കര്മിങ്ങള്‍ കാണും. ‎അവസാനം അകവും പുറവും അറിയുന്നവന്റെ അടുത്തേക്ക് നിങ്ങള്‍ ‎ചെന്നെത്തും. അപ്പോള്‍ നിങ്ങള്‍ പ്രവര്ത്തിനച്ചുകൊണ്ടിരുന്നതിനെപ്പറ്റി അവന്‍ ‎നിങ്ങളെ വിവരമറിയിക്കും. ‎
\end{malayalam}}
\flushright{\begin{Arabic}
\quranayah[9][106]
\end{Arabic}}
\flushleft{\begin{malayalam}
അല്ലാഹുവിന്റെ തീരുമാനത്തിനായി പ്രശ്നം മാറ്റിവെക്കപ്പെട്ട മറ്റൊരു ‎കൂട്ടരുമുണ്ട്. ഒന്നുകില്‍ അവന്‍ അവരെ ശിക്ഷിക്കും. അല്ലെങ്കില്‍ അവരുടെ ‎പശ്ചാത്താപം സ്വീകരിക്കും. അല്ലാഹു എല്ലാം അറിയുന്നവനും ‎യുക്തിമാനുമാണ്. ‎
\end{malayalam}}
\flushright{\begin{Arabic}
\quranayah[9][107]
\end{Arabic}}
\flushleft{\begin{malayalam}
ദ്രോഹംവരുത്താനും സത്യനിഷേധത്തെ സഹായിക്കാനും ‎വിശ്വാസികള്ക്കിംടയില്‍ ഭിന്നതയുണ്ടാക്കാനും നേരത്തെ അല്ലാഹുവോടും ‎അവന്റെ ദൂതനോടും യുദ്ധംചെയ്തവന് താവളമൊരുക്കാനുമായി പള്ളി‎‎യുണ്ടാക്കിയവരും അവരിലുണ്ട്. നല്ലതല്ലാതൊന്നും ഞങ്ങള്‍ ‎ഉദ്ദേശിച്ചിട്ടില്ലെന്ന് അവര്‍ ആണയിട്ടു പറയും. എന്നാല്‍ തീര്ച്ചുയായും അവര്‍ ‎കള്ളം പറയുന്നവരാണെന്ന് അല്ലാഹു സാക്ഷ്യം വഹിക്കുന്നു. ‎
\end{malayalam}}
\flushright{\begin{Arabic}
\quranayah[9][108]
\end{Arabic}}
\flushleft{\begin{malayalam}
നീ ഒരിക്കലും അതില്‍ നമസ്കരിക്കരുത്. തുടക്കം മുതല്ക്കു തന്നെ ‎ദൈവഭക്തിയില്‍ പടുത്തുയര്ത്തരപ്പെട്ട പള്ളിയാണ് നിനക്ക് നിന്നു ‎നമസ്കരിക്കാന്‍ ഏറ്റം അര്ഹം്. വിശുദ്ധി വരിക്കാനിഷ്ടപ്പെടുന്നവരുള്ളത് ‎അവിടെയാണ്. അല്ലാഹു വിശുദ്ധി വരിക്കുന്നവരെ ഇഷ്ടപ്പെടുന്നു. ‎
\end{malayalam}}
\flushright{\begin{Arabic}
\quranayah[9][109]
\end{Arabic}}
\flushleft{\begin{malayalam}
ഒരാള്‍ അല്ലാഹുവോടുള്ള കറയറ്റ ഭക്തിയിലും അവന്റെ പ്രീതിയിലും ‎തന്റെ കെട്ടിടം സ്ഥാപിച്ചു. മറ്റൊരാള്‍ അടിമണ്ണിളകി പൊളിഞ്ഞുവീഴാന്‍ ‎പോകുന്ന മണല്ത്തസട്ടിന്റെ വക്കില്‍ കെട്ടിടം പണിതു. അങ്ങനെയത് ‎അവനെയും കൊണ്ട് നേരെ നരകത്തീയില്‍ തകര്ന്നു വീഴുകയും ചെയ്തു. ‎ഇവരില്‍ ആരാണുത്തമന്‍? അക്രമികളായ ജനത്തെ അല്ലാഹു ‎നേര്വ്ഴിയിലാക്കുകയില്ല. ‎
\end{malayalam}}
\flushright{\begin{Arabic}
\quranayah[9][110]
\end{Arabic}}
\flushleft{\begin{malayalam}
അവര്‍ പടുത്തുയര്ത്തി്യ അവരുടെ ആ കെട്ടിടം അവരുടെ ‎മനസ്സുകളിലെന്നും ശങ്കയുണര്ത്തി്ക്കൊണ്ടേയിരിക്കും. അവരുടെ ഹൃദയങ്ങള്‍ ‎ശിഥിലമായിത്തീരും വരെ അതിനറുതിയില്ല. അല്ലാഹു എല്ലാം ‎അറിയുന്നവനും യുക്തിമാനുമാണ്. ‎
\end{malayalam}}
\flushright{\begin{Arabic}
\quranayah[9][111]
\end{Arabic}}
\flushleft{\begin{malayalam}
അല്ലാഹു സത്യവിശ്വാസികളില്‍ നിന്ന് അവര്ക്ക് ‎സ്വര്ഗുമുണ്ടെന്നവ്യവസ്ഥയില്‍ അവരുടെ ദേഹവും ധനവും വിലയ്ക്കു ‎വാങ്ങിയിരിക്കുന്നു. അവര്‍ അല്ലാഹുവിന്റെ മാര്ഗയത്തില്‍ യുദ്ധം ചെയ്യുന്നു. ‎അങ്ങനെ വധിക്കുകയും വധിക്കപ്പെടുകയും ചെയ്യുന്നു. അവര്ക്ക് ‎സ്വര്ഗ മുണ്ടെന്നത് അല്ലാഹു തന്റെ മേല്‍ പാലിക്കല്‍ ബാധ്യതയായി ‎നിശ്ചയിച്ച സത്യനിഷ്ഠമായ വാഗ്ദാനമാണ്. തൌറാത്തിലും ഇഞ്ചീലി‎‎ലും ഖുര്ആഷനിലും അതുണ്ട്. അല്ലാഹുവെക്കാള്‍ കരാര്‍ ‎പാലിക്കുന്നവനായി ആരുണ്ട്? അതിനാല്‍ നിങ്ങള്‍ നടത്തിയ കച്ചവട ‎ഇടപാടില്‍ സന്തോഷിച്ചുകൊള്ളുക. അതിമഹത്തായ വിജയവും അതുതന്നെ. ‎
\end{malayalam}}
\flushright{\begin{Arabic}
\quranayah[9][112]
\end{Arabic}}
\flushleft{\begin{malayalam}
പശ്ചാത്തപിച്ചു മടങ്ങുന്നവര്‍, അല്ലാഹുവെ ‎കീഴ്വണങ്ങിക്കൊണ്ടിരിക്കുന്നവര്‍, അവനെ കീര്ത്തി ച്ചുകൊണ്ടിരിക്കുന്നവര്‍, ‎വ്രതമനുഷ്ഠിക്കുന്നവര്‍, നമിക്കുകയും സാഷ്ടാംഗം പ്രണമിക്കുകയും ‎ചെയ്യുന്നവര്‍, നന്മ കല്പിിക്കുകയും തിന്മ വിലക്കുകയും ചെയ്യുന്നവര്‍, ‎അല്ലാഹുവിന്റെ നിയമപരിധികള്‍ പാലിക്കുന്നവര്‍, ഇവരൊക്കെയാണവര്‍. ‎സത്യവിശ്വാസികളെ ശുഭവാര്ത്തല അറിയിക്കുക. ‎
\end{malayalam}}
\flushright{\begin{Arabic}
\quranayah[9][113]
\end{Arabic}}
\flushleft{\begin{malayalam}
ബഹുദൈവവിശ്വാസികള്‍, കത്തിക്കാളുന്ന നരകത്തീയിന്റെ ‎അവകാശികളാണെന്ന് വ്യക്തമായിക്കഴിഞ്ഞശേഷം അവരുടെ ‎പാപമോചനത്തിന് പ്രാര്ഥികക്കാന്‍ പ്രവാചകന്നും സത്യവിശ്വാസികള്ക്കും ‎അനുവാദമില്ല. അവര്‍ അടുത്ത ബന്ധുക്കളാണെങ്കില്‍ പോലും. ‎
\end{malayalam}}
\flushright{\begin{Arabic}
\quranayah[9][114]
\end{Arabic}}
\flushleft{\begin{malayalam}
ഇബ്റാഹീം തന്റെ പിതാവിന്റെ പാപമോചനത്തിനായി പ്രാര്ഥിാച്ചത് ‎അദ്ദേഹം പിതാവിനോട് ചെയ്ത പ്രതിജ്ഞയുടെ പേരില്‍ മാത്രമായിരുന്നു. ‎അയാള്‍ അല്ലാഹുവിന്റെ ശത്രുവാണെന്ന് വ്യക്തമായപ്പോള്‍ അദ്ദേഹം ‎അയാളെ കയ്യൊഴിച്ചു. ഇബ്റാഹീം ഏറെ പശ്ചാത്താപമുള്ളവനും ‎സഹനശാലിയുമാണ്. ‎
\end{malayalam}}
\flushright{\begin{Arabic}
\quranayah[9][115]
\end{Arabic}}
\flushleft{\begin{malayalam}
ഒരു ജനതയെ നേര്വയഴിയിലാക്കിയശേഷം അവര്‍ സൂക്ഷിക്കേണ്ടത് ‎എന്തൊക്കെയാണെന്ന് വ്യക്തമാക്കിക്കൊടുക്കുന്നതുവരെ അല്ലാഹു അവരെ ‎പിഴച്ചവരായി കണക്കാക്കുകയില്ല. അല്ലാഹു എല്ലാ കാര്യങ്ങളെക്കുറിച്ചും ‎നന്നായറിയുന്നവനാണ്. ‎
\end{malayalam}}
\flushright{\begin{Arabic}
\quranayah[9][116]
\end{Arabic}}
\flushleft{\begin{malayalam}
സംശയമില്ല; ആകാശഭൂമികളുടെ ആധിപത്യം അല്ലാഹുവിന് മാത്രമാണ്. ‎അവന്‍ ജീവിപ്പിക്കുകയും മരിപ്പിക്കുകയും ചെയ്യുന്നു. അല്ലാഹുവല്ലാതെ ‎നിങ്ങള്ക്ക്് ഒരു രക്ഷകനും സഹായിയുമില്ല. ‎
\end{malayalam}}
\flushright{\begin{Arabic}
\quranayah[9][117]
\end{Arabic}}
\flushleft{\begin{malayalam}
പ്രവാചകന്നും പ്രയാസഘട്ടത്തില്‍ അദ്ദേഹത്തെ പിന്പാറ്റിയ ‎മുഹാജിറുകള്ക്കും അന്സാുറുകള്ക്കും അല്ലാഹു മാപ്പേകിയിരിക്കുന്നു. ‎അവരിലൊരു വിഭാഗത്തിന്റെ മനസ്സ് ഇത്തിരി പതറിപ്പോയിരുന്നുവെങ്കിലും! ‎പിന്നീട് അല്ലാഹു അവര്ക്ക് പൊറുത്തുകൊടുത്തു. തീര്ച്ച്യായും അല്ലാഹു ‎അവരോട് ഏറെ കൃപയുള്ളവനും പരമദയാലുവുമാണ്. ‎
\end{malayalam}}
\flushright{\begin{Arabic}
\quranayah[9][118]
\end{Arabic}}
\flushleft{\begin{malayalam}
തീരുമാനം മാറ്റിവെക്കപ്പെട്ട ആ മൂന്നാളുകള്ക്കും അവന്‍ ‎മാപ്പേകിയിരിക്കുന്നു. ഭൂമി ഏറെ വിശാലമായിരുന്നിട്ടുകൂടി അതവര്ക്ക് ‎ഇടുങ്ങിയതായിത്തീര്ന്നുപ. തങ്ങളുടെ മനസ്സുകള്തിന്നെ അവര്ക്ക്ട ‎കദനഭാരത്താല്‍ ദുര്വ്ഹമായിമാറി. അല്ലാഹുവിന്റെ പിടിയില്‍ നിന്ന് ‎രക്ഷപ്പെടാന്‍ അവനില്ത്തൂന്നെ അഭയം തേടലല്ലാതെ മാര്ഗനമില്ലെന്ന് അവര്ക്ക് ‎ബോധ്യമായി. അപ്പോള്‍ അല്ലാഹു അവരോട് കരുണ കാണിച്ചു. അവര്‍ ‎പശ്ചാത്തപിച്ചു മടങ്ങാന്‍. സംശയമില്ല; അല്ലാഹു പശ്ചാത്താപം ധാരാളമായി ‎സ്വീകരിക്കുന്നവനാണ്. പരമദയാലുവും. ‎
\end{malayalam}}
\flushright{\begin{Arabic}
\quranayah[9][119]
\end{Arabic}}
\flushleft{\begin{malayalam}
വിശ്വസിച്ചവരേ, നിങ്ങള്‍ അല്ലാഹുവെ സൂക്ഷിക്കുക. സത്യവാന്മാരോട് ‎സഹവസിക്കുക. ‎
\end{malayalam}}
\flushright{\begin{Arabic}
\quranayah[9][120]
\end{Arabic}}
\flushleft{\begin{malayalam}
മദീനക്കാര്ക്കുംന അവരുടെ പരിസരത്തുള്ള ഗ്രാമീണ അറബികള്ക്കും ‎അല്ലാഹുവിന്റെ ദൂതനെ വിട്ട് വീട്ടിലിരിക്കാനോ അദ്ദേഹത്തിന്റെ ജീവന്റെ ‎കാര്യം അവഗണിച്ച് തങ്ങളുടെ സ്വന്തം കാര്യം നോക്കാനോ അനുവാദമില്ല. ‎അല്ലാഹുവിന്റെ മാര്ഗ്ത്തില്‍ അവരെ ബാധിക്കുന്ന വിശപ്പ്, ദാഹം, ക്ഷീണം, ‎സത്യനിഷേധികളെ പ്രകോപിപ്പിക്കുന്ന ഇടങ്ങളിലൊക്കെയുള്ള അവരുടെ ‎സാന്നിധ്യം; എതിരാളിക്ക് ഏല്പിൊക്കുന്ന നാശം, ഇതൊക്കെയും അവരുടെ ‎പേരില്‍ സല്ക്കിര്മ്മായി രേഖപ്പെടുത്താതിരിക്കുകയില്ല എന്നതിനാലാണത്. ‎സല്ക്കധര്മിപകളുടെ പ്രതിഫലം അല്ലാഹു നഷ്ടപ്പെടുത്തുകയില്ല; തീര്ച്ചാ. ‎
\end{malayalam}}
\flushright{\begin{Arabic}
\quranayah[9][121]
\end{Arabic}}
\flushleft{\begin{malayalam}
അവര്‍ ചെലവഴിക്കുന്നത് ചെറുതായാലും വലുതായാലും അതും, ‎ഏതെങ്കിലും താഴ്വരയിലൂടെ അവര്‍ മുറിച്ചുകടക്കുന്നതും അവര്ക്ത ‎പുണ്യമായി രേഖപ്പെടുത്താതിരിക്കില്ല. അല്ലാഹു അവര്ക്ക് അവര്‍ ‎ചെയ്തുകൊണ്ടിരുന്ന അത്യുത്തമ വൃത്തികള്ക്ക്ു മഹത്തായ പ്രതിഫലം ‎നല്കാുനാണിത്. ‎
\end{malayalam}}
\flushright{\begin{Arabic}
\quranayah[9][122]
\end{Arabic}}
\flushleft{\begin{malayalam}
സത്യവിശ്വാസികള്‍ ഒന്നടങ്കം യുദ്ധത്തിന് പുറപ്പെടാവതല്ല. അവരില്‍ ഓരോ ‎വിഭാഗത്തില്‍ നിന്നും ഓരോ സംഘം മതത്തില്‍ അറിവുനേടാന്‍ ‎ഇറങ്ങിപ്പുറപ്പെടാത്തതെന്ത്? തങ്ങളുടെ ജനം അവരുടെ അടുത്തേക്ക് ‎മടങ്ങിവന്നാല്‍ അവര്ക്ക് താക്കീത് നല്കാരനുള്ള അറിവു നേടാനാണത്. ‎അതുവഴി അവര്‍ സൂക്ഷ്മത പുലര്ത്തുാന്നവരായേക്കാം. ‎
\end{malayalam}}
\flushright{\begin{Arabic}
\quranayah[9][123]
\end{Arabic}}
\flushleft{\begin{malayalam}
വിശ്വസിച്ചവരേ, നിങ്ങളുടെ അടുത്തുള്ള ആ സത്യനിഷേധികളോട് നിങ്ങള്‍ ‎യുദ്ധം ചെയ്യുക. അവര്‍ നിങ്ങളില്‍ കാര്ക്കേശ്യം കാണട്ടെ. അറിയുക: ‎അല്ലാഹു സൂക്ഷ്മതയുള്ളവരോടൊപ്പമാണ്. ‎
\end{malayalam}}
\flushright{\begin{Arabic}
\quranayah[9][124]
\end{Arabic}}
\flushleft{\begin{malayalam}
ഏതെങ്കിലും ഒരധ്യായം അവതീര്ണ്മായാല്‍ അവരില്‍ ചിലര്‍ ‎പരിഹാസത്തോടെ ചോദിക്കും: "നിങ്ങളില്‍ ആര്ക്കാളണ് ഇതുവഴി വിശ്വാസം ‎വര്ധിസച്ചത്?” എന്നാല്‍ അറിയുക: തീര്ച്ചഇയായും അത് സത്യവിശ്വാസികളുടെ ‎വിശ്വാസം വര്ധിചപ്പിച്ചിരിക്കുന്നു. അവരതില്‍ സന്തോഷിക്കുന്നവരുമാണ്. ‎
\end{malayalam}}
\flushright{\begin{Arabic}
\quranayah[9][125]
\end{Arabic}}
\flushleft{\begin{malayalam}
എന്നാല്‍ ദീനം പിടിച്ച മനസ്സിന്റെ ഉടമകള്ക്ക്് അത് തങ്ങളുടെ ‎മാലിന്യത്തിലേക്ക് കൂടുതല്‍ മാലിന്യം കൂട്ടിച്ചേര്ക്കു കയാണുണ്ടായത്. അവര്‍ ‎സത്യനിഷേധികളായിത്തന്നെ മരണമടയും. ‎
\end{malayalam}}
\flushright{\begin{Arabic}
\quranayah[9][126]
\end{Arabic}}
\flushleft{\begin{malayalam}
കൊല്ലംതോറും ഒന്നോ രണ്ടോ തവണ തങ്ങള്‍ പരീക്ഷണത്തിലകപ്പെടുന്നത് ‎അവര്‍ കാണുന്നില്ലേ? എന്നിട്ടും അവര്‍ പശ്ചാത്തപിച്ചു മടങ്ങുന്നില്ല. അവര്‍ ‎ചിന്തിച്ചറിയുന്നുമില്ല. ‎
\end{malayalam}}
\flushright{\begin{Arabic}
\quranayah[9][127]
\end{Arabic}}
\flushleft{\begin{malayalam}
ഓരോ അധ്യായം അവതരിക്കുമ്പോഴും നിങ്ങളെ ആരെങ്കിലും ‎കാണുന്നുണ്ടോയെന്ന ഭാവത്തില്‍ അവരന്യോന്യം നോക്കുന്നു. പിന്നീടവര്‍ ‎പിന്തിരിഞ്ഞു പോകുന്നു. അല്ലാഹു അവരുടെ മനസ്സുകളെ ‎തെറ്റിച്ചുകളഞ്ഞിരിക്കുന്നു. അവര്‍ കാര്യം മനസ്സിലാക്കാത്ത ‎ജനമായതിനാലാണത്. ‎
\end{malayalam}}
\flushright{\begin{Arabic}
\quranayah[9][128]
\end{Arabic}}
\flushleft{\begin{malayalam}
തീര്ച്ചയായും നിങ്ങള്ക്കിതാ നിങ്ങളില്നിന്നു തന്നെയുള്ള ഒരു ദൈവദൂതന്‍ ‎വന്നിരിക്കുന്നു. നിങ്ങള്‍ കഷ്ടപ്പെടുന്നത് അസഹ്യമായി അനുഭവപ്പെടുന്നവനും ‎നിങ്ങളുടെ കാര്യത്തില്‍ അതീവതല്പ‍രനുമാണവന്‍. സത്യവിശ്വാസികളോട് ‎ഏറെ കൃപയും കാരുണ്യവുമുള്ളവനും. ‎
\end{malayalam}}
\flushright{\begin{Arabic}
\quranayah[9][129]
\end{Arabic}}
\flushleft{\begin{malayalam}
എന്നിട്ടും അവര്‍ പുറന്തിരിഞ്ഞു നില്ക്കു കയാണെങ്കില്‍ പറയുക: എനിക്ക് ‎അല്ലാഹു മതി. അവനല്ലാതെ ദൈവമില്ല. ഞാന്‍ അവനില്‍ ‎ഭരമേല്പിമച്ചിരിക്കുന്നു. മഹത്തായ സിംഹാസനത്തിന്റെ നാഥനാണവന്‍. ‎
\end{malayalam}}
\chapter{\textmalayalam{യൂനുസ്}}
\begin{Arabic}
\Huge{\centerline{\basmalah}}\end{Arabic}
\flushright{\begin{Arabic}
\quranayah[10][1]
\end{Arabic}}
\flushleft{\begin{malayalam}
അലിഫ്-ലാം-റാഅ്. ഇത് ജ്ഞാന സമ്പന്നമായ വേദപുസ്തകത്തിലെ വചനങ്ങളാണ്.
\end{malayalam}}
\flushright{\begin{Arabic}
\quranayah[10][2]
\end{Arabic}}
\flushleft{\begin{malayalam}
തങ്ങളില്‍ നിന്നുതന്നെയുള്ള ഒരാള്‍ക്കു നാം ദിവ്യസന്ദേശം നല്‍കിയത്. ജനങ്ങള്‍ക്കൊരദ്ഭുതമായി തോന്നുന്നോ? ജനങ്ങള്‍ക്ക് മുന്നറിയിപ്പ് നല്‍കാനാണിത്. സത്യവിശ്വാസികള്‍ക്ക് തങ്ങളുടെ നാഥങ്കല്‍ സത്യത്തിനര്‍ഹമായ പദവിയുണ്ടെന്ന സുവാര്‍ത്ത അറിയിക്കാനും. സത്യനിഷേധികള്‍ പറഞ്ഞു: "ഇയാള്‍ വ്യക്തമായും ഒരു മായാജാലക്കാരന്‍ തന്നെ.”
\end{malayalam}}
\flushright{\begin{Arabic}
\quranayah[10][3]
\end{Arabic}}
\flushleft{\begin{malayalam}
ആകാശഭൂമികളെ ആറുനാളുകളിലായി പടച്ചുണ്ടാക്കിയ അല്ലാഹുവാണ് നിങ്ങളുടെ നാഥന്‍; സംശയമില്ല. പിന്നീട് അവന്‍ അധികാരപീഠത്തിലിരുന്ന് കാര്യങ്ങള്‍ നിയന്ത്രിച്ചുകൊണ്ടിരിക്കുന്നു. അവന്റെ അനുവാദം കിട്ടിയ ശേഷമല്ലാതെ ശിപാര്‍ശ ചെയ്യുന്ന ആരുമില്ല. അവനാണ് നിങ്ങളുടെ നാഥനായ അല്ലാഹു. അതിനാല്‍ അവനുമാത്രം വഴിപ്പെടുക. ഇതൊന്നും നിങ്ങള്‍ ചിന്തിച്ചു മനസ്സിലാക്കുന്നില്ലേ?
\end{malayalam}}
\flushright{\begin{Arabic}
\quranayah[10][4]
\end{Arabic}}
\flushleft{\begin{malayalam}
അവനിലേക്കാണ് നിങ്ങളുടെയൊക്കെ മടക്കം. ഇത് അല്ലാഹുവിന്റെ തെറ്റുപറ്റാത്ത വാഗ്ദാനമാണ്. തീര്‍ച്ചയായും അവനാണ് സൃഷ്ടികര്‍മം ആരംഭിക്കുന്നത്. പിന്നെ അതാവര്‍ത്തിക്കുകയും ചെയ്യുന്നു. സത്യവിശ്വാസം സ്വീകരിക്കുകയും സല്‍ക്കര്‍മങ്ങള്‍ പ്രവര്‍ത്തിക്കുകയും ചെയ്യുന്നവര്‍ക്ക് ന്യായമായ പ്രതിഫലം നല്‍കാനാണിത്. എന്നാല്‍ സത്യനിഷേധികള്‍ക്ക് തിളച്ചുമറിയുന്ന പാനീയമാണുണ്ടാവുക. നോവേറിയ ശിക്ഷയും. അവര്‍ സത്യത്തെ നിഷേധിച്ചുകൊണ്ടിരുന്നതിനാലാണിത്.
\end{malayalam}}
\flushright{\begin{Arabic}
\quranayah[10][5]
\end{Arabic}}
\flushleft{\begin{malayalam}
അവനാണ് സൂര്യനെ പ്രകാശമണിയിച്ചത്. ചന്ദ്രനെ പ്രശോഭിപ്പിച്ചതും അവന്‍ തന്നെ. അതിന് അവന്‍ വൃദ്ധിക്ഷയങ്ങള്‍ നിശ്ചയിച്ചിരിക്കുന്നു. അതുവഴി നിങ്ങള്‍ക്ക് കൊല്ലങ്ങളുടെ എണ്ണവും കണക്കും അറിയാന്‍. യാഥാര്‍ഥ്യ നിഷ്ഠമായല്ലാതെ അല്ലാഹു ഇതൊന്നും സൃഷ്ടിച്ചിട്ടില്ല. കാര്യം ഗ്രഹിക്കുന്ന ജനത്തിനായി അല്ലാഹു തെളിവുകള്‍ വിശദീകരിക്കുകയാണ്.
\end{malayalam}}
\flushright{\begin{Arabic}
\quranayah[10][6]
\end{Arabic}}
\flushleft{\begin{malayalam}
രാപ്പകലുകള്‍ മാറിമാറി വരുന്നതിലും ആകാശഭൂമികളില്‍ അല്ലാഹു സൃഷ്ടിച്ച മറ്റെല്ലാറ്റിലും ശ്രദ്ധ പുലര്‍ത്തുന്ന ജനത്തിന് ധാരാളം തെളിവുകളുണ്ട്.
\end{malayalam}}
\flushright{\begin{Arabic}
\quranayah[10][7]
\end{Arabic}}
\flushleft{\begin{malayalam}
നമ്മെ കണ്ടുമുട്ടുമെന്ന് പ്രതീക്ഷിക്കാത്തവര്‍, ഐഹികജീവിതംകൊണ്ട് തൃപ്തിയടഞ്ഞവര്‍, അതില്‍തന്നെ സമാധാനം കണ്ടെത്തിയവര്‍, നമ്മുടെ പ്രമാണങ്ങളെപ്പറ്റി അശ്രദ്ധ കാണിച്ചവര്‍-
\end{malayalam}}
\flushright{\begin{Arabic}
\quranayah[10][8]
\end{Arabic}}
\flushleft{\begin{malayalam}
അവരുടെയൊക്കെ താവളം നരകമാണ്. അവര്‍ പ്രവര്‍ത്തിച്ചുകൊണ്ടിരുന്നതിന്റെ പ്രതിഫലമാണത്.
\end{malayalam}}
\flushright{\begin{Arabic}
\quranayah[10][9]
\end{Arabic}}
\flushleft{\begin{malayalam}
എന്നാല്‍ സത്യവിശ്വാസം സ്വീകരിക്കുകയും സല്‍ക്കര്‍മങ്ങള്‍ പ്രവര്‍ത്തിക്കുകയും ചെയ്തവരെ, അവരുടെ സത്യവിശ്വാസം കാരണം അവരുടെ നാഥന്‍ നേര്‍വഴിയില്‍ നയിക്കും. അനുഗൃഹീതമായ സ്വര്‍ഗീയാരാമങ്ങളില്‍ അവരുടെ താഴ്ഭാഗത്തൂടെ അരുവികള്‍ ഒഴുകിക്കൊണ്ടിരിക്കും.
\end{malayalam}}
\flushright{\begin{Arabic}
\quranayah[10][10]
\end{Arabic}}
\flushleft{\begin{malayalam}
അവിടെ അവരുടെ പ്രാര്‍ഥന “അല്ലാഹുവേ, നീയെത്ര പരിശുദ്ധന്‍” എന്നായിരിക്കും. അവിടെ അവര്‍ക്കുള്ള അഭിവാദ്യം “സമാധാനം” എന്നും അവരുടെ പ്രാര്‍ഥനയുടെ സമാപനം “ലോകനാഥനായ അല്ലാഹുവിന് സ്തുതി”യെന്നുമായിരിക്കും.
\end{malayalam}}
\flushright{\begin{Arabic}
\quranayah[10][11]
\end{Arabic}}
\flushleft{\begin{malayalam}
ജനം ഭൌതിക നേട്ടത്തിന് തിടുക്കം കൂട്ടുന്നപോലെ അവര്‍ക്ക് വിപത്ത് വരുത്താന്‍ അല്ലാഹുവും ധൃതി കാട്ടുകയാണെങ്കില്‍ അവരുടെ കാലാവധി എന്നോ കഴിഞ്ഞുപോയേനെ. എന്നാല്‍, നാമുമായി കണ്ടുമുട്ടുമെന്ന് കരുതാത്തവരെ അവരുടെ അതിക്രമങ്ങളില്‍ അന്ധമായി വിഹരിക്കാന്‍ നാം അയച്ചുവിടുകയാണ്.
\end{malayalam}}
\flushright{\begin{Arabic}
\quranayah[10][12]
\end{Arabic}}
\flushleft{\begin{malayalam}
മനുഷ്യനെ വല്ല വിപത്തും ബാധിച്ചാല്‍ അവന്‍ നിന്നോ ഇരുന്നോ കിടന്നോ നമ്മോട് പ്രാര്‍ഥിച്ചുകൊണ്ടിരിക്കും. അങ്ങനെ അവനെ ആ വിപത്തില്‍ നിന്ന് നാം രക്ഷപ്പെടുത്തിയാല്‍ പിന്നെ അവനകപ്പെട്ട വിഷമസന്ധിയിലവന്‍ നമ്മോടു പ്രാര്‍ഥിച്ചിട്ടേയില്ലെന്ന വിധം നടന്നകലുന്നു. അതിരു കവിയുന്നവര്‍ക്ക് അവരുടെ ചെയ്തികള്‍ അവ്വിധം അലംകൃതമായി തോന്നുന്നു.
\end{malayalam}}
\flushright{\begin{Arabic}
\quranayah[10][13]
\end{Arabic}}
\flushleft{\begin{malayalam}
നിങ്ങള്‍ക്കു മുമ്പുള്ള പല തലമുറകളെയും അവര്‍ അതിക്രമം കാണിച്ചപ്പോള്‍ നാം നശിപ്പിച്ചിട്ടുണ്ട്. വ്യക്തമായ പ്രമാണങ്ങളുമായി അവരിലേക്കുള്ള നമ്മുടെ ദൂതന്മാര്‍ അവരെ സമീപിച്ചു. എന്നാല്‍ അവര്‍ വിശ്വസിച്ചതേയില്ല. അവ്വിധമാണ് കുറ്റവാളികളായ ജനത്തിന് നാം പ്രതിഫലം നല്‍കുന്നത്.
\end{malayalam}}
\flushright{\begin{Arabic}
\quranayah[10][14]
\end{Arabic}}
\flushleft{\begin{malayalam}
പിന്നെ അവര്‍ക്കുശേഷം നിങ്ങളെ നാം ഭൂമിയില്‍ പ്രതിനിധികളാക്കി. നിങ്ങളെങ്ങനെ ചെയ്യുന്നുവെന്ന് നോക്കിക്കാണാന്‍.
\end{malayalam}}
\flushright{\begin{Arabic}
\quranayah[10][15]
\end{Arabic}}
\flushleft{\begin{malayalam}
നമ്മുടെ സുവ്യക്തമായ വചനങ്ങള്‍ അവരെ ഓതിക്കേള്‍പ്പിക്കുമ്പോള്‍ നാമുമായി കണ്ടുമുട്ടുമെന്ന് കരുതാത്തവര്‍ പറയും: "നീ ഇതല്ലാത്ത മറ്റൊരു ഖുര്‍ആന്‍ കൊണ്ടുവരിക. അല്ലെങ്കില്‍ ഇതില്‍ മാറ്റങ്ങള്‍ വരുത്തുക.” പറയുക: "എന്റെ സ്വന്തം വകയായി അതില്‍ ഭേദഗതി വരുത്താന്‍ എനിക്കവകാശമില്ല. എനിക്ക് ബോധനമായി കിട്ടുന്നത് പിന്‍പറ്റുക മാത്രമാണ് ഞാന്‍ ചെയ്യുന്നത്. എന്റെ നാഥനെ ഞാന്‍ ധിക്കരിക്കുകയാണെങ്കില്‍ അതിഭയങ്കരമായ ഒരു നാളിലെ ശിക്ഷ എന്നെ ബാധിക്കുമെന്ന് ഞാന്‍ ഭയപ്പെടുന്നു.”
\end{malayalam}}
\flushright{\begin{Arabic}
\quranayah[10][16]
\end{Arabic}}
\flushleft{\begin{malayalam}
പറയുക: "അല്ലാഹു ഇച്ഛിച്ചിരുന്നെങ്കില്‍ ഞാനിത് നിങ്ങളെ ഓതിക്കേള്‍പ്പിക്കുമായിരുന്നില്ല. ഇതിനെക്കുറിച്ച് നിങ്ങളെ അറിയിക്കുകപോലുമില്ലായിരുന്നു. ഇതിനുമുമ്പ് കുറേക്കാലം ഞാന്‍ നിങ്ങള്‍ക്കിടയില്‍ കഴിഞ്ഞുകൂടിയതാണല്ലോ. നിങ്ങള്‍ ആലോചിക്കുന്നില്ലേ?”
\end{malayalam}}
\flushright{\begin{Arabic}
\quranayah[10][17]
\end{Arabic}}
\flushleft{\begin{malayalam}
അല്ലാഹുവിന്റെ പേരില്‍ കള്ളം കെട്ടിച്ചമക്കുകയോ അവന്റെ വചനങ്ങളെ കള്ളമാക്കി തള്ളുകയോ ചെയ്തവനെക്കാള്‍ കടുത്ത അക്രമി ആരുണ്ട്? പാപികള്‍ ഒരിക്കലും വിജയിക്കുകയില്ല.
\end{malayalam}}
\flushright{\begin{Arabic}
\quranayah[10][18]
\end{Arabic}}
\flushleft{\begin{malayalam}
അവര്‍ അല്ലാഹുവിന് പുറമെ, തങ്ങള്‍ക്ക് ദോഷമോ ഗുണമോ വരുത്താത്ത വസ്തുക്കളെ പൂജിച്ചുകൊണ്ടിരിക്കുകയാണ്. അവരവകാശപ്പെടുന്നു: "ഇവയൊക്കെ അല്ലാഹുവിന്റെ അടുത്ത് ഞങ്ങളുടെ ശിപാര്‍ശകരാണ്.” ചോദിക്കുക: ആകാശഭൂമികളിലുള്ളതായി അല്ലാഹുവിനറിയാത്ത കാര്യങ്ങള്‍ നിങ്ങള്‍ അവനെ അറിയിച്ചുകൊടുക്കുകയാണോ? അവര്‍ പങ്കുചേര്‍ക്കുന്നതില്‍ നിന്നൊക്കെ എത്രയോ പരിശുദ്ധനും പരമോന്നതനുമാണ് അല്ലാഹു.
\end{malayalam}}
\flushright{\begin{Arabic}
\quranayah[10][19]
\end{Arabic}}
\flushleft{\begin{malayalam}
മനുഷ്യരൊക്കെ ഒരു സമുദായമായിരുന്നു. പിന്നെ അവര്‍ ഭിന്നിച്ചു. നിന്റെ നാഥനില്‍നിന്നുള്ള പ്രഖ്യാപനം നേരത്തെ ഉണ്ടായിരുന്നില്ലെങ്കില്‍ അവര്‍ ഭിന്നിച്ചുകൊണ്ടിരിക്കുന്ന കാര്യത്തില്‍ ഇതിനകം തന്നെ തീര്‍പ്പ് കല്‍പിക്കപ്പെടുമായിരുന്നു.
\end{malayalam}}
\flushright{\begin{Arabic}
\quranayah[10][20]
\end{Arabic}}
\flushleft{\begin{malayalam}
അവര്‍ ചോദിക്കുന്നു: "ഈ പ്രവാചകന് തന്റെ നാഥനില്‍ നിന്ന് ഒരടയാളം ഇറക്കിക്കിട്ടാത്തതെന്ത്?” പറയുക: അഭൌതികമായ അറിവ് അല്ലാഹുവിന് മാത്രമേയുള്ളൂ. അതിനാല്‍ നിങ്ങള്‍ കാത്തിരിക്കുക. ഞാനും നിങ്ങളോടൊപ്പം കാത്തിരിക്കാം.
\end{malayalam}}
\flushright{\begin{Arabic}
\quranayah[10][21]
\end{Arabic}}
\flushleft{\begin{malayalam}
ജനങ്ങള്‍ക്ക് ദുരിതാനുഭവങ്ങള്‍ക്കു ശേഷം നാം അനുഗ്രഹം അനുഭവിക്കാനവസരം നല്‍കിയാല്‍ ഉടനെ അവര്‍ നമ്മുടെ പ്രമാണങ്ങളുടെ കാര്യത്തില്‍ കുതന്ത്രം കാണിക്കുന്നു. പറയുക: അല്ലാഹു അതിവേഗം തന്ത്രം പ്രയോഗിക്കുന്നവനാണ്. നമ്മുടെ ദൂതന്മാര്‍ നിങ്ങള്‍ കാണിച്ചുകൊണ്ടിരിക്കുന്ന കുതന്ത്രങ്ങളെല്ലാം രേഖപ്പെടുത്തിവെക്കും; തീര്‍ച്ച.
\end{malayalam}}
\flushright{\begin{Arabic}
\quranayah[10][22]
\end{Arabic}}
\flushleft{\begin{malayalam}
കരയിലും കടലിലും നിങ്ങള്‍ക്ക് സഞ്ചരിക്കാനവസരമൊരുക്കിയത് ആ അല്ലാഹുതന്നെയാണ്. അങ്ങനെ നിങ്ങള്‍ കപ്പലിലായിരിക്കെ, സുഖകരമായ കാറ്റുവീശി. യാത്രക്കാരെയും കൊണ്ട് കപ്പല്‍ നീങ്ങിത്തുടങ്ങി. അവരതില്‍ സന്തുഷ്ടരായി. പെട്ടെന്നൊരു കൊടുങ്കാറ്റടിച്ചു. എല്ലാ ഭാഗത്തുനിന്നും തിരമാലകള്‍ അവരുടെ നേരെ ആഞ്ഞു വീശി. കൊടുങ്കാറ്റ് തങ്ങളെ വലയം ചെയ്തതായി അവര്‍ക്കുതോന്നി. അപ്പോള്‍ തങ്ങളുടെ വണക്കം അല്ലാഹുവിന് മാത്രം സമര്‍പ്പിച്ചുകൊണ്ട് അവര്‍ അവനോട് പ്രാര്‍ഥിച്ചു: "ഞങ്ങളെ നീ ഇതില്‍നിന്ന് രക്ഷപ്പെടുത്തിയാല്‍ ഉറപ്പായും ഞങ്ങള്‍ നന്ദിയുള്ളവരായിരിക്കും.”
\end{malayalam}}
\flushright{\begin{Arabic}
\quranayah[10][23]
\end{Arabic}}
\flushleft{\begin{malayalam}
അങ്ങനെ അല്ലാഹു അവരെ രക്ഷപ്പെടുത്തി. അപ്പോള്‍ അവരതാ അന്യായമായി ഭൂമിയില്‍ അതിക്രമം പ്രവര്‍ത്തിക്കുന്നു. മനുഷ്യരേ, നിങ്ങളുടെ അതിക്രമം നിങ്ങള്‍ക്കെതിരെ തന്നെയാണ്. നിങ്ങള്‍ക്കത് നല്‍കുക ഐഹികജീവിതത്തിലെ സുഖാസ്വാദനമാണ്. പിന്നെ നമ്മുടെ അടുത്തേക്കാണ് നിങ്ങളുടെ മടക്കം. അപ്പോള്‍ നിങ്ങള്‍ ചെയ്തുകൊണ്ടിരുന്നതിനെപ്പറ്റി നിങ്ങളെ നാം വിവരമറിയിക്കും.
\end{malayalam}}
\flushright{\begin{Arabic}
\quranayah[10][24]
\end{Arabic}}
\flushleft{\begin{malayalam}
ഐഹികജീവിതത്തിന്റെ ഉപമ യിതാ: മാനത്തുനിന്നു നാം മഴ പെയ്യിച്ചു. അതുവഴി ഭൂമിയില്‍ സസ്യങ്ങള്‍ ഇടകലര്‍ന്നു വളര്‍ന്നു. മനുഷ്യര്‍ക്കും കന്നുകാലികള്‍ക്കും തിന്നാന്‍. അങ്ങനെ ഭൂമി അതിന്റെ ചമയങ്ങളണിയുകയും ചേതോഹരമാവുകയും ചെയ്തു. അവയൊക്കെ അനുഭവിക്കാന്‍ തങ്ങള്‍ കഴിവുറ്റവരായിരിക്കുന്നുവെന്ന് അതിന്റെ ഉടമകള്‍ കരുതി. അപ്പോള്‍ രാത്രിയോ പകലോ നമ്മുടെ കല്‍പന വന്നെത്തുന്നു. അങ്ങനെ നാമതിനെ നിശ്ശേഷം നശിപ്പിക്കുന്നു; ഇന്നലെ അവിടെ ഒന്നുംതന്നെ ഉണ്ടായിരുന്നിട്ടില്ലാത്തവിധം. ചിന്തിച്ചു മനസ്സിലാക്കുന്ന ജനതക്കുവേണ്ടിയാണ് നാം ഇവ്വിധം തെളിവുകള്‍ വിശദീകരിക്കുന്നത്.
\end{malayalam}}
\flushright{\begin{Arabic}
\quranayah[10][25]
\end{Arabic}}
\flushleft{\begin{malayalam}
അല്ലാഹു സമാധാനത്തിന്റെ ഭവനത്തിലേക്ക് ക്ഷണിക്കുന്നു. അവനിച്ഛിക്കുന്നവരെ അവന്‍ നേര്‍വഴിയില്‍ നയിക്കുന്നു.
\end{malayalam}}
\flushright{\begin{Arabic}
\quranayah[10][26]
\end{Arabic}}
\flushleft{\begin{malayalam}
നന്മ ചെയ്തവര്‍ക്ക് നല്ല പ്രതിഫലമുണ്ട്. അവര്‍ക്കതില്‍ വര്‍ധനവുമുണ്ട്. അവരുടെ മുഖത്തെ ഇരുളോ നിന്ദ്യതയോ ബാധിക്കുകയില്ല. അവരാണ് സ്വര്‍ഗാവകാശികള്‍. അവരവിടെ സ്ഥിരവാസികളായിരിക്കും.
\end{malayalam}}
\flushright{\begin{Arabic}
\quranayah[10][27]
\end{Arabic}}
\flushleft{\begin{malayalam}
എന്നാല്‍ തിന്മകള്‍ ചെയ്തുകൂട്ടിനയവരോ, തിന്മക്കുള്ള പ്രതിഫലം അതിനു തുല്യം തന്നെയായിരിക്കും. അപമാനം അവരെ ബാധിക്കും. അല്ലാഹുവില്‍ നിന്ന് അവരെ രക്ഷിക്കാന്‍ ആരുമുണ്ടാവില്ല. അവരുടെ മുഖങ്ങള്‍ ഇരുള്‍മുറ്റിയ രാവിന്റെ കഷ്ണംകൊണ്ട് പൊതിഞ്ഞ പോലിരിക്കും. അവരാണ് നരകാവകാശികള്‍. അവരതില്‍ സ്ഥിരവാസികളായിരിക്കും.
\end{malayalam}}
\flushright{\begin{Arabic}
\quranayah[10][28]
\end{Arabic}}
\flushleft{\begin{malayalam}
നാം അവരെയെല്ലാം ഒരുമിച്ചുകൂട്ടുന്ന ദിനം. അന്ന് നാം ബഹുദൈവവിശ്വാസികളോടു പറയും: "നിങ്ങളും നിങ്ങള്‍ പങ്കാളികളാക്കിവെച്ചവരും അവിടെത്തന്നെ നില്‍ക്കുക.” പിന്നീട് നാം അവരെ പരസ്പരം വേര്‍പ്പെടുത്തും. അവര്‍ പങ്കുചേര്‍ത്തിരുന്നവര്‍ പറയും: "നിങ്ങള്‍ ഞങ്ങളെ ആരാധിച്ചിരുന്നില്ല.
\end{malayalam}}
\flushright{\begin{Arabic}
\quranayah[10][29]
\end{Arabic}}
\flushleft{\begin{malayalam}
"അതിനാല്‍ ഞങ്ങള്‍ക്കും നിങ്ങള്‍ക്കുമിടയില്‍ സാക്ഷിയായി അല്ലാഹു മതി. നിങ്ങളുടെ ആരാധനയെപ്പറ്റി ഞങ്ങള്‍ തീര്‍ത്തും അശ്രദ്ധരായിരുന്നു.”
\end{malayalam}}
\flushright{\begin{Arabic}
\quranayah[10][30]
\end{Arabic}}
\flushleft{\begin{malayalam}
അന്ന് അവിടെവെച്ചു ഓരോ മനുഷ്യനും താന്‍ നേരത്തെ ചെയ്തുകൂട്ടിയതിന്റെ രുചി അനുഭവിച്ചറിയും. എല്ലാവരും തങ്ങളുടെ യഥാര്‍ഥ രക്ഷകനായ അല്ലാഹുവിങ്കലേക്ക് മടക്കപ്പെടും. അവര്‍ കെട്ടിയുണ്ടാക്കിയ കള്ളത്തരങ്ങളൊക്കെയും അവരില്‍നിന്ന് തെന്നിമാറിപ്പോകും.
\end{malayalam}}
\flushright{\begin{Arabic}
\quranayah[10][31]
\end{Arabic}}
\flushleft{\begin{malayalam}
ചോദിക്കുക: ആകാശഭൂമികളില്‍ നിന്ന് നിങ്ങള്‍ക്ക് അന്നം നല്‍കുന്നത് ആരാണ്? കേള്‍വിയും കാഴ്ചയും ആരുടെ അധീനതയിലാണ്? ജീവനില്ലാത്തതില്‍ നിന്ന് ജീവനുള്ളതിനെയും ജീവനുള്ളതില്‍നിന്ന് ജീവനില്ലാത്തതിനെയും പുറത്തെടുക്കുന്നതാരാണ്? കാര്യങ്ങളൊക്കെ നിയന്ത്രിക്കുന്നതാരാണ്? അവര്‍ പറയും: "അല്ലാഹു.” അവരോടു ചോദിക്കുക: "എന്നിട്ടും നിങ്ങള്‍ സൂക്ഷ്മതയുള്ളവരാവുന്നില്ലേ?”
\end{malayalam}}
\flushright{\begin{Arabic}
\quranayah[10][32]
\end{Arabic}}
\flushleft{\begin{malayalam}
അവനാണ് നിങ്ങളുടെ യഥാര്‍ഥ സംരക്ഷകനായ അല്ലാഹു. അതിനാല്‍ യഥാര്‍ഥത്തിനപ്പുറം ദുര്‍മാര്‍ഗമല്ലാതെ മറ്റെന്താണുള്ളത്? എന്നിട്ടും നിങ്ങള്‍ എങ്ങോട്ടാണ് വഴിതെറ്റിപ്പോകുന്നത്?
\end{malayalam}}
\flushright{\begin{Arabic}
\quranayah[10][33]
\end{Arabic}}
\flushleft{\begin{malayalam}
അങ്ങനെ ധിക്കാരികളുടെ കാര്യത്തില്‍ “അവര്‍ വിശ്വസിക്കുകയില്ല” എന്ന നിന്റെ നാഥന്റെ വചനം സത്യമായി പുലര്‍ന്നിരിക്കുന്നു.
\end{malayalam}}
\flushright{\begin{Arabic}
\quranayah[10][34]
\end{Arabic}}
\flushleft{\begin{malayalam}
ചോദിക്കുക: നിങ്ങള്‍ ദൈവത്തില്‍ പങ്കാളികളാക്കിയവരില്‍ സൃഷ്ടി ആരംഭിക്കുകയും പിന്നെ അതാവര്‍ത്തിക്കുകയും ചെയ്യുന്ന ആരെങ്കിലുമുണ്ടോ? പറയുക: അല്ലാഹു മാത്രമാണ് സൃഷ്ടികര്‍മമാരംഭിക്കുന്നതും പിന്നീട് അതാവര്‍ത്തിക്കുന്നതും. എന്നിട്ടും നിങ്ങളെങ്ങോട്ടാണ് വഴിതെറ്റിപ്പോകുന്നത്?
\end{malayalam}}
\flushright{\begin{Arabic}
\quranayah[10][35]
\end{Arabic}}
\flushleft{\begin{malayalam}
ചോദിക്കുക: നിങ്ങള്‍ പങ്കാളികളാക്കിയ ദൈവങ്ങളില്‍ സത്യത്തിലേക്ക് നയിക്കുന്ന വല്ലവരുമുണ്ടോ? പറയുക: അല്ലാഹുവാണ് സത്യത്തിലേക്ക് നയിക്കുന്നവന്‍. അപ്പോള്‍ സത്യത്തിലേക്ക് നയിക്കുന്നവനോ, അതല്ല മാര്‍ഗദര്‍ശനം നല്‍കപ്പെട്ടാലല്ലാതെ സ്വയം നേര്‍വഴി കാണാന്‍ കഴിയാത്തവനോ പിന്‍പറ്റാന്‍ ഏറ്റം അര്‍ഹന്‍? നിങ്ങള്‍ക്കെന്തു പറ്റി? എങ്ങനെയൊക്കെയാണ് നിങ്ങള്‍ തീരുമാനമെടുക്കുന്നത്!
\end{malayalam}}
\flushright{\begin{Arabic}
\quranayah[10][36]
\end{Arabic}}
\flushleft{\begin{malayalam}
അവരിലേറെപ്പേരും ഊഹാപോഹത്തെ മാത്രമാണ് ആശ്രയിക്കുന്നത്. സത്യം മനസ്സിലാക്കാന്‍ ഊഹം ഒട്ടും ഉപകരിക്കുകയില്ല. അല്ലാഹു അവര്‍ ചെയ്യുന്നതൊക്കെയും നന്നായറിയുന്നവനാണ്; തീര്‍ച്ച.
\end{malayalam}}
\flushright{\begin{Arabic}
\quranayah[10][37]
\end{Arabic}}
\flushleft{\begin{malayalam}
അല്ലാഹുവല്ലാത്തവര്‍ക്ക് പടച്ചുണ്ടാക്കാനാവുന്നതല്ല ഈ ഖുര്‍ആന്‍. മുമ്പുള്ള വേദപുസ്തകങ്ങളെ സത്യപ്പെടുത്തുന്നതും ദൈവിക വചനങ്ങളുടെ വിശദീകരണവുമാണിത്. ഇതിലൊട്ടും സംശയിക്കേണ്ടതില്ല. ഇതു ലോകനാഥനില്‍ നിന്നുള്ളതുതന്നെയാണ്.
\end{malayalam}}
\flushright{\begin{Arabic}
\quranayah[10][38]
\end{Arabic}}
\flushleft{\begin{malayalam}
അതല്ല; ഇതു പ്രവാചകന്‍ കെട്ടിച്ചമച്ചതാണെന്നാണോ അവര്‍ പറയുന്നത്? പറയുക: "അങ്ങനെയെങ്കില്‍ അതിനു സമാനമായ ഒരധ്യായം നിങ്ങള്‍ കൊണ്ടുവരിക. അല്ലാഹുവെക്കൂടാതെ നിങ്ങള്‍ക്ക് കിട്ടാവുന്നവരെയൊക്കെ സഹായത്തിനു വിളിച്ചുകൊള്ളുക; നിങ്ങള്‍ സത്യവാന്മാരെങ്കില്‍!”
\end{malayalam}}
\flushright{\begin{Arabic}
\quranayah[10][39]
\end{Arabic}}
\flushleft{\begin{malayalam}
എന്നാല്‍ കാര്യമിതാണ്. തങ്ങള്‍ക്ക് അറിയാന്‍ കഴിയാത്തവയെയൊക്കെ അവര്‍ തള്ളിപ്പറഞ്ഞു. ഏതൊന്നിന്റെ അനുഭവസാക്ഷ്യം തങ്ങള്‍ക്കു വന്നെത്തിയിട്ടില്ലയോ അതിനെയും അവര്‍ തള്ളിപ്പറഞ്ഞു. ഇതുപോലെയാണ് അവരുടെ മുമ്പുള്ളവരും കള്ളമാക്കിത്തള്ളിയത്. നോക്കൂ: ആ അക്രമികളുടെ അന്ത്യം എവ്വിധമായിരുന്നുവെന്ന്.
\end{malayalam}}
\flushright{\begin{Arabic}
\quranayah[10][40]
\end{Arabic}}
\flushleft{\begin{malayalam}
ഈ ഖുര്‍ആനില്‍ വിശ്വസിക്കുന്ന വര്‍ അവരിലുണ്ട്. വിശ്വസിക്കാത്തവരുമുണ്ട്. കുഴപ്പക്കാരെക്കുറിച്ച് നന്നായറിയുന്നവനാണ് നിന്റെ നാഥന്‍.
\end{malayalam}}
\flushright{\begin{Arabic}
\quranayah[10][41]
\end{Arabic}}
\flushleft{\begin{malayalam}
അവര്‍ നിന്നെ നിഷേധിച്ചു തള്ളുകയാണെങ്കില്‍ പറയുക: "എനിക്ക് എന്റെ കര്‍മം. നിങ്ങള്‍ക്ക് നിങ്ങളുടെ കര്‍മം. ഞാന്‍ പ്രവര്‍ത്തിക്കുന്നതിന്റെ ബാധ്യത നിങ്ങള്‍ക്കില്ല. നിങ്ങള്‍ പ്രവര്‍ത്തിക്കുന്നതിന്റെ ബാധ്യത എനിക്കുമില്ല.”
\end{malayalam}}
\flushright{\begin{Arabic}
\quranayah[10][42]
\end{Arabic}}
\flushleft{\begin{malayalam}
അവരില്‍ നിന്റെ വാക്കുകള്‍ ശ്രദ്ധിച്ചു കേട്ടുകൊണ്ടിരിക്കുന്നവരുമുണ്ട്. എന്നാല്‍ ബധിരന്മാരെ കേള്‍പ്പിക്കാന്‍ നിനക്കാവുമോ? അവര്‍ തീരെ ചിന്തിക്കാത്തവരുമാണെങ്കില്‍.
\end{malayalam}}
\flushright{\begin{Arabic}
\quranayah[10][43]
\end{Arabic}}
\flushleft{\begin{malayalam}
അവരില്‍ നിന്നെ ഉറ്റുനോക്കുന്നചിലരുമുണ്ട്. എന്നാല്‍ കണ്ണുപൊട്ടന്മാരെ നേര്‍വഴി കാണിക്കാന്‍ നിനക്കാവുമോ? അവര്‍ ഒന്നും കാണാന്‍ ഒരുക്കവുമല്ലെങ്കില്‍!
\end{malayalam}}
\flushright{\begin{Arabic}
\quranayah[10][44]
\end{Arabic}}
\flushleft{\begin{malayalam}
നിശ്ചയമായും അല്ലാഹു മനുഷ്യരോട് അക്രമം കാണിക്കുന്നില്ല. മറിച്ച് ജനം തങ്ങളോടുതന്നെ അനീതി കാണിക്കുകയാണ്.
\end{malayalam}}
\flushright{\begin{Arabic}
\quranayah[10][45]
\end{Arabic}}
\flushleft{\begin{malayalam}
അല്ലാഹു അവരെ ഒരുമിച്ചുകൂട്ടുന്ന നാളിലെ സ്ഥിതിയോര്‍ക്കുക: അന്നവര്‍ക്കു തോന്നും; തങ്ങള്‍ പരസ്പരം തിരിച്ചറിയാന്‍ മാത്രം പകലില്‍ ഇത്തിരിനേരമേ ഭൂമിയില്‍ താമസിച്ചിട്ടുള്ളൂവെന്ന്. അല്ലാഹുവുമായി കണ്ടുമുട്ടുമെന്ന കാര്യം കള്ളമാക്കി തള്ളിയവര്‍ കൊടിയ നഷ്ടത്തിലകപ്പെട്ടിരിക്കുന്നു. അവര്‍ നേര്‍വഴിയിലായിരുന്നില്ല.
\end{malayalam}}
\flushright{\begin{Arabic}
\quranayah[10][46]
\end{Arabic}}
\flushleft{\begin{malayalam}
നാം അവര്‍ക്കു താക്കീത് നല്‍കിക്കൊണ്ടിരിക്കുന്ന വിപത്തുകളില്‍ ചിലത് നിനക്ക് ഈ ജീവിതത്തില്‍തന്നെ നാം കാണിച്ചുതന്നേക്കാം. അല്ലെങ്കില്‍ അതിനുമുമ്പേ നിന്നെ മരിപ്പിച്ചേക്കാം. ഏതായാലും അവരുടെ മടക്കം നമ്മിലേക്കാണ്. പിന്നെ, അവര്‍ ചെയ്തുകൊണ്ടിരിക്കുന്നതിനൊക്കെ അല്ലാഹു സാക്ഷിയായിരിക്കും.
\end{malayalam}}
\flushright{\begin{Arabic}
\quranayah[10][47]
\end{Arabic}}
\flushleft{\begin{malayalam}
ഓരോ സമുദായത്തിനും ഓരോ ദൂതനുണ്ട്. അങ്ങനെ ഓരോ സമുദായത്തിലേക്കും അവരുടെ ദൂതന്‍ വന്നെത്തിയപ്പോള്‍ അവര്‍ക്കിടയില്‍ നീതിപൂര്‍വകമായ വിധിത്തീര്‍പ്പുണ്ടാക്കി. അവര്‍ അല്‍പവും അനീതിക്കിരയായതുമില്ല.
\end{malayalam}}
\flushright{\begin{Arabic}
\quranayah[10][48]
\end{Arabic}}
\flushleft{\begin{malayalam}
അവര്‍ ചോദിക്കുന്നുവല്ലോ: "ഈ വാഗ്ദാനം എപ്പോഴാണ് പുലരുക; നിങ്ങള്‍ സത്യവാന്മാരെങ്കില്‍.”
\end{malayalam}}
\flushright{\begin{Arabic}
\quranayah[10][49]
\end{Arabic}}
\flushleft{\begin{malayalam}
പറയുക: "എനിക്കു തന്നെ ഗുണമോ ദോഷമോ വരുത്താന്‍ എനിക്കാവില്ല. അല്ലാഹു ഇച്ഛിച്ചാലല്ലാതെ.” ഓരോ ജനതക്കും ഒരു നിശ്ചിത അവധിയുണ്ട്. അവരുടെ അവധി വന്നെത്തിയാല്‍ പിന്നെ ഇത്തിരിനേരം പോലും വൈകിക്കാനവര്‍ക്കാവില്ല. നേരത്തെയാക്കാനും കഴിയില്ല.
\end{malayalam}}
\flushright{\begin{Arabic}
\quranayah[10][50]
\end{Arabic}}
\flushleft{\begin{malayalam}
ചോദിക്കുക: അല്ലാഹുവിന്റെ ശിക്ഷ രാവോ പകലോ നിങ്ങള്‍ക്കു വന്നെത്തിയാല്‍ എന്തുണ്ടാവുമെന്ന് നിങ്ങള്‍ ആലോചിച്ചുനോക്കിയിട്ടുണ്ടോ? അതില്‍നിന്ന് ഏത് ശിക്ഷക്കായിരിക്കും കുറ്റവാളികള്‍ തിടുക്കം കൂട്ടുക?
\end{malayalam}}
\flushright{\begin{Arabic}
\quranayah[10][51]
\end{Arabic}}
\flushleft{\begin{malayalam}
ആ ശിക്ഷ സംഭവിക്കുമ്പോഴേ നിങ്ങള്‍ വിശ്വസിക്കൂ എന്നാണോ? അപ്പോള്‍ അവരോടു ചോദിക്കും: "ഇപ്പോഴാണോ വിശ്വസിക്കുന്നത്? നിങ്ങള്‍ ഈ ശിക്ഷക്ക് തിടുക്കം കൂട്ടുകയായിരുന്നുവല്ലോ.”
\end{malayalam}}
\flushright{\begin{Arabic}
\quranayah[10][52]
\end{Arabic}}
\flushleft{\begin{malayalam}
പിന്നെ ആ അക്രമികളോട് പറയും: നിങ്ങള്‍ ശാശ്വത ശിക്ഷ അനുഭവിച്ചുകൊള്ളുക. നിങ്ങള്‍ സമ്പാദിച്ചുകൊണ്ടിരുന്നതിനല്ലാതെ നിങ്ങള്‍ക്ക് പ്രതിഫലം കിട്ടുമോ?
\end{malayalam}}
\flushright{\begin{Arabic}
\quranayah[10][53]
\end{Arabic}}
\flushleft{\begin{malayalam}
അവര്‍ നിന്നോട് അന്വേഷിക്കുന്നു: ഇതു സത്യമാണോ എന്ന്. പറയുക: "അതെ. എന്റെ നാഥന്‍ സാക്ഷി. തീര്‍ച്ചയായും ഇതു സത്യംതന്നെ. നിങ്ങള്‍ക്കിതിനെ പരാജയപ്പെടുത്താനാവില്ല.”
\end{malayalam}}
\flushright{\begin{Arabic}
\quranayah[10][54]
\end{Arabic}}
\flushleft{\begin{malayalam}
അക്രമം പ്രവര്‍ത്തിച്ചവരുടെ വശം ഭൂമിയിലുള്ളതെല്ലാം ഉണ്ടെന്ന് കരുതുക; എങ്കില്‍, ശിക്ഷ നേരില്‍ കാണുമ്പോള്‍ അതൊക്കെയും പിഴയായി നല്‍കാന്‍ അവര്‍ തയ്യാറാകും. അവര്‍ ഖേദം ഉള്ളിലൊളിപ്പിച്ചുവെക്കുകയും ചെയ്യും. അവര്‍ക്കിടയില്‍ നീതിപൂര്‍വം വിധി തീര്‍പ്പുണ്ടാവും. അവരോടൊട്ടും അനീതിയുണ്ടാവുകയില്ല.
\end{malayalam}}
\flushright{\begin{Arabic}
\quranayah[10][55]
\end{Arabic}}
\flushleft{\begin{malayalam}
അറിയുക: തീര്‍ച്ചയായും ആകാശഭൂമികളിലുള്ളതൊക്കെയും അല്ലാഹുവിന്റേതാണ്. അറിയുക: അല്ലാഹുവിന്റെ വാഗ്ദാനം സത്യമാണ്. എങ്കിലും ഏറെപ്പേരും കാര്യം മനസ്സിലാക്കുന്നില്ല.
\end{malayalam}}
\flushright{\begin{Arabic}
\quranayah[10][56]
\end{Arabic}}
\flushleft{\begin{malayalam}
അവനാണ് ജീവിപ്പിക്കുകയും മരിപ്പിക്കുകയും ചെയ്യുന്നത്. നിങ്ങളെല്ലാം മടങ്ങിച്ചെല്ലുന്നത് അവങ്കലേക്കാണ്.
\end{malayalam}}
\flushright{\begin{Arabic}
\quranayah[10][57]
\end{Arabic}}
\flushleft{\begin{malayalam}
മനുഷ്യരേ, നിങ്ങള്‍ക്ക് നിങ്ങളുടെ നാഥനില്‍ നിന്നുള്ള സദുപദേശം വന്നെത്തിയിരിക്കുന്നു. അത് നിങ്ങളുടെ മനസ്സുകളുടെ രോഗത്തിനുള്ള ശമനമാണ്. ഒപ്പം സത്യവിശ്വാസികള്‍ക്ക് നേര്‍വഴി കാട്ടുന്നതും മഹത്തായ അനുഗ്രഹവും.
\end{malayalam}}
\flushright{\begin{Arabic}
\quranayah[10][58]
\end{Arabic}}
\flushleft{\begin{malayalam}
പറയൂ: അല്ലാഹുവിന്റെ അനുഗ്രഹവും കാരുണ്യവും കൊണ്ടാണ് അവനങ്ങനെ ചെയ്തത്. അതിനാല്‍ അവര്‍ സന്തോഷിച്ചുകൊള്ളട്ടെ. അതാണവര്‍ നേടിക്കൊണ്ടിരിക്കുന്നതിനെക്കാളെല്ലാം ഉത്തമം.
\end{malayalam}}
\flushright{\begin{Arabic}
\quranayah[10][59]
\end{Arabic}}
\flushleft{\begin{malayalam}
പറയുക: അല്ലാഹു നിങ്ങള്‍ക്കിറക്കിത്തന്ന ആഹാരത്തെപ്പറ്റി നിങ്ങളാലോചിച്ചുനോക്കിയിട്ടുണ്ടോ? എന്നിട്ട് നിങ്ങള്‍ അവയില്‍ ചിലതിനെ നിഷിദ്ധമാക്കി. മറ്റു ചിലതിനെ അനുവദനീയവുമാക്കി. ചോദിക്കുക: ഇങ്ങനെ ചെയ്യാന്‍ അല്ലാഹു നിങ്ങള്‍ക്ക് അനുവാദം തന്നിട്ടുണ്ടോ? അതോ, നിങ്ങള്‍ അല്ലാഹുവിന്റെ പേരില്‍ കള്ളം കെട്ടിച്ചമക്കുകയാണോ?
\end{malayalam}}
\flushright{\begin{Arabic}
\quranayah[10][60]
\end{Arabic}}
\flushleft{\begin{malayalam}
അല്ലാഹുവിന്റെ പേരില്‍ കള്ളം കെട്ടിയുണ്ടാക്കുന്നവരുടെ ഉയിര്‍ത്തെഴുന്നേല്‍പുനാളിലെ മനോഗതി എന്തായിരിക്കുമെന്നാണ് ഭാവിക്കുന്നത്? അല്ലാഹു ജനത്തോട് അത്യുദാരനാണ്; എന്നാല്‍ അവരിലേറെപ്പേരും നന്ദികാണിക്കുന്നില്ല.
\end{malayalam}}
\flushright{\begin{Arabic}
\quranayah[10][61]
\end{Arabic}}
\flushleft{\begin{malayalam}
നീ ഏതുകാര്യത്തിലാവട്ടെ; ഖുര്‍ആനില്‍നിന്ന് എന്തെങ്കിലും ഓതിക്കേള്‍പ്പിക്കുകയാകട്ടെ; നിങ്ങള്‍ ഏത് പ്രവൃത്തി ചെയ്യുകയാകട്ടെ; നിങ്ങളതില്‍ ഏര്‍പ്പെടുമ്പോഴെല്ലാം നാം നിങ്ങളുടെമേല്‍ സാക്ഷിയായി ഉണ്ടാവാതിരിക്കില്ല. ആകാശഭൂമികളിലെ അണുപോലുള്ളതോ അതിനെക്കാള്‍ ചെറുതോ വലുതോ ആയ ഒന്നും നിന്റെ നാഥന്റെ ശ്രദ്ധയില്‍പെടാതെയില്ല. വ്യക്തമായ പ്രമാണത്തില്‍ രേഖപ്പെടുത്താത്ത ഒന്നും തന്നെയില്ല.
\end{malayalam}}
\flushright{\begin{Arabic}
\quranayah[10][62]
\end{Arabic}}
\flushleft{\begin{malayalam}
അറിയുക: അല്ലാഹുവിന്റെ ഉറ്റവരാരും പേടിക്കേണ്ടതില്ല. ദുഃഖിക്കേണ്ടതുമില്ല.
\end{malayalam}}
\flushright{\begin{Arabic}
\quranayah[10][63]
\end{Arabic}}
\flushleft{\begin{malayalam}
സത്യവിശ്വാസം സ്വീകരിച്ചവരും സൂക്ഷ്മത പാലിക്കുന്നവരുമാണവര്‍.
\end{malayalam}}
\flushright{\begin{Arabic}
\quranayah[10][64]
\end{Arabic}}
\flushleft{\begin{malayalam}
ഇഹലോകത്തും പരലോകത്തും അവര്‍ക്ക് ശുഭവാര്‍ത്തയുണ്ട്. അല്ലാഹുവിന്റെ വചനങ്ങള്‍ തിരുത്താനാവാത്തതാണ്. ആ ശുഭവാര്‍ത്ത തന്നെയാണ് അതിമഹത്തായ വിജയം.
\end{malayalam}}
\flushright{\begin{Arabic}
\quranayah[10][65]
\end{Arabic}}
\flushleft{\begin{malayalam}
അവരുടെ വാക്കുകളൊന്നും നിന്നെ ദുഃഖിപ്പിക്കേണ്ടതില്ല. തീര്‍ച്ചയായും പ്രതാപമൊക്കെയും അല്ലാഹുവിനാണ്. അവന്‍ എല്ലാം കേള്‍ക്കുന്നവനും അറിയുന്നവനുമാണ്.
\end{malayalam}}
\flushright{\begin{Arabic}
\quranayah[10][66]
\end{Arabic}}
\flushleft{\begin{malayalam}
അറിയുക: ആകാശഭൂമികളിലുള്ളവരൊക്കെയും അല്ലാഹുവിനുള്ളതാണ്. അല്ലാഹുവിനു പുറമെ മറ്റു പങ്കാളികളോട് പ്രാര്‍ഥിക്കുന്നവര്‍ എന്തിനെയാണ് പിന്തുടരുന്നത്? ഊഹത്തെ മാത്രമാണ് അവര്‍ പിന്തുടരുന്നത്. കള്ളം കെട്ടിയുണ്ടാക്കുക മാത്രമാണ് അവര്‍ ചെയ്യുന്നത്.
\end{malayalam}}
\flushright{\begin{Arabic}
\quranayah[10][67]
\end{Arabic}}
\flushleft{\begin{malayalam}
നിങ്ങള്‍ക്ക് ശാന്തി നേടാനായി രാവിനെ നിശ്ചയിച്ചു തന്നതും പകലിനെ പ്രകാശപൂരിതമാക്കിയതും അവനാണ്. കേട്ടറിയുന്ന ജനത്തിന് അതില്‍ ധാരാളം ദൃഷ്ടാന്തങ്ങളുണ്ട്.
\end{malayalam}}
\flushright{\begin{Arabic}
\quranayah[10][68]
\end{Arabic}}
\flushleft{\begin{malayalam}
അല്ലാഹു പുത്രനെ സ്വീകരിച്ചിരിക്കുന്നുവെന്ന് അവര്‍ പറയുന്നു. എന്നാല്‍ അവന്‍ പരിശുദ്ധനാണ്. സ്വയം പൂര്‍ണനും. ആകാശഭൂമികളിലുള്ളതൊക്കെയും അവന്റേതാണ്. ഈ വാദത്തിന് നിങ്ങളുടെ പക്കല്‍ ഒരു പ്രമാണവുമില്ല. അല്ലാഹുവിന്റെ പേരില്‍ നിങ്ങള്‍ക്കറിയാത്ത കാര്യങ്ങള്‍ പറഞ്ഞുണ്ടാക്കുകയാണോ നിങ്ങള്‍?
\end{malayalam}}
\flushright{\begin{Arabic}
\quranayah[10][69]
\end{Arabic}}
\flushleft{\begin{malayalam}
പറയുക: നിശ്ചയമായും അല്ലാഹുവിന്റെ പേരില്‍ കള്ളം കെട്ടിയുണ്ടാക്കുന്നവര്‍ വിജയിക്കുകയില്ല.
\end{malayalam}}
\flushright{\begin{Arabic}
\quranayah[10][70]
\end{Arabic}}
\flushleft{\begin{malayalam}
അവര്‍ക്കുണ്ടാവുക ഈ ലോകത്തെ ഇത്തിരി സുഖാനുഭവം മാത്രമാണ്. ഒടുക്കം അവരുടെ മടക്കം നമ്മിലേക്കാണ്. പിന്നീട് നാമവരെ കഠിന ശിക്ഷ അനുഭവിപ്പിക്കും. അവര്‍ സത്യനിഷേധികളായതിനാലാണിത്.
\end{malayalam}}
\flushright{\begin{Arabic}
\quranayah[10][71]
\end{Arabic}}
\flushleft{\begin{malayalam}
നീ അവരെ നൂഹിന്റെ കഥ ഓതിക്കേള്‍പ്പിക്കുക. അദ്ദേഹം തന്റെ ജനതയോട് പറഞ്ഞ സന്ദര്‍ഭം: "എന്റെ ജനമേ, എന്റെ സാന്നിധ്യവും ദൈവിക വചനങ്ങളെ സംബന്ധിച്ച എന്റെ ഉണര്‍ത്തലും നിങ്ങള്‍ക്ക് ഏറെ ദുസ്സഹമായിത്തോന്നുന്നുവെങ്കില്‍ ഞാനിതാ അല്ലാഹുവില്‍ ഭരമേല്‍പിക്കുന്നു. നിങ്ങളുടെ കാര്യം നിങ്ങളും നിങ്ങള്‍ സങ്കല്‍പിച്ചുണ്ടാക്കിയ പങ്കാളികളുംകൂടി തീരുമാനിച്ചുകൊള്ളുക. പിന്നെ നിങ്ങളുടെ തീരുമാനം നിങ്ങള്‍ക്കൊരിക്കലും അവ്യക്തമാകരുത്. എന്നിട്ട് നിങ്ങളത് എനിക്കെതിരെ നടപ്പാക്കിക്കൊള്ളുക. എനിക്കൊട്ടും അവധി തരേണ്ടതില്ല.
\end{malayalam}}
\flushright{\begin{Arabic}
\quranayah[10][72]
\end{Arabic}}
\flushleft{\begin{malayalam}
"അഥവാ, നിങ്ങള്‍ പിന്തിരിയുന്നുവെങ്കില്‍ എനിക്കെന്ത്; ഞാന്‍ നിങ്ങളോട് ഒരു പ്രതിഫലവും ആവശ്യപ്പെട്ടിട്ടില്ലല്ലോ. എനിക്കുള്ള പ്രതിഫലം അല്ലാഹുവിങ്കല്‍ മാത്രമാണ്. ഞാന്‍ മുസ്ലിം ആയിരിക്കാന്‍ കല്‍പിക്കപ്പെട്ടിരിക്കുന്നു.”
\end{malayalam}}
\flushright{\begin{Arabic}
\quranayah[10][73]
\end{Arabic}}
\flushleft{\begin{malayalam}
എന്നിട്ടും അവരദ്ദേഹത്തെ തള്ളിപ്പറഞ്ഞു. അപ്പോള്‍ അദ്ദേഹത്തെയും അദ്ദേഹത്തോടൊപ്പമുള്ളവരെയും നാം കപ്പലില്‍ രക്ഷപ്പെടുത്തി. നാമവരെ ഭൂമിയിലെ പ്രതിനിധികളാക്കി. നമ്മുടെ പ്രമാണങ്ങളെ കള്ളമാക്കി തള്ളിയവരെ മുക്കിക്കൊന്നു. അപ്പോള്‍ നോക്കൂ: താക്കീത് നല്‍കപ്പെട്ട അക്കൂട്ടരുടെ ഒടുക്കം എവ്വിധമായിരുന്നുവെന്ന്.
\end{malayalam}}
\flushright{\begin{Arabic}
\quranayah[10][74]
\end{Arabic}}
\flushleft{\begin{malayalam}
പിന്നീട് അദ്ദേഹത്തിനുശേഷം നിരവധി ദൂതന്മാരെ നാം തങ്ങളുടെ ജനതകളിലേക്കയച്ചു. അങ്ങനെ അവരുടെ അടുത്ത് വ്യക്തമായ പ്രമാണങ്ങളുമായി അവര്‍ വന്നെത്തി. എങ്കിലും നേരത്തെ കള്ളമാക്കി തള്ളിയതില്‍ വിശ്വസിക്കാനവര്‍ തയ്യാറായിരുന്നില്ല. അവ്വിധം അതിക്രമികളുടെ മനസ്സുകള്‍ക്ക് നാം മുദ്ര വെക്കുന്നു.
\end{malayalam}}
\flushright{\begin{Arabic}
\quranayah[10][75]
\end{Arabic}}
\flushleft{\begin{malayalam}
പിന്നീട് അവര്‍ക്കുശേഷം നാം മൂസയെയും ഹാറൂനെയും നമ്മുടെ പ്രമാണങ്ങളുമായി ഫറവോന്റെയും അവന്റെ പ്രമാണിമാരുടെയും അടുത്തേക്കയച്ചു. അപ്പോള്‍ അവര്‍ അഹങ്കരിക്കുകയാണുണ്ടായത്. അവര്‍ കുറ്റവാളികളായ ജനമായിരുന്നു.
\end{malayalam}}
\flushright{\begin{Arabic}
\quranayah[10][76]
\end{Arabic}}
\flushleft{\begin{malayalam}
അങ്ങനെ നമ്മുടെ ഭാഗത്തുനിന്നുള്ള സത്യം അവര്‍ക്ക് വന്നെത്തി. അപ്പോള്‍ അവര്‍ പറഞ്ഞു: "ഇത് വ്യക്തമായ മായാജാലംതന്നെ; തീര്‍ച്ച.”
\end{malayalam}}
\flushright{\begin{Arabic}
\quranayah[10][77]
\end{Arabic}}
\flushleft{\begin{malayalam}
മൂസാ പറഞ്ഞു: "സത്യം നിങ്ങള്‍ക്ക് വന്നെത്തിയപ്പോള്‍ അതേപ്പറ്റിയാണോ നിങ്ങളിങ്ങനെ പറയുന്നത്? ഇത് മായാജാലമാണെന്നോ? മായാജാലക്കാര്‍ ഒരിക്കലും വിജയിക്കുകയില്ല.”
\end{malayalam}}
\flushright{\begin{Arabic}
\quranayah[10][78]
\end{Arabic}}
\flushleft{\begin{malayalam}
അവര്‍ പറഞ്ഞു: "ഞങ്ങളുടെ പൂര്‍വ പിതാക്കള്‍ ഏതൊരു മാര്‍ഗം മുറുകെപ്പിടിക്കുന്നതായി ഞങ്ങള്‍ കണ്ടുവോ അതില്‍നിന്ന് ഞങ്ങളെ തെറ്റിച്ചുകളയാനാണോ നീ ഞങ്ങളുടെയടുത്ത് വന്നത്? ഭൂമിയില്‍ നിങ്ങളിരുവരുടെയും മേധാവിത്വം സ്ഥാപിക്കാനും? എന്നാല്‍ ഞങ്ങളൊരിക്കലും നിങ്ങളിരുവരിലും വിശ്വസിക്കുന്നവരാവുകയില്ല.”
\end{malayalam}}
\flushright{\begin{Arabic}
\quranayah[10][79]
\end{Arabic}}
\flushleft{\begin{malayalam}
ഫറവോന്‍ പറഞ്ഞു: "അറിവുള്ള എല്ലാ ജാലവിദ്യക്കാരെയും നിങ്ങള്‍ എന്റെ അടുത്തെത്തിക്കുക.”
\end{malayalam}}
\flushright{\begin{Arabic}
\quranayah[10][80]
\end{Arabic}}
\flushleft{\begin{malayalam}
അങ്ങനെ ജാലവിദ്യക്കാര്‍ വന്നപ്പോള്‍ മൂസ അവരോടു പറഞ്ഞു: നിങ്ങള്‍ക്ക് ഇടാനുള്ളത് ഇടുക.”
\end{malayalam}}
\flushright{\begin{Arabic}
\quranayah[10][81]
\end{Arabic}}
\flushleft{\begin{malayalam}
അവര്‍ ഇട്ടപ്പോള്‍ മൂസ പറഞ്ഞു: നിങ്ങള്‍ ഈ കാണിച്ചതൊക്കെ വെറും ജാലവിദ്യയാണ്. അല്ലാഹു അതിനെ തോല്‍പിക്കും; തീര്‍ച്ച. സംശയമില്ല: നാശകാരികളുടെ ചെയ്തികളെ അല്ലാഹു ഫലവത്താക്കുകയില്ല.
\end{malayalam}}
\flushright{\begin{Arabic}
\quranayah[10][82]
\end{Arabic}}
\flushleft{\begin{malayalam}
അല്ലാഹു തന്റെ വചനങ്ങളിലൂടെ സത്യത്തെ സ്ഥാപിക്കുന്നു. കുറ്റവാളികള്‍ക്ക് എത്രതന്നെ അതനിഷ്ടകരമാണെങ്കിലും!
\end{malayalam}}
\flushright{\begin{Arabic}
\quranayah[10][83]
\end{Arabic}}
\flushleft{\begin{malayalam}
മൂസായില്‍ അദ്ദേഹത്തിന്റെ ജനതയിലെ ഏതാനും ചെറുപ്പക്കാരല്ലാതെ ആരും വിശ്വസിച്ചില്ല. ഫറവോനും അവരുടെ പ്രമാണിമാരും തങ്ങളെ പീഡിപ്പിച്ചേക്കുമോയെന്ന പേടിയിലായിരുന്നു അവര്‍. ഫറവോന്‍ ഭൂമിയില്‍ ഔദ്ധത്യം നടിക്കുന്നവനായിരുന്നു; അതോടൊപ്പം പരിധിവിട്ടവനും.
\end{malayalam}}
\flushright{\begin{Arabic}
\quranayah[10][84]
\end{Arabic}}
\flushleft{\begin{malayalam}
മൂസാ പറഞ്ഞു: "എന്റെ ജനമേ, നിങ്ങള്‍ അല്ലാഹുവില്‍ വിശ്വസിക്കുന്നവരാണെങ്കില്‍ അവനില്‍ ഭരമേല്‍പിക്കുക. നിങ്ങള്‍ മുസ്ലിംകളെങ്കില്‍!”
\end{malayalam}}
\flushright{\begin{Arabic}
\quranayah[10][85]
\end{Arabic}}
\flushleft{\begin{malayalam}
അപ്പോഴവര്‍ പറഞ്ഞു: "ഞങ്ങള്‍ അല്ലാഹുവില്‍ ഭരമേല്‍പിച്ചിരിക്കുന്നു. ഞങ്ങളുടെ നാഥാ, ഞങ്ങളെ നീ അക്രമികളായ ഈ ജനത്തിന്റെ പീഡനങ്ങള്‍ക്കിരയാക്കരുതേ.
\end{malayalam}}
\flushright{\begin{Arabic}
\quranayah[10][86]
\end{Arabic}}
\flushleft{\begin{malayalam}
"നിന്റെ കാരുണ്യത്താല്‍ ഞങ്ങളെ നീ സത്യനിഷേധികളായ ഈ ജനതയില്‍നിന്ന് രക്ഷിക്കേണമേ.”
\end{malayalam}}
\flushright{\begin{Arabic}
\quranayah[10][87]
\end{Arabic}}
\flushleft{\begin{malayalam}
മൂസാക്കും അദ്ദേഹത്തിന്റെ സഹോദരന്നും നാം ബോധനം നല്‍കി: നിങ്ങളിരുവരും നിങ്ങളുടെ ജനതക്കായി ഈജിപ്തില്‍ ഏതാനും വീടുകള്‍ തയ്യാറാക്കുക. നിങ്ങളുടെ വീടുകളെ നിങ്ങള്‍ ഖിബ്ലകളാക്കുക. നമസ്കാരം നിഷ്ഠയോടെ നിര്‍വഹിക്കുക. സത്യവിശ്വാസികളെ ശുഭവാര്‍ത്ത അറിയിക്കുകയും ചെയ്യുക.
\end{malayalam}}
\flushright{\begin{Arabic}
\quranayah[10][88]
\end{Arabic}}
\flushleft{\begin{malayalam}
മൂസാ പറഞ്ഞു: "ഞങ്ങളുടെ നാഥാ! ഫറവോന്നും അവന്റെ പ്രമാണിമാര്‍ക്കും നീ ഐഹിക ജീവിതത്തില്‍ പ്രൌഢിയും പണവും നല്‍കിയിരിക്കുന്നു. ഞങ്ങളുടെ നാഥാ! ജനങ്ങളെ നിന്റെ മാര്‍ഗത്തില്‍ നിന്ന് തെറ്റിക്കാനാണ് അവരതുപയോഗിക്കുന്നത്. ഞങ്ങളുടെ നാഥാ! അവരുടെ സമ്പത്ത് നീ നശിപ്പിച്ചുകളയേണമേ. നോവേറിയ ശിക്ഷ കാണുംവരെ വിശ്വസിക്കാനാവാത്തവിധം അവരുടെ മനസ്സുകളെ കടുത്തതാക്കേണമേ.”
\end{malayalam}}
\flushright{\begin{Arabic}
\quranayah[10][89]
\end{Arabic}}
\flushleft{\begin{malayalam}
അല്ലാഹു പറഞ്ഞു: "നിങ്ങളിരുവരുടെയും പ്രാര്‍ഥന സ്വീകരിച്ചിരിക്കുന്നു. അതിനാല്‍ സ്ഥൈര്യത്തോടെയിരിക്കുക. വിവരമില്ലാത്തവരുടെ പാത പിന്തുടരരുത്.”
\end{malayalam}}
\flushright{\begin{Arabic}
\quranayah[10][90]
\end{Arabic}}
\flushleft{\begin{malayalam}
ഇസ്രയേല്‍ മക്കളെ നാം കടല്‍ കടത്തി. അപ്പോള്‍ ഫറവോനും അവന്റെ സൈന്യവും അക്രമിക്കാനും ദ്രോഹിക്കാനുമായി അവരെ പിന്തുടര്‍ന്നു. അങ്ങനെ മുങ്ങിച്ചാകുമെന്നായപ്പോള്‍ ഫറവോന്‍ പറഞ്ഞു: "ഇസ്രയേല്‍ മക്കള്‍ വിശ്വസിച്ചവനല്ലാതെ ദൈവമില്ലെന്ന് ഞാനിതാ വിശ്വസിച്ചിരിക്കുന്നു. ഞാന്‍ മുസ്ലിംകളില്‍ പെട്ടവനാകുന്നു.”
\end{malayalam}}
\flushright{\begin{Arabic}
\quranayah[10][91]
\end{Arabic}}
\flushleft{\begin{malayalam}
അല്ലാഹു പറഞ്ഞു: ഇപ്പോഴോ? ഇതുവരെയും നീ ധിക്കരിച്ചു. നീ കുഴപ്പക്കാരില്‍ പെട്ടവനായിരുന്നു.
\end{malayalam}}
\flushright{\begin{Arabic}
\quranayah[10][92]
\end{Arabic}}
\flushleft{\begin{malayalam}
നിന്റെ ശേഷക്കാര്‍ക്ക് ഒരു പാഠമായിരിക്കാന്‍ വേണ്ടി ഇന്നു നിന്റെ ജഡത്തെ നാം രക്ഷപ്പെടുത്തും. സംശയമില്ല; മനുഷ്യരിലേറെപ്പേരും നമ്മുടെ ദൃഷ്ടാന്തങ്ങളെപ്പറ്റി അശ്രദ്ധരാണ്.
\end{malayalam}}
\flushright{\begin{Arabic}
\quranayah[10][93]
\end{Arabic}}
\flushleft{\begin{malayalam}
തീര്‍ച്ചയായും ഇസ്രയേല്‍ മക്കള്‍ക്ക് നാം മെച്ചപ്പെട്ട താവളമൊരുക്കിക്കൊടുത്തു. വിശിഷ്ടമായ വിഭവങ്ങള്‍ ആഹാരമായി നല്‍കി. വേദവിജ്ഞാനം വന്നുകിട്ടുംവരെ അവര്‍ ഭിന്നിച്ചിരുന്നില്ല. ഉറപ്പായും ഉയിര്‍ത്തെഴുന്നേല്‍പ് നാളില്‍ അവര്‍ ഭിന്നിച്ചുകൊണ്ടിരുന്ന കാര്യത്തില്‍ നിന്റെ നാഥന്‍ അവര്‍ക്കിടയില്‍ തീര്‍പ്പ് കല്‍പിക്കും.
\end{malayalam}}
\flushright{\begin{Arabic}
\quranayah[10][94]
\end{Arabic}}
\flushleft{\begin{malayalam}
നിനക്കു നാം അവതരിപ്പിച്ചുതന്നതിനെ സംബന്ധിച്ച് നിനക്കെന്തെങ്കിലും സംശയമുണ്ടെങ്കില്‍ നിനക്കുമുമ്പെ വേദപാരായണം നടത്തിവരുന്നവരോട് ചോദിച്ചു നോക്കൂ. തീര്‍ച്ചയായും നിന്റെ നാഥനില്‍ നിന്ന് സത്യമാണ് നിനക്ക് വന്നെത്തിയിരിക്കുന്നത്. അതിനാല്‍ നീ സംശയാലുക്കളില്‍ പെട്ടുപോകരുത്.
\end{malayalam}}
\flushright{\begin{Arabic}
\quranayah[10][95]
\end{Arabic}}
\flushleft{\begin{malayalam}
അല്ലാഹുവിന്റെ പ്രമാണങ്ങളെ കള്ളമാക്കി തള്ളിയവരിലും നീ അകപ്പെടരുത്. അങ്ങനെ സംഭവിച്ചാല്‍ നീ പരാജിതരുടെ കൂട്ടത്തില്‍ പെട്ടുപോകും.
\end{malayalam}}
\flushright{\begin{Arabic}
\quranayah[10][96]
\end{Arabic}}
\flushleft{\begin{malayalam}
ഏതൊരു ജനത്തിന്റെമേല്‍ നിന്റെ നാഥന്റെ വചനം സത്യമായി പുലര്‍ന്നുവോ അവര്‍ വിശ്വസിക്കുകയില്ല.
\end{malayalam}}
\flushright{\begin{Arabic}
\quranayah[10][97]
\end{Arabic}}
\flushleft{\begin{malayalam}
എല്ലാ തെളിവും അവര്‍ക്കു വന്നുകിട്ടിയാലും നോവേറിയ ശിക്ഷ നേരില്‍ കാണുംവരെ അവര്‍ വിശ്വസിക്കുകയില്ല.
\end{malayalam}}
\flushright{\begin{Arabic}
\quranayah[10][98]
\end{Arabic}}
\flushleft{\begin{malayalam}
ഏതെങ്കിലും നാട് ശിക്ഷ കണ്ട് ഭയന്ന് സത്യവിശ്വാസം സ്വീകരിക്കുകയും അങ്ങനെ അതവര്‍ക്ക് ഉപകരിക്കുകയും ചെയ്ത അനുഭവമുണ്ടോ? യൂനുസിന്റെ ജനതയുടേതൊഴികെ. അവര്‍ വിശ്വസിച്ചപ്പോള്‍ ഐഹിക ജീവിതത്തിലെ നിന്ദ്യമായ ശിക്ഷ നാമവരില്‍ നിന്ന് എടുത്തുമാറ്റി. ഒരു നിശ്ചിതകാലംവരെ നാമവര്‍ക്ക് സുഖജീവിതം നല്‍കുകയും ചെയ്തു.
\end{malayalam}}
\flushright{\begin{Arabic}
\quranayah[10][99]
\end{Arabic}}
\flushleft{\begin{malayalam}
നിന്റെ നാഥന്‍ ഇച്ഛിച്ചിരുന്നെങ്കില്‍ ഭൂമിയിലുള്ളവരൊക്കെയും സത്യവിശ്വാസം സ്വീകരിക്കുമായിരുന്നു. എന്നിരിക്കെ ജനങ്ങള്‍ വിശ്വാസികളാകാന്‍ നീ അവരെ നിര്‍ബന്ധിക്കുകയോ?
\end{malayalam}}
\flushright{\begin{Arabic}
\quranayah[10][100]
\end{Arabic}}
\flushleft{\begin{malayalam}
ദൈവഹിതമനുസരിച്ചല്ലാതെ ആര്‍ക്കും സത്യവിശ്വാസം സ്വീകരിക്കാനാവില്ല. ആലോചിച്ച് മനസ്സിലാക്കാത്തവര്‍ക്ക് അല്ലാഹു നിന്ദ്യത വരുത്തിവെക്കും.
\end{malayalam}}
\flushright{\begin{Arabic}
\quranayah[10][101]
\end{Arabic}}
\flushleft{\begin{malayalam}
പറയുക: ആകാശഭൂമികളിലുള്ളതെന്തൊക്കെയാണെന്ന് നോക്കൂ. എന്നാല്‍ വിശ്വസിക്കാത്ത ജനത്തിന് തെളിവുകളും താക്കീതുകളും കൊണ്ടെന്തു ഫലം?
\end{malayalam}}
\flushright{\begin{Arabic}
\quranayah[10][102]
\end{Arabic}}
\flushleft{\begin{malayalam}
അതിനാല്‍ ഇവര്‍ക്കെന്താണ് പ്രതീക്ഷിക്കാനുള്ളത്? ഇവരുടെ മുമ്പെ കഴിഞ്ഞുപോയവര്‍ അനുഭവിച്ച ദുരന്തനാളുകള്‍ പോലുള്ളതല്ലാതെ? പറയൂ:"നിങ്ങള്‍ കാത്തിരിക്കുക. നിങ്ങളോടൊപ്പം ഞാനും കാത്തിരിക്കുന്നുണ്ട്.”
\end{malayalam}}
\flushright{\begin{Arabic}
\quranayah[10][103]
\end{Arabic}}
\flushleft{\begin{malayalam}
പിന്നീട് നാം നമ്മുടെ ദൂതന്മാരെയും വിശ്വാസികളെയും രക്ഷപ്പെടുത്തും. ഇവ്വിധം വിശ്വാസികളെ രക്ഷപ്പെടുത്തുക എന്നത് നമ്മുടെ ബാധ്യതയാണ്.
\end{malayalam}}
\flushright{\begin{Arabic}
\quranayah[10][104]
\end{Arabic}}
\flushleft{\begin{malayalam}
പറയൂ: "ജനങ്ങളേ, എന്റെ മാര്‍ഗത്തെ സംബന്ധിച്ച് ഇനിയും നിങ്ങള്‍ സംശയത്തിലാണെങ്കില്‍ അറിയുക: അല്ലാഹുവിനു പുറമെ നിങ്ങള്‍ പൂജിക്കുന്നവയെ ഞാന്‍ പൂജിക്കുന്നില്ല. എന്നാല്‍, നിങ്ങളെ മരിപ്പിക്കുന്ന അല്ലാഹുവിനെ ഞാന്‍ ആരാധിക്കുന്നു. സത്യവിശ്വാസികളിലുള്‍പ്പെടാനാണ് എന്നോട് കല്‍പിച്ചിരിക്കുന്നത്.”
\end{malayalam}}
\flushright{\begin{Arabic}
\quranayah[10][105]
\end{Arabic}}
\flushleft{\begin{malayalam}
“നിന്റെ മുഖം ചാഞ്ഞുപോകാതെ ഈ മാര്‍ഗത്തിന് നേരെ ഉറപ്പിച്ചുനിര്‍ത്തണ”മെന്നും “നീ ഒരിക്കലും ബഹുദൈവ വിശ്വാസികളില്‍ പെട്ടുപോകരുതെ”ന്നും എന്നോടു കല്‍പിച്ചിരിക്കുന്നു.
\end{malayalam}}
\flushright{\begin{Arabic}
\quranayah[10][106]
\end{Arabic}}
\flushleft{\begin{malayalam}
അല്ലാഹുവിനു പുറമെ നിനക്ക് ഉപകാരമോ ഉപദ്രവമോ ചെയ്യാനാവാത്ത ഒന്നിനോടും നീ പ്രാര്‍ഥിക്കരുത്. അങ്ങനെ ചെയ്താല്‍ നീ അതിക്രമികളില്‍പ്പെടും; തീര്‍ച്ച.
\end{malayalam}}
\flushright{\begin{Arabic}
\quranayah[10][107]
\end{Arabic}}
\flushleft{\begin{malayalam}
അല്ലാഹു നിനക്കു വല്ല വിപത്തും വരുത്തുന്നുവെങ്കില്‍ അതു തട്ടിമാറ്റാന്‍ അവനല്ലാതാരുമില്ല. അവന്‍ നിനക്കു വല്ല ഗുണവും ഉദ്ദേശിക്കുന്നുവെങ്കില്‍ അവന്റെ അനുഗ്രഹം തട്ടിമാറ്റാനും ആര്‍ക്കുമാവില്ല. തന്റെ ദാസന്മാരില്‍ താനിച്ഛിക്കുന്നവര്‍ക്ക് അവനത് നല്‍കുന്നു. അവന്‍ ഏറെ പൊറുക്കുന്നവനും ദയാപരനുമാണ്.
\end{malayalam}}
\flushright{\begin{Arabic}
\quranayah[10][108]
\end{Arabic}}
\flushleft{\begin{malayalam}
പറയുക: മനുഷ്യരേ, നിങ്ങള്‍ക്ക് നിങ്ങളുടെ നാഥനില്‍ നിന്നുള്ള സത്യം ഇതാ വന്നെത്തിയിരിക്കുന്നു. അതിനാല്‍ ആര്‍ സന്മാര്‍ഗം സ്വീകരിക്കുന്നുവോ അതിന്റെ നേട്ടം അവനുതന്നെയാണ്. ആരെങ്കിലും ദുര്‍മാര്‍ഗത്തിലാവുകയാണെങ്കില്‍ ആ വഴികേടിന്റെ ദുരന്തവും അവനുതന്നെ. ഇക്കാര്യത്തില്‍ എനിക്കു നിങ്ങളുടെമേല്‍ ഒരുവിധ ഉത്തരവാദിത്വവുമില്ല.
\end{malayalam}}
\flushright{\begin{Arabic}
\quranayah[10][109]
\end{Arabic}}
\flushleft{\begin{malayalam}
നിനക്ക് ബോധനമായി ലഭിച്ചദിവ്യസന്ദേശം പിന്‍പറ്റുക. അല്ലാഹു തീര്‍പ്പുകല്‍പിക്കുംവരെ ക്ഷമ പാലിക്കുക. തീര്‍പ്പുകല്‍പിക്കുന്നവരില്‍ അത്യുത്തമന്‍ അവനാണല്ലോ.
\end{malayalam}}
\chapter{\textmalayalam{ഹൂദ്}}
\begin{Arabic}
\Huge{\centerline{\basmalah}}\end{Arabic}
\flushright{\begin{Arabic}
\quranayah[11][1]
\end{Arabic}}
\flushleft{\begin{malayalam}
അലിഫ്- ലാം -റാഅ.് ഇത് വേദപുസ്തകമാകുന്നു. ഇതിലെ സൂക്തങ്ങള്‍ സുഭദ്രമാക്കിയിരിക്കുന്നു. പിന്നെ അവയെ വിശദീകരിക്കുകയും ചെയ്തിരിക്കുന്നു. യുക്തിമാനും സൂക്ഷ്മജ്ഞനുമായ അല്ലാഹുവില്‍ നിന്നുളളതാണിത്.
\end{malayalam}}
\flushright{\begin{Arabic}
\quranayah[11][2]
\end{Arabic}}
\flushleft{\begin{malayalam}
അതിനാല്‍ നിങ്ങള്‍ അല്ലാഹുവിനു മാത്രം വഴിപ്പെടുക. ഞാന്‍ നിങ്ങളിലേക്ക് അവനയച്ച മുന്നറിയിപ്പുകാരനും ശുഭവാര്‍ത്ത അറിയിക്കുന്നവനുമാണ്.
\end{malayalam}}
\flushright{\begin{Arabic}
\quranayah[11][3]
\end{Arabic}}
\flushleft{\begin{malayalam}
നിങ്ങള്‍ നിങ്ങളുടെ നാഥനായ അല്ലാഹുവോട് പാപമോചനം തേടുക. അവങ്കലേക്ക് പശ്ചാത്തപിച്ചു മടങ്ങുക. എങ്കില്‍ ഒരു നിശ്ചിതകാലം വരെ അവന്‍ നിങ്ങള്‍ക്ക് ഉത്തമമായ ജീവിത വിഭവം നല്‍കും. ശ്രേഷ്ഠത പുലര്‍ത്തുന്നവര്‍ക്ക് തങ്ങളുടെ ശ്രേഷ്ഠതക്കൊത്ത പ്രതിഫലമുണ്ട്. അഥവാ, നിങ്ങള്‍ പിന്തിരിയുന്നുവെങ്കില്‍ ഭീകരമായ ഒരു നാളിലെ ശിക്ഷ നിങ്ങള്‍ക്കുണ്ടാകുമെന്ന് ഞാന്‍ ഭയപ്പെടുന്നു.
\end{malayalam}}
\flushright{\begin{Arabic}
\quranayah[11][4]
\end{Arabic}}
\flushleft{\begin{malayalam}
നിങ്ങളുടെ മടക്കം അല്ലാഹുവിങ്കലേക്കാണ്. അവന്‍ എല്ലാറ്റിനും കഴിവുറ്റവനാണ്.
\end{malayalam}}
\flushright{\begin{Arabic}
\quranayah[11][5]
\end{Arabic}}
\flushleft{\begin{malayalam}
അറിയുക: അവനില്‍ നിന്ന് മറച്ചുപിടിക്കാനായി അവര്‍ തങ്ങളുടെ നെഞ്ചുകള്‍ ചുരുട്ടിക്കൂട്ടുന്നു. എന്നാല്‍ ഓര്‍ക്കുക: അവര്‍ തങ്ങളുടെ വസ്ത്രങ്ങള്‍ കൊണ്ടു മൂടുമ്പോഴും അവര്‍ രഹസ്യമാക്കുന്നതും പരസ്യമാക്കുന്നതുമെല്ലാം അവനറിയുന്നു. നെഞ്ചകത്തുള്ളതൊക്കെ അറിയുന്നവനാണവന്‍; തീര്‍ച്ച.
\end{malayalam}}
\flushright{\begin{Arabic}
\quranayah[11][6]
\end{Arabic}}
\flushleft{\begin{malayalam}
ഭൂമിയിലെ എല്ലാ ജീവജാലങ്ങളുടെയും ആഹാരച്ചുമതല അല്ലാഹുവിനാണ്. അവ എവിടെക്കഴിയുന്നുവെന്നും അവസാനം എവിടെക്കാണെത്തിച്ചേരുന്നതെന്നും അവനറിയുന്നു. എല്ലാം സുവ്യക്തമായ ഒരു ഗ്രന്ഥത്തിലുണ്ട്.
\end{malayalam}}
\flushright{\begin{Arabic}
\quranayah[11][7]
\end{Arabic}}
\flushleft{\begin{malayalam}
ആറു നാളുകളിലായി ആകാശഭൂമികളെ സൃഷ്ടിച്ചത് അവനാണ്. അവന്റെ സിംഹാസനം ജലപ്പരപ്പിലായിരുന്നു. നിങ്ങളില്‍ സല്‍ക്കര്‍മം ചെയ്യുന്നത് ആരെന്ന് പരീക്ഷിക്കാനാണത്. മരണശേഷം നിങ്ങളെ ഉയിര്‍ത്തെഴുന്നേല്‍പിക്കുമെന്ന് നീ പറഞ്ഞാല്‍ അവരിലെ അവിശ്വസിച്ചവര്‍ പറയും: ഇത് സ്പഷ്ടമായ മായാജാലം മാത്രമാണ്.
\end{malayalam}}
\flushright{\begin{Arabic}
\quranayah[11][8]
\end{Arabic}}
\flushleft{\begin{malayalam}
ഒരു നിശ്ചിത അവധിവരെ നാം അവരുടെ ശിക്ഷ വൈകിച്ചാല്‍ അവരിങ്ങനെ പറയും: "അതിനെ തടഞ്ഞുനിര്‍ത്തിയതെന്താണ്?” അറിയുക: അത് വന്നെത്തുന്ന ദിവസം ഒരു നിലക്കും അവരില്‍ നിന്നത് തട്ടി മാറ്റപ്പെടുന്നതല്ല. ഏതൊന്നിനെ അവര്‍ പരിഹസിച്ചുകൊണ്ടിരിക്കുന്നുവോ, അതവരില്‍ വന്നു പതിക്കുക തന്നെ ചെയ്യും.
\end{malayalam}}
\flushright{\begin{Arabic}
\quranayah[11][9]
\end{Arabic}}
\flushleft{\begin{malayalam}
നാം മനുഷ്യനെ നമ്മില്‍ നിന്നുള്ള അനുഗ്രഹം ആസ്വദിപ്പിക്കുകയും പിന്നെ അത് എടുത്ത് മാറ്റുകയും ചെയ്താല്‍ അവന്‍ വല്ലാതെ നിരാശനും നന്ദികെട്ടവനുമായിത്തീരുന്നു.
\end{malayalam}}
\flushright{\begin{Arabic}
\quranayah[11][10]
\end{Arabic}}
\flushleft{\begin{malayalam}
അഥവാ, നാമവനെ ദുരന്തം അനുഭവിപ്പിച്ച ശേഷം അനുഗ്രഹം ആസ്വദിപ്പിച്ചാല്‍ അവന്‍ പറയും: "എന്റെ ദുരന്തങ്ങളൊക്കെ പോയിമറഞ്ഞിരിക്കുന്നു.” അങ്ങനെ അവന്‍ ആഹ്ളാദഭരിതനും അഹങ്കാരിയുമായിത്തീരുന്നു.
\end{malayalam}}
\flushright{\begin{Arabic}
\quranayah[11][11]
\end{Arabic}}
\flushleft{\begin{malayalam}
സഹനമവലംബിക്കുകയും സല്‍ക്കര്‍മങ്ങള്‍ പ്രവര്‍ത്തിക്കുകയും ചെയ്തവരൊഴികെ. അവര്‍ക്കാണ് പാപമോചനം. മഹത്തായ പ്രതിഫലവും.
\end{malayalam}}
\flushright{\begin{Arabic}
\quranayah[11][12]
\end{Arabic}}
\flushleft{\begin{malayalam}
“ഇയാള്‍ക്ക് ഒരു നിധി ഇറക്കിക്കൊടുക്കാത്തതെന്ത്? അല്ലെങ്കില്‍ ഇയാളോടൊപ്പം ഒരു മലക്ക് വരാത്തതെന്ത്?” എന്നൊക്കെ അവര്‍ പറയുന്നതുകാരണം നിനക്കു ബോധനമായി ലഭിച്ച സന്ദേശങ്ങളില്‍ ചിലത് നീ വിട്ടുകളഞ്ഞേക്കാം. അല്ലെങ്കിലതുവഴി നിനക്ക് മനോവിഷമമുണ്ടായേക്കാം. എന്നാല്‍ നീ ഒരു മുന്നറിയിപ്പുകാരന്‍ മാത്രമാണ്. അല്ലാഹുവോ സര്‍വ സംഗതികള്‍ക്കും ചുമതലപ്പെട്ടവനും.
\end{malayalam}}
\flushright{\begin{Arabic}
\quranayah[11][13]
\end{Arabic}}
\flushleft{\begin{malayalam}
അതല്ല; ഇത് ഇദ്ദേഹം കെട്ടിച്ചമച്ചതാണെന്നാണോ അവര്‍ വാദിക്കുന്നത്? പറയുക: എങ്കില്‍ ഇതുപോലുള്ള പത്ത് അധ്യായം നിങ്ങള്‍ കെട്ടിച്ചമച്ച് കൊണ്ടുവരിക. അതിനായി അല്ലാഹുവിനു പുറമെ നിങ്ങള്‍ക്ക് കിട്ടാവുന്നവരെയൊക്കെ വിളിച്ചുകൊള്ളുക. നിങ്ങള്‍ സത്യവാന്മാരെങ്കില്‍.
\end{malayalam}}
\flushright{\begin{Arabic}
\quranayah[11][14]
\end{Arabic}}
\flushleft{\begin{malayalam}
അഥവാ അവര്‍ നിങ്ങളുടെ വെല്ലുവിളിക്ക് ഉത്തരം നല്‍കുന്നില്ലെങ്കില്‍ അറിയുക: അല്ലാഹു അറിഞ്ഞുകൊണ്ടുതന്നെയാണ് ഇത് അവതരിപ്പിച്ചിരിക്കുന്നത്. അവനല്ലാതെ ദൈവമില്ല. ഇനിയെങ്കിലും നിങ്ങള്‍ മുസ്ലിംകളാവുന്നുണ്ടോ?
\end{malayalam}}
\flushright{\begin{Arabic}
\quranayah[11][15]
\end{Arabic}}
\flushleft{\begin{malayalam}
ആരെങ്കിലും ഐഹികജീവിതവും അതിന്റെ ആര്‍ഭാടങ്ങളും മാത്രമാണ് ആഗ്രഹിക്കുന്നതെങ്കില്‍ നാമവരുടെ കര്‍മഫലങ്ങളൊക്കെ ഇവിടെ വെച്ച് തന്നെ പൂര്‍ണമായി നല്‍കും. അതിലവര്‍ക്കൊട്ടും കുറവു വരുത്തില്ല.
\end{malayalam}}
\flushright{\begin{Arabic}
\quranayah[11][16]
\end{Arabic}}
\flushleft{\begin{malayalam}
എന്നാല്‍ പരലോകത്ത് നരകത്തീ മാത്രമാണവര്‍ക്കുണ്ടാവുക. അവരിവിടെ ചെയ്തുകൂട്ടിയതൊക്കെയും നിഷ്ഫലമായിരിക്കുന്നു. അവര്‍ പ്രവര്‍ത്തിച്ചുകൊണ്ടിരുന്നതെല്ലാം പാഴ്വേലകളായി പരിണമിച്ചിരിക്കുന്നു.
\end{malayalam}}
\flushright{\begin{Arabic}
\quranayah[11][17]
\end{Arabic}}
\flushleft{\begin{malayalam}
ഒരാള്‍ക്ക് തന്റെ നാഥനില്‍ നിന്നുള്ള വ്യക്തമായ തെളിവു ലഭിച്ചു. അതേ തുടര്‍ന്ന് തന്റെ നാഥനില്‍ നിന്നുള്ള ഒരു സാക്ഷി അയാള്‍ക്ക് പിന്തുണ നല്‍കുകയും ചെയ്തു. അതിനു മുമ്പേ മാതൃകയും ദിവ്യാനുഗ്രഹവുമായി മൂസാക്ക് ഗ്രന്ഥം വന്നെത്തിയിട്ടുമുണ്ട്. ഇയാളും ഭൌതിക പൂജകരെപ്പോലെ അത് തള്ളിക്കളയുമോ? അവരതില്‍ വിശ്വസിക്കുക തന്നെ ചെയ്യും. എന്നാല്‍ വിവിധ വിഭാഗങ്ങളില്‍ ആരെങ്കിലും അതിനെ നിഷേധിക്കുകയാണെങ്കില്‍ അവരുടെ വാഗ്ദത്ത സ്ഥലം നരകത്തീയായിരിക്കും. അതിനാല്‍ നീ ഇതില്‍ സംശയിക്കരുത്. തീര്‍ച്ചയായും ഇത് നിന്റെ നാഥനില്‍ നിന്നുള്ള സത്യമാണ്. എന്നിട്ടും ജനങ്ങളിലേറെപേരും വിശ്വസിക്കുന്നില്ല.
\end{malayalam}}
\flushright{\begin{Arabic}
\quranayah[11][18]
\end{Arabic}}
\flushleft{\begin{malayalam}
അല്ലാഹുവിന്റെ പേരില്‍ കള്ളം കെട്ടിച്ചമച്ചുണ്ടാക്കിയവനെക്കാള്‍ കൊടിയ അക്രമി ആരുണ്ട്? അവര്‍ തങ്ങളുടെ നാഥന്റെ സന്നിധിയില്‍ കൊണ്ടുവരപ്പെടും. അപ്പോള്‍ സാക്ഷികള്‍ പറയും: "ഇവരാണ് തങ്ങളുടെ നാഥന്റെ പേരില്‍ കള്ളം കെട്ടിച്ചമച്ചവര്‍.” അറിയുക: അക്രമികളുടെ മേല്‍ അല്ലാഹുവിന്റെ കൊടിയ ശാപമുണ്ട്.
\end{malayalam}}
\flushright{\begin{Arabic}
\quranayah[11][19]
\end{Arabic}}
\flushleft{\begin{malayalam}
അല്ലാഹുവിന്റെ മാര്‍ഗത്തില്‍ നിന്ന് ജനത്തെ തടയുന്നവരും അവന്റെ വഴി വികലമാക്കാനാഗ്രഹിക്കുന്നവരുമാണവര്‍. പരലോകത്തെ തള്ളിപ്പറയുന്നവരും.
\end{malayalam}}
\flushright{\begin{Arabic}
\quranayah[11][20]
\end{Arabic}}
\flushleft{\begin{malayalam}
അവര്‍ ഈ ഭൂമിയില്‍ അല്ലാഹുവെ തോല്‍പിക്കാന്‍ മാത്രം വളര്‍ന്നിട്ടില്ല. അല്ലാഹുവല്ലാതെ അവര്‍ക്ക് മറ്റു രക്ഷകരില്ല. അവര്‍ക്ക് ഇരട്ടി ശിക്ഷയുണ്ട്. അവര്‍ക്കൊന്നും കേള്‍ക്കാന്‍ കഴിഞ്ഞിരുന്നില്ല. അവരൊന്നും കണ്ടറിയുന്നവരുമായിരുന്നില്ല.
\end{malayalam}}
\flushright{\begin{Arabic}
\quranayah[11][21]
\end{Arabic}}
\flushleft{\begin{malayalam}
തങ്ങള്‍ക്കു തന്നെ നഷ്ടം വരുത്തിവെച്ചവരാണവര്‍. അവര്‍ കെട്ടിച്ചമച്ചിരുന്നതെല്ലാം അവരില്‍ നിന്ന് ഏറെ അകന്നുപോയിരിക്കുന്നു.
\end{malayalam}}
\flushright{\begin{Arabic}
\quranayah[11][22]
\end{Arabic}}
\flushleft{\begin{malayalam}
സംശയമില്ല; അവര്‍ തന്നെയാണ് പരലോകത്ത് പരാജയപ്പെട്ടവര്‍.
\end{malayalam}}
\flushright{\begin{Arabic}
\quranayah[11][23]
\end{Arabic}}
\flushleft{\begin{malayalam}
എന്നാല്‍ സത്യവിശ്വാസം സ്വീകരിക്കുകയും സല്‍ക്കര്‍മങ്ങള്‍ പ്രവര്‍ത്തിക്കുകയും തങ്ങളുടെ നാഥങ്കലേക്ക് വിനയത്തോടെ തിരിച്ചുചെല്ലുകയും ചെയ്തവരാണ് സ്വര്‍ഗാവകാശികള്‍. അവരവിടെ സ്ഥിരവാസികളായിരിക്കും.
\end{malayalam}}
\flushright{\begin{Arabic}
\quranayah[11][24]
\end{Arabic}}
\flushleft{\begin{malayalam}
ഈ രണ്ടു വിഭാഗത്തിന്റെ ഉപമ ഇവ്വിധമത്രെ: ഒരുവന്‍ അന്ധനും ബധിരനും; അപരന്‍ കാഴ്ചയും കേള്‍വിയുമുള്ളവനും. ഈ ഉപമയിലെ ഇരുവരും ഒരുപോലെയാണോ? നിങ്ങള്‍ ആലോചിച്ചു നോക്കുന്നില്ലേ?
\end{malayalam}}
\flushright{\begin{Arabic}
\quranayah[11][25]
\end{Arabic}}
\flushleft{\begin{malayalam}
നൂഹിനെ നാം തന്റെ ജനതയിലേക്കയച്ചു. അദ്ദേഹം പറഞ്ഞു: "ഞാന്‍ നിങ്ങള്‍ക്ക് വ്യക്തമായ മുന്നറിയിപ്പു നല്‍കുന്നവനാണ്.
\end{malayalam}}
\flushright{\begin{Arabic}
\quranayah[11][26]
\end{Arabic}}
\flushleft{\begin{malayalam}
"നിങ്ങള്‍ അല്ലാഹുവിനെയല്ലാതെ വഴിപ്പെടരുത്. നോവേറിയ ശിക്ഷ ഒരുനാള്‍ നിങ്ങള്‍ക്കുണ്ടാവുമെന്ന് തീര്‍ച്ചയായും ഞാന്‍ ഭയപ്പെടുന്നു.”
\end{malayalam}}
\flushright{\begin{Arabic}
\quranayah[11][27]
\end{Arabic}}
\flushleft{\begin{malayalam}
അപ്പോള്‍ അദ്ദേഹത്തിന്റെ ജനതയിലെ സത്യനിഷേധികളായ പ്രമാണിമാര്‍ പറഞ്ഞു: "ഞങ്ങളുടെ നോട്ടത്തില്‍ നീ ഞങ്ങളെപ്പോലുള്ള ഒരു മനുഷ്യന്‍ മാത്രമാണ്. ഞങ്ങളിലെ നിസ്സാരന്മാര്‍ മാത്രമാണ്, കാര്യവിചാരമില്ലാതെ നിന്നെ പിന്തുടര്‍ന്നതായി ഞങ്ങള്‍ കാണുന്നത്. ഞങ്ങളെക്കാളേറെ ഒരു ശ്രേഷ്ഠതയും നിങ്ങളില്‍ ഞങ്ങള്‍ കാണുന്നില്ല. മാത്രമല്ല; നിങ്ങള്‍ കള്ളവാദികളാണെന്ന് ഞങ്ങള്‍ കരുതുന്നു.”
\end{malayalam}}
\flushright{\begin{Arabic}
\quranayah[11][28]
\end{Arabic}}
\flushleft{\begin{malayalam}
അദ്ദേഹം പറഞ്ഞു: "എന്റെ ജനമേ, നിങ്ങള്‍ ആലോചിച്ചു നോക്കിയിട്ടുണ്ടോ? ഞാനെന്റെ നാഥനില്‍ നിന്നുള്ള സ്പഷ്ടമായ പ്രമാണങ്ങള്‍ മുറുകെ പിടിക്കുന്നവനാണ്; അവന്‍ തന്റെ അനുഗ്രഹമെനിക്ക് തന്നിരിക്കുന്നു; നിങ്ങള്‍ക്കത് കാണാന്‍ കഴിയുന്നില്ലെങ്കില്‍ ഞാനെന്തു ചെയ്യാനാണ്? നിങ്ങള്‍ക്കത് ഇഷ്ടമില്ലാതിരിക്കെ നിങ്ങളതംഗീകരിക്കാന്‍ ഞങ്ങള്‍ നിര്‍ബന്ധിക്കുകയോ?
\end{malayalam}}
\flushright{\begin{Arabic}
\quranayah[11][29]
\end{Arabic}}
\flushleft{\begin{malayalam}
"എന്റെ ജനമേ, ഇതിന്റെ പേരില്‍ ഞാന്‍ നിങ്ങളോട് സ്വത്തൊന്നും ചോദിക്കുന്നില്ല. എന്റെ പ്രതിഫലം അല്ലാഹുവിങ്കല്‍ മാത്രമാണ്. വിശ്വസിച്ചവരെ ആട്ടിയകറ്റുന്നവനല്ല ഞാന്‍. തീര്‍ച്ചയായും അവര്‍ തങ്ങളുടെ നാഥനുമായി സന്ധിക്കും. എന്നാല്‍ നിങ്ങളെ തികഞ്ഞ അവിവേകികളായാണ് ഞാന്‍ കാണുന്നത്.
\end{malayalam}}
\flushright{\begin{Arabic}
\quranayah[11][30]
\end{Arabic}}
\flushleft{\begin{malayalam}
"എന്റെ ജനമേ, ഞാന്‍ അവരെ ആട്ടിയകറ്റിയാല്‍ അല്ലാഹുവിന്റെ ശിക്ഷയില്‍നിന്ന് ആരാണെന്നെ രക്ഷിക്കുക? നിങ്ങളിക്കാര്യം മനസ്സിലാക്കുന്നില്ലേ?
\end{malayalam}}
\flushright{\begin{Arabic}
\quranayah[11][31]
\end{Arabic}}
\flushleft{\begin{malayalam}
"അല്ലാഹുവിന്റെ ഖജനാവുകള്‍ എന്റെ വശമുണ്ടെന്ന് ഞാന്‍ നിങ്ങളോട് പറയുന്നില്ല. എനിക്ക് അഭൌതിക കാര്യങ്ങളറിയുകയുമില്ല. ഞാന്‍ മലക്കാണെന്നു വാദിക്കുന്നുമില്ല. നിങ്ങളുടെ കണ്ണില്‍ നിസ്സാരരായി കാണുന്നവര്‍ക്ക് അല്ലാഹു യാതൊരു ഗുണവും നല്‍കുകയില്ല എന്നു പറയാനും ഞാനില്ല. അവരുടെ മനസ്സുകളിലുള്ളത് നന്നായറിയുന്നവന്‍ അല്ലാഹുവാണ്. ഇതൊന്നുമംഗീകരിക്കുന്നില്ലെങ്കില്‍ ഞാന്‍ അതിക്രമികളില്‍പെട്ടവനായിത്തീരും; തീര്‍ച്ച.”
\end{malayalam}}
\flushright{\begin{Arabic}
\quranayah[11][32]
\end{Arabic}}
\flushleft{\begin{malayalam}
അവര്‍ പറഞ്ഞു: "നൂഹേ, നീ ഞങ്ങളോട് തര്‍ക്കിച്ചു. വളരെക്കൂടുതലായി തര്‍ക്കിച്ചു. അതിനാല്‍ നീ ഭീഷണിപ്പെടുത്തിക്കൊണ്ടിരിക്കുന്ന ആ ശിക്ഷയിങ്ങ് കൊണ്ടുവരിക. നീ സത്യവാദിയാണെങ്കില്‍!”
\end{malayalam}}
\flushright{\begin{Arabic}
\quranayah[11][33]
\end{Arabic}}
\flushleft{\begin{malayalam}
നൂഹ് പറഞ്ഞു: "അല്ലാഹു ഇച്ഛിച്ചെങ്കില്‍ അവന്‍ തന്നെയാണ് നിങ്ങള്‍ക്കത് കൊണ്ടുവരിക. അപ്പോഴവനെ തോല്‍പിക്കാന്‍ നിങ്ങള്‍ക്കാവില്ല.
\end{malayalam}}
\flushright{\begin{Arabic}
\quranayah[11][34]
\end{Arabic}}
\flushleft{\begin{malayalam}
"അല്ലാഹു നിങ്ങളെ വഴിതെറ്റിച്ചു കളയാനിച്ഛിക്കുന്നുവെങ്കില്‍ ഞാന്‍ നിങ്ങളെ എത്ര ഉപദേശിച്ചാലും ആ ഉപദേശം നിങ്ങള്‍ക്ക് ഉപകരിക്കുകയില്ല. അവനാണ് നിങ്ങളുടെ നാഥന്‍. അവങ്കലേക്കാണ് നിങ്ങള്‍ തിരിച്ചുചെല്ലേണ്ടത്.”
\end{malayalam}}
\flushright{\begin{Arabic}
\quranayah[11][35]
\end{Arabic}}
\flushleft{\begin{malayalam}
നബിയേ, അതല്ല; “അയാളിത് സ്വയം കെട്ടിച്ചമച്ചതാണെ”ന്നാണോ അവര്‍ പറയുന്നത്? പറയുക: "ഞാനത് കെട്ടിച്ചമച്ചതാണെങ്കില്‍ എന്റെ പാപത്തിന്റെ ദോഷഫലം എനിക്കുതന്നെയായിരിക്കും. നിങ്ങള്‍ ചെയ്യുന്ന കുറ്റങ്ങളില്‍ നിന്ന് ഞാന്‍ തീര്‍ത്തും മുക്തനാണ്.”
\end{malayalam}}
\flushright{\begin{Arabic}
\quranayah[11][36]
\end{Arabic}}
\flushleft{\begin{malayalam}
നൂഹിന് ദിവ്യസന്ദേശം ലഭിച്ചു: നിന്റെ ജനതയില്‍ ഇതുവരെ വിശ്വസിച്ചുകഴിഞ്ഞവരല്ലാതെ ഇനിയാരും വിശ്വസിക്കുകയില്ല. അതിനാല്‍ അവര്‍ ചെയ്തുകൊണ്ടിരിക്കുന്നതിനെ സംബന്ധിച്ച് നീ സങ്കടപ്പെടേണ്ടതില്ല.
\end{malayalam}}
\flushright{\begin{Arabic}
\quranayah[11][37]
\end{Arabic}}
\flushleft{\begin{malayalam}
നമ്മുടെ മേല്‍നോട്ടത്തിലും നമ്മുടെ നിര്‍ദേശമനുസരിച്ചും നീ കപ്പലുണ്ടാക്കുക. അക്രമം കാണിച്ചവരുടെ കാര്യത്തില്‍ നീയെന്നോടൊന്നും പറയരുത്. അവര്‍ മുങ്ങിച്ചാവുകതന്നെ ചെയ്യും.
\end{malayalam}}
\flushright{\begin{Arabic}
\quranayah[11][38]
\end{Arabic}}
\flushleft{\begin{malayalam}
അദ്ദേഹം കപ്പലുണ്ടാക്കുന്നു. ആ ജനതയിലെ പ്രമാണിക്കൂട്ടം അദ്ദേഹത്തിനരികിലൂടെ നടന്നുപോയപ്പോഴെല്ലാം അദ്ദേഹത്തെ പരിഹസിച്ചു. അദ്ദേഹം പറഞ്ഞു: "ഇപ്പോള്‍ നിങ്ങള്‍ ഞങ്ങളെ പരിഹസിക്കുന്നു. ഒരുനാള്‍ നിങ്ങള്‍ പരിഹസിക്കുന്നപോലെ ഞങ്ങള്‍ നിങ്ങളെയും പരിഹസിക്കും.
\end{malayalam}}
\flushright{\begin{Arabic}
\quranayah[11][39]
\end{Arabic}}
\flushleft{\begin{malayalam}
"അപമാനകരമായ ശിക്ഷ ആര്‍ക്കാണ് വന്നെത്തുകയെന്നും സ്ഥിരമായ ശിക്ഷ ആരുടെ മേലാണ് വന്നു പതിക്കുകയെന്നും നിങ്ങള്‍ വൈകാതെ അറിയും.”
\end{malayalam}}
\flushright{\begin{Arabic}
\quranayah[11][40]
\end{Arabic}}
\flushleft{\begin{malayalam}
അങ്ങനെ നമ്മുടെ വിധി വന്നു. അടുപ്പില്‍ ഉറവ പൊട്ടി. അപ്പോള്‍ നാം പറഞ്ഞു: "എല്ലാ ജന്തുവര്‍ഗത്തില്‍നിന്നും ഈരണ്ടു ഇണകളെ അതില്‍ കയറ്റുക. നിന്റെ കുടുംബത്തെയും. നേരത്തെ തീരുമാന പ്രഖ്യാപനം ഉണ്ടായവരെയൊഴികെ. വിശ്വസിച്ചവരെയും കയറ്റുക.” വളരെ കുറച്ചു പേരല്ലാതെ അദ്ദേഹത്തോടൊപ്പം വിശ്വസിച്ചവരായി ഉണ്ടായിരുന്നില്ല.
\end{malayalam}}
\flushright{\begin{Arabic}
\quranayah[11][41]
\end{Arabic}}
\flushleft{\begin{malayalam}
അദ്ദേഹം പറഞ്ഞു: "നിങ്ങളതില്‍ കയറുക. അതിന്റെ നീക്കവും നില്‍പുമെല്ലാം അല്ലാഹുവിന്റെ നാമത്തിലാണ്. എന്റെ നാഥന്‍ ഏറെ പൊറുക്കുന്നവനും പരമദയാലുവുമാണ്.”
\end{malayalam}}
\flushright{\begin{Arabic}
\quranayah[11][42]
\end{Arabic}}
\flushleft{\begin{malayalam}
പര്‍വതങ്ങള്‍ പോലുള്ള തിരമാലകള്‍ക്കിടയിലൂടെ അത് അവരെയും കൊണ്ട് സഞ്ചരിക്കുകയായിരുന്നു. നൂഹ് തന്റെ മകനെ വിളിച്ചു- അവന്‍ വളരെ ദൂരെയായിരുന്നു- "എന്റെ കുഞ്ഞുമോനേ, നീ ഞങ്ങളുടെ കൂടെ ഇതില്‍ കയറുക. നീ സത്യനിഷേധികളോടൊപ്പമാകരുതേ.”
\end{malayalam}}
\flushright{\begin{Arabic}
\quranayah[11][43]
\end{Arabic}}
\flushleft{\begin{malayalam}
അവന്‍ പറഞ്ഞു: "ഞാനൊരു മലയില്‍ അഭയം തേടിക്കൊള്ളാം. അതെന്നെ വെള്ളപ്പൊക്കത്തില്‍ നിന്ന് രക്ഷിച്ചുകൊള്ളും.” നൂഹ് പറഞ്ഞു: "ഇന്ന് ദൈവ വിധിയില്‍നിന്ന് രക്ഷിക്കുന്ന ഒന്നുമില്ല. അവന്‍ കരുണ കാണിക്കുന്നവരൊഴികെ.” അപ്പോഴേക്കും അവര്‍ക്കിടയില്‍ തിരമാല മറയിട്ടു. അങ്ങനെ അവന്‍ മുങ്ങിമരിച്ചവരില്‍ പെട്ടുപോയി.
\end{malayalam}}
\flushright{\begin{Arabic}
\quranayah[11][44]
\end{Arabic}}
\flushleft{\begin{malayalam}
അപ്പോള്‍ കല്‍പനയുണ്ടായി: "ഓ ഭൂമി, നിന്നിലെ വെള്ളമൊക്കെ നീ കുടിച്ചുതീര്‍ക്കൂ. ആകാശമേ, മഴ നിര്‍ത്തൂ.” വെള്ളം വറ്റുകയും കല്‍പന നടപ്പാവുകയും ചെയ്തു. കപ്പല്‍ ജൂദി പര്‍വതത്തിന്മേല്‍ ചെന്നു നിന്നു. അപ്പോള്‍ ഇങ്ങനെ അരുളപ്പാടുണ്ടായി: "അക്രമികളായ ജനതക്കു നാശം!”
\end{malayalam}}
\flushright{\begin{Arabic}
\quranayah[11][45]
\end{Arabic}}
\flushleft{\begin{malayalam}
നൂഹ് തന്റെ നാഥനെ വിളിച്ചു പറഞ്ഞു: "നാഥാ! എന്റെ മകന്‍ എന്റെ കുടുംബത്തില്‍പെട്ടവന്‍ തന്നെയാണല്ലോ. തീര്‍ച്ചയായും നിന്റെ വാഗ്ദാനം സത്യവുമാണ്. നീയോ വിധികര്‍ത്താക്കളില്‍ ഏറ്റവും നന്നായി വിധി കല്‍പിക്കുന്നവനും.”
\end{malayalam}}
\flushright{\begin{Arabic}
\quranayah[11][46]
\end{Arabic}}
\flushleft{\begin{malayalam}
അല്ലാഹു പറഞ്ഞു: "നൂഹേ, നിശ്ചയമായും അവന്‍ നിന്റെ കുടുംബത്തില്‍ പെട്ടവനല്ല. അവന്‍ ദുര്‍വൃത്തിയാകുന്നു. അതിനാല്‍ യാഥാര്‍ഥ്യം എന്തെന്ന് നിനക്കറിയാത്ത കാര്യം നീ എന്നോടാവശ്യപ്പെടരുത്. അവിവേകികളില്‍ പെടരുതെന്ന് ഞാനിതാ നിന്നെ ഉപദേശിക്കുന്നു.”
\end{malayalam}}
\flushright{\begin{Arabic}
\quranayah[11][47]
\end{Arabic}}
\flushleft{\begin{malayalam}
നൂഹ് പറഞ്ഞു: "എന്റെ നാഥാ, എനിക്കറിയാത്ത കാര്യം നിന്നോട് ആവശ്യപ്പെടുന്നതില്‍ നിന്ന് ഞാനിതാ നിന്നിലഭയം തേടുന്നു. നീ എനിക്ക് പൊറുത്തുതരികയും എന്നോട് കരുണ കാണിക്കുകയും ചെയ്യുന്നില്ലെങ്കില്‍ ഞാന്‍ നഷ്ടപ്പെട്ടവനായിത്തീരും.”
\end{malayalam}}
\flushright{\begin{Arabic}
\quranayah[11][48]
\end{Arabic}}
\flushleft{\begin{malayalam}
അദ്ദേഹത്തോടു പറഞ്ഞു: "നൂഹേ, നീ കരക്കിറങ്ങുക. നമ്മില്‍ നിന്നുള്ള സമാധാനം നിനക്കുണ്ട്. നിനക്കും നിന്നോടൊപ്പമുള്ള ചില സമൂഹങ്ങള്‍ക്കും നമ്മുടെ അനുഗ്രഹവുമുണ്ട്. എന്നാല്‍ മറ്റു ചില സമൂഹങ്ങളുണ്ട്. അവര്‍ക്ക് നാം താല്‍ക്കാലിക ജീവിതസുഖം നല്‍കും. പിന്നെ നമ്മില്‍ നിന്നുള്ള നോവേറിയ ശിക്ഷ അവരെ ബാധിക്കുകയും ചെയ്യും.
\end{malayalam}}
\flushright{\begin{Arabic}
\quranayah[11][49]
\end{Arabic}}
\flushleft{\begin{malayalam}
നബിയേ, ഇതൊക്കെ അദൃശ്യ കാര്യങ്ങളെ സംബന്ധിച്ച വര്‍ത്തമാനങ്ങളില്‍പെട്ടതാണ്. നിനക്കു നാമത് ബോധനം നല്‍കുന്നു. നീയോ നിന്റെ ജനതയോ ആരും തന്നെ ഇതിനു മുമ്പ് ഇതേക്കുറിച്ച് അറിയുമായിരുന്നില്ല. അതിനാല്‍ ക്ഷമിക്കുക. സംശയമില്ല; അവസാനഫലം ഭക്തന്മാര്‍ക്കനുഗുണമായിരിക്കും.
\end{malayalam}}
\flushright{\begin{Arabic}
\quranayah[11][50]
\end{Arabic}}
\flushleft{\begin{malayalam}
ആദ് ജനതയിലേക്ക് അവരുടെ സഹോദരന്‍ ഹൂദിനെ നാം നിയോഗിച്ചു. അദ്ദേഹം പറഞ്ഞു: "എന്റെ ജനമേ, നിങ്ങള്‍ അല്ലാഹുവിന് വഴിപ്പെടുക. അവനല്ലാതെ നിങ്ങള്‍ക്കൊരു ദൈവമില്ല. നിങ്ങള്‍ കെട്ടിച്ചമച്ചു കള്ളം പറയുന്നവര്‍ മാത്രമാണ്.
\end{malayalam}}
\flushright{\begin{Arabic}
\quranayah[11][51]
\end{Arabic}}
\flushleft{\begin{malayalam}
"എന്റെ ജനമേ, ഇതിന്റെ പേരില്‍ ഞാന്‍ നിങ്ങളോടൊരു പ്രതിഫലവും ആവശ്യപ്പെടുന്നില്ല. എനിക്കുള്ള പ്രതിഫലം എന്നെ പടച്ചവന്റെതുമാത്രമാണ്. നിങ്ങള്‍ ആലോചിക്കുന്നില്ലേ?
\end{malayalam}}
\flushright{\begin{Arabic}
\quranayah[11][52]
\end{Arabic}}
\flushleft{\begin{malayalam}
"എന്റെ ജനമേ, നിങ്ങള്‍ നിങ്ങളുടെ നാഥനോട് മാപ്പിരക്കുക. പിന്നെ അവങ്കലേക്ക് പശ്ചാത്തപിച്ചു മടങ്ങുക. എങ്കിലവന്‍ നിങ്ങള്‍ക്ക് മാനത്തുനിന്ന് വേണ്ടുവോളം മഴ വീഴ്ത്തിത്തരും. നിങ്ങള്‍ക്ക് ഇപ്പോഴുള്ള ശക്തി വളരെയേറെ വര്‍ധിപ്പിച്ചുതരും. അതിനാല്‍ പാപികളായി പിന്തിരിഞ്ഞുപോവരുത്.”
\end{malayalam}}
\flushright{\begin{Arabic}
\quranayah[11][53]
\end{Arabic}}
\flushleft{\begin{malayalam}
അവര്‍ പറഞ്ഞു: "ഹൂദേ, നീ ഞങ്ങള്‍ക്ക് വ്യക്തമായൊരു തെളിവും കൊണ്ടുവന്നിട്ടില്ല. നിന്റെ വാക്കുകേട്ട് മാത്രം ഞങ്ങള്‍ ഞങ്ങളുടെ ദൈവങ്ങളെ വെടിയുകയുമില്ല. ഞങ്ങള്‍ നിന്നിലൊട്ടും വിശ്വസിക്കുന്നുമില്ല.
\end{malayalam}}
\flushright{\begin{Arabic}
\quranayah[11][54]
\end{Arabic}}
\flushleft{\begin{malayalam}
"ഞങ്ങള്‍ക്കു പറയാനുള്ളതിതാണ്: നിനക്കു ഞങ്ങളുടെ ദൈവങ്ങളിലാരുടെയോ ദോഷബാധയേറ്റിരിക്കുന്നു.” ഹൂദ് പറഞ്ഞു: "ഞാന്‍ അല്ലാഹുവെ സാക്ഷിയാക്കുന്നു. നിങ്ങളും സാക്ഷ്യം വഹിക്കുക. നിങ്ങളവനില്‍ പങ്കുചേര്‍ക്കുന്നതില്‍ നിന്നൊക്കെ ഞാന്‍ മുക്തനാകുന്നു....
\end{malayalam}}
\flushright{\begin{Arabic}
\quranayah[11][55]
\end{Arabic}}
\flushleft{\begin{malayalam}
അല്ലാഹുവെക്കൂടാതെ. അതിനാല്‍ നിങ്ങളെല്ലാവരും ചേര്‍ന്ന് എനിക്കെതിരെ തന്ത്രം പ്രയോഗിച്ചുകൊള്ളുക. നിങ്ങള്‍ എനിക്കൊട്ടും അവധി തരേണ്ടതില്ല.
\end{malayalam}}
\flushright{\begin{Arabic}
\quranayah[11][56]
\end{Arabic}}
\flushleft{\begin{malayalam}
"ഞാനിതാ അല്ലാഹുവില്‍ ഭരമേല്‍പിച്ചിരിക്കുന്നു. എന്റെയും നിങ്ങളുടെയും നാഥനാണവന്‍. ഒരു ജന്തുവുമില്ല; അതിന്റെ മൂര്‍ധാവ് അവന്റെ പിടിയിലായിക്കൊണ്ടല്ലാതെ. എന്റെ നാഥന്‍ നേര്‍വഴിയിലാകുന്നു; തീര്‍ച്ച.
\end{malayalam}}
\flushright{\begin{Arabic}
\quranayah[11][57]
\end{Arabic}}
\flushleft{\begin{malayalam}
"ഏതൊരു സന്ദേശവുമായാണോ ഞാന്‍ നിങ്ങളിലേക്ക് നിയോഗിതനായത് അതു ഞാന്‍ നിങ്ങള്‍ക്ക് എത്തിച്ചുതന്നിരിക്കുന്നു. ഇനി നിങ്ങള്‍ പിന്തിരിഞ്ഞുപോവുകയാണെങ്കില്‍ അറിയുക: നിങ്ങള്‍ക്കു പകരം മറ്റൊരു ജനതയെ എന്റെ നാഥന്‍ കൊണ്ടുവരിക തന്നെ ചെയ്യും. അവനൊരു ദ്രോഹവും വരുത്താന്‍ നിങ്ങള്‍ക്കാവില്ല. എന്റെ നാഥന്‍ എല്ലാ കാര്യത്തിനും മേല്‍നോട്ടം വഹിക്കുന്നവനാണ്.”
\end{malayalam}}
\flushright{\begin{Arabic}
\quranayah[11][58]
\end{Arabic}}
\flushleft{\begin{malayalam}
നമ്മുടെ വിധിവന്നപ്പോള്‍ ഹൂദിനെയും അദ്ദേഹത്തോടൊപ്പം വിശ്വസിച്ചവരെയും നമ്മുടെ അനുഗ്രഹത്താല്‍ നാം രക്ഷപ്പെടുത്തി. കൊടിയ ശിക്ഷയില്‍നിന്ന് നാമവരെ മോചിപ്പിച്ചു.
\end{malayalam}}
\flushright{\begin{Arabic}
\quranayah[11][59]
\end{Arabic}}
\flushleft{\begin{malayalam}
അതാണ് ആദ് ജനത. തങ്ങളുടെ നാഥന്റെ പ്രമാണങ്ങളെ അവര്‍ നിഷേധിച്ചു. അവന്റെ ദൂതന്മാരെ ധിക്കരിച്ചു. ധിക്കാരികളായ എല്ലാ സ്വേച്ഛാധിപതികളുടെയും കല്‍പന പിന്‍പറ്റുകയും ചെയ്തു.
\end{malayalam}}
\flushright{\begin{Arabic}
\quranayah[11][60]
\end{Arabic}}
\flushleft{\begin{malayalam}
എന്നാല്‍ ഐഹികജീവിതത്തിലും ഉയിര്‍ത്തെഴുന്നേല്‍പുനാളിലും ശാപം അവരെ പിന്തുടരും. അറിയുക: ആദ് ജനത തങ്ങളുടെ നാഥനെ തള്ളിപ്പറഞ്ഞു. അതിനാല്‍, ഹൂദിന്റെ ജനതയായ ആദിന് നാശം!
\end{malayalam}}
\flushright{\begin{Arabic}
\quranayah[11][61]
\end{Arabic}}
\flushleft{\begin{malayalam}
സമൂദ് ഗോത്രത്തിലേക്ക് അവരുടെ സഹോദരന്‍ സ്വാലിഹിനെ നാം നിയോഗിച്ചു. അദ്ദേഹം പറഞ്ഞു: "എന്റെ ജനമേ, നിങ്ങള്‍ അല്ലാഹുവിനു വഴിപ്പെടുക. അവനല്ലാതെ നിങ്ങള്‍ക്കൊരു ദൈവമില്ല. അവന്‍ നിങ്ങളെ ഭൂമിയില്‍ നിന്ന് സൃഷ്ടിച്ചു വളര്‍ത്തി. നിങ്ങളെ അവിടെ കുടിയിരുത്തുകയും ചെയ്തു. അതിനാല്‍ നിങ്ങളവനോട് മാപ്പിരക്കുക. പിന്നെ അവങ്കലേക്ക് പശ്ചാത്തപിച്ചു മടങ്ങുക. നിശ്ചയമായും എന്റെ നാഥന്‍ നിങ്ങള്‍ക്ക് ഏറെ അടുത്തവനത്രെ. ഉത്തരം നല്‍കുന്നവനും അവന്‍ തന്നെ.”
\end{malayalam}}
\flushright{\begin{Arabic}
\quranayah[11][62]
\end{Arabic}}
\flushleft{\begin{malayalam}
അവര്‍ പറഞ്ഞു: "സ്വാലിഹേ, ഇതിനുമുമ്പ് നീ ഞങ്ങള്‍ക്കിടയില്‍ ഏറെ വേണ്ടപ്പെട്ടവനായിരുന്നു. നീയിപ്പോള്‍ ഞങ്ങളുടെ പൂര്‍വികര്‍ പൂജിച്ചിരുന്നവയെ ഞങ്ങള്‍ പൂജിക്കുന്നത് വിലക്കുകയാണോ? നീ ഞങ്ങളെ ക്ഷണിച്ചുകൊണ്ടിരിക്കുന്ന കാര്യത്തെപ്പറ്റി ഞങ്ങള്‍ സങ്കീര്‍ണമായ സംശയത്തിലാണ്.”
\end{malayalam}}
\flushright{\begin{Arabic}
\quranayah[11][63]
\end{Arabic}}
\flushleft{\begin{malayalam}
സ്വാലിഹ് പറഞ്ഞു: "എന്റെ ജനമേ, നിങ്ങള്‍ ചിന്തിച്ചിട്ടുണ്ടോ? ഞാന്‍ എന്റെ നാഥനില്‍നിന്നുള്ള വ്യക്തമായ പ്രമാണം മുറുകെപ്പിടിക്കുന്നു. അവന്റെ അനുഗ്രഹം അവനെനിക്കു നല്‍കിയിരിക്കുന്നു. എന്നിട്ടും ഞാന്‍ അല്ലാഹുവെ ധിക്കരിക്കുകയാണെങ്കില്‍ അവന്റെ കഠിനമായ ശിക്ഷയില്‍ നിന്ന് ആരാണെന്നെ രക്ഷിക്കുക? എനിക്ക് കൂടുതല്‍ നഷ്ടം വരുത്താനല്ലാതെ നിങ്ങള്‍ക്കെന്തു ചെയ്യാന്‍ കഴിയും?
\end{malayalam}}
\flushright{\begin{Arabic}
\quranayah[11][64]
\end{Arabic}}
\flushleft{\begin{malayalam}
"എന്റെ ജനമേ, ഇതാ അല്ലാഹുവിന്റെ ഒട്ടകം. നിങ്ങള്‍ക്കുള്ള ദൃഷ്ടാന്തമാണിത്. അല്ലാഹുവിന്റെ ഭൂമിയില്‍ മേഞ്ഞുനടക്കാന്‍ നിങ്ങളതിനെ വിട്ടേക്കുക. നിങ്ങളതിനൊരു ദ്രോഹവും വരുത്തരുത്. അങ്ങനെ ചെയ്താല്‍ അടുത്തുതന്നെ കടുത്ത ശിക്ഷ നിങ്ങളെ പിടികൂടും.”
\end{malayalam}}
\flushright{\begin{Arabic}
\quranayah[11][65]
\end{Arabic}}
\flushleft{\begin{malayalam}
എന്നിട്ടും അവരതിനെ അറുകൊല ചെയ്തു. അപ്പോള്‍ സ്വാലിഹ് പറഞ്ഞു: "നിങ്ങളിനി മൂന്നുദിവസം മാത്രം നിങ്ങളുടെ വീടുകളില്‍ സുഖിച്ചുകഴിയുക. ഒട്ടും പിഴവുപറ്റാത്ത സമയ നിര്‍ണയമാണിത്.”
\end{malayalam}}
\flushright{\begin{Arabic}
\quranayah[11][66]
\end{Arabic}}
\flushleft{\begin{malayalam}
അങ്ങനെ നമ്മുടെ വിധി വന്നപ്പോള്‍ സ്വാലിഹിനെയും അദ്ദേഹത്തോടൊപ്പമുള്ള വിശ്വാസികളേയും നമ്മുടെ കാരുണ്യത്താല്‍ നാം രക്ഷപ്പെടുത്തി. അന്നാളിലെ അപമാനത്തില്‍ നിന്നും നാമവരെ മോചിപ്പിച്ചു. നിന്റെ നാഥന്‍ ശക്തനും അജയ്യനുമാണ്.
\end{malayalam}}
\flushright{\begin{Arabic}
\quranayah[11][67]
\end{Arabic}}
\flushleft{\begin{malayalam}
അക്രമം കാണിച്ചവരെ ഘോരഗര്‍ജനം പിടികൂടി. അങ്ങനെ പ്രഭാതത്തിലവര്‍ തങ്ങളുടെ വീടുകളില്‍ കമിഴ്ന്നു വീണുകിടക്കുന്നവരായിത്തീര്‍ന്നു.
\end{malayalam}}
\flushright{\begin{Arabic}
\quranayah[11][68]
\end{Arabic}}
\flushleft{\begin{malayalam}
അവരവിടെ പാര്‍ത്തിട്ടേയില്ലെന്ന പോലെയായി. അറിയുക: സമൂദ് ഗോത്രം തങ്ങളുടെ നാഥനെ ധിക്കരിച്ചു. അതിനാല്‍ സമൂദ് ഗോത്രത്തിന് നാശം!
\end{malayalam}}
\flushright{\begin{Arabic}
\quranayah[11][69]
\end{Arabic}}
\flushleft{\begin{malayalam}
നമ്മുടെ ദൂതന്മാര്‍ ശുഭവൃത്താന്തവുമായി ഇബ്റാഹീമിനെ സമീപിച്ചു. അവര്‍ പറഞ്ഞു: "സലാം.” അദ്ദേഹം പറഞ്ഞു: "സലാം.” ഒട്ടും വൈകാതെ അദ്ദേഹം വേവിച്ചു പാകംചെയ്ത ഒരു കാളക്കുട്ടിയെ കൊണ്ടുവന്നു.
\end{malayalam}}
\flushright{\begin{Arabic}
\quranayah[11][70]
\end{Arabic}}
\flushleft{\begin{malayalam}
അവരുടെ കൈകള്‍ അതിലേക്ക് നീളുന്നില്ലെന്ന് കണ്ടപ്പോള്‍ അദ്ദേഹത്തിന് അവരെ സംബന്ധിച്ച് സംശയമായി. അവരെപ്പറ്റി പേടി തോന്നുകയും ചെയ്തു. അവര്‍ പറഞ്ഞു: "പേടിക്കേണ്ട. ഞങ്ങള്‍ ലൂത്വിന്റെ ജനതയിലേക്ക് നിയോഗിക്കപ്പെട്ടവരാണ്.”
\end{malayalam}}
\flushright{\begin{Arabic}
\quranayah[11][71]
\end{Arabic}}
\flushleft{\begin{malayalam}
ഇബ്റാഹീമിന്റെ ഭാര്യ അവിടെ നില്‍ക്കുന്നുണ്ടായിരുന്നു. അവര്‍ ചിരിച്ചു. അപ്പോള്‍ അവരെ ഇസ്ഹാഖിനെ പറ്റിയും ഇസ്ഹാഖിന് പിറകെ യഅ്ഖൂബിനെ പറ്റിയും നാം ശുഭവാര്‍ത്ത അറിയിച്ചു.
\end{malayalam}}
\flushright{\begin{Arabic}
\quranayah[11][72]
\end{Arabic}}
\flushleft{\begin{malayalam}
അവര്‍ പറഞ്ഞു: "എന്ത്! ഞാന്‍ പടുകിഴവിയായിരിക്കുന്നു. ഇനി പ്രസവിക്കുകയോ? എന്റെ ഭര്‍ത്താവും ഇതാ പടുവൃദ്ധനായിരിക്കുന്നു. ഇതൊരദ്ഭുതകരമായ കാര്യം തന്നെ.”
\end{malayalam}}
\flushright{\begin{Arabic}
\quranayah[11][73]
\end{Arabic}}
\flushleft{\begin{malayalam}
ആ ദൂതന്മാര്‍ പറഞ്ഞു: "അല്ലാഹുവിന്റെ വിധിയില്‍ നീ അദ്ഭുതപ്പെടുകയോ? ഇബ്റാഹീമിന്റെ വീട്ടുകാരേ, നിങ്ങള്‍ക്ക് അല്ലാഹുവിന്റെ കാരുണ്യവും അനുഗ്രഹവുമുണ്ടാവട്ടെ. അവന്‍ സ്തുത്യര്‍ഹനും ഏറെ മഹത്വമുള്ളവനുമാണ്.”
\end{malayalam}}
\flushright{\begin{Arabic}
\quranayah[11][74]
\end{Arabic}}
\flushleft{\begin{malayalam}
അങ്ങനെ ഇബ്റാഹീമിന്റെ പരിഭ്രമം വിട്ടുമാറുകയും ശുഭവാര്‍ത്ത വന്നെത്തുകയും ചെയ്തപ്പോള്‍ ലൂത്വിന്റെ ജനതയുടെ കാര്യത്തില്‍ അദ്ദേഹം നമ്മോടു തര്‍ക്കിക്കാന്‍ തുടങ്ങി.
\end{malayalam}}
\flushright{\begin{Arabic}
\quranayah[11][75]
\end{Arabic}}
\flushleft{\begin{malayalam}
ഉറപ്പായും ഇബ്റാഹീം ക്ഷമാശീലനും ഏറെ ദയാലുവുമാണ്. സദാ പശ്ചാത്തപിക്കുന്നവനും.
\end{malayalam}}
\flushright{\begin{Arabic}
\quranayah[11][76]
\end{Arabic}}
\flushleft{\begin{malayalam}
ഇബ്റാഹീമേ; ഇതങ്ങ് വിട്ടേക്കുക. നിശ്ചയമായും നിന്റെ നാഥന്റെ വിധി വന്നുകഴിഞ്ഞു. ആര്‍ക്കും തടുക്കാനാവാത്ത ശിക്ഷ അവര്‍ക്ക് വന്നെത്തുക തന്നെ ചെയ്യും.
\end{malayalam}}
\flushright{\begin{Arabic}
\quranayah[11][77]
\end{Arabic}}
\flushleft{\begin{malayalam}
നമ്മുടെ ദൂതന്മാര്‍ ലൂത്വിന്റെ അടുത്തെത്തി. അവരുടെ വരവില്‍ അദ്ദേഹം അതീവ ദുഃഖിതനായി. അവരെക്കുറിച്ചോര്‍ത്ത് അദ്ദേഹത്തിന്റെ മനസ്സ് നൊമ്പരം കൊണ്ടു. അദ്ദേഹം പറഞ്ഞു: "ഇത് പ്രയാസകരമായ ദിനംതന്നെ.”
\end{malayalam}}
\flushright{\begin{Arabic}
\quranayah[11][78]
\end{Arabic}}
\flushleft{\begin{malayalam}
ലൂത്വിന്റെ ജനത അദ്ദേഹത്തിന്റെയടുത്തേക്ക് ഓടിയടുത്തു. നേരത്തെ തന്നെ അവര്‍ നീചവൃത്തികള്‍ ചെയ്യുന്നവരായിരുന്നു. ലൂത്വ് പറഞ്ഞു: "എന്റെ ജനമേ, ഇതാ എന്റെ പെണ്‍കുട്ടികള്‍. ഇവരാണ് നിങ്ങള്‍ക്ക് കൂടുതല്‍ വിശുദ്ധിയുള്ളവര്‍. അതിനാല്‍ നിങ്ങള്‍ അല്ലാഹുവെ സൂക്ഷിക്കുക. എന്റെ അതിഥികളുടെ കാര്യത്തില്‍ എന്നെ മാനക്കേടിലാക്കാതിരിക്കുക. നിങ്ങളില്‍ വിവേകമുള്ള ഒരാളുമില്ലേ?”
\end{malayalam}}
\flushright{\begin{Arabic}
\quranayah[11][79]
\end{Arabic}}
\flushleft{\begin{malayalam}
അവര്‍ പറഞ്ഞു: "നിന്റെ പെണ്‍മക്കളെക്കൊണ്ട് ഞങ്ങള്‍ക്കൊരു പ്രയോജനവുമില്ലെന്ന് നിനക്കുതന്നെ അറിയാമല്ലോ. ഞങ്ങളെന്താണാഗ്രഹിക്കുന്നതെന്നും നിനക്കറിയാം.”
\end{malayalam}}
\flushright{\begin{Arabic}
\quranayah[11][80]
\end{Arabic}}
\flushleft{\begin{malayalam}
ലൂത്വ് പറഞ്ഞു: "നിങ്ങളെ നേരിടാന്‍ എനിക്കു കരുത്തുണ്ടായിരുന്നെങ്കില്‍! അല്ലെങ്കില്‍ ശക്തമായ ഒരു താങ്ങ് എനിക്ക് അവലംബിക്കാനുണ്ടായിരുന്നെങ്കില്‍.”
\end{malayalam}}
\flushright{\begin{Arabic}
\quranayah[11][81]
\end{Arabic}}
\flushleft{\begin{malayalam}
മലക്കുകള്‍ പറഞ്ഞു: "ലൂത്വേ, ഞങ്ങള്‍ നിന്റെ നാഥന്റെ ദൂതന്മാരാണ്. ഈ ആളുകള്‍ക്കൊരിക്കലും നിന്നെ തൊടാനാവില്ല. അതിനാല്‍ രാവേറെക്കഴിഞ്ഞാല്‍ നീ നിന്റെ കുടുംബത്തെ കൂട്ടി പുറപ്പെടുക. നിങ്ങളിലാരും തിരിഞ്ഞുനോക്കരുത്. പക്ഷേ, നിന്റെ ഭാര്യ കൂടെ വരുന്നതല്ല. അക്കൂട്ടര്‍ക്കുള്ള ശിക്ഷ അവളെയും ബാധിക്കും. അവരുടെ നാശത്തിന്റെ നിശ്ചിതസമയം പ്രഭാതമാണ്. പ്രഭാതം അടുത്തുതന്നെയല്ലേ?
\end{malayalam}}
\flushright{\begin{Arabic}
\quranayah[11][82]
\end{Arabic}}
\flushleft{\begin{malayalam}
അങ്ങനെ നമ്മുടെ കല്‍പന വന്നെത്തി. നാം ആ നാടിനെ കീഴ്മേല്‍ മറിച്ചു. അട്ടിയട്ടിയായി ചൂളവെച്ച മണ്‍കട്ടകള്‍ നാം ആ നാടിനുമേല്‍ വര്‍ഷിച്ചു.
\end{malayalam}}
\flushright{\begin{Arabic}
\quranayah[11][83]
\end{Arabic}}
\flushleft{\begin{malayalam}
ആ കട്ടകള്‍ നിന്റെ നാഥന്റെ അടുക്കല്‍വെച്ച് അടയാളപ്പെടുത്തിയവയാണ്. ഈ ശിക്ഷയാവട്ടെ; അത് ഈ അതിക്രമികളില്‍ നിന്ന് ഒട്ടും വിദൂരമല്ല.
\end{malayalam}}
\flushright{\begin{Arabic}
\quranayah[11][84]
\end{Arabic}}
\flushleft{\begin{malayalam}
മദ്യന്‍ നിവാസികളിലേക്ക് അവരുടെ സഹോദരന്‍ ശുഐബിനെ നാം നിയോഗിച്ചു. അദ്ദേഹം പറഞ്ഞു: "എന്റെ ജനമേ, നിങ്ങള്‍ അല്ലാഹുവിന് വഴിപ്പെടുക. അവനല്ലാതെ നിങ്ങള്‍ക്കൊരു ദൈവമില്ല. നിങ്ങള്‍ അളവിലും തൂക്കത്തിലും കുറവു വരുത്തരുത്. ഞാന്‍ നിങ്ങളെ കാണുന്നത് സുസ്ഥിതിയിലാണ്. അതോടൊപ്പം നിങ്ങളെയാകെ വലയം ചെയ്യുന്ന ശിക്ഷ നിങ്ങള്‍ക്കുണ്ടാകുമോയെന്ന് ഞാന്‍ ഭയപ്പെടുകയും ചെയ്യുന്നു.
\end{malayalam}}
\flushright{\begin{Arabic}
\quranayah[11][85]
\end{Arabic}}
\flushleft{\begin{malayalam}
"എന്റെ ജനമേ, നിങ്ങള്‍ നീതിയോടെ അളവിലും തൂക്കത്തിലും തികവു വരുത്തുക. നിങ്ങള്‍ ജനങ്ങള്‍ക്ക് അവരുടെ ചരക്കുകളില്‍ കുറവു വരുത്തരുത്. ഭൂമിയില്‍ കുഴപ്പക്കാരായി കൂത്താടി നടക്കരുത്.
\end{malayalam}}
\flushright{\begin{Arabic}
\quranayah[11][86]
\end{Arabic}}
\flushleft{\begin{malayalam}
"അല്ലാഹു നിങ്ങള്‍ക്കായി കരുതിവെക്കുന്നതാണ് നിങ്ങള്‍ക്കുത്തമം. നിങ്ങള്‍ സത്യവിശ്വാസികളെങ്കില്‍! ഞാന്‍ നിങ്ങളുടെ മേല്‍നോട്ടക്കാരനല്ല.”
\end{malayalam}}
\flushright{\begin{Arabic}
\quranayah[11][87]
\end{Arabic}}
\flushleft{\begin{malayalam}
അവര്‍ പറഞ്ഞു: "ശുഐബേ, നമ്മുടെ പിതാക്കന്മാര്‍ പൂജിച്ചുപോരുന്നവയെ ഞങ്ങളുപേക്ഷിക്കണമെന്നും ഞങ്ങളുടെ ധനം ഞങ്ങളുടെ ഇഷ്ടംപോലെ ഞങ്ങള്‍ കൈകാര്യം ചെയ്യരുതെന്നും നിന്നോട് കല്‍പിക്കുന്നത് നിന്റെ നമസ്കാരമാണോ? നീ വല്ലാത്തൊരു വിവേകശാലിയും സന്മാര്‍ഗി യും തന്നെ!”
\end{malayalam}}
\flushright{\begin{Arabic}
\quranayah[11][88]
\end{Arabic}}
\flushleft{\begin{malayalam}
ശുഐബ് പറഞ്ഞു: "എന്റെ ജനമേ, നിങ്ങള്‍ ആലോചിച്ചിട്ടുണ്ടോ; ഞാന്‍ എന്റെ നാഥനില്‍ നിന്നുള്ള സ്പഷ്ടമായ പ്രമാണം മുറുകെ പിടിക്കുന്നവനാണ്. അവന്‍ എനിക്കു തന്റെ പക്കല്‍നിന്നുള്ള ഉത്തമ വിഭവം നല്‍കിയിരിക്കുന്നു. എന്നിട്ടും ഞാന്‍ നന്ദികെട്ടവനാവുകയോ? ഞാന്‍ നിങ്ങളെ വിലക്കുന്ന അതേ കാര്യം തന്നെ നിങ്ങള്‍ക്കെതിരായി ചെയ്യാന്‍ ഞാനുദ്ദേശിക്കുന്നില്ല. കഴിയാവുന്നിടത്തോളം നിങ്ങള്‍ക്ക് നന്മവരുത്തണമെന്നേ ഞാനുദ്ദേശിക്കുന്നുള്ളൂ. അല്ലാഹുവിലൂടെയല്ലാതെ എനിക്കൊന്നിനും ഒരു കഴിവും കിട്ടുന്നില്ല. ഞാന്‍ അവനില്‍ ഭരമേല്‍പിച്ചിരിക്കുന്നു. അവങ്കലേക്കുതന്നെ ഞാന്‍ എളിമയോടെ മടങ്ങിപ്പോവുകയും ചെയ്യും.
\end{malayalam}}
\flushright{\begin{Arabic}
\quranayah[11][89]
\end{Arabic}}
\flushleft{\begin{malayalam}
"എന്റെ ജനമേ, എന്നോടുള്ള എതിര്‍പ്പ്, നൂഹിന്റെയും സ്വാലിഹിന്റെയും ലൂത്വിന്റെയും ജനതക്ക് ബാധിച്ചതുപോലുള്ള ശിക്ഷ നിങ്ങളെയും ബാധിക്കാന്‍ ഇടവരുത്താതിരിക്കട്ടെ. ലൂത്വിന്റെ ജനത നിങ്ങളില്‍നിന്ന് ഏറെയൊന്നും അകലെയല്ലല്ലോ.
\end{malayalam}}
\flushright{\begin{Arabic}
\quranayah[11][90]
\end{Arabic}}
\flushleft{\begin{malayalam}
"നിങ്ങള്‍ നിങ്ങളുടെ നാഥനോട് മാപ്പിരക്കുക. എന്നിട്ട് അവനിലേക്ക് പശ്ചാത്തപിച്ചു മടങ്ങുക. തീര്‍ച്ചയായും എന്റെ നാഥന്‍ പരമദയാലുവാണ്. ഏറെ സ്നേഹമുള്ളവനും.”
\end{malayalam}}
\flushright{\begin{Arabic}
\quranayah[11][91]
\end{Arabic}}
\flushleft{\begin{malayalam}
അവര്‍ പറഞ്ഞു: "ശുഐബേ, നീ പറയുന്നവയില്‍ ഏറെയും ഞങ്ങള്‍ക്ക് മനസ്സിലാകുന്നേയില്ല. തീര്‍ച്ചയായും ഞങ്ങളറിയുന്നു; ഞങ്ങളെക്കാള്‍ ഏറെ ദുര്‍ബലനാണ് നീയെന്ന്. നിന്റെ കുടുംബമില്ലായിരുന്നെങ്കില്‍ എന്നോ നിന്നെ ഞങ്ങള്‍ കല്ലെറിഞ്ഞു കൊല്ലുമായിരുന്നു. ഞങ്ങളെ സംബന്ധിച്ചിടത്തോളം നീയൊട്ടും അജയ്യനല്ല.”
\end{malayalam}}
\flushright{\begin{Arabic}
\quranayah[11][92]
\end{Arabic}}
\flushleft{\begin{malayalam}
ശുഐബ് ചോദിച്ചു: "എന്റെ ജനമേ, എന്റെ കുടുംബമാണോ അല്ലാഹുവിനെക്കാള്‍ നിങ്ങള്‍ക്ക് പ്രധാനം? അങ്ങനെ നിങ്ങളവനെ നിസ്സാരമാക്കി പുറംതള്ളുകയാണോ? എന്റെ നാഥന്‍ നിങ്ങള്‍ ചെയ്യുന്നതിനെക്കുറിച്ചൊക്കെ സൂക്ഷ്മമായി അറിയുന്നവനാണ്; തീര്‍ച്ച.
\end{malayalam}}
\flushright{\begin{Arabic}
\quranayah[11][93]
\end{Arabic}}
\flushleft{\begin{malayalam}
"എന്റെ ജനമേ, നിങ്ങള്‍ നിങ്ങള്‍ക്ക് തോന്നുംപോലെ പ്രവര്‍ത്തിച്ചുകൊള്ളുക. തീര്‍ച്ചയായും ഞാനും പ്രവര്‍ത്തിച്ചുകൊണ്ടിരിക്കുകയാണ്. ആര്‍ക്കാണ് അപമാനകരമായ ശിക്ഷ വന്നെത്തുകയെന്നും ആരാണ് കള്ളം പറയുന്നതെന്നും ഉറപ്പായും നിങ്ങള്‍ അടുത്തുതന്നെ അറിയും. നിങ്ങള്‍ കാത്തിരുന്നുകൊള്ളുക. ഞാനും നിങ്ങളോടൊപ്പം കാത്തിരിക്കാം.”
\end{malayalam}}
\flushright{\begin{Arabic}
\quranayah[11][94]
\end{Arabic}}
\flushleft{\begin{malayalam}
അവസാനം നമ്മുടെ വിധി വന്നപ്പോള്‍ ശുഐബിനെയും അദ്ദേഹത്തോടൊപ്പം വിശ്വസിച്ചവരെയും നാം നമ്മുടെ കാരുണ്യത്താല്‍ രക്ഷപ്പെടുത്തി. അക്രമം കാണിച്ചവരെ ഘോരഗര്‍ജനം പിടികൂടി. അങ്ങനെ അവര്‍ പ്രഭാതത്തില്‍ തങ്ങളുടെ വീടുകളില്‍ കമിഴ്ന്നുവീണു കിടക്കുന്നവരായിത്തീര്‍ന്നു;
\end{malayalam}}
\flushright{\begin{Arabic}
\quranayah[11][95]
\end{Arabic}}
\flushleft{\begin{malayalam}
അവരവിടെ പാര്‍ത്തിട്ടേയില്ലെന്ന പോലെ. അറിയുക: മദ്യന്‍ വാസികള്‍ പൂര്‍ണമായും തൂത്തെറിയപ്പെട്ടു. സമൂദ് ഗോത്രം തൂത്തെറിയപ്പെട്ടപോലെത്തന്നെ.
\end{malayalam}}
\flushright{\begin{Arabic}
\quranayah[11][96]
\end{Arabic}}
\flushleft{\begin{malayalam}
മൂസായെ നാം നമ്മുടെ പ്രമാണങ്ങളും വ്യക്തമായ അടയാളങ്ങളുമായി അയച്ചു.
\end{malayalam}}
\flushright{\begin{Arabic}
\quranayah[11][97]
\end{Arabic}}
\flushleft{\begin{malayalam}
ഫറവോന്റെയും അവന്റെ പ്രമാണിമാരുടെയും അടുത്തേക്ക്. എന്നിട്ടും അവര്‍ ഫറവോന്റെ കല്‍പന പിന്‍പറ്റുകയാണുണ്ടായത്. ഫറവോന്റെ കല്‍പനയോ, അതൊട്ടും വിവേകപൂര്‍വമായിരുന്നില്ല.
\end{malayalam}}
\flushright{\begin{Arabic}
\quranayah[11][98]
\end{Arabic}}
\flushleft{\begin{malayalam}
ഉയിര്‍ത്തെഴുന്നേല്‍പുനാളില്‍ ഫറവോന്‍ തന്റെ ജനതയുടെ മുന്നിലുണ്ടായിരിക്കും. അങ്ങനെ അവനവരെ നരകത്തീയിലേക്ക് നയിക്കും. ചെന്നെത്താവുന്നതില്‍ ഏറ്റവും ചീത്തയായ ഇടമാണത്.
\end{malayalam}}
\flushright{\begin{Arabic}
\quranayah[11][99]
\end{Arabic}}
\flushleft{\begin{malayalam}
ഈ ലോകത്ത് ശാപം അവരെ പിന്തുടര്‍ന്നു. ഉയിര്‍ത്തെഴുന്നേല്‍പുനാളിലും അതങ്ങനെത്തന്നെ. കിട്ടാവുന്നതില്‍വെച്ച് ഏറ്റം മോശമായ സമ്മാനമാണത്.
\end{malayalam}}
\flushright{\begin{Arabic}
\quranayah[11][100]
\end{Arabic}}
\flushleft{\begin{malayalam}
വിവിധ നാടുകളിലെ സംഭവകഥകളില്‍ ചിലതാണിത്. നാമത് നിനക്ക് വിവരിച്ചുതരുന്നു. ആ നാടുകളില്‍ ചിലത് ഇന്നും നിലനില്‍ക്കുന്നുണ്ട്. ചിലത് നിശ്ശേഷം നശിപ്പിക്കപ്പെട്ടിരിക്കുന്നു.
\end{malayalam}}
\flushright{\begin{Arabic}
\quranayah[11][101]
\end{Arabic}}
\flushleft{\begin{malayalam}
നാം അവരോട് ഒരതിക്രമവും കാണിച്ചിട്ടില്ല. അവര്‍ തങ്ങളോടുതന്നെ അതിക്രമം കാണിക്കുകയായിരുന്നു. നിന്റെ നാഥന്റെ വിധിവന്നു. അപ്പോള്‍ അല്ലാഹുവെ വിട്ട് അവര്‍ പ്രാര്‍ഥിച്ചുകൊണ്ടിരുന്ന ദൈവങ്ങള്‍ക്കൊന്നും അവര്‍ക്കൊരുപകാരവും ചെയ്യാനായില്ല. അവര്‍ക്കവ നാശമല്ലാതൊന്നും വര്‍ധിപ്പിച്ചുകൊടുത്തതുമില്ല.
\end{malayalam}}
\flushright{\begin{Arabic}
\quranayah[11][102]
\end{Arabic}}
\flushleft{\begin{malayalam}
നാട്ടുകാര്‍ അക്രമികളായിരിക്കെ അവരെ പിടികൂടുകയാണെങ്കില്‍ ഇവ്വിധമാണ് നിന്റെ നാഥന്‍ പിടികൂടുക. അവന്റെ പിടുത്തം നോവേറിയതും കഠിനവും തന്നെ.
\end{malayalam}}
\flushright{\begin{Arabic}
\quranayah[11][103]
\end{Arabic}}
\flushleft{\begin{malayalam}
പരലോകശിക്ഷ പേടിക്കുന്നവര്‍ക്ക് തീര്‍ച്ചയായും ഇതില്‍ വ്യക്തമായ തെളിവുണ്ട്. മുഴുവന്‍ മനുഷ്യരും ഒരിടത്തൊരുമിച്ചുകൂടുന്ന ദിനമാണതുണ്ടാവുക. എല്ലാറ്റിനും സാക്ഷ്യമുണ്ടാകുന്ന ദിനമാണത്.
\end{malayalam}}
\flushright{\begin{Arabic}
\quranayah[11][104]
\end{Arabic}}
\flushleft{\begin{malayalam}
നിശ്ചയിക്കപ്പെട്ട ഒരവധി വരെയല്ലാതെ നാമത് നീട്ടിവെക്കുകയില്ല.
\end{malayalam}}
\flushright{\begin{Arabic}
\quranayah[11][105]
\end{Arabic}}
\flushleft{\begin{malayalam}
അത് വന്നെത്തുന്ന ദിനം അല്ലാഹുവിന്റെ അനുമതിയോടെയല്ലാതെ ആര്‍ക്കും ഒന്നും പറയാനാവില്ല. അവരില്‍ കുറേ പേര്‍ നിര്‍ഭാഗ്യവാന്മാരായിരിക്കും. കുറേപേര്‍ സൌഭാഗ്യവാന്മാരും.
\end{malayalam}}
\flushright{\begin{Arabic}
\quranayah[11][106]
\end{Arabic}}
\flushleft{\begin{malayalam}
നിര്‍ഭാഗ്യവാന്മാര്‍ നരകത്തിലായിരിക്കും. അവര്‍ക്കവിടെ നെടുവീര്‍പ്പും തേങ്ങലുകളുമാണുണ്ടാവുക.
\end{malayalam}}
\flushright{\begin{Arabic}
\quranayah[11][107]
\end{Arabic}}
\flushleft{\begin{malayalam}
ആകാശഭൂമികള്‍ ഉള്ളേടത്തോളംകാലം അവരവിടെ സ്ഥിരവാസികളായിരിക്കും. നിന്റെ നാഥന്‍ ഇച്ഛിച്ച കാലമൊഴികെ. തീര്‍ച്ചയായും നിന്റെ നാഥന്‍ താനിച്ഛിക്കുന്നത് നടപ്പാക്കുന്നവനാണ്.
\end{malayalam}}
\flushright{\begin{Arabic}
\quranayah[11][108]
\end{Arabic}}
\flushleft{\begin{malayalam}
എന്നാല്‍ സൌഭാഗ്യവാന്മാര്‍ സ്വര്‍ഗത്തിലായിരിക്കും. ആകാശഭൂമികള്‍ ഉള്ളേടത്തോളം കാലം അവരതില്‍ നിത്യവാസികളായിരിക്കും. നിന്റെ നാഥന്‍ ഇച്ഛിക്കുന്ന കാലമൊഴികെ. ഒടുക്കമില്ലാത്ത സമ്മാനമായിരിക്കും അത്.
\end{malayalam}}
\flushright{\begin{Arabic}
\quranayah[11][109]
\end{Arabic}}
\flushleft{\begin{malayalam}
ഇക്കൂട്ടര്‍ പൂജിച്ചുകൊണ്ടിരിക്കുന്നവയെ സംബന്ധിച്ച് നിനക്കൊരിക്കലും സംശയം വേണ്ടാ. മുമ്പ് ഇവരുടെ പിതാക്കന്മാര്‍ പൂജിച്ചിരുന്നപോലെത്തന്നെയാണ് ഇന്നിവരും പൂജ നടത്തുന്നത്. ഇവരുടെ വിഹിതം ഒട്ടും കുറവുവരുത്താതെ നാമവര്‍ക്ക് അവരുടെ ശിക്ഷ നല്‍കുന്നതാണ്.
\end{malayalam}}
\flushright{\begin{Arabic}
\quranayah[11][110]
\end{Arabic}}
\flushleft{\begin{malayalam}
മൂസാക്കു നാം വേദഗ്രന്ഥം നല്‍കി. അപ്പോഴതിലും അഭിപ്രായവ്യത്യാസങ്ങളുണ്ടായി. നിന്റെ നാഥനില്‍ നിന്ന് നേരത്തെ തീരുമാന പ്രഖ്യാപനം ഉണ്ടായിട്ടില്ലായിരുന്നുവെങ്കില്‍ അവര്‍ക്കിടയില്‍ അക്കാര്യത്തില്‍ ഇപ്പോള്‍ തന്നെ വിധി കല്‍പിക്കുമായിരുന്നു. തീര്‍ച്ചയായും അവരിക്കാര്യത്തില്‍ ആശങ്കാകുലമായ സംശയത്തിലാണ്.
\end{malayalam}}
\flushright{\begin{Arabic}
\quranayah[11][111]
\end{Arabic}}
\flushleft{\begin{malayalam}
അവരില്‍ ഓരോരുത്തര്‍ക്കും നിന്റെ നാഥന്‍ അവരുടെ പ്രവര്‍ത്തനത്തിന്റെ ഫലം പൂര്‍ണമായി നല്‍കുക തന്നെ ചെയ്യും. നിശ്ചയമായും അവര്‍ പ്രവര്‍ത്തിക്കുന്നതിനെപ്പറ്റി നന്നായറിയുന്നവനാണവന്‍.
\end{malayalam}}
\flushright{\begin{Arabic}
\quranayah[11][112]
\end{Arabic}}
\flushleft{\begin{malayalam}
നിന്നോടു കല്‍പിച്ചവിധം നീയും നിന്നോടൊപ്പം പശ്ചാത്തപിച്ചു മടങ്ങിയവരും നേര്‍വഴിയില്‍ ഉറച്ചു നില്‍ക്കുക. നിങ്ങള്‍ പരിധി ലംഘിക്കരുത്. തീര്‍ച്ചയായും നിങ്ങള്‍ ചെയ്യുന്നത് സൂക്ഷ്മമായി കാണുന്നവനാണവന്‍.
\end{malayalam}}
\flushright{\begin{Arabic}
\quranayah[11][113]
\end{Arabic}}
\flushleft{\begin{malayalam}
അതിക്രമം കാണിച്ചവരുടെ ഭാഗത്തേക്ക് നിങ്ങള്‍ ചായരുത്. അങ്ങനെ ചെയ്താല്‍ നരകം നിങ്ങളെ പിടികൂടും. അല്ലാഹുവിനെ കൂടാതെ നിങ്ങള്‍ക്ക് രക്ഷകരായി ആരുമില്ല. പിന്നീട് നിങ്ങള്‍ക്കൊരു സഹായവും ലഭിക്കുകയുമില്ല.
\end{malayalam}}
\flushright{\begin{Arabic}
\quranayah[11][114]
\end{Arabic}}
\flushleft{\begin{malayalam}
പകലിന്റെ രണ്ടറ്റങ്ങളിലും രാവിന്റെ ആദ്യയാമത്തിലും നീ നമസ്കാരം നിഷ്ഠയോടെ നിര്‍വഹിക്കുക. തീര്‍ച്ചയായും, സദ്വൃത്തികള്‍ ദുര്‍വൃത്തികളെ ദൂരീകരിക്കും. ആലോചിച്ചറിയുന്നവര്‍ക്കുള്ള ഉദ്ബോധനമാണിത്.
\end{malayalam}}
\flushright{\begin{Arabic}
\quranayah[11][115]
\end{Arabic}}
\flushleft{\begin{malayalam}
ക്ഷമിക്കുക. സല്‍ക്കര്‍മികള്‍ക്കുള്ള പ്രതിഫലം അല്ലാഹു ഒട്ടും നഷ്ടപ്പെടുത്തുകയില്ല; ഉറപ്പ്.
\end{malayalam}}
\flushright{\begin{Arabic}
\quranayah[11][116]
\end{Arabic}}
\flushleft{\begin{malayalam}
നിങ്ങള്‍ക്കു മുമ്പ് കഴിഞ്ഞുപോയ തലമുറകളില്‍ ഭൂമിയില്‍ കുഴപ്പമുണ്ടാക്കുന്നത് തടയുന്ന ഉത്തമ പാരമ്പര്യമുള്ള ഒരു വിഭാഗം ഉണ്ടാവാതിരുന്നതെന്തുകൊണ്ട്? അവരില്‍ നിന്നും നാം രക്ഷപ്പെടുത്തിയ വളരെ കുറച്ചുപേരൊഴികെ. അക്രമികള്‍ തങ്ങള്‍ക്കു കിട്ടിയ സുഖസൌകര്യങ്ങളുടെ പിറകെ പോവുകയാണുണ്ടായത്. അവര്‍ കുറ്റവാളികളായിരുന്നു.
\end{malayalam}}
\flushright{\begin{Arabic}
\quranayah[11][117]
\end{Arabic}}
\flushleft{\begin{malayalam}
നാട്ടുകാര്‍ സല്‍കൃത്യങ്ങള്‍ ചെയ്യുന്നവരായിരിക്കെ അല്ലാഹു അക്രമമായി ആ നാടുകളെ നശിപ്പിക്കുകയില്ല.
\end{malayalam}}
\flushright{\begin{Arabic}
\quranayah[11][118]
\end{Arabic}}
\flushleft{\begin{malayalam}
നിന്റെ നാഥന്‍ ഇച്ഛിച്ചിരുന്നുവെങ്കില്‍ അവന്‍ മുഴുവന്‍ മനുഷ്യരെയും ഒരൊറ്റ സമുദായമാക്കുമായിരുന്നു. എന്നാല്‍ അവര്‍ ഭിന്നിച്ചുകൊണ്ടേയിരിക്കും.
\end{malayalam}}
\flushright{\begin{Arabic}
\quranayah[11][119]
\end{Arabic}}
\flushleft{\begin{malayalam}
നിന്റെ നാഥന്‍ അനുഗ്രഹിച്ചവരൊഴികെ. അതിനുവേണ്ടിയാണ് അവനവരെ സൃഷ്ടിച്ചത്. “ജിന്നുവര്‍ഗത്തിലും മനുഷ്യവര്‍ഗത്തിലും പെട്ടവരെക്കൊണ്ട് നാം നരകത്തെ നിറക്കുക തന്നെ ചെയ്യു”മെന്ന നിന്റെ നാഥന്റെ പ്രഖ്യാപനം യാഥാര്‍ഥ്യമായിരിക്കുന്നു.
\end{malayalam}}
\flushright{\begin{Arabic}
\quranayah[11][120]
\end{Arabic}}
\flushleft{\begin{malayalam}
ദൈവദൂതന്മാരുടെ വാര്‍ത്തകളില്‍നിന്ന് നിന്റെ മനസ്സിന് ദൃഢത നല്‍കുന്നതെല്ലാം നിനക്കു നാം പറഞ്ഞുതരുന്നു. ഇതിലൂടെ യഥാര്‍ഥ ജ്ഞാനവും സത്യവിശ്വാസികള്‍ക്കുള്ള സദുപദേശവും ഉദ്ബോധനവും നിനക്ക് വന്നെത്തിയിരിക്കുന്നു.
\end{malayalam}}
\flushright{\begin{Arabic}
\quranayah[11][121]
\end{Arabic}}
\flushleft{\begin{malayalam}
വിശ്വസിക്കാത്തവരോട് പറയുക: നിങ്ങള്‍ നിങ്ങളുടെ നിലപാടനുസരിച്ച് പ്രവര്‍ത്തിക്കുക. ഞങ്ങളും പ്രവര്‍ത്തിക്കുന്നതാണ്.
\end{malayalam}}
\flushright{\begin{Arabic}
\quranayah[11][122]
\end{Arabic}}
\flushleft{\begin{malayalam}
നിങ്ങള്‍ കാത്തിരിക്കുക. ഉറപ്പായും ഞങ്ങളും കാത്തിരിക്കാം.
\end{malayalam}}
\flushright{\begin{Arabic}
\quranayah[11][123]
\end{Arabic}}
\flushleft{\begin{malayalam}
ആകാശഭൂമികളില്‍ മറഞ്ഞിരിക്കുന്നതൊക്കെയും അല്ലാഹുവിനുള്ളതാണ്. അവസാനം എല്ലാം മടങ്ങിയെത്തുന്നതും അവങ്കലേക്കുതന്നെ. അതിനാല്‍ നീ അവനുമാത്രം വഴിപ്പെടുക. അവനില്‍ ഭരമേല്‍പിക്കുക. നിങ്ങള്‍ ചെയ്യുന്നതിനെപ്പറ്റിയൊന്നും നിന്റെ നാഥന്‍ ഒട്ടും അശ്രദ്ധനല്ല.
\end{malayalam}}
\chapter{\textmalayalam{യൂസുഫ്}}
\begin{Arabic}
\Huge{\centerline{\basmalah}}\end{Arabic}
\flushright{\begin{Arabic}
\quranayah[12][1]
\end{Arabic}}
\flushleft{\begin{malayalam}
അലിഫ്-ലാം-റാഅ്. സുവ്യക്തമായ വേദപുസ്തകത്തിലെ വചനങ്ങളാണിവ.
\end{malayalam}}
\flushright{\begin{Arabic}
\quranayah[12][2]
\end{Arabic}}
\flushleft{\begin{malayalam}
നാമിതിനെ അറബി ഭാഷയില്‍ വായനക്കുള്ള പുസ്തകമായി ഇറക്കിയിരിക്കുന്നു. നിങ്ങള്‍ നന്നായി ചിന്തിച്ചു മനസ്സിലാക്കാന്‍.
\end{malayalam}}
\flushright{\begin{Arabic}
\quranayah[12][3]
\end{Arabic}}
\flushleft{\begin{malayalam}
ഈ ഖുര്‍ആന്‍ ബോധനമായി നല്‍കിയതിലൂടെ നാം നിനക്ക് നല്ല ചരിത്രകഥകള്‍ വിവരിച്ചു തരികയാണ്. ഇതിനുമുമ്പ് നീ ഇതൊന്നുമറിയാത്തവരുടെ കൂട്ടത്തിലായിരുന്നു.
\end{malayalam}}
\flushright{\begin{Arabic}
\quranayah[12][4]
\end{Arabic}}
\flushleft{\begin{malayalam}
യൂസുഫ് തന്റെ പിതാവിനോട് പറഞ്ഞ സന്ദര്‍ഭം: "പ്രിയ പിതാവേ, പതിനൊന്ന് നക്ഷത്രങ്ങളും സൂര്യനും ചന്ദ്രനും എനിക്കു സാഷ്ടാംഗം ചെയ്യുന്നതായി ഞാന്‍ സ്വപ്നം കണ്ടിരിക്കുന്നു.”
\end{malayalam}}
\flushright{\begin{Arabic}
\quranayah[12][5]
\end{Arabic}}
\flushleft{\begin{malayalam}
പിതാവു പറഞ്ഞു: "മോനേ, ഈ സ്വപ്നത്തെപ്പറ്റി ഒരിക്കലും നിന്റെ സഹോദരന്മാരോട് പറയരുത്. അവര്‍ നിനക്കെതിരെ ഗൂഢതന്ത്രം പ്രയോഗിച്ചേക്കും. പിശാച് മനുഷ്യന്റെ പ്രത്യക്ഷ ശത്രുവാണ്.”
\end{malayalam}}
\flushright{\begin{Arabic}
\quranayah[12][6]
\end{Arabic}}
\flushleft{\begin{malayalam}
അവ്വിധം നിന്റെ നാഥന്‍ നിന്നെ തെരഞ്ഞെടുക്കും. നിന്നെ അവന്‍ സ്വപ്ന വ്യാഖ്യാനം പഠിപ്പിക്കും. നിനക്കും യഅ്ഖൂബ് കുടുംബത്തിനും അവന്റെ അനുഗ്രഹങ്ങള്‍ പൂര്‍ത്തീകരിച്ചു തരും; നിന്റെ രണ്ടു പൂര്‍വപിതാക്കളായ ഇബ്റാഹീമിനും ഇസ്ഹാഖിനും അത് പൂര്‍ത്തീകരിച്ചു കൊടുത്തപോലെ. തീര്‍ച്ചയായും നിന്റെ നാഥന്‍ എല്ലാം അറിയുന്നവനും യുക്തിമാനുമാണ്.
\end{malayalam}}
\flushright{\begin{Arabic}
\quranayah[12][7]
\end{Arabic}}
\flushleft{\begin{malayalam}
ഉറപ്പായും യൂസുഫിലും അദ്ദേഹത്തിന്റെ സഹോദരന്മാരിലും അന്വേഷിച്ചറിയുന്നവര്‍ക്ക് നിരവധി തെളിവുകളുണ്ട്.
\end{malayalam}}
\flushright{\begin{Arabic}
\quranayah[12][8]
\end{Arabic}}
\flushleft{\begin{malayalam}
അവര്‍ പറഞ്ഞ സന്ദര്‍ഭം: "യൂസുഫും അവന്റെ സഹോദരനുമാണ് നമ്മെക്കാള്‍ പിതാവിന് പ്രിയപ്പെട്ടവര്‍. നാം വലിയൊരു സംഘമായിരുന്നിട്ടും. നമ്മുടെ പിതാവ് വ്യക്തമായ വഴികേടില്‍തന്നെ.
\end{malayalam}}
\flushright{\begin{Arabic}
\quranayah[12][9]
\end{Arabic}}
\flushleft{\begin{malayalam}
"നിങ്ങള്‍ യൂസുഫിനെ കൊന്നുകളയുക. അല്ലെങ്കില്‍ ഏതെങ്കിലും ഒരിടത്ത് കൊണ്ടുപോയി തള്ളുക. അതോടെ പിതാവിന്റെ അടുപ്പം നിങ്ങള്‍ക്കു മാത്രമായി കിട്ടും. അതിനു ശേഷം നിങ്ങള്‍ക്ക് നല്ലവരായിത്തീരുകയും ചെയ്യാം.”
\end{malayalam}}
\flushright{\begin{Arabic}
\quranayah[12][10]
\end{Arabic}}
\flushleft{\begin{malayalam}
അപ്പോള്‍ അവരിലൊരാള്‍ പറഞ്ഞു: "യൂസുഫിനെ കൊല്ലരുത്. നിങ്ങള്‍ക്ക് വല്ലതും ചെയ്യണമെന്നുണ്ടെങ്കില്‍ അവനെ കിണറിന്റെ ആഴത്തിലെറിയുക. വല്ല യാത്രാസംഘവും അവനെ കണ്ടെടുത്തുകൊള്ളും.”
\end{malayalam}}
\flushright{\begin{Arabic}
\quranayah[12][11]
\end{Arabic}}
\flushleft{\begin{malayalam}
അവര്‍ പറഞ്ഞു: "ഞങ്ങളുടെ പിതാവേ, അങ്ങക്കെന്തുപറ്റി? യൂസുഫിന്റെ കാര്യത്തില്‍ അങ്ങു ഞങ്ങളെ വിശ്വസിക്കാത്തതെന്ത്? തീര്‍ച്ചയായും ഞങ്ങള്‍ അവന്റെ ഗുണകാംക്ഷികളാണ്.
\end{malayalam}}
\flushright{\begin{Arabic}
\quranayah[12][12]
\end{Arabic}}
\flushleft{\begin{malayalam}
"നാളെ അവനെ ഞങ്ങളോടൊപ്പമയച്ചാലും. അവന്‍ തിന്നുരസിച്ചുല്ലസിക്കട്ടെ. ഉറപ്പായും ഞങ്ങളവനെ കാത്തുരക്ഷിച്ചുകൊള്ളും.”
\end{malayalam}}
\flushright{\begin{Arabic}
\quranayah[12][13]
\end{Arabic}}
\flushleft{\begin{malayalam}
പിതാവ് പറഞ്ഞു: "നിങ്ങളവനെ കൊണ്ടുപോകുന്നത് എന്നെ ദുഃഖിതനാക്കും. അവനെ ചെന്നായ തിന്നുമോ എന്നാണെന്റെ പേടി. നിങ്ങള്‍ അവനെ ശ്രദ്ധിക്കാതെ പോയേക്കുമെന്നും.”
\end{malayalam}}
\flushright{\begin{Arabic}
\quranayah[12][14]
\end{Arabic}}
\flushleft{\begin{malayalam}
അവര്‍ പറഞ്ഞു: "ഞങ്ങള്‍ വലിയ ഒരു സംഘമുണ്ടായിരിക്കെ അവനെ ചെന്നായ തിന്നുകയാണെങ്കില്‍ ഞങ്ങള്‍ കൊടിയ നഷ്ടം പറ്റിയവരായിരിക്കും; തീര്‍ച്ച.”
\end{malayalam}}
\flushright{\begin{Arabic}
\quranayah[12][15]
\end{Arabic}}
\flushleft{\begin{malayalam}
അങ്ങനെ അവരവനെ കൊണ്ടുപോയി. കിണറ്റിന്റെ ആഴത്തില്‍ തള്ളാന്‍ കൂട്ടായി തീരുമാനിച്ചു. അപ്പോള്‍ നാം അവന് ബോധനം നല്‍കി: അവരുടെ ഈ ചെയ്തിയെക്കുറിച്ച് നീ അവര്‍ക്ക് വഴിയെ വിവരിച്ചു കൊടുക്കുകതന്നെ ചെയ്യും. അവര്‍ അന്നേരം അതേക്കുറിച്ച് ഒട്ടും ബോധവാന്മാരായിരിക്കുകയില്ല.
\end{malayalam}}
\flushright{\begin{Arabic}
\quranayah[12][16]
\end{Arabic}}
\flushleft{\begin{malayalam}
സന്ധ്യാസമയത്ത് അവര്‍ തങ്ങളുടെ പിതാവിന്റെ അടുത്ത് കരഞ്ഞുകൊണ്ടു വന്നു.
\end{malayalam}}
\flushright{\begin{Arabic}
\quranayah[12][17]
\end{Arabic}}
\flushleft{\begin{malayalam}
അവര്‍ പറഞ്ഞു: "ഞങ്ങളുടെ ഉപ്പാ, യൂസുഫിനെ ഞങ്ങളുടെ സാധനങ്ങള്‍ക്കരികെ നിര്‍ത്തി ഞങ്ങള്‍ മല്‍സരിക്കാന്‍ പോയതായിരുന്നു. അപ്പോള്‍ അവനെ ഒരു ചെന്നായ തിന്നുകളഞ്ഞു. അങ്ങ് ഞങ്ങളെ വിശ്വസിക്കുകയില്ല. ഞങ്ങള്‍ എത്ര സത്യം പറയുന്നവരായാലും.
\end{malayalam}}
\flushright{\begin{Arabic}
\quranayah[12][18]
\end{Arabic}}
\flushleft{\begin{malayalam}
യൂസുഫിന്റെ കുപ്പായത്തില്‍ കള്ളച്ചോര പുരട്ടിയാണവര്‍ വന്നത്. പിതാവ് പറഞ്ഞു: "നിങ്ങളുടെ മനസ്സ് ഒരു കാര്യം ചെയ്യാന്‍ നിങ്ങളെ പ്രേരിപ്പിച്ചു. ഇനി നന്നായി ക്ഷമിക്കുകതന്നെ. നിങ്ങള്‍ പറഞ്ഞ കാര്യത്തിന്റെ നിജസ്ഥിതി അറിയുന്നതിലെന്നെ സഹായിക്കാനുള്ളത് അല്ലാഹു മാത്രം.”
\end{malayalam}}
\flushright{\begin{Arabic}
\quranayah[12][19]
\end{Arabic}}
\flushleft{\begin{malayalam}
ഒരു യാത്രാസംഘം വന്നു. അവര്‍ തങ്ങളുടെ വെള്ളം കോരിയെ അയച്ചു. അയാള്‍ തന്റെ തൊട്ടി ഇറക്കി. അയാള്‍ പറഞ്ഞു: "ഹാ, എന്തൊരദ്ഭുതം! ഇതാ ഒരു കുട്ടി?” അവര്‍ ആ കുട്ടിയെ ഒരു കച്ചവടച്ചരക്കാക്കി ഒളിപ്പിച്ചുവെച്ചു. അവര്‍ ചെയ്തുകൊണ്ടിരുന്നതിനെപ്പറ്റി നന്നായറിയുന്നവനാണ് അല്ലാഹു.
\end{malayalam}}
\flushright{\begin{Arabic}
\quranayah[12][20]
\end{Arabic}}
\flushleft{\begin{malayalam}
അവരവനെ കുറഞ്ഞ വിലയ്ക്ക് വിറ്റു. ഏതാനും നാണയത്തുട്ടുകള്‍ക്ക്. അവനില്‍ ഒട്ടും താല്‍പര്യമില്ലാത്തവരായിരുന്നു അവര്‍.
\end{malayalam}}
\flushright{\begin{Arabic}
\quranayah[12][21]
\end{Arabic}}
\flushleft{\begin{malayalam}
ഈജിപ്തില്‍ നിന്ന് അവനെ വാങ്ങിയവന്‍ തന്റെ പത്നിയോടു പറഞ്ഞു: "ഇവനെ നല്ല നിലയില്‍ പോറ്റി വളര്‍ത്തുക. ഇവന്‍ നമുക്കുപകരിച്ചേക്കാം. അല്ലെങ്കില്‍ നമുക്കിവനെ നമ്മുടെ മകനായി കണക്കാക്കാം.” അങ്ങനെ യൂസുഫിന് നാം അന്നാട്ടില്‍ സൌകര്യമൊരുക്കിക്കൊടുത്തു. സ്വപ്നവ്യാഖ്യാനം അവനെ പഠിപ്പിക്കാന്‍ കൂടിയാണത്. അല്ലാഹു തന്റെ തീരുമാനം കൃത്യമായി നടത്തുക തന്നെ ചെയ്യും. എങ്കിലും മനുഷ്യരിലേറെപ്പേരും അതറിയുന്നില്ല.
\end{malayalam}}
\flushright{\begin{Arabic}
\quranayah[12][22]
\end{Arabic}}
\flushleft{\begin{malayalam}
അവന്‍ പൂര്‍ണവളര്‍ച്ച പ്രാപിച്ചപ്പോള്‍ നാമവന് തീരുമാനശക്തിയും അറിവും നല്‍കി. അങ്ങനെയാണ് നാം സച്ചരിതര്‍ക്ക് പ്രതിഫലം നല്‍കുന്നത്.
\end{malayalam}}
\flushright{\begin{Arabic}
\quranayah[12][23]
\end{Arabic}}
\flushleft{\begin{malayalam}
യൂസുഫ് പാര്‍ക്കുന്ന പുരയിലെ പെണ്ണ് അയാളെ വശീകരിക്കാന്‍ ശ്രമിച്ചു. വാതിലുകളടച്ച് അവള്‍ പറഞ്ഞു: "വരൂ.” അവന്‍ പറഞ്ഞു: "അല്ലാഹു ശരണം; അവനാണെന്റെ നാഥന്‍. അവനെനിക്കു നല്ല സ്ഥാനം നല്‍കിയിരിക്കുന്നു. അതിക്രമികള്‍ ഒരിക്കലും വിജയിക്കുകയില്ല.”
\end{malayalam}}
\flushright{\begin{Arabic}
\quranayah[12][24]
\end{Arabic}}
\flushleft{\begin{malayalam}
അവള്‍ അദ്ദേഹത്തെ കാമിച്ചു. തന്റെ നാഥന്റെ പ്രമാണം കണ്ടിരുന്നില്ലെങ്കില്‍ അദ്ദേഹം അവളെയും കാമിക്കുമായിരുന്നു. അവ്വിധം സംഭവിച്ചത് തിന്മയും നീചകൃത്യവും നാം അദ്ദേഹത്തില്‍ നിന്ന് തട്ടിമാറ്റാനാണ്. തീര്‍ച്ചയായും അദ്ദേഹം നമ്മുടെ തെരഞ്ഞെടുക്കപ്പെട്ട ദാസന്മാരില്‍ പെട്ടവനത്രെ.
\end{malayalam}}
\flushright{\begin{Arabic}
\quranayah[12][25]
\end{Arabic}}
\flushleft{\begin{malayalam}
അവരിരുവരും വാതില്‍ക്കലേക്കോടി. അവള്‍ പിന്നില്‍ നിന്ന് അദ്ദേഹത്തിന്റെ കുപ്പായം വലിച്ചുകീറി. വാതില്‍ക്കല്‍ അവളുടെ ഭര്‍ത്താവിനെ ഇരുവരും കണ്ടുമുട്ടി. അവള്‍ പറഞ്ഞു: "നിങ്ങളുടെ ഭാര്യയുടെ നേരെ അരുതായ്മ ആഗ്രഹിച്ച ഇയാള്‍ക്കുള്ള ശിക്ഷയെന്താണ്? ഒന്നുകിലവനെ തടവിലിടണം. അല്ലെങ്കില്‍ നോവേറിയ മറ്റെന്തെങ്കിലും ശിക്ഷ നല്‍കണം.”
\end{malayalam}}
\flushright{\begin{Arabic}
\quranayah[12][26]
\end{Arabic}}
\flushleft{\begin{malayalam}
യൂസുഫ് പറഞ്ഞു: "അവളാണെന്നെ വശീകരിക്കാന്‍ ശ്രമിച്ചത്.” ആ സ്ത്രീയുടെ ബന്ധുവായ ഒരു സാക്ഷി തെളിവുന്നയിച്ചു: അവന്റെ കുപ്പായം മുന്‍വശത്താണ് കീറിയതെങ്കില്‍ അവള്‍ പറഞ്ഞത് സത്യമാണ്. അവന്‍ കള്ളം പറഞ്ഞവനും.
\end{malayalam}}
\flushright{\begin{Arabic}
\quranayah[12][27]
\end{Arabic}}
\flushleft{\begin{malayalam}
"അഥവാ, അവന്റെ കുപ്പായം പിന്‍വശത്താണ് കീറിയതെങ്കില്‍ അവള്‍ പറഞ്ഞത് കള്ളമാണ്. അവന്‍ സത്യം പറഞ്ഞവനും.”
\end{malayalam}}
\flushright{\begin{Arabic}
\quranayah[12][28]
\end{Arabic}}
\flushleft{\begin{malayalam}
യൂസുഫിന്റെ കുപ്പായം പിന്‍ഭാഗം കീറിയതായി കണ്ടപ്പോള്‍ ഭര്‍ത്താവ് പറഞ്ഞു: "ഇത് നിങ്ങള്‍ സ്ത്രീകളുടെ കുതന്ത്രത്തില്‍പ്പെട്ടതാണ്. നിങ്ങളുടെ കുതന്ത്രം ഭയങ്കരം തന്നെ.
\end{malayalam}}
\flushright{\begin{Arabic}
\quranayah[12][29]
\end{Arabic}}
\flushleft{\begin{malayalam}
"യൂസുഫ്, നീയിത് അവഗണിച്ചേക്കുക.” സ്ത്രീയോട്: "നീ നിന്റെ തെറ്റിന് മാപ്പിരക്കുക. തീര്‍ച്ചയായും നീയാണ് തെറ്റുകാരി.”
\end{malayalam}}
\flushright{\begin{Arabic}
\quranayah[12][30]
\end{Arabic}}
\flushleft{\begin{malayalam}
പട്ടണത്തിലെ പെണ്ണുങ്ങള്‍ പറഞ്ഞു: "പ്രഭുവിന്റെ പത്നി തന്റെ വേലക്കാരനെ വശീകരിക്കാന്‍ നോക്കുകയാണ്. കാമം അവളുടെ മനസ്സിനെ കീഴടക്കിയിരിക്കുന്നു. നമ്മുടെ വീക്ഷണത്തില്‍ അവള്‍ വ്യക്തമായ വഴികേടിലാണ്.”
\end{malayalam}}
\flushright{\begin{Arabic}
\quranayah[12][31]
\end{Arabic}}
\flushleft{\begin{malayalam}
അവരുടെ തന്ത്രത്തെപ്പറ്റി കേട്ട പ്രഭുപത്നി അവരുടെ അടുത്തേക്ക് ആളെ അയച്ചു. അവര്‍ക്ക് ചാരിയിരിക്കാന്‍ അവള്‍ ഇരിപ്പിടങ്ങളൊരുക്കി. അവരിലോരോരുത്തര്‍ക്കും ഓരോ കത്തി കൊടുക്കുകയും ചെയ്തു. അവള്‍ യൂസുഫിനോടു പറഞ്ഞു: "ആ സ്ത്രീകളുടെ മുന്നിലേക്ക് ചെല്ലുക.” അവര്‍ അദ്ദേഹത്തെ കണ്ടപ്പോള്‍ വിസ്മയഭരിതരാവുകയും തങ്ങളുടെ കൈകള്‍ സ്വയം മുറിപ്പെടുത്തുകയും ചെയ്തു. അവര്‍ പറഞ്ഞുപോയി: "അല്ലാഹു എത്ര മഹാന്‍! ഇത് മനുഷ്യനല്ല. ഇത് മാന്യനായ ഒരു മലക്കല്ലാതാരുമല്ല”
\end{malayalam}}
\flushright{\begin{Arabic}
\quranayah[12][32]
\end{Arabic}}
\flushleft{\begin{malayalam}
പ്രഭുപത്നി പറഞ്ഞു: "ഇദ്ദേഹത്തിന്റെ കാര്യത്തിലാണ് നിങ്ങളെന്നെ ആക്ഷേപിച്ചുകൊണ്ടിരിക്കുന്നത്. തീര്‍ച്ചയായും ഞാനിദ്ദേഹത്തെ വശപ്പെടുത്താന്‍ ശ്രമിച്ചിട്ടുണ്ട്. എന്നാല്‍ ഇദ്ദേഹം വഴങ്ങിയില്ല. ഞാന്‍ കല്‍പിക്കുംവിധം ചെയ്തില്ലെങ്കില്‍ ഉറപ്പായും ഞാനിവനെ ജയിലിലടക്കും. അങ്ങനെ ഇവന്‍ നിന്ദ്യനായിത്തീരും.”
\end{malayalam}}
\flushright{\begin{Arabic}
\quranayah[12][33]
\end{Arabic}}
\flushleft{\begin{malayalam}
യൂസുഫ് പറഞ്ഞു: "എന്റെ നാഥാ, ഇവരെന്നെ ക്ഷണിക്കുന്നത് ഏതൊന്നിലേക്കാണോ അതിനേക്കാള്‍ എനിക്കിഷ്ടം തടവറയാണ്. ഇവരുടെ കുതന്ത്രം നീയെന്നില്‍ നിന്ന് തട്ടിമാറ്റുന്നില്ലെങ്കില്‍ ഞാന്‍ അവരുടെ കെണിയില്‍ കുടുങ്ങി അവിവേകികളില്‍പ്പെട്ടവനായേക്കാം.”
\end{malayalam}}
\flushright{\begin{Arabic}
\quranayah[12][34]
\end{Arabic}}
\flushleft{\begin{malayalam}
അദ്ദേഹത്തിന്റെ പ്രാര്‍ഥന നാഥന്‍ സ്വീകരിച്ചു. അദ്ദേഹത്തില്‍നിന്ന് അവരുടെ കുതന്ത്രത്തെ അവന്‍ തട്ടിമാറ്റി. അല്ലാഹു എല്ലാം കേള്‍ക്കുന്നവനും അറിയുന്നവനുമാണ്.
\end{malayalam}}
\flushright{\begin{Arabic}
\quranayah[12][35]
\end{Arabic}}
\flushleft{\begin{malayalam}
പിന്നീട് യൂസുഫിന്റെ നിരപരാധിത്വത്തിന്റെ തെളിവുകള്‍ കണ്ടറിഞ്ഞ ശേഷവും അദ്ദേഹത്തെ നിശ്ചിത അവധിവരെ ജയിലിലടക്കണമെന്ന് അവര്‍ക്ക് തോന്നി.
\end{malayalam}}
\flushright{\begin{Arabic}
\quranayah[12][36]
\end{Arabic}}
\flushleft{\begin{malayalam}
അദ്ദേഹത്തോടൊപ്പം മറ്റു രണ്ടു ചെറുപ്പക്കാരും ജയിലിലകപ്പെട്ടു. അവരിലൊരാള്‍ പറഞ്ഞു: "ഞാന്‍ മദ്യം പിഴിഞ്ഞെടുക്കുന്നതായി സ്വപ്നം കണ്ടിരിക്കുന്നു.” മറ്റെയാള്‍ പറഞ്ഞു: "ഞാനെന്റെ തലയില്‍ റൊട്ടി ചുമക്കുന്നതായും പക്ഷികള്‍ അതില്‍ നിന്ന് തിന്നുന്നതായും സ്വപ്നം കണ്ടിരിക്കുന്നു. ഞങ്ങള്‍ക്ക് ഇതിന്റെ വ്യാഖ്യാനം പറഞ്ഞുതരിക. താങ്കളെ നല്ല ഒരാളായാണ് ഞങ്ങള്‍ കാണുന്നത്.”
\end{malayalam}}
\flushright{\begin{Arabic}
\quranayah[12][37]
\end{Arabic}}
\flushleft{\begin{malayalam}
യൂസുഫ് പറഞ്ഞു: "നിങ്ങള്‍ക്ക് തിന്നാനുള്ള അന്നം വന്നെത്തും മുമ്പെ ഞാനതിന്റെ പൊരുള്‍ നിങ്ങള്‍ക്ക് വിവരിച്ചു തരാതിരിക്കില്ല. എനിക്കെന്റെ നാഥന്‍ പഠിപ്പിച്ചുതന്നവയില്‍പ്പെട്ടതാണത്. അല്ലാഹുവില്‍ വിശ്വസിക്കാത്തവരും പരലോകത്തെ നിഷേധിക്കുന്നവരുമായ ഈ ജനത്തിന്റെ മാര്‍ഗം ഞാന്‍ കൈവെടിഞ്ഞിരിക്കുന്നു.
\end{malayalam}}
\flushright{\begin{Arabic}
\quranayah[12][38]
\end{Arabic}}
\flushleft{\begin{malayalam}
"എന്റെ പിതാക്കളായ ഇബ്റാഹീമിന്റെയും ഇസ്ഹാഖിന്റെയും യഅ്ഖൂബിന്റെയും മാര്‍ഗമാണ് ഞാന്‍ പിന്‍പറ്റുന്നത്. അല്ലാഹുവില്‍ ഒന്നിനെയും പങ്കുചേര്‍ക്കാന്‍ നമുക്ക് അനുവാദമില്ല. അല്ലാഹു ഞങ്ങള്‍ക്കും മറ്റു മുഴുവന്‍ മനുഷ്യര്‍ക്കും നല്‍കിയ അനുഗ്രഹങ്ങളില്‍പ്പെട്ടതാണിത്. എങ്കിലും മനുഷ്യരിലേറെപ്പേരും നന്ദി കാണിക്കുന്നില്ല.
\end{malayalam}}
\flushright{\begin{Arabic}
\quranayah[12][39]
\end{Arabic}}
\flushleft{\begin{malayalam}
"എന്റെ ജയില്‍ക്കൂട്ടുകാരേ, വ്യത്യസ്തരായ പല പല ദൈവങ്ങളാണോ ഉത്തമം? അതോ സര്‍വാധിനാഥനും ഏകനുമായ അല്ലാഹുവോ?
\end{malayalam}}
\flushright{\begin{Arabic}
\quranayah[12][40]
\end{Arabic}}
\flushleft{\begin{malayalam}
"അവനെക്കൂടാതെ നിങ്ങള്‍ പൂജിച്ചുകൊണ്ടിരിക്കുന്നവയൊക്കെയും നിങ്ങളും നിങ്ങളുടെ പൂര്‍വപിതാക്കളും വ്യാജമായി പടച്ചുണ്ടാക്കിയ ചില പേരുകളല്ലാതൊന്നുമല്ല. അല്ലാഹു അതിനൊന്നിനും ഒരു പ്രമാണവും ഇറക്കിത്തന്നിട്ടില്ല. വിധിക്കധികാരം അല്ലാഹുവിന് മാത്രമാണ്. അവനെയല്ലാതെ യാതൊന്നിനെയും നിങ്ങള്‍ വഴങ്ങരുതെന്ന് അവനാജ്ഞാപിച്ചിരിക്കുന്നു. ഏറ്റം ശരിയായ ജീവിതക്രമം അതാണ്. എങ്കിലും ഏറെ മനുഷ്യരും അതറിയുന്നില്ല.
\end{malayalam}}
\flushright{\begin{Arabic}
\quranayah[12][41]
\end{Arabic}}
\flushleft{\begin{malayalam}
"എന്റെ ജയില്‍ക്കൂട്ടുകാരേ, നിങ്ങളിലൊരാള്‍ തന്റെ യജമാനന് മദ്യം വിളമ്പിക്കൊണ്ടിരിക്കും. മറ്റയാള്‍ കുരിശിലേറ്റപ്പെടും. അങ്ങനെ അയാളുടെ തലയില്‍ നിന്ന് പക്ഷികള്‍ കൊത്തിത്തിന്നും. നിങ്ങളിരുവരും വിധി തേടിയ കാര്യം തീരുമാനിക്കപ്പെട്ടുകഴിഞ്ഞിരിക്കുന്നു.”
\end{malayalam}}
\flushright{\begin{Arabic}
\quranayah[12][42]
\end{Arabic}}
\flushleft{\begin{malayalam}
അവരിരുവരില്‍ രക്ഷപ്പെടുമെന്ന് താന്‍ കരുതിയ ആളോട് യൂസുഫ് പറഞ്ഞു: "നീ നിന്റെ യജമാനനോട് എന്നെപ്പറ്റി പറയുക.” എങ്കിലും യജമാനനോട് അതേക്കുറിച്ച് പറയുന്ന കാര്യം പിശാച് അയാളെ മറപ്പിച്ചു. അതിനാല്‍ യൂസുഫ് ഏതാനും കൊല്ലം ജയിലില്‍ കഴിഞ്ഞു.
\end{malayalam}}
\flushright{\begin{Arabic}
\quranayah[12][43]
\end{Arabic}}
\flushleft{\begin{malayalam}
ഒരിക്കല്‍ രാജാവ് പറഞ്ഞു: "ഞാനൊരു സ്വപ്നം കണ്ടിരിക്കുന്നു; ഏഴു തടിച്ചു കൊഴുത്ത പശുക്കള്‍. അവയെ ഏഴു മെലിഞ്ഞ പശുക്കള്‍ തിന്നുകൊണ്ടിരിക്കുന്നു. അതോടൊപ്പം ഏഴു പച്ചക്കതിരുകളും ഏഴു ഉണങ്ങിയ കതിരുകളും. അതിനാല്‍ വിദ്വാന്മാരേ, എന്റെ ഈ സ്വപ്നത്തിന്റെ പൊരുള്‍ എനിക്ക് പറഞ്ഞുതരിക. നിങ്ങള്‍ സ്വപ്നവ്യാഖ്യാതാക്കളാണെങ്കില്‍!”
\end{malayalam}}
\flushright{\begin{Arabic}
\quranayah[12][44]
\end{Arabic}}
\flushleft{\begin{malayalam}
അവര്‍ പറഞ്ഞു: "ഇതൊക്കെ പാഴ്ക്കിനാവുകളാണ്. ഞങ്ങള്‍ അത്തരം പാഴ്ക്കിനാവുകളുടെ വ്യാഖ്യാനം അറിയുന്നവരല്ല.”
\end{malayalam}}
\flushright{\begin{Arabic}
\quranayah[12][45]
\end{Arabic}}
\flushleft{\begin{malayalam}
ആ രണ്ടു ജയില്‍ക്കൂട്ടുകാരില്‍ രക്ഷപ്പെട്ടവന്‍ കുറേക്കാലത്തിനു ശേഷം ഓര്‍മിച്ചു പറഞ്ഞു: "അതിന്റെ വ്യാഖ്യാനം ഞാന്‍ നിങ്ങള്‍ക്ക് അറിയിച്ചു തരാം. നിങ്ങള്‍ എന്നെ ചുമതലപ്പെടുത്തി അയച്ചാലും.”
\end{malayalam}}
\flushright{\begin{Arabic}
\quranayah[12][46]
\end{Arabic}}
\flushleft{\begin{malayalam}
അയാള്‍ പറഞ്ഞു: "സത്യസന്ധനായ യൂസുഫേ, എനിക്ക് ഇതിലൊരു വിധി തരിക. ഏഴു തടിച്ചുകൊഴുത്ത പശുക്കള്‍; ഏഴു മെലിഞ്ഞ പശുക്കള്‍ അവയെ തിന്നുന്നു. പിന്നെ ഏഴു പച്ച കതിരുകളും ഏഴു ഉണങ്ങിയ കതിരുകളും. ജനങ്ങള്‍ക്ക് കാര്യം ഗ്രഹിക്കാനായി എനിക്ക് ആ വിശദീകരണവുമായി ജനങ്ങളുടെ അടുത്തേക്ക് തിരിച്ചുപോകാമല്ലോ.”
\end{malayalam}}
\flushright{\begin{Arabic}
\quranayah[12][47]
\end{Arabic}}
\flushleft{\begin{malayalam}
യൂസുഫ് പറഞ്ഞു: "ഏഴുകൊല്ലം നിങ്ങള്‍ തുടര്‍ച്ചയായി കൃഷി ചെയ്യും. അങ്ങനെ നിങ്ങള്‍ കൊയ്തെടുക്കുന്നവ അവയുടെ കതിരില്‍ തന്നെ സൂക്ഷിച്ചുവെക്കുക. നിങ്ങള്‍ക്ക് ആഹരിക്കാനാവശ്യമായ അല്‍പമൊഴികെ.
\end{malayalam}}
\flushright{\begin{Arabic}
\quranayah[12][48]
\end{Arabic}}
\flushleft{\begin{malayalam}
"പിന്നീട് അതിനുശേഷം കഷ്ടതയുടെ ഏഴാണ്ടുകളുണ്ടാകും. അക്കാലത്തേക്കായി നിങ്ങള്‍ കരുതിവെച്ചവ നിങ്ങളന്ന് തിന്നുതീര്‍ക്കും. നിങ്ങള്‍ പ്രത്യേകം സൂക്ഷിച്ചുവെച്ച അല്‍പമൊഴികെ.
\end{malayalam}}
\flushright{\begin{Arabic}
\quranayah[12][49]
\end{Arabic}}
\flushleft{\begin{malayalam}
"പിന്നീട് അതിനു ശേഷം ഒരു കൊല്ലംവരും. അന്ന് ആളുകള്‍ക്ക് സുഭിക്ഷതയുണ്ടാകും. അവര്‍ തങ്ങള്‍ക്കാവശ്യമുള്ളത് പിഴിഞ്ഞെടുക്കുകയും ചെയ്യും.”
\end{malayalam}}
\flushright{\begin{Arabic}
\quranayah[12][50]
\end{Arabic}}
\flushleft{\begin{malayalam}
രാജാവ് പറഞ്ഞു: "നിങ്ങള്‍ യൂസുഫിനെ എന്റെ അടുത്തു കൊണ്ടുവരിക.” യൂസുഫിന്റെ അടുത്ത് ദൂതന്‍ ചെന്നപ്പോള്‍ അദ്ദേഹം പറഞ്ഞു: "നീ നിന്റെ യജമാനന്റെ അടുത്തേക്കു തന്നെ തിരിച്ചു പോവുക. എന്നിട്ട് അദ്ദേഹത്തോടു ചോദിക്കുക; സ്വന്തം കൈകള്‍ക്ക് മുറിവുണ്ടാക്കിയ ആ സ്ത്രീകളുടെ സ്ഥിതിയെന്തെന്ന്. എന്റെ നാഥന്‍ അവരുടെ കുതന്ത്രത്തെപ്പറ്റി നന്നായറിയുന്നവനാണ്; തീര്‍ച്ച.”
\end{malayalam}}
\flushright{\begin{Arabic}
\quranayah[12][51]
\end{Arabic}}
\flushleft{\begin{malayalam}
രാജാവ് സ്ത്രീകളോട് ചോദിച്ചു: "യൂസുഫിനെ വശപ്പെടുത്താന്‍ ശ്രമിച്ചപ്പോള്‍ നിങ്ങളുടെ അനുഭവമെന്തായിരുന്നു?” അവര്‍ പറഞ്ഞു: "മഹത്വം അല്ലാഹുവിനു തന്നെ. യൂസുഫിനെപ്പറ്റി മോശമായതൊന്നും ഞങ്ങള്‍ക്കറിയില്ല.” പ്രഭുവിന്റെ പത്നി പറഞ്ഞു: "ഇപ്പോള്‍ സത്യം വെളിപ്പെട്ടിരിക്കുന്നു. ഞാന്‍ അദ്ദേഹത്തെ വശപ്പെടുത്താന്‍ സ്വയം ശ്രമിക്കുകയായിരുന്നു. തീര്‍ച്ചയായും അദ്ദേഹം സത്യവാനാണ്.”
\end{malayalam}}
\flushright{\begin{Arabic}
\quranayah[12][52]
\end{Arabic}}
\flushleft{\begin{malayalam}
യൂസുഫ് പറഞ്ഞു: "പ്രഭുവില്ലാത്ത നേരത്ത് ഞാനദ്ദേഹത്തെ വഞ്ചിച്ചിട്ടില്ലെന്ന് അദ്ദേഹം അറിയാനാണ് ഞാനങ്ങനെ ചെയ്തത്. വഞ്ചകരുടെ കുതന്ത്രങ്ങളെ അല്ലാഹു ഒരിക്കലും ലക്ഷ്യത്തിലെത്തിക്കുകയില്ല.
\end{malayalam}}
\flushright{\begin{Arabic}
\quranayah[12][53]
\end{Arabic}}
\flushleft{\begin{malayalam}
"ഞാനെന്റെ മനസ്സ് കുറ്റമറ്റതാണെന്നവകാശപ്പെടുന്നില്ല. തീര്‍ച്ചയായും മനുഷ്യമനസ്സ് തിന്മക്കു പ്രേരിപ്പിക്കുന്നതു തന്നെ. എന്റെ നാഥന്‍ അനുഗ്രഹിച്ചവരുടേതൊഴികെ. എന്റെ നാഥന്‍ ഏറെ പൊറുക്കുന്നവനും പരമദയാലുവുമാണ്; തീര്‍ച്ച.”
\end{malayalam}}
\flushright{\begin{Arabic}
\quranayah[12][54]
\end{Arabic}}
\flushleft{\begin{malayalam}
രാജാവ് കല്‍പിച്ചു: "നിങ്ങള്‍ അദ്ദേഹത്തെ എന്റെ അടുത്തെത്തിക്കുക. ഞാനദ്ദേഹത്തെ എന്റെ പ്രത്യേകക്കാരനായി സ്വീകരിക്കട്ടെ.” അങ്ങനെ അദ്ദേഹവുമായി സംസാരിച്ചപ്പോള്‍ രാജാവ് പറഞ്ഞു: "താങ്കളിന്ന് നമ്മുടെയടുത്ത് ഉന്നതസ്ഥാനീയനാണ്. നമ്മുടെ വിശ്വസ്തനും.”
\end{malayalam}}
\flushright{\begin{Arabic}
\quranayah[12][55]
\end{Arabic}}
\flushleft{\begin{malayalam}
യൂസുഫ് പറഞ്ഞു: "രാജ്യത്തെ ഖജനാവുകളുടെ ചുമതല എന്നെ ഏല്‍പിക്കുക. തീര്‍ച്ചയായും ഞാനതു പരിരക്ഷിക്കുന്നവനും അതിനാവശ്യമായ അറിവുള്ളവനുമാണ്.”
\end{malayalam}}
\flushright{\begin{Arabic}
\quranayah[12][56]
\end{Arabic}}
\flushleft{\begin{malayalam}
അവ്വിധം നാം യൂസുഫിന് അന്നാട്ടില്‍ അദ്ദേഹം ഉദ്ദേശിക്കുന്നിടമെല്ലാം ഉപയോഗിക്കാന്‍ കഴിയുമാറ് സൌകര്യം ചെയ്തുകൊടുത്തു. നാം ഉദ്ദേശിക്കുന്നവര്‍ക്ക് നമ്മുടെ കാരുണ്യം നല്‍കുന്നു. സല്‍ക്കര്‍മികള്‍ക്കുള്ള പ്രതിഫലം നാമൊട്ടും പാഴാക്കുകയില്ല.
\end{malayalam}}
\flushright{\begin{Arabic}
\quranayah[12][57]
\end{Arabic}}
\flushleft{\begin{malayalam}
എന്നാല്‍ സത്യവിശ്വാസം സ്വീകരിക്കുകയും സൂക്ഷ്മത പാലിക്കുകയും ചെയ്യുന്നവര്‍ക്ക് പരലോകത്തെ പ്രതിഫലമാണ് ഉത്തമം.
\end{malayalam}}
\flushright{\begin{Arabic}
\quranayah[12][58]
\end{Arabic}}
\flushleft{\begin{malayalam}
യൂസുഫിന്റെ സഹോദരന്മാര്‍ വന്നു. അവര്‍ അദ്ദേഹത്തിന്റെ അടുത്തെത്തി. അപ്പോള്‍ അദ്ദേഹം അവരെ തിരിച്ചറിഞ്ഞു. എന്നാല്‍ അവര്‍ക്ക് അദ്ദേഹത്തെ മനസ്സിലായില്ല.
\end{malayalam}}
\flushright{\begin{Arabic}
\quranayah[12][59]
\end{Arabic}}
\flushleft{\begin{malayalam}
അദ്ദേഹം അവര്‍ക്കാവശ്യമായ ചരക്കുകളൊരുക്കിക്കൊടുത്തു. എന്നിട്ടിങ്ങനെ പറഞ്ഞു: "നിങ്ങളുടെ പിതാവൊത്ത സഹോദരനെ എന്റെയടുത്ത് കൊണ്ടുവരണം. ഞാന്‍ അളവില്‍ തികവ് വരുത്തുന്നതും ഏറ്റവും നല്ല നിലയില്‍ ആതിഥ്യമരുളുന്നതും നിങ്ങള്‍ കാണുന്നില്ലേ?
\end{malayalam}}
\flushright{\begin{Arabic}
\quranayah[12][60]
\end{Arabic}}
\flushleft{\begin{malayalam}
"നിങ്ങളവനെ എന്റെ അടുത്ത് കൊണ്ടുവന്നില്ലെങ്കില്‍ നിങ്ങള്‍ക്കിനി ഇവിടെ നിന്ന് ധാന്യം അളന്നു തരുന്നതല്ല. നിങ്ങള്‍ എന്റെ അടുത്ത് വരികയും വേണ്ട.”
\end{malayalam}}
\flushright{\begin{Arabic}
\quranayah[12][61]
\end{Arabic}}
\flushleft{\begin{malayalam}
അവര്‍ പറഞ്ഞു: "അവന്റെ കാര്യത്തില്‍ പിതാവിനെ സമ്മതിപ്പിക്കാന്‍ ഞങ്ങള്‍ ശ്രമിക്കാം. തീര്‍ച്ചയായും ഞങ്ങളങ്ങനെ ചെയ്യാം.”
\end{malayalam}}
\flushright{\begin{Arabic}
\quranayah[12][62]
\end{Arabic}}
\flushleft{\begin{malayalam}
യൂസുഫ് തന്റെ ഭൃത്യന്മാരോടു പറഞ്ഞു: "അവര്‍ പകരം തന്ന ചരക്കുകള്‍ അവരുടെ ഭാണ്ഡങ്ങളില്‍ തന്നെ വെച്ചേക്കുക. അവര്‍ തങ്ങളുടെ കുടുംബത്തില്‍ തിരിച്ചെത്തിയാലത് തിരിച്ചറിഞ്ഞുകൊള്ളും. അവര്‍ വീണ്ടും വന്നേക്കും.”
\end{malayalam}}
\flushright{\begin{Arabic}
\quranayah[12][63]
\end{Arabic}}
\flushleft{\begin{malayalam}
അവര്‍ തങ്ങളുടെ പിതാവിന്റെ അടുത്ത് മടങ്ങിയെത്തിയപ്പോള്‍ പറഞ്ഞു: "ഞങ്ങളുടെ പിതാവേ, ഞങ്ങള്‍ക്ക് അളന്നുകിട്ടുന്നത് തടയപ്പെട്ടിരിക്കുന്നു. അതിനാല്‍ ഞങ്ങളോടൊത്ത് ഞങ്ങളുടെ സഹോദരനെ കൂടി അയച്ചുതരിക. എങ്കില്‍ ഞങ്ങള്‍ക്ക് ധാന്യം അളന്നുകിട്ടും. തീര്‍ച്ചയായും ഞങ്ങളവനെ വേണ്ടപോലെ കാത്തുരക്ഷിക്കും.”
\end{malayalam}}
\flushright{\begin{Arabic}
\quranayah[12][64]
\end{Arabic}}
\flushleft{\begin{malayalam}
പിതാവ് പറഞ്ഞു: "അവന്റെ കാര്യത്തില്‍ എനിക്ക് നിങ്ങളെ വിശ്വസിക്കാനാവുമോ? നേരത്തെ അവന്റെ സഹോദരന്റെ കാര്യത്തില്‍ നിങ്ങളെ വിശ്വസിച്ചപോലെയല്ലേ ഇതും? അല്ലാഹുവാണ് ഏറ്റവും നല്ല സംരക്ഷകന്‍. അവന്‍ കാരുണികരില്‍ പരമകാരുണികനാകുന്നു.”
\end{malayalam}}
\flushright{\begin{Arabic}
\quranayah[12][65]
\end{Arabic}}
\flushleft{\begin{malayalam}
അവര്‍ തങ്ങളുടെ കെട്ടുകള്‍ തുറന്നുനോക്കിയപ്പോള്‍ തങ്ങള്‍ കൊണ്ടുപോയ ചരക്കുകള്‍ തങ്ങള്‍ക്കു തന്നെ തിരിച്ചുകിട്ടിയതായി കണ്ടു. അപ്പോഴവര്‍ പറഞ്ഞു: "ഞങ്ങളുടെ പിതാവേ, നമുക്കിനിയെന്തുവേണം? നമ്മുടെ ചരക്കുകളിതാ നമുക്കു തന്നെ തിരിച്ചുകിട്ടിയിരിക്കുന്നു. ഞങ്ങള്‍ പോയി കുടുംബത്തിന് ആവശ്യമായ ആഹാരസാധനങ്ങള്‍ കൊണ്ടുവരാം. ഞങ്ങളുടെ സഹോദരനെ കാത്തുരക്ഷിക്കുകയും ചെയ്യാം. ഒരൊട്ടകത്തിന് ചുമക്കാവുന്നത്ര ധാന്യം നമുക്കു കൂടുതല്‍ കിട്ടുമല്ലോ. അത്രയും കൂടുതല്‍ അളന്നുകിട്ടുകയെന്നത് വളരെ വേഗം സാധിക്കുന്ന കാര്യമത്രെ.”
\end{malayalam}}
\flushright{\begin{Arabic}
\quranayah[12][66]
\end{Arabic}}
\flushleft{\begin{malayalam}
പിതാവ് പറഞ്ഞു: "നിങ്ങള്‍ വല്ല അപകടത്തിലും അകപ്പെട്ടില്ലെങ്കില്‍ അവനെ എന്റെ അടുത്ത് തിരിച്ചുകൊണ്ടുവരുമെന്ന് അല്ലാഹുവിന്റെ പേരില്‍ നിങ്ങള്‍ ഉറപ്പ് തരുംവരെ ഞാനവനെ നിങ്ങളോടൊപ്പം അയക്കുകയില്ല.” അങ്ങനെ അവരദ്ദേഹത്തിന് ഉറപ്പ് നല്‍കിയപ്പോള്‍ അദ്ദേഹം പറഞ്ഞു: "നാം ഇപ്പറയുന്നതിന് കാവല്‍ നില്‍ക്കുന്നവന്‍ അല്ലാഹുവാണ്.”
\end{malayalam}}
\flushright{\begin{Arabic}
\quranayah[12][67]
\end{Arabic}}
\flushleft{\begin{malayalam}
അദ്ദേഹം അവരോട് പറഞ്ഞു: "എന്റെ മക്കളേ, നിങ്ങള്‍ ഒരേ വാതിലിലൂടെ പ്രവേശിക്കരുത്. വ്യത്യസ്ത വാതിലുകളിലൂടെ പ്രവേശിക്കുക. ദൈവവിധിയില്‍ നിന്ന്ഒന്നുപോലും നിങ്ങളില്‍ നിന്ന് തടഞ്ഞുനിര്‍ത്താന്‍ എനിക്കു സാധ്യമല്ല. വിധിനിശ്ചയം അല്ലാഹുവിന്റേതു മാത്രമാണല്ലോ. ഞാനിതാ അവനില്‍ ഭരമേല്‍പിക്കുന്നു. ഭരമേല്‍പിക്കുന്നവര്‍ അവനിലാണ് ഭരമേല്‍പിക്കേണ്ടത്.”
\end{malayalam}}
\flushright{\begin{Arabic}
\quranayah[12][68]
\end{Arabic}}
\flushleft{\begin{malayalam}
അവരുടെ പിതാവ് കല്‍പിച്ചപോലെ അവര്‍ പ്രവേശിച്ചപ്പോള്‍ അല്ലാഹുവിന്റെ വിധിയില്‍ നിന്ന്ഒന്നും അവരില്‍നിന്ന് തടഞ്ഞുനിര്‍ത്താന്‍ അദ്ദേഹത്തിനു സാധിച്ചില്ല. യഅ്ഖൂബിന്റെ മനസ്സിലുണ്ടായിരുന്ന ഒരാഗ്രഹം അദ്ദേഹം പൂര്‍ത്തീകരിച്ചുവെന്നു മാത്രം. നാം പഠിപ്പിച്ചുകൊടുത്തതിനാല്‍ അദ്ദേഹം അറിവുള്ളവനാണ്. എന്നാല്‍ മനുഷ്യരിലേറെപ്പേരും അറിയുന്നില്ല.
\end{malayalam}}
\flushright{\begin{Arabic}
\quranayah[12][69]
\end{Arabic}}
\flushleft{\begin{malayalam}
അവര്‍ യൂസുഫിന്റെ സന്നിധിയില്‍ പ്രവേശിച്ചപ്പോള്‍ അദ്ദേഹം തന്റെ സഹോദരനെ അടുത്തുവരുത്തി. എന്നിട്ട് അവനോട് പറഞ്ഞു: "ഞാന്‍ നിന്റെ സഹോദരനാണ്. ഇവര്‍ ചെയ്തുകൂട്ടിയതിനെക്കുറിച്ചൊന്നും നീയിനി ദുഃഖിക്കേണ്ടതില്ല.”
\end{malayalam}}
\flushright{\begin{Arabic}
\quranayah[12][70]
\end{Arabic}}
\flushleft{\begin{malayalam}
അങ്ങനെ അദ്ദേഹം ചരക്കുകള്‍ ഒരുക്കിക്കൊടുത്തപ്പോള്‍ തന്റെ സഹോദരന്റെ ഭാണ്ഡത്തില്‍ പാനപാത്രം എടുത്തുവെച്ചു. പിന്നീട് ഒരു വിളംബരക്കാരന്‍ വിളിച്ചുപറഞ്ഞു: "ഹേ, യാത്രാസംഘമേ, നിങ്ങള്‍ കള്ളന്മാരാണ്.”
\end{malayalam}}
\flushright{\begin{Arabic}
\quranayah[12][71]
\end{Arabic}}
\flushleft{\begin{malayalam}
അവരുടെ നേരെ തിരിഞ്ഞ് യാത്രാസംഘം ചോദിച്ചു: "എന്താണ് നിങ്ങള്‍ക്ക് നഷ്ടപ്പെട്ടത്?”
\end{malayalam}}
\flushright{\begin{Arabic}
\quranayah[12][72]
\end{Arabic}}
\flushleft{\begin{malayalam}
അവര്‍ പറഞ്ഞു: "രാജാവിന്റെ പാനപാത്രം നഷ്ടപ്പെട്ടിരിക്കുന്നു. അത് കൊണ്ടുവന്നുതരുന്നവന് ഒരൊട്ടകത്തിന് ചുമക്കാവുന്നത്ര ധാന്യം സമ്മാനമായി കിട്ടും.” "ഞാനതിന് ബാധ്യസ്ഥനാണ്.”
\end{malayalam}}
\flushright{\begin{Arabic}
\quranayah[12][73]
\end{Arabic}}
\flushleft{\begin{malayalam}
യാത്രാസംഘം പറഞ്ഞു: "അല്ലാഹു സത്യം! നിങ്ങള്‍ക്കറിയാമല്ലോ, നാട്ടില്‍ നാശമുണ്ടാക്കാന്‍ വന്നവരല്ല ഞങ്ങള്‍; ഞങ്ങള്‍ കള്ളന്മാരുമല്ല.”
\end{malayalam}}
\flushright{\begin{Arabic}
\quranayah[12][74]
\end{Arabic}}
\flushleft{\begin{malayalam}
അവര്‍ ചോദിച്ചു: "നിങ്ങള്‍ കള്ളം പറഞ്ഞവരാണെങ്കില്‍ എന്തു ശിക്ഷയാണ് നല്‍കേണ്ടത്?”
\end{malayalam}}
\flushright{\begin{Arabic}
\quranayah[12][75]
\end{Arabic}}
\flushleft{\begin{malayalam}
യാത്രാസംഘം പറഞ്ഞു: "അതിനുള്ള ശിക്ഷയിതാണ്: ആരുടെ ഭാണ്ഡത്തില്‍ നിന്നാണോ അത് കണ്ടുകിട്ടുന്നത് അവനെ പിടിച്ചുവെക്കണം. അങ്ങനെയാണ് ഞങ്ങള്‍ അക്രമികള്‍ക്ക് ശിക്ഷ നല്‍കാറുള്ളത്.”
\end{malayalam}}
\flushright{\begin{Arabic}
\quranayah[12][76]
\end{Arabic}}
\flushleft{\begin{malayalam}
യൂസുഫ് തന്റെ സഹോദരന്റെ ഭാണ്ഡം പരിശോധിക്കുന്നതിനു മുമ്പ് അവരുടെ ഭാണ്ഡങ്ങള്‍ പരിശോധിക്കാന്‍ തുടങ്ങി. അവസാനമത് തന്റെ സഹോദരന്റെ ഭാണ്ഡത്തില്‍ നിന്ന് പുറത്തെടുത്തു. അവ്വിധം നാം യൂസുഫിനുവേണ്ടി തന്ത്രം പ്രയോഗിച്ചു. അല്ലാഹു ഉദ്ദേശിച്ചിരുന്നെങ്കിലല്ലാതെ രാജാവിന്റെ നിയമമനുസരിച്ച് യൂസുഫിന് തന്റെ സഹോദരനെ പിടിച്ചുവെക്കാന്‍ സാധിക്കുമായിരുന്നില്ല. നാം ഇച്ഛിക്കുന്നവരെ നാം പല പദവികളിലും ഉയര്‍ത്തുന്നു. അറിവുള്ളവര്‍ക്കെല്ലാം ഉപരിയായി സര്‍വജ്ഞനായി അല്ലാഹുവുണ്ട്.
\end{malayalam}}
\flushright{\begin{Arabic}
\quranayah[12][77]
\end{Arabic}}
\flushleft{\begin{malayalam}
സഹോദരന്മാര്‍ പറഞ്ഞു: "അവന്‍ കട്ടുവെങ്കില്‍ അവന്റെ സഹോദരനും മുമ്പ് കട്ടിട്ടുണ്ട്.” യൂസുഫ് ഇതൊക്കെ തന്റെ മനസ്സിലൊളിപ്പിച്ചുവെച്ചു. യാഥാര്‍ഥ്യം അവരോട് വെളിപ്പെടുത്തിയില്ല. അദ്ദേഹം ഇത്രമാത്രം പറഞ്ഞു: "നിങ്ങളുടെ നിലപാട് നന്നെ മോശംതന്നെ. നിങ്ങള്‍ പറഞ്ഞുണ്ടാക്കുന്നതിനെപ്പറ്റിയൊക്കെ നന്നായറിയാവുന്നവനാണ് അല്ലാഹു.”
\end{malayalam}}
\flushright{\begin{Arabic}
\quranayah[12][78]
\end{Arabic}}
\flushleft{\begin{malayalam}
അവര്‍ പറഞ്ഞു: "പ്രഭോ, ഇവന് വയോവൃദ്ധനായ പിതാവുണ്ട്. അതിനാല്‍ ഇവന്ന് പകരമായി അങ്ങ് ഞങ്ങളിലാരെയെങ്കിലും പിടിച്ചുവെച്ചാലും. ഞങ്ങള്‍ അങ്ങയെ കാണുന്നത് അങ്ങേയറ്റം സന്മനസ്സുള്ളവനായാണ്.”
\end{malayalam}}
\flushright{\begin{Arabic}
\quranayah[12][79]
\end{Arabic}}
\flushleft{\begin{malayalam}
യൂസുഫ് പറഞ്ഞു: "അല്ലാഹുവില്‍ ശരണം! നമ്മുടെ സാധനം ആരുടെ കയ്യിലാണോ കണ്ടെത്തിയത് അവനെയല്ലാതെ മറ്റാരെയെങ്കിലും പിടിച്ചുവെക്കുകയോ? എങ്കില്‍ ഞങ്ങള്‍ അതിക്രമികളായിത്തീരും.”
\end{malayalam}}
\flushright{\begin{Arabic}
\quranayah[12][80]
\end{Arabic}}
\flushleft{\begin{malayalam}
സഹോദരനെ സംബന്ധിച്ച് നിരാശരായപ്പോള്‍ അവര്‍ മാറിയിരുന്ന് കൂടിയാലോചിച്ചു. അവരിലെ മുതിര്‍ന്നവന്‍ പറഞ്ഞു: "നിങ്ങള്‍ക്കറിഞ്ഞുകൂടേ; നിങ്ങളുടെ പിതാവ് അല്ലാഹുവിന്റെ പേരില്‍ നിങ്ങളോട് ഉറപ്പ് വാങ്ങിയ കാര്യം. മുമ്പ് യൂസുഫിന്റെ കാര്യത്തില്‍ നിങ്ങള്‍ അക്രമം കാണിച്ചിട്ടുണ്ടെന്നും. അതിനാല്‍ എന്റെ പിതാവ് എനിക്കനുവാദം തരികയോ അല്ലെങ്കില്‍ അല്ലാഹു എന്റെ കാര്യം തീരുമാനിക്കുകയോ ചെയ്യുംവരെ ഞാന്‍ ഈ നാട് വിടുകയില്ല. വിധികര്‍ത്താക്കളില്‍ ഉത്തമന്‍ അല്ലാഹുവാണല്ലോ.
\end{malayalam}}
\flushright{\begin{Arabic}
\quranayah[12][81]
\end{Arabic}}
\flushleft{\begin{malayalam}
"നിങ്ങള്‍ നിങ്ങളുടെ പിതാവിന്റെ അടുത്ത് മടങ്ങിച്ചെന്ന് പറയുക: “ഞങ്ങളുടെ പിതാവേ, അങ്ങയുടെ മകന്‍ കളവു നടത്തി. ഞങ്ങള്‍ മനസ്സിലാക്കിയതിന്റെ അടിസ്ഥാനത്തില്‍ മാത്രമാണ് ഞങ്ങള്‍ സാക്ഷ്യം വഹിച്ചത്. അദൃശ്യകാര്യം ഞങ്ങള്‍ക്ക് അറിയില്ലല്ലോ.
\end{malayalam}}
\flushright{\begin{Arabic}
\quranayah[12][82]
\end{Arabic}}
\flushleft{\begin{malayalam}
“ഞങ്ങള്‍ താമസിച്ചുപോന്ന നാട്ടുകാരോട് ചോദിച്ചു നോക്കുക. ഞങ്ങളോടൊന്നിച്ചുണ്ടായിരുന്ന യാത്രാസംഘത്തോടും അങ്ങയ്ക്ക് അന്വേഷിക്കാം. ഞങ്ങള്‍ സത്യമേ പറയുന്നുള്ളൂ.”
\end{malayalam}}
\flushright{\begin{Arabic}
\quranayah[12][83]
\end{Arabic}}
\flushleft{\begin{malayalam}
പിതാവ് പറഞ്ഞു: "അല്ല, നിങ്ങളുടെ മനസ്സ് നിങ്ങളെ ഒരു കാര്യത്തിന് പ്രേരിപ്പിച്ചു. അതു നിങ്ങള്‍ക്ക് ചേതോഹരമായി തോന്നി. അതിനാല്‍ നന്നായി ക്ഷമിക്കുക തന്നെ. ഒരുവേള അല്ലാഹു അവരെയെല്ലാവരെയും എന്റെ അടുത്തെത്തിച്ചേക്കാം. അവന്‍ എല്ലാം അറിയുന്നവനും യുക്തിജ്ഞനും തന്നെ.”
\end{malayalam}}
\flushright{\begin{Arabic}
\quranayah[12][84]
\end{Arabic}}
\flushleft{\begin{malayalam}
അദ്ദേഹം അവരില്‍നിന്ന് പിന്തിരിഞ്ഞ് ഇങ്ങനെ പറഞ്ഞു: "ഹാ, യൂസുഫിന്റെ കാര്യമെത്ര കഷ്ടം!” ദുഃഖം കൊണ്ട് അദ്ദേഹത്തിന്റെ ഇരുകണ്ണുകളും വെളുത്തുവിളറി. അദ്ദേഹം അതീവ ദുഃഖിതനായി.
\end{malayalam}}
\flushright{\begin{Arabic}
\quranayah[12][85]
\end{Arabic}}
\flushleft{\begin{malayalam}
അവര്‍ പറഞ്ഞു: "അല്ലാഹു സത്യം! അങ്ങ് യൂസുഫിനെത്തന്നെ ഓര്‍ത്തുകൊണ്ടേയിരിക്കുകയാണ്. അങ്ങ് പറ്റെ അവശനാവുകയോ ജീവന്‍ വെടിയുകയോ ചെയ്യുമെന്ന് ഞങ്ങളാശങ്കിക്കുന്നു.”
\end{malayalam}}
\flushright{\begin{Arabic}
\quranayah[12][86]
\end{Arabic}}
\flushleft{\begin{malayalam}
അദ്ദേഹം പറഞ്ഞു: "എന്റെ വേദനയെയും വ്യസനത്തെയും സംബന്ധിച്ച് ഞാന്‍ അല്ലാഹുവോട് മാത്രമാണ് ആവലാതിപ്പെടുന്നത്. നിങ്ങള്‍ക്കറിയാത്ത പലതും അല്ലാഹുവില്‍നിന്ന് ഞാനറിയുന്നു.
\end{malayalam}}
\flushright{\begin{Arabic}
\quranayah[12][87]
\end{Arabic}}
\flushleft{\begin{malayalam}
"എന്റെ മക്കളേ, നിങ്ങള്‍ പോയി യൂസുഫിനെയും അവന്റെ സഹോദരനെയും സംബന്ധിച്ച് അന്വേഷിച്ചു നോക്കുക. അല്ലാഹുവിങ്കല്‍ നിന്നുള്ള കാരുണ്യത്തെ സംബന്ധിച്ച് നിരാശരാവരുത്. സത്യനിഷേധികളായ ജനമല്ലാതെ അല്ലാഹുവിന്റെ കാരുണ്യത്തെ സംബന്ധിച്ച് നിരാശരാവുകയില്ല.”
\end{malayalam}}
\flushright{\begin{Arabic}
\quranayah[12][88]
\end{Arabic}}
\flushleft{\begin{malayalam}
അങ്ങനെ അവര്‍ യൂസുഫിന്റെ അടുത്ത് കടന്നുചെന്നു. അവര്‍ പറഞ്ഞു: "പ്രഭോ, ഞങ്ങളെയും ഞങ്ങളുടെ കുടുംബത്തെയും വറുതി ബാധിച്ചിരിക്കുന്നു. താണതരം ചരക്കുമായാണ് ഞങ്ങള്‍ വന്നിരിക്കുന്നത്. അതിനാല്‍ അങ്ങ് ഞങ്ങള്‍ക്ക് അളവ് പൂര്‍ത്തീകരിച്ചുതരണം. ഞങ്ങള്‍ക്ക് ദാനമായും നല്‍കണം. ധര്‍മിഷ്ഠര്‍ക്ക് അല്ലാഹു അര്‍ഹമായ പ്രതിഫലം നല്‍കും; തീര്‍ച്ച.”
\end{malayalam}}
\flushright{\begin{Arabic}
\quranayah[12][89]
\end{Arabic}}
\flushleft{\begin{malayalam}
അദ്ദേഹം പറഞ്ഞു: "നിങ്ങള്‍ അവിവേകികളായിരുന്നപ്പോള്‍ യൂസുഫിനോടും അവന്റെ സഹോദരനോടും ചെയ്തതെന്താണെന്ന് അറിയാമോ?”
\end{malayalam}}
\flushright{\begin{Arabic}
\quranayah[12][90]
\end{Arabic}}
\flushleft{\begin{malayalam}
അവര്‍ ചോദിച്ചു: "താങ്കള്‍ തന്നെയാണോ യൂസുഫ്?” അദ്ദേഹം പറഞ്ഞു: "ഞാന്‍ തന്നെയാണ് യൂസുഫ്. ഇതെന്റെ സഹോദരനും. അല്ലാഹു ഞങ്ങളോട് ഔദാര്യം കാണിച്ചിരിക്കുന്നു. ആര്‍ സൂക്ഷ്മത പുലര്‍ത്തുകയും ക്ഷമ പാലിക്കുകയും ചെയ്യുന്നുവോ അത്തരം സദ്വൃത്തരുടെ പ്രതിഫലം അല്ലാഹു നഷ്ടപ്പെടുത്തുകയില്ല; തീര്‍ച്ച.”
\end{malayalam}}
\flushright{\begin{Arabic}
\quranayah[12][91]
\end{Arabic}}
\flushleft{\begin{malayalam}
അവര്‍ പറഞ്ഞു: "അല്ലാഹുവാണ് സത്യം! അല്ലാഹു താങ്കള്‍ക്ക് ഞങ്ങളെക്കാള്‍ ശ്രേഷ്ഠത കല്‍പിച്ചിരിക്കുന്നു. തീര്‍ച്ചയായും ഞങ്ങള്‍ തെറ്റുകാരായിരുന്നു.”
\end{malayalam}}
\flushright{\begin{Arabic}
\quranayah[12][92]
\end{Arabic}}
\flushleft{\begin{malayalam}
അദ്ദേഹം പറഞ്ഞു: "ഇന്നു നിങ്ങള്‍ക്കെതിരെ പ്രതികാരമൊന്നുമില്ല. അല്ലാഹു നിങ്ങള്‍ക്ക് മാപ്പ് നല്‍കട്ടെ. അവന്‍ കാരുണികരില്‍ പരമകാരുണികനല്ലോ.
\end{malayalam}}
\flushright{\begin{Arabic}
\quranayah[12][93]
\end{Arabic}}
\flushleft{\begin{malayalam}
"നിങ്ങള്‍ എന്റെ ഈ കുപ്പായവുമായി പോവുക. എന്നിട്ടത് എന്റെ പിതാവിന്റെ മുഖത്ത് ഇട്ടുകൊടുക്കുക. അപ്പോള്‍ അദ്ദേഹം കാഴ്ചയുള്ളവനായിത്തീരും. പിന്നെ നിങ്ങള്‍ നിങ്ങളുടെ എല്ലാ കുടുംബക്കാരെയുംകൊണ്ട് എന്റെയടുത്ത് വരിക.”
\end{malayalam}}
\flushright{\begin{Arabic}
\quranayah[12][94]
\end{Arabic}}
\flushleft{\begin{malayalam}
യാത്രാസംഘം അവിടം വിട്ടപ്പോള്‍ അവരുടെ പിതാവ് പറഞ്ഞു: "ഉറപ്പായും യൂസുഫിന്റെ വാസന ഞാനനുഭവിക്കുന്നു. നിങ്ങളെന്നെ ബുദ്ധിഭ്രമം ബാധിച്ചവനായി ആക്ഷേപിക്കുന്നില്ലെങ്കില്‍!”
\end{malayalam}}
\flushright{\begin{Arabic}
\quranayah[12][95]
\end{Arabic}}
\flushleft{\begin{malayalam}
വീട്ടുകാര്‍ പറഞ്ഞു: "അല്ലാഹു തന്നെ സത്യം! അങ്ങ് ഇപ്പോഴും അങ്ങയുടെ ആ പഴയ ബുദ്ധിഭ്രമത്തില്‍ തന്നെ.”
\end{malayalam}}
\flushright{\begin{Arabic}
\quranayah[12][96]
\end{Arabic}}
\flushleft{\begin{malayalam}
പിന്നീട് ശുഭവാര്‍ത്ത അറിയിക്കുന്നയാള്‍ വന്നു. അയാള്‍ ആ കുപ്പായം അദ്ദേഹത്തിന്റെ മുഖത്തിട്ടുകൊടുത്തു. അദ്ദേഹം കാഴ്ചയുള്ളവനായി. അദ്ദേഹം പറഞ്ഞു: "ഞാന്‍ നിങ്ങളോടു പറഞ്ഞിരുന്നില്ലേ; നിങ്ങള്‍ക്കറിയാത്ത പലതും ഞാന്‍ അല്ലാഹുവില്‍ നിന്ന് അറിയുന്നുവെന്ന്.”
\end{malayalam}}
\flushright{\begin{Arabic}
\quranayah[12][97]
\end{Arabic}}
\flushleft{\begin{malayalam}
അവര്‍ പറഞ്ഞു: "ഞങ്ങളുടെ പിതാവേ, അങ്ങ് ഞങ്ങള്‍ക്കുവേണ്ടി, ഞങ്ങളുടെ പാപമോചനത്തിനായി പ്രാര്‍ഥിക്കേണമേ; തീര്‍ച്ചയായും ഞങ്ങള്‍ കുറ്റവാളികളായിരുന്നു.”
\end{malayalam}}
\flushright{\begin{Arabic}
\quranayah[12][98]
\end{Arabic}}
\flushleft{\begin{malayalam}
അദ്ദേഹം പറഞ്ഞു: "നിങ്ങള്‍ക്കുവേണ്ടി ഞാനെന്റെ നാഥനോട് പാപമോചനത്തിനായി പ്രാര്‍ഥിക്കാം. അവന്‍ ഏറെ പൊറുക്കുന്നവനും പരമ ദയാലുവും തന്നെ; തീര്‍ച്ച.”
\end{malayalam}}
\flushright{\begin{Arabic}
\quranayah[12][99]
\end{Arabic}}
\flushleft{\begin{malayalam}
പിന്നീട് അവരെല്ലാം യൂസുഫിന്റെ സന്നിധിയില്‍ പ്രവേശിച്ചു. യൂസുഫ് തന്റെ മാതാപിതാക്കളെ തന്നിലേക്കു ചേര്‍ത്തുനിര്‍ത്തി. അദ്ദേഹം പറഞ്ഞു: "വരിക. നിര്‍ഭയരായി ഈ പട്ടണത്തില്‍ പ്രവേശിച്ചുകൊള്ളുക. അല്ലാഹു ഇച്ഛിക്കുന്നുവെങ്കില്‍.”
\end{malayalam}}
\flushright{\begin{Arabic}
\quranayah[12][100]
\end{Arabic}}
\flushleft{\begin{malayalam}
അദ്ദേഹം തന്റെ മാതാപിതാക്കളെ സിംഹാസനത്തില്‍ കയറ്റിയിരുത്തി. അവര്‍ അദ്ദേഹത്തിന്റെ മുമ്പില്‍ പ്രണാമമര്‍പ്പിച്ചു. അദ്ദേഹം പറഞ്ഞു: "എന്റെ പിതാവേ, ഞാന്‍ പണ്ടു കണ്ട ആ സ്വപ്നത്തിന്റെ സാക്ഷാല്‍ക്കാരമാണിത്. എന്റെ നാഥന്‍ അത് യാഥാര്‍ഥ്യമാക്കിയിരിക്കുന്നു. എന്നെ തടവറയില്‍നിന്ന് മോചിപ്പിച്ചപ്പോഴും എനിക്കും എന്റെ സഹോദരങ്ങള്‍ക്കുമിടയില്‍ പിശാച് അകല്‍ച്ചയുണ്ടാക്കിയശേഷം അവന്‍ നിങ്ങളെയെല്ലാം മരുഭൂമിയില്‍ നിന്നിവിടെ കൊണ്ടുവന്നപ്പോഴും അവന്‍ എന്നോട് വളരെയേറെ ഔദാര്യം കാണിച്ചിരിക്കുന്നു. തീര്‍ച്ചയായും എന്റെ നാഥന്‍ താനിച്ഛിക്കുന്ന കാര്യങ്ങള്‍ സൂക്ഷ്മമായി നടപ്പാക്കുന്നവനാണ്. അവന്‍ എല്ലാം അറിയുന്നവനും യുക്തിജ്ഞനും തന്നെ.
\end{malayalam}}
\flushright{\begin{Arabic}
\quranayah[12][101]
\end{Arabic}}
\flushleft{\begin{malayalam}
"എന്റെ നാഥാ, നീ എനിക്ക് അധികാരം നല്‍കി. സ്വപ്നകഥകളുടെ വ്യാഖ്യാനം പഠിപ്പിച്ചു. ആകാശഭൂമികളെ പടച്ചവനേ, ഇഹത്തിലും പരത്തിലും നീയാണെന്റെ രക്ഷകന്‍. നീയെന്നെ മുസ്ലിമായി മരിപ്പിക്കേണമേ, സജ്ജനങ്ങളിലുള്‍പ്പെടുത്തേണമേ.”
\end{malayalam}}
\flushright{\begin{Arabic}
\quranayah[12][102]
\end{Arabic}}
\flushleft{\begin{malayalam}
നബിയേ, ഇക്കഥ അഭൌതിക ജ്ഞാനങ്ങളില്‍പെട്ടതാണ്. നാമത് നിനക്ക് ബോധനമായി നല്‍കുന്നു. അവര്‍ കൂടിയിരുന്ന് കുതന്ത്രം മെനഞ്ഞ് തങ്ങളുടെ കാര്യം തീരുമാനിച്ചപ്പോള്‍ നീ അവരുടെ അടുത്തുണ്ടായിരുന്നില്ല.
\end{malayalam}}
\flushright{\begin{Arabic}
\quranayah[12][103]
\end{Arabic}}
\flushleft{\begin{malayalam}
എന്നാല്‍ നീ എത്രതന്നെ ആഗ്രഹിച്ചാലും ജനങ്ങളിലേറെപ്പേരും വിശ്വാസികളാവുകയില്ല.
\end{malayalam}}
\flushright{\begin{Arabic}
\quranayah[12][104]
\end{Arabic}}
\flushleft{\begin{malayalam}
നീ അവരോട് ഇതിന്റെ പേരില്‍ പ്രതിഫലമൊന്നും ചോദിക്കുന്നില്ല. ഇത് ലോകര്‍ക്കാകമാനമുള്ള ഒരുദ്ബോധനം മാത്രമാണ്.
\end{malayalam}}
\flushright{\begin{Arabic}
\quranayah[12][105]
\end{Arabic}}
\flushleft{\begin{malayalam}
ആകാശങ്ങളിലും ഭൂമിയിലും എത്രയെത്ര അടയാളങ്ങളുണ്ട്. ആളുകള്‍ അവയ്ക്കരികിലൂടെ നടന്നുനീങ്ങുന്നു. എന്നിട്ടും അവരവയെ അപ്പാടെ അവഗണിക്കുകയാണ്.
\end{malayalam}}
\flushright{\begin{Arabic}
\quranayah[12][106]
\end{Arabic}}
\flushleft{\begin{malayalam}
അവരില്‍ ഏറെ പേരും അല്ലാഹുവില്‍ വിശ്വസിക്കുന്നില്ല; അവനില്‍ മറ്റുള്ളവയെ പങ്കുചേര്‍ക്കുന്നവരായിക്കൊണ്ടല്ലാതെ.
\end{malayalam}}
\flushright{\begin{Arabic}
\quranayah[12][107]
\end{Arabic}}
\flushleft{\begin{malayalam}
അവരെ ആവരണം ചെയ്യുന്ന അല്ലാഹുവിന്റെ ശിക്ഷ അവര്‍ക്ക് വന്നെത്തുന്നതിനെ സംബന്ധിച്ച് അവര്‍ നിര്‍ഭയരായിരിക്കയാണോ? അല്ലെങ്കില്‍ അവരോര്‍ക്കാത്ത നേരത്ത് പെട്ടെന്ന് അന്ത്യദിനം അവര്‍ക്ക് വന്നുപെടുന്നതിനെപ്പറ്റി?
\end{malayalam}}
\flushright{\begin{Arabic}
\quranayah[12][108]
\end{Arabic}}
\flushleft{\begin{malayalam}
പറയുക: ഇതാണെന്റെ വഴി; തികഞ്ഞ ഉള്‍ക്കാഴ്ചയോടെയാണ് ഞാന്‍ അല്ലാഹുവിലേക്ക് ക്ഷണിക്കുന്നത്. ഞാനും എന്നെ അനുഗമിച്ചവരും. അല്ലാഹു എത്ര പരിശുദ്ധന്‍. ഞാന്‍ അല്ലാഹുവില്‍ പങ്കുചേര്‍ക്കുന്നവരില്‍പെട്ടവനല്ല; തീര്‍ച്ച.
\end{malayalam}}
\flushright{\begin{Arabic}
\quranayah[12][109]
\end{Arabic}}
\flushleft{\begin{malayalam}
ചില പുരുഷന്മാരെയല്ലാതെ നിനക്കുമുമ്പു നാം ദൂതന്മാരായി നിയോഗിച്ചിട്ടില്ല. നാം അവര്‍ക്ക് ബോധനം നല്‍കി. അവര്‍ വിവിധ രാജ്യങ്ങളില്‍ നിന്നുള്ളവരായിരുന്നു. എന്നിട്ടും ഇക്കൂട്ടര്‍ ഭൂമിയില്‍ സഞ്ചരിച്ചുനോക്കുന്നില്ലേ? അങ്ങനെ അവര്‍ക്കു മുമ്പുണ്ടായിരുന്നവരുടെ ഒടുക്കം എവ്വിധമായിരുന്നുവെന്ന് നോക്കിക്കാണുന്നില്ലേ? ഭക്തി പുലര്‍ത്തുന്നവര്‍ക്ക് കൂടുതലുത്തമം പരലോകഭവനമാണ്. ഇതൊന്നും നിങ്ങള്‍ ചിന്തിക്കുന്നില്ലേ?
\end{malayalam}}
\flushright{\begin{Arabic}
\quranayah[12][110]
\end{Arabic}}
\flushleft{\begin{malayalam}
അങ്ങനെ ആ ദൈവദൂതന്മാര്‍ ആശയറ്റവരാവുകയും അവര്‍ തങ്ങളോട് പറഞ്ഞത് കളവാണെന്ന് ജനം കരുതുകയും ചെയ്തപ്പോള്‍ നമ്മുടെ സഹായം അവര്‍ക്ക് വന്നെത്തി. അങ്ങനെ നാം ഇച്ഛിച്ചവര്‍ രക്ഷപ്പെട്ടു. കുറ്റവാളികളായ ജനത്തില്‍ നിന്ന് നമ്മുടെ ശിക്ഷ തട്ടിമാറ്റപ്പെടുകയില്ല.
\end{malayalam}}
\flushright{\begin{Arabic}
\quranayah[12][111]
\end{Arabic}}
\flushleft{\begin{malayalam}
അവരുടെ ഈ കഥകളില്‍ ചിന്തിക്കുന്നവര്‍ക്ക്് തീര്‍ച്ചയായും ഗുണപാഠമുണ്ട്. ഇവയൊന്നും കെട്ടിച്ചമച്ചുണ്ടാക്കുന്ന വര്‍ത്തമാനമല്ല. മറിച്ച്, അതിന്റെ മുമ്പുള്ള വേദങ്ങളെ സത്യപ്പെടുത്തുന്നതാണ്. എല്ലാ കാര്യങ്ങള്‍ക്കുമുള്ള വിശദീകരണവുമാണ്. ഒപ്പം വിശ്വസിക്കുന്ന ജനത്തിന് വഴികാട്ടിയും മഹത്തായ അനുഗ്രഹവും.
\end{malayalam}}
\chapter{\textmalayalam{‍റഅദ് ( ഇടിനാദം )}}
\begin{Arabic}
\Huge{\centerline{\basmalah}}\end{Arabic}
\flushright{\begin{Arabic}
\quranayah[13][1]
\end{Arabic}}
\flushleft{\begin{malayalam}
അലിഫ് - ലാം - മീം - റാഅ്. ഇത് വേദപുസ്തകത്തിലെ വചനങ്ങളാണ്. നിന്റെ നാഥനില്‍ നിന്ന് നിനക്ക് അവതരിച്ചത്. തീര്‍ത്തും സത്യമാണിത്. എങ്കിലും ജനങ്ങളിലേറെപ്പേരും വിശ്വസിക്കുന്നവരല്ല.
\end{malayalam}}
\flushright{\begin{Arabic}
\quranayah[13][2]
\end{Arabic}}
\flushleft{\begin{malayalam}
നിങ്ങള്‍ കാണുന്ന താങ്ങൊന്നുമില്ലാതെ ആകാശങ്ങളെ ഉയര്‍ത്തിനിര്‍ത്തിയവന്‍ അല്ലാഹുവാണ്. പിന്നെ അവന്‍ സിംഹാസനസ്ഥനായി. അവന്‍ സൂര്യ ചന്ദ്രന്മാരെ അധീനപ്പെടുത്തിയിരിക്കുന്നു. എല്ലാം നിശ്ചിത കാലപരിധിയില്‍ ചരിച്ചുകൊണ്ടിരിക്കുകയാണ്. അവന്‍ കാര്യങ്ങളെല്ലാം നിയന്ത്രിച്ചുകൊണ്ടിരിക്കുന്നു. ഈ തെളിവുകളെല്ലാം വിവരിച്ചുതരികയും ചെയ്യുന്നു. നിങ്ങളുടെ നാഥനുമായി സന്ധിക്കുന്നതിനെ സംബന്ധിച്ച് നിങ്ങള്‍ ദൃഢബോധ്യമുള്ളവരാകാന്‍.
\end{malayalam}}
\flushright{\begin{Arabic}
\quranayah[13][3]
\end{Arabic}}
\flushleft{\begin{malayalam}
അവനാണ് ഈ ഭൂമിയെ വിശാലമാക്കിയത്. അവനതില്‍ നീങ്ങിപ്പോകാത്ത പര്‍വതങ്ങളുണ്ടാക്കി; നദികളും. അവന്‍ തന്നെ എല്ലാ പഴങ്ങളിലും ഈരണ്ട് ഇണകളെ സൃഷ്ടിച്ചു. അവന്‍ രാവ് കൊണ്ട് പകലിനെ മൂടുന്നു. ചിന്തിക്കുന്ന ജനത്തിന് ഇതിലൊക്കെയും അടയാളങ്ങളുണ്ട്.
\end{malayalam}}
\flushright{\begin{Arabic}
\quranayah[13][4]
\end{Arabic}}
\flushleft{\begin{malayalam}
ഭൂമിയില്‍ അടുത്തടുത്തുള്ള ഖണ്ഡങ്ങളുണ്ട്. മുന്തിരിത്തോപ്പുകളുണ്ട്. കൃഷിയുണ്ട്. ഒറ്റയായും കൂട്ടായും വളരുന്ന ഈത്തപ്പനകളുണ്ട്. എല്ലാറ്റിനെയും നനയ്ക്കുന്നത് ഒരേ വെള്ളമാണ്. എന്നിട്ടും ചില പഴങ്ങളുടെ രുചി മറ്റുചിലതിന്റേതിനെക്കാള്‍ നാം വിശിഷ്ടമാക്കിയിരിക്കുന്നു. ചിന്തിക്കുന്ന ജനത്തിന് ഇതിലൊക്കെയും ധാരാളം ദൃഷ്ടാന്തങ്ങളുണ്ട്.
\end{malayalam}}
\flushright{\begin{Arabic}
\quranayah[13][5]
\end{Arabic}}
\flushleft{\begin{malayalam}
നീ അദ്ഭുതപ്പെടുന്നുവെങ്കില്‍ ജനത്തിന്റെ ഈ വാക്കാണ് ഏറെ അദ്ഭുതകരമായിട്ടുള്ളത്: "നാം മരിച്ചു മണ്ണായിക്കഴിഞ്ഞാല്‍ വീണ്ടും പുതുതായി സൃഷ്ടിക്കപ്പെടുമെന്നോ?” അവരാണ് തങ്ങളുടെ നാഥനില്‍ അവിശ്വസിച്ചവര്‍. അവരുടെ കണ്ഠങ്ങളില്‍ ചങ്ങലകളുണ്ട്. നരകാവകാശികളും അവര്‍ തന്നെ. അവരതില്‍ നിത്യവാസികളായിരിക്കും.
\end{malayalam}}
\flushright{\begin{Arabic}
\quranayah[13][6]
\end{Arabic}}
\flushleft{\begin{malayalam}
ഇക്കൂട്ടര്‍ നിന്നോട് നന്മക്കു മുമ്പെ തിന്മക്കായി തിടുക്കം കൂട്ടുന്നു. എന്നാല്‍ ഇവര്‍ക്കു മുമ്പ് ഗുണപാഠമുള്‍ക്കൊള്ളുന്ന ശിക്ഷകള്‍ എത്രയോ കഴിഞ്ഞുപോയിട്ടുണ്ട്. ജനം അതിക്രമം കാണിച്ചിട്ടും നിന്റെ നാഥന്‍ അവര്‍ക്ക് ഏറെ മാപ്പേകിയിട്ടുമുണ്ട്. നിന്റെ നാഥന്‍ കഠിനമായി ശിക്ഷിക്കുന്നവനുമാണ്.
\end{malayalam}}
\flushright{\begin{Arabic}
\quranayah[13][7]
\end{Arabic}}
\flushleft{\begin{malayalam}
സത്യനിഷേധികള്‍ ചോദിക്കുന്നു: "ഇയാള്‍ക്ക് ഇയാളുടെ നാഥനില്‍ നിന്ന് ഒരു ദൃഷ്ടാന്തവും ഇറക്കിക്കിട്ടാത്തതെന്ത്?” എന്നാല്‍ നീ ഒരു മുന്നറിയിപ്പുകാരന്‍ മാത്രമാണ്. എല്ലാ ജനതക്കുമുണ്ട് ഒരു വഴികാട്ടി.
\end{malayalam}}
\flushright{\begin{Arabic}
\quranayah[13][8]
\end{Arabic}}
\flushleft{\begin{malayalam}
ഓരോ സ്ത്രീയും ഗര്‍ഭാശയത്തില്‍ ചുമക്കുന്നതെന്തെന്ന് അല്ലാഹു അറിയുന്നു. ഗര്‍ഭാശയങ്ങള്‍ കുറവ് വരുത്തുന്നതും അധികരിപ്പിക്കുന്നതും അവന്നറിയാം. എല്ലാ കാര്യങ്ങള്‍ക്കും അവന്റെയടുത്ത് വ്യക്തമായ വ്യവസ്ഥകളുണ്ട്.
\end{malayalam}}
\flushright{\begin{Arabic}
\quranayah[13][9]
\end{Arabic}}
\flushleft{\begin{malayalam}
അവന്‍ ഒളിഞ്ഞതും തെളിഞ്ഞതും അറിയുന്നവനാണ്. മഹാനും ഉന്നതനുമാണ്.
\end{malayalam}}
\flushright{\begin{Arabic}
\quranayah[13][10]
\end{Arabic}}
\flushleft{\begin{malayalam}
നിങ്ങളിലെ മെല്ലെ സംസാരിക്കുന്നവനും ഉറക്കെ സംസാരിക്കുന്നവനും രാവില്‍ മറഞ്ഞിരിക്കുന്നവനും പകലില്‍ ഇറങ്ങിനടക്കുന്നവനുമെല്ലാം അവനെ സംബന്ധിച്ചിടത്തോളം സമമാണ്.
\end{malayalam}}
\flushright{\begin{Arabic}
\quranayah[13][11]
\end{Arabic}}
\flushleft{\begin{malayalam}
എല്ലാ ഓരോ മനുഷ്യന്റെയും മുന്നിലും പിന്നിലും അവന്നായി നിയോഗിക്കപ്പെട്ട മേല്‍നോട്ടക്കാരുണ്ട്. അല്ലാഹുവിന്റെ കല്‍പന പ്രകാരം അവരവനെ ശ്രദ്ധിച്ചുകൊണ്ടിരിക്കുന്നു. അല്ലാഹു ഒരു ജനതയുടെയും അവസ്ഥയില്‍ മാറ്റം വരുത്തുകയില്ല; അവര്‍ തങ്ങളുടെ സ്ഥിതി സ്വയം മാറ്റുംവരെ. എന്നാല്‍ അല്ലാഹു ഒരു ജനതക്ക് വല്ല ദുരിതവും വരുത്താനുദ്ദേശിച്ചാല്‍ ആര്‍ക്കും അത് തടുക്കാനാവില്ല. അവനെക്കൂടാതെ അവര്‍ക്ക് രക്ഷകനുമില്ല.
\end{malayalam}}
\flushright{\begin{Arabic}
\quranayah[13][12]
\end{Arabic}}
\flushleft{\begin{malayalam}
പേടിയും പ്രതീക്ഷയുമുണ്ടാക്കുന്ന മിന്നല്‍പ്പിണര്‍ നിങ്ങള്‍ക്കു കാണിച്ചുതരുന്നത് അവനാണ്. ജലവാഹിനികളായ കനത്ത കാര്‍മേഘങ്ങളുണ്ടാക്കുന്നതും അവന്‍ തന്നെ.
\end{malayalam}}
\flushright{\begin{Arabic}
\quranayah[13][13]
\end{Arabic}}
\flushleft{\begin{malayalam}
ഇടിനാദം അവനെ സ്തുതിച്ചുകൊണ്ട് അവന്റെ വിശുദ്ധിയെ വാഴ്ത്തുന്നു. അവനെ സംബന്ധിച്ച ഭയത്താല്‍ മലക്കുകളും അതുതന്നെ ചെയ്യുന്നു. അവന്‍ ഘോരഗര്‍ജനമുള്ള ഇടിവാളുകളയക്കുന്നു. അങ്ങനെ അവര്‍ അല്ലാഹുവിന്റെ കാര്യത്തില്‍ തര്‍ക്കിച്ചുകൊണ്ടിരിക്കെ അവനുദ്ദേശിക്കുന്നവരില്‍ അത് പതിക്കുന്നു. അതിശക്തമായി തന്ത്രം പ്രയോഗിക്കുന്നവനാണവന്‍.
\end{malayalam}}
\flushright{\begin{Arabic}
\quranayah[13][14]
\end{Arabic}}
\flushleft{\begin{malayalam}
അവനോടുള്ളതുമാത്രമാണ് യഥാര്‍ഥ പ്രാര്‍ഥന. അവനെക്കൂടാതെ ഇക്കൂട്ടര്‍ ആരോടൊക്കെ പ്രാര്‍ഥിച്ചുകൊണ്ടിരിക്കുന്നുവോ അവര്‍ക്കൊന്നും ഒരുത്തരവും നല്‍കാനാവില്ല. വെള്ളത്തിലേക്ക് ഇരുകൈകളും നീട്ടി അത് വായിലെത്താന്‍ കാത്തിരിക്കുന്നവനെപ്പോലെയാണവര്‍. വെള്ളം അങ്ങോട്ടെത്തുകയില്ലല്ലോ. സത്യനിഷേധികളുടെ പ്രാര്‍ഥന പൂര്‍ണമായും പാഴായതുതന്നെ.
\end{malayalam}}
\flushright{\begin{Arabic}
\quranayah[13][15]
\end{Arabic}}
\flushleft{\begin{malayalam}
ആകാശഭൂമികളിലുള്ളവരൊക്കെയും ഇഷ്ടത്തോടെയോ നിര്‍ബന്ധിതമായോ സാഷ്ടാംഗം ചെയ്തുകൊണ്ടിരിക്കുന്നതും അല്ലാഹുവിനാണ്. അവരുടെ നിഴലുകള്‍ പോലും രാവിലെയും വൈകുന്നേരവും അതുതന്നെ ചെയ്യുന്നു.
\end{malayalam}}
\flushright{\begin{Arabic}
\quranayah[13][16]
\end{Arabic}}
\flushleft{\begin{malayalam}
ചോദിക്കുക: ആരാണ് ആകാശഭൂമികളുടെ നാഥന്‍! പറയുക: അല്ലാഹു. അവരോട് പറയുക: എന്നിട്ടും സ്വന്തത്തിനുപോലും ഗുണമോ ദോഷമോ വരുത്താനാവാത്തവരെയാണോ നിങ്ങള്‍ അല്ലാഹുവെക്കൂടാതെ രക്ഷാധികാരികളാക്കിയിരിക്കുന്നത്? ചോദിക്കുക: കണ്ണുപൊട്ടനും കാഴ്ചയുള്ളവനും ഒരുപോലെയാണോ? ഇരുളും വെളിച്ചവും സമമാണോ? അതല്ല; അവരുടെ സാങ്കല്‍പിക സഹദൈവങ്ങള്‍ അല്ലാഹു സൃഷ്ടിക്കുന്നതുപോലെത്തന്നെ സൃഷ്ടി നടത്തുകയും അതുകണ്ട് ഇരുവിഭാഗത്തിന്റെയും സൃഷ്ടികളവര്‍ക്ക് തിരിച്ചറിയാതാവുകയുമാണോ ഉണ്ടായത്? പറയുക: എല്ലാ വസ്തുക്കളുടെയും സ്രഷ്ടാവ് അല്ലാഹുവാണ്. അവന്‍ ഏകനും എല്ലാറ്റിനെയും അതിജയിക്കുന്നവനുമാണ്.
\end{malayalam}}
\flushright{\begin{Arabic}
\quranayah[13][17]
\end{Arabic}}
\flushleft{\begin{malayalam}
അവന്‍ മാനത്തുനിന്നു വെള്ളമിറക്കി. അങ്ങനെ അരുവികളിലൂടെ അവയുടെ വലുപ്പത്തിന്റെ തോതനുസരിച്ച് അതൊഴുകി. ആ പ്രവാഹത്തിന്റെ ഉപരിതലത്തില്‍ പതയുണ്ട്. ആഭരണങ്ങളോ ഉപകരണങ്ങളോ ഉണ്ടാക്കാനായി അവര്‍ തീയിലിട്ടുരുക്കുന്നവയില്‍നിന്നും ഇതുപോലുള്ള പതയുണ്ടാകാറുണ്ട്. ഇവ്വിധമാണ് അല്ലാഹു സത്യത്തെയും അസത്യത്തെയും ഉപമിക്കുന്നത്. എന്നാല്‍ ആ പത വറ്റിപ്പോകുന്നു. ജനങ്ങള്‍ക്കുപകരിക്കുന്നത് ഭൂമിയില്‍ ബാക്കിയാവുകയും ചെയ്യും. അവ്വിധം അല്ലാഹു ഉദാഹരണങ്ങള്‍ സമര്‍പ്പിക്കുന്നു.
\end{malayalam}}
\flushright{\begin{Arabic}
\quranayah[13][18]
\end{Arabic}}
\flushleft{\begin{malayalam}
തങ്ങളുടെ നാഥന്റെ ക്ഷണം സ്വീകരിച്ചവര്‍ക്ക് എല്ലാ നന്മയുമുണ്ട്. അവന്റെ ക്ഷണം സ്വീകരിക്കാത്തവരോ, അവര്‍ക്ക് ഭൂമിയിലുള്ള സകലതും അതോടൊപ്പം അത്ര വേറെയും ഉണ്ടായാല്‍ പോലും ശിക്ഷ ഒഴിവാകാന്‍ അതൊക്കെയും അവര്‍ പിഴയായി ഒടുക്കുമായിരുന്നു. അവര്‍ക്കാണ് കടുത്ത വിചാരണയുള്ളത്. അവരുടെ താവളം നരകമാണ്. എത്ര ചീത്ത സങ്കേതം!
\end{malayalam}}
\flushright{\begin{Arabic}
\quranayah[13][19]
\end{Arabic}}
\flushleft{\begin{malayalam}
അപ്പോള്‍ നിന്റെ നാഥന്‍ നിനക്കിറക്കിത്തന്നത് സത്യമാണെന്ന് അറിയുന്നവന്‍ കുരുടനെപ്പോലെയാകുമോ? വിചാരശാലികള്‍ മാത്രമേ കാര്യങ്ങള്‍ നന്നായി മനസ്സിലാക്കുകയുള്ളൂ.
\end{malayalam}}
\flushright{\begin{Arabic}
\quranayah[13][20]
\end{Arabic}}
\flushleft{\begin{malayalam}
അല്ലാഹുവോടുള്ള വാഗ്ദാനം പൂര്‍ണമായും നിറവേറ്റുന്നവരാണവര്‍. കരാര്‍ ലംഘിക്കാത്തവരും.
\end{malayalam}}
\flushright{\begin{Arabic}
\quranayah[13][21]
\end{Arabic}}
\flushleft{\begin{malayalam}
ചേര്‍ത്തുവെക്കാന്‍ അല്ലാഹു കല്‍പിച്ച ബന്ധങ്ങളെയൊക്കെ കൂട്ടിയിണക്കുന്നവരാണവര്‍. തങ്ങളുടെ നാഥനെ ഭയപ്പെടുന്നവരും. കടുത്ത വിചാരണയെ പേടിക്കുന്നവരുമാണ്.
\end{malayalam}}
\flushright{\begin{Arabic}
\quranayah[13][22]
\end{Arabic}}
\flushleft{\begin{malayalam}
അവര്‍ തങ്ങളുടെ നാഥന്റെ പ്രീതി കാംക്ഷിച്ച് ക്ഷമപാലിക്കുന്നവരുമാണ്. നമസ്കാരം നിഷ്ഠയോടെ നിര്‍വഹിക്കുന്നവരും നാം നല്‍കിയ വിഭവങ്ങളില്‍ നിന്ന് രഹസ്യമായും പരസ്യമായും ചെലവഴിക്കുന്നവരുമാണ്. തിന്മയെ നന്മകൊണ്ടു തടയുന്നവരും. അവര്‍ക്കുള്ളതാണ് പരലോക നേട്ടം.
\end{malayalam}}
\flushright{\begin{Arabic}
\quranayah[13][23]
\end{Arabic}}
\flushleft{\begin{malayalam}
അതായത് സ്ഥിരവാസത്തിനുള്ള സ്വര്‍ഗീയാരാമങ്ങള്‍. അവരും അവരുടെ മാതാപിതാക്കളിലും ഇണകളിലും മക്കളിലുമുള്ള സദ്വൃത്തരും അതില്‍ പ്രവേശിക്കും. മലക്കുകള്‍ എല്ലാ കവാടങ്ങളിലൂടെയും അവരുടെ അടുത്തെത്തും.
\end{malayalam}}
\flushright{\begin{Arabic}
\quranayah[13][24]
\end{Arabic}}
\flushleft{\begin{malayalam}
മലക്കുകള്‍ പറയും: "നിങ്ങള്‍ ക്ഷമപാലിച്ചതിനാല്‍ നിങ്ങള്‍ക്ക് സമാധാനമുണ്ടാവട്ടെ.” ആ പരലോക ഭവനം എത്ര അനുഗ്രഹപൂര്‍ണം!
\end{malayalam}}
\flushright{\begin{Arabic}
\quranayah[13][25]
\end{Arabic}}
\flushleft{\begin{malayalam}
അല്ലാഹുവോടുള്ള കരാര്‍ ഉറപ്പിച്ചശേഷം ലംഘിക്കുകയും അവന്‍ കൂട്ടിയിണക്കാന്‍ കല്‍പിച്ചവയെ അറുത്തുമാറ്റുകയും ഭൂമിയില്‍ കുഴപ്പമുണ്ടാക്കുകയും ചെയ്യുന്നവര്‍ക്ക് ശാപം. അവര്‍ക്കുണ്ടാവുക ഏറ്റവും ചീത്തയായ പാര്‍പ്പിടമാണ്.
\end{malayalam}}
\flushright{\begin{Arabic}
\quranayah[13][26]
\end{Arabic}}
\flushleft{\begin{malayalam}
അല്ലാഹു അവനിച്ഛിക്കുന്നവര്‍ക്ക് വിഭവങ്ങള്‍ സമൃദ്ധമായി നല്‍കുന്നു. വേറെ ചിലര്‍ക്കത് പരിമിതപ്പെടുത്തുന്നു. അവര്‍ ഈലോകജീവിതം കൊണ്ടുതന്നെ തൃപ്തിപ്പെട്ടിരിക്കുന്നു. എന്നാല്‍ പരലോകത്തെ അപേക്ഷിച്ച് ഐഹിക ജീവിതം നന്നെ തുച്ഛമായ വിഭവം മാത്രമാണ്.
\end{malayalam}}
\flushright{\begin{Arabic}
\quranayah[13][27]
\end{Arabic}}
\flushleft{\begin{malayalam}
സത്യനിഷേധികള്‍ പറയുന്നു: "ഇയാള്‍ക്ക് എന്തുകൊണ്ടാണ് ഇയാളുടെ നാഥനില്‍നിന്ന് ഒരടയാളവും ഇറക്കിക്കിട്ടാത്തത്?” പറയുക: "തീര്‍ച്ചയായും അല്ലാഹു അവനിച്ഛിക്കുന്നവരെ വഴികേടിലാക്കുന്നു. പശ്ചാത്തപിച്ചു മടങ്ങുന്നവരെ അവന്‍ തന്നിലേക്കുള്ള നേര്‍വഴിയില്‍ നയിക്കുകയും ചെയ്യുന്നു.”
\end{malayalam}}
\flushright{\begin{Arabic}
\quranayah[13][28]
\end{Arabic}}
\flushleft{\begin{malayalam}
സത്യവിശ്വാസം സ്വീകരിക്കുകയും ദൈവസ്മരണയാല്‍ മനസ്സുകള്‍ ശാന്തമാവുകയും ചെയ്യുന്നവരാണവര്‍. അറിയുക: ദൈവസ്മരണകൊണ്ട് മാത്രമാണ് മനസ്സുകള്‍ ശാന്തമാകുന്നത്.
\end{malayalam}}
\flushright{\begin{Arabic}
\quranayah[13][29]
\end{Arabic}}
\flushleft{\begin{malayalam}
സത്യവിശ്വാസം സ്വീകരിക്കുകയും സല്‍ക്കര്‍മങ്ങള്‍ പ്രവര്‍ത്തിക്കുകയും ചെയ്തവര്‍ ഭാഗ്യവാന്മാര്‍. അവര്‍ക്ക് തിരിച്ചെത്താനുള്ളത് ഏറ്റം മികച്ച താവളം തന്നെ.
\end{malayalam}}
\flushright{\begin{Arabic}
\quranayah[13][30]
\end{Arabic}}
\flushleft{\begin{malayalam}
അവ്വിധം, നിന്നെ നാമൊരു സമുദായത്തിലേക്ക് ദൂതനായി നിയോഗിച്ചിരിക്കുന്നു. ഇതിനുമുമ്പും നിരവധി സമുദായങ്ങള്‍ കഴിഞ്ഞുപോയിട്ടുണ്ട്. നാം നിനക്കു ബോധനമായി നല്‍കിയ സന്ദേശം നീയവര്‍ക്ക് വായിച്ചുകേള്‍പ്പിക്കാന്‍ വേണ്ടിയാണിത്. അവരോ, ദയാപരനായ ദൈവത്തെ തള്ളിപ്പറയുന്നു. പറയുക: അവനാണെന്റെ നാഥന്‍! അവനല്ലാതെ ദൈവമില്ല. ഞാന്‍ അവനില്‍ ഭരമേല്‍പിച്ചിരിക്കുന്നു. എന്റെ തിരിച്ചുപോക്കും അവനിലേക്കുതന്നെ.
\end{malayalam}}
\flushright{\begin{Arabic}
\quranayah[13][31]
\end{Arabic}}
\flushleft{\begin{malayalam}
പര്‍വതങ്ങളെ ചലിപ്പിക്കുകയോ ഭൂമിയെ തുണ്ടം തുണ്ടമാക്കി മുറിക്കുകയോ മരിച്ചവരോടു സംസാരിക്കുകയോ ചെയ്യാന്‍ കഴിവുറ്റഒരു ഖുര്‍ആന്‍ ഉണ്ടായാല്‍പ്പോലും അവരതില്‍ വിശ്വസിക്കുമായിരുന്നില്ല. എന്നാല്‍ കാര്യങ്ങളൊക്കെ അല്ലാഹുവിന്റെ നിയന്ത്രണത്തിലാണ്. സത്യവിശ്വാസികള്‍ മനസ്സിലാക്കുന്നില്ലേ; അല്ലാഹു ഇച്ഛിച്ചിരുന്നെങ്കില്‍ അവന്‍ മുഴുവന്‍ മനുഷ്യരെയും നേര്‍വഴിയിലാക്കുമായിരുന്നു. സത്യനിഷേധികള്‍ക്ക് തങ്ങള്‍ പ്രവര്‍ത്തിച്ചുകൊണ്ടിരുന്നതിന്റെ ഫലമായി എന്തെങ്കിലും വിപത്ത് ബാധിച്ചുകൊണ്ടിരിക്കും. അല്ലെങ്കില്‍ അവരുടെ വീടുകള്‍ക്ക് അടുത്തുതന്നെ ദുരിതം വന്നുപതിക്കും; അല്ലാഹുവിന്റെ വാഗ്ദാനം പുലരുംവരെ. അല്ലാഹു ഒരിക്കലും വാഗ്ദാനം ലംഘിക്കുകയില്ല.
\end{malayalam}}
\flushright{\begin{Arabic}
\quranayah[13][32]
\end{Arabic}}
\flushleft{\begin{malayalam}
നിനക്കു മുമ്പും നിരവധി ദൈവദൂതന്മാര്‍ പരിഹസിക്കപ്പെട്ടിട്ടുണ്ട്. അപ്പോഴൊക്കെ നാം സത്യനിഷേധികള്‍ക്ക് അവസരം നീട്ടിക്കൊടുത്തുകൊണ്ടിരുന്നു. പിന്നീട് നാം അവരെ പിടികൂടി. നോക്കൂ: എന്റെ ശിക്ഷ എവ്വിധമായിരുന്നുവെന്ന്!
\end{malayalam}}
\flushright{\begin{Arabic}
\quranayah[13][33]
\end{Arabic}}
\flushleft{\begin{malayalam}
അപ്പോള്‍ ഓരോ ആത്മാവും സമ്പാദിച്ചുകൊണ്ടിരിക്കുന്നതിന് മേല്‍നോട്ടം വഹിക്കുന്നവനോടാണോ ഈ ധിക്കാരം? അവര്‍ അല്ലാഹുവിന് പങ്കാളികളെ ആരോപിച്ചിരിക്കുന്നു. പറയുക: നിങ്ങള്‍ അവരുടെ പേരുകളൊന്നു പറഞ്ഞുതരിക. അല്ല; അല്ലാഹുവിന് ഭൂമിയില്‍ അറിയാത്ത കാര്യം അറിയിച്ചുകൊടുക്കുകയാണോ നിങ്ങള്‍? അതല്ല; തോന്നുന്നതൊക്കെ വിളിച്ചുപറയുകയാണോ? എന്നാല്‍ വസ്തുത അതൊന്നുമല്ല; സത്യനിഷേധികള്‍ക്ക് അവരുടെ കുതന്ത്രം കൌതുകകരമായി തോന്നിയിരിക്കുന്നു. സത്യപാതയില്‍നിന്ന് അവര്‍ തടയപ്പെടുകയും ചെയ്തിരിക്കുന്നു. അല്ലാഹു ആരെയെങ്കിലും ദുര്‍മാര്‍ഗത്തിലാക്കുകയാണെങ്കില്‍ പിന്നെ അവനെ നേര്‍വഴിയിലാക്കുന്ന ആരുമില്ല.
\end{malayalam}}
\flushright{\begin{Arabic}
\quranayah[13][34]
\end{Arabic}}
\flushleft{\begin{malayalam}
അവര്‍ക്ക് ഐഹികജീവിതത്തില്‍ അര്‍ഹമായ ശിക്ഷയുണ്ട്. പരലോക ശിക്ഷയോ ഏറെ ദുരിത പൂര്‍ണവും. അല്ലാഹുവില്‍ നിന്ന് അവരെ രക്ഷിക്കാന്‍ ആരുമുണ്ടാവില്ല.
\end{malayalam}}
\flushright{\begin{Arabic}
\quranayah[13][35]
\end{Arabic}}
\flushleft{\begin{malayalam}
ഭക്തന്മാര്‍ക്കു വാഗ്ദാനം ചെയ്യപ്പെട്ട സ്വര്‍ഗത്തിന്റെ ഉപമ ഇതാണ്: അതിന്റെ താഴ്ഭാഗത്തൂടെ ആറുകളൊഴുകിക്കൊണ്ടിരിക്കും. അതിലെ കനികളും തണലും ശാശ്വതമായിരിക്കും. ദൈവഭക്തന്മാരുടെ മടക്കം അവിടേക്കാണ്. സത്യനിഷേധികളുടെ ഒടുക്കമോ നരകത്തീയിലും.
\end{malayalam}}
\flushright{\begin{Arabic}
\quranayah[13][36]
\end{Arabic}}
\flushleft{\begin{malayalam}
നാം നേരത്തെ വേദപുസ്തകം നല്‍കിയവര്‍ നിനക്ക് ഇറക്കിയ ഈ വേദപുസ്തകത്തില്‍ സന്തുഷ്ടരാണ്. എന്നാല്‍ സഖ്യകക്ഷികളില്‍ ചിലര്‍ ഇതിന്റെ ചില ഭാഗങ്ങള്‍ അംഗീകരിക്കാത്തവരാണ്. പറയുക: "ഞാന്‍ അല്ലാഹുവിനു മാത്രം വഴിപ്പെടാനാണ് കല്‍പിക്കപ്പെട്ടിരിക്കുന്നത്. അവനില്‍ ഒന്നും പങ്കുചേര്‍ക്കാതിരിക്കാനും. അതിനാല്‍ ഞാന്‍ ക്ഷണിക്കുന്നത് അവനിലേക്കാണ്. എന്റെ മടക്കവും അവങ്കലേക്കുതന്നെ.”
\end{malayalam}}
\flushright{\begin{Arabic}
\quranayah[13][37]
\end{Arabic}}
\flushleft{\begin{malayalam}
ഇവ്വിധം നാം ഇതിനെ ഒരു ന്യായപ്രമാണമായി അറബിഭാഷയില്‍ ഇറക്കിയിരിക്കുന്നു. നിനക്ക് ഈ അറിവ് വന്നെത്തിയശേഷവും നീ അവരുടെ ഇച്ഛകളെ പിന്‍പറ്റുകയാണെങ്കില്‍ അല്ലാഹുവിന്റെ ശിക്ഷയില്‍നിന്ന് നിന്നെ കാക്കുന്ന ഒരു രക്ഷകനോ കാവല്‍ക്കാരനോ നിനക്കുണ്ടാവില്ല.
\end{malayalam}}
\flushright{\begin{Arabic}
\quranayah[13][38]
\end{Arabic}}
\flushleft{\begin{malayalam}
നിനക്കുമുമ്പും നാം ദൂതന്മാരെ നിയോഗിച്ചിട്ടുണ്ട്. അവര്‍ക്കു നാം ഇണകളെയും സന്താനങ്ങളെയും നല്‍കിയിട്ടുമുണ്ട്. ഒരു ദൈവദൂതന്നും അല്ലാഹുവിന്റെ അനുമതിയോടെയല്ലാതെ ഒരു ദൃഷ്ടാന്തവും കൊണ്ടുവരാനാവില്ല. എല്ലാ കാലഘട്ടത്തിനും ഒരു പ്രമാണമുണ്ട്.
\end{malayalam}}
\flushright{\begin{Arabic}
\quranayah[13][39]
\end{Arabic}}
\flushleft{\begin{malayalam}
അല്ലാഹു അവനിച്ഛിക്കുന്നതിനെ മായ്ച്ചുകളയുന്നു. അവനിച്ഛിക്കുന്നത് നിലനിര്‍ത്തുകയും ചെയ്യുന്നു. എല്ലാറ്റിനും ആധാരമായ മൂലപ്രമാണം അവന്റെ അടുത്താണുള്ളത്.
\end{malayalam}}
\flushright{\begin{Arabic}
\quranayah[13][40]
\end{Arabic}}
\flushleft{\begin{malayalam}
അവര്‍ക്കു നാം മുന്നറിയിപ്പ് നല്‍കിക്കൊണ്ടിരിക്കുന്ന ശിക്ഷയില്‍ ചിലത് നിനക്കു നാം കാണിച്ചുതന്നേക്കാം. അല്ലെങ്കില്‍ അതിനുമുമ്പെ നിന്റെ ജീവിതം നാം അവസാനിപ്പിച്ചേക്കാം. എന്തായാലും നമ്മുടെ സന്ദേശം എത്തിക്കേണ്ട ചുമതല മാത്രമേ നിനക്കുള്ളൂ. കണക്കുനോക്കുന്ന പണി നമ്മുടേതാണ്.
\end{malayalam}}
\flushright{\begin{Arabic}
\quranayah[13][41]
\end{Arabic}}
\flushleft{\begin{malayalam}
നാം ഈ ഭൂമിയെ അതിന്റെ നാനാഭാഗത്തുനിന്നും ചുരുക്കിക്കൊണ്ടുവരുന്നത് അവര്‍ കാണുന്നില്ലേ? അല്ലാഹു എല്ലാം തീരുമാനിക്കുന്നു. അവന്റെ തീരുമാനം മാറ്റിമറിക്കാനാരുമില്ല. അവന്‍ അതിവേഗം കണക്കുനോക്കുന്നവനാണ്.
\end{malayalam}}
\flushright{\begin{Arabic}
\quranayah[13][42]
\end{Arabic}}
\flushleft{\begin{malayalam}
ഇവര്‍ക്കു മുമ്പുണ്ടായിരുന്നവരും പല തന്ത്രങ്ങളും പയറ്റിയിട്ടുണ്ട്. എന്നാല്‍ ഫലവത്തായ തന്ത്രങ്ങളൊക്കെയും അല്ലാഹുവിന്റേതാണ്. ഓരോ മനുഷ്യനും നേടിക്കൊണ്ടിരിക്കുന്നതെന്തെന്നു അവന്‍ നന്നായറിയുന്നു. ശോഭനമായ അന്ത്യം ആരുടേതാണെന്ന് ഈ സത്യനിഷേധികള്‍ അടുത്തുതന്നെ അറിയും.
\end{malayalam}}
\flushright{\begin{Arabic}
\quranayah[13][43]
\end{Arabic}}
\flushleft{\begin{malayalam}
സത്യനിഷേധികള്‍ പറയുന്നു, നിന്നെ ദൈവം അയച്ചതല്ലെന്ന്. പറയുക: എനിക്കും നിങ്ങള്‍ക്കുമിടയില്‍ സാക്ഷിയായി അല്ലാഹു മതി. വേദവിജ്ഞാനമുള്ളവരും.
\end{malayalam}}
\chapter{\textmalayalam{ഇബ്രാഹീം}}
\begin{Arabic}
\Huge{\centerline{\basmalah}}\end{Arabic}
\flushright{\begin{Arabic}
\quranayah[14][1]
\end{Arabic}}
\flushleft{\begin{malayalam}
അലിഫ് - ലാം - റാഅ്. ഇത് നാം നിനക്കിറക്കിയ വേദപുസ്തകമാണ്. ജനങ്ങളെ അവരുടെ നാഥന്റെ അനുമതിയോടെ ഇരുളില്‍നിന്ന് വെളിച്ചത്തിലേക്ക് നയിക്കാന്‍. പ്രതാപിയും സ്തുത്യര്‍ഹനുമായവന്റെ മാര്‍ഗത്തിലേക്ക്.
\end{malayalam}}
\flushright{\begin{Arabic}
\quranayah[14][2]
\end{Arabic}}
\flushleft{\begin{malayalam}
ആകാശഭൂമികളിലുള്ളവയുടെയെല്ലാം ഉടമയായ അല്ലാഹുവിന്റെ മാര്‍ഗത്തിലേക്ക്. സത്യനിഷേധികള്‍ക്ക് കഠിനശിക്ഷയുടെ കൊടും നാശമാണുണ്ടാവുക.
\end{malayalam}}
\flushright{\begin{Arabic}
\quranayah[14][3]
\end{Arabic}}
\flushleft{\begin{malayalam}
പരലോകത്തെക്കാള്‍ ഇഹലോക ജീവിതത്തെ സ്നേഹിക്കുന്നവരാണവര്‍. ദൈവമാര്‍ഗത്തില്‍ നിന്ന് ജനത്തെ തടഞ്ഞുനിര്‍ത്തുന്നവരും ദൈവമാര്‍ഗം വികലമാകണമെന്നാഗ്രഹിക്കുന്നവരുമാണ്. അവര്‍ വഴികേടില്‍ ഏറെദൂരം പിന്നിട്ടിരിക്കുന്നു.
\end{malayalam}}
\flushright{\begin{Arabic}
\quranayah[14][4]
\end{Arabic}}
\flushleft{\begin{malayalam}
നാം നിയോഗിച്ച ഒരു ദൂതന്നും തന്റെ ജനതയുടെ ഭാഷയിലല്ലാതെ സന്ദേശം നല്‍കിയിട്ടില്ല. അവര്‍ക്കത് വിവരിച്ചുകൊടുക്കാനാണ് അങ്ങനെ ചെയ്തത്. അല്ലാഹു അവനിച്ഛിക്കുന്നവരെ വഴികേടിലാക്കുന്നു. അവനിച്ഛിക്കുന്നവരെ നേര്‍വഴിയിലാക്കുകയും ചെയ്യുന്നു. അവന്‍ ഏറെ പ്രതാപിയും യുക്തിമാനും തന്നെ.
\end{malayalam}}
\flushright{\begin{Arabic}
\quranayah[14][5]
\end{Arabic}}
\flushleft{\begin{malayalam}
മൂസയെ നാം നമ്മുടെ വചനങ്ങളുമായി അയച്ചു. നാം പറഞ്ഞു: നീ നിന്റെ ജനത്തെ ഇരുളില്‍നിന്ന് വെളിച്ചത്തിലേക്കു നയിക്കുക. അല്ലാഹുവിന്റെ സവിശേഷമായ നാളുകളെപ്പറ്റി അവരെ ഓര്‍മിപ്പിക്കുക. തികഞ്ഞ ക്ഷമയുള്ളവര്‍ക്കും നിറഞ്ഞ നന്ദിയുള്ളവര്‍ക്കും അതില്‍ നിരവധി തെളിവുകളുണ്ട്.
\end{malayalam}}
\flushright{\begin{Arabic}
\quranayah[14][6]
\end{Arabic}}
\flushleft{\begin{malayalam}
മൂസ തന്റെ ജനതയോടു പറഞ്ഞ സന്ദര്‍ഭം: "അല്ലാഹു നിങ്ങള്‍ക്കേകിയ അനുഗ്രഹങ്ങള്‍ ഓര്‍ക്കുക: ഫറവോന്റെ ആള്‍ക്കാരില്‍ നിന്ന് അവന്‍ നിങ്ങളെ രക്ഷിച്ച കാര്യം. അവര്‍ നിങ്ങളെ കഠിനമായി പീഡിപ്പിക്കുകയായിരുന്നു. നിങ്ങളുടെ ആണ്‍മക്കളെ അറുകൊല നടത്തുകയും പെണ്ണുങ്ങളെ ജീവിക്കാന്‍ വിടുകയുമായിരുന്നു. നിങ്ങള്‍ക്കതില്‍ നിങ്ങളുടെ നാഥനില്‍ നിന്നുള്ള വമ്പിച്ച പരീക്ഷണമുണ്ട്.
\end{malayalam}}
\flushright{\begin{Arabic}
\quranayah[14][7]
\end{Arabic}}
\flushleft{\begin{malayalam}
"നിങ്ങളുടെ നാഥനിങ്ങനെ വിളംബരം ചെയ്ത സന്ദര്‍ഭം: “നിങ്ങള്‍ നന്ദി കാണിക്കുകയാണെങ്കില്‍ ഞാന്‍ നിങ്ങള്‍ക്ക് അനുഗ്രഹങ്ങള്‍ ധാരാളമായി നല്‍കും; അഥവാ, നന്ദികേടു കാണിക്കുകയാണെങ്കില്‍ എന്റെ ശിക്ഷ കടുത്തതായിരിക്കുകയും ചെയ്യും.”
\end{malayalam}}
\flushright{\begin{Arabic}
\quranayah[14][8]
\end{Arabic}}
\flushleft{\begin{malayalam}
മൂസ പറഞ്ഞു: "നിങ്ങളും ഭൂമിയിലുള്ളവരൊക്കെയും സത്യനിഷേധികളായാല്‍പ്പോലും അല്ലാഹു തീര്‍ത്തും സ്വയംപര്യാപ്തനാണ്. സ്തുത്യര്‍ഹനും.”
\end{malayalam}}
\flushright{\begin{Arabic}
\quranayah[14][9]
\end{Arabic}}
\flushleft{\begin{malayalam}
നിങ്ങളുടെ മുന്‍ഗാമികളുടെ വര്‍ത്തമാനം നിങ്ങള്‍ക്ക് വന്നെത്തിയിട്ടില്ലേ; നൂഹിന്റെ ജനതയുടെയും ആദ്, സമൂദ് ഗോത്രങ്ങളുടെയും അവര്‍ക്കുശേഷമുള്ള, കൃത്യമായി അല്ലാഹുവിനു മാത്രമറിയാവുന്ന സമുദായങ്ങളുടെയും വാര്‍ത്ത. അവരിലേക്കുള്ള നമ്മുടെ ദൂതന്മാര്‍ വ്യക്തമായ തെളിവുകളുമായി അവരുടെയടുത്ത് ചെന്നു. അപ്പോഴവര്‍ കൈവിരലുകള്‍ തങ്ങളുടെ തന്നെ വായില്‍ തിരുകിക്കയറ്റി. എന്നിട്ടിങ്ങനെ പറഞ്ഞു: "ഏതൊരു സന്ദേശവുമായാണോ നിങ്ങളെ അയച്ചിരിക്കുന്നത് അതിനെ ഞങ്ങളിതാ കള്ളമാക്കിത്തള്ളുന്നു. ഏതൊന്നിലേക്കാണോ ഞങ്ങളെ നിങ്ങള്‍ വിളിക്കുന്നത് അതേപ്പറ്റി ഞങ്ങള്‍ ആശങ്കാപൂര്‍ണമായ സംശയത്തിലാണ്.”
\end{malayalam}}
\flushright{\begin{Arabic}
\quranayah[14][10]
\end{Arabic}}
\flushleft{\begin{malayalam}
അവര്‍ക്കുള്ള ദൈവദൂതന്മാര്‍ പറഞ്ഞു: "ആകാശഭൂമികളുടെ സ്രഷ്ടാവായ അല്ലാഹുവിന്റെ കാര്യത്തിലാണോ നിങ്ങള്‍ക്കു സംശയം? അറിയുക: നിങ്ങളുടെ പാപങ്ങള്‍ പൊറുത്തുതരാനും നിശ്ചിത അവധിവരെ നിങ്ങള്‍ക്ക് അവസരം നീട്ടിത്തരാനുമായി അവന്‍ നിങ്ങളെ ക്ഷണിച്ചുകൊണ്ടിരിക്കുന്നു.” ആ ജനം പറഞ്ഞു: "നിങ്ങള്‍ ഞങ്ങളെപ്പോലുള്ള മനുഷ്യര്‍ മാത്രമാണ്. ഞങ്ങളുടെ പിതാക്കള്‍ പൂജിച്ചിരുന്നവയില്‍ നിന്ന് ഞങ്ങളെ പിന്തിരിപ്പിക്കാനാണ് നിങ്ങളുദ്ദേശിക്കുന്നത്. അതിനാല്‍ വ്യക്തമായ എന്തെങ്കിലും തെളിവ് കൊണ്ടുവരൂ.”
\end{malayalam}}
\flushright{\begin{Arabic}
\quranayah[14][11]
\end{Arabic}}
\flushleft{\begin{malayalam}
അവര്‍ക്കുള്ള ദൈവദൂതന്മാര്‍ അവരോടു പറഞ്ഞു: "ഞങ്ങള്‍ നിങ്ങളെപ്പോലുള്ള മനുഷ്യര്‍ മാത്രമാണ്. എന്നാല്‍ അല്ലാഹു തന്റെ ദാസന്മാരില്‍ താനിച്ഛിക്കുന്നവരെ പ്രത്യേകം അനുഗ്രഹിക്കുന്നു. ദൈവഹിതമനുസരിച്ചല്ലാതെ നിങ്ങള്‍ക്ക് ഒരു തെളിവും കൊണ്ടുവന്നുതരാന്‍ ഞങ്ങള്‍ക്കാവില്ല. വിശ്വാസികള്‍ അല്ലാഹുവിലാണ് ഭരമേല്‍പിക്കേണ്ടത്.
\end{malayalam}}
\flushright{\begin{Arabic}
\quranayah[14][12]
\end{Arabic}}
\flushleft{\begin{malayalam}
"ഞങ്ങള്‍ എന്തിന് അല്ലാഹുവില്‍ ഭരമേല്‍പിക്കാതിരിക്കണം? ഞങ്ങളെ അവന്‍ ഞങ്ങള്‍ക്കാവശ്യമായ നേര്‍വഴിയിലാക്കിയിരിക്കുന്നു. നിങ്ങള്‍ ഞങ്ങള്‍ക്കേല്‍പിക്കുന്ന ദ്രോഹം ഞങ്ങള്‍ ക്ഷമിക്കുക തന്നെ ചെയ്യും. ഭരമേല്‍പിക്കുന്നവരൊക്കെയും അല്ലാഹുവില്‍ ഭരമേല്‍പിച്ചുകൊള്ളട്ടെ.”
\end{malayalam}}
\flushright{\begin{Arabic}
\quranayah[14][13]
\end{Arabic}}
\flushleft{\begin{malayalam}
സത്യനിഷേധികള്‍ തങ്ങളുടെ ദൈവദൂതന്മാരോടു പറഞ്ഞു: "നിങ്ങളെ ഞങ്ങള്‍ ഞങ്ങളുടെ നാട്ടില്‍ നിന്ന് പുറത്താക്കും. അല്ലെങ്കില്‍ നിങ്ങള്‍ ഞങ്ങളുടെ മതത്തിലേക്കുതന്നെ തിരിച്ചുവരണം.” അപ്പോള്‍ അവരുടെ നാഥന്‍ അവര്‍ക്ക് ബോധനം നല്‍കി: "ഈ അക്രമികളെ നാം നശിപ്പിക്കുകതന്നെ ചെയ്യും.
\end{malayalam}}
\flushright{\begin{Arabic}
\quranayah[14][14]
\end{Arabic}}
\flushleft{\begin{malayalam}
"അവര്‍ക്കുശേഷം നിങ്ങളെ നാം ഈ നാട്ടില്‍ താമസിപ്പിക്കും. വിധിദിനത്തിലെ എന്റെ സ്ഥാനത്തെ ഭയപ്പെടുകയും എന്റെ താക്കീതിനെ പേടിക്കുകയും ചെയ്യുന്നവര്‍ക്കുള്ള ഔദാര്യമാണിത്.”
\end{malayalam}}
\flushright{\begin{Arabic}
\quranayah[14][15]
\end{Arabic}}
\flushleft{\begin{malayalam}
ആ ദൈവദൂതന്മാര്‍ വിജയത്തിനായി പ്രാര്‍ഥിച്ചു. ധിക്കാരികളായ സ്വേഛാധിപതികളൊക്കെ തോറ്റമ്പി.
\end{malayalam}}
\flushright{\begin{Arabic}
\quranayah[14][16]
\end{Arabic}}
\flushleft{\begin{malayalam}
ഇതിനു പിന്നാലെ കത്തിയെരിയുന്ന നരകത്തീയുണ്ട്. ചോരയും ചലവും ചേര്‍ന്ന നീരാണവിടെ കുടിക്കാന്‍ കിട്ടുക.
\end{malayalam}}
\flushright{\begin{Arabic}
\quranayah[14][17]
\end{Arabic}}
\flushleft{\begin{malayalam}
അത് കുടിച്ചിറക്കാനവന്‍ ശ്രമിക്കും. എന്നാല്‍, വളരെ വിഷമിച്ചേ അവന്നത് തൊണ്ടയില്‍ നിന്നിറക്കാനാവൂ. നാനാഭാഗത്തുനിന്നും മരണം അവന്റെ നേരെ വരും. എന്നാലൊട്ടു മരിക്കുകയുമില്ല. ഇതിനുപിറകെ കഠിനമായ ശിക്ഷ വേറെയുമുണ്ട്.
\end{malayalam}}
\flushright{\begin{Arabic}
\quranayah[14][18]
\end{Arabic}}
\flushleft{\begin{malayalam}
തങ്ങളുടെ നാഥനെ കള്ളമാക്കിത്തള്ളിയവരുടെ ഉദാഹരണമിതാ: അവരുടെ പ്രവര്‍ത്തനങ്ങള്‍, കൊടുങ്കാറ്റുള്ള നാളില്‍ കാറ്റടിച്ചു പാറിപ്പോയ വെണ്ണീറുപോലെയാണ്. അവര്‍ നേടിയതൊന്നും അവര്‍ക്ക് ഉപകരിക്കുകയില്ല. ഇതുതന്നെയാണ് അതിരുകളില്ലാത്ത മാര്‍ഗഭ്രംശം.
\end{malayalam}}
\flushright{\begin{Arabic}
\quranayah[14][19]
\end{Arabic}}
\flushleft{\begin{malayalam}
വളരെ കൃത്യതയോടെ അല്ലാഹു ആകാശഭൂമികളെ സൃഷ്ടിച്ചത് നീ കാണുന്നില്ലേ. അവനിച്ഛിക്കുന്നുവെങ്കില്‍ നിങ്ങളെ തുടച്ചുമാറ്റി പകരം പുതിയ സൃഷ്ടികളെ അവന്‍ കൊണ്ടുവരും.
\end{malayalam}}
\flushright{\begin{Arabic}
\quranayah[14][20]
\end{Arabic}}
\flushleft{\begin{malayalam}
അല്ലാഹുവിനിതൊട്ടും പ്രയാസകരമല്ല
\end{malayalam}}
\flushright{\begin{Arabic}
\quranayah[14][21]
\end{Arabic}}
\flushleft{\begin{malayalam}
അവരെല്ലാവരും അല്ലാഹുവിങ്കല്‍ മറയില്ലാതെ പ്രത്യക്ഷപ്പെടും. അപ്പോള്‍ ഈ ലോകത്ത് ദുര്‍ബലരായിരുന്നവര്‍, അഹങ്കരിച്ചുകഴിഞ്ഞിരുന്നവരോടു പറയും: "തീര്‍ച്ചയായും ഞങ്ങള്‍ നിങ്ങളുടെ അനുയായികളായിരുന്നുവല്ലോ. അതിനാലിപ്പോള്‍ അല്ലാഹുവിന്റെ ശിക്ഷയില്‍നിന്ന് ഞങ്ങള്‍ക്ക് എന്തെങ്കിലും ഇളവ് ഉണ്ടാക്കിത്തരുമോ?” അവര്‍ പറയും: "അല്ലാഹു ഞങ്ങള്‍ക്കു വല്ല രക്ഷാമാര്‍ഗവും കാണിച്ചുതന്നിരുന്നെങ്കില്‍ ഞങ്ങള്‍ നിങ്ങള്‍ക്കും രക്ഷാമാര്‍ഗം കാണിച്ചുതരുമായിരുന്നു. ഇനി നാം വെപ്രാളപ്പെടുന്നതും ക്ഷമ പാലിക്കുന്നതും സമമാണ്. നമുക്കു രക്ഷപ്പെടാനൊരു പഴുതുമില്ല.”
\end{malayalam}}
\flushright{\begin{Arabic}
\quranayah[14][22]
\end{Arabic}}
\flushleft{\begin{malayalam}
വിധി തീര്‍പ്പുണ്ടായിക്കഴിഞ്ഞാല്‍ പിശാച് പറയും: "അല്ലാഹു നിങ്ങള്‍ക്ക് സത്യമായ വാഗ്ദാനമാണ് നല്‍കിയത്. ഞാനും നിങ്ങള്‍ക്ക് വാഗ്ദാനം നല്‍കിയിരുന്നു. പക്ഷേ, ഞാനത് ലംഘിച്ചു. എനിക്ക് നിങ്ങളുടെമേല്‍ ഒരധികാരവുമുണ്ടായിരുന്നില്ല. ഞാന്‍ നിങ്ങളെ ക്ഷണിച്ചുവെന്നുമാത്രം. അപ്പോള്‍ നിങ്ങളെനിക്ക് ഉത്തരം നല്‍കി. അതിനാല്‍ നിങ്ങള്‍ എന്നെ കുറ്റപ്പെടുത്തേണ്ട. നിങ്ങളെത്തന്നെ കുറ്റപ്പെടുത്തിയാല്‍ മതി. എനിക്കു നിങ്ങളെ രക്ഷിക്കാനാവില്ല. നിങ്ങള്‍ക്ക് എന്നെയും രക്ഷിക്കാനാവില്ല. നേരത്തെ നിങ്ങളെന്നെ അല്ലാഹുവിന് പങ്കാളിയാക്കിയിരുന്നതിനെ ഞാനിതാ നിഷേധിക്കുന്നു.” തീര്‍ച്ചയായും അക്രമികള്‍ക്ക് നോവേറിയ ശിക്ഷയുണ്ട്.
\end{malayalam}}
\flushright{\begin{Arabic}
\quranayah[14][23]
\end{Arabic}}
\flushleft{\begin{malayalam}
സത്യവിശ്വാസം സ്വീകരിക്കുകയും സല്‍ക്കര്‍മങ്ങള്‍ പ്രവര്‍ത്തിക്കുകയും ചെയ്തവര്‍ താഴ്ഭാഗത്തൂടെ ആറുകളൊഴുകുന്ന സ്വര്‍ഗീയാരാമങ്ങളില്‍ പ്രവേശിക്കും. തങ്ങളുടെ നാഥന്റെ ഹിതമനുസരിച്ച് അവരവിടെ നിത്യവാസികളായിരിക്കും. അവിടെയവരുടെഅഭിവാദ്യം സമാധാനത്തിന്റേതായിരിക്കും.
\end{malayalam}}
\flushright{\begin{Arabic}
\quranayah[14][24]
\end{Arabic}}
\flushleft{\begin{malayalam}
ഉത്തമ വചനത്തിന് അല്ലാഹു നല്‍കിയ ഉദാഹരണം എങ്ങനെയെന്ന് നീ കാണുന്നില്ലേ? അത് നല്ല ഒരു മരംപോലെയാണ്. അതിന്റെ വേരുകള്‍ ഭൂമിയില്‍ ആണ്ടിറങ്ങിയിരിക്കുന്നു. ശാഖകള്‍ അന്തരീക്ഷത്തില്‍ പടര്‍ന്നുപന്തലിച്ചു നില്‍ക്കുന്നു.
\end{malayalam}}
\flushright{\begin{Arabic}
\quranayah[14][25]
\end{Arabic}}
\flushleft{\begin{malayalam}
എല്ലാ കാലത്തും അത് അതിന്റെ നാഥന്റെ അനുമതിയോടെ ഫലങ്ങള്‍ നല്‍കിക്കൊണ്ടിരിക്കുന്നു. അല്ലാഹു ജനങ്ങള്‍ക്ക് ഉപമകള്‍ വിശദീകരിച്ചുകൊടുക്കുന്നു. അവര്‍ ചിന്തിച്ചറിയാന്‍.
\end{malayalam}}
\flushright{\begin{Arabic}
\quranayah[14][26]
\end{Arabic}}
\flushleft{\begin{malayalam}
ചീത്ത വചനത്തിന്റെ ഉപമ ഒരു ക്ഷുദ്ര വൃക്ഷത്തിന്റേതാണ്. ഭൂതലത്തില്‍ നിന്ന് അത് വേരോടെ പിഴുതെറിയപ്പെട്ടിരിക്കുന്നു. അതിനെ ഉറപ്പിച്ചുനിര്‍ത്തുന്ന ഒന്നുമില്ല.
\end{malayalam}}
\flushright{\begin{Arabic}
\quranayah[14][27]
\end{Arabic}}
\flushleft{\begin{malayalam}
സത്യവിശ്വാസം സ്വീകരിച്ചവര്‍ക്ക് അല്ലാഹു സുസ്ഥിരമായ വചനത്താല്‍ സ്ഥൈര്യം നല്‍കുന്നു; ഇഹലോകജീവിതത്തിലും പരലോകത്തും. അക്രമികളെ അല്ലാഹു വഴികേടിലാക്കുന്നു. അല്ലാഹു അവനിച്ഛിക്കുന്നതെന്തും ചെയ്യുന്നു.
\end{malayalam}}
\flushright{\begin{Arabic}
\quranayah[14][28]
\end{Arabic}}
\flushleft{\begin{malayalam}
അല്ലാഹുവിന്റെ അനുഗ്രഹത്തിനു നന്ദികാണിക്കാത്തവരെ നീ കണ്ടില്ലേ? അവര്‍ തങ്ങളുടെ ജനതയെ നാശത്തിന്റെ താവളത്തിലേക്ക് തള്ളിയിട്ടു.
\end{malayalam}}
\flushright{\begin{Arabic}
\quranayah[14][29]
\end{Arabic}}
\flushleft{\begin{malayalam}
അഥവാ, നരകത്തിലേക്ക്. അവരതില്‍ കത്തിയെരിയും. അതെത്ര ചീത്ത താവളം!
\end{malayalam}}
\flushright{\begin{Arabic}
\quranayah[14][30]
\end{Arabic}}
\flushleft{\begin{malayalam}
അവര്‍ അല്ലാഹുവിന് ചില സമന്മാരെ സങ്കല്‍പിച്ചുവെച്ചിരിക്കുന്നു. അവന്റെ മാര്‍ഗത്തില്‍ നിന്ന്ജനത്തെ തെറ്റിക്കാന്‍. പറയുക: നിങ്ങള്‍ സുഖിച്ചോളൂ. തീര്‍ച്ചയായും നിങ്ങളുടെ മടക്കം നരകത്തീയിലേക്കാണ്.
\end{malayalam}}
\flushright{\begin{Arabic}
\quranayah[14][31]
\end{Arabic}}
\flushleft{\begin{malayalam}
സത്യവിശ്വാസം സ്വീകരിച്ച എന്റെ ദാസന്മാരോടു പറയുക: കൊള്ളക്കൊടുക്കകളും ചങ്ങാത്തവും നടക്കാത്തദിനം വന്നെത്തും മുമ്പെ അവര്‍ നമസ്കാരം നിഷ്ഠയോടെ നിര്‍വഹിക്കട്ടെ. നാമവര്‍ക്ക് നല്‍കിയതില്‍നിന്ന് രഹസ്യമായും പരസ്യമായും ചെലവഴിക്കുകയും ചെയ്യട്ടെ.
\end{malayalam}}
\flushright{\begin{Arabic}
\quranayah[14][32]
\end{Arabic}}
\flushleft{\begin{malayalam}
അല്ലാഹുവാണ് ആകാശഭൂമികളെ സൃഷ്ടിച്ചവന്‍. അവന്‍ മാനത്തുനിന്നു മഴ പെയ്യിച്ചു. അതുവഴി നിങ്ങള്‍ക്ക് ആഹരിക്കാന്‍ കായ്കനികള്‍ ഉല്‍പാദിപ്പിച്ചു. ദൈവനിശ്ചയപ്രകാരം സമുദ്രത്തില്‍ സഞ്ചരിക്കാന്‍ നിങ്ങള്‍ക്ക് അവന്‍ കപ്പലുകള്‍ അധീനപ്പെടുത്തിത്തന്നു. നദികളെയും അവന്‍ നിങ്ങള്‍ക്കു വിധേയമാക്കി.
\end{malayalam}}
\flushright{\begin{Arabic}
\quranayah[14][33]
\end{Arabic}}
\flushleft{\begin{malayalam}
നിരന്തരം ചരിച്ചുകൊണ്ടിരിക്കുന്ന സൂര്യചന്ദ്രന്മാരെയും അവന്‍ നിങ്ങള്‍ക്ക് അധീനപ്പെടുത്തിത്തന്നു. രാപ്പകലുകളെയും നിങ്ങള്‍ക്ക് വിധേയമാക്കി.
\end{malayalam}}
\flushright{\begin{Arabic}
\quranayah[14][34]
\end{Arabic}}
\flushleft{\begin{malayalam}
നിങ്ങള്‍ക്ക് ആവശ്യമുള്ളതൊക്കെ അവന്‍ നിങ്ങള്‍ക്ക് നല്‍കിയിരിക്കുന്നു. അല്ലാഹുവിന്റെ അനുഗ്രഹം നിങ്ങള്‍ക്ക് എണ്ണിക്കണക്കാക്കാനാവില്ല. തീര്‍ച്ചയായും മനുഷ്യന്‍ കടുത്ത അക്രമിയും വളരെ നന്ദികെട്ടവനും തന്നെ.
\end{malayalam}}
\flushright{\begin{Arabic}
\quranayah[14][35]
\end{Arabic}}
\flushleft{\begin{malayalam}
ഇബ്റാഹീം പറഞ്ഞ സന്ദര്‍ഭം: "എന്റെ നാഥാ! നീ ഈ നാടിനെ നിര്‍ഭയത്വമുള്ളതാക്കേണമേ. എന്നെയും എന്റെ മക്കളെയും വിഗ്രഹപൂജയില്‍ നിന്നകറ്റി നിര്‍ത്തേണമേ.
\end{malayalam}}
\flushright{\begin{Arabic}
\quranayah[14][36]
\end{Arabic}}
\flushleft{\begin{malayalam}
"എന്റെ നാഥാ! ഈ വിഗ്രഹങ്ങള്‍ ഏറെപ്പേരെ വഴികേടിലാക്കിയിരിക്കുന്നു. അതിനാല്‍ എന്നെ പിന്തുടരുന്നവന്‍ എന്റെ ആളാണ്. ആരെങ്കിലും എന്നെ ധിക്കരിക്കുന്നുവെങ്കില്‍, നാഥാ, നീ എറെ പൊറുക്കുന്നവനും ദയാപരനുമല്ലോ.”
\end{malayalam}}
\flushright{\begin{Arabic}
\quranayah[14][37]
\end{Arabic}}
\flushleft{\begin{malayalam}
"ഞങ്ങളുടെ നാഥാ! എന്റെ മക്കളില്‍ ചിലരെ, കൃഷിയില്ലാത്ത ഈ താഴ്വരയില്‍, നിന്റെ ആദരണീയ മന്ദിരത്തിനടുത്ത് ഞാന്‍ താമസിപ്പിച്ചിരിക്കുന്നു. ഞങ്ങളുടെ നാഥാ! അവര്‍ നമസ്കാരം നിഷ്ഠയോടെ നിര്‍വഹിക്കാനാണത്. അതിനാല്‍ നീ ജനമനസ്സുകളില്‍ അവരോട് അടുപ്പമുണ്ടാക്കേണമേ. അവര്‍ക്ക് ആഹാരമായി കായ്കനികള്‍ നല്‍കേണമേ. അവര്‍ നന്ദി കാണിച്ചേക്കാം.
\end{malayalam}}
\flushright{\begin{Arabic}
\quranayah[14][38]
\end{Arabic}}
\flushleft{\begin{malayalam}
"ഞങ്ങളുടെ നാഥാ! ഞങ്ങള്‍ മറച്ചുവെക്കുന്നതും തെളിയിച്ചുകാണിക്കുന്നതുമെല്ലാം നീയറിയുന്നു.” അല്ലാഹുവില്‍നിന്ന് മറഞ്ഞിരിക്കുന്നതായി ഒന്നുമില്ല- ഭൂമിയിലും ആകാശത്തും.
\end{malayalam}}
\flushright{\begin{Arabic}
\quranayah[14][39]
\end{Arabic}}
\flushleft{\begin{malayalam}
"വയസ്സുകാലത്ത് എനിക്ക് ഇസ്മാഈലിനെയും ഇസ്ഹാഖിനെയും സമ്മാനിച്ച അല്ലാഹുവിന് സ്തുതി. തീര്‍ച്ചയായും എന്റെ നാഥന്‍ പ്രാര്‍ഥന കേള്‍ക്കുന്നവനാണ്.
\end{malayalam}}
\flushright{\begin{Arabic}
\quranayah[14][40]
\end{Arabic}}
\flushleft{\begin{malayalam}
"എന്റെ നാഥാ! എന്നെ നീ നമസ്കാരം നിഷ്ഠയോടെ നിര്‍വഹിക്കുന്നവനാക്കേണമേ. എന്റെ മക്കളില്‍ നിന്നും അത്തരക്കാരെ ഉണ്ടാക്കേണമേ; ഞങ്ങളുടെ നാഥാ! എന്റെ ഈ പ്രാര്‍ഥന നീ സ്വീകരിച്ചാലും.
\end{malayalam}}
\flushright{\begin{Arabic}
\quranayah[14][41]
\end{Arabic}}
\flushleft{\begin{malayalam}
"ഞങ്ങളുടെ നാഥാ! വിചാരണ നാളില്‍ നീ എനിക്കും എന്റെ മാതാപിതാക്കള്‍ക്കും മുഴുവന്‍ സത്യവിശ്വാസികള്‍ക്കും മാപ്പേകണമേ.”
\end{malayalam}}
\flushright{\begin{Arabic}
\quranayah[14][42]
\end{Arabic}}
\flushleft{\begin{malayalam}
അക്രമികള്‍ പ്രവര്‍ത്തിച്ചുകൊണ്ടിരിക്കുന്നതിനെപ്പറ്റി അല്ലാഹു അശ്രദ്ധനാണെന്ന് നിങ്ങള്‍ കരുതരുത്. അവന്‍ അവരെ കണ്ണുകള്‍ തുറിച്ചുപോകുന്ന ഒരു നാളിലേക്ക് പിന്തിച്ചിടുന്നുവെന്നേയുള്ളൂ.
\end{malayalam}}
\flushright{\begin{Arabic}
\quranayah[14][43]
\end{Arabic}}
\flushleft{\begin{malayalam}
അന്ന് അവര്‍ പരിഭ്രാന്തരായി തലപൊക്കിപ്പിടിച്ച് പാഞ്ഞടുക്കും. അവരുടെ തുറിച്ച ദൃഷ്ടികള്‍ അവരിലേക്ക് മടങ്ങുകയില്ല. അവരുടെ ഹൃദയങ്ങള്‍ ശൂന്യമായിരിക്കും.
\end{malayalam}}
\flushright{\begin{Arabic}
\quranayah[14][44]
\end{Arabic}}
\flushleft{\begin{malayalam}
ജനത്തിനു ശിക്ഷ വന്നെത്തുന്ന ദിവസത്തെ സംബന്ധിച്ച് നീ അവരെ താക്കീതു ചെയ്യുക. അതിക്രമം പ്രവര്‍ത്തിച്ചവര്‍ അപ്പോള്‍ പറയും: "ഞങ്ങളുടെ നാഥാ! അടുത്ത ഒരവധിവരെ ഞങ്ങള്‍ക്കു നീ അവസരം നല്‍കേണമേ! എങ്കില്‍ നിന്റെ വിളിക്ക് ഞങ്ങളുത്തരം നല്‍കാം. നിന്റെ ദൂതന്മാരെ പിന്തുടരുകയും ചെയ്യാം.” അവര്‍ക്കുള്ള മറുപടി ഇതായിരിക്കും: "ഞങ്ങള്‍ക്കൊരു മാറ്റവുമുണ്ടാവുകയില്ലെന്ന് നേരത്തെ ആണയിട്ടു പറഞ്ഞിരുന്നില്ലേ നിങ്ങള്‍?”
\end{malayalam}}
\flushright{\begin{Arabic}
\quranayah[14][45]
\end{Arabic}}
\flushleft{\begin{malayalam}
തങ്ങളോടുതന്നെ അതിക്രമം കാണിച്ച ഒരു ജനവിഭാഗത്തിന്റെ പാര്‍പ്പിടങ്ങളിലാണല്ലോ നിങ്ങള്‍ താമസിച്ചിരുന്നത്. അവരെ നാമെന്തു ചെയ്തുവെന്ന് നിങ്ങള്‍ വ്യക്തമായി മനസ്സിലാക്കിയിട്ടുണ്ട്. നിങ്ങള്‍ക്കു നാം വ്യക്തമായ ഉപമകള്‍ വഴി കാര്യം വിശദീകരിച്ചുതന്നിട്ടുമുണ്ട്.
\end{malayalam}}
\flushright{\begin{Arabic}
\quranayah[14][46]
\end{Arabic}}
\flushleft{\begin{malayalam}
അവര്‍ തങ്ങളുടെ കൌശലം പരമാവധി പ്രയോഗിച്ചു. എന്നാല്‍ അവര്‍ക്കെതിരിലുള്ള കൌശലം അല്ലാഹുവിങ്കലുണ്ട്; അവരുടെ കുതന്ത്രം പര്‍വതങ്ങളെ പിഴുതുമാറ്റാന്‍ പോന്നതാണെങ്കിലും.
\end{malayalam}}
\flushright{\begin{Arabic}
\quranayah[14][47]
\end{Arabic}}
\flushleft{\begin{malayalam}
അല്ലാഹു തന്റെ ദൂതന്മാര്‍ക്ക് നല്‍കിയ വാഗ്ദാനം ലംഘിക്കുമെന്ന് നീ ഒരിക്കലും കരുതരുത്. തീര്‍ച്ചയായും അല്ലാഹു പ്രതാപിയാണ്. പ്രതികാരനടപടി സ്വീകരിക്കുന്നവനും.
\end{malayalam}}
\flushright{\begin{Arabic}
\quranayah[14][48]
\end{Arabic}}
\flushleft{\begin{malayalam}
ഈ ഭൂമി ഒരുനാള്‍ ഭൂമിയല്ലാതായിത്തീരും. ആകാശങ്ങളും അവയല്ലാതായിമാറും. ഏകനും എല്ലാറ്റിനെയും അടക്കിഭരിക്കുന്നവനുമായ അല്ലാഹുവിന്റെ മുന്നില്‍ അവയെല്ലാം മറയില്ലാതെ പ്രത്യക്ഷപ്പെടും.
\end{malayalam}}
\flushright{\begin{Arabic}
\quranayah[14][49]
\end{Arabic}}
\flushleft{\begin{malayalam}
അന്ന് കുറ്റവാളികളെ നിനക്കു കാണാം. അവര്‍ ചങ്ങലകളില്‍ പരസ്പരം ബന്ധിക്കപ്പെട്ടവരായിരിക്കും.
\end{malayalam}}
\flushright{\begin{Arabic}
\quranayah[14][50]
\end{Arabic}}
\flushleft{\begin{malayalam}
അവരുടെ കുപ്പായങ്ങള്‍ കട്ടിത്താറുകൊണ്ടുള്ളവയായിരിക്കും. തീനാളങ്ങള്‍ അവരുടെ മുഖങ്ങളെ പൊതിയും.
\end{malayalam}}
\flushright{\begin{Arabic}
\quranayah[14][51]
\end{Arabic}}
\flushleft{\begin{malayalam}
എല്ലാ ഓരോരുത്തര്‍ക്കും അവര്‍ സമ്പാദിച്ചതിന്റെ പ്രതിഫലം അല്ലാഹു നല്‍കാന്‍ വേണ്ടിയാണിത്. അല്ലാഹു അതിവേഗം കണക്കുനോക്കുന്നവനാണ്; തീര്‍ച്ച.
\end{malayalam}}
\flushright{\begin{Arabic}
\quranayah[14][52]
\end{Arabic}}
\flushleft{\begin{malayalam}
ഇത് മുഴുവന്‍ മനുഷ്യര്‍ക്കുമുള്ള സന്ദേശമാണ്. ഇതിലൂടെ അവര്‍ക്ക് മുന്നറിയിപ്പ് നല്‍കാന്‍. അവന്‍ ഏകനായ ദൈവം മാത്രമാണെന്ന് അവരറിയാന്‍. വിചാരശാലികള്‍ ചിന്തിച്ചു മനസ്സിലാക്കാനും.
\end{malayalam}}
\chapter{\textmalayalam{ഹിജ്റ്}}
\begin{Arabic}
\Huge{\centerline{\basmalah}}\end{Arabic}
\flushright{\begin{Arabic}
\quranayah[15][1]
\end{Arabic}}
\flushleft{\begin{malayalam}
അലിഫ് - ലാം - റാഅ്. വേദപുസ്തകത്തിലെ അഥവാ, സുവ്യക്തമായ ഖുര്‍ആനിലെ വചനങ്ങളാണിവ.
\end{malayalam}}
\flushright{\begin{Arabic}
\quranayah[15][2]
\end{Arabic}}
\flushleft{\begin{malayalam}
തങ്ങള്‍ മുസ്ലിംകളായിരുന്നെങ്കില്‍ എന്ന് സത്യനിഷേധികള്‍ കൊതിച്ചുപോകുന്ന അവസ്ഥയുണ്ടാകും.
\end{malayalam}}
\flushright{\begin{Arabic}
\quranayah[15][3]
\end{Arabic}}
\flushleft{\begin{malayalam}
അവരെ നീ വിട്ടേക്കുക. അവര്‍ തിന്നും സുഖിച്ചും വ്യാമോഹങ്ങള്‍ക്കടിപ്പെട്ടും കഴിയട്ടെ. വൈകാതെ അവര്‍ എല്ലാം അറിയും.
\end{malayalam}}
\flushright{\begin{Arabic}
\quranayah[15][4]
\end{Arabic}}
\flushleft{\begin{malayalam}
നിശ്ചിതമായ അവധി നല്‍കിക്കൊണ്ടല്ലാതെ നാം ഒരു നാടിനെയും നശിപ്പിച്ചിട്ടില്ല.
\end{malayalam}}
\flushright{\begin{Arabic}
\quranayah[15][5]
\end{Arabic}}
\flushleft{\begin{malayalam}
ഒരു സമുദായവും നിശ്ചിത അവധിക്കുമുമ്പ് നശിക്കുകയില്ല. അവധിയെത്തിയാല്‍ പിന്നെ പിന്തിക്കുകയുമില്ല.
\end{malayalam}}
\flushright{\begin{Arabic}
\quranayah[15][6]
\end{Arabic}}
\flushleft{\begin{malayalam}
സത്യനിഷേധികള്‍ പറഞ്ഞു: "ഉദ്ബോധനം ഇറക്കിക്കിട്ടിയവനേ, നീയൊരു ഭ്രാന്തന്‍ തന്നെ.”
\end{malayalam}}
\flushright{\begin{Arabic}
\quranayah[15][7]
\end{Arabic}}
\flushleft{\begin{malayalam}
"നീ സത്യവാനെങ്കില്‍ ഞങ്ങളുടെ അടുത്ത് മലക്കുകളെ കൊണ്ടുവരാത്തതെന്ത്?”
\end{malayalam}}
\flushright{\begin{Arabic}
\quranayah[15][8]
\end{Arabic}}
\flushleft{\begin{malayalam}
എന്നാല്‍ ന്യായമായ ആവശ്യത്തിനല്ലാതെ നാം മലക്കുകളെ ഇറക്കുകയില്ല. ഇറക്കിയാല്‍ പിന്നെ അവര്‍ക്ക് അവസരം നല്‍കുകയുമില്ല.
\end{malayalam}}
\flushright{\begin{Arabic}
\quranayah[15][9]
\end{Arabic}}
\flushleft{\begin{malayalam}
തീര്‍ച്ചയായും നാമാണ് ഈ ഖുര്‍ആന്‍ ഇറക്കിയത്. നാം തന്നെ അതിനെ കാത്തുരക്ഷിക്കുകയും ചെയ്യും.
\end{malayalam}}
\flushright{\begin{Arabic}
\quranayah[15][10]
\end{Arabic}}
\flushleft{\begin{malayalam}
നിനക്കുമുമ്പ് പൂര്‍വികരായ പല വിഭാഗങ്ങളിലും നാം ദൂതന്മാരെ നിയോഗിച്ചിട്ടുണ്ട്.
\end{malayalam}}
\flushright{\begin{Arabic}
\quranayah[15][11]
\end{Arabic}}
\flushleft{\begin{malayalam}
അവരുടെ അടുത്ത് ദൈവദൂതന്‍ ചെന്നപ്പോഴെല്ലാം അവരദ്ദേഹത്തെ പരിഹസിക്കാതിരുന്നിട്ടില്ല.
\end{malayalam}}
\flushright{\begin{Arabic}
\quranayah[15][12]
\end{Arabic}}
\flushleft{\begin{malayalam}
അവ്വിധമാണ് നാം കുറ്റവാളികളുടെ മനസ്സുകളില്‍ നാം പരിഹാസം കടത്തിവിടുന്നത്.
\end{malayalam}}
\flushright{\begin{Arabic}
\quranayah[15][13]
\end{Arabic}}
\flushleft{\begin{malayalam}
എന്നിട്ടും അവരതില്‍ വിശ്വസിക്കുന്നില്ല. പൂര്‍വികരും ഇമ്മട്ടില്‍ തന്നെയായിരുന്നു.
\end{malayalam}}
\flushright{\begin{Arabic}
\quranayah[15][14]
\end{Arabic}}
\flushleft{\begin{malayalam}
നാമവര്‍ക്ക് മാനത്തുനിന്നൊരു വാതില്‍ തുറന്നുകൊടുത്തുവെന്ന് വെക്കുക. അങ്ങനെ അവരതിലൂടെ കയറിപ്പോയിക്കൊണ്ടിരിക്കുന്നുവെന്നും.
\end{malayalam}}
\flushright{\begin{Arabic}
\quranayah[15][15]
\end{Arabic}}
\flushleft{\begin{malayalam}
എന്നാല്‍പ്പോലും അവര്‍ പറയും: "നമ്മുടെ കണ്ണുകള്‍ക്ക് മയക്കം ബാധിച്ചതാണ്. അല്ല നാം മാരണത്തിനിരയായ ജനമത്രെ.”
\end{malayalam}}
\flushright{\begin{Arabic}
\quranayah[15][16]
\end{Arabic}}
\flushleft{\begin{malayalam}
ആകാശത്തു നാം രാശികളുണ്ടാക്കിയിരിക്കുന്നു. കാണികള്‍ക്ക് അത് അലംകൃതമാക്കുകയും ചെയ്തിരിക്കുന്നു.
\end{malayalam}}
\flushright{\begin{Arabic}
\quranayah[15][17]
\end{Arabic}}
\flushleft{\begin{malayalam}
ശപിക്കപ്പെട്ട സകല പിശാചുക്കളില്‍നിന്നും നാമതിനെ കാത്തുരക്ഷിച്ചിരിക്കുന്നു;
\end{malayalam}}
\flushright{\begin{Arabic}
\quranayah[15][18]
\end{Arabic}}
\flushleft{\begin{malayalam}
കട്ടുകേള്‍ക്കുന്നവനില്‍ നിന്നൊഴികെ. അങ്ങനെ ചെയ്യുമ്പോള്‍ തീക്ഷ്ണമായ ജ്വാല അവനെ പിന്തുടരുന്നു.
\end{malayalam}}
\flushright{\begin{Arabic}
\quranayah[15][19]
\end{Arabic}}
\flushleft{\begin{malayalam}
ഭൂമിയെ നാം വിശാലമാക്കി. അതില്‍ മലകളെ ഉറപ്പിച്ചുനിര്‍ത്തി. അതില്‍ നാം നാനാതരം വസ്തുക്കള്‍ കൃത്യമായ പരിമാണത്തോടെ മുളപ്പിച്ചു.
\end{malayalam}}
\flushright{\begin{Arabic}
\quranayah[15][20]
\end{Arabic}}
\flushleft{\begin{malayalam}
നാമതില്‍ നിങ്ങള്‍ക്ക് ജീവനോപാധികള്‍ ഉണ്ടാക്കിവെച്ചിരിക്കുന്നു. നിങ്ങള്‍ ആഹാരം കൊടുക്കാത്തവയ്ക്കും.
\end{malayalam}}
\flushright{\begin{Arabic}
\quranayah[15][21]
\end{Arabic}}
\flushleft{\begin{malayalam}
എല്ലാറ്റിന്റെയും ജീവിതോപാധികളുടെ പത്തായം നമ്മുടെ വശമാണ്. നീതിപൂര്‍വം നിശ്ചിത തോതില്‍ നാമതു ഇറക്കിക്കൊടുക്കുന്നു.
\end{malayalam}}
\flushright{\begin{Arabic}
\quranayah[15][22]
\end{Arabic}}
\flushleft{\begin{malayalam}
നാം മേഘവാഹിനികളായ കാറ്റിനെ അയക്കുന്നു. അങ്ങനെ മാനത്തുനിന്ന് വെള്ളമിറക്കുന്നു. നാം നിങ്ങളെയത് കുടിപ്പിക്കുന്നു. അതൊന്നും ശേഖരിച്ചുവെക്കുന്നത് നിങ്ങളല്ലല്ലോ.
\end{malayalam}}
\flushright{\begin{Arabic}
\quranayah[15][23]
\end{Arabic}}
\flushleft{\begin{malayalam}
തീര്‍ച്ചയായും നാമാണ് ജീവിപ്പിക്കുന്നതും മരിപ്പിക്കുന്നതും. എല്ലാറ്റിനെയും അനന്തരമെടുക്കുന്നതും നാം തന്നെ.
\end{malayalam}}
\flushright{\begin{Arabic}
\quranayah[15][24]
\end{Arabic}}
\flushleft{\begin{malayalam}
നിങ്ങളില്‍നിന്ന് നേരത്തെ കടന്നുപോയവര്‍ ആരെന്ന് നമുക്ക് നന്നായറിയാം. പിറകെ വരുന്നവരാരെന്നും നാമറിയുന്നു.
\end{malayalam}}
\flushright{\begin{Arabic}
\quranayah[15][25]
\end{Arabic}}
\flushleft{\begin{malayalam}
നിസ്സംശയം; നിന്റെ നാഥന്‍ അവരെയൊക്കെ ഒരുമിച്ചുകൂട്ടും. അവന്‍ യുക്തിമാനും എല്ലാം അറിയുന്നവനും തന്നെ.
\end{malayalam}}
\flushright{\begin{Arabic}
\quranayah[15][26]
\end{Arabic}}
\flushleft{\begin{malayalam}
നിശ്ചയമായും മനുഷ്യനെ നാം, മുട്ടിയാല്‍ മുഴങ്ങുന്ന, ഗന്ധമുള്ള കറുത്ത കളിമണ്ണില്‍ നിന്നു സൃഷ്ടിച്ചു.
\end{malayalam}}
\flushright{\begin{Arabic}
\quranayah[15][27]
\end{Arabic}}
\flushleft{\begin{malayalam}
അതിനുമുമ്പ് ജിന്നുകളെ നാം അത്യുഷ്ണമുള്ള തീജ്ജ്വാലയില്‍നിന്ന് സൃഷ്ടിച്ചു.
\end{malayalam}}
\flushright{\begin{Arabic}
\quranayah[15][28]
\end{Arabic}}
\flushleft{\begin{malayalam}
നിന്റെ നാഥന്‍ മലക്കുകളോട് പറഞ്ഞ സന്ദര്‍ഭം: നിശ്ചയമായും മുട്ടിയാല്‍ മുഴങ്ങുന്ന, ഗന്ധമുള്ള, കറുത്ത കളിമണ്ണില്‍ നിന്ന് നാം മനുഷ്യനെ സൃഷ്ടിക്കാന്‍ പോവുകയാണ്.
\end{malayalam}}
\flushright{\begin{Arabic}
\quranayah[15][29]
\end{Arabic}}
\flushleft{\begin{malayalam}
അങ്ങനെ ഞാനവനെ രൂപപ്പെടുത്തുകയും എന്റെ ആത്മാവില്‍ നിന്ന് അവനിലൂതുകയും ചെയ്താല്‍ നിങ്ങളെല്ലാവരും അവന് പ്രണാമമര്‍പ്പിക്കുന്നവരായിത്തീരണം.
\end{malayalam}}
\flushright{\begin{Arabic}
\quranayah[15][30]
\end{Arabic}}
\flushleft{\begin{malayalam}
അങ്ങനെ മലക്കുകളൊക്കെ പ്രണമിച്ചു.
\end{malayalam}}
\flushright{\begin{Arabic}
\quranayah[15][31]
\end{Arabic}}
\flushleft{\begin{malayalam}
ഇബ്ലീസൊഴികെ. പ്രണാമമര്‍പ്പിക്കുന്നവരോടൊപ്പം ചേരാന്‍ അവന്‍ വിസമ്മതിച്ചു.
\end{malayalam}}
\flushright{\begin{Arabic}
\quranayah[15][32]
\end{Arabic}}
\flushleft{\begin{malayalam}
അല്ലാഹു ചോദിച്ചു: "പ്രണാമം ചെയ്തവരോടൊപ്പം ചേരാതിരിക്കാന്‍ നിന്നെ പ്രേരിപ്പിച്ചതെന്ത്?”
\end{malayalam}}
\flushright{\begin{Arabic}
\quranayah[15][33]
\end{Arabic}}
\flushleft{\begin{malayalam}
ഇബ്ലീസ് പറഞ്ഞു: "മുട്ടിയാല്‍ മുഴങ്ങുന്ന, ഗന്ധമുള്ള കറുത്ത കളിമണ്ണില്‍ നിന്ന് നീ സൃഷ്ടിച്ച മനുഷ്യനെ പ്രണമിക്കേണ്ടവനല്ല ഞാന്‍.”
\end{malayalam}}
\flushright{\begin{Arabic}
\quranayah[15][34]
\end{Arabic}}
\flushleft{\begin{malayalam}
അല്ലാഹു കല്‍പിച്ചു: "എങ്കില്‍ നീ ഇവിടെനിന്നിറങ്ങിപ്പോവുക. നീ ഭ്രഷ്ടനാണ്.
\end{malayalam}}
\flushright{\begin{Arabic}
\quranayah[15][35]
\end{Arabic}}
\flushleft{\begin{malayalam}
"ന്യായവിധിയുടെ നാള്‍വരെ നിനക്കു ശാപമുണ്ടായിരിക്കും.”
\end{malayalam}}
\flushright{\begin{Arabic}
\quranayah[15][36]
\end{Arabic}}
\flushleft{\begin{malayalam}
അവന്‍ പറഞ്ഞു: "എന്റെ നാഥാ, അവര്‍ ഉയിര്‍ത്തെഴുന്നേല്‍ക്കുന്ന നാള്‍വരെ എനിക്ക് അവധി തന്നാലും.”
\end{malayalam}}
\flushright{\begin{Arabic}
\quranayah[15][37]
\end{Arabic}}
\flushleft{\begin{malayalam}
അല്ലാഹു അറിയിച്ചു: "നിനക്ക് അവസരം തന്നിരിക്കുന്നു.
\end{malayalam}}
\flushright{\begin{Arabic}
\quranayah[15][38]
\end{Arabic}}
\flushleft{\begin{malayalam}
"നിശ്ചിതസമയം വന്നെത്തുന്ന ദിനംവരെ.”
\end{malayalam}}
\flushright{\begin{Arabic}
\quranayah[15][39]
\end{Arabic}}
\flushleft{\begin{malayalam}
അവന്‍ പറഞ്ഞു: "എന്റെ നാഥാ, നീ എന്നെ വഴികേടിലാക്കി. അതേപോലെ ഭൂമിയില്‍ ഞാനവര്‍ക്ക് ചീത്തവൃത്തികള്‍ ചേതോഹരമായിത്തോന്നിപ്പിക്കും. അവരെയൊക്കെ ദുര്‍മാര്‍ഗത്തിലാക്കുകയും ചെയ്യും; തീര്‍ച്ച.
\end{malayalam}}
\flushright{\begin{Arabic}
\quranayah[15][40]
\end{Arabic}}
\flushleft{\begin{malayalam}
"അവരിലെ നിന്റെ ആത്മാര്‍ഥതയുള്ള ദാസന്മാരെയൊഴികെ.”
\end{malayalam}}
\flushright{\begin{Arabic}
\quranayah[15][41]
\end{Arabic}}
\flushleft{\begin{malayalam}
അല്ലാഹു പറഞ്ഞു: "ഇതാണ് എന്നിലേക്കെത്താനുള്ള നേര്‍ വഴി.
\end{malayalam}}
\flushright{\begin{Arabic}
\quranayah[15][42]
\end{Arabic}}
\flushleft{\begin{malayalam}
"എന്റെ അടിമകളുടെ മേല്‍ നിനക്കൊരു സ്വാധീനവുമില്ല. നിന്നെ പിന്തുടര്‍ന്ന വഴിപിഴച്ചവരിലൊഴികെ.
\end{malayalam}}
\flushright{\begin{Arabic}
\quranayah[15][43]
\end{Arabic}}
\flushleft{\begin{malayalam}
"തീര്‍ച്ചയായും നരകമാണ് അവര്‍ക്ക് വാഗ്ദാനം ചെയ്യപ്പെട്ട ഇടം.”
\end{malayalam}}
\flushright{\begin{Arabic}
\quranayah[15][44]
\end{Arabic}}
\flushleft{\begin{malayalam}
അതിന് ഏഴു വാതിലുകളുണ്ട്. ഓരോ വാതിലിലൂടെയും പ്രവേശിക്കാന്‍ അവരില്‍നിന്ന് പ്രത്യേകം വീതിക്കപ്പെട്ട ഓരോ വിഭാഗമുണ്ട്.
\end{malayalam}}
\flushright{\begin{Arabic}
\quranayah[15][45]
\end{Arabic}}
\flushleft{\begin{malayalam}
ഉറപ്പായും സൂക്ഷ്മത പാലിക്കുന്നവര്‍ സ്വര്‍ഗീയാരാമങ്ങളിലും അരുവികളിലുമായിരിക്കും.
\end{malayalam}}
\flushright{\begin{Arabic}
\quranayah[15][46]
\end{Arabic}}
\flushleft{\begin{malayalam}
അവരോടു പറയും: "നിര്‍ഭയരായി സമാധാനത്തോടെ നിങ്ങളതില്‍ പ്രവേശിച്ചുകൊള്ളുക.”
\end{malayalam}}
\flushright{\begin{Arabic}
\quranayah[15][47]
\end{Arabic}}
\flushleft{\begin{malayalam}
അവരുടെ ഹൃദയങ്ങളിലുണ്ടായേക്കാവുന്ന വിദ്വേഷം നാം നീക്കിക്കളയും. പരസ്പരം സഹോദരങ്ങളായി ചാരുകട്ടിലുകളിലവര്‍ അഭിമുഖമായി ഇരിക്കും.
\end{malayalam}}
\flushright{\begin{Arabic}
\quranayah[15][48]
\end{Arabic}}
\flushleft{\begin{malayalam}
അവിടെ അവരെ ക്ഷീണം ബാധിക്കുകയില്ല. അവിടെനിന്നവര്‍ പുറന്തള്ളപ്പെടുകയുമില്ല.
\end{malayalam}}
\flushright{\begin{Arabic}
\quranayah[15][49]
\end{Arabic}}
\flushleft{\begin{malayalam}
ഞാന്‍ ഏറെ പൊറുക്കുന്നവനും പരമ ദയാലുവുമാണെന്ന് എന്റെ ദാസന്മാരെ അറിയിക്കുക;
\end{malayalam}}
\flushright{\begin{Arabic}
\quranayah[15][50]
\end{Arabic}}
\flushleft{\begin{malayalam}
തീര്‍ച്ചയായും എന്റെ ശിക്ഷയാണ് ഏറ്റം നോവേറിയ ശിക്ഷയെന്നും.
\end{malayalam}}
\flushright{\begin{Arabic}
\quranayah[15][51]
\end{Arabic}}
\flushleft{\begin{malayalam}
ഇബ്റാഹീമിന്റെ അതിഥികളെപ്പറ്റിയും നീ അവര്‍ക്കു പറഞ്ഞുകൊടുക്കുക.
\end{malayalam}}
\flushright{\begin{Arabic}
\quranayah[15][52]
\end{Arabic}}
\flushleft{\begin{malayalam}
അവര്‍ അദ്ദേഹത്തിന്റെ അടുത്തുചെന്ന സന്ദര്‍ഭം: അപ്പോള്‍ അവര്‍ പറഞ്ഞു: "താങ്കള്‍ക്കു സമാധാനം.” അദ്ദേഹം പറഞ്ഞു: "സത്യമായും ഞങ്ങള്‍ക്കു നിങ്ങളെപ്പറ്റി പേടിതോന്നുന്നു.”
\end{malayalam}}
\flushright{\begin{Arabic}
\quranayah[15][53]
\end{Arabic}}
\flushleft{\begin{malayalam}
അവര്‍ പറഞ്ഞു: "താങ്കള്‍ പേടിക്കേണ്ട. ജ്ഞാനമുള്ള ഒരു പുത്രനെ സംബന്ധിച്ച ശുഭവാര്‍ത്ത ഞങ്ങളിതാ താങ്കളെ അറിയിക്കുന്നു.”
\end{malayalam}}
\flushright{\begin{Arabic}
\quranayah[15][54]
\end{Arabic}}
\flushleft{\begin{malayalam}
അദ്ദേഹം പറഞ്ഞു: "ഈ വയസ്സുകാലത്താണോ നിങ്ങളെന്നെ പുത്രനെ സംബന്ധിച്ച ശുഭവാര്‍ത്ത അറിയിക്കുന്നത്? എന്തൊരു ശുഭവാര്‍ത്തയാണ് നിങ്ങള്‍ ഈ നല്‍കുന്നത്?”
\end{malayalam}}
\flushright{\begin{Arabic}
\quranayah[15][55]
\end{Arabic}}
\flushleft{\begin{malayalam}
അവര്‍ പറഞ്ഞു: "ഞങ്ങള്‍ താങ്കള്‍ക്കു നല്‍കുന്നത് ശരിയായ ശുഭവാര്‍ത്ത തന്നെ. അതിനാല്‍ താങ്കള്‍ നിരാശനാവാതിരിക്കുക.”
\end{malayalam}}
\flushright{\begin{Arabic}
\quranayah[15][56]
\end{Arabic}}
\flushleft{\begin{malayalam}
ഇബ്റാഹീം പറഞ്ഞു: "തന്റെ നാഥന്റെ അനുഗ്രഹത്തെക്കുറിച്ച് ആരാണ് നിരാശനാവുക? വഴിപിഴച്ചവരൊഴികെ.”
\end{malayalam}}
\flushright{\begin{Arabic}
\quranayah[15][57]
\end{Arabic}}
\flushleft{\begin{malayalam}
ഇബ്റാഹീം ചോദിച്ചു: "അല്ലയോ ദൂതന്മാരേ, നിങ്ങളുടെ പ്രധാന ദൌത്യമെന്താണ്?”
\end{malayalam}}
\flushright{\begin{Arabic}
\quranayah[15][58]
\end{Arabic}}
\flushleft{\begin{malayalam}
അവര്‍ പറഞ്ഞു: "കുറ്റവാളികളായ ഒരു ജനതയിലേക്കാണ് ഞങ്ങളെ നിയോഗിച്ചിരിക്കുന്നത്.”
\end{malayalam}}
\flushright{\begin{Arabic}
\quranayah[15][59]
\end{Arabic}}
\flushleft{\begin{malayalam}
ലൂത്വിന്റെ കുടുംബം അതില്‍ നിന്നൊഴിവാണ്. അവരെയൊക്കെ നാം രക്ഷപ്പെടുത്തും.
\end{malayalam}}
\flushright{\begin{Arabic}
\quranayah[15][60]
\end{Arabic}}
\flushleft{\begin{malayalam}
അദ്ദേഹത്തിന്റെ ഭാര്യയെ ഒഴികെ. അവള്‍ പിന്തിനില്‍ക്കുന്നവരിലായിരിക്കുമെന്ന് നാം തീരുമാനിച്ചിരിക്കുന്നു.
\end{malayalam}}
\flushright{\begin{Arabic}
\quranayah[15][61]
\end{Arabic}}
\flushleft{\begin{malayalam}
അങ്ങനെ ആ മലക്കുകള്‍ ലൂത്വിന്റെ ആളുകളുടെ അടുക്കലെത്തിയപ്പോള്‍.
\end{malayalam}}
\flushright{\begin{Arabic}
\quranayah[15][62]
\end{Arabic}}
\flushleft{\begin{malayalam}
അദ്ദേഹം പറഞ്ഞു: "നിങ്ങള്‍ അപരിചിതരായ ആളുകളാണല്ലോ.”
\end{malayalam}}
\flushright{\begin{Arabic}
\quranayah[15][63]
\end{Arabic}}
\flushleft{\begin{malayalam}
അവര്‍ പറഞ്ഞു: "ഈ ജനം സംശയിച്ചുകൊണ്ടിരുന്ന കാര്യവുമായാണ് ഞങ്ങള്‍ വന്നിരിക്കുന്നത്.
\end{malayalam}}
\flushright{\begin{Arabic}
\quranayah[15][64]
\end{Arabic}}
\flushleft{\begin{malayalam}
"ഞങ്ങള്‍ സത്യവുമായാണ് താങ്കളുടെ അടുത്ത് വന്നിരിക്കുന്നത്. തീര്‍ച്ചയായും ഞങ്ങള്‍ സത്യം പറയുന്നവരാണ്.
\end{malayalam}}
\flushright{\begin{Arabic}
\quranayah[15][65]
\end{Arabic}}
\flushleft{\begin{malayalam}
"അതിനാല്‍ രാവിന്റെ ഒരു ഖണ്ഡം മാത്രം ബാക്കിനില്‍ക്കെ താങ്കള്‍ കുടുംബത്തെയും കൂട്ടി ഇവിടം വിടുക. താങ്കള്‍ അവരുടെ പിന്നില്‍ നടക്കണം. ആരും തിരിഞ്ഞുനോക്കരുത്. ആവശ്യപ്പെടുന്നേടത്തേക്ക് പോവുക.”
\end{malayalam}}
\flushright{\begin{Arabic}
\quranayah[15][66]
\end{Arabic}}
\flushleft{\begin{malayalam}
അഥവാ, അടുത്ത പ്രഭാതത്തോടെ ഇക്കൂട്ടരുടെ മുരടു മുറിച്ചുമാറ്റുമെന്ന് നാം അദ്ദേഹത്തെ ഖണ്ഡിതമായി അറിയിച്ചു.
\end{malayalam}}
\flushright{\begin{Arabic}
\quranayah[15][67]
\end{Arabic}}
\flushleft{\begin{malayalam}
അപ്പോഴേക്കും നഗരവാസികള്‍ ആഹ്ളാദഭരിതരായി വന്നെത്തി.
\end{malayalam}}
\flushright{\begin{Arabic}
\quranayah[15][68]
\end{Arabic}}
\flushleft{\begin{malayalam}
ലൂത്വ് പറഞ്ഞു: "നിശ്ചയമായും ഇവരെന്റെ വിരുന്നുകാരാണ്. അതിനാല്‍ നിങ്ങളെന്നെ വഷളാക്കരുതേ.
\end{malayalam}}
\flushright{\begin{Arabic}
\quranayah[15][69]
\end{Arabic}}
\flushleft{\begin{malayalam}
"അല്ലാഹുവെ ഓര്‍ത്ത് നിങ്ങളെന്നെ മാനക്കേടിലാക്കാതിരിക്കുക.”
\end{malayalam}}
\flushright{\begin{Arabic}
\quranayah[15][70]
\end{Arabic}}
\flushleft{\begin{malayalam}
അവര്‍ പറഞ്ഞു: "ജനങ്ങളുടെ കാര്യത്തിലിടപെടരുതെന്ന് നിന്നെ ഞങ്ങള്‍ വിലക്കിയിരുന്നില്ലേ?”
\end{malayalam}}
\flushright{\begin{Arabic}
\quranayah[15][71]
\end{Arabic}}
\flushleft{\begin{malayalam}
അദ്ദേഹം പറഞ്ഞു: "നിങ്ങള്‍ എന്തെങ്കിലും ചെയ്തേ അടങ്ങൂ എങ്കില്‍ ഇതാ ഇവര്‍, എന്റെ പെണ്‍മക്കള്‍.”
\end{malayalam}}
\flushright{\begin{Arabic}
\quranayah[15][72]
\end{Arabic}}
\flushleft{\begin{malayalam}
നിന്റെ ജീവിതമാണ് സത്യം! അവര്‍ തങ്ങളുടെ ലഹരിയില്‍ മതിമറന്ന് എന്തൊക്കെയോ ചെയ്യുകയാണ്.
\end{malayalam}}
\flushright{\begin{Arabic}
\quranayah[15][73]
\end{Arabic}}
\flushleft{\begin{malayalam}
പ്രഭാതോദയത്തോടെ ഒരു ഘോരഗര്‍ജനം അവരെ പിടികൂടി.
\end{malayalam}}
\flushright{\begin{Arabic}
\quranayah[15][74]
\end{Arabic}}
\flushleft{\begin{malayalam}
അങ്ങനെ ആ നാടിനെ നാം കീഴ്മേല്‍ മറിച്ചു. പിന്നെ നാം ചുട്ടുപഴുത്ത കല്ലുകള്‍ അവരുടെമേല്‍ വീഴ്ത്തി.
\end{malayalam}}
\flushright{\begin{Arabic}
\quranayah[15][75]
\end{Arabic}}
\flushleft{\begin{malayalam}
കാര്യങ്ങള്‍ വേര്‍തിരിച്ചറിയാന്‍ കഴിയുന്നവര്‍ക്ക് തീര്‍ച്ചയായും ഇതില്‍ ധാരാളം തെളിവുകളുണ്ട്.
\end{malayalam}}
\flushright{\begin{Arabic}
\quranayah[15][76]
\end{Arabic}}
\flushleft{\begin{malayalam}
ആ നാട് ഇന്നും ജനസഞ്ചാരമുള്ള വഴിയിലാണ്.
\end{malayalam}}
\flushright{\begin{Arabic}
\quranayah[15][77]
\end{Arabic}}
\flushleft{\begin{malayalam}
തീര്‍ച്ചയായും സത്യവിശ്വാസികള്‍ക്കിതില്‍ മഹത്തായ അടയാളമുണ്ട്.
\end{malayalam}}
\flushright{\begin{Arabic}
\quranayah[15][78]
\end{Arabic}}
\flushleft{\begin{malayalam}
ഉറപ്പായും ഐക്കവാസികള്‍ അക്രമികളായിരുന്നു.
\end{malayalam}}
\flushright{\begin{Arabic}
\quranayah[15][79]
\end{Arabic}}
\flushleft{\begin{malayalam}
അതിനാല്‍ അവരെയും നാം ശിക്ഷിച്ചു. തീര്‍ച്ചയായും ഈ രണ്ടു നാടുകളും തുറസ്സായ വഴിയില്‍തന്നെയാണുള്ളത്.
\end{malayalam}}
\flushright{\begin{Arabic}
\quranayah[15][80]
\end{Arabic}}
\flushleft{\begin{malayalam}
ഹിജ്റ് ദേശക്കാരും ദൈവദൂതന്മാരെ തള്ളിപ്പറഞ്ഞു.
\end{malayalam}}
\flushright{\begin{Arabic}
\quranayah[15][81]
\end{Arabic}}
\flushleft{\begin{malayalam}
നാമവര്‍ക്ക് നമ്മുടെ തെളിവുകള്‍ നല്‍കി. എന്നാല്‍ അവര്‍ അവയെ അവഗണിക്കുകയായിരുന്നു.
\end{malayalam}}
\flushright{\begin{Arabic}
\quranayah[15][82]
\end{Arabic}}
\flushleft{\begin{malayalam}
അവര്‍ പര്‍വതങ്ങളിലെ പാറകള്‍ തുരന്ന് വീടുകളുണ്ടാക്കി. അവരവിടെ നിര്‍ഭയരായി കഴിയുകയായിരുന്നു.
\end{malayalam}}
\flushright{\begin{Arabic}
\quranayah[15][83]
\end{Arabic}}
\flushleft{\begin{malayalam}
അങ്ങനെ ഒരു പ്രഭാതവേളയില്‍ ഘോരഗര്‍ജനം അവരെ പിടികൂടി.
\end{malayalam}}
\flushright{\begin{Arabic}
\quranayah[15][84]
\end{Arabic}}
\flushleft{\begin{malayalam}
അപ്പോള്‍ അവര്‍ നേടിയതൊന്നും അവര്‍ക്ക് ഉപകരിച്ചില്ല.
\end{malayalam}}
\flushright{\begin{Arabic}
\quranayah[15][85]
\end{Arabic}}
\flushleft{\begin{malayalam}
ആകാശഭൂമികളെയും അവയ്ക്കിടയിലുള്ളവയെയും നാം ന്യായമായ ആവശ്യത്തിനല്ലാതെ സൃഷ്ടിച്ചിട്ടില്ല. തീര്‍ച്ചയായും അന്ത്യസമയം വന്നെത്തുക തന്നെ ചെയ്യും. അതിനാല്‍ നീ വിട്ടുവീഴ്ച കാണിക്കുക. മാന്യമായ വിട്ടുവീഴ്ച.
\end{malayalam}}
\flushright{\begin{Arabic}
\quranayah[15][86]
\end{Arabic}}
\flushleft{\begin{malayalam}
നിശ്ചയമായും നിന്റെ നാഥന്‍ എല്ലാറ്റിനെയും സൃഷ്ടിച്ചവനാണ്. എല്ലാം അറിയുന്നവനും.
\end{malayalam}}
\flushright{\begin{Arabic}
\quranayah[15][87]
\end{Arabic}}
\flushleft{\begin{malayalam}
ആവര്‍ത്തിച്ച് പാരായണം ചെയ്യുന്ന ഏഴു സൂക്തങ്ങള്‍ നിനക്കു നാം നല്‍കിയിട്ടുണ്ട്. മഹത്തായ ഈ ഖുര്‍ആനും.
\end{malayalam}}
\flushright{\begin{Arabic}
\quranayah[15][88]
\end{Arabic}}
\flushleft{\begin{malayalam}
അവരിലെ വിവിധ വിഭാഗങ്ങള്‍ക്ക് നാം നല്‍കിയ സുഖഭോഗങ്ങളില്‍ നീ കണ്ണുവെക്കേണ്ടതില്ല. അവരെപ്പറ്റി ദുഃഖിക്കേണ്ടതുമില്ല. സത്യവിശ്വാസികള്‍ക്ക് നീ നിന്റെ ചിറക് താഴ്ത്തിക്കൊടുക്കുക.
\end{malayalam}}
\flushright{\begin{Arabic}
\quranayah[15][89]
\end{Arabic}}
\flushleft{\begin{malayalam}
നീ ഇങ്ങനെ പറയുകയും ചെയ്യുക: "തീര്‍ച്ചയായും ഞാന്‍ വ്യക്തമായ മുന്നറിയിപ്പുകാരന്‍ മാത്രമാണ്.”
\end{malayalam}}
\flushright{\begin{Arabic}
\quranayah[15][90]
\end{Arabic}}
\flushleft{\begin{malayalam}
ശൈഥില്യം സൃഷ്ടിച്ചവര്‍ക്ക് നാം താക്കീതു നല്‍കി. അതുപോലെയാണിതും.
\end{malayalam}}
\flushright{\begin{Arabic}
\quranayah[15][91]
\end{Arabic}}
\flushleft{\begin{malayalam}
ഖുര്‍ആനെ വ്യത്യസ്ത തട്ടുകളാക്കി മാറ്റിയവരാണവര്‍.
\end{malayalam}}
\flushright{\begin{Arabic}
\quranayah[15][92]
\end{Arabic}}
\flushleft{\begin{malayalam}
നിന്റെ നാഥന്‍ സാക്ഷി. അവരെയൊക്കെ നാം വിചാരണ ചെയ്യും.
\end{malayalam}}
\flushright{\begin{Arabic}
\quranayah[15][93]
\end{Arabic}}
\flushleft{\begin{malayalam}
അവര്‍ പ്രവര്‍ത്തിച്ചുകൊണ്ടിരുന്നതിനെപ്പറ്റി.
\end{malayalam}}
\flushright{\begin{Arabic}
\quranayah[15][94]
\end{Arabic}}
\flushleft{\begin{malayalam}
അതിനാല്‍ നിന്നോടാവശ്യപ്പെട്ടതെന്തോ, അത് ഉറക്കെ പ്രഖ്യാപിക്കുക. ബഹുദൈവവാദികളെ തീര്‍ത്തും അവഗണിക്കുക.
\end{malayalam}}
\flushright{\begin{Arabic}
\quranayah[15][95]
\end{Arabic}}
\flushleft{\begin{malayalam}
കളിയാക്കുന്നവരില്‍നിന്ന് നിന്നെ കാക്കാന്‍ നാം തന്നെ മതി.
\end{malayalam}}
\flushright{\begin{Arabic}
\quranayah[15][96]
\end{Arabic}}
\flushleft{\begin{malayalam}
അല്ലാഹുവോടൊപ്പം മറ്റു ദൈവങ്ങളെ സങ്കല്‍പിക്കുന്നവരാണവര്‍. അതിന്റെ ഫലം അടുത്തുതന്നെ അവരറിയും.
\end{malayalam}}
\flushright{\begin{Arabic}
\quranayah[15][97]
\end{Arabic}}
\flushleft{\begin{malayalam}
അവര്‍ പറഞ്ഞുപരത്തുന്നതു കാരണം നിന്റെ മനസ്സ് തിടുങ്ങുന്നുണ്ടെന്ന് നാം അറിയുന്നു.
\end{malayalam}}
\flushright{\begin{Arabic}
\quranayah[15][98]
\end{Arabic}}
\flushleft{\begin{malayalam}
അതിനാല്‍ നീ നിന്റെ നാഥനെ കീര്‍ത്തിച്ച് അവന്റെ വിശുദ്ധി വാഴ്ത്തുക. അവന് പ്രണാമമര്‍പ്പിക്കുന്നവരില്‍ പെടുകയും ചെയ്യുക.
\end{malayalam}}
\flushright{\begin{Arabic}
\quranayah[15][99]
\end{Arabic}}
\flushleft{\begin{malayalam}
നീ നിന്റെ നാഥന്ന് വഴിപ്പെടുക. ആ ഉറപ്പായ കാര്യം നിനക്കു വന്നെത്തുംവരെ.
\end{malayalam}}
\chapter{\textmalayalam{നഹ്ല്‍ ( തേനീച്ച )}}
\begin{Arabic}
\Huge{\centerline{\basmalah}}\end{Arabic}
\flushright{\begin{Arabic}
\quranayah[16][1]
\end{Arabic}}
\flushleft{\begin{malayalam}
അല്ലാഹുവിന്റെ തീരുമാനം വന്നിരിക്കുന്നു. അതിനാല്‍ നിങ്ങളിനി അതിന് ധൃതികാണിക്കേണ്ട. അവര്‍ പങ്കുചേര്‍ക്കുന്നവരില്‍ നിന്നെല്ലാം അല്ലാഹു ഏറെ പരിശുദ്ധനും ഉന്നതനുമാണ്.
\end{malayalam}}
\flushright{\begin{Arabic}
\quranayah[16][2]
\end{Arabic}}
\flushleft{\begin{malayalam}
അല്ലാഹു തന്റെ ദാസന്മാരില്‍ നിന്ന് താനിച്ഛിക്കുന്നവരുടെ മേല്‍ തന്റെ തീരുമാനപ്രകാരം ദിവ്യചൈതന്യവുമായി മലക്കുകളെ ഇറക്കുന്നു. "നിങ്ങള്‍ ജനങ്ങള്‍ക്ക് മുന്നറിയിപ്പ് നല്‍കുക: ഞാനല്ലാതെ ദൈവമില്ല. അതിനാല്‍ എന്നെ സൂക്ഷിച്ചു ജീവിക്കുക.”
\end{malayalam}}
\flushright{\begin{Arabic}
\quranayah[16][3]
\end{Arabic}}
\flushleft{\begin{malayalam}
ആകാശഭൂമികളെ അവന്‍ യാഥാര്‍ഥ്യബോധത്തോടെ സൃഷ്ടിച്ചിരിക്കുന്നു. അവര്‍ പങ്കുചേര്‍ക്കുന്നവക്കെല്ലാം അതീതനാണവന്‍.
\end{malayalam}}
\flushright{\begin{Arabic}
\quranayah[16][4]
\end{Arabic}}
\flushleft{\begin{malayalam}
അല്ലാഹു മനുഷ്യനെ സൃഷ്ടിച്ചത് ഒരു ശുക്ളകണത്തില്‍ നിന്നാണ്. എന്നിട്ടും അവനിതാ തികഞ്ഞ താര്‍ക്കികനായിരിക്കുന്നു.
\end{malayalam}}
\flushright{\begin{Arabic}
\quranayah[16][5]
\end{Arabic}}
\flushleft{\begin{malayalam}
അല്ലാഹു കന്നുകാലികളെ സൃഷ്ടിച്ചു. അവയില്‍ നിങ്ങള്‍ക്ക് തണുപ്പകറ്റാനുള്ള വസ്ത്രമുണ്ട്. മറ്റുപകാരങ്ങളും. നിങ്ങളവയെ തിന്നുകയും ചെയ്യുന്നു.
\end{malayalam}}
\flushright{\begin{Arabic}
\quranayah[16][6]
\end{Arabic}}
\flushleft{\begin{malayalam}
നിങ്ങള്‍ കൌതുകത്തോടെയാണ് അവയെ മേച്ചില്‍സ്ഥലത്തുനിന്ന് തിരിച്ചുകൊണ്ടുവരുന്നത്. മേയാന്‍ വിടുന്നതും അവ്വിധംതന്നെ.
\end{malayalam}}
\flushright{\begin{Arabic}
\quranayah[16][7]
\end{Arabic}}
\flushleft{\begin{malayalam}
കടുത്ത ശാരീരിക പ്രയാസത്തോടെയല്ലാതെ നിങ്ങള്‍ക്ക് ചെന്നെത്താനാവാത്ത നാട്ടിലേക്ക് അവ നിങ്ങളുടെ ഭാരങ്ങള്‍ ചുമന്നുകൊണ്ടുപോവുന്നു. നിങ്ങളുടെ നാഥന്‍ അതീവ ദയാലുവും പരമകാരുണികനുമാണ്.
\end{malayalam}}
\flushright{\begin{Arabic}
\quranayah[16][8]
\end{Arabic}}
\flushleft{\begin{malayalam}
അവന്‍ കുതിരകളെയും കോവര്‍ കഴുതകളെയും കഴുതകളെയും സൃഷ്ടിച്ചു. നിങ്ങള്‍ക്ക് യാത്രക്കുപയോഗിക്കാനും അലങ്കാരമായും. നിങ്ങള്‍ക്കറിയാത്ത പലതും അവന്‍ സൃഷ്ടിക്കുന്നു.
\end{malayalam}}
\flushright{\begin{Arabic}
\quranayah[16][9]
\end{Arabic}}
\flushleft{\begin{malayalam}
നേര്‍വഴി കാണിക്കല്‍ അല്ലാഹുവിന്റെ ബാധ്യതയത്രെ. വഴികളില്‍ പിഴച്ചവയുമുണ്ട്. അല്ലാഹു ഇച്ഛിച്ചിരുന്നെങ്കില്‍ നിങ്ങളെയൊക്കെ അവന്‍ നേര്‍വഴിയിലാക്കുമായിരുന്നു.
\end{malayalam}}
\flushright{\begin{Arabic}
\quranayah[16][10]
\end{Arabic}}
\flushleft{\begin{malayalam}
അവനാണ് മാനത്തുനിന്ന് വെള്ളമിറക്കിയത്. നിങ്ങള്‍ക്കുള്ള കുടിവെള്ളമതാണ്. നിങ്ങള്‍ കാലികളെ മേയാനുപയോഗിക്കുന്ന ചെടികളുണ്ടാവുന്നതും അതിലൂടെയാണ്.
\end{malayalam}}
\flushright{\begin{Arabic}
\quranayah[16][11]
\end{Arabic}}
\flushleft{\begin{malayalam}
അതുവഴി അവന്‍ നിങ്ങള്‍ക്ക് കൃഷിയും ഒലീവും ഈന്തപ്പനയും മുന്തിരിയും മുളപ്പിച്ചുതരുന്നു. എല്ലായിനം കായ്കനികളും. ചിന്തിക്കുന്ന ജനത്തിന് ഇതിലെല്ലാം ധാരാളം തെളിവുകളുണ്ട്.
\end{malayalam}}
\flushright{\begin{Arabic}
\quranayah[16][12]
\end{Arabic}}
\flushleft{\begin{malayalam}
അവന്‍ രാപ്പകലുകളെയും സൂര്യചന്ദ്രന്മാരെയും നിങ്ങള്‍ക്ക് അധീനമാക്കിത്തന്നു. അവന്റെ കല്‍പനപ്രകാരം എല്ലാ നക്ഷത്രങ്ങളും വിധേയമാക്കപ്പെട്ടിരിക്കുന്നു. ചിന്തിക്കുന്ന ജനത്തിന് ഇതില്‍ ധാരാളം തെളിവുകളുണ്ട്.
\end{malayalam}}
\flushright{\begin{Arabic}
\quranayah[16][13]
\end{Arabic}}
\flushleft{\begin{malayalam}
അവന്‍ ഭൂമിയില്‍ നിങ്ങള്‍ക്കായി വിവിധ വര്‍ണങ്ങളില്‍ നിരവധി വസ്തുക്കള്‍ സൃഷ്ടിച്ചുവെച്ചിട്ടുണ്ട്. പാഠമുള്‍ക്കൊള്ളുന്ന ജനത്തിന് അവയിലും മഹത്തായ തെളിവുണ്ട്.
\end{malayalam}}
\flushright{\begin{Arabic}
\quranayah[16][14]
\end{Arabic}}
\flushleft{\begin{malayalam}
അവന്‍ സമുദ്രത്തെ നിങ്ങള്‍ക്ക് വിധേയമാക്കിത്തന്നു. നിങ്ങളതില്‍നിന്ന് പുതുമാംസം ഭക്ഷിക്കാനും നിങ്ങള്‍ക്കണിയാനുള്ള ആഭരണങ്ങള്‍ കണ്ടെടുക്കാനും. കപ്പല്‍ അതിലെ അലമാലകളെ കീറിമുറിച്ച് സഞ്ചരിക്കുന്നത് നീ കാണുന്നുണ്ടല്ലോ: നിങ്ങള്‍ അല്ലാഹുവിന്റെ ഔദാര്യം തേടാന്‍ വേണ്ടി. നിങ്ങള്‍ അവനോട് നന്ദി കാണിക്കുന്നവരാകാനും.
\end{malayalam}}
\flushright{\begin{Arabic}
\quranayah[16][15]
\end{Arabic}}
\flushleft{\begin{malayalam}
ഭൂമിയില്‍ ഊന്നിയുറച്ചു നില്‍ക്കുന്ന മലകള്‍ നാം സ്ഥാപിച്ചിരിക്കുന്നു; ഭൂമി നിങ്ങളെയുംകൊണ്ട് ആടിയുലയാതിരിക്കാന്‍. അവന്‍ പുഴകളും പെരുവഴികളുമുണ്ടാക്കി. നിങ്ങള്‍ നേര്‍വഴി പ്രാപിക്കാന്‍.
\end{malayalam}}
\flushright{\begin{Arabic}
\quranayah[16][16]
\end{Arabic}}
\flushleft{\begin{malayalam}
കൂടാതെ വേറെയും വഴിയടയാളങ്ങളുണ്ട്. നക്ഷത്രങ്ങള്‍ മുഖേനയും അവര്‍ വഴികണ്ടെത്തുന്നു.
\end{malayalam}}
\flushright{\begin{Arabic}
\quranayah[16][17]
\end{Arabic}}
\flushleft{\begin{malayalam}
അപ്പോള്‍ പടച്ചവന്‍ പടക്കാത്തവരെപ്പോലെയാണോ? നിങ്ങള്‍ ചിന്തിച്ചു മനസ്സിലാക്കുന്നില്ലേ?
\end{malayalam}}
\flushright{\begin{Arabic}
\quranayah[16][18]
\end{Arabic}}
\flushleft{\begin{malayalam}
അല്ലാഹുവിന്റെ അനുഗ്രഹങ്ങള്‍ എണ്ണുകയാണെങ്കില്‍ നിങ്ങള്‍ക്കവ തിട്ടപ്പെടുത്താനാവില്ല. തീര്‍ച്ചയായും അല്ലാഹു ഏറെ പൊറുക്കുന്നവനും ദയാപരനുമാണ്.
\end{malayalam}}
\flushright{\begin{Arabic}
\quranayah[16][19]
\end{Arabic}}
\flushleft{\begin{malayalam}
നിങ്ങള്‍ ഒളിപ്പിച്ചുവെക്കുന്നതും തെളിയിച്ചു കാണിക്കുന്നതും അല്ലാഹു അറിയുന്നു.
\end{malayalam}}
\flushright{\begin{Arabic}
\quranayah[16][20]
\end{Arabic}}
\flushleft{\begin{malayalam}
അല്ലാഹുവെക്കൂടാതെ അവര്‍ വിളിച്ചു പ്രാര്‍ഥിക്കുന്നവരാരും ഒന്നും സൃഷ്ടിക്കുന്നില്ല. എന്നല്ല; അവര്‍ തന്നെ സൃഷ്ടിക്കപ്പെടുന്നവരാണ്.
\end{malayalam}}
\flushright{\begin{Arabic}
\quranayah[16][21]
\end{Arabic}}
\flushleft{\begin{malayalam}
അവര്‍ മൃതശരീരങ്ങളാണ്. ജീവനില്ലാത്തവര്‍. തങ്ങള്‍ ഉയിര്‍ത്തെഴുന്നേല്‍പിക്കപ്പെടുക എപ്പോഴെന്നുപോലും അവരറിയുന്നില്ല.
\end{malayalam}}
\flushright{\begin{Arabic}
\quranayah[16][22]
\end{Arabic}}
\flushleft{\begin{malayalam}
നിങ്ങളുടെ ദൈവം ഏകദൈവമാണ്. എന്നാല്‍ പരലോകത്തില്‍ വിശ്വസിക്കാത്തവരുടെ ഹൃദയങ്ങള്‍ അതിനെ നിഷേധിക്കുന്നവയാണ്. അവര്‍ അഹങ്കാരികളാണ്.
\end{malayalam}}
\flushright{\begin{Arabic}
\quranayah[16][23]
\end{Arabic}}
\flushleft{\begin{malayalam}
അവര്‍ ഒളിപ്പിച്ചുവെക്കുന്നതും തെളിയിച്ചുകാണിക്കുന്നതും അല്ലാഹു അറിയുന്നു. അഹങ്കാരികളെ അവന്‍ ഇഷ്ടപ്പെടുന്നില്ല; തീര്‍ച്ച.
\end{malayalam}}
\flushright{\begin{Arabic}
\quranayah[16][24]
\end{Arabic}}
\flushleft{\begin{malayalam}
നിങ്ങളുടെ നാഥന്‍ ഇറക്കിത്തന്നത് എന്താണെന്ന് ചോദിച്ചാല്‍ അവര്‍ പറയും: "പൂര്‍വികരുടെ പഴങ്കഥകള്‍.”
\end{malayalam}}
\flushright{\begin{Arabic}
\quranayah[16][25]
\end{Arabic}}
\flushleft{\begin{malayalam}
ഉയിര്‍ത്തെഴുന്നേല്‍പുനാളില്‍ തങ്ങളുടെ പാപഭാരം പൂര്‍ണമായും ചുമക്കാനാണിത് ഇടവരുത്തുക. ഒരു വിവരവുമില്ലാതെ തങ്ങള്‍ വഴിപിഴപ്പിച്ചുകൊണ്ടിരിക്കുന്നവരുടെ പാപഭാരങ്ങളില്‍ ഒരു പങ്കും അവര്‍ പേറേണ്ടിവരും. അറിയുക: എത്ര ചീത്ത ഭാരമാണ് അവര്‍ ചുമന്നുകൊണ്ടിരിക്കുന്നത്!
\end{malayalam}}
\flushright{\begin{Arabic}
\quranayah[16][26]
\end{Arabic}}
\flushleft{\begin{malayalam}
അവര്‍ക്കു മുമ്പുള്ളവരും ഇങ്ങനെ പല തന്ത്രങ്ങളും പ്രയോഗിച്ചിട്ടുണ്ട്. എന്നാല്‍ അവര്‍ കെട്ടിയുണ്ടാക്കിയ കൌശലപ്പുരയുടെ അസ്ഥിവാരം തന്നെ അല്ലാഹു തകര്‍ത്തുകളഞ്ഞു. അതോടെ അതിന്റെ മേല്‍പ്പുര അവരുടെ മേല്‍ തകര്‍ന്നുവീണു. അവര്‍ക്കറിയാത്ത ഭാഗത്തുനിന്നാണ് ശിക്ഷകള്‍ അവര്‍ക്കു മേല്‍ വന്നുവീണത്.
\end{malayalam}}
\flushright{\begin{Arabic}
\quranayah[16][27]
\end{Arabic}}
\flushleft{\begin{malayalam}
പിന്നീട് ഉയിര്‍ത്തെഴുന്നേല്‍പ് നാളില്‍ അല്ലാഹു അവരെ നിന്ദ്യരാക്കും. അവന്‍ അവരോടിങ്ങനെ ചോദിക്കും: "ഇപ്പോള്‍ എന്റെ പങ്കാളികളെവിടെ? അവര്‍ക്കു വേണ്ടിയായിരുന്നുവല്ലോ നിങ്ങള്‍ ചേരിതിരിഞ്ഞു തര്‍ക്കിച്ചിരുന്നത്?” അറിവുള്ളവര്‍ പറയും: "ഇന്ന് നിന്ദ്യതയും ശിക്ഷയും സത്യനിഷേധികള്‍ക്കു തന്നെ.”
\end{malayalam}}
\flushright{\begin{Arabic}
\quranayah[16][28]
\end{Arabic}}
\flushleft{\begin{malayalam}
തങ്ങളോട് തന്നെ ദ്രോഹം ചെയ്തുകൊണ്ടിരിക്കെ മലക്കുകള്‍ ജീവന്‍ പിടിച്ചെടുക്കുമ്പോള്‍ അവര്‍ അല്ലാഹുവിന് കീഴ്പെടും. “ഞങ്ങള്‍ തെറ്റൊന്നും ചെയ്തിരുന്നില്ലല്ലോ” എന്നു പറയുകയും ചെയ്യും. എന്നാല്‍; നിശ്ചയമായും നിങ്ങള്‍ പ്രവര്‍ത്തിച്ചുകൊണ്ടിരുന്നതിനെപ്പറ്റി നന്നായറിയുന്നവനാണ് അല്ലാഹു.
\end{malayalam}}
\flushright{\begin{Arabic}
\quranayah[16][29]
\end{Arabic}}
\flushleft{\begin{malayalam}
അതിനാല്‍ നരക കവാടങ്ങളിലൂടെ കടന്നുകൊള്ളുക. നിങ്ങളവിടെ നിത്യവാസികളായിരിക്കും. അഹങ്കാരികളുടേത് എത്ര ചീത്ത സങ്കേതം!
\end{malayalam}}
\flushright{\begin{Arabic}
\quranayah[16][30]
\end{Arabic}}
\flushleft{\begin{malayalam}
സൂക്ഷ്മത പാലിച്ചവരോട് ചോദിക്കും: "നിങ്ങളുടെ നാഥന്‍ എന്താണ് ഇറക്കിത്തന്നത്?” അപ്പോഴവര്‍ പറയും: "നല്ലതു തന്നെ.” സുകൃതം ചെയ്തവര്‍ക്ക് ഈ ലോകത്തുതന്നെ സദ്ഫലമുണ്ട്. പരലോക ഭവനമോ കൂടുതലുത്തമവും. ഭക്തന്മാര്‍ക്കുള്ള ഭവനം എത്ര മഹത്തരം!
\end{malayalam}}
\flushright{\begin{Arabic}
\quranayah[16][31]
\end{Arabic}}
\flushleft{\begin{malayalam}
സ്ഥിരവാസത്തിനുള്ള സ്വര്‍ഗീയാരാമങ്ങളാണത്. അവരതില്‍ പ്രവേശിക്കും. അതിന്റെ താഴ്ഭാഗത്തൂടെ ആറുകളൊഴുകിക്കൊണ്ടിരിക്കും. അവരാഗ്രഹിക്കുന്നതൊക്കെ അവര്‍ക്കവിടെ കിട്ടും. അവ്വിധമാണ് അല്ലാഹു സൂക്ഷ്മതയുള്ളവര്‍ക്ക് പ്രതിഫലം നല്‍കുന്നത്.
\end{malayalam}}
\flushright{\begin{Arabic}
\quranayah[16][32]
\end{Arabic}}
\flushleft{\begin{malayalam}
വിശുദ്ധരായിരിക്കെ മലക്കുകള്‍ മരിപ്പിക്കുന്നവരാണവര്‍. മലക്കുകള്‍ അവരോട് പറയും: "നിങ്ങള്‍ക്കു ശാന്തി! നിങ്ങള്‍ സ്വര്‍ഗത്തില്‍ പ്രവേശിച്ചുകൊള്ളുക. നിങ്ങള്‍ പ്രവര്‍ത്തിച്ചുകൊണ്ടിരുന്നതിന്റെ പ്രതിഫലമാണിത്.”
\end{malayalam}}
\flushright{\begin{Arabic}
\quranayah[16][33]
\end{Arabic}}
\flushleft{\begin{malayalam}
ഈ ജനത്തിനെന്താണ് പ്രതീക്ഷിക്കാനുള്ളത്? അവരുടെ അടുത്ത് മലക്കുകള്‍ വരുന്നതോ അല്ലെങ്കില്‍ നിന്റെ നാഥന്റെ കല്‍പന വന്നെത്തുന്നതോ അല്ലാതെ. അവ്വിധം തന്നെയാണ് അവര്‍ക്ക് മുമ്പുള്ളവരും ചെയ്തത്. അല്ലാഹു അവരോട് ഒരക്രമവും ചെയ്തിട്ടില്ല. അവര്‍ തങ്ങളോടുതന്നെ അക്രമം കാണിക്കുകയായിരുന്നു.
\end{malayalam}}
\flushright{\begin{Arabic}
\quranayah[16][34]
\end{Arabic}}
\flushleft{\begin{malayalam}
അങ്ങനെ അവര്‍ ചെയ്തതിന്റെ ദുരന്തഫലങ്ങള്‍ അവരെ ബാധിച്ചു. അവര്‍ കളിയാക്കിത്തള്ളിയിരുന്ന ശിക്ഷ അവരെ വലയം ചെയ്തു.
\end{malayalam}}
\flushright{\begin{Arabic}
\quranayah[16][35]
\end{Arabic}}
\flushleft{\begin{malayalam}
ബഹുദൈവ വിശ്വാസികള്‍ പറഞ്ഞു: "അല്ലാഹു ഇച്ഛിച്ചിരുന്നെങ്കില്‍ ഞങ്ങളോ ഞങ്ങളുടെ പിതാക്കന്മാരോ അവനെക്കൂടാതെ ഒന്നിനെയും പൂജിക്കുമായിരുന്നില്ല. അവന്റെ വിധിയില്ലാതെ ഒന്നും നിഷിദ്ധമാക്കുമായിരുന്നില്ല.” അവര്‍ക്കു മുമ്പുള്ളവരും ഇതുതന്നെ ചെയ്തിട്ടുണ്ട്. സന്ദേശം വ്യക്തമായി എത്തിച്ചുകൊടുക്കുകയെന്നതല്ലാത്ത എന്തു ബാധ്യതയാണ് ദൈവദൂതന്മാര്‍ക്കുള്ളത്?
\end{malayalam}}
\flushright{\begin{Arabic}
\quranayah[16][36]
\end{Arabic}}
\flushleft{\begin{malayalam}
നിശ്ചയമായും എല്ലാ സമുദായത്തിലും നാം ദൂതനെ നിയോഗിച്ചിട്ടുണ്ട്. അവരൊക്കെ പറഞ്ഞതിതാണ്: "നിങ്ങള്‍ അല്ലാഹുവിന് വഴിപ്പെടുക; വ്യാജ ദൈവങ്ങളെ വര്‍ജിക്കുക.” അങ്ങനെ അവരില്‍ ചിലരെ അല്ലാഹു നേര്‍വഴിയിലാക്കി. മറ്റു ചിലരെ ദുര്‍മാര്‍ഗം കീഴ്പ്പെടുത്തുകയും ചെയ്തു. അതിനാല്‍ നിങ്ങള്‍ ഭൂമിയിലൂടെ സഞ്ചരിക്കൂ. എന്നിട്ട് സത്യത്തെ നിഷേധിച്ചുതള്ളിയവരുടെ ഒടുക്കം എവ്വിധമായിരുന്നുവെന്ന് നോക്കിക്കാണുക.
\end{malayalam}}
\flushright{\begin{Arabic}
\quranayah[16][37]
\end{Arabic}}
\flushleft{\begin{malayalam}
അവരെ നേര്‍വഴിയിലാക്കണമെന്ന് നീയെത്ര തന്നെ ആഗ്രഹിച്ചാലും, അല്ലാഹു വഴികേടിലാക്കുന്നവരെ അവന്‍ നേര്‍വഴിയിലാക്കുകയില്ല. അവര്‍ക്ക് സഹായികളായി ആരുമില്ല.
\end{malayalam}}
\flushright{\begin{Arabic}
\quranayah[16][38]
\end{Arabic}}
\flushleft{\begin{malayalam}
അല്ലാഹുവിന്റെ പേരില്‍ തങ്ങള്‍ക്കാവും വിധം ദൃഢതയോടെ ആണയിട്ട് അവര്‍ പറയുന്നു: "മരിച്ചവരെ അല്ലാഹു വീണ്ടും ജീവിപ്പിച്ചെഴുന്നേല്‍പിക്കുകയില്ല.” എന്നാല്‍ അങ്ങനെയല്ല. അതൊരു വാഗ്ദാനമാണ്. അതിന്റെ പൂര്‍ത്തീകരണം അല്ലാഹു തന്റെ ബാധ്യതയായി ഏറ്റെടുത്തിരിക്കുന്നു. എന്നാല്‍ മനുഷ്യരിലേറെപ്പേരും അതറിയുന്നില്ല.
\end{malayalam}}
\flushright{\begin{Arabic}
\quranayah[16][39]
\end{Arabic}}
\flushleft{\begin{malayalam}
തങ്ങള്‍ ഭിന്നിച്ചുകൊണ്ടിരിക്കുന്നവയുടെ നിജസ്ഥിതി സത്യനിഷേധികള്‍ക്ക് വിവരിച്ചുകൊടുക്കാനാണിത്. തങ്ങള്‍ കള്ളം പറയുന്നവരായിരുന്നുവെന്ന് അവര്‍ക്ക് ബോധ്യമാകാനും.
\end{malayalam}}
\flushright{\begin{Arabic}
\quranayah[16][40]
\end{Arabic}}
\flushleft{\begin{malayalam}
ഒരു വസ്തു ഉണ്ടാകണമെന്ന് നാം ഉദ്ദേശിച്ചാല്‍ “ഉണ്ടാകൂ” എന്നു കല്‍പിക്കുകയേ വേണ്ടൂ, അപ്പോഴേക്കും അതുണ്ടാകുന്നു.
\end{malayalam}}
\flushright{\begin{Arabic}
\quranayah[16][41]
\end{Arabic}}
\flushleft{\begin{malayalam}
മര്‍ദനത്തിനിരയായശേഷം അല്ലാഹുവിന്റെ മാര്‍ഗത്തില്‍ പലായനം ചെയ്തവര്‍ക്ക് നാം ഈ ലോകത്ത് മെച്ചമായ പാര്‍പ്പിടം ഒരുക്കിക്കൊടുക്കുക തന്നെ ചെയ്യും. പരലോകത്തെ പ്രതിഫലമോ, അതിമഹത്തരവും. അവരിതെല്ലാം അറിഞ്ഞിരുന്നെങ്കില്‍!
\end{malayalam}}
\flushright{\begin{Arabic}
\quranayah[16][42]
\end{Arabic}}
\flushleft{\begin{malayalam}
അവരോ, ക്ഷമ പാലിക്കുകയും തങ്ങളുടെ നാഥനില്‍ ഭരമേല്‍പിക്കുകയും ചെയ്തവരാണ്.
\end{malayalam}}
\flushright{\begin{Arabic}
\quranayah[16][43]
\end{Arabic}}
\flushleft{\begin{malayalam}
ചില പുരുഷന്മാരെയല്ലാതെ നിനക്കു മുമ്പ് നാം ദൂതന്മാരായി ആരെയും നിയോഗിച്ചിട്ടില്ല. നാം അവര്‍ക്ക് സന്ദേശം നല്‍കുന്നു. ഇതൊന്നും നിങ്ങള്‍ക്കറിയില്ലെങ്കില്‍ നേരത്തെ ഉദ്ബോധനം ലഭിച്ചവരോടു ചോദിച്ചറിയുക.
\end{malayalam}}
\flushright{\begin{Arabic}
\quranayah[16][44]
\end{Arabic}}
\flushleft{\begin{malayalam}
വ്യക്തമായ പ്രമാണങ്ങളും വേദപുസ്തകങ്ങളുമായാണ് നാമവരെ നിയോഗിച്ചത്. ഇപ്പോള്‍ നിനക്കും നാമിതാ ഈ വേദപുസ്തകം ഇറക്കിത്തന്നിരിക്കുന്നു. ജനങ്ങള്‍ക്കായി അവതീര്‍ണമായത് നീയവര്‍ക്ക് വിശദീകരിച്ചുകൊടുക്കാന്‍. അങ്ങനെ ജനം ചിന്തിച്ചുമനസ്സിലാക്കട്ടെ!
\end{malayalam}}
\flushright{\begin{Arabic}
\quranayah[16][45]
\end{Arabic}}
\flushleft{\begin{malayalam}
നീചമായ കുതന്ത്രങ്ങള്‍ പ്രയോഗിക്കുന്നവരിപ്പോള്‍ സമാശ്വസിക്കുകയാണോ; അല്ലാഹു അവരെ ഭൂമിയില്‍ ആഴ്ത്തിക്കളയുകയില്ലെന്ന്? അല്ലെങ്കില്‍ വിചാരിക്കാത്ത ഭാഗത്തുനിന്ന് ശിക്ഷ അവര്‍ക്ക് വന്നെത്തുകയില്ലെന്ന്?
\end{malayalam}}
\flushright{\begin{Arabic}
\quranayah[16][46]
\end{Arabic}}
\flushleft{\begin{malayalam}
അല്ലെങ്കില്‍, അവരുടെ പോക്കുവരവിനിടയില്‍ അവര്‍ക്ക് അതിജയിക്കാനാവാത്ത വിധം അല്ലാഹു അവരെ പിടികൂടുകയില്ലെന്നാണോ അവരാശ്വസിക്കുന്നത്?
\end{malayalam}}
\flushright{\begin{Arabic}
\quranayah[16][47]
\end{Arabic}}
\flushleft{\begin{malayalam}
അതുമല്ലെങ്കില്‍ അവര്‍ പേടിച്ചുവിറച്ചുകൊണ്ടിരിക്കെ അല്ലാഹു അവരെ പിടികൂടുകയില്ലെന്ന്? എന്നാല്‍ നിങ്ങളുടെ നാഥന്‍ ഏറെ കരുണയുള്ളവനും പരമദയാലുവുമാണ്.
\end{malayalam}}
\flushright{\begin{Arabic}
\quranayah[16][48]
\end{Arabic}}
\flushleft{\begin{malayalam}
പടച്ചവന്‍ പടച്ച പദാര്‍ഥങ്ങളുടെ നിഴലുകള്‍ പോലും ഇടത്തോട്ടും വലത്തോട്ടും ചാഞ്ഞും ചെരിഞ്ഞും ഏറെ വിനീതമായി അല്ലാഹുവിന് പ്രണാമമര്‍പ്പിക്കുന്നത് ഇവര്‍ കാണുന്നില്ലേ?
\end{malayalam}}
\flushright{\begin{Arabic}
\quranayah[16][49]
\end{Arabic}}
\flushleft{\begin{malayalam}
വിണ്ണിലും മണ്ണിലുമുള്ള ജീവികളൊക്കെയും അല്ലാഹുവിന് പ്രണാമമര്‍പ്പിക്കുന്നു. മലക്കുകള്‍പോലും താന്‍പോരിമ നടിക്കാതെ അവനെ പ്രണമിക്കുന്നു.
\end{malayalam}}
\flushright{\begin{Arabic}
\quranayah[16][50]
\end{Arabic}}
\flushleft{\begin{malayalam}
അവരൊക്കെയും തങ്ങളുടെ മീതെയുള്ള നാഥനെ ഭയപ്പെടുന്നു. അവന്‍ കല്‍പിക്കുന്നതൊക്കെയും അവര്‍ പ്രാവര്‍ത്തികമാക്കുന്നു.
\end{malayalam}}
\flushright{\begin{Arabic}
\quranayah[16][51]
\end{Arabic}}
\flushleft{\begin{malayalam}
അല്ലാഹു കല്‍പിച്ചിരിക്കുന്നു: രണ്ട് ദൈവങ്ങളെ നിങ്ങള്‍ സ്വീകരിക്കരുത്. ഒരൊറ്റ ദൈവമേയുള്ളൂ. അതിനാല്‍ നിങ്ങള്‍ എന്നെ മാത്രം ഭയപ്പെടുക.
\end{malayalam}}
\flushright{\begin{Arabic}
\quranayah[16][52]
\end{Arabic}}
\flushleft{\begin{malayalam}
ആകാശഭൂമികളിലുള്ളതെല്ലാം അല്ലാഹുവിന്റേതാണ്. അവിരാമമായ വണക്കം അവനു മാത്രം. എന്നിട്ടും അല്ലാഹു അല്ലാത്തവരോടാണോ നിങ്ങള്‍ ഭക്തിപുലര്‍ത്തുന്നത്?
\end{malayalam}}
\flushright{\begin{Arabic}
\quranayah[16][53]
\end{Arabic}}
\flushleft{\begin{malayalam}
നിങ്ങള്‍ക്കുണ്ടാവുന്ന ഏതനുഗ്രഹവും അല്ലാഹുവില്‍ നിന്നുള്ളതാണ്. പിന്നീട് നിങ്ങള്‍ക്ക് വല്ല വിപത്തും വന്നുപെട്ടാല്‍ അവങ്കലേക്കു തന്നെയാണ് നിങ്ങള്‍ വേവലാതികളോടെ പാഞ്ഞടുക്കുന്നത്.
\end{malayalam}}
\flushright{\begin{Arabic}
\quranayah[16][54]
\end{Arabic}}
\flushleft{\begin{malayalam}
പിന്നെ, നിങ്ങളില്‍ നിന്ന് ആ വിപത്ത് അവന്‍ നീക്കിക്കളഞ്ഞാല്‍ നിങ്ങളിലൊരു വിഭാഗം തങ്ങളുടെ നാഥനില്‍ മറ്റുള്ളവരെ പങ്കുചേര്‍ക്കുന്നു.
\end{malayalam}}
\flushright{\begin{Arabic}
\quranayah[16][55]
\end{Arabic}}
\flushleft{\begin{malayalam}
നാം അവര്‍ക്കു നല്‍കിയതിനോടുള്ള നന്ദികേട് കാണിക്കലാണിത്. അതിനാല്‍ നിങ്ങള്‍ സുഖിച്ചുകൊള്ളുക. വൈകാതെ എല്ലാം നിങ്ങളറിയും.
\end{malayalam}}
\flushright{\begin{Arabic}
\quranayah[16][56]
\end{Arabic}}
\flushleft{\begin{malayalam}
നാം അവര്‍ക്കു നല്‍കിയതില്‍നിന്ന് ഒരു വിഹിതം അവര്‍ക്കുതന്നെ അറിയാത്ത ചിലതിന്നായി അവര്‍ നീക്കിവെക്കുന്നു. അല്ലാഹുവാണ് സത്യം. നിങ്ങള്‍ കൃത്രിമമായി കെട്ടിച്ചമക്കുന്നതിനെ സംബന്ധിച്ച് നിങ്ങള്‍ ചോദ്യം ചെയ്യപ്പെടുകതന്നെ ചെയ്യും.
\end{malayalam}}
\flushright{\begin{Arabic}
\quranayah[16][57]
\end{Arabic}}
\flushleft{\begin{malayalam}
അല്ലാഹുവിന് പെണ്‍മക്കളുണ്ടെന്ന് അവര്‍ ആരോപിക്കുന്നു- അവന്‍ എത്ര പരിശുദ്ധന്‍! അവര്‍ക്കോ അവര്‍ ഇഷ്ടപ്പെടുന്നതും.
\end{malayalam}}
\flushright{\begin{Arabic}
\quranayah[16][58]
\end{Arabic}}
\flushleft{\begin{malayalam}
അവരിലൊരാള്‍ക്ക് പെണ്‍കുഞ്ഞ് പിറന്നതായി സന്തോഷവാര്‍ത്ത ലഭിച്ചാല്‍ ദുഃഖത്താല്‍ അവന്റെ മുഖം കറുത്തിരുളും.
\end{malayalam}}
\flushright{\begin{Arabic}
\quranayah[16][59]
\end{Arabic}}
\flushleft{\begin{malayalam}
തനിക്കു ലഭിച്ച സന്തോഷവാര്‍ത്തയുണ്ടാക്കുന്ന അപമാനത്താല്‍ അവന്‍ ആളുകളില്‍ നിന്ന് ഒളിഞ്ഞുമറയുന്നു. അയാളുടെ പ്രശ്നം, അപമാനം സഹിച്ച് അതിനെ നിലനിര്‍ത്തണമോ അതല്ല മണ്ണില്‍ കുഴിച്ചുമൂടണമോ എന്നതാണ്. അറിയുക: അവരുടെ തീരുമാനം വളരെ നീചം തന്നെ!
\end{malayalam}}
\flushright{\begin{Arabic}
\quranayah[16][60]
\end{Arabic}}
\flushleft{\begin{malayalam}
പരലോകത്തില്‍ വിശ്വസിക്കാത്തവര്‍ക്കാണ് പതിതാവസ്ഥ. അത്യുന്നതാവസ്ഥ അല്ലാഹുവിനാണ്. അവന്‍ അജയ്യനും യുക്തിമാനുമാണ്.
\end{malayalam}}
\flushright{\begin{Arabic}
\quranayah[16][61]
\end{Arabic}}
\flushleft{\begin{malayalam}
ജനത്തെ അവരുടെ അക്രമത്തിന്റെ പേരില്‍ അല്ലാഹു പെട്ടെന്ന് പിടികൂടുകയാണെങ്കില്‍ ഭൂമുഖത്ത് ഒരു ജീവിയെയും അവന്‍ വിട്ടേക്കുമായിരുന്നില്ല. എന്നാല്‍ നിശ്ചിത അവധിവരെ അവര്‍ക്ക് അവന്‍ അവസരം അനുവദിക്കുകയാണ്. അങ്ങനെ അവരുടെ കാലാവധി വന്നെത്തിയാല്‍ പിന്നെ ഒരു നിമിഷംപോലും അവര്‍ക്കത് വൈകിക്കാനാവില്ല. നേരത്തെയാക്കാനും സാധ്യമല്ല.
\end{malayalam}}
\flushright{\begin{Arabic}
\quranayah[16][62]
\end{Arabic}}
\flushleft{\begin{malayalam}
അവര്‍ തങ്ങള്‍ക്കായി ഇഷ്ടപ്പെടാത്ത വസ്തുക്കള്‍ അല്ലാഹുവിനുള്ളതായി സങ്കല്‍പിക്കുന്നു. ഏറ്റവും നല്ലത് മാത്രമാണ് തങ്ങള്‍ക്കുണ്ടാവുകയെന്ന് അവരുടെ നാവുകള്‍ കള്ളംപറയുന്നു. സംശയമില്ല; അവര്‍ക്കുള്ളത് നരകമാണ്. മറ്റാരെക്കാളും മുമ്പെ അവരാണവിടേക്ക് നയിക്കപ്പെടുക.
\end{malayalam}}
\flushright{\begin{Arabic}
\quranayah[16][63]
\end{Arabic}}
\flushleft{\begin{malayalam}
അല്ലാഹുവാണ് സത്യം. നിനക്കു മുമ്പ് പല സമുദായങ്ങളിലേക്കും നാം ദൂതന്മാരെ അയച്ചിട്ടുണ്ട്. അപ്പോഴൊക്കെ ആ ജനത്തിന്റെ ദുര്‍വൃത്തികള്‍ പിശാച് അവര്‍ക്ക് ചേതോഹരമാക്കിത്തോന്നിക്കുകയായിരുന്നു. അതിനാല്‍ അവനാണ് ഇന്ന് അവരുടെ രക്ഷാധികാരി. അവര്‍ക്ക് നോവേറിയ ശിക്ഷയുണ്ട്.
\end{malayalam}}
\flushright{\begin{Arabic}
\quranayah[16][64]
\end{Arabic}}
\flushleft{\begin{malayalam}
നിനക്കു നാം വേദപുസ്തകം ഇറക്കിത്തന്നത് അവര്‍ക്കിടയില്‍ അഭിപ്രായ ഭിന്നതകളുള്ള കാര്യങ്ങളുടെ യാഥാര്‍ഥ്യം അവര്‍ക്ക് വിവരിച്ചുകൊടുക്കാനാണ്. വിശ്വസിക്കുന്ന ജനത്തിന് നേര്‍വഴി കാട്ടാനും. ഒപ്പം അനുഗ്രഹമായും.
\end{malayalam}}
\flushright{\begin{Arabic}
\quranayah[16][65]
\end{Arabic}}
\flushleft{\begin{malayalam}
അല്ലാഹു മാനത്തുനിന്ന് മഴ പെയ്യിച്ചു. അതുവഴി അവന്‍ ജീവനറ്റ ഭൂമിക്ക് ജീവനേകി. സംശയമില്ല; കേട്ടറിയുന്ന ജനത്തിന് ഇതില്‍ ദൃഷ്ടാന്തമുണ്ട്.
\end{malayalam}}
\flushright{\begin{Arabic}
\quranayah[16][66]
\end{Arabic}}
\flushleft{\begin{malayalam}
നിശ്ചയമായും കന്നുകാലികളിലും നിങ്ങള്‍ക്ക് പാഠമുണ്ട്. അവയുടെ വയറ്റിലുള്ളതില്‍ നിന്ന്, ചാണകത്തിനും ചോരക്കുമിടയില്‍നിന്ന് നിങ്ങളെ നാം ശുദ്ധമായ പാല്‍ കുടിപ്പിക്കുന്നു. കുടിക്കുന്നവര്‍ക്കെല്ലാം ആനന്ദദായകമാണത്.
\end{malayalam}}
\flushright{\begin{Arabic}
\quranayah[16][67]
\end{Arabic}}
\flushleft{\begin{malayalam}
ഈന്തപ്പനയുടെയും മുന്തിരിവള്ളിയുടെയും പഴങ്ങളില്‍ നിന്ന് നിങ്ങള്‍ ലഹരി പദാര്‍ഥവും നല്ല ആഹാരവും ഉണ്ടാക്കുന്നു. ചിന്തിക്കുന്ന ജനത്തിന് തീര്‍ച്ചയായും അതില്‍ അടയാളമുണ്ട്.
\end{malayalam}}
\flushright{\begin{Arabic}
\quranayah[16][68]
\end{Arabic}}
\flushleft{\begin{malayalam}
നിന്റെ നാഥന്‍ തേനീച്ചക്ക് ബോധനം നല്‍കി; "മലകളിലും മരങ്ങളിലും മനുഷ്യര്‍ കെട്ടിയുയര്‍ത്തുന്ന പന്തലുകളിലും നിങ്ങള്‍ കൂടുണ്ടാക്കുക.
\end{malayalam}}
\flushright{\begin{Arabic}
\quranayah[16][69]
\end{Arabic}}
\flushleft{\begin{malayalam}
"പിന്നെ എല്ലാത്തരം ഫലങ്ങളില്‍നിന്നും ഭക്ഷിക്കുക. അങ്ങനെ നിന്റെ നാഥന്‍ പാകപ്പെടുത്തിവച്ച വഴികളില്‍ പ്രവേശിക്കുക.” അവയുടെ വയറുകളില്‍ നിന്ന് വര്‍ണവൈവിധ്യമുള്ള പാനീയം സ്രവിക്കുന്നു. അതില്‍ മനുഷ്യര്‍ക്ക് രോഗശമനമുണ്ട്. ചിന്തിക്കുന്ന ജനത്തിന് ഇതിലും ദൃഷ്ടാന്തമുണ്ട്.
\end{malayalam}}
\flushright{\begin{Arabic}
\quranayah[16][70]
\end{Arabic}}
\flushleft{\begin{malayalam}
അല്ലാഹു നിങ്ങളെ സൃഷ്ടിച്ചു. പിന്നെ നിങ്ങളെ അവന്‍ മരിപ്പിക്കുന്നു. നിങ്ങളില്‍ ചിലര്‍ അങ്ങേയറ്റത്തെ വാര്‍ധക്യത്തിലേക്ക് തള്ളപ്പെടുന്നു. പലതും അറിയാവുന്ന അവസ്ഥക്കുശേഷം ഒന്നും അറിയാത്ത സ്ഥിതിയിലെത്താനാണിത്. അല്ലാഹു എല്ലാം അറിയുന്നവനാണ്. എല്ലാറ്റിനും കഴിവുറ്റവനും.
\end{malayalam}}
\flushright{\begin{Arabic}
\quranayah[16][71]
\end{Arabic}}
\flushleft{\begin{malayalam}
ആഹാരകാര്യത്തില്‍ അല്ലാഹു നിങ്ങളില്‍ ചിലരെ മറ്റു ചിലരെക്കാള്‍ മികവുറ്റവരാക്കിയിരിക്കുന്നു. എന്നാല്‍ മികവ് ലഭിച്ചവര്‍ തങ്ങളുടെ വിഭവം തങ്ങളുടെ ഭൃത്യന്മാര്‍ക്ക് വിട്ടുകൊടുക്കുന്നതിലൂടെ അവരെയൊക്കെ തങ്ങളെപ്പോലെ അതില്‍ സമന്മാരാക്കുന്നില്ല. അപ്പോള്‍ പിന്നെ അല്ലാഹുവിന്റെ അനുഗ്രഹത്തെയാണോ അവര്‍ നിഷേധിക്കുന്നത്?
\end{malayalam}}
\flushright{\begin{Arabic}
\quranayah[16][72]
\end{Arabic}}
\flushleft{\begin{malayalam}
അല്ലാഹു നിങ്ങള്‍ക്ക് നിങ്ങളുടെ വര്‍ഗത്തില്‍ നിന്നുതന്നെ ഇണകളെ ഉണ്ടാക്കിത്തന്നു. നിങ്ങള്‍ക്ക് നിങ്ങളുടെ ഇണകളിലൂടെ പുത്രന്മാരെയും നല്‍കി. പൌത്രന്മാരെയും. വിശിഷ്ട വസ്തുക്കള്‍ ആഹാരമായി തന്നു. എന്നിട്ടും ഇക്കൂട്ടര്‍ അസത്യത്തില്‍ വിശ്വസിക്കുകയാണോ? അല്ലാഹുവിന്റെ അനുഗ്രഹത്തെ അപ്പാടെ തള്ളിപ്പറയുകയും?
\end{malayalam}}
\flushright{\begin{Arabic}
\quranayah[16][73]
\end{Arabic}}
\flushleft{\begin{malayalam}
ആകാശഭൂമികളില്‍ നിന്ന് അവര്‍ക്ക് ആഹാരമൊന്നും നല്‍കാത്തവരെയും ഒന്നിനും കഴിയാത്തവരെയുമാണ് അല്ലാഹുവെക്കൂടാതെ അവര്‍ പൂജിച്ചുകൊണ്ടിരിക്കുന്നത്.
\end{malayalam}}
\flushright{\begin{Arabic}
\quranayah[16][74]
\end{Arabic}}
\flushleft{\begin{malayalam}
അതിനാല്‍ നിങ്ങള്‍ അല്ലാഹുവെ മറ്റൊന്നുമായി സാദൃശ്യപ്പെടുത്തരുത്. തീര്‍ച്ചയായും അല്ലാഹു എല്ലാം അറിയുന്നു. നിങ്ങള്‍ അറിയുന്നുമില്ല.
\end{malayalam}}
\flushright{\begin{Arabic}
\quranayah[16][75]
\end{Arabic}}
\flushleft{\begin{malayalam}
അല്ലാഹു ഒരുദാഹരണം സമര്‍പ്പിക്കുന്നു: ഒരാള്‍ മറ്റൊരാളുടെ ഉടമയിലുള്ള അടിമയാണ്. അയാള്‍ക്കൊന്നിനും കഴിയില്ല; മറ്റൊരാള്‍, നാം നമ്മുടെ വകയായി നല്‍കിയ ഉത്തമമായ ആഹാരപദാര്‍ഥങ്ങളില്‍ നിന്ന് രഹസ്യമായും പരസ്യമായും ചെലവഴിച്ചുകൊണ്ടിരിക്കുന്നു. അവരിരുവരും തുല്യരാണോ? അല്ലാഹുവിന് സ്തുതി. എങ്കിലും അവരിലേറെ പേരും കാര്യം മനസ്സിലാക്കുന്നില്ല.
\end{malayalam}}
\flushright{\begin{Arabic}
\quranayah[16][76]
\end{Arabic}}
\flushleft{\begin{malayalam}
അല്ലാഹു മറ്റൊരുദാഹരണം കൂടി നല്‍കുന്നു: രണ്ടാളുകള്‍. അവരിലൊരുവന്‍ ഊമയാണ്. ഒന്നിനും കഴിയാത്തവന്‍. അവന്‍ തന്റെ യജമാനന് ഒരു ഭാരമാണ്. അയാള്‍ അവനെ എവിടേക്കയച്ചാലും അവനൊരു നന്മയും വരുത്തുകയില്ല. അയാളും, സ്വയം നേര്‍വഴിയില്‍ നിലയുറപ്പിച്ച് നീതി കല്‍പിക്കുന്നവനും ഒരുപോലെയാണോ?
\end{malayalam}}
\flushright{\begin{Arabic}
\quranayah[16][77]
\end{Arabic}}
\flushleft{\begin{malayalam}
ആകാശഭൂമികളില്‍ ഒളിഞ്ഞിരിക്കുന്നവയൊക്കെയും നന്നായറിയുന്നവന്‍ അല്ലാഹു മാത്രമാണ്. ആ അന്ത്യസമയം ഇമവെട്ടുംപോലെ മാത്രമാണ്. അല്ലെങ്കില്‍ അതിനെക്കാള്‍ വേഗതയുള്ളത്. അല്ലാഹു എല്ലാ കാര്യങ്ങള്‍ക്കും കഴിവുറ്റവനാണ്.
\end{malayalam}}
\flushright{\begin{Arabic}
\quranayah[16][78]
\end{Arabic}}
\flushleft{\begin{malayalam}
അല്ലാഹു നിങ്ങളെ മാതാക്കളുടെ ഉദരങ്ങളില്‍ നിന്ന് ഒന്നും അറിയാത്തവരായി പുറത്തേക്ക് കൊണ്ടുവന്നു; പിന്നെ നിങ്ങള്‍ക്ക് അവന്‍ കേള്‍വിയും കാഴ്ചകളും ഹൃദയങ്ങളും നല്‍കി. നിങ്ങള്‍ നന്ദിയുള്ളവരാകാന്‍.
\end{malayalam}}
\flushright{\begin{Arabic}
\quranayah[16][79]
\end{Arabic}}
\flushleft{\begin{malayalam}
ഇവര്‍ പറവകളെ കാണുന്നില്ലേ? അന്തരീക്ഷത്തില്‍ അവ എവ്വിധം അധീനമാക്കപ്പെട്ടിരിക്കുന്നുവെന്ന്. അല്ലാഹുവല്ലാതെ ആരും അവയെ താങ്ങിനിര്‍ത്തുന്നില്ല. വിശ്വസിക്കുന്ന ജനത്തിന് ഇതില്‍ ധാരാളം ദൃഷ്ടാന്തങ്ങളുണ്ട്.
\end{malayalam}}
\flushright{\begin{Arabic}
\quranayah[16][80]
\end{Arabic}}
\flushleft{\begin{malayalam}
അല്ലാഹു നിങ്ങളുടെ വീടുകളെ നിങ്ങള്‍ക്കുള്ള വിശ്രമസ്ഥലങ്ങളാക്കി. മൃഗത്തോലുകളില്‍നിന്ന് അവന്‍ നിങ്ങള്‍ക്ക് പാര്‍പ്പിടങ്ങളുണ്ടാക്കിത്തന്നു. നിങ്ങളുടെ യാത്രാ നാളുകളിലും താവളമടിക്കുന്ന ദിനങ്ങളിലും നിങ്ങളവ അനായാസം ഉപയോഗപ്പെടുത്തുന്നു. ചെമ്മരിയാടുകളുടെയും ഒട്ടകങ്ങളുടെയും കോലാടുകളുടെയും രോമങ്ങളില്‍നിന്ന് നിശ്ചിതകാലംവരെ ഉപയോഗിക്കാവുന്ന വീട്ടുപകരണങ്ങള്‍ അവനുണ്ടാക്കിത്തന്നു. ഉപകാരപ്രദമായ മറ്റു വസ്തുക്കളും.
\end{malayalam}}
\flushright{\begin{Arabic}
\quranayah[16][81]
\end{Arabic}}
\flushleft{\begin{malayalam}
അല്ലാഹു താന്‍ സൃഷ്ടിച്ച നിരവധി വസ്തുക്കളാല്‍ നിങ്ങള്‍ക്ക് തണലുണ്ടാക്കി. പര്‍വതങ്ങളില്‍ അവന്‍ നിങ്ങള്‍ക്ക് അഭയസ്ഥാനങ്ങളുമുണ്ടാക്കി. നിങ്ങളെ ചൂടില്‍ നിന്ന് കാത്തുരക്ഷിക്കുന്ന വസ്ത്രങ്ങള്‍ നല്‍കി. യുദ്ധവേളയില്‍ സംരക്ഷണമേകുന്ന കവചങ്ങളും പ്രദാനം ചെയ്തു. ഇവ്വിധം അല്ലാഹു തന്റെ അനുഗ്രഹം നിങ്ങള്‍ക്ക് പൂര്‍ത്തീകരിച്ചുതരുന്നു; നിങ്ങള്‍ അനുസരണമുള്ളവരാകാന്‍.
\end{malayalam}}
\flushright{\begin{Arabic}
\quranayah[16][82]
\end{Arabic}}
\flushleft{\begin{malayalam}
എന്നിട്ടും അവര്‍ പിന്മാറുകയാണെങ്കില്‍ ഓര്‍ക്കുക: സത്യസന്ദേശം വ്യക്തമായി എത്തിച്ചുകൊടുക്കുന്നതല്ലാത്ത ഒരുത്തരവാദിത്വവും നിനക്കില്ല.
\end{malayalam}}
\flushright{\begin{Arabic}
\quranayah[16][83]
\end{Arabic}}
\flushleft{\begin{malayalam}
അല്ലാഹുവിന്റെ അളവറ്റ അനുഗ്രഹങ്ങള്‍ അവരറിയുന്നുണ്ട്. എന്നിട്ടും അവരതിനെ തള്ളിപ്പറയുകയാണ്. അവരിലേറെപ്പേരും നന്ദികെട്ടവരാണ്.
\end{malayalam}}
\flushright{\begin{Arabic}
\quranayah[16][84]
\end{Arabic}}
\flushleft{\begin{malayalam}
എല്ലാ ഓരോ സമുദായത്തില്‍നിന്നും ഓരോ സാക്ഷിയെ നാം ഉയിര്‍ത്തെഴുന്നേല്‍പിക്കുന്ന ദിവസം. അന്നു പിന്നെ ഒഴികഴിവു പറയാന്‍ സത്യനിഷേധികള്‍ക്ക് ഒരവസരവും നല്‍കുകയില്ല. അവരില്‍നിന്ന് പശ്ചാത്താപം ആവശ്യപ്പെടുകയുമില്ല.
\end{malayalam}}
\flushright{\begin{Arabic}
\quranayah[16][85]
\end{Arabic}}
\flushleft{\begin{malayalam}
അക്രമം പ്രവര്‍ത്തിച്ചവര്‍ ശിക്ഷ നേരില്‍ കണ്ടാല്‍ പിന്നീട് അവര്‍ക്ക് അതിലൊരിളവും നല്‍കുകയില്ല. അവര്‍ക്കൊട്ടും അവധി ലഭിക്കുകയുമില്ല.
\end{malayalam}}
\flushright{\begin{Arabic}
\quranayah[16][86]
\end{Arabic}}
\flushleft{\begin{malayalam}
ബഹുദൈവ വിശ്വാസികള്‍ തങ്ങള്‍ അല്ലാഹുവില്‍ പങ്കാളികളാക്കിയിരുന്നവരെ കാണുമ്പോള്‍ പറയും: "ഞങ്ങളുടെ നാഥാ! നിന്നെക്കൂടാതെ ഞങ്ങള്‍ വിളിച്ചു പ്രാര്‍ഥിക്കാറുണ്ടായിരുന്ന ഞങ്ങളുടെ പങ്കാളികളാണിവര്‍.” അപ്പോള്‍ ആ പങ്കാളികള്‍ അവരോടിങ്ങനെ പറയും: "നിങ്ങള്‍ കള്ളം പറയുന്നവരാണ്.”
\end{malayalam}}
\flushright{\begin{Arabic}
\quranayah[16][87]
\end{Arabic}}
\flushleft{\begin{malayalam}
അന്ന് അവരെല്ലാം അല്ലാഹുവിന് കീഴൊതുങ്ങും. അവര്‍ കെട്ടിച്ചമച്ചിരുന്നവയെല്ലാം അവരില്‍നിന്ന് അകന്നുപോകും.
\end{malayalam}}
\flushright{\begin{Arabic}
\quranayah[16][88]
\end{Arabic}}
\flushleft{\begin{malayalam}
സത്യത്തെ നിഷേധിച്ചുതള്ളുകയും അല്ലാഹുവിന്റെ മാര്‍ഗത്തില്‍നിന്ന് ജനങ്ങളെ തടയുകയും ചെയ്തവര്‍ക്ക് നാം ശിക്ഷക്കു മേല്‍ ശിക്ഷ കൂട്ടിക്കൊടുക്കും. അവര്‍ നാശം വരുത്തിക്കൊണ്ടിരുന്നതിനാലാണിത്.
\end{malayalam}}
\flushright{\begin{Arabic}
\quranayah[16][89]
\end{Arabic}}
\flushleft{\begin{malayalam}
ഓരോ സമുദായത്തിലും അവര്‍ക്കെതിരായി നിലകൊള്ളുന്ന സാക്ഷിയെ അവരില്‍ നിന്നു തന്നെ നാം നിയോഗിക്കുന്ന ദിവസമാണത്. ഇക്കൂട്ടര്‍ക്കെതിരെ സാക്ഷിയായി നിന്നെ നാം കൊണ്ടുവരുന്നതുമാണ്. നിനക്ക് നാം ഈ വേദപുസ്തകം ഇറക്കിത്തന്നിരിക്കുന്നു. ഇതില്‍ സകല സംഗതികള്‍ക്കുമുള്ള വിശദീകരണമുണ്ട്. വഴിപ്പെട്ടു ജീവിക്കുന്നവര്‍ക്ക് വഴികാട്ടിയും അനുഗ്രഹവും ശുഭവൃത്താന്തവുമാണിത്.
\end{malayalam}}
\flushright{\begin{Arabic}
\quranayah[16][90]
\end{Arabic}}
\flushleft{\begin{malayalam}
നീതിപാലിക്കണമെന്നും നന്മ ചെയ്യണമെന്നും കുടുംബ ബന്ധമുള്ളവര്‍ക്ക് സഹായം നല്‍കണമെന്നും അല്ലാഹു കല്‍പിക്കുന്നു. നീചവും നിഷിദ്ധവും അതിക്രമവും വിലക്കുകയും ചെയ്യുന്നു. അവന്‍ നിങ്ങളെ ഉപദേശിക്കുകയാണ്. നിങ്ങള്‍ കാര്യം മനസ്സിലാക്കാന്‍.
\end{malayalam}}
\flushright{\begin{Arabic}
\quranayah[16][91]
\end{Arabic}}
\flushleft{\begin{malayalam}
നിങ്ങള്‍ അല്ലാഹുവോട് പ്രതിജ്ഞ ചെയ്താല്‍ പൂര്‍ണമായും പാലിക്കുക. അല്ലാഹുവെ സാക്ഷിയാക്കി നിങ്ങള്‍ ചെയ്തുറപ്പിക്കുന്ന സത്യങ്ങളൊന്നും ലംഘിക്കരുത്. നിങ്ങള്‍ ചെയ്യുന്നതൊക്കെയും അല്ലാഹു അറിയുന്നുണ്ട്.
\end{malayalam}}
\flushright{\begin{Arabic}
\quranayah[16][92]
\end{Arabic}}
\flushleft{\begin{malayalam}
ഭദ്രതയോടെ നൂല്‍ നൂറ്റ ശേഷം അത് പല തുണ്ടുകളാക്കി പൊട്ടിച്ചുകളഞ്ഞവളെപ്പോലെ നിങ്ങളാകരുത്. ഒരു ജനവിഭാഗം മറ്റൊരു ജനവിഭാഗത്തേക്കാള്‍ കൂടുതല്‍ നേടാനായി നിങ്ങള്‍ നിങ്ങളുടെ ശപഥങ്ങളെ പരസ്പരം വഞ്ചനോപാധിയാക്കരുത്. അതിലൂടെ അല്ലാഹു നിങ്ങളെ പരീക്ഷിക്കുക മാത്രമാണ് ചെയ്യുന്നത്. ഉയിര്‍ത്തെഴുന്നേല്‍പു നാളില്‍ നിങ്ങള്‍ ഭിന്നിച്ചിരുന്ന കാര്യങ്ങളുടെ നിജസ്ഥിതി നിങ്ങള്‍ക്കവന്‍ വ്യക്തമാക്കിത്തരികതന്നെ ചെയ്യും.
\end{malayalam}}
\flushright{\begin{Arabic}
\quranayah[16][93]
\end{Arabic}}
\flushleft{\begin{malayalam}
അല്ലാഹു ഇച്ഛിച്ചിരുന്നെങ്കില്‍ നിങ്ങളെ അവന്‍ ഒരൊറ്റ സമുദായമാക്കുമായിരുന്നു. എന്നാല്‍ അവനിച്ഛിക്കുന്നവരെ അവന്‍ വഴികേടിലാക്കുന്നു. അവനിച്ഛിക്കുന്നവരെ നേര്‍വഴിയിലാക്കുകയും ചെയ്യുന്നു. നിങ്ങള്‍ പ്രവര്‍ത്തിച്ചുകൊണ്ടിരിക്കുന്നതിനെപ്പറ്റി തീര്‍ച്ചയായും നിങ്ങള്‍ ചോദിക്കപ്പെടും.
\end{malayalam}}
\flushright{\begin{Arabic}
\quranayah[16][94]
\end{Arabic}}
\flushleft{\begin{malayalam}
നിങ്ങള്‍ നിങ്ങളുടെ ശപഥങ്ങളെ പരസ്പരം വഞ്ചനോപാധിയാക്കരുത്. അങ്ങനെ ചെയ്താല്‍ സത്യത്തില്‍ നിലയുറപ്പിച്ച ശേഷം കാലിടറാനും അല്ലാഹുവിന്റെ മാര്‍ഗത്തില്‍ നിന്ന് ജനങ്ങളെ തടഞ്ഞതിന്റെ പേരില്‍ ദുരിതമനുഭവിക്കാനും അതിടവരുത്തും. പിന്നെ നിങ്ങള്‍ക്ക് അതിഭയങ്കരമായ ശിക്ഷയുണ്ടാകും.
\end{malayalam}}
\flushright{\begin{Arabic}
\quranayah[16][95]
\end{Arabic}}
\flushleft{\begin{malayalam}
അല്ലാഹുവുമായുള്ള പ്രതിജ്ഞകള്‍ നിങ്ങള്‍ നിസ്സാരവിലയ്ക്ക് വില്‍ക്കരുത്. സംശയംവേണ്ട; അല്ലാഹുവിന്റെ അടുത്തുള്ളതു തന്നെയാണ് നിങ്ങള്‍ക്കുത്തമം. നിങ്ങള്‍ കാര്യം മനസ്സിലാക്കുന്നവരെങ്കില്‍!
\end{malayalam}}
\flushright{\begin{Arabic}
\quranayah[16][96]
\end{Arabic}}
\flushleft{\begin{malayalam}
നിങ്ങളുടെ വശമുള്ളത് തീര്‍ന്നുപോകും. ബാക്കിയാവുന്നത് അല്ലാഹുവിന്റെ വശമുള്ളത് മാത്രം. തീര്‍ച്ചയായും ക്ഷമപാലിക്കുന്നവര്‍ക്ക് നാം അവരുടെ നന്മനിറഞ്ഞ കര്‍മങ്ങള്‍ക്ക് അര്‍ഹമായ പ്രതിഫലം നല്‍കും.
\end{malayalam}}
\flushright{\begin{Arabic}
\quranayah[16][97]
\end{Arabic}}
\flushleft{\begin{malayalam}
പുരുഷനോ സ്ത്രീയോ ആരാവട്ടെ. സത്യവിശ്വാസിയായിരിക്കെ സല്‍ക്കര്‍മം പ്രവര്‍ത്തിക്കുന്നവര്‍ക്ക് നിശ്ചയമായും നാം മെച്ചപ്പെട്ട ജീവിതം നല്‍കും. അവര്‍ പ്രവര്‍ത്തിച്ചുകൊണ്ടിരുന്നതില്‍ ഏറ്റം ഉത്തമമായതിന് അനുസൃതമായ പ്രതിഫലവും നാമവര്‍ക്ക് കൊടുക്കും.
\end{malayalam}}
\flushright{\begin{Arabic}
\quranayah[16][98]
\end{Arabic}}
\flushleft{\begin{malayalam}
നീ ഖുര്‍ആന്‍ പാരായണം ചെയ്യുമ്പോള്‍ ശപിക്കപ്പെട്ട പിശാചില്‍ നിന്ന് അല്ലാഹുവോട് ശരണം തേടുക.
\end{malayalam}}
\flushright{\begin{Arabic}
\quranayah[16][99]
\end{Arabic}}
\flushleft{\begin{malayalam}
സത്യവിശ്വാസം സ്വീകരിക്കുകയും തങ്ങളുടെ നാഥനില്‍ ഭരമേല്‍പിക്കുകയും ചെയ്യുന്നവരുടെ മേല്‍ പിശാചിന് ഒട്ടും സ്വാധീനമില്ല.
\end{malayalam}}
\flushright{\begin{Arabic}
\quranayah[16][100]
\end{Arabic}}
\flushleft{\begin{malayalam}
പിശാചിനെ രക്ഷാധികാരിയാക്കുകയും അവന്റെ പ്രേരണമൂലം അല്ലാഹുവില്‍ പങ്കുചേര്‍ക്കുകയും ചെയ്യുന്നവരുടെ മേല്‍ മാത്രമാണ് അവന് സ്വാധീനമുള്ളത്.
\end{malayalam}}
\flushright{\begin{Arabic}
\quranayah[16][101]
\end{Arabic}}
\flushleft{\begin{malayalam}
ഒരു വചനത്തിനു പകരമായി മറ്റൊരു വചനം നാം അവതരിപ്പിക്കുമ്പോള്‍ - താന്‍ എന്താണ് അവതരിപ്പിക്കുന്നതെന്ന് അല്ലാഹുവിന് നന്നായറിയാം - അവര്‍ പറയും: "നീ ഇത് കൃത്രിമമായി കെട്ടിയുണ്ടാക്കുന്നവന്‍ മാത്രമാണ്.” എന്നാല്‍ യാഥാര്‍ഥ്യം അതല്ല; അവരിലേറെപ്പേരും കാര്യമറിയുന്നില്ല.
\end{malayalam}}
\flushright{\begin{Arabic}
\quranayah[16][102]
\end{Arabic}}
\flushleft{\begin{malayalam}
പറയുക: നിന്റെ നാഥങ്കല്‍ നിന്ന് പരിശുദ്ധാത്മാവ് വളരെ കണിശതയോടെ ഇറക്കിത്തന്നതാണിത്. അത് സത്യവിശ്വാസം സ്വീകരിച്ചവരെ അതിലുറപ്പിച്ചുനിര്‍ത്തുന്നു. വഴിപ്പെട്ടു ജീവിക്കുന്നവര്‍ക്കത് വഴികാട്ടിയാണ്. ശുഭവാര്‍ത്തയും.
\end{malayalam}}
\flushright{\begin{Arabic}
\quranayah[16][103]
\end{Arabic}}
\flushleft{\begin{malayalam}
ഇദ്ദേഹത്തിനിത് പഠിപ്പിച്ചുകൊടുക്കുന്നത് വെറുമൊരു മനുഷ്യന്‍ മാത്രമാണെന്ന് ഇക്കൂട്ടര്‍ പ്രചരിപ്പിക്കുന്നതായി നിശ്ചയമായും നമുക്കറിയാം. എന്നാല്‍ ഇവര്‍ ദുസ്സൂചന നല്‍കിക്കൊണ്ടിരിക്കുന്നയാളുടെ ഭാഷ അറബിയല്ല. ഇതോ, തെളിഞ്ഞ അറബി ഭാഷയിലും.
\end{malayalam}}
\flushright{\begin{Arabic}
\quranayah[16][104]
\end{Arabic}}
\flushleft{\begin{malayalam}
അല്ലാഹുവിന്റെ വചനങ്ങളില്‍ വിശ്വസിക്കാത്തവരെ അല്ലാഹു നേര്‍വഴിയിലാക്കുകയില്ല; തീര്‍ച്ച. അവര്‍ക്ക് നോവേറിയ ശിക്ഷയുണ്ട്.
\end{malayalam}}
\flushright{\begin{Arabic}
\quranayah[16][105]
\end{Arabic}}
\flushleft{\begin{malayalam}
അല്ലാഹുവിന്റെ വചനങ്ങളില്‍ വിശ്വസിക്കാത്തവര്‍ തന്നെയാണ് കള്ളം കെട്ടിച്ചമക്കുന്നത്. നുണ പറയുന്നവരും അവര്‍ തന്നെ.
\end{malayalam}}
\flushright{\begin{Arabic}
\quranayah[16][106]
\end{Arabic}}
\flushleft{\begin{malayalam}
അല്ലാഹുവില്‍ വിശ്വസിച്ചശേഷം അവിശ്വസിച്ചവന്‍, തുറന്ന മനസ്സോടെ സത്യനിഷേധം അംഗീകരിച്ചവരാണെങ്കില്‍ അവരുടെ മേല്‍ ദൈവകോപമുണ്ട്. കടുത്ത ശിക്ഷയും. എന്നാല്‍ തങ്ങളുടെ മനസ്സ് സത്യവിശ്വാസത്തില്‍ ശാന്തി നേടിയതായിരിക്കെ നിര്‍ബന്ധിതരായി അങ്ങനെ ചെയ്യുന്നവര്‍ക്കിതു ബാധകമല്ല.
\end{malayalam}}
\flushright{\begin{Arabic}
\quranayah[16][107]
\end{Arabic}}
\flushleft{\begin{malayalam}
അവര്‍ ഐഹികജീവിതത്തെ പരലോകത്തെക്കാള്‍ ഇഷ്ടപ്പെടുന്നതിനാലാണിത്. സത്യനിഷേധികളായ ജനത്തെ അല്ലാഹു നേര്‍വഴിയിലാക്കുകയില്ല.
\end{malayalam}}
\flushright{\begin{Arabic}
\quranayah[16][108]
\end{Arabic}}
\flushleft{\begin{malayalam}
അല്ലാഹു ഹൃദയങ്ങളും കാതുകളും കണ്ണുകളും കൊട്ടിയടച്ചു മുദ്രവെച്ചവരാണവര്‍. തികഞ്ഞ അശ്രദ്ധയില്‍ കഴിയുന്നവരും.
\end{malayalam}}
\flushright{\begin{Arabic}
\quranayah[16][109]
\end{Arabic}}
\flushleft{\begin{malayalam}
സംശയം വേണ്ട; പരലോകത്ത് നഷ്ടം പറ്റിയവരും അവര്‍ തന്നെ.
\end{malayalam}}
\flushright{\begin{Arabic}
\quranayah[16][110]
\end{Arabic}}
\flushleft{\begin{malayalam}
നേരെമറിച്ച് അങ്ങേയറ്റം പീഡിതരായശേഷം സ്വദേശം വെടിഞ്ഞ് പലായനം നടത്തുകയും പിന്നീട് സമരത്തിലേര്‍പ്പെടുകയും ക്ഷമപാലിക്കുകയും ചെയ്തവരെ സംബന്ധിച്ചേടത്തോളം നിന്റെ നാഥന്‍ ഏറെ പൊറുക്കുന്നവനും പരമദയാലുവും തന്നെ; തീര്‍ച്ച.
\end{malayalam}}
\flushright{\begin{Arabic}
\quranayah[16][111]
\end{Arabic}}
\flushleft{\begin{malayalam}
ഒരുദിനം അതുണ്ടാകും. അന്ന് എല്ലാ മനുഷ്യരും സ്വന്തം കാര്യത്തിനുവേണ്ടി വാദിച്ചുകൊണ്ടിരിക്കും. എല്ലാ ഓരോരുത്തര്‍ക്കും തങ്ങളുടെ കര്‍മഫലം പൂര്‍ണമായി നല്‍കപ്പെടും. ആരും ഒരുവിധ അനീതിക്കുമിരയാവുകയുമില്ല.
\end{malayalam}}
\flushright{\begin{Arabic}
\quranayah[16][112]
\end{Arabic}}
\flushleft{\begin{malayalam}
അല്ലാഹു ഒരു നാടിന്റെ ഉദാഹരണം എടുത്തുകാണിക്കുന്നു. അത് നിര്‍ഭയവും ശാന്തവുമായിരുന്നു. അവിടേക്കാവശ്യമായ ആഹാരം നാനാഭാഗത്തുനിന്നും സമൃദ്ധമായി വന്നുകൊണ്ടിരുന്നു. എന്നിട്ടും ആ നാട് അല്ലാഹുവിന്റെ അനുഗ്രഹങ്ങളോട് നന്ദികേടു കാണിച്ചു. അപ്പോള്‍ അല്ലാഹു അതിനെ വിശപ്പിന്റെയും ഭയത്തിന്റെയും ആവരണമണിയിച്ചു. അവര്‍ പ്രവര്‍ത്തിച്ചുകൊണ്ടിരുന്നതിന്റെ ഫലമായി.
\end{malayalam}}
\flushright{\begin{Arabic}
\quranayah[16][113]
\end{Arabic}}
\flushleft{\begin{malayalam}
അവരില്‍നിന്നുതന്നെയുള്ള ഒരു ദൈവദൂതന്‍ അവരുടെ അടുത്തു ചെന്നു. അപ്പോള്‍ അവരദ്ദേഹത്തെ കളവാക്കി. അവരങ്ങനെ അക്രമികളായി. ശിക്ഷ അവരെ പിടികൂടുകയും ചെയ്തു.
\end{malayalam}}
\flushright{\begin{Arabic}
\quranayah[16][114]
\end{Arabic}}
\flushleft{\begin{malayalam}
അതിനാല്‍ അല്ലാഹു നിങ്ങള്‍ക്കു നല്‍കിയ വിഭവങ്ങളില്‍ അനുവദനീയവും ഉത്തമവുമായത് തിന്നുകൊള്ളുക. അല്ലാഹുവിന്റെ അനുഗ്രഹത്തിന് നന്ദി കാണിക്കുക. നിങ്ങള്‍ അവനുമാത്രം വഴിപ്പെടുന്നവരെങ്കില്‍!
\end{malayalam}}
\flushright{\begin{Arabic}
\quranayah[16][115]
\end{Arabic}}
\flushleft{\begin{malayalam}
ശവം, രക്തം, പന്നിമാംസം, അല്ലാഹു അല്ലാത്തവരുടെ പേരില്‍ അറുക്കപ്പെട്ടത് ഇവ മാത്രമാണ് അല്ലാഹു നിങ്ങള്‍ക്ക് നിഷിദ്ധമാക്കിയത്. അഥവാ, ആരെങ്കിലും നിര്‍ബന്ധിതനായാല്‍, അവന്‍ അതാഗ്രഹിക്കുന്നവനോ അത്യാവശ്യത്തിലേറെ തിന്നുന്നവനോ അല്ലെങ്കില്‍, അല്ലാഹു ഏറെ പൊറുക്കുന്നവനും പരമകാരുണികനുമാകുന്നു.
\end{malayalam}}
\flushright{\begin{Arabic}
\quranayah[16][116]
\end{Arabic}}
\flushleft{\begin{malayalam}
നിങ്ങളുടെ നാവുകള്‍ വിശേഷിപ്പിക്കുന്നതിന്റെ അടിസ്ഥാനത്തില്‍, “ഇത് അനുവദനീയം, ഇത് നിഷിദ്ധം” എന്നിങ്ങനെ കള്ളം പറയാതിരിക്കുക. നിങ്ങള്‍ അല്ലാഹുവിന്റെ പേരില്‍ കള്ളം കെട്ടിച്ചമക്കലാകുമത്. അല്ലാഹുവിന്റെ പേരില്‍ കള്ളം കെട്ടിച്ചമക്കുന്നവര്‍ ഒരിക്കലും വിജയിക്കുകയില്ല; തീര്‍ച്ച.
\end{malayalam}}
\flushright{\begin{Arabic}
\quranayah[16][117]
\end{Arabic}}
\flushleft{\begin{malayalam}
വളരെ തുച്ഛമായ സുഖാനുഭവമാണ് അവര്‍ക്കുണ്ടാവുക. പിന്നെ അവര്‍ക്ക് നോവേറിയ ശിക്ഷയുണ്ട്.
\end{malayalam}}
\flushright{\begin{Arabic}
\quranayah[16][118]
\end{Arabic}}
\flushleft{\begin{malayalam}
നിനക്ക് നേരത്തെ നാം വിവരിച്ചുതന്നവ ജൂതന്മാര്‍ക്കും നാം നിഷിദ്ധമാക്കുകയുണ്ടായി. നാം അവരോടൊട്ടും അനീതി ചെയ്തിട്ടില്ല. അവര്‍ തങ്ങളോടു തന്നെ അനീതി ചെയ്യുകയായിരുന്നു.
\end{malayalam}}
\flushright{\begin{Arabic}
\quranayah[16][119]
\end{Arabic}}
\flushleft{\begin{malayalam}
എന്നാല്‍, അറിവില്ലായ്മ കാരണം അബദ്ധം പ്രവര്‍ത്തിക്കുകയും പിന്നീട് പശ്ചാത്തപിക്കുകയും ജീവിതം നന്നാക്കിത്തീര്‍ക്കുകയും ചെയ്തവരോട്, അതിനു ശേഷവും നിന്റെ നാഥന്‍ ഏറെ പൊറുക്കുന്നവനും ദയാപരനും തന്നെ; തീര്‍ച്ച.
\end{malayalam}}
\flushright{\begin{Arabic}
\quranayah[16][120]
\end{Arabic}}
\flushleft{\begin{malayalam}
ഇബ്റാഹീം സ്വയം ഒരു സമുദായമായിരുന്നു. അദ്ദേഹം അല്ലാഹുവിന് വഴങ്ങി ജീവിക്കുന്നവനായിരുന്നു. ചൊവ്വായ പാതയില്‍ ഉറച്ചുനില്‍ക്കുന്നവനും. അദ്ദേഹം ബഹുദൈവവിശ്വാസികളില്‍ പെട്ടവനായിരുന്നില്ല.
\end{malayalam}}
\flushright{\begin{Arabic}
\quranayah[16][121]
\end{Arabic}}
\flushleft{\begin{malayalam}
അദ്ദേഹം അല്ലാഹുവിന്റെ അനുഗ്രഹങ്ങള്‍ക്ക് നന്ദി കാണിക്കുന്നവനായിരുന്നു. അല്ലാഹു അദ്ദേഹത്തെ തെരഞ്ഞെടുക്കുകയും ഏറ്റം നേരായ വഴിയില്‍ നയിക്കുകയും ചെയ്തു.
\end{malayalam}}
\flushright{\begin{Arabic}
\quranayah[16][122]
\end{Arabic}}
\flushleft{\begin{malayalam}
ഇഹലോകത്ത് അദ്ദേഹത്തിനു നാം നന്മ നല്‍കി. പരലോകത്തോ, ഉറപ്പായും അദ്ദേഹം സച്ചരിതരിലായിരിക്കും.
\end{malayalam}}
\flushright{\begin{Arabic}
\quranayah[16][123]
\end{Arabic}}
\flushleft{\begin{malayalam}
പിന്നീട് നിനക്കു നാം ബോധനം നല്‍കി, ഏറ്റം ചൊവ്വായപാതയില്‍ നിലയുറപ്പിച്ച ഇബ്റാഹീമിന്റെ മാര്‍ഗം പിന്തുടരണമെന്ന്. അദ്ദേഹം ബഹുദൈവവിശ്വാസികളില്‍ പെട്ടവനായിരുന്നില്ല.
\end{malayalam}}
\flushright{\begin{Arabic}
\quranayah[16][124]
\end{Arabic}}
\flushleft{\begin{malayalam}
ശാബത്ത് ദിനാചരണം അക്കാര്യത്തില്‍ ഭിന്നിച്ചവരുടെ മേല്‍ മാത്രമാണ് നാം നടപ്പാക്കിയത്. നിന്റെ നാഥന്‍ അവര്‍ക്കിടയില്‍ ഭിന്നതയുള്ള കാര്യങ്ങളിലൊക്കെയും ഉയിര്‍ത്തെഴുന്നേല്‍പുനാളില്‍ തീര്‍പ്പ് കല്‍പിക്കും; തീര്‍ച്ച.
\end{malayalam}}
\flushright{\begin{Arabic}
\quranayah[16][125]
\end{Arabic}}
\flushleft{\begin{malayalam}
യുക്തികൊണ്ടും സദുപദേശം കൊണ്ടും നീ ജനത്തെ നിന്റെ നാഥന്റെ മാര്‍ഗത്തിലേക്ക് ക്ഷണിക്കുക. ഏറ്റം നല്ല നിലയില്‍ അവരുമായി സംവാദം നടത്തുക. നിശ്ചയമായും നിന്റെ നാഥന്‍ തന്റെ നേര്‍വഴി വിട്ട് പിഴച്ചുപോയവരെ സംബന്ധിച്ച് നന്നായറിയുന്നവനാണ്. നേര്‍വഴി പ്രാപിച്ചവരെപ്പറ്റിയും സൂക്ഷ്മമായി അറിയുന്നവനാണവന്‍.
\end{malayalam}}
\flushright{\begin{Arabic}
\quranayah[16][126]
\end{Arabic}}
\flushleft{\begin{malayalam}
നിങ്ങള്‍ പ്രതികാരം ചെയ്യുന്നുവെങ്കില്‍ ഇങ്ങോട്ട് അക്രമിക്കപ്പെട്ടതിന് തുല്യമായി അങ്ങോട്ടും ശിക്ഷാനടപടികള്‍ സ്വീകരിക്കുക. എന്നാല്‍ നിങ്ങള്‍ ക്ഷമിക്കുകയാണെങ്കില്‍ അറിയുക: അതു തന്നെയാണ് ക്ഷമാശീലര്‍ക്ക് കൂടുതലുത്തമം.
\end{malayalam}}
\flushright{\begin{Arabic}
\quranayah[16][127]
\end{Arabic}}
\flushleft{\begin{malayalam}
നീ ക്ഷമിക്കുക. അല്ലാഹുവിന്റെ മഹത്തായ അനുഗ്രഹം കൊണ്ടു മാത്രമാണ് നിനക്ക് ക്ഷമിക്കാന്‍ കഴിയുന്നത്. അവരെപ്പറ്റി നീ ദുഃഖിക്കരുത്. അവരുടെ കുതന്ത്രങ്ങളെപ്പറ്റി വിഷമിക്കുകയും വേണ്ട.
\end{malayalam}}
\flushright{\begin{Arabic}
\quranayah[16][128]
\end{Arabic}}
\flushleft{\begin{malayalam}
സംശയമില്ല; അല്ലാഹു ഭക്തന്മാരോടൊപ്പമാണ്. സച്ചരിതരായിക്കഴിയുന്നവരോടൊപ്പം.
\end{malayalam}}
\chapter{\textmalayalam{ഇസ്റാഅ് ( നിശായാത്ര )}}
\begin{Arabic}
\Huge{\centerline{\basmalah}}\end{Arabic}
\flushright{\begin{Arabic}
\quranayah[17][1]
\end{Arabic}}
\flushleft{\begin{malayalam}
തന്റെ ദാസനെ മസ്ജിദുല്‍ ഹറാമില്‍നിന്ന് മസ്ജിദുല്‍ അഖ്സായിലേക്ക്-അതിന്റെ പരിസരം നാം അനുഗൃഹീതമാക്കിയിരിക്കുന്നു-ഒരു രാവില്‍ കൊണ്ടുപോയവന്‍ ഏറെ പരിശുദ്ധന്‍ തന്നെ. നമ്മുടെ ചില ദൃഷ്ടാന്തങ്ങള്‍ അദ്ദേഹത്തിന് കാണിച്ചുകൊടുക്കാന്‍ വേണ്ടിയാണത്. അവന്‍ എല്ലാം കേള്‍ക്കുന്നവനും കാണുന്നവനുമാണ്.
\end{malayalam}}
\flushright{\begin{Arabic}
\quranayah[17][2]
\end{Arabic}}
\flushleft{\begin{malayalam}
മൂസാക്കു നാം വേദപുസ്തകം നല്‍കി. അതിനെ ഇസ്രയേല്‍ മക്കള്‍ക്ക് വഴികാട്ടിയാക്കി. എന്നെക്കൂടാതെ ആരെയും കൈകാര്യകര്‍ത്താവാക്കരുതെന്ന ശാസന അതിലുണ്ട്.
\end{malayalam}}
\flushright{\begin{Arabic}
\quranayah[17][3]
\end{Arabic}}
\flushleft{\begin{malayalam}
നാം നൂഹിനോടൊപ്പം കപ്പലില്‍ കയറ്റിയവരുടെ സന്തതികളാണ് നിങ്ങള്‍. നൂഹ് വളരെ നന്ദിയുള്ള ദാസനായിരുന്നു.
\end{malayalam}}
\flushright{\begin{Arabic}
\quranayah[17][4]
\end{Arabic}}
\flushleft{\begin{malayalam}
ഇസ്രയേല്‍ മക്കള്‍ രണ്ടു തവണ ഭൂമിയില്‍ കുഴപ്പമുണ്ടാക്കുമെന്നും ധിക്കാരം കാണിക്കുമെന്നും നാം മൂലപ്രമാണത്തില്‍ രേഖപ്പെടുത്തിയിട്ടുണ്ട്.
\end{malayalam}}
\flushright{\begin{Arabic}
\quranayah[17][5]
\end{Arabic}}
\flushleft{\begin{malayalam}
അങ്ങനെ ആ രണ്ട് സന്ദര്‍ഭങ്ങളില്‍ ആദ്യത്തേതിന്റെ അവസരമെത്തിയപ്പോള്‍ നാം നിങ്ങള്‍ക്കെതിരെ നമ്മുടെ ദാസന്മാരിലെ അതിശക്തരായ ആക്രമണകാരികളെ അയച്ചു. അവര്‍ നിങ്ങളുടെ വീടുകള്‍ക്കിടയില്‍പോലും നിങ്ങളെ പരതിനടന്നു. അനിവാര്യമായി സംഭവിക്കേണ്ടിയിരുന്ന ഒരു വാഗ്ദാനം തന്നെയായിരുന്നു അത്.
\end{malayalam}}
\flushright{\begin{Arabic}
\quranayah[17][6]
\end{Arabic}}
\flushleft{\begin{malayalam}
പിന്നീട് നിങ്ങള്‍ക്കു നാം അവരുടെമേല്‍ വീണ്ടും വിജയം നല്‍കി. സമ്പത്തും സന്താനങ്ങളും നല്‍കി സഹായിച്ചു. നിങ്ങളെ കൂടുതല്‍ അംഗബലമുള്ളവരാക്കുകയും ചെയ്തു.
\end{malayalam}}
\flushright{\begin{Arabic}
\quranayah[17][7]
\end{Arabic}}
\flushleft{\begin{malayalam}
നിങ്ങള്‍ നന്മ പ്രവര്‍ത്തിച്ചാല്‍ അതിന്റെ ഗുണം നിങ്ങള്‍ക്കുതന്നെയാണ്. തിന്മ ചെയ്താല്‍ അതിന്റെ ദോഷവും നിങ്ങള്‍ക്കുതന്നെ. നിങ്ങളെ അറിയിച്ച രണ്ടു സന്ദര്‍ഭങ്ങളില്‍ അവസാനത്തേതിന്റെ സമയമായപ്പോള്‍ നിങ്ങളെ മറ്റു ശത്രുക്കള്‍ കീഴ്പ്പെടുത്തി; അവര്‍ നിങ്ങളുടെ മുഖം ചീത്തയാക്കാനും ആദ്യതവണ പള്ളിയില്‍ കടന്നുവന്നപോലെ ഇത്തവണയും കടന്നുചെല്ലാനും കയ്യില്‍ ക്കിട്ടിയതെല്ലാം തകര്‍ത്തുകളയാനും വേണ്ടി.
\end{malayalam}}
\flushright{\begin{Arabic}
\quranayah[17][8]
\end{Arabic}}
\flushleft{\begin{malayalam}
ഇനിയും നിങ്ങളുടെ നാഥന്‍ നിങ്ങളോടു കരുണ കാണിച്ചേക്കാം. അഥവാ നിങ്ങള്‍ പഴയ നിലപാട് ആവര്‍ത്തിച്ചാല്‍ നാം നമ്മുടെ ശിക്ഷയും ആവര്‍ത്തിക്കും. സംശയമില്ല; നരകത്തെ നാം സത്യനിഷേധികള്‍ക്കുള്ള തടവറയാക്കിയിരിക്കുന്നു.
\end{malayalam}}
\flushright{\begin{Arabic}
\quranayah[17][9]
\end{Arabic}}
\flushleft{\begin{malayalam}
ഈ ഖുര്‍ആന്‍ ഏറ്റവും നേരായ വഴി കാണിച്ചുതരുന്നു. സല്‍ക്കര്‍മങ്ങള്‍ പ്രവര്‍ത്തിക്കുന്ന സത്യവിശ്വാസികള്‍ക്ക് അതിമഹത്തായ പ്രതിഫലമുണ്ടെന്ന് ശുഭവാര്‍ത്ത അറിയിക്കുന്നു.
\end{malayalam}}
\flushright{\begin{Arabic}
\quranayah[17][10]
\end{Arabic}}
\flushleft{\begin{malayalam}
പരലോകത്തില്‍ വിശ്വസിക്കാത്തവര്‍ക്ക് നാം നോവേറിയ ശിക്ഷ ഒരുക്കിവെച്ചിട്ടുണ്ടെന്ന് മുന്നറിയിപ്പ് നല്‍കുകയും ചെയ്യുന്നു.
\end{malayalam}}
\flushright{\begin{Arabic}
\quranayah[17][11]
\end{Arabic}}
\flushleft{\begin{malayalam}
മനുഷ്യന്‍ നന്മക്കുവേണ്ടിയെന്നപോലെ തിന്മക്കുവേണ്ടിയും പ്രാര്‍ഥിക്കുന്നു. അവന്‍ വല്ലാത്ത ധൃതിക്കാരന്‍ തന്നെ.
\end{malayalam}}
\flushright{\begin{Arabic}
\quranayah[17][12]
\end{Arabic}}
\flushleft{\begin{malayalam}
നാം രാവിനെയും പകലിനെയും രണ്ട് അടയാളങ്ങളാക്കിയിരിക്കുന്നു. അങ്ങനെ നാം രാവാകുന്ന ദൃഷ്ടാന്തത്തിന്റെ നിറംകെടുത്തി. പകലാകുന്ന ദൃഷ്ടാന്തത്തെ പ്രകാശപൂരിതമാക്കി. നിങ്ങള്‍ നിങ്ങളുടെ നാഥനില്‍ നിന്നുള്ള അനുഗ്രഹം തേടാനാണിത്. നിങ്ങള്‍ കൊല്ലങ്ങളുടെ എണ്ണവും കണക്കും മനസ്സിലാക്കാനും. അങ്ങനെ സകല സംഗതികളും നാം വ്യക്തമായി വേര്‍തിരിച്ചുവെച്ചിരിക്കുന്നു.
\end{malayalam}}
\flushright{\begin{Arabic}
\quranayah[17][13]
\end{Arabic}}
\flushleft{\begin{malayalam}
ഓരോ മനുഷ്യന്റെയും ഭാഗധേയത്തെ നാം അവന്റെ കഴുത്തില്‍ തന്നെ ബന്ധിച്ചിരിക്കുന്നു. ഉയിര്‍ത്തെഴുന്നേല്‍പുനാളില്‍ നാം അവനുവേണ്ടി ഒരു കര്‍മരേഖ പുറത്തിറക്കും. അത് തുറന്നുവെച്ചതായി അവനു കാണാം.
\end{malayalam}}
\flushright{\begin{Arabic}
\quranayah[17][14]
\end{Arabic}}
\flushleft{\begin{malayalam}
നിന്റെ ഈ കര്‍മപുസ്തകമൊന്നു വായിച്ചുനോക്കൂ. ഇന്നു നിന്റെ കണക്കുനോക്കാന്‍ നീ തന്നെ മതി.
\end{malayalam}}
\flushright{\begin{Arabic}
\quranayah[17][15]
\end{Arabic}}
\flushleft{\begin{malayalam}
ആര്‍ നേര്‍വഴി സ്വീകരിക്കുന്നുവോ, അതിന്റെ ഗുണം അവനുതന്നെയാണ്. ആര്‍ വഴികേടിലാകുന്നുവോ അതിന്റെ ദോഷവും അവനുതന്നെ. ആരും മറ്റൊരുത്തന്റെ ഭാരം ചുമക്കുകയില്ല. ദൂതനെ നിയോഗിക്കും വരെ നാമാരെയും ശിക്ഷിക്കുകയുമില്ല.
\end{malayalam}}
\flushright{\begin{Arabic}
\quranayah[17][16]
\end{Arabic}}
\flushleft{\begin{malayalam}
ഒരു നാടിനെ നശിപ്പിക്കണമെന്ന് നാമുദ്ദേശിച്ചാല്‍ അവിടത്തെ സുഖലോലുപരോട് നാം കല്പിക്കും. അങ്ങനെ അവരവിടെ അധര്‍മം പ്രവര്‍ത്തിക്കും. അതോടെ അവിടം ശിക്ഷാര്‍ഹമായിത്തീരുന്നു. അങ്ങനെ, നാമതിനെ തകര്‍ത്ത് തരിപ്പണമാക്കുന്നു.
\end{malayalam}}
\flushright{\begin{Arabic}
\quranayah[17][17]
\end{Arabic}}
\flushleft{\begin{malayalam}
നൂഹിനുശേഷം എത്രയെത്ര തലമുറകളെയാണ് നാം നശിപ്പിച്ചത്? തന്റെ ദാസന്മാരുടെ പാപങ്ങളെപ്പറ്റി സൂക്ഷ്മമായി അറിയുന്നവനും കാണുന്നവനുമായി നിന്റെ നാഥന്‍ തന്നെ മതി!
\end{malayalam}}
\flushright{\begin{Arabic}
\quranayah[17][18]
\end{Arabic}}
\flushleft{\begin{malayalam}
ആരെങ്കിലും പെട്ടെന്ന് കിട്ടുന്ന നേട്ടങ്ങളാണ് കൊതിക്കുന്നതെങ്കില്‍ നാം അയാള്‍ക്ക് അതുടനെത്തന്നെ നല്‍കുന്നു; നാം ഇച്ഛിക്കുന്നവര്‍ക്ക് നാം ഇച്ഛിക്കുന്ന അളവില്‍. പിന്നെ നാമവന്ന് നല്‍കുക നരകത്തീയാണ്. നിന്ദ്യനും ദിവ്യകാരുണ്യം നിഷേധിക്കപ്പെട്ടവനുമായി അവനവിടെ കത്തിയെരിയും.
\end{malayalam}}
\flushright{\begin{Arabic}
\quranayah[17][19]
\end{Arabic}}
\flushleft{\begin{malayalam}
എന്നാല്‍ ആരെങ്കിലും സത്യവിശ്വാസം സ്വീകരിക്കുകയും പരലോകമാഗ്രഹിക്കുകയും അതിനായി ശ്രമിക്കുകയുമാണെങ്കില്‍ അറിയുക: അത്തരക്കാരുടെ പരിശ്രമം ഏറെ നന്ദിയര്‍ഹിക്കുന്നതുതന്നെ.
\end{malayalam}}
\flushright{\begin{Arabic}
\quranayah[17][20]
\end{Arabic}}
\flushleft{\begin{malayalam}
ഇവര്‍ക്കും അവര്‍ക്കും നാം ഈ ലോകത്ത് സഹായം നല്‍കും. നിന്റെ നാഥന്റെ ദാനമാണത്. നിന്റെ നാഥന്റെ ദാനം തടയാന്‍ ആര്‍ക്കുമാവില്ല.
\end{malayalam}}
\flushright{\begin{Arabic}
\quranayah[17][21]
\end{Arabic}}
\flushleft{\begin{malayalam}
ഇവിടെ നാം ചിലരെ മറ്റു ചിലരേക്കാള്‍ എവ്വിധമാണ് ശ്രേഷ്ഠരാക്കിയതെന്ന് നോക്കൂ. എന്നാല്‍ ഏറ്റം മഹിതമായ പദവിയും ഏറ്റം ഉല്‍കൃഷ്ടമായ അവസ്ഥയുമുള്ളത് പരലോകജീവിതത്തിലാണ്.
\end{malayalam}}
\flushright{\begin{Arabic}
\quranayah[17][22]
\end{Arabic}}
\flushleft{\begin{malayalam}
നീ അല്ലാഹുവോടൊപ്പം മറ്റൊരു ദൈവത്തെയും സ്വീകരിക്കരുത്. അങ്ങനെ ചെയ്താല്‍ നീ നിന്ദ്യനും നിരാകരിക്കപ്പെട്ടവനുമായി കുത്തിയിരിക്കേണ്ടിവരും.
\end{malayalam}}
\flushright{\begin{Arabic}
\quranayah[17][23]
\end{Arabic}}
\flushleft{\begin{malayalam}
നിന്റെ നാഥന്‍ വിധിച്ചിരിക്കുന്നു: നിങ്ങള്‍ അവനെയല്ലാതെ വഴിപ്പെടരുത്. മാതാപിതാക്കള്‍ക്ക് നന്മ ചെയ്യുക. അവരില്‍ ഒരാളോ രണ്ടുപേരുമോ വാര്‍ധക്യം ബാധിച്ച് നിന്നോടൊപ്പമുണ്ടെങ്കില്‍ അവരോട് “ഛെ” എന്നുപോലും പറയരുത്. പരുഷമായി സംസാരിക്കരുത്. ഇരുവരോടും ആദരവോടെ സംസാരിക്കുക.
\end{malayalam}}
\flushright{\begin{Arabic}
\quranayah[17][24]
\end{Arabic}}
\flushleft{\begin{malayalam}
കാരുണ്യപൂര്‍വം വിനയത്തിന്റെ ചിറക് ഇരുവര്‍ക്കും താഴ്ത്തിക്കൊടുക്കുക. അതോടൊപ്പം ഇങ്ങനെ പ്രാര്‍ഥിക്കുക: "എന്റെ നാഥാ! കുട്ടിക്കാലത്ത് അവരിരുവരും എന്നെ പോറ്റിവളര്‍ത്തിയപോലെ നീ അവരോട് കരുണ കാണിക്കേണമേ.”
\end{malayalam}}
\flushright{\begin{Arabic}
\quranayah[17][25]
\end{Arabic}}
\flushleft{\begin{malayalam}
നിങ്ങളുടെ നാഥന്‍ നിങ്ങളുടെ മനസ്സിലുള്ളത് നന്നായറിയുന്നവനാണ്. നിങ്ങള്‍ സച്ചരിതരാവുകയാണെങ്കില്‍ നിശ്ചയമായും ഖേദിച്ച് സത്യത്തിലേക്ക് തിരിച്ചുവരുന്നവര്‍ക്ക് അവന്‍ ഏറെ പൊറുത്തുകൊടുക്കുന്നവനാണ്.
\end{malayalam}}
\flushright{\begin{Arabic}
\quranayah[17][26]
\end{Arabic}}
\flushleft{\begin{malayalam}
അടുത്ത കുടുംബക്കാരന്ന് അവന്റെ അവകാശം കൊടുക്കുക. അഗതിക്കും വഴിപോക്കന്നുമുള്ളത് അവര്‍ക്കും. എന്നാല്‍ ധനം ധൂര്‍ത്തടിക്കരുത്.
\end{malayalam}}
\flushright{\begin{Arabic}
\quranayah[17][27]
\end{Arabic}}
\flushleft{\begin{malayalam}
നിശ്ചയം ധൂര്‍ത്തന്മാര്‍ പിശാചുക്കളുടെ സഹോദരങ്ങളാകുന്നു. പിശാചോ തന്റെ നാഥനോട് നന്ദികെട്ടവനും.
\end{malayalam}}
\flushright{\begin{Arabic}
\quranayah[17][28]
\end{Arabic}}
\flushleft{\begin{malayalam}
നിന്റെ നാഥനില്‍ നിന്ന് നീയാഗ്രഹിക്കുന്ന അനുഗ്രഹം പ്രതീക്ഷിച്ച് നിനക്ക് അവരുടെ ആവശ്യം അവഗണിക്കേണ്ടിവന്നാല്‍ നീ അവരോട് സൌമ്യമായി ആശ്വാസവാക്കു പറയണം.
\end{malayalam}}
\flushright{\begin{Arabic}
\quranayah[17][29]
\end{Arabic}}
\flushleft{\begin{malayalam}
നിന്റെ കൈ നീ പിരടിയില്‍ കെട്ടിവെക്കരുത്. അതിനെ മുഴുവനായി നിവര്‍ത്തിയിടുകയുമരുത്. അങ്ങനെ ചെയ്താല്‍ നീ നിന്ദിതനും ദുഃഖിതനുമായിത്തീരും.
\end{malayalam}}
\flushright{\begin{Arabic}
\quranayah[17][30]
\end{Arabic}}
\flushleft{\begin{malayalam}
നിന്റെ നാഥന്‍ അവനിച്ഛിക്കുന്നവര്‍ക്ക് ജീവിതവിഭവം ധാരാളമായി നല്‍കുന്നു. മറ്റു ചിലര്‍ക്ക് അതില്‍ കുറവ് വരുത്തുകയും ചെയ്യുന്നു. അവന്‍ തന്റെ ദാസന്മാരെ നന്നായറിയുന്നവനും കാണുന്നവനുമാണ്.
\end{malayalam}}
\flushright{\begin{Arabic}
\quranayah[17][31]
\end{Arabic}}
\flushleft{\begin{malayalam}
പട്ടിണി പേടിച്ച് നിങ്ങള്‍ നിങ്ങളുടെ കുട്ടികളെ കൊല്ലരുത്. അവര്‍ക്കും നിങ്ങള്‍ക്കും അന്നം നല്‍കുന്നത് നാമാണ്. അവരെ കൊല്ലുന്നത് കൊടിയകുറ്റം തന്നെ.
\end{malayalam}}
\flushright{\begin{Arabic}
\quranayah[17][32]
\end{Arabic}}
\flushleft{\begin{malayalam}
നിങ്ങള്‍ വ്യഭിചാരത്തോടടുക്കുകപോലുമരുത്. അത് നീചമാണ്. ഹീനമായ മാര്‍ഗവും.
\end{malayalam}}
\flushright{\begin{Arabic}
\quranayah[17][33]
\end{Arabic}}
\flushleft{\begin{malayalam}
അല്ലാഹു ആദരിച്ച മനുഷ്യജീവനെ അന്യായമായി നിങ്ങള്‍ ഹനിക്കരുത്. ആരെങ്കിലും അന്യായമായി വധിക്കപ്പെട്ടാല്‍ അവന്റെ അവകാശികള്‍ക്കു നാം പ്രതിക്രിയക്ക് അധികാരം നല്‍കിയിരിക്കുന്നു. എന്നാല്‍, അവന്‍ കൊലയില്‍ അതിരുകവിയരുത്. തീര്‍ച്ചയായും അവന്‍ സഹായം ലഭിക്കുന്നവനാകുന്നു.
\end{malayalam}}
\flushright{\begin{Arabic}
\quranayah[17][34]
\end{Arabic}}
\flushleft{\begin{malayalam}
അനാഥന്റെ ധനത്തോട് നിങ്ങളടുക്കാതിരിക്കുക; ഏറ്റം നല്ല നിലയിലല്ലാതെ. അവന്‍ കാര്യവിവരമുള്ളവനാകും വരെ. നിങ്ങള്‍ കരാര്‍ പാലിക്കുക. കരാറിനെക്കുറിച്ച് ചോദ്യം ചെയ്യപ്പെടും; തീര്‍ച്ച.
\end{malayalam}}
\flushright{\begin{Arabic}
\quranayah[17][35]
\end{Arabic}}
\flushleft{\begin{malayalam}
നിങ്ങള്‍ അളന്നുകൊടുക്കുമ്പോള്‍ അളവില്‍ തികവ് വരുത്തുക. കൃത്യതയുള്ള തുലാസ്സുകൊണ്ട് തൂക്കിക്കൊടുക്കുക. അതാണ് ഏറ്റം നല്ലത്. അന്ത്യഫലം ഏറ്റം മികച്ചതാകാനുള്ള വഴിയും അതു തന്നെ.
\end{malayalam}}
\flushright{\begin{Arabic}
\quranayah[17][36]
\end{Arabic}}
\flushleft{\begin{malayalam}
നിനക്കറിയാത്തവയെ നീ പിന്‍പറ്റരുത്. കാതും കണ്ണും മനസ്സുമെല്ലാം ചോദ്യം ചെയ്യപ്പെടുന്നവതന്നെ.
\end{malayalam}}
\flushright{\begin{Arabic}
\quranayah[17][37]
\end{Arabic}}
\flushleft{\begin{malayalam}
നീ ഭൂമിയില്‍ അഹങ്കരിച്ചുനടക്കരുത്. ഭൂമിയെ പിളര്‍ക്കാനൊന്നും നിനക്കാവില്ല. പര്‍വതങ്ങളോളം പൊക്കംവെക്കാനും നിനക്കു സാധ്യമല്ല; ഉറപ്പ്.
\end{malayalam}}
\flushright{\begin{Arabic}
\quranayah[17][38]
\end{Arabic}}
\flushleft{\begin{malayalam}
ഇവയിലെ മോശമായവയെല്ലാം നിന്റെ നാഥന്‍ വെറുത്തകറ്റിയവയാണ്.
\end{malayalam}}
\flushright{\begin{Arabic}
\quranayah[17][39]
\end{Arabic}}
\flushleft{\begin{malayalam}
നിന്റെ നാഥന്‍ നിനക്കു ബോധനം നല്‍കിയ ജ്ഞാനത്തില്‍ പെട്ടതാണിത്. നീ അല്ലാഹുവോടൊപ്പം മറ്റൊരു ദൈവത്തെയും സ്വീകരിക്കരുത്. അങ്ങനെ ചെയ്താല്‍ നീ നിന്ദ്യനും ദിവ്യാനുഗ്രഹം വിലക്കപ്പെട്ടവനുമായി നരകത്തിലെറിയപ്പെടും.
\end{malayalam}}
\flushright{\begin{Arabic}
\quranayah[17][40]
\end{Arabic}}
\flushleft{\begin{malayalam}
നിങ്ങളുടെ നാഥന്‍ നിങ്ങള്‍ക്ക് പുത്രന്മാരെ തരികയും തനിക്കുവേണ്ടി മലക്കുകളില്‍നിന്ന് പുത്രിമാരെ സ്വീകരിക്കുകയുമാണോ ചെയ്തത്? വളരെ ഗുരുതരമായ വാക്കാണ് നിങ്ങള്‍ പറയുന്നത്.
\end{malayalam}}
\flushright{\begin{Arabic}
\quranayah[17][41]
\end{Arabic}}
\flushleft{\begin{malayalam}
ജനം ചിന്തിച്ചു മനസ്സിലാക്കാനായി നാം ഈ ഖുര്‍ആനില്‍ കാര്യങ്ങള്‍ വിവിധ രൂപത്തില്‍ വിശദീകരിച്ചിരിക്കുന്നു. എന്നിട്ടും ഇത് സത്യത്തില്‍ നിന്നുള്ള അവരുടെ അകല്‍ച്ച വര്‍ധിപ്പിക്കുക മാത്രമാണ് ചെയ്യുന്നത്.
\end{malayalam}}
\flushright{\begin{Arabic}
\quranayah[17][42]
\end{Arabic}}
\flushleft{\begin{malayalam}
പറയുക: അവര്‍ വാദിക്കുംപോലെ അല്ലാഹുവോടൊപ്പം മറ്റു ദൈവങ്ങളുണ്ടായിരുന്നെങ്കില്‍ ഉറപ്പായും അവര്‍ സിംഹാസനാധിപന്റെ സ്ഥാനത്തെത്താന്‍ സകലമാര്‍ഗങ്ങളും തേടുമായിരുന്നു.
\end{malayalam}}
\flushright{\begin{Arabic}
\quranayah[17][43]
\end{Arabic}}
\flushleft{\begin{malayalam}
അവര്‍ പറഞ്ഞുപരത്തുന്നതില്‍ നിന്നെല്ലാം അല്ലാഹു എത്രയോ പരിശുദ്ധനാണ്. അവയ്ക്കെല്ലാമുപരി അവന്‍ എത്രയോ ഉന്നതനായിരിക്കുന്നു.
\end{malayalam}}
\flushright{\begin{Arabic}
\quranayah[17][44]
\end{Arabic}}
\flushleft{\begin{malayalam}
ഏഴാകാശങ്ങളും ഭൂമിയും അവയിലുള്ളവരൊക്കെയും അവന്റെ വിശുദ്ധിയെ വാഴ്ത്തുന്നു. അവനെ സ്തുതിക്കുന്നതോടൊപ്പം അവന്റെ പരിശുദ്ധിയെ പ്രകീര്‍ത്തിക്കാത്ത യാതൊന്നുമില്ല. പക്ഷേ, അവരുടെ പ്രകീര്‍ത്തനം നിങ്ങള്‍ക്കു മനസ്സിലാവുകയില്ല. അവന്‍ വളരെ സഹനമുള്ളവനും ഏറെ പൊറുക്കുന്നവനുമാണ്.
\end{malayalam}}
\flushright{\begin{Arabic}
\quranayah[17][45]
\end{Arabic}}
\flushleft{\begin{malayalam}
നീ ഖുര്‍ആന്‍ പാരായണം ചെയ്യുമ്പോള്‍ നിനക്കും പരലോകത്തില്‍ വിശ്വസിക്കാത്തവര്‍ക്കുമിടയില്‍ നാം അദൃശ്യമായ ഒരു മറയിടുന്നു.
\end{malayalam}}
\flushright{\begin{Arabic}
\quranayah[17][46]
\end{Arabic}}
\flushleft{\begin{malayalam}
അത് മനസ്സിലാക്കാനാവാത്ത വിധം അവരുടെ ഹൃദയങ്ങള്‍ക്കു നാം മൂടിയിടുന്നു. കാതുകള്‍ക്ക് അടപ്പിടുന്നു. നിന്റെ നാഥനെ മാത്രം ഈ ഖുര്‍ആനില്‍ നീ പരാമര്‍ശിക്കുമ്പോള്‍ അവര്‍ വെറുപ്പോടെ പിന്തിരിഞ്ഞുപോകുന്നു.
\end{malayalam}}
\flushright{\begin{Arabic}
\quranayah[17][47]
\end{Arabic}}
\flushleft{\begin{malayalam}
നിന്റെ വാക്കുകള്‍ അവര്‍ ചെവികൊടുത്ത് കേള്‍ക്കുമ്പോള്‍ യഥാര്‍ഥത്തില്‍ എന്താണവര്‍ ശ്രദ്ധിച്ചുകേട്ടുകൊണ്ടിരുന്നതെന്ന് നമുക്ക് നന്നായറിയാം. അവര്‍ സ്വകാര്യം പറയുമ്പോള്‍ എന്താണവര്‍ പറയുന്നതെന്നും. ഈ അക്രമികള്‍ പറഞ്ഞുകൊണ്ടിരുന്നത് “നിങ്ങള്‍ പിന്തുടരുന്നത് മാരണം ബാധിച്ച ഒരു മനുഷ്യനെ മാത്രമാണെ”ന്നാണ്.
\end{malayalam}}
\flushright{\begin{Arabic}
\quranayah[17][48]
\end{Arabic}}
\flushleft{\begin{malayalam}
നോക്കൂ! എവ്വിധമാണ് അവര്‍ നിനക്ക് ഉപമകള്‍ ചമക്കുന്നത്? അങ്ങനെ അവര്‍ പിഴച്ചുപോയിരിക്കുന്നു. അതിനാലവര്‍ക്ക് നേര്‍വഴി പ്രാപിക്കാനാവില്ല.
\end{malayalam}}
\flushright{\begin{Arabic}
\quranayah[17][49]
\end{Arabic}}
\flushleft{\begin{malayalam}
അവര്‍ ചോദിക്കുന്നു: "ഞങ്ങള്‍ എല്ലുകളും നുരുമ്പിയ തുരുമ്പുകളുമായി മാറിയാല്‍ പിന്നെയും പുതിയ സൃഷ്ടിയായി ഉയിര്‍ത്തെഴുന്നേല്‍പിക്കപ്പെടുമെന്നോ?”
\end{malayalam}}
\flushright{\begin{Arabic}
\quranayah[17][50]
\end{Arabic}}
\flushleft{\begin{malayalam}
പറയുക: "നിങ്ങള്‍ കല്ലോ ഇരുമ്പോ ആയിക്കൊള്ളുക.
\end{malayalam}}
\flushright{\begin{Arabic}
\quranayah[17][51]
\end{Arabic}}
\flushleft{\begin{malayalam}
"അതല്ലെങ്കില്‍ നിങ്ങളുടെ മനസ്സുകളില്‍ കൂടുതല്‍ വലുതായി ത്തോന്നുന്ന മറ്റു വല്ല സൃഷ്ടിയുമായിത്തീരുക; എന്നാലും നിങ്ങളെ ഉയിര്‍ത്തെഴുന്നേല്‍പിക്കും.” അപ്പോഴവര്‍ ചോദിക്കും: "ആരാണ് ഞങ്ങളെ ജീവിതത്തിലേക്ക് തിരിച്ചുകൊണ്ടുവരിക?” പറയുക: "നിങ്ങളെ ആദ്യം സൃഷ്ടിച്ചവന്‍ തന്നെ.” അന്നേരമവര്‍ നിന്റെ നേരെ തലയാട്ടിക്കൊണ്ട് ചോദിക്കും: "എപ്പോഴാണ് അതുണ്ടാവുക?” പറയുക: "അടുത്തുതന്നെ ആയേക്കാം.”
\end{malayalam}}
\flushright{\begin{Arabic}
\quranayah[17][52]
\end{Arabic}}
\flushleft{\begin{malayalam}
അവന്‍ നിങ്ങളെ വിളിക്കുകയും നിങ്ങള്‍ അവനെ സ്തുതിച്ചുകൊണ്ട് ഉത്തരം നല്‍കുകയും ചെയ്യുന്ന ദിവസം. അപ്പോള്‍ അല്‍പകാലം മാത്രമേ നിങ്ങള്‍ ഭൂമിയില്‍ കഴിച്ചുകൂട്ടിയിട്ടുള്ളൂവെന്ന് നിങ്ങള്‍ക്കു തോന്നും.
\end{malayalam}}
\flushright{\begin{Arabic}
\quranayah[17][53]
\end{Arabic}}
\flushleft{\begin{malayalam}
നീ എന്റെ ദാസന്മാരോടു പറയുക: അവര്‍ പറയുന്നത് ഏറ്റം മികച്ച വാക്കുകളാകട്ടെ. തീര്‍ച്ചയായും പിശാച് അവര്‍ക്കിടയില്‍ കുഴപ്പം കുത്തിപ്പൊക്കുന്നു. പിശാച് മനുഷ്യന്റെ പ്രത്യക്ഷ ശത്രുതന്നെ.
\end{malayalam}}
\flushright{\begin{Arabic}
\quranayah[17][54]
\end{Arabic}}
\flushleft{\begin{malayalam}
നിങ്ങളുടെ നാഥന്‍ നിങ്ങളെപ്പറ്റി നന്നായറിയുന്നവനാണ്. അവനിച്ഛിക്കുന്നുവെങ്കില്‍ അവന്‍ നിങ്ങളോട് കരുണകാണിക്കും. അവനിച്ഛിക്കുന്നുവെങ്കില്‍ നിങ്ങളെ ശിക്ഷിക്കുകയും ചെയ്യും. നാം നിന്നെ അവരുടെ കൈകാര്യകര്‍ത്താവായി നിയോഗിച്ചിട്ടില്ല.
\end{malayalam}}
\flushright{\begin{Arabic}
\quranayah[17][55]
\end{Arabic}}
\flushleft{\begin{malayalam}
ആകാശഭൂമികളിലുള്ളവരെക്കുറിച്ചൊക്കെ നന്നായറിയുന്നവന്‍ നിന്റെ നാഥനാണ്. തീര്‍ച്ചയായും നാം പ്രവാചകന്മാരില്‍ ചിലര്‍ക്ക് മറ്റു ചിലരേക്കാള്‍ ശ്രേഷ്ഠത നല്‍കിയിട്ടുണ്ട്. ദാവൂദിന് നാം സങ്കീര്‍ത്തനം നല്‍കി.
\end{malayalam}}
\flushright{\begin{Arabic}
\quranayah[17][56]
\end{Arabic}}
\flushleft{\begin{malayalam}
പറയുക: അല്ലാഹുവെക്കൂടാതെ ദൈവങ്ങളെന്ന് നിങ്ങള്‍ വാദിച്ചുവരുന്നവരോട് പ്രാര്‍ഥിച്ചു നോക്കൂ. നിങ്ങളില്‍ നിന്ന് ഒരു ദുരിതവും തട്ടിമാറ്റാനവര്‍ക്കു സാധ്യമല്ല. ഒന്നിനും ഒരു മാറ്റവും വരുത്താന്‍ അവര്‍ക്കാവില്ല.
\end{malayalam}}
\flushright{\begin{Arabic}
\quranayah[17][57]
\end{Arabic}}
\flushleft{\begin{malayalam}
ഇക്കൂട്ടര്‍ ആരെയാണോ വിളിച്ചുപ്രാര്‍ഥിക്കുന്നത് അവര്‍ സ്വയംതന്നെ തങ്ങളുടെ നാഥന്റെ സാമീപ്യംനേടാന്‍ വഴിതേടിക്കൊണ്ടിരിക്കുകയാണ്. അവരില്‍ അല്ലാഹുവുമായി ഏറ്റവും അടുത്തവരുടെ അവസ്ഥയിതാണ്: അവര്‍ അവന്റെ കാരുണ്യം കൊതിക്കുന്നു. അവന്റെ ശിക്ഷയെ ഭയപ്പെടുന്നു. നിന്റെ നാഥന്റെ ശിക്ഷ പേടിക്കപ്പെടേണ്ടതുതന്നെ; തീര്‍ച്ച.
\end{malayalam}}
\flushright{\begin{Arabic}
\quranayah[17][58]
\end{Arabic}}
\flushleft{\begin{malayalam}
ഉയിര്‍ത്തെഴുന്നേല്‍പുനാളിന് മുമ്പായി നാം നശിപ്പിക്കുകയോ കഠിനമായി ശിക്ഷിക്കുകയോ ചെയ്യാത്ത ഒരു നാടുമുണ്ടാവുകയില്ല. അത് മൂലപ്രമാണത്തില്‍ രേഖപ്പെടുത്തിയ കാര്യമാണ്.
\end{malayalam}}
\flushright{\begin{Arabic}
\quranayah[17][59]
\end{Arabic}}
\flushleft{\begin{malayalam}
ദൃഷ്ടാന്തങ്ങള്‍ അയക്കുന്നതില്‍ നിന്നു നമ്മെ തടയുന്നത് ഇവര്‍ക്കു മുമ്പുണ്ടായിരുന്നവര്‍ അത്തരം ദൃഷ്ടാന്തങ്ങളെ നിഷേധിച്ചു തള്ളിക്കളഞ്ഞുവെന്നതു മാത്രമാണ്. സമൂദ് ഗോത്രത്തിനു നാം പ്രത്യക്ഷ അടയാളമായി ഒട്ടകത്തെ നല്‍കി. എന്നാല്‍ അവരതിനോട് അതിക്രമം കാണിക്കുകയാണുണ്ടായത്. നാം ദൃഷ്ടാന്തങ്ങളയക്കുന്നത് ഭയപ്പെടുത്താന്‍ വേണ്ടി മാത്രമാണ്.
\end{malayalam}}
\flushright{\begin{Arabic}
\quranayah[17][60]
\end{Arabic}}
\flushleft{\begin{malayalam}
നിന്റെ നാഥന്‍ മനുഷ്യരെയൊന്നടങ്കം വലയം ചെയ്തിരിക്കുന്നുവെന്ന് നാം നിന്നോട് പറഞ്ഞ സന്ദര്‍ഭം ഓര്‍ക്കുക. നിനക്കു നാം കാണിച്ചുതന്ന ആ കാഴ്ച നാം ജനങ്ങള്‍ക്ക് ഒരു പരീക്ഷണമാക്കുകയാണ് ചെയ്തത്. ഖുര്‍ആനില്‍ ശപിക്കപ്പെട്ട ആ വൃക്ഷവും അങ്ങനെതന്നെ. നാം അവരെ ഭയപ്പെടുത്തുകയാണ്. എന്നാല്‍ അതവരില്‍ ധിക്കാരം വളര്‍ത്തുക മാത്രമേ ചെയ്യുന്നുള്ളൂ.
\end{malayalam}}
\flushright{\begin{Arabic}
\quranayah[17][61]
\end{Arabic}}
\flushleft{\begin{malayalam}
നിങ്ങള്‍ ആദമിന് സാഷ്ടാംഗം ചെയ്യുകയെന്ന് നാം മലക്കുകളോട് പറഞ്ഞ സന്ദര്‍ഭം! അപ്പോഴവര്‍ സാഷ്ടാംഗം പ്രണമിച്ചു. ഇബ്ലീസൊഴികെ. അവന്‍ പറഞ്ഞു: "നീ കളിമണ്ണുകൊണ്ടുണ്ടാക്കിയവന് ഞാന്‍ സാഷ്ടാംഗം ചെയ്യുകയോ?”
\end{malayalam}}
\flushright{\begin{Arabic}
\quranayah[17][62]
\end{Arabic}}
\flushleft{\begin{malayalam}
ഇബ്ലീസ് പറഞ്ഞു: "എന്നേക്കാള്‍ നീ ഇവനെ ആദരണീയനാക്കി. ഇവന്‍ അതിനര്‍ഹനാണോയെന്ന് നീയെന്നെ അറിയിക്കുക. ഉയിര്‍ത്തെഴുന്നേല്‍പുനാള്‍ വരെ നീയെനിക്കു സമയം അനുവദിക്കുകയാണെങ്കില്‍ അവന്റെ സന്താനങ്ങളില്‍ അല്‍പം ചിലരെയൊഴികെ എല്ലാവരെയും ഞാന്‍ ആ പദവിയില്‍നിന്ന് പിഴുതെറിയുക തന്നെ ചെയ്യും.”
\end{malayalam}}
\flushright{\begin{Arabic}
\quranayah[17][63]
\end{Arabic}}
\flushleft{\begin{malayalam}
അല്ലാഹു പറഞ്ഞു: "നീ പോയിക്കൊള്ളുക. അവരില്‍ നിന്നാരെങ്കിലും നിന്നെ പിന്തുടരുകയാണെങ്കില്‍ നിങ്ങള്‍ക്കുള്ള പ്രതിഫലം നരകമായിരിക്കും. ഇത് തികവൊത്ത പ്രതിഫലംതന്നെ.
\end{malayalam}}
\flushright{\begin{Arabic}
\quranayah[17][64]
\end{Arabic}}
\flushleft{\begin{malayalam}
"നിന്റെ ഒച്ചവെക്കലിലൂടെ അവരില്‍നിന്ന് നിനക്ക് കഴിയാവുന്നവരെയൊക്കെ നീ തെറ്റിച്ചുകൊള്ളുക. അവര്‍ക്കെതിരെ നീ നിന്റെ കുതിരപ്പടയെയും കാലാള്‍പ്പടയെയും ഒരുമിച്ചുകൂട്ടുക. സമ്പത്തിലും സന്താനങ്ങളിലും അവരോടൊപ്പം കൂട്ടുചേര്‍ന്നുകൊള്ളുക. അവര്‍ക്ക് നീ മോഹന വാഗ്ദാനങ്ങള്‍ നല്‍കുകയും ചെയ്യുക.” പിശാചിന്റെ അവരോടുള്ള വാഗ്ദാനം കൊടും ചതിയല്ലാതൊന്നുമല്ല.
\end{malayalam}}
\flushright{\begin{Arabic}
\quranayah[17][65]
\end{Arabic}}
\flushleft{\begin{malayalam}
"നിശ്ചയമായും എന്റെ ദാസന്മാരുടെ മേല്‍ നിനക്ക് ഒരധികാരവുമില്ല. ഭരമേല്‍പിക്കപ്പെടാന്‍ നിന്റെ നാഥന്‍ തന്നെ മതി.”
\end{malayalam}}
\flushright{\begin{Arabic}
\quranayah[17][66]
\end{Arabic}}
\flushleft{\begin{malayalam}
നിങ്ങള്‍ക്കായി കടലിലൂടെ കപ്പലോടിക്കുന്നവനാണ് നിങ്ങളുടെ നാഥന്‍. നിങ്ങളവന്റെ ഔദാര്യം തേടിപ്പിടിക്കാന്‍വേണ്ടി. അവന്‍ നിങ്ങളോട് അളവറ്റ കാരുണ്യവാനാണ്.
\end{malayalam}}
\flushright{\begin{Arabic}
\quranayah[17][67]
\end{Arabic}}
\flushleft{\begin{malayalam}
കടലില്‍ നിങ്ങളെ വല്ല വിപത്തും ബാധിച്ചാല്‍ അല്ലാഹുവെക്കൂടാതെ നിങ്ങള്‍ വിളിച്ചു പ്രാര്‍ഥിക്കുന്നവയെല്ലാം അപ്രത്യക്ഷമാകുന്നു. എന്നാല്‍ അവന്‍ നിങ്ങളെ കരയിലേക്ക് രക്ഷപ്പെടുത്തിയാല്‍ നിങ്ങള്‍ അവനില്‍നിന്ന് തിരിഞ്ഞുകളയുന്നു. മനുഷ്യന്‍ ഏറെ നന്ദികെട്ടവന്‍ തന്നെ.
\end{malayalam}}
\flushright{\begin{Arabic}
\quranayah[17][68]
\end{Arabic}}
\flushleft{\begin{malayalam}
കരയുടെ ഓരത്തുതന്നെ അവന്‍ നിങ്ങളെ ആഴ്ത്തിക്കളയുന്നുവെന്നു വെക്കുക. അല്ലെങ്കില്‍ അവന്‍ നിങ്ങളുടെ നേരെ ചരല്‍മഴ വീഴ്ത്തുന്നുവെന്ന്. ഇതില്‍നിന്നെല്ലാം നിങ്ങളെ രക്ഷിക്കുന്ന ആരെയും കണ്ടെത്താന്‍ കഴിയുന്നില്ലെന്നും! ഇതേക്കുറിച്ചൊക്കെ നിങ്ങള്‍ തീര്‍ത്തും നിര്‍ഭയരാണോ?
\end{malayalam}}
\flushright{\begin{Arabic}
\quranayah[17][69]
\end{Arabic}}
\flushleft{\begin{malayalam}
അല്ലെങ്കില്‍ മറ്റൊരിക്കല്‍ അവന്‍ നിങ്ങളെ കടലിലേക്ക് തിരിച്ചുകൊണ്ടുപോകുന്നു; അങ്ങനെ നിങ്ങള്‍ നന്ദികേട് കാണിച്ചതിന് ശിക്ഷയായി നിങ്ങളുടെ നേരെ കൊടുങ്കാറ്റയച്ച് നിങ്ങളെ അതില്‍ മുക്കിക്കളയുന്നു; പിന്നീട് അക്കാര്യത്തില്‍ നിങ്ങള്‍ക്കായി നമുക്കെതിരെ നടപടിയെടുക്കാന്‍ നിങ്ങള്‍ക്കാരെയും കണ്ടെത്താനാവുന്നുമില്ല- ഇത്തരമൊരവസ്ഥയെക്കുറിച്ചും നിങ്ങള്‍ നിര്‍ഭയരാണോ?
\end{malayalam}}
\flushright{\begin{Arabic}
\quranayah[17][70]
\end{Arabic}}
\flushleft{\begin{malayalam}
ഉറപ്പായും ആദം സന്തതികളെ നാം ആദരിച്ചിരിക്കുന്നു. അവര്‍ക്കു നാം കടലിലും കരയിലും സഞ്ചരിക്കാനായി വാഹനങ്ങളൊരുക്കി. ഉത്തമ വിഭവങ്ങള്‍ ആഹാരമായി നല്‍കി. നാം സൃഷ്ടിച്ച നിരവധി സൃഷ്ടികളെക്കാള്‍ നാമവര്‍ക്ക് മഹത്വമേകുകയും ചെയ്തു.
\end{malayalam}}
\flushright{\begin{Arabic}
\quranayah[17][71]
\end{Arabic}}
\flushleft{\begin{malayalam}
ഒരു ദിനം! അന്നു നാം എല്ലാ ഓരോ ജനവിഭാഗത്തെയും തങ്ങളുടെ നേതാവിനോടൊപ്പം ഒരിടത്ത് വിളിച്ചുകൂട്ടും. അന്ന് കര്‍മപുസ്തകം വലതുകയ്യില്‍ നല്‍കപ്പെടുന്നവര്‍ തങ്ങളുടെ രേഖ വായിച്ചുനോക്കും. അവരൊട്ടും അനീതിക്കിരയാവില്ല.
\end{malayalam}}
\flushright{\begin{Arabic}
\quranayah[17][72]
\end{Arabic}}
\flushleft{\begin{malayalam}
ഈ ലോകത്ത് കണ്ണു കാണാത്തവനെപ്പോലെ കഴിയുന്നവന്‍ പരലോകത്ത് കണ്ണുപൊട്ടനായിരിക്കും. പറ്റെ വഴി പിഴച്ചവനും.
\end{malayalam}}
\flushright{\begin{Arabic}
\quranayah[17][73]
\end{Arabic}}
\flushleft{\begin{malayalam}
സംശയമില്ല; നാം നിനക്ക് ബോധനം നല്‍കിയ സന്ദേശങ്ങളില്‍ നിന്ന് നിന്നെ തെറ്റിച്ച് നാശത്തിലകപ്പെടുത്താന്‍ അവരൊരുങ്ങിയിരിക്കുന്നു. നീ സ്വയം കെട്ടിച്ചമച്ചവ നമ്മുടെ പേരില്‍വെച്ചുകെട്ടണമെന്നാണവരാഗ്രഹിക്കുന്നത്. അങ്ങനെ ചെയ്താല്‍ ഉറപ്പായും അവര്‍ നിന്നെ ഉറ്റ മിത്രമായി സ്വീകരിക്കും.
\end{malayalam}}
\flushright{\begin{Arabic}
\quranayah[17][74]
\end{Arabic}}
\flushleft{\begin{malayalam}
നിന്നെ നാം ഉറപ്പിച്ചു നിര്‍ത്തിയില്ലായിരുന്നെങ്കില്‍ നീ അവരുടെ പക്ഷത്തേക്ക് അല്‍പസ്വല്‍പം ചാഞ്ഞുപോകുമായിരുന്നു.
\end{malayalam}}
\flushright{\begin{Arabic}
\quranayah[17][75]
\end{Arabic}}
\flushleft{\begin{malayalam}
അങ്ങനെ സംഭവിച്ചിരുന്നെങ്കില്‍ നിന്നെ നാം ഇഹത്തിലും പരത്തിലും ഇരട്ടി ശിക്ഷ ആസ്വദിപ്പിക്കും. അപ്പോള്‍ നമുക്കെതിരെ നിന്നെ സഹായിക്കാന്‍ ആരേയും കണ്ടെത്താനാവില്ല.
\end{malayalam}}
\flushright{\begin{Arabic}
\quranayah[17][76]
\end{Arabic}}
\flushleft{\begin{malayalam}
നിന്നെ ഭൂമിയില്‍നിന്ന് പറിച്ചെടുത്ത് പുറത്തെറിയാന്‍ അവര്‍ തയ്യാറെടുത്തിരിക്കുന്നു. അങ്ങനെ സംഭവിച്ചാല്‍ നിനക്കു ശേഷം അല്‍പകാലമല്ലാതെ അവരവിടെ താമസിക്കാന്‍ പോകുന്നില്ല.
\end{malayalam}}
\flushright{\begin{Arabic}
\quranayah[17][77]
\end{Arabic}}
\flushleft{\begin{malayalam}
നിനക്കു മുമ്പ് നാം അയച്ച നമ്മുടെ ദൂതന്മാരുടെ കാര്യത്തിലുണ്ടായ നടപടിക്രമം തന്നെയാണിത്. നമ്മുടെ നടപടിക്രമങ്ങളില്‍ ഒരു വ്യത്യാസവും നിനക്കു കാണാനാവില്ല.
\end{malayalam}}
\flushright{\begin{Arabic}
\quranayah[17][78]
\end{Arabic}}
\flushleft{\begin{malayalam}
സൂര്യന്‍ തെറ്റുന്നതു മുതല്‍ രാവ് ഇരുളും വരെ നമസ്കാരം നിഷ്ഠയോടെ നിര്‍വഹിക്കുക; ഖുര്‍ആന്‍ പാരായണം ചെയ്തുള്ള പ്രഭാത നമസ്കാരവും. തീര്‍ച്ചയായും പ്രഭാത പ്രാര്‍ഥനയിലെ ഖുര്‍ആന്‍ പാരായണം സാക്ഷ്യം വഹിക്കപ്പെടുന്നതാണ്.
\end{malayalam}}
\flushright{\begin{Arabic}
\quranayah[17][79]
\end{Arabic}}
\flushleft{\begin{malayalam}
രാവില്‍ ഖുര്‍ആന്‍ പാരായണം ചെയ്ത് തഹജ്ജുദ് നമസ്കരിക്കുക. ഇത് നിനക്ക് കൂടുതല്‍ അനുഗ്രഹം നേടിത്തരുന്ന ഒന്നാണ്. അതുവഴി നിന്റെ നാഥന്‍ നിന്നെ സ്തുത്യര്‍ഹമായ സ്ഥാനത്തേക്കുയര്‍ത്തിയേക്കാം.
\end{malayalam}}
\flushright{\begin{Arabic}
\quranayah[17][80]
\end{Arabic}}
\flushleft{\begin{malayalam}
നീ പ്രാര്‍ഥിക്കുക: "എന്റെ നാഥാ, നീ എന്റെ പ്രവേശനം സത്യത്തോടൊപ്പമാക്കേണമേ! എന്റെ പുറപ്പാടും സത്യത്തോടൊപ്പമാക്കേണമേ. നിന്നില്‍ നിന്നുള്ള ഒരധികാരശക്തിയെ എനിക്ക് സഹായിയായി നല്‍കേണമേ.”
\end{malayalam}}
\flushright{\begin{Arabic}
\quranayah[17][81]
\end{Arabic}}
\flushleft{\begin{malayalam}
പ്രഖ്യാപിക്കുക: സത്യം വന്നു. മിഥ്യ തകര്‍ന്നു. മിഥ്യ തകരാനുള്ളതു തന്നെ.
\end{malayalam}}
\flushright{\begin{Arabic}
\quranayah[17][82]
\end{Arabic}}
\flushleft{\begin{malayalam}
ഈ ഖുര്‍ആനിലൂടെ നാം, സത്യവിശ്വാസികള്‍ക്ക് ആശ്വാസവും കാരുണ്യവും നല്‍കുന്ന ചിലത് ഇറക്കിക്കൊണ്ടിരിക്കുന്നു. എന്നാല്‍ അതിക്രമികള്‍ക്കിത് നഷ്ടമല്ലാതൊന്നും വര്‍ധിപ്പിക്കുന്നില്ല.
\end{malayalam}}
\flushright{\begin{Arabic}
\quranayah[17][83]
\end{Arabic}}
\flushleft{\begin{malayalam}
മനുഷ്യന് നാം അനുഗ്രഹമേകിയാല്‍ അവന്‍ തിരിഞ്ഞുകളയുന്നു. തനിക്കു തോന്നിയപോലെ തെന്നിമാറിപ്പോകുന്നു. അവന് വല്ല വിപത്തും വന്നാലോ നിരാശനാവുകയും ചെയ്യുന്നു.
\end{malayalam}}
\flushright{\begin{Arabic}
\quranayah[17][84]
\end{Arabic}}
\flushleft{\begin{malayalam}
പറയുക: ഓരോരുത്തരും തങ്ങളുടെ മനോനിലക്കനുസരിച്ച് പ്രവര്‍ത്തിച്ചുകൊണ്ടിരിക്കുന്നു. ആരാണ് ഏറ്റം ശരിയായ മാര്‍ഗത്തിലെന്ന് നന്നായറിയുന്നവന്‍ നിങ്ങളുടെ നാഥന്‍ മാത്രമാണ്.
\end{malayalam}}
\flushright{\begin{Arabic}
\quranayah[17][85]
\end{Arabic}}
\flushleft{\begin{malayalam}
ആത്മാവിനെപ്പറ്റി അവര്‍ നിന്നോട് ചോദിക്കുന്നു. പറയുക: ആത്മാവ് എന്റെ നാഥന്റെ വരുതിയിലുള്ള കാര്യമാണ്. വിജ്ഞാനത്തില്‍നിന്ന് വളരെ കുറച്ചേ നിങ്ങള്‍ക്ക് നല്‍കിയിട്ടുള്ളൂ.
\end{malayalam}}
\flushright{\begin{Arabic}
\quranayah[17][86]
\end{Arabic}}
\flushleft{\begin{malayalam}
നാം ഇച്ഛിക്കുകയാണെങ്കില്‍ നിനക്കു നാം ബോധനമായി നല്‍കിയ സന്ദേശം നാം തന്നെ പിന്‍വലിക്കുമായിരുന്നു. പിന്നെ നമുക്കെതിരെ നിന്നെ സഹായിക്കാന്‍ ഒരു രക്ഷകനെയും നിനക്കു കണ്ടെത്താനാവില്ല.
\end{malayalam}}
\flushright{\begin{Arabic}
\quranayah[17][87]
\end{Arabic}}
\flushleft{\begin{malayalam}
അങ്ങനെ സംഭവിക്കാത്തത് നിന്റെ നാഥന്റെ കാരുണ്യംകൊണ്ടു മാത്രമാണ്. നിന്നോടുള്ള അവന്റെ അനുഗ്രഹം വളരെ വലുതാണ്.
\end{malayalam}}
\flushright{\begin{Arabic}
\quranayah[17][88]
\end{Arabic}}
\flushleft{\begin{malayalam}
പറയുക: മനുഷ്യരും ജിന്നുകളും ഒത്തൊരുമിച്ചു ശ്രമിച്ചാലും ഈ ഖുര്‍ആന്‍ പോലൊന്ന് കൊണ്ടുവരാനാവില്ല- അവരെല്ലാം പരസ്പരം പിന്തുണച്ചാലും ശരി.
\end{malayalam}}
\flushright{\begin{Arabic}
\quranayah[17][89]
\end{Arabic}}
\flushleft{\begin{malayalam}
ഈ ഖുര്‍ആനില്‍ മനുഷ്യര്‍ക്കായി എല്ലാവിധ ഉപമകളും നാം വിവിധ രൂപേണ വിവരിച്ചിട്ടുണ്ട്. എന്നിട്ടും മനുഷ്യരിലേറെ പേരും അവയെ തള്ളിക്കളഞ്ഞു. സത്യനിഷേധത്തിലുറച്ചുനിന്നു.
\end{malayalam}}
\flushright{\begin{Arabic}
\quranayah[17][90]
\end{Arabic}}
\flushleft{\begin{malayalam}
അവര്‍ പറഞ്ഞു: "നീ ഞങ്ങള്‍ക്കായി ഭൂമിയില്‍നിന്ന് ഒരു ഉറവ ഒഴുക്കിത്തരുംവരെ ഞങ്ങള്‍ നിന്നെ വിശ്വസിക്കുകയില്ല;
\end{malayalam}}
\flushright{\begin{Arabic}
\quranayah[17][91]
\end{Arabic}}
\flushleft{\begin{malayalam}
"അല്ലെങ്കില്‍ നിനക്ക് ഈന്തപ്പനയുടെയും മുന്തിരിയുടെയും ഒരു തോട്ടമുണ്ടാവുകയും അവയ്ക്കിടയിലൂടെ നീ അരുവികള്‍ ഒഴുക്കുകയും ചെയ്യുക;
\end{malayalam}}
\flushright{\begin{Arabic}
\quranayah[17][92]
\end{Arabic}}
\flushleft{\begin{malayalam}
"അല്ലെങ്കില്‍ നീ വാദിക്കുംപോലെ ആകാശത്തെ ഞങ്ങളുടെ മേല്‍ കഷ്ണങ്ങളാക്കി വീഴ്ത്തുക; അല്ലാഹുവെയും മലക്കുകളെയും ഞങ്ങളുടെ മുന്നില്‍ നേരിട്ട് കൊണ്ടുവരിക;
\end{malayalam}}
\flushright{\begin{Arabic}
\quranayah[17][93]
\end{Arabic}}
\flushleft{\begin{malayalam}
"അതുമല്ലെങ്കില്‍ നീ നിനക്കായി സ്വര്‍ണ നിര്‍മിതമായ കൊട്ടാരമുണ്ടാക്കുക; നീ ആകാശത്തേക്ക് കയറിപ്പോവുക; ഞങ്ങള്‍ക്കു വായിക്കാവുന്ന ഒരു ഗ്രന്ഥം ഇറക്കിത്തരുന്നതുവരെ നീ മാനത്തേക്കു കയറിപ്പോയതായി ഞങ്ങള്‍ വിശ്വസിക്കുകയില്ല.” പറയുക: "എന്റെ നാഥന്‍ പരിശുദ്ധന്‍! ഞാന്‍ സന്ദേശവാഹകനായ ഒരു മനുഷ്യന്‍ മാത്രമാകുന്നു?”
\end{malayalam}}
\flushright{\begin{Arabic}
\quranayah[17][94]
\end{Arabic}}
\flushleft{\begin{malayalam}
ജനങ്ങള്‍ക്ക് നേര്‍വഴി വന്നെത്തിയപ്പോഴെല്ലാം അതില്‍ വിശ്വസിക്കാന്‍ അവര്‍ക്ക് തടസ്സമായത് “അല്ലാഹു ഒരു മനുഷ്യനെയാണോ തന്റെ ദൂതനായി നിയോഗിച്ചിരിക്കുന്നത്” എന്ന അവരുടെ വാദം മാത്രമാണ്.
\end{malayalam}}
\flushright{\begin{Arabic}
\quranayah[17][95]
\end{Arabic}}
\flushleft{\begin{malayalam}
പറയുക: ഭൂമിയിലുള്ളത് ശാന്തരായി നടന്നുനീങ്ങുന്ന മലക്കുകളായിരുന്നുവെങ്കില്‍ നിശ്ചയമായും അവരിലേക്കു നാം ആകാശത്തുനിന്ന് ഒരു മലക്കിനെത്തന്നെ ദൂതനായി ഇറക്കുമായിരുന്നു.
\end{malayalam}}
\flushright{\begin{Arabic}
\quranayah[17][96]
\end{Arabic}}
\flushleft{\begin{malayalam}
പറയുക: എനിക്കും നിങ്ങള്‍ക്കുമിടയില്‍ സാക്ഷിയായി അല്ലാഹു മതി. തീര്‍ച്ചയായും അവന്‍ തന്റെ ദാസന്മാരെ സൂക്ഷ്മമായി അറിയുന്നവനും കാണുന്നവനുമാണ്.
\end{malayalam}}
\flushright{\begin{Arabic}
\quranayah[17][97]
\end{Arabic}}
\flushleft{\begin{malayalam}
അല്ലാഹു നേര്‍വഴിയിലാക്കുന്നവന്‍ മാത്രമാണ് സന്മാര്‍ഗം പ്രാപിച്ചവന്‍. അവന്‍ ദുര്‍മാര്‍ഗത്തിലകപ്പെടുത്തുന്നവര്‍ക്ക് അവനെക്കൂടാതെ ഒരു രക്ഷകനെയും നിനക്കു കണ്ടെത്താനാവില്ല. ഉയിര്‍ത്തെഴുന്നേല്‍പു നാളില്‍ നാമവരെ മുഖം നിലത്തു കുത്തി വലിച്ചിഴച്ച് കൊണ്ടുവരും. അവരപ്പോള്‍ അന്ധരും ഊമകളും ബധിരരുമായിരിക്കും. അവരുടെ സങ്കേതം നരകമാണ്. അതിലെ അഗ്നി അണയുമ്പോഴൊക്കെ നാമത് ആളിക്കത്തിക്കും.
\end{malayalam}}
\flushright{\begin{Arabic}
\quranayah[17][98]
\end{Arabic}}
\flushleft{\begin{malayalam}
അവര്‍ നമ്മുടെ പ്രമാണങ്ങളെ കള്ളമാക്കി തള്ളിയതിന്റെ പ്രതിഫലമാണത്. “ഞങ്ങള്‍ എല്ലും തുരുമ്പുമായശേഷം പുതിയൊരു സൃഷ്ടിയായി ഉയിര്‍ത്തെഴുന്നേല്‍പിക്കപ്പെടുമോ” എന്ന് ചോദിച്ചതിന്റെയും.
\end{malayalam}}
\flushright{\begin{Arabic}
\quranayah[17][99]
\end{Arabic}}
\flushleft{\begin{malayalam}
ആകാശഭൂമികളെ സൃഷ്ടിച്ച അല്ലാഹു ഇവരെപ്പോലുള്ളവരെയും സൃഷ്ടിക്കാന്‍ ശക്തനാണെന്ന് ഇവരെന്തുകൊണ്ട് മനസ്സിലാക്കുന്നില്ല? അല്ലാഹു ഇവര്‍ക്കൊരവധി നിശ്ചയിച്ചിട്ടുണ്ട്. അതിലൊട്ടും സംശയമില്ല. എന്നാല്‍ അതിനെ തള്ളിപ്പറയാനല്ലാതെ അതിക്രമികള്‍ക്കാവുന്നില്ല.
\end{malayalam}}
\flushright{\begin{Arabic}
\quranayah[17][100]
\end{Arabic}}
\flushleft{\begin{malayalam}
പറയുക: എന്റെ നാഥന്റെ കാരുണ്യത്തിന്റെ ഖജനാവുകള്‍ നിങ്ങളുടെ അധീനതയിലായിരുന്നുവെങ്കില്‍ ചെലവഴിച്ചു തീര്‍ന്നുപോകുമോയെന്ന് പേടിച്ച് നിങ്ങളത് മുറുക്കിപ്പിടിക്കുമായിരുന്നു. മനുഷ്യന്‍ പറ്റെ പിശുക്കന്‍ തന്നെ.
\end{malayalam}}
\flushright{\begin{Arabic}
\quranayah[17][101]
\end{Arabic}}
\flushleft{\begin{malayalam}
മൂസാക്കു നാം പ്രത്യക്ഷത്തില്‍ കാണാവുന്ന ഒമ്പതു തെളിവുകള്‍ നല്‍കി. നീ ഇസ്രയേല്യരോട് ചോദിച്ചു നോക്കുക: അദ്ദേഹം അവരിലേക്ക് ചെന്ന സന്ദര്‍ഭം; അപ്പോള്‍ ഫറവോന്‍ പറഞ്ഞു: "മൂസാ, നിന്നെ മാരണം ബാധിച്ചവനായാണ് ഞാന്‍ കരുതുന്നത്.”
\end{malayalam}}
\flushright{\begin{Arabic}
\quranayah[17][102]
\end{Arabic}}
\flushleft{\begin{malayalam}
മൂസാ പറഞ്ഞു: "ഉള്‍ക്കാഴ്ചയുണ്ടാക്കാന്‍ പോന്ന ഈ അടയാളങ്ങള്‍ ഇറക്കിയത് ആകാശഭൂമികളുടെ നാഥനല്ലാതെ മറ്റാരുമല്ലെന്ന് താങ്കള്‍ക്കു തന്നെ നന്നായറിയാവുന്നതാണല്ലോ. ഫറവോന്‍, താങ്കള്‍ തുലഞ്ഞവനാണെന്നാണ് ഞാന്‍ കരുതുന്നത്.”
\end{malayalam}}
\flushright{\begin{Arabic}
\quranayah[17][103]
\end{Arabic}}
\flushleft{\begin{malayalam}
അപ്പോള്‍ അവരെ നാട്ടില്‍നിന്ന് വിരട്ടിയോടിക്കാന്‍ ഫറവോന്‍ തീരുമാനിച്ചു. എന്നാല്‍ അവനെയും അവന്റെ കൂട്ടാളികളെയും നാം മുക്കിക്കൊന്നു.
\end{malayalam}}
\flushright{\begin{Arabic}
\quranayah[17][104]
\end{Arabic}}
\flushleft{\begin{malayalam}
അതിനുശേഷം നാം ഇസ്രയേല്‍ മക്കളോടു പറഞ്ഞു: "നിങ്ങള്‍ ഈ നാട്ടില്‍ പാര്‍ത്തുകൊള്ളുക. പിന്നീട് പരലോകത്തിന്റെ വാഗ്ദത്ത സമയം വന്നെത്തിയാല്‍ നിങ്ങളെയെല്ലാം ഒരുമിച്ചുകൂട്ടി കൂട്ടത്തോടെ കൊണ്ടുവരുന്നതാണ്.”
\end{malayalam}}
\flushright{\begin{Arabic}
\quranayah[17][105]
\end{Arabic}}
\flushleft{\begin{malayalam}
നാം ഈ ഖുര്‍ആന്‍ ഇറക്കിയത് സത്യസന്ദേശവുമായാണ്. സത്യനിഷ്ഠമായിത്തന്നെ അത് ഇറങ്ങുകയും ചെയ്തിരിക്കുന്നു. ശുഭവാര്‍ത്ത അറിയിക്കുന്നവനും താക്കീതു നല്‍കുന്നവനുമായല്ലാതെ നിന്നെ നാം അയച്ചിട്ടില്ല.
\end{malayalam}}
\flushright{\begin{Arabic}
\quranayah[17][106]
\end{Arabic}}
\flushleft{\begin{malayalam}
ഈ ഖുര്‍ആനിനെ നാം പല ഭാഗങ്ങളായി വേര്‍തിരിച്ചിരിക്കുന്നു. നീ ജനങ്ങള്‍ക്ക് സാവധാനം ഓതിക്കൊടുക്കാന്‍ വേണ്ടിയാണിത്. നാമതിനെ ക്രമേണയായി ഇറക്കിത്തന്നിരിക്കുന്നു.
\end{malayalam}}
\flushright{\begin{Arabic}
\quranayah[17][107]
\end{Arabic}}
\flushleft{\begin{malayalam}
പറയുക: നിങ്ങള്‍ക്കിത് വിശ്വസിക്കുകയോ വിശ്വസിക്കാതിരിക്കുകയോ ചെയ്യാം. എന്നാല്‍ ഇതിനു മുമ്പെ ദിവ്യജ്ഞാനം ലഭിച്ചവര്‍ ഇത് വായിച്ചുകേള്‍ക്കുമ്പോള്‍ മുഖം കുത്തി സാഷ്ടാംഗം പ്രണമിക്കുന്നതാണ്.
\end{malayalam}}
\flushright{\begin{Arabic}
\quranayah[17][108]
\end{Arabic}}
\flushleft{\begin{malayalam}
അവര്‍ പറയും: ഞങ്ങളുടെ നാഥന്‍ എത്ര പരിശുദ്ധന്‍! ഞങ്ങളുടെ നാഥന്റെ വാഗ്ദാനം നിറവേറ്റപ്പെടുന്നതു തന്നെ.
\end{malayalam}}
\flushright{\begin{Arabic}
\quranayah[17][109]
\end{Arabic}}
\flushleft{\begin{malayalam}
അവര്‍ കരഞ്ഞുകൊണ്ട് മുഖം കുത്തിവീഴുന്നു. അതവരുടെ ഭയഭക്തി വര്‍ധിപ്പിക്കുന്നു.
\end{malayalam}}
\flushright{\begin{Arabic}
\quranayah[17][110]
\end{Arabic}}
\flushleft{\begin{malayalam}
പറയുക: നിങ്ങള്‍ “അല്ലാഹു” എന്ന് വിളിച്ചോളൂ. അല്ലെങ്കില്‍ “പരമകാരുണികനെ”ന്ന് വിളിച്ചോളൂ. നിങ്ങള്‍ ഏതു പേരു വിളിച്ചു പ്രാര്‍ഥിച്ചാലും തരക്കേടില്ല. ഉത്തമ നാമങ്ങളൊക്കെയും അവന്നുള്ളതാണ്. നിന്റെ നമസ്കാരം വളരെ ഉറക്കെയാക്കരുത്. വളരെ പതുക്കെയുമാക്കരുത്. അവയ്ക്കിടയില്‍ മധ്യമാര്‍ഗമവലംബിക്കുക.
\end{malayalam}}
\flushright{\begin{Arabic}
\quranayah[17][111]
\end{Arabic}}
\flushleft{\begin{malayalam}
അവന്‍ ആരെയും പുത്രനായി സ്വീകരിച്ചിട്ടില്ല. ആധിപത്യത്തില്‍ അവന് പങ്കാളിയുമില്ല. മാനക്കേടില്‍നിന്നു കാക്കാന്‍ ഒരു സഹായിയും അവന്നാവശ്യമില്ല. അങ്ങനെയുള്ള “അല്ലാഹുവിനു സ്തുതി” എന്നു നീ പറയുക. അവന്റെ മഹത്വം കീര്‍ത്തിക്കുക.
\end{malayalam}}
\chapter{\textmalayalam{അല്‍ കഹഫ് ( ഗുഹ‍ )}}
\begin{Arabic}
\Huge{\centerline{\basmalah}}\end{Arabic}
\flushright{\begin{Arabic}
\quranayah[18][1]
\end{Arabic}}
\flushleft{\begin{malayalam}
അല്ലാഹുവിന് സ്തുതി. തന്റെ ദാസന്ന് വേദപുസ്തകം ഇറക്കിക്കൊടുത്തവനാണവന്‍. അതിലൊരു വക്രതയും വരുത്താത്തവനും.
\end{malayalam}}
\flushright{\begin{Arabic}
\quranayah[18][2]
\end{Arabic}}
\flushleft{\begin{malayalam}
തികച്ചും ഋജുവായ വേദമാണിത്. അല്ലാഹുവിന്റെ കൊടിയ ശിക്ഷയെക്കുറിച്ച് മുന്നറിയിപ്പ് നല്‍കാനാണിത്. സല്‍ക്കര്‍മങ്ങള്‍ പ്രവര്‍ത്തിക്കുന്ന സത്യവിശ്വാസികള്‍ക്ക് ഉത്തമമായ പ്രതിഫലമുണ്ടെന്ന് ശുഭവാര്‍ത്ത അറിയിക്കാനും.
\end{malayalam}}
\flushright{\begin{Arabic}
\quranayah[18][3]
\end{Arabic}}
\flushleft{\begin{malayalam}
ആ പ്രതിഫലം എക്കാലവും അനുഭവിച്ചുകഴിയുന്നവരാണവര്‍.
\end{malayalam}}
\flushright{\begin{Arabic}
\quranayah[18][4]
\end{Arabic}}
\flushleft{\begin{malayalam}
അല്ലാഹു പുത്രനെ സ്വീകരിച്ചിരിക്കുന്നുവെന്ന് വാദിക്കുന്നവരെ താക്കീതു ചെയ്യാനുള്ളതുമാണ് ഈ വേദപുസ്തകം.
\end{malayalam}}
\flushright{\begin{Arabic}
\quranayah[18][5]
\end{Arabic}}
\flushleft{\begin{malayalam}
അവര്‍ക്കോ അവരുടെ പിതാക്കള്‍ക്കോ അതേക്കുറിച്ച് ഒന്നുമറിയില്ല. അവരുടെ വായില്‍നിന്ന് വരുന്ന വാക്ക് അത്യന്തം ഗുരുതരമാണ്. പച്ചക്കള്ളമാണവര്‍ പറയുന്നത്.
\end{malayalam}}
\flushright{\begin{Arabic}
\quranayah[18][6]
\end{Arabic}}
\flushleft{\begin{malayalam}
ഈ സന്ദേശത്തില്‍ അവര്‍ വിശ്വസിക്കുന്നില്ലെങ്കില്‍ അവരുടെ പിറകെ കടുത്ത ദുഃഖത്തോടെ നടന്നലഞ്ഞ് നീ ജീവനൊടുക്കിയേക്കാം
\end{malayalam}}
\flushright{\begin{Arabic}
\quranayah[18][7]
\end{Arabic}}
\flushleft{\begin{malayalam}
ഭൂമുഖത്തുള്ളതൊക്കെ നാം അതിന് അലങ്കാരമാക്കിയിരിക്കുന്നു. മനുഷ്യരില്‍ ആരാണ് ഏറ്റവും നല്ല കര്‍മങ്ങളിലേര്‍പ്പെടുന്നതെന്ന് പരീക്ഷിക്കാനാണിത്.
\end{malayalam}}
\flushright{\begin{Arabic}
\quranayah[18][8]
\end{Arabic}}
\flushleft{\begin{malayalam}
അവസാനം നാം അതിലുള്ളതൊക്കെയും നശിപ്പിച്ച് അതിനെ തരിശായ പ്രദേശമാക്കും; ഉറപ്പ്.
\end{malayalam}}
\flushright{\begin{Arabic}
\quranayah[18][9]
\end{Arabic}}
\flushleft{\begin{malayalam}
അതല്ല; ഗുഹയുടെയും റഖീമിന്റെയും ആള്‍ക്കാര്‍ നമ്മുടെ മഹത്തായ ദൃഷ്ടാന്തങ്ങളിലെ വലിയൊരദ്ഭുതമായിരുന്നുവെന്ന് നീ കരുതിയോ?
\end{malayalam}}
\flushright{\begin{Arabic}
\quranayah[18][10]
\end{Arabic}}
\flushleft{\begin{malayalam}
ആ ചെറുപ്പക്കാര്‍ ഗുഹയില്‍ അഭയം പ്രാപിച്ച സന്ദര്‍ഭം. അപ്പോഴവര്‍ പ്രാര്‍ഥിച്ചു: "ഞങ്ങളുടെ നാഥാ! നിന്റെ ഭാഗത്തുനിന്നുള്ള കാരുണ്യം ഞങ്ങള്‍ക്കു നീ കനിഞ്ഞേകണമേ. ഞങ്ങള്‍ ചെയ്യേണ്ട കാര്യം നേരാംവിധം നടത്താന്‍ ഞങ്ങള്‍ക്കു നീ സൌകര്യമൊരുക്കിത്തരേണമേ.”
\end{malayalam}}
\flushright{\begin{Arabic}
\quranayah[18][11]
\end{Arabic}}
\flushleft{\begin{malayalam}
അങ്ങനെ കുറേയേറെ കൊല്ലം അതേ ഗുഹയില്‍ നാം അവരെ ഉറക്കിക്കിടത്തി.
\end{malayalam}}
\flushright{\begin{Arabic}
\quranayah[18][12]
\end{Arabic}}
\flushleft{\begin{malayalam}
പിന്നീട് നാം അവരെ ഉണര്‍ത്തി. ആ ഇരുകക്ഷികളില്‍ ആരാണ് തങ്ങളുടെ ഗുഹാവാസക്കാലം കൃത്യമായി അറിയുകയെന്ന് മനസ്സിലാക്കാന്‍.
\end{malayalam}}
\flushright{\begin{Arabic}
\quranayah[18][13]
\end{Arabic}}
\flushleft{\begin{malayalam}
അവരുടെ വിവരം നിനക്കു നാം ശരിയാംവിധം വിശദീകരിച്ചു തരാം: തങ്ങളുടെ നാഥനില്‍ വിശ്വസിച്ച ഒരുപറ്റം ചെറുപ്പക്കാരായിരുന്നു അവര്‍. അവര്‍ക്കു നാം നേര്‍വഴിയില്‍ വമ്പിച്ച വളര്‍ച്ച നല്‍കി.
\end{malayalam}}
\flushright{\begin{Arabic}
\quranayah[18][14]
\end{Arabic}}
\flushleft{\begin{malayalam}
"ഞങ്ങളുടെ നാഥന്‍ ആകാശഭൂമികളുടെ നാഥനാണ്. അവനെക്കൂടാതെ മറ്റൊരു ദൈവത്തോടും ഞങ്ങള്‍ പ്രാര്‍ഥിക്കുകയില്ല. അങ്ങനെ ചെയ്താല്‍ തീര്‍ച്ചയായും ഞങ്ങള്‍ അന്യായം പറഞ്ഞവരായിത്തീരും” എന്ന് അവര്‍ എഴുന്നേറ്റു നിന്ന് പ്രഖ്യാപിച്ചപ്പോള്‍ നാം അവരുടെ മനസ്സുകള്‍ക്ക് കരുത്തേകി.
\end{malayalam}}
\flushright{\begin{Arabic}
\quranayah[18][15]
\end{Arabic}}
\flushleft{\begin{malayalam}
അവര്‍ പറഞ്ഞു: നമ്മുടെ ഈ ജനം അല്ലാഹുവെക്കൂടാതെ പല ദൈവങ്ങളെയും സങ്കല്‍പിച്ചുവെച്ചിരിക്കുന്നു. എന്നിട്ടും അവരതിന് വ്യക്തമായ തെളിവുകളൊന്നും കൊണ്ടുവരാത്തതെന്ത്? അല്ലാഹുവിന്റെ പേരില്‍ കള്ളം കെട്ടിച്ചമക്കുന്നവനേക്കാള്‍ കടുത്ത അക്രമി ആരുണ്ട്?
\end{malayalam}}
\flushright{\begin{Arabic}
\quranayah[18][16]
\end{Arabic}}
\flushleft{\begin{malayalam}
"നിങ്ങളിപ്പോള്‍ അവരെയും അല്ലാഹുവെക്കൂടാതെ അവര്‍ ആരാധിച്ചുകൊണ്ടിരിക്കുന്നവയെയും കൈവെടിഞ്ഞിരിക്കയാണല്ലോ. അതിനാല്‍ നിങ്ങള്‍ ആ ഗുഹയില്‍ അഭയം തേടിക്കൊള്ളുക. നിങ്ങളുടെ നാഥന്‍ തന്റെ അനുഗ്രഹം നിങ്ങള്‍ക്ക് ചൊരിഞ്ഞുതരും. നിങ്ങളുടെ കാര്യം നിങ്ങള്‍ക്ക് സുഗമവും സൌകര്യപ്രദവുമാക്കിത്തരും.”
\end{malayalam}}
\flushright{\begin{Arabic}
\quranayah[18][17]
\end{Arabic}}
\flushleft{\begin{malayalam}
സൂര്യന്‍ ഉദയവേളയില്‍ ആ ഗുഹയുടെ വലതുഭാഗത്തേക്ക് മാറിപ്പോകുന്നതായും അസ്തമയസമയത്ത് അവരെ വിട്ടുകടന്ന് ഇടത്തോട്ടുപോകുന്നതായും നിനക്കു കാണാം. അവരോ, ഗുഹക്കകത്ത് വിശാലമായ ഒരിടത്താകുന്നു. ഇത് അല്ലാഹുവിന്റെ അടയാളങ്ങളില്‍ പെട്ടതാണ്. അല്ലാഹു ആരെ നേര്‍വഴിയിലാക്കുന്നുവോ അവനാണ് സന്മാര്‍ഗം പ്രാപിച്ചവന്‍. അവന്‍ ആരെ വഴികേടിലാക്കുന്നുവോ അവനെ നേര്‍വഴിയിലാക്കുന്ന ഒരു രക്ഷകനേയും നിനക്കു കണ്ടെത്താനാവില്ല.
\end{malayalam}}
\flushright{\begin{Arabic}
\quranayah[18][18]
\end{Arabic}}
\flushleft{\begin{malayalam}
അവര്‍ ഉണര്‍ന്നിരിക്കുന്നവരാണെന്ന് നിനക്കു തോന്നും. യഥാര്‍ഥത്തിലവര്‍ ഉറങ്ങുന്നവരാണ്. നാമവരെ വലത്തോട്ടും ഇടത്തോട്ടും തിരിച്ചുകിടത്തിക്കൊണ്ടിരിക്കുന്നു. അവരുടെ നായ മുന്‍കാലുകള്‍ നീട്ടി ഗുഹാമുഖത്ത് ഇരിപ്പുണ്ട്. നീയെങ്ങാനും അവരെ എത്തിനോക്കിയാല്‍ ഉറപ്പായും അവരില്‍ നിന്ന് പുറംതിരിഞ്ഞോടുമായിരുന്നു. അവരെപ്പറ്റി പേടിച്ചരണ്ടവനായിത്തീരുകയും ചെയ്യും.
\end{malayalam}}
\flushright{\begin{Arabic}
\quranayah[18][19]
\end{Arabic}}
\flushleft{\begin{malayalam}
അങ്ങനെ നാം അവരെ ഉണര്‍ത്തിയെഴുന്നേല്‍പിച്ചു. അവര്‍ അന്യോന്യം അന്വേഷിച്ചറിയാന്‍. അവരിലൊരാള്‍ ചോദിച്ചു: "നിങ്ങളെത്ര കാലമിങ്ങനെ കഴിച്ചുകൂട്ടി?” മറ്റുള്ളവര്‍ പറഞ്ഞു: "നാം ഒരു ദിവസം കഴിച്ചുകൂട്ടിയിട്ടുണ്ടാവും. അല്ലെങ്കില്‍ അതില്‍നിന്ന് അല്‍പസമയം.” വേറെ ചിലര്‍ പറഞ്ഞു: നിങ്ങളുടെ നാഥനാണ് നിങ്ങള്‍ എത്രകാലമിങ്ങനെ കഴിഞ്ഞുവെന്ന് നന്നായറിയുന്നവന്‍. ഏതായാലും നിങ്ങളിലൊരാളെ നിങ്ങളുടെ ഈ വെള്ളിനാണയങ്ങളുമായി നഗരത്തിലേക്കയക്കുക. അവിടെ എവിടെയാണ് ഏറ്റവും നല്ല ഭക്ഷണമുള്ളതെന്ന് അവന്‍ നോക്കട്ടെ. എന്നിട്ടവിടെ നിന്ന് അവന്‍ നിങ്ങള്‍ക്ക് വല്ല ആഹാരവും വാങ്ങിക്കൊണ്ടുവരട്ടെ. അവന്‍ തികഞ്ഞ ജാഗ്രത പാലിക്കണം. നിങ്ങളെപ്പറ്റി അവന്‍ ആരെയും ഒരു വിവരവും അറിയിക്കരുത്.
\end{malayalam}}
\flushright{\begin{Arabic}
\quranayah[18][20]
\end{Arabic}}
\flushleft{\begin{malayalam}
നിങ്ങളെപ്പറ്റി വല്ല വിവരവും കിട്ടിയാല്‍ അവര്‍ നിങ്ങളെ എറിഞ്ഞുകൊല്ലും. അല്ലെങ്കില്‍ അവരുടെ മതത്തിലേക്ക് തിരിച്ചുപോകാനവര്‍ നിര്‍ബന്ധിക്കും. അങ്ങനെ വന്നാല്‍ പിന്നെ, നിങ്ങളൊരിക്കലും വിജയം വരിക്കുകയില്ല.
\end{malayalam}}
\flushright{\begin{Arabic}
\quranayah[18][21]
\end{Arabic}}
\flushleft{\begin{malayalam}
അങ്ങനെ അവരെ കണ്ടെത്താന്‍ നാം അവസരമൊരുക്കി. അല്ലാഹുവിന്റെ വാഗ്ദാനം സത്യമാണെന്നും അന്ത്യസമയം വരുമെന്ന കാര്യത്തില്‍ യാതൊരു സംശയവുമില്ലെന്നും അവരറിയാന്‍ വേണ്ടി. അവരന്യോന്യം ഗുഹാവാസികളുടെ കാര്യത്തില്‍ തര്‍ക്കിച്ച സന്ദര്‍ഭം ഓര്‍ക്കുക. ചിലര്‍ പറഞ്ഞു: "നിങ്ങള്‍ അവര്‍ക്കുമീതെ ഒരു കെട്ടിടമുണ്ടാക്കുക. അവരെപ്പറ്റി നന്നായറിയുന്നവന്‍ അവരുടെ നാഥനാണ്.” എന്നാല്‍ അവരുടെ കാര്യത്തില്‍ സ്വാധീനമുള്ളവര്‍ പറഞ്ഞു: "നാം അവര്‍ക്കു മീതെ ഒരാരാധനാലയം ഉണ്ടാക്കുകതന്നെ ചെയ്യും.”
\end{malayalam}}
\flushright{\begin{Arabic}
\quranayah[18][22]
\end{Arabic}}
\flushleft{\begin{malayalam}
ചിലര്‍ പറയും: "അവര്‍ മൂന്നാളായിരുന്നു. നാലാമത്തേത് അവരുടെ നായയും.” വേറെ ചിലര്‍ പറയും: "അവര്‍ അഞ്ചാളുകളാണ്. ആറാമത്തേത് അവരുടെ നായയും.” ഇതൊക്കെയും അഭൌതിക കാര്യങ്ങളെ സംബന്ധിച്ച ഊഹം മാത്രമാണ്. ഇനിയും ചിലര്‍ പറയും: "അവര്‍ ഏഴുപേരാണ്. എട്ടാമത്തേത് അവരുടെ നായയും.” പറയുക: "എന്റെ നാഥനാണ് അവരുടെ എണ്ണത്തെപ്പറ്റി ഏറ്റം നന്നായറിയുന്നവന്‍.” അല്‍പം ചിലര്‍ക്കൊഴികെ ആര്‍ക്കും അവരെപ്പറ്റി അറിയില്ല. അതിനാല്‍ വ്യക്തമായ അറിവിന്റെ അടിസ്ഥാനത്തിലല്ലാതെ അവരുടെ കാര്യത്തില്‍ നീ തര്‍ക്കിക്കരുത്. ജനങ്ങളിലാരോടും നീ അവരുടെ കാര്യത്തില്‍ അഭിപ്രായം ചോദിക്കരുത്.
\end{malayalam}}
\flushright{\begin{Arabic}
\quranayah[18][23]
\end{Arabic}}
\flushleft{\begin{malayalam}
ഒരു കാര്യത്തെക്കുറിച്ചും തീര്‍ച്ചയായും “നാളെ ഞാനത് ചെയ്യു”മെന്ന് നീ പറയരുത്;
\end{malayalam}}
\flushright{\begin{Arabic}
\quranayah[18][24]
\end{Arabic}}
\flushleft{\begin{malayalam}
“അല്ലാഹു ഇച്ഛിച്ചെങ്കില്‍” എന്ന് പറഞ്ഞല്ലാതെ. അഥവാ മറന്നുപോയാല്‍ ഉടനെ നീ നിന്റെ നാഥനെ ഓര്‍ക്കുക. എന്നിട്ടിങ്ങനെ പറയുക: "എന്റെ നാഥന്‍ എന്നെ ഇതിനെക്കാള്‍ നേരായ വഴിക്കു നയിച്ചേക്കാം.”
\end{malayalam}}
\flushright{\begin{Arabic}
\quranayah[18][25]
\end{Arabic}}
\flushleft{\begin{malayalam}
അവര്‍ തങ്ങളുടെ ഗുഹയില്‍ മുന്നൂറു കൊല്ലം താമസിച്ചു. ചിലര്‍ അതില്‍ ഒമ്പതു വര്‍ഷം കൂട്ടിപ്പറഞ്ഞു.
\end{malayalam}}
\flushright{\begin{Arabic}
\quranayah[18][26]
\end{Arabic}}
\flushleft{\begin{malayalam}
പറയുക: അവര്‍ താമസിച്ചതിനെ സംബന്ധിച്ച് ഏറ്റം നന്നായറിയുക അല്ലാഹുവിനാണ്. ആകാശഭൂമികളുടെ രഹസ്യങ്ങള്‍ അറിയുന്നത് അവന്ന് മാത്രമാണ്. അവന്‍ എന്തൊരു കാഴ്ചയുള്ളവന്‍! എത്ര നന്നായി കേള്‍ക്കുന്നവന്‍! ആര്‍ക്കും അവനല്ലാതെ ഒരു രക്ഷകനുമില്ല. തന്റെ ആധിപത്യത്തില്‍ അവനാരെയും പങ്കുചേര്‍ക്കുകയില്ല.
\end{malayalam}}
\flushright{\begin{Arabic}
\quranayah[18][27]
\end{Arabic}}
\flushleft{\begin{malayalam}
നിനക്കു ബോധനമായി ലഭിച്ച നിന്റെ നാഥന്റെ വേദപുസ്തകം നീ വായിച്ചുകേള്‍പ്പിക്കുക. അവന്റെ വചനങ്ങളില്‍ ഭേദഗതി വരുത്തുന്ന ആരുമില്ല. അവനല്ലാത്ത ഒരഭയകേന്ദ്രം കണ്ടെത്താനും നിനക്കാവില്ല.
\end{malayalam}}
\flushright{\begin{Arabic}
\quranayah[18][28]
\end{Arabic}}
\flushleft{\begin{malayalam}
തങ്ങളുടെ നാഥന്റെ പ്രീതി പ്രതീക്ഷിച്ച് രാവിലെയും വൈകുന്നേരവും അവനോട് പ്രാര്‍ഥിച്ചുകൊണ്ടിരിക്കുന്നവരോടൊപ്പം നീ നിന്റെ മനസ്സിനെ ഉറപ്പിച്ചുനിര്‍ത്തുക. ഇഹലോക ജീവിതത്തിന്റെ മോടി തേടി നിന്റെ കണ്ണുകള്‍ അവരില്‍നിന്നും തെറ്റിപ്പോവാതിരിക്കട്ടെ. നമ്മുടെ സ്മരണയെ സംബന്ധിച്ച് അശ്രദ്ധരാവുന്നവനെയും തന്നിഷ്ടത്തെ പിന്‍പറ്റുന്നവനെയും പരിധി ലംഘിച്ച് ജീവിക്കുന്നവനെയും നീ അനുസരിച്ചുപോകരുത്.
\end{malayalam}}
\flushright{\begin{Arabic}
\quranayah[18][29]
\end{Arabic}}
\flushleft{\begin{malayalam}
പറയുക: ഇത് നിങ്ങളുടെ നാഥനില്‍നിന്നുള്ള സത്യമാണ്. ഇഷ്ടമുള്ളവര്‍ക്ക് വിശ്വസിക്കാം, ഇഷ്ടമുള്ളവര്‍ക്ക് അവിശ്വസിക്കാം; അക്രമികള്‍ക്കു നാം നരകത്തീ തയ്യാറാക്കിവെച്ചിട്ടുണ്ട്. അതിന്റെ ജ്വാലകള്‍ അവരെ വലയം ചെയ്തുകഴിഞ്ഞു. അവിടെ അവര്‍ വെള്ളത്തിനു കേഴുകയാണെങ്കില്‍ അവര്‍ക്ക് കുടിക്കാന്‍ കിട്ടുക ഉരുകിയ ലോഹം പോലുള്ള പാനീയമായിരിക്കും. അതവരുടെ മുഖങ്ങളെ കരിച്ചുകളയും. അതൊരു നശിച്ച പാനീയം തന്നെ! അവിടം വളരെ ചീത്തയായ താവളമാണ്.
\end{malayalam}}
\flushright{\begin{Arabic}
\quranayah[18][30]
\end{Arabic}}
\flushleft{\begin{malayalam}
സത്യവിശ്വാസം സ്വീകരിക്കുകയും സല്‍ക്കര്‍മങ്ങള്‍ പ്രവര്‍ത്തിക്കുകയും ചെയ്തവരോ, തീര്‍ച്ചയായും അത്തരം സല്‍പ്രവൃത്തികള്‍ ചെയ്യുന്ന ആരുടെയും പ്രതിഫലം നാം പാഴാക്കുകയില്ല.
\end{malayalam}}
\flushright{\begin{Arabic}
\quranayah[18][31]
\end{Arabic}}
\flushleft{\begin{malayalam}
അവര്‍ക്ക് സ്ഥിരവാസത്തിനുള്ള സ്വര്‍ഗീയാരാമങ്ങളുണ്ട്. അവരുടെ താഴ്ഭാഗത്തൂടെ ആറുകളൊഴുകിക്കൊണ്ടിരിക്കും. അവിടെയവര്‍ സ്വര്‍ണവളകളണിയിക്കപ്പെടും. നേര്‍ത്തതും കനത്തതുമായ പച്ചപ്പട്ടുകളാണ് അവിടെയവര്‍ ധരിക്കുക. കട്ടിലുകളില്‍ ചാരിയിരുന്നാണ് അവര്‍ വിശ്രമിക്കുക. എത്ര മഹത്തായ പ്രതിഫലം! എത്ര നല്ല സങ്കേതം!
\end{malayalam}}
\flushright{\begin{Arabic}
\quranayah[18][32]
\end{Arabic}}
\flushleft{\begin{malayalam}
നീ അവര്‍ക്ക് രണ്ടാളുകളുടെ ഉദാഹരണം പറഞ്ഞുകൊടുക്കുക: അവരിലൊരാള്‍ക്ക് നാം രണ്ടു മുന്തിരിത്തോട്ടങ്ങള്‍ നല്‍കി. അവയ്ക്കു ചുറ്റും ഈന്തപ്പനകള്‍ വളര്‍ത്തി. അവയ്ക്കിടയില്‍ ധാന്യകൃഷിയിടവും ഉണ്ടാക്കി.
\end{malayalam}}
\flushright{\begin{Arabic}
\quranayah[18][33]
\end{Arabic}}
\flushleft{\begin{malayalam}
രണ്ടു തോട്ടങ്ങളും ധാരാളം വിളവുല്‍പാദിപ്പിച്ചു. അതിലൊരു കുറവും ഉണ്ടായില്ല. അവയ്ക്കിടയിലൂടെ നാം പുഴ ഒഴുക്കുകയും ചെയ്തു.
\end{malayalam}}
\flushright{\begin{Arabic}
\quranayah[18][34]
\end{Arabic}}
\flushleft{\begin{malayalam}
കര്‍ഷകന് നല്ല വരുമാനമുണ്ടായി. അപ്പോള്‍ അയാള്‍ തന്റെ കൂട്ടുകാരനോട് സംസാരിക്കവെ പറഞ്ഞു: "ഞാനാണ് നിന്നെക്കാള്‍ സമ്പത്തും സംഘബലവുമുള്ളവന്‍.”
\end{malayalam}}
\flushright{\begin{Arabic}
\quranayah[18][35]
\end{Arabic}}
\flushleft{\begin{malayalam}
അങ്ങനെ തന്നോടുതന്നെ അതിക്രമം ചെയ്തവനായി അയാള്‍ തന്റെ തോട്ടത്തില്‍ പ്രവേശിച്ചു. അയാള്‍ പറഞ്ഞു: "ഇതൊന്നും ഒരിക്കലും നശിച്ചുപോകുമെന്ന് ഞാന്‍ കരുതുന്നില്ല.
\end{malayalam}}
\flushright{\begin{Arabic}
\quranayah[18][36]
\end{Arabic}}
\flushleft{\begin{malayalam}
"അന്ത്യനാള്‍ വന്നെത്തുമെന്നും ഞാന്‍ കരുതുന്നില്ല. അഥവാ, എനിക്കെന്റെ നാഥന്റെ അടുത്തേക്ക് തിരിച്ചുചെല്ലേണ്ടി വന്നാല്‍ തന്നെ, അവിടെ ഇതിനെക്കാള്‍ മെച്ചപ്പെട്ട ഇടമെനിക്കു ലഭിക്കും.”
\end{malayalam}}
\flushright{\begin{Arabic}
\quranayah[18][37]
\end{Arabic}}
\flushleft{\begin{malayalam}
അവന്റെ കൂട്ടുകാരന്‍ ഇതിനെ എതിര്‍ത്തുകൊണ്ട് പറഞ്ഞു: "നിന്നെ മണ്ണില്‍നിന്നും പിന്നെ ബീജകണത്തില്‍നിന്നും സൃഷ്ടിക്കുകയും അങ്ങനെ ഒരു പൂര്‍ണമനുഷ്യനാക്കി രൂപപ്പെടുത്തുകയും ചെയ്ത നാഥനെയാണോ നീ തള്ളിപ്പറയുന്നത്?
\end{malayalam}}
\flushright{\begin{Arabic}
\quranayah[18][38]
\end{Arabic}}
\flushleft{\begin{malayalam}
"എന്നാല്‍ അവനാണ്; അഥവാ അല്ലാഹുവാണ് എന്റെ നാഥന്‍. ഞാന്‍ ആരെയും എന്റെ നാഥന്റെ പങ്കാളിയാക്കുകയില്ല.
\end{malayalam}}
\flushright{\begin{Arabic}
\quranayah[18][39]
\end{Arabic}}
\flushleft{\begin{malayalam}
"നീ നിന്റെ തോട്ടത്തില്‍ പ്രവേശിച്ചപ്പോള്‍ നിനക്കിങ്ങനെ പറഞ്ഞുകൂടായിരുന്നോ: “ഇത് അല്ലാഹു ഇച്ഛിച്ചതാണ്. അല്ലാഹുവെക്കൊണ്ടല്ലാതെ യാതൊരു ശക്തിയും സ്വാധീനവും ഇല്ല.” നിന്നെക്കാള്‍ സമ്പത്തും സന്താനങ്ങളും കുറഞ്ഞവനായി നീ എന്നെ കാണുന്നുവെങ്കില്‍;
\end{malayalam}}
\flushright{\begin{Arabic}
\quranayah[18][40]
\end{Arabic}}
\flushleft{\begin{malayalam}
"എന്റെ നാഥന്‍ എനിക്ക് നിന്റെ തോട്ടത്തെക്കാള്‍ നല്ലത് നല്‍കിയേക്കാം. നിന്റെ തോട്ടത്തിന്റെ നേരെ അവന്‍ മാനത്തുനിന്നു വല്ല വിപത്തുമയച്ചേക്കാം. അങ്ങനെ അത് തരിശായ ചതുപ്പുനിലമായേക്കാം. .
\end{malayalam}}
\flushright{\begin{Arabic}
\quranayah[18][41]
\end{Arabic}}
\flushleft{\begin{malayalam}
"അല്ലെങ്കില്‍ അതിലെ വെള്ളം പിന്നീടൊരിക്കലും നിനക്കു തിരിച്ചുകൊണ്ടുവരാനാവാത്ത വിധം വററിവരണ്ടെന്നും വരാം.”
\end{malayalam}}
\flushright{\begin{Arabic}
\quranayah[18][42]
\end{Arabic}}
\flushleft{\begin{malayalam}
അവസാനം അവന്റെ കായ്കനികള്‍ നാശത്തിനിരയായി. തോട്ടം പന്തലോടുകൂടി നിലംപൊത്തി. അതുകണ്ട് അയാള്‍ താനതില്‍ ചെലവഴിച്ചതിന്റെ പേരില്‍ ഖേദത്താല്‍ കൈമലര്‍ത്തി. അയാളിങ്ങനെ വിലപിച്ചു: "ഞാനെന്റെ നാഥനില്‍ ആരെയും പങ്ക് ചേര്‍ത്തില്ലായിരുന്നെങ്കില്‍ എത്ര നന്നായേനേ.”
\end{malayalam}}
\flushright{\begin{Arabic}
\quranayah[18][43]
\end{Arabic}}
\flushleft{\begin{malayalam}
അല്ലാഹുവെക്കൂടാതെ അയാളെ സഹായിക്കാന്‍ ആരുമുണ്ടായില്ല. ആ നാശത്തെ നേരിടാന്‍ അവനു കഴിഞ്ഞതുമില്ല.
\end{malayalam}}
\flushright{\begin{Arabic}
\quranayah[18][44]
\end{Arabic}}
\flushleft{\begin{malayalam}
അവിടെ രക്ഷാധികാരം സാക്ഷാല്‍ അല്ലാഹുവിന് മാത്രമാണ്. ഉത്തമമായ പ്രതിഫലം നല്‍കുന്നതവനാണ്. മെച്ചപ്പെട്ട പര്യവസാനത്തിലെത്തിക്കുന്നതും അവന്‍ തന്നെ.
\end{malayalam}}
\flushright{\begin{Arabic}
\quranayah[18][45]
\end{Arabic}}
\flushleft{\begin{malayalam}
ഇഹലോകജീവിതത്തിന്റെ ഉദാഹരണം നീ അവര്‍ക്ക് വിവരിച്ചുകൊടുക്കുക: നാം മാനത്തുനിന്ന് മഴ വീഴ്ത്തി. അതുവഴി സസ്യങ്ങള്‍ ഇടകലര്‍ന്നു വളര്‍ന്നു. താമസിയാതെ അതൊക്കെ കാറ്റില്‍ പറക്കുന്ന തുരുമ്പായിമാറി. അല്ലാഹു എല്ലാ കാര്യങ്ങള്‍ക്കും കഴിവുറ്റവനാണ്.
\end{malayalam}}
\flushright{\begin{Arabic}
\quranayah[18][46]
\end{Arabic}}
\flushleft{\begin{malayalam}
സമ്പത്തും സന്താനങ്ങളും ഐഹികജീവിതത്തിന്റെ അലങ്കാരമാണ്. എന്നും നിലനില്‍ക്കുന്ന സല്‍ക്കര്‍മങ്ങള്‍ക്കാണ് നിന്റെ നാഥന്റെയടുത്ത് ഉത്തമമായ പ്രതിഫലമുള്ളത്. നല്ല പ്രതീക്ഷ നല്‍കുന്നതും അതുതന്നെ.
\end{malayalam}}
\flushright{\begin{Arabic}
\quranayah[18][47]
\end{Arabic}}
\flushleft{\begin{malayalam}
നാം പര്‍വതങ്ങളെ ചലിപ്പിക്കുന്ന ദിവസത്തെ ഓര്‍ക്കുക. അപ്പോള്‍ ഭൂമി തെളിഞ്ഞ് തരിശായതായി നിനക്കു കാണാം. അന്ന് അവരെയൊക്കെയും നാം ഒരുമിച്ചുകൂട്ടും. അവരിലാരെയും ഒഴിവാക്കുകയില്ല.
\end{malayalam}}
\flushright{\begin{Arabic}
\quranayah[18][48]
\end{Arabic}}
\flushleft{\begin{malayalam}
നിന്റെ നാഥന്റെ മുന്നില്‍ അവരൊക്കെയും അണിയണിയായി നിര്‍ത്തപ്പെടും. അപ്പോഴവന്‍ പറയും: നിങ്ങളെ നാം ആദ്യതവണ സൃഷ്ടിച്ചപോലെ നിങ്ങളിതാ നമ്മുടെ അടുത്ത് വന്നിരിക്കുന്നു. ഇത്തരമൊരു സന്ദര്‍ഭം നിങ്ങള്‍ക്കു നാം ഉണ്ടാക്കുകയേയില്ല എന്നാണല്ലോ നിങ്ങള്‍ വാദിച്ചുകൊണ്ടിരുന്നത്.
\end{malayalam}}
\flushright{\begin{Arabic}
\quranayah[18][49]
\end{Arabic}}
\flushleft{\begin{malayalam}
കര്‍മപുസ്തകം നിങ്ങളുടെ മുന്നില്‍ വെക്കും. അതിലുള്ളവയെപ്പറ്റി പേടിച്ചരണ്ടവരായി പാപികളെ നീ കാണും. അവര്‍ പറയും: "അയ്യോ, ഞങ്ങള്‍ക്കു നാശം! ഇതെന്തൊരു കര്‍മരേഖ! ചെറുതും വലുതുമായ ഒന്നുംതന്നെ ഇത് വിട്ടുകളഞ്ഞിട്ടില്ലല്ലോ.” അവര്‍ പ്രവര്‍ത്തിച്ചതൊക്കെയും തങ്ങളുടെ മുന്നില്‍ വന്നെത്തിയതായി അവര്‍ കാണുന്നു. നിന്റെ നാഥന്‍ ആരോടും അനീതി കാണിക്കുകയില്ല.
\end{malayalam}}
\flushright{\begin{Arabic}
\quranayah[18][50]
\end{Arabic}}
\flushleft{\begin{malayalam}
നാം മലക്കുകളോട് പറഞ്ഞ സന്ദര്‍ഭം: "നിങ്ങള്‍ ആദമിന് സാഷ്ടാംഗം പ്രണമിക്കുക.” അവര്‍ പ്രണമിച്ചു; ഇബ്ലീസ് ഒഴികെ. അവന്‍ ജിന്നുകളില്‍പെട്ടവനായിരുന്നു. അവന്‍ തന്റെ നാഥന്റെ കല്‍പന ധിക്കരിച്ചു. എന്നിട്ടും നിങ്ങള്‍ എന്നെ വെടിഞ്ഞ് അവനെയും അവന്റെ സന്തതികളെയുമാണോ രക്ഷാധികാരികളാക്കുന്നത്? അവര്‍ നിങ്ങളുടെ ശത്രുക്കളാണ്. അതിക്രമികള്‍ അല്ലാഹുവിന് പകരം വെച്ചത് വളരെ ചീത്തതന്നെ.
\end{malayalam}}
\flushright{\begin{Arabic}
\quranayah[18][51]
\end{Arabic}}
\flushleft{\begin{malayalam}
ആകാശഭൂമികളുടെ സൃഷ്ടിപ്പിന് ഞാന്‍ അവരെ സാക്ഷികളാക്കിയിട്ടില്ല. അവരെ സൃഷ്ടിച്ചപ്പോഴും ഞാനങ്ങനെ ചെയ്തിട്ടില്ല. വഴിപിഴപ്പിക്കുന്നവരെ തുണയായി സ്വീകരിക്കുന്നവനല്ല ഞാന്‍.
\end{malayalam}}
\flushright{\begin{Arabic}
\quranayah[18][52]
\end{Arabic}}
\flushleft{\begin{malayalam}
“എന്റെ പങ്കാളികളായി നിങ്ങള്‍ സങ്കല്‍പിച്ചുവെച്ചിരുന്നവരെ വിളിച്ചുനോക്കൂ” എന്ന് അല്ലാഹു പറയുന്ന ദിനം. അന്ന് ഇവര്‍ അവരെ വിളിക്കും. എന്നാല്‍ അവര്‍ ഇവര്‍ക്ക് ഉത്തരം നല്‍കുന്നതല്ല. അവര്‍ക്കിടയില്‍ നാം ഒരു നാശക്കുഴിയൊരുക്കിയിരിക്കുന്നു.
\end{malayalam}}
\flushright{\begin{Arabic}
\quranayah[18][53]
\end{Arabic}}
\flushleft{\begin{malayalam}
അന്ന് കുറ്റവാളികള്‍ നരകം നേരില്‍ കാണും. തങ്ങളതില്‍ പതിക്കാന്‍ പോകയാണെന്ന് അവര്‍ മനസ്സിലാക്കും. അതില്‍നിന്ന് രക്ഷപ്പെടാനൊരു മാര്‍ഗവും അവര്‍ക്ക് കണ്ടെത്താനാവില്ല.
\end{malayalam}}
\flushright{\begin{Arabic}
\quranayah[18][54]
\end{Arabic}}
\flushleft{\begin{malayalam}
ഈ ഖുര്‍ആനില്‍ നാം നിരവധി ഉദാഹരണങ്ങള്‍ വിവിധ രീതികളില്‍ ജനങ്ങള്‍ക്ക് വിവരിച്ചുതന്നിരിക്കുന്നു. എന്നാല്‍ മനുഷ്യന്‍ അതിരറ്റ തര്‍ക്കപ്രകൃതക്കാരന്‍ തന്നെ.
\end{malayalam}}
\flushright{\begin{Arabic}
\quranayah[18][55]
\end{Arabic}}
\flushleft{\begin{malayalam}
നേര്‍വഴി വന്നെത്തിയപ്പോള്‍ അതില്‍ വിശ്വസിക്കുകയും തങ്ങളുടെ നാഥനോട് പാപമോചനം തേടുകയും ചെയ്യുന്നതില്‍നിന്ന് ജനത്തെ തടഞ്ഞത്, പൂര്‍വികരുടെ കാര്യത്തിലുണ്ടായ നടപടി തങ്ങളുടെ കാര്യത്തിലും ഉണ്ടാവണം; അഥവാ, ശിക്ഷ തങ്ങള്‍ നേരില്‍ കാണണം എന്ന അവരുടെ നിലപാടു മാത്രമാണ്.
\end{malayalam}}
\flushright{\begin{Arabic}
\quranayah[18][56]
\end{Arabic}}
\flushleft{\begin{malayalam}
ശുഭവാര്‍ത്ത അറിയിക്കുന്നവരും താക്കീതു നല്‍കുന്നവരുമായല്ലാതെ നാം ദൈവദൂതന്മാരെ അയച്ചിട്ടില്ല. സത്യനിഷേധികള്‍ മിഥ്യാവാദങ്ങളുമായി സത്യത്തെ തകര്‍ക്കാന്‍ തര്‍ക്കിച്ചുകൊണ്ടേയിരിക്കുന്നു. അവരെന്റെ വചനങ്ങളെയും അവര്‍ക്കു നല്‍കിയ താക്കീതുകളെയും പുച്ഛിച്ചു തള്ളുന്നു.
\end{malayalam}}
\flushright{\begin{Arabic}
\quranayah[18][57]
\end{Arabic}}
\flushleft{\begin{malayalam}
തന്റെ നാഥന്റെ വചനങ്ങള്‍ ഓര്‍മിപ്പിക്കുമ്പോള്‍ അതിനെ അവഗണിച്ചു തള്ളുകയും തന്റെ കൈകള്‍ നേരത്തെ ചെയ്തുവെച്ചത് മറന്നുകളയുകയും ചെയ്തവനെക്കാള്‍ കടുത്ത അക്രമി ആരുണ്ട്? അവര്‍ക്കു കാര്യം ഗ്രഹിക്കാനാവാത്ത വിധം അവരുടെ ഹൃദയങ്ങള്‍ക്കു നാം മൂടികളിട്ടിരിക്കുന്നു. അവരുടെ കാതുകളില്‍ അടപ്പിട്ടിരിക്കുന്നു. നീ അവരെ നേര്‍വഴിയിലേക്ക് എത്രതന്നെ വിളിച്ചാലും അവരൊരിക്കലും സന്മാര്‍ഗം സ്വീകരിക്കുകയില്ല.
\end{malayalam}}
\flushright{\begin{Arabic}
\quranayah[18][58]
\end{Arabic}}
\flushleft{\begin{malayalam}
നിന്റെ നാഥന്‍ ഏറെ പൊറുക്കുന്നവനും കരുണാമയനുമാണ്. അവര്‍ ചെയ്തുകൂട്ടിയതിന്റെ പേരില്‍ അവരെയവന്‍ പിടികൂടുകയാണെങ്കില്‍ അവര്‍ക്കവന്‍ വളരെ പെട്ടെന്നു തന്നെ ശിക്ഷ നല്‍കുമായിരുന്നു. എന്നാല്‍ അവര്‍ക്കൊരു നിശ്ചിത കാലാവധിയുണ്ട്. അതിനെ മറികടക്കാന്‍ ഒരഭയകേന്ദ്രവും കണ്ടെത്താനവര്‍ക്കാവില്ല.
\end{malayalam}}
\flushright{\begin{Arabic}
\quranayah[18][59]
\end{Arabic}}
\flushleft{\begin{malayalam}
ആ നാടുകള്‍ അതിക്രമം കാണിച്ചപ്പോള്‍ നാം അവയെ നശിപ്പിച്ചു. അവയുടെ നാശത്തിനു നാം നിശ്ചിത കാലപരിധി വെച്ചിട്ടുണ്ടായിരുന്നു.
\end{malayalam}}
\flushright{\begin{Arabic}
\quranayah[18][60]
\end{Arabic}}
\flushleft{\begin{malayalam}
മൂസ തന്റെ ഭൃത്യനോട് പറഞ്ഞു: "രണ്ടു നദികളുടെ സംഗമസ്ഥാനത്തെത്തുംവരെ ഞാന്‍ ഈ യാത്ര തുടര്‍ന്നുകൊണ്ടേയിരിക്കും. അല്ലെങ്കില്‍ അളവറ്റ കാലം ഞാന്‍ സഞ്ചരിച്ചുകൊണ്ടേയിരിക്കും.”
\end{malayalam}}
\flushright{\begin{Arabic}
\quranayah[18][61]
\end{Arabic}}
\flushleft{\begin{malayalam}
അങ്ങനെ അവര്‍ ആ സംഗമസ്ഥാനത്തെത്തിയപ്പോള്‍ ഇരുവരും തങ്ങളുടെ മത്സ്യത്തെ ക്കുറിച്ചോര്‍ത്തില്ല. മത്സ്യം പുറത്തുകടന്ന് തുരങ്കത്തിലൂടെയെന്നവണ്ണം വെള്ളത്തില്‍ പോയി.
\end{malayalam}}
\flushright{\begin{Arabic}
\quranayah[18][62]
\end{Arabic}}
\flushleft{\begin{malayalam}
അങ്ങനെയവര്‍ അവിടംവിട്ട് മുന്നോട്ട് പോയി. അപ്പോള്‍ മൂസ തന്റെ ഭൃത്യനോട് പറഞ്ഞു: "നമ്മുടെ പ്രാതല്‍ കൊണ്ടുവരൂ! ഈ യാത്രകാരണം നാം നന്നെ ക്ഷീണിച്ചിരിക്കുന്നു.”
\end{malayalam}}
\flushright{\begin{Arabic}
\quranayah[18][63]
\end{Arabic}}
\flushleft{\begin{malayalam}
അയാള്‍ പറഞ്ഞു: "അങ്ങ് കണ്ടോ? നാം ആ പാറക്കല്ലില്‍ അഭയം തേടിയ നേരത്ത് ഞാന്‍ ആ മത്സ്യത്തെ പറ്റെയങ്ങ് മറന്നുപോയി. അക്കാര്യം പറയാന്‍ എന്നെ മറപ്പിച്ചത് പിശാചല്ലാതാരുമല്ല. മത്സ്യം കടലില്‍ അദ്ഭുതകരമാം വിധം അതിന്റെ വഴി തേടുകയും ചെയ്തു.”
\end{malayalam}}
\flushright{\begin{Arabic}
\quranayah[18][64]
\end{Arabic}}
\flushleft{\begin{malayalam}
മൂസ പറഞ്ഞു: "അതു തന്നെയാണ് നാം തേടിക്കൊണ്ടിരുന്നത്.” അങ്ങനെ അവരിരുവരും തങ്ങളുടെ കാലടിപ്പാടുകള്‍ നോക്കി തിരിച്ചുനടന്നു.
\end{malayalam}}
\flushright{\begin{Arabic}
\quranayah[18][65]
\end{Arabic}}
\flushleft{\begin{malayalam}
അപ്പോള്‍ അവിടെയവര്‍ നമ്മുടെ ദാസന്മാരിലൊരാളെ കണ്ടെത്തി. അദ്ദേഹത്തിന് നാം നമ്മുടെ കാരുണ്യം നല്‍കിയിരുന്നു. നമ്മുടെ സവിശേഷ ജ്ഞാനം പഠിപ്പിക്കുകയും ചെയ്തിരുന്നു.
\end{malayalam}}
\flushright{\begin{Arabic}
\quranayah[18][66]
\end{Arabic}}
\flushleft{\begin{malayalam}
മൂസ അദ്ദേഹത്തോടു ചോദിച്ചു: "ഞാന്‍ താങ്കളെ പിന്തുടരട്ടെയോ? താങ്കള്‍ക്കു കൈവന്ന സവിശേഷ ജ്ഞാനത്തില്‍നിന്ന് എന്നെയും പഠിപ്പിക്കുമോ?”
\end{malayalam}}
\flushright{\begin{Arabic}
\quranayah[18][67]
\end{Arabic}}
\flushleft{\begin{malayalam}
അദ്ദേഹം പറഞ്ഞു: "താങ്കള്‍ക്ക് എന്നോടൊപ്പം ക്ഷമിച്ചുകഴിയാന്‍ സാധിക്കുകയില്ല.
\end{malayalam}}
\flushright{\begin{Arabic}
\quranayah[18][68]
\end{Arabic}}
\flushleft{\begin{malayalam}
"അകംപൊരുളറിഞ്ഞിട്ടില്ലാത്ത കാര്യത്തില്‍ താങ്കളെങ്ങനെ ക്ഷമിച്ചിരിക്കും”
\end{malayalam}}
\flushright{\begin{Arabic}
\quranayah[18][69]
\end{Arabic}}
\flushleft{\begin{malayalam}
മൂസ പറഞ്ഞു: "അല്ലാഹു ഇച്ഛിച്ചെങ്കില്‍ താങ്കള്‍ക്കെന്നെ എല്ലാം ക്ഷമിക്കുന്നവനായി കണ്ടെത്താം. ഞാന്‍ താങ്കളുടെ കല്‍പനയൊന്നും ധിക്കരിക്കുകയില്ല.”
\end{malayalam}}
\flushright{\begin{Arabic}
\quranayah[18][70]
\end{Arabic}}
\flushleft{\begin{malayalam}
അദ്ദേഹം പറഞ്ഞു: "താങ്കള്‍ എന്നെ അനുഗമിക്കുന്നുവെങ്കില്‍ ഒരു കാര്യത്തെക്കുറിച്ചും ഞാനത് വിശദീകരിച്ചുതരുന്നത് വരെ എന്നോട് ചോദിക്കരുത്.”
\end{malayalam}}
\flushright{\begin{Arabic}
\quranayah[18][71]
\end{Arabic}}
\flushleft{\begin{malayalam}
അങ്ങനെ അവരിരുവരും യാത്രയായി. അവര്‍ ഒരു കപ്പലില്‍ കയറിയപ്പോള്‍ അദ്ദേഹം ആ കപ്പലിന് ഒരു ദ്വാരമുണ്ടാക്കി. മൂസ ചോദിച്ചു: "താങ്കളെന്തിനാണ് കപ്പലിന് ദ്വാരമുണ്ടാക്കുന്നത്? ഇതിലുള്ളവരെയൊക്കെ മുക്കിക്കൊല്ലാനാണോ? താങ്കള്‍ ഇച്ചെയ്തത് ഗുരുതരമായ കാര്യം തന്നെ.”
\end{malayalam}}
\flushright{\begin{Arabic}
\quranayah[18][72]
\end{Arabic}}
\flushleft{\begin{malayalam}
അദ്ദേഹം പറഞ്ഞു: "അപ്പോഴേ ഞാന്‍ പറഞ്ഞിരുന്നില്ലേ; താങ്കള്‍ക്കെന്റെ കൂടെ ക്ഷമിച്ചുകഴിയാന്‍ സാധ്യമല്ലെന്ന്?”
\end{malayalam}}
\flushright{\begin{Arabic}
\quranayah[18][73]
\end{Arabic}}
\flushleft{\begin{malayalam}
മൂസ പറഞ്ഞു: "ഞാന്‍ മറന്നുപോയതാണ്. ഇതിന്റെ പേരില്‍ താങ്കളെന്നെ പിടികൂടരുത്! എന്റെ കാര്യത്തില്‍ പ്രയാസകരമായ ഒന്നിനും താങ്കള്‍ നിര്‍ബന്ധിക്കരുത്.”
\end{malayalam}}
\flushright{\begin{Arabic}
\quranayah[18][74]
\end{Arabic}}
\flushleft{\begin{malayalam}
അവര്‍ യാത്ര തുടര്‍ന്നു. വഴിയില്‍ അവരൊരു ബാലനെ കണ്ടുമുട്ടി. അദ്ദേഹം അവനെ കൊന്നുകളഞ്ഞു. മൂസ പറഞ്ഞു: "താങ്കളെന്തിനാണ് ഒരു നിരപരാധിയെ കൊന്നത്? അതും മറ്റൊരാളെ കൊന്നതിന് പകരമായല്ലാതെ. ഉറപ്പായും താങ്കള്‍ ഇച്ചെയ്തത് കടുത്ത ക്രൂരത തന്നെ.”
\end{malayalam}}
\flushright{\begin{Arabic}
\quranayah[18][75]
\end{Arabic}}
\flushleft{\begin{malayalam}
അദ്ദേഹം പറഞ്ഞു: "ഞാന്‍ താങ്കളോട് പറഞ്ഞിരുന്നില്ലേ; താങ്കള്‍ക്കെന്റെ കൂടെ ക്ഷമിച്ചു കഴിയാന്‍ സാധ്യമല്ലെന്ന്?”
\end{malayalam}}
\flushright{\begin{Arabic}
\quranayah[18][76]
\end{Arabic}}
\flushleft{\begin{malayalam}
മൂസ പറഞ്ഞു: "ഇനിയും ഞാന്‍ താങ്കളോട് എന്തെങ്കിലും ചോദിക്കുകയാണെങ്കില്‍ അന്നേരം താങ്കളെന്നെ കൂടെ കൂട്ടേണ്ടതില്ല. താങ്കള്‍ക്കതിന് എന്നില്‍നിന്ന് വേണ്ടത്ര കാരണം കിട്ടിക്കഴിഞ്ഞു.”
\end{malayalam}}
\flushright{\begin{Arabic}
\quranayah[18][77]
\end{Arabic}}
\flushleft{\begin{malayalam}
പിന്നെയും അവരിരുവരും മുന്നോട്ടുനീങ്ങി. അങ്ങനെ ഒരു നാട്ടിലെത്തിയപ്പോള്‍ ആ നാട്ടുകാരോട് അവര്‍ അന്നം ചോദിച്ചു. എന്നാല്‍ അവര്‍ക്ക് ആതിഥ്യം നല്‍കാന്‍ നാട്ടുകാര്‍ സന്നദ്ധരായില്ല. അവിടെ പൊളിഞ്ഞുവീഴാറായ ഒരു മതില്‍ അവര്‍ കണ്ടു. അദ്ദേഹം അതു നേരെയാക്കി. മൂസ പറഞ്ഞു: "താങ്കള്‍ക്കു വേണമെങ്കില്‍ ഇതിന് പ്രതിഫലം വാങ്ങാമായിരുന്നു.”
\end{malayalam}}
\flushright{\begin{Arabic}
\quranayah[18][78]
\end{Arabic}}
\flushleft{\begin{malayalam}
അദ്ദേഹം പറഞ്ഞു: "ഇത് ഞാനും താങ്കളും തമ്മില്‍ വേര്‍പിരിയാനുള്ള അവസരമാണ്. ഇനി താങ്കള്‍ക്ക് ക്ഷമിക്കാന്‍ പറ്റാതിരുന്ന കാര്യങ്ങളുടെ പൊരുള്‍ ഞാന്‍ വിശദീകരിച്ചുതരാം:
\end{malayalam}}
\flushright{\begin{Arabic}
\quranayah[18][79]
\end{Arabic}}
\flushleft{\begin{malayalam}
"ആ കപ്പലില്ലേ; അത് കടലില്‍ കഠിനാധ്വാനം ചെയ്തുകഴിയുന്ന ഏതാനും പാവങ്ങളുടേതാണ്. അതിനാല്‍ ഞാനത് കേടുവരുത്തണമെന്ന് കരുതി. കാരണം അവര്‍ക്ക് മുന്നില്‍ എല്ലാ നല്ല കപ്പലും ബലാല്‍ക്കാരം പിടിച്ചെടുക്കുന്ന ഒരു രാജാവുണ്ടായിരുന്നു.
\end{malayalam}}
\flushright{\begin{Arabic}
\quranayah[18][80]
\end{Arabic}}
\flushleft{\begin{malayalam}
"ആ ബാലന്റെ കാര്യമിതാണ്: അവന്റെ മാതാപിതാക്കള്‍ സത്യവിശ്വാസികളായിരുന്നു. എന്നാല്‍ ബാലന്‍ അവരെ അതിക്രമത്തിനും സത്യനിഷേധത്തിനും നിര്‍ബന്ധിതരാക്കുമെന്ന് നാം ഭയപ്പെട്ടു.
\end{malayalam}}
\flushright{\begin{Arabic}
\quranayah[18][81]
\end{Arabic}}
\flushleft{\begin{malayalam}
"അവരുടെ നാഥന്‍ അവനുപകരം അവനെക്കാള്‍ സദാചാര ശുദ്ധിയുള്ളവനും കുടുംബത്തോടു കൂടുതല്‍ അടുത്ത് ബന്ധപ്പെടുന്നവനുമായ ഒരു മകനെ നല്‍കണമെന്ന് നാം ആഗ്രഹിച്ചു.
\end{malayalam}}
\flushright{\begin{Arabic}
\quranayah[18][82]
\end{Arabic}}
\flushleft{\begin{malayalam}
"പിന്നെ ആ മതിലിന്റെ കാര്യം: അത് ആ പട്ടണത്തിലെ രണ്ട് അനാഥക്കുട്ടികളുടേതാണ്. അതിനടിയില്‍ അവര്‍ക്കായി കരുതിവെച്ച ഒരു നിധിയുണ്ട്. അവരുടെ പിതാവ് നല്ലൊരു മനുഷ്യനായിരുന്നു. അതിനാല്‍ അവരിരുവരും പ്രായപൂര്‍ത്തിയെത്തി തങ്ങളുടെ നിധി പുറത്തെടുക്കണമെന്ന് നിന്റെ നാഥന്‍ ആഗ്രഹിച്ചു. ഇതൊക്കെയും നിന്റെ നാഥന്റെ അനുഗ്രഹമത്രെ. ഞാനെന്റെ സ്വന്തം ഹിതമനുസരിച്ച് ചെയ്തതല്ല ഇതൊന്നും. താങ്കള്‍ക്കു ക്ഷമിക്കാന്‍ കഴിയാതിരുന്ന കാര്യങ്ങളുടെ പൊരുളിതാണ്.”
\end{malayalam}}
\flushright{\begin{Arabic}
\quranayah[18][83]
\end{Arabic}}
\flushleft{\begin{malayalam}
അവര്‍ നിന്നോട് ദുല്‍ഖര്‍നൈനിയെക്കുറിച്ചു ചോദിക്കുന്നു. പറയുക: "അദ്ദേഹത്തെ സംബന്ധിച്ച വിവരം ഞാന്‍ നിങ്ങളെ വായിച്ചുകേള്‍പ്പിക്കാം.”
\end{malayalam}}
\flushright{\begin{Arabic}
\quranayah[18][84]
\end{Arabic}}
\flushleft{\begin{malayalam}
നാം അദ്ദേഹത്തിന് ഭൂമിയില്‍ അധികാരം നല്‍കി. സകലവിധ സൌകര്യങ്ങളും ചെയ്തുകൊടുത്തു.
\end{malayalam}}
\flushright{\begin{Arabic}
\quranayah[18][85]
\end{Arabic}}
\flushleft{\begin{malayalam}
പിന്നെ അദ്ദേഹം ഒരു വഴിക്ക് യാത്ര തിരിച്ചു.
\end{malayalam}}
\flushright{\begin{Arabic}
\quranayah[18][86]
\end{Arabic}}
\flushleft{\begin{malayalam}
അങ്ങനെ സൂര്യാസ്തമയ സ്ഥാനത്തെത്തിയപ്പോള്‍ ചേറു നിറഞ്ഞ ജലാശയത്തില്‍ സൂര്യന്‍ മറഞ്ഞുപോകുന്നത് അദ്ദേഹം കണ്ടു. അതിനടുത്ത് ഒരു ജനവിഭാഗത്തെയും അദ്ദേഹം കണ്ടെത്തി. നാം പറഞ്ഞു: "ഓ, ദുല്‍ഖര്‍ നൈന്‍! വേണമെങ്കില്‍ നിനക്കിവരെ ശിക്ഷിക്കാം. അല്ലെങ്കില്‍ ഇവരില്‍ നന്മ ചൊരിയാം.”
\end{malayalam}}
\flushright{\begin{Arabic}
\quranayah[18][87]
\end{Arabic}}
\flushleft{\begin{malayalam}
ദുല്‍ഖര്‍നൈന്‍ പറഞ്ഞു: "അക്രമം പ്രവര്‍ത്തിക്കുന്നവനെ നാം ശിക്ഷിക്കും. പിന്നീട് അവന്‍ തന്റെ നാഥനിലേക്ക് മടക്കപ്പെടും. അപ്പോള്‍ അവന്റെ നാഥന്‍ അവന് കൂടുതല്‍ കടുത്തശിക്ഷ നല്‍കും.”
\end{malayalam}}
\flushright{\begin{Arabic}
\quranayah[18][88]
\end{Arabic}}
\flushleft{\begin{malayalam}
എന്നാല്‍ സത്യവിശ്വാസം സ്വീകരിക്കുകയും സല്‍ക്കര്‍മങ്ങള്‍ പ്രവര്‍ത്തിക്കുകയും ചെയ്തവന്ന് അത്യുത്തമമായ പ്രതിഫലമുണ്ട്. അവനു നാം നല്‍കുന്ന കല്‍പന ഏറെ എളുപ്പമുള്ളതായിരിക്കും.
\end{malayalam}}
\flushright{\begin{Arabic}
\quranayah[18][89]
\end{Arabic}}
\flushleft{\begin{malayalam}
പിന്നീട് അദ്ദേഹം മറ്റൊരു പാത പിന്തുടര്‍ന്നു.
\end{malayalam}}
\flushright{\begin{Arabic}
\quranayah[18][90]
\end{Arabic}}
\flushleft{\begin{malayalam}
അങ്ങനെ സൂര്യോദയ സ്ഥാനത്തെത്തിയപ്പോള്‍ അത് ഒരു ജനതയുടെ മേല്‍ ഉദിച്ചുയരുന്നതായി അദ്ദേഹം കണ്ടു. സൂര്യന്നും അവര്‍ക്കുമിടയില്‍ ഒരു മറയും നാം ഉണ്ടാക്കിയിട്ടില്ല.
\end{malayalam}}
\flushright{\begin{Arabic}
\quranayah[18][91]
\end{Arabic}}
\flushleft{\begin{malayalam}
അപ്രകാരം ദുല്‍ഖര്‍നൈനിയുടെ വശമുള്ളതെന്താണെന്നത് സംബന്ധിച്ച സൂക്ഷ്മമായ അറിവ് നമുക്കുണ്ടായിരുന്നു.
\end{malayalam}}
\flushright{\begin{Arabic}
\quranayah[18][92]
\end{Arabic}}
\flushleft{\begin{malayalam}
പിന്നെ അദ്ദേഹം വേറൊരു വഴിയിലൂടെ സഞ്ചരിച്ചു.
\end{malayalam}}
\flushright{\begin{Arabic}
\quranayah[18][93]
\end{Arabic}}
\flushleft{\begin{malayalam}
അങ്ങനെ രണ്ടു മലനിരകള്‍ക്കിടയിലെത്തിയപ്പോള്‍ അദ്ദേഹം അവയ്ക്കടുത്തായി വേറൊരു ജനവിഭാഗത്തെ കണ്ടെത്തി. പറയുന്നതൊന്നും മനസ്സിലാക്കാനാവാത്ത ജനം!
\end{malayalam}}
\flushright{\begin{Arabic}
\quranayah[18][94]
\end{Arabic}}
\flushleft{\begin{malayalam}
അവര്‍ പറഞ്ഞു: "അല്ലയോ ദുല്‍ഖര്‍നൈന്‍; യഅ്ജൂജും മഅ്ജൂജും നാട്ടില്‍ നാശമുണ്ടാക്കിക്കൊണ്ടിരിക്കുകയാണ്. അങ്ങ് അവര്‍ക്കും ഞങ്ങള്‍ക്കുമിടയില്‍ ഒരു ഭിത്തിയുണ്ടാക്കിത്തരണം. ആ വ്യവസ്ഥയില്‍ ഞങ്ങള്‍ അങ്ങയ്ക്ക് നികുതി നിശ്ചയിച്ചു തരട്ടെയോ?”
\end{malayalam}}
\flushright{\begin{Arabic}
\quranayah[18][95]
\end{Arabic}}
\flushleft{\begin{malayalam}
അദ്ദേഹം പറഞ്ഞു: "എന്റെ നാഥന്‍ എനിക്ക് അധീനപ്പെടുത്തിത്തന്നത് അതിനെക്കാളെല്ലാം മെച്ചപ്പെട്ടതാണ്. അതിനാല്‍ നിങ്ങളെന്നെ സഹായിക്കേണ്ടത് ശാരീരികാധ്വാനംകൊണ്ടാണ്. നിങ്ങള്‍ക്കും അവര്‍ക്കുമിടയില്‍ ഞാനൊരു ഭിത്തി ഉണ്ടാക്കിത്തരാം.
\end{malayalam}}
\flushright{\begin{Arabic}
\quranayah[18][96]
\end{Arabic}}
\flushleft{\begin{malayalam}
"എനിക്കു നിങ്ങള്‍ ഇരുമ്പുകട്ടികള്‍ കൊണ്ടുവന്നു തരിക.” അങ്ങനെ രണ്ടു മലകള്‍ക്കിടയിലെ വിടവ് നികത്തി നിരത്തിയപ്പോള്‍ അദ്ദേഹം പറഞ്ഞു: "നിങ്ങള്‍ കാറ്റ് ഊതുക.” അതോടെ ഇരുമ്പുഭിത്തി പഴുത്തു തീപോലെയായി. അപ്പോള്‍ അദ്ദേഹം കല്‍പിച്ചു: "നിങ്ങളെനിക്ക് ഉരുക്കിയ ചെമ്പുകൊണ്ടുവന്നു തരൂ! ഞാനത് ഇതിന്മേല്‍ ഒഴിക്കട്ടെ.”
\end{malayalam}}
\flushright{\begin{Arabic}
\quranayah[18][97]
\end{Arabic}}
\flushleft{\begin{malayalam}
പിന്നെ യഅ്ജൂജൂ മഅ്ജൂജുകള്‍ക്ക് അത് കയറി മറിയാന്‍ കഴിഞ്ഞിരുന്നില്ല. അതിന് തുളയുണ്ടാക്കാനും അവര്‍ക്കായില്ല.
\end{malayalam}}
\flushright{\begin{Arabic}
\quranayah[18][98]
\end{Arabic}}
\flushleft{\begin{malayalam}
ദുല്‍ഖര്‍നൈന്‍ പറഞ്ഞു: "ഇതെന്റെ നാഥന്റെ കാരുണ്യമാണ്. എന്നാല്‍ എന്റെ നാഥന്റെ വാഗ്ദത്തസമയം വന്നെത്തിയാല്‍ അവനതിനെ തകര്‍ത്ത് നിരപ്പാക്കും. എന്റെ നാഥന്റെ വാഗ്ദാനം തീര്‍ത്തും സത്യമാണ്.”
\end{malayalam}}
\flushright{\begin{Arabic}
\quranayah[18][99]
\end{Arabic}}
\flushleft{\begin{malayalam}
അന്ന് അവരില്‍ ചിലരെ മറ്റു ചിലര്‍ക്കെതിരെ തിരമാലകള്‍ കണക്കെ ഇരച്ചുവരുന്നവരാക്കും. പിന്നെ കാഹളത്തില്‍ ഊതും. അങ്ങനെ നാം മുഴുവനാളുകളെയും ഒരിടത്തൊരുമിച്ചുകൂട്ടും.
\end{malayalam}}
\flushright{\begin{Arabic}
\quranayah[18][100]
\end{Arabic}}
\flushleft{\begin{malayalam}
അന്ന് സത്യനിഷേധികള്‍ക്ക് നാം നരകത്തെ ശരിയാംവിധം നേര്‍ക്കുനേരെ കാണിച്ചുകൊടുക്കും.
\end{malayalam}}
\flushright{\begin{Arabic}
\quranayah[18][101]
\end{Arabic}}
\flushleft{\begin{malayalam}
അവരുടെ കണ്ണുകള്‍ക്ക് എന്റെ സന്ദേശത്തിന്റെ മുന്നില്‍ മറയുണ്ടായിരുന്നു. അവര്‍ക്കത് കേട്ടുമനസ്സിലാക്കാന്‍ കഴിഞ്ഞിരുന്നില്ല.
\end{malayalam}}
\flushright{\begin{Arabic}
\quranayah[18][102]
\end{Arabic}}
\flushleft{\begin{malayalam}
എന്നെ വെടിഞ്ഞ് എന്റെ ദാസന്മാരെ തങ്ങളുടെ രക്ഷകരാക്കാമെന്ന് സത്യനിഷേധികള്‍ കരുതുന്നുണ്ടോ? എന്നാല്‍ സംശയം വേണ്ട; സത്യനിഷേധികളെ സല്‍ക്കരിക്കാന്‍ നാം നരകത്തീ ഒരുക്കിവെച്ചിട്ടുണ്ട്.
\end{malayalam}}
\flushright{\begin{Arabic}
\quranayah[18][103]
\end{Arabic}}
\flushleft{\begin{malayalam}
പറയുക: തങ്ങളുടെ കര്‍മങ്ങള്‍ തീര്‍ത്തും നഷ്ടപ്പെട്ടവരായി മാറിയവരാരെന്ന് ഞാന്‍ നിങ്ങളെ അറിയിച്ചുതരട്ടെയോ?
\end{malayalam}}
\flushright{\begin{Arabic}
\quranayah[18][104]
\end{Arabic}}
\flushleft{\begin{malayalam}
ഇഹലോകജീവിതത്തില്‍ തങ്ങളുടെ പ്രവര്‍ത്തനങ്ങളൊക്കെ പിഴച്ചു പോയവരാണവര്‍. അതോടൊപ്പം തങ്ങള്‍ ചെയ്യുന്നതെല്ലാം നല്ലതാണെന്ന് കരുതുന്നവരും.
\end{malayalam}}
\flushright{\begin{Arabic}
\quranayah[18][105]
\end{Arabic}}
\flushleft{\begin{malayalam}
തങ്ങളുടെ നാഥന്റെ വചനങ്ങളെയും അവനുമായി കണ്ടുമുട്ടുമെന്നതിനെയും കള്ളമാക്കി തള്ളിയവരാണവര്‍. അതിനാല്‍ അവരുടെ പ്രവര്‍ത്തനങ്ങള്‍ പാഴായിരിക്കുന്നു. ഉയിര്‍ത്തെഴുന്നേല്‍പുനാളില്‍ നാം അവയ്ക്ക് ഒട്ടും പരിഗണന കല്‍പിക്കുകയില്ല.
\end{malayalam}}
\flushright{\begin{Arabic}
\quranayah[18][106]
\end{Arabic}}
\flushleft{\begin{malayalam}
അതാണ് അവര്‍ക്കുള്ള പ്രതിഫലം, നരകം; സത്യത്തെ നിഷേധിക്കുകയും നമ്മുടെ പ്രമാണങ്ങളെയും പ്രവാചകന്മാരെയും പുച്ഛിക്കുകയും ചെയ്തതിനുള്ള ശിക്ഷ!
\end{malayalam}}
\flushright{\begin{Arabic}
\quranayah[18][107]
\end{Arabic}}
\flushleft{\begin{malayalam}
സത്യവിശ്വാസം സ്വീകരിക്കുകയും സല്‍ക്കര്‍മങ്ങള്‍ പ്രവര്‍ത്തിക്കുകയും ചെയ്തവര്‍ക്ക് സല്‍ക്കാര വിഭവമായി സ്വര്‍ഗീയാരാമങ്ങളാണുണ്ടാവുക.
\end{malayalam}}
\flushright{\begin{Arabic}
\quranayah[18][108]
\end{Arabic}}
\flushleft{\begin{malayalam}
അവരവിടെ സ്ഥിരവാസികളായിരിക്കും. അവിടംവിട്ട് പോകാന്‍ അവരാഗ്രഹിക്കുകയില്ല.
\end{malayalam}}
\flushright{\begin{Arabic}
\quranayah[18][109]
\end{Arabic}}
\flushleft{\begin{malayalam}
പറയുക: സമുദ്രം എന്റെ നാഥന്റെ വചനങ്ങള്‍ കുറിക്കാനുള്ള മഷിയാവുകയാണെങ്കില്‍ എന്റെ നാഥന്റെ വചനങ്ങള്‍ തീരും മുമ്പെ തീര്‍ച്ചയായും അത് തീര്‍ന്നുപോകുമായിരുന്നു. അത്രയും കൂടി സമുദ്രജലം നാം സഹായത്തിനായി വേറെ കൊണ്ടുവന്നാലും ശരി!
\end{malayalam}}
\flushright{\begin{Arabic}
\quranayah[18][110]
\end{Arabic}}
\flushleft{\begin{malayalam}
പറയുക: ഞാന്‍ നിങ്ങളെപ്പോലുള്ള ഒരു മനുഷ്യന്‍ മാത്രമാണ്. നിങ്ങളുടെ ദൈവം ഏകദൈവം മാത്രമാണെന്ന് എനിക്ക് ദിവ്യബോധനം ലഭിക്കുന്നുണ്ട്. അതിനാല്‍ ആരെങ്കിലും തന്റെ നാഥനുമായി കണ്ടുമുട്ടണമെന്ന് ആഗ്രഹിക്കുന്നുവെങ്കില്‍ അവന്‍ സല്‍ക്കര്‍മങ്ങള്‍ ചെയ്തുകൊള്ളട്ടെ. തന്റെ നാഥനെ വഴിപ്പെടുന്ന കാര്യത്തില്‍ ആരെയും പങ്കുചേര്‍ക്കാതിരിക്കട്ടെ.
\end{malayalam}}
\chapter{\textmalayalam{മര്‍യം}}
\begin{Arabic}
\Huge{\centerline{\basmalah}}\end{Arabic}
\flushright{\begin{Arabic}
\quranayah[19][1]
\end{Arabic}}
\flushleft{\begin{malayalam}
കാഫ്-ഹാ-യാ-ഐന്‍-സ്വാദ്.
\end{malayalam}}
\flushright{\begin{Arabic}
\quranayah[19][2]
\end{Arabic}}
\flushleft{\begin{malayalam}
നിന്റെ നാഥന്‍ തന്റെ ദാസന്‍ സകരിയ്യയോടു കാണിച്ച കാരുണ്യത്തെ സംബന്ധിച്ച വിവരണമാണിത്.
\end{malayalam}}
\flushright{\begin{Arabic}
\quranayah[19][3]
\end{Arabic}}
\flushleft{\begin{malayalam}
അദ്ദേഹം തന്റെ നാഥനെ പതുക്കെ വിളിച്ചു പ്രാര്‍ഥിച്ച സന്ദര്‍ഭം.
\end{malayalam}}
\flushright{\begin{Arabic}
\quranayah[19][4]
\end{Arabic}}
\flushleft{\begin{malayalam}
അദ്ദേഹം പറഞ്ഞു: "എന്റെ നാഥാ! എന്റെ എല്ലുകള്‍ ദുര്‍ബലമായിരിക്കുന്നു. എന്റെ തല നരച്ചു തിളങ്ങുന്നതുമായിരിക്കുന്നു. നാഥാ; ഞാന്‍ നിന്നോട് പ്രാര്‍ഥിച്ചതൊന്നും നടക്കാതിരുന്നിട്ടില്ല.
\end{malayalam}}
\flushright{\begin{Arabic}
\quranayah[19][5]
\end{Arabic}}
\flushleft{\begin{malayalam}
"എനിക്കു പിറകെ വരാനിരിക്കുന്ന ബന്ധുക്കളെയോര്‍ത്ത് ഞാന്‍ ഭയപ്പെടുന്നു. എന്റെ ഭാര്യ വന്ധ്യയാണ്. അതിനാല്‍ നിന്റെ കാരുണ്യത്താല്‍ എനിക്കൊരു പിന്‍ഗാമിയെ പ്രദാനം ചെയ്യണമേ!
\end{malayalam}}
\flushright{\begin{Arabic}
\quranayah[19][6]
\end{Arabic}}
\flushleft{\begin{malayalam}
"അവനെന്റെ അനന്തരാവകാശിയാകണം. യഅ്ഖൂബ് കുടുംബത്തിന്റെയും പിന്മുറക്കാരനാകണം. എന്റെ നാഥാ, നീ അവനെ നിനക്കിഷ്ടപ്പെട്ടവനാക്കേണമേ.”
\end{malayalam}}
\flushright{\begin{Arabic}
\quranayah[19][7]
\end{Arabic}}
\flushleft{\begin{malayalam}
"സകരിയ്യാ, നിശ്ചയമായും നിന്നെയിതാ നാം ഒരു പുത്രനെ സംബന്ധിച്ച ശുഭവാര്‍ത്ത അറിയിക്കുന്നു. അവന്റെ പേര് യഹ്യാ എന്നായിരിക്കും. ഇതിനു മുമ്പ് നാം ആരെയും അവന്റെ പേരുള്ളവരാക്കിയിട്ടില്ല.”
\end{malayalam}}
\flushright{\begin{Arabic}
\quranayah[19][8]
\end{Arabic}}
\flushleft{\begin{malayalam}
അദ്ദേഹം പറഞ്ഞു: "എന്റെ നാഥാ, എനിക്കെങ്ങനെ പുത്രനുണ്ടാകും? എന്റെ ഭാര്യ വന്ധ്യയാണ്. ഞാനോ പ്രായാധിക്യത്താല്‍ പരവശനും.”
\end{malayalam}}
\flushright{\begin{Arabic}
\quranayah[19][9]
\end{Arabic}}
\flushleft{\begin{malayalam}
അല്ലാഹു അറിയിച്ചു: അതൊക്കെ ശരിതന്നെ. നിന്റെ നാഥന്‍ അരുള്‍ ചെയ്യുന്നു: എനിക്കത് നന്നെ നിസ്സാരമാണ്. നേരത്തെ നീ ഒന്നുമായിരുന്നില്ല. എന്നിട്ടും ഇതിനുമുമ്പ് നിന്നെ നാം സൃഷ്ടിച്ചല്ലോ.
\end{malayalam}}
\flushright{\begin{Arabic}
\quranayah[19][10]
\end{Arabic}}
\flushleft{\begin{malayalam}
സകരിയ്യാ പറഞ്ഞു: "നാഥാ, നീ എനിക്കൊരടയാളം കാണിച്ചു തരേണമേ?” അല്ലാഹു അറിയിച്ചു: "നിനക്കിപ്പോള്‍ വൈകല്യമൊന്നുമില്ല. എന്നാലും നീ മൂന്നുനാള്‍ ജനങ്ങളോട് മിണ്ടാതിരിക്കും. അതാണ് നിനക്കുള്ള അടയാളം.”
\end{malayalam}}
\flushright{\begin{Arabic}
\quranayah[19][11]
\end{Arabic}}
\flushleft{\begin{malayalam}
അങ്ങനെ അദ്ദേഹം പ്രാര്‍ഥനാ മണ്ഡപത്തില്‍ നിന്നിറങ്ങി തന്റെ ജനത്തിന്റെ അടുത്തേക്ക് പോയി. എന്നിട്ട് അദ്ദേഹം ആംഗ്യത്തിലൂടെ നിര്‍ദേശിച്ചു: "നിങ്ങള്‍ രാവിലെയും വൈകുന്നേരവും അല്ലാഹുവിന്റെ വിശുദ്ധി വാഴ്ത്തുക.”
\end{malayalam}}
\flushright{\begin{Arabic}
\quranayah[19][12]
\end{Arabic}}
\flushleft{\begin{malayalam}
"ഓ യഹ്യാ, വേദപുസ്തകം കരുത്തോടെ മുറുകെപ്പിടിക്കുക.” കുട്ടിയായിരിക്കെ തന്നെ നാമവന്ന് ജ്ഞാനം നല്‍കി.
\end{malayalam}}
\flushright{\begin{Arabic}
\quranayah[19][13]
\end{Arabic}}
\flushleft{\begin{malayalam}
നമ്മില്‍ നിന്നുള്ള ദയയും വിശുദ്ധിയും സമ്മാനിച്ചു. അദ്ദേഹം തികഞ്ഞ ഭക്തനായിരുന്നു;
\end{malayalam}}
\flushright{\begin{Arabic}
\quranayah[19][14]
\end{Arabic}}
\flushleft{\begin{malayalam}
തന്റെ മാതാപിതാക്കള്‍ക്ക് നന്മ ചെയ്യുന്നവനും. അദ്ദേഹം ക്രൂരനായിരുന്നില്ല. അനുസരണമില്ലാത്തവനുമായിരുന്നില്ല.
\end{malayalam}}
\flushright{\begin{Arabic}
\quranayah[19][15]
\end{Arabic}}
\flushleft{\begin{malayalam}
ജനനനാളിലും മരണദിനത്തിലും, ജീവനോടെ ഉയിര്‍ത്തെഴുന്നേല്‍ക്കുന്ന നാളിലും അദ്ദേഹത്തിനു സമാധാനം!
\end{malayalam}}
\flushright{\begin{Arabic}
\quranayah[19][16]
\end{Arabic}}
\flushleft{\begin{malayalam}
ഈ വേദപുസ്തകത്തില്‍ മര്‍യമിന്റെ കാര്യം വിവരിക്കുക. അവര്‍ തന്റെ സ്വന്തക്കാരില്‍ നിന്നകലെ കിഴക്കൊരിടത്ത് കഴിഞ്ഞുകൂടിയ കാലം.
\end{malayalam}}
\flushright{\begin{Arabic}
\quranayah[19][17]
\end{Arabic}}
\flushleft{\begin{malayalam}
സ്വന്തക്കാരില്‍ നിന്നൊളിഞ്ഞിരിക്കാന്‍ അവരൊരു മറയുണ്ടാക്കി. അപ്പോള്‍ നാം നമ്മുടെ മലക്കിനെ മര്‍യമിന്റെ അടുത്തേക്കയച്ചു. മലക്ക് അവരുടെ മുമ്പില്‍ തികഞ്ഞ മനുഷ്യരൂപത്തില്‍ പ്രത്യക്ഷമായി.
\end{malayalam}}
\flushright{\begin{Arabic}
\quranayah[19][18]
\end{Arabic}}
\flushleft{\begin{malayalam}
അവര്‍ പറഞ്ഞു: "ഞാന്‍ നിങ്ങളില്‍നിന്ന് പരമകാരുണികനായ അല്ലാഹുവില്‍ അഭയം തേടുന്നു. നിങ്ങളൊരു ഭക്തനെങ്കില്‍?”
\end{malayalam}}
\flushright{\begin{Arabic}
\quranayah[19][19]
\end{Arabic}}
\flushleft{\begin{malayalam}
മലക്ക് പറഞ്ഞു: "നിനക്ക് പരിശുദ്ധനായൊരു പുത്രനെ പ്രദാനം ചെയ്യാന്‍ നിന്റെ നാഥന്‍ നിയോഗിച്ച ദൂതന്‍ മാത്രമാണ് ഞാന്‍.”
\end{malayalam}}
\flushright{\begin{Arabic}
\quranayah[19][20]
\end{Arabic}}
\flushleft{\begin{malayalam}
അവര്‍ പറഞ്ഞു: "എനിക്കെങ്ങനെ പുത്രനുണ്ടാകും? ഇന്നോളം ഒരാണും എന്നെ തൊട്ടിട്ടില്ല. ഞാന്‍ ദുര്‍നടപ്പുകാരിയുമല്ല.”
\end{malayalam}}
\flushright{\begin{Arabic}
\quranayah[19][21]
\end{Arabic}}
\flushleft{\begin{malayalam}
മലക്ക് പറഞ്ഞു: "അതൊക്കെ ശരിതന്നെ. എന്നാലും അതുണ്ടാവും. നിന്റെ നാഥന്‍ പറയുന്നു: നമുക്കത് നന്നെ നിസ്സാരമായ കാര്യമാണ്. ആ കുട്ടിയെ ജനങ്ങള്‍ക്കൊരടയാളവും നമ്മില്‍ നിന്നുള്ള കാരുണ്യവുമാക്കാനാണ് നാം അങ്ങനെ ചെയ്യുന്നത്. അത് തീരുമാനിക്കപ്പെട്ട കാര്യമാണ്.”
\end{malayalam}}
\flushright{\begin{Arabic}
\quranayah[19][22]
\end{Arabic}}
\flushleft{\begin{malayalam}
അങ്ങനെ അവര്‍ ആ കുഞ്ഞിനെ ഗര്‍ഭം ധരിച്ചു. ഗര്‍ഭം ചുമന്ന് അവര്‍ അകലെ ഒറ്റക്കൊരിടത്ത് മാറിത്താമസിച്ചു.
\end{malayalam}}
\flushright{\begin{Arabic}
\quranayah[19][23]
\end{Arabic}}
\flushleft{\begin{malayalam}
പിന്നെ പേറ്റുനോവുണ്ടായപ്പോള്‍ അവര്‍ ഒരീന്തപ്പനയുടെ അടുത്തേക്കുപോയി. അവര്‍ പറഞ്ഞു: "അയ്യോ കഷ്ടം! ഇതിനു മുമ്പേ തന്നെ ഞാന്‍ മരിച്ചിരുന്നെങ്കില്‍! എന്റെ ഓര്‍മപോലും മാഞ്ഞുപോയിരുന്നെങ്കില്‍!”
\end{malayalam}}
\flushright{\begin{Arabic}
\quranayah[19][24]
\end{Arabic}}
\flushleft{\begin{malayalam}
അപ്പോള്‍ താഴ്ഭാഗത്തുനിന്ന് അവരോട് വിളിച്ചുപറഞ്ഞു: "നീ ദുഃഖിക്കേണ്ട. നിന്റെ നാഥന്‍ നിന്റെ താഴ്ഭാഗത്ത് ഒരരുവി ഉണ്ടാക്കിത്തന്നിരിക്കുന്നു.
\end{malayalam}}
\flushright{\begin{Arabic}
\quranayah[19][25]
\end{Arabic}}
\flushleft{\begin{malayalam}
"നീ ആ ഈന്തപ്പന മരമൊന്നു പിടിച്ചു കുലുക്കുക. അത് നിനക്ക് പഴുത്തു പാകമായ പഴം വീഴ്ത്തിത്തരും
\end{malayalam}}
\flushright{\begin{Arabic}
\quranayah[19][26]
\end{Arabic}}
\flushleft{\begin{malayalam}
"അങ്ങനെ നീ തിന്നുകയും കുടിക്കുകയും കണ്‍കുളിര്‍ക്കുകയും ചെയ്യുക. അഥവാ, നീയിനി വല്ലവരെയും കാണുകയാണെങ്കില്‍ അവരോട് ഇങ്ങനെ പറഞ്ഞേക്കുക: “ഞാന്‍ പരമകാരുണികനായ അല്ലാഹുവിനു വേണ്ടി നോമ്പെടുക്കാമെന്ന് നേര്‍ച്ചയാക്കിയിട്ടുണ്ട്. അതിനാല്‍ ഞാന്‍ ഇന്ന് ആരോടും സംസാരിക്കുകയില്ല.”
\end{malayalam}}
\flushright{\begin{Arabic}
\quranayah[19][27]
\end{Arabic}}
\flushleft{\begin{malayalam}
പിന്നെ അവര്‍ ആ കുഞ്ഞിനെയെടുത്ത് തന്റെ ജനത്തിന്റെ അടുത്തു ചെന്നു. അവര്‍ പറഞ്ഞുതുടങ്ങി: "മര്‍യമേ, കൊടിയ കുറ്റമാണല്ലോ നീ ചെയ്തിരിക്കുന്നത്.
\end{malayalam}}
\flushright{\begin{Arabic}
\quranayah[19][28]
\end{Arabic}}
\flushleft{\begin{malayalam}
"ഹാറൂന്റെ സോദരീ, നിന്റെ പിതാവ് വൃത്തികെട്ടവനായിരുന്നില്ല. നിന്റെ മാതാവ് പിഴച്ചവളുമായിരുന്നില്ല.”
\end{malayalam}}
\flushright{\begin{Arabic}
\quranayah[19][29]
\end{Arabic}}
\flushleft{\begin{malayalam}
അപ്പോള്‍ മര്‍യം തന്റെ കുഞ്ഞിനു നേരെ വിരല്‍ ചൂണ്ടി. അവര്‍ ചോദിച്ചു: "തൊട്ടിലില്‍ കിടക്കുന്ന കുട്ടിയോട് ഞങ്ങളെങ്ങനെ സംസാരിക്കും?”
\end{malayalam}}
\flushright{\begin{Arabic}
\quranayah[19][30]
\end{Arabic}}
\flushleft{\begin{malayalam}
കുഞ്ഞ് പറഞ്ഞു: " ഞാന്‍ അല്ലാഹുവിന്റെ ദാസനാണ്. അവനെനിക്കു വേദപുസ്തകം നല്‍കിയിരിക്കുന്നു. എന്നെ പ്രവാചകനാക്കുകയും ചെയ്തിരിക്കുന്നു.
\end{malayalam}}
\flushright{\begin{Arabic}
\quranayah[19][31]
\end{Arabic}}
\flushleft{\begin{malayalam}
"ഞാന്‍ എവിടെയായിരുന്നാലും അവനെന്നെ അനുഗൃഹീതനാക്കിയിരിക്കുന്നു. ഞാന്‍ ജീവിച്ചിരിക്കുന്നേടത്തോളം കാലം നമസ്കരിക്കാനും സകാത്ത് നല്‍കാനും അവനെന്നോട് കല്‍പിച്ചിരിക്കുന്നു.
\end{malayalam}}
\flushright{\begin{Arabic}
\quranayah[19][32]
\end{Arabic}}
\flushleft{\begin{malayalam}
"അല്ലാഹു എന്നെ എന്റെ മാതാവിനോട് നന്നായി വര്‍ത്തിക്കുന്നവനാക്കിയിരിക്കുന്നു. അവനെന്നെ ക്രൂരനും ഭാഗ്യംകെട്ടവനുമാക്കിയിട്ടില്ല.
\end{malayalam}}
\flushright{\begin{Arabic}
\quranayah[19][33]
\end{Arabic}}
\flushleft{\begin{malayalam}
"എന്റെ ജനനദിനത്തിലും മരണദിവസത്തിലും ഉയിര്‍ത്തെഴുന്നേല്‍പ് നാളിലും എനിക്ക് സമാധാനം!”
\end{malayalam}}
\flushright{\begin{Arabic}
\quranayah[19][34]
\end{Arabic}}
\flushleft{\begin{malayalam}
അതാണ് മര്‍യമിന്റെ മകന്‍ ഈസാ. ജനം തര്‍ക്കിച്ചുകൊണ്ടിരിക്കുന്ന കാര്യത്തിലുള്ള സത്യസന്ധമായ വിവരണമാണിത്.
\end{malayalam}}
\flushright{\begin{Arabic}
\quranayah[19][35]
\end{Arabic}}
\flushleft{\begin{malayalam}
പുത്രനെ സ്വീകരിക്കുകയെന്നത് അല്ലാഹുവിനു ചേര്‍ന്നതല്ല. അവനെത്ര പരിശുദ്ധന്‍. അവനൊരു കാര്യം തീരുമാനിച്ചാല്‍ അതിനോട് “ഉണ്ടാവുക” എന്ന വചനമേ വേണ്ടൂ. അതോടെ അതുണ്ടാവുന്നു.
\end{malayalam}}
\flushright{\begin{Arabic}
\quranayah[19][36]
\end{Arabic}}
\flushleft{\begin{malayalam}
ഈസാ പറഞ്ഞു: "സംശയമില്ല; അല്ലാഹു എന്റെയും നിങ്ങളുടെയും നാഥനാണ്. അതിനാല്‍ അവനു വഴിപ്പെടുക. ഇതാണ് നേര്‍വഴി.”
\end{malayalam}}
\flushright{\begin{Arabic}
\quranayah[19][37]
\end{Arabic}}
\flushleft{\begin{malayalam}
എന്നാല്‍ അവര്‍ ഭിന്നിച്ച് വിവിധ വിഭാഗങ്ങളായി. ആ ഭീകരനാളിനെ കണ്ടുമുട്ടുമ്പോള്‍ അതിനെ തള്ളിപ്പറഞ്ഞവര്‍ക്കെല്ലാം കടുത്ത വിപത്താണുണ്ടാവുക.
\end{malayalam}}
\flushright{\begin{Arabic}
\quranayah[19][38]
\end{Arabic}}
\flushleft{\begin{malayalam}
അവര്‍ നമ്മുടെ അടുത്ത് വരുംദിനം അവര്‍ക്കെന്തൊരു കേള്‍വിയും കാഴ്ചയുമായിരിക്കും? എന്നാലിന്ന് ആ അക്രമികള്‍ പ്രകടമായ വഴികേടിലാണ്.
\end{malayalam}}
\flushright{\begin{Arabic}
\quranayah[19][39]
\end{Arabic}}
\flushleft{\begin{malayalam}
തീരാ ദുഃഖത്തിന്റെ ആ ദുര്‍ ദിനത്തെപ്പറ്റി അവര്‍ക്ക് മുന്നറിയിപ്പു നല്‍കുക. കാര്യം അന്തിമമായി തീരുമാനിക്കപ്പെടുന്ന ദിനമാണത്. എന്നാല്‍ അവര്‍ അതേക്കുറിച്ച് തീര്‍ത്തും അശ്രദ്ധയിലാണ്. അവര്‍ വിശ്വസിക്കുന്നുമില്ല.
\end{malayalam}}
\flushright{\begin{Arabic}
\quranayah[19][40]
\end{Arabic}}
\flushleft{\begin{malayalam}
അവസാനം ഭൂമിയുടെയും അതിലുള്ളവരുടെയും അവകാശിയാകുന്നത് നാം തന്നെയാണ്. എല്ലാവരും തിരിച്ചെത്തുന്നതും നമ്മുടെ അടുത്തേക്കു തന്നെ.
\end{malayalam}}
\flushright{\begin{Arabic}
\quranayah[19][41]
\end{Arabic}}
\flushleft{\begin{malayalam}
ഈ വേദപുസ്തകത്തില്‍ ഇബ്റാഹീമിന്റെ കഥയും നീ വിവരിച്ചു കൊടുക്കുക: സംശയമില്ല; അദ്ദേഹം സത്യവാനും പ്രവാചകനുമായിരുന്നു.
\end{malayalam}}
\flushright{\begin{Arabic}
\quranayah[19][42]
\end{Arabic}}
\flushleft{\begin{malayalam}
അദ്ദേഹം തന്റെ പിതാവിനോട് പറഞ്ഞ സന്ദര്‍ഭം: "എന്റുപ്പാ, കേള്‍ക്കാനോ കാണാനോ അങ്ങയ്ക്കെന്തെങ്കിലും ഉപകാരം ചെയ്യാനോ കഴിയാത്ത വസ്തുക്കളെ അങ്ങ് എന്തിനാണ് പൂജിച്ചുകൊണ്ടിരിക്കുന്നത്?
\end{malayalam}}
\flushright{\begin{Arabic}
\quranayah[19][43]
\end{Arabic}}
\flushleft{\begin{malayalam}
"എന്റുപ്പാ, അങ്ങയ്ക്ക് വന്നുകിട്ടിയിട്ടില്ലാത്ത അറിവ് എനിക്കു വന്നെത്തിയിട്ടുണ്ട്. അതിനാല്‍ എന്നെ പിന്തുടരുക. ഞാന്‍ അങ്ങയ്ക്ക് നേര്‍വഴി കാണിച്ചുതരാം.
\end{malayalam}}
\flushright{\begin{Arabic}
\quranayah[19][44]
\end{Arabic}}
\flushleft{\begin{malayalam}
"എന്റുപ്പാ, അങ്ങ് പിശാചിന് വഴിപ്പെടരുത്. തീര്‍ച്ചയായും പിശാച് പരമകാരുണികനായ അല്ലാഹുവെ ധിക്കരിച്ചവനാണ്.
\end{malayalam}}
\flushright{\begin{Arabic}
\quranayah[19][45]
\end{Arabic}}
\flushleft{\begin{malayalam}
"പ്രിയ പിതാവേ, പരമകാരുണികനായ അല്ലാഹുവില്‍ നിന്നുള്ള വല്ല ശിക്ഷയും അങ്ങയെ ഉറപ്പായും പിടികൂടുമെന്ന് ഞാന്‍ ഭയപ്പെടുന്നു. അപ്പോള്‍ അങ്ങ് പിശാചിന്റെ ഉറ്റമിത്രമായി മാറും.”
\end{malayalam}}
\flushright{\begin{Arabic}
\quranayah[19][46]
\end{Arabic}}
\flushleft{\begin{malayalam}
അയാള്‍ ചോദിച്ചു: "ഇബ്റാഹീമേ, നീ എന്റെ ദൈവങ്ങളെ വെറുക്കുകയാണോ? എങ്കില്‍ ഉടനെത്തന്നെ ഇതവസാനിപ്പിക്കുക. അല്ലെങ്കില്‍ നിന്നെ ഞാന്‍ കല്ലെറിഞ്ഞാട്ടും. നീ എന്നെന്നേക്കുമായി എന്നെ വിട്ടുപോകണം”
\end{malayalam}}
\flushright{\begin{Arabic}
\quranayah[19][47]
\end{Arabic}}
\flushleft{\begin{malayalam}
ഇബ്റാഹീം പറഞ്ഞു: "അങ്ങയ്ക്ക് സലാം. അങ്ങയ്ക്കു പൊറുത്തുതരാന്‍ ഞാനെന്റെ നാഥനോട് പ്രാര്‍ഥിക്കാം. സംശയമില്ല; അവനെന്നോട് ഏറെ കനിവുറ്റവനാണ്.
\end{malayalam}}
\flushright{\begin{Arabic}
\quranayah[19][48]
\end{Arabic}}
\flushleft{\begin{malayalam}
"നിങ്ങളെയും അല്ലാഹുവെക്കൂടാതെ നിങ്ങള്‍ വിളിച്ചു പ്രാര്‍ഥിക്കുന്നവയെയും ഞാനിതാ നിരാകരിക്കുന്നു. ഞാനെന്റെ നാഥനോടു മാത്രം പ്രാര്‍ഥിക്കുന്നു. എന്റെ നാഥനെ പ്രാര്‍ഥിക്കുന്നതു കാരണം ഞാനൊരിക്കലും പരാജിതനാവില്ലെന്ന് ഉറപ്പിക്കാം.”
\end{malayalam}}
\flushright{\begin{Arabic}
\quranayah[19][49]
\end{Arabic}}
\flushleft{\begin{malayalam}
അങ്ങനെ ഇബ്റാഹീം അവരെയും അല്ലാഹു അല്ലാത്ത അവരുടെ ആരാധ്യരെയും വെടിഞ്ഞുപോയപ്പോള്‍ അദ്ദേഹത്തിനു നാം ഇസ്ഹാഖിനെയും യഅ്ഖൂബിനെയും നല്‍കി. അവരെയെല്ലാം പ്രവാചകന്മാരാക്കുകയും ചെയ്തു.
\end{malayalam}}
\flushright{\begin{Arabic}
\quranayah[19][50]
\end{Arabic}}
\flushleft{\begin{malayalam}
അവരില്‍ നാം നമ്മുടെ അനുഗ്രഹങ്ങള്‍ വര്‍ഷിച്ചു. അവരുടെ സല്‍ക്കീര്‍ത്തി ഉയര്‍ത്തി.
\end{malayalam}}
\flushright{\begin{Arabic}
\quranayah[19][51]
\end{Arabic}}
\flushleft{\begin{malayalam}
ഈ വേദപുസ്തകത്തില്‍ മൂസയുടെ കഥയും പരാമര്‍ശിക്കുക: തീര്‍ച്ചയായും അദ്ദേഹം തെരഞ്ഞെടുക്കപ്പെട്ട വ്യക്തിയായിരുന്നു. ദൂതനും പ്രവാചകനുമായിരുന്നു.
\end{malayalam}}
\flushright{\begin{Arabic}
\quranayah[19][52]
\end{Arabic}}
\flushleft{\begin{malayalam}
ത്വൂര്‍ മലയുടെ വലതുവശത്തുനിന്നു നാം അദ്ദേഹത്തെ വിളിച്ചു. രഹസ്യ സംഭാഷണത്തിനായി നാം അദ്ദേഹത്തെ നമ്മിലേക്കടുപ്പിച്ചു.
\end{malayalam}}
\flushright{\begin{Arabic}
\quranayah[19][53]
\end{Arabic}}
\flushleft{\begin{malayalam}
നമ്മുടെ അനുഗ്രഹത്താല്‍ നാം അദ്ദേഹത്തിന് തന്റെ സഹോദരനെ- പ്രവാചകനായ ഹാറൂനിനെ- സഹായിയായി നല്‍കി.
\end{malayalam}}
\flushright{\begin{Arabic}
\quranayah[19][54]
\end{Arabic}}
\flushleft{\begin{malayalam}
ഈ വേദപുസ്തകത്തില്‍ ഇസ്മാഈലിന്റെ കാര്യവും പരാമര്‍ശിക്കുക: തീര്‍ച്ചയായും അദ്ദേഹം വാഗ്ദാനം നന്നായി പാലിക്കുന്നവനായിരുന്നു. ദൂതനും പ്രവാചകനുമായിരുന്നു.
\end{malayalam}}
\flushright{\begin{Arabic}
\quranayah[19][55]
\end{Arabic}}
\flushleft{\begin{malayalam}
അദ്ദേഹം തന്റെ ആള്‍ക്കാരോട് നമസ്കാരം നിര്‍വഹിക്കാനും സകാത്ത് നല്‍കാനും കല്‍പിച്ചു. അദ്ദേഹം തന്റെ നാഥന്ന് ഏറെ പ്രിയപ്പെട്ടവനായിരുന്നു.
\end{malayalam}}
\flushright{\begin{Arabic}
\quranayah[19][56]
\end{Arabic}}
\flushleft{\begin{malayalam}
ഈ വേദപുസ്തകത്തില്‍ ഇദ്രീസിനെപ്പറ്റിയും പരാമര്‍ശിക്കുക: നിശ്ചയമായും അദ്ദേഹം സത്യസന്ധനും പ്രവാചകനുമായിരുന്നു.
\end{malayalam}}
\flushright{\begin{Arabic}
\quranayah[19][57]
\end{Arabic}}
\flushleft{\begin{malayalam}
നാം അദ്ദേഹത്തെ ഉന്നതസ്ഥാനത്തേക്കുയര്‍ത്തി.
\end{malayalam}}
\flushright{\begin{Arabic}
\quranayah[19][58]
\end{Arabic}}
\flushleft{\begin{malayalam}
ഇവരാണ് അല്ലാഹു അനുഗ്രഹിച്ച പ്രവാചകന്മാര്‍. ആദം സന്തതികളില്‍ പെട്ടവര്‍. നൂഹിനോടൊപ്പം നാം കപ്പലില്‍ കയറ്റിയവരുടെയും; ഇബ്റാഹീമിന്റെയും ഇസ്രയേലിന്റെയും വംശത്തില്‍ നിന്നുള്ളവരാണിവര്‍. നാം നേര്‍വഴിയില്‍ നയിക്കുകയും പ്രത്യേകം തെരഞ്ഞെടുക്കുകയും ചെയ്തവരില്‍ പെട്ടവരും. പരമകാരുണികനായ അല്ലാഹുവിന്റെ വചനങ്ങള്‍ വായിച്ചുകേള്‍ക്കുമ്പോള്‍ സാഷ്ടാംഗം പ്രണമിച്ചും കരഞ്ഞും നിലം പതിക്കുന്നവരായിരുന്നു ഇവര്‍.
\end{malayalam}}
\flushright{\begin{Arabic}
\quranayah[19][59]
\end{Arabic}}
\flushleft{\begin{malayalam}
പിന്നീട് ഇവര്‍ക്കു പിറകെ പിഴച്ച ഒരു തലമുറ രംഗത്തുവന്നു. അവര്‍ നമസ്കാരം പാഴാക്കി. തന്നിഷ്ടങ്ങള്‍ക്കൊത്ത് ജീവിച്ചു. തങ്ങളുടെ ദുര്‍വൃത്തികളുടെ ദുരന്തഫലം അവരെ വൈകാതെ ബാധിക്കും.
\end{malayalam}}
\flushright{\begin{Arabic}
\quranayah[19][60]
\end{Arabic}}
\flushleft{\begin{malayalam}
പശ്ചാത്തപിക്കുകയും സത്യവിശ്വാസം സ്വീകരിക്കുകയും സല്‍ക്കര്‍മങ്ങള്‍ പ്രവര്‍ത്തിക്കുകയും ചെയ്തവരെയൊഴികെ. അവര്‍ സ്വര്‍ഗീയാരാമങ്ങളില്‍ പ്രവേശിക്കും. അവരോട് ഒട്ടും അനീതിയുണ്ടാവില്ല.
\end{malayalam}}
\flushright{\begin{Arabic}
\quranayah[19][61]
\end{Arabic}}
\flushleft{\begin{malayalam}
അവര്‍ക്ക് സ്ഥിരവാസത്തിനുള്ള സ്വര്‍ഗീയാരാമങ്ങളുണ്ട്. പരമകാരുണികനായ അല്ലാഹു തന്റെ ദാസന്മാര്‍ക്ക് അഭൌതികജ്ഞാനത്തിലൂടെ നല്‍കിയ വാഗ്ദാനമാണിത്. അവന്റെ വാഗ്ദാനം നടപ്പാകുക തന്നെ ചെയ്യും.
\end{malayalam}}
\flushright{\begin{Arabic}
\quranayah[19][62]
\end{Arabic}}
\flushleft{\begin{malayalam}
അവരവിടെ ഒരനാവശ്യവും കേള്‍ക്കുകയില്ല; സമാധാനത്തിന്റെ അഭിവാദ്യമല്ലാതെ. തങ്ങളുടെ ആഹാരവിഭവങ്ങള്‍ രാവിലെയും വൈകുന്നേരവും മുടങ്ങാതെ അവര്‍ക്ക് കിട്ടിക്കൊണ്ടിരിക്കും.
\end{malayalam}}
\flushright{\begin{Arabic}
\quranayah[19][63]
\end{Arabic}}
\flushleft{\begin{malayalam}
നമ്മുടെ ദാസന്മാരിലെ ഭക്തന്മാര്‍ക്ക് നാം അവകാശമായി നല്‍കുന്ന സ്വര്‍ഗമാണത്.
\end{malayalam}}
\flushright{\begin{Arabic}
\quranayah[19][64]
\end{Arabic}}
\flushleft{\begin{malayalam}
"നിന്റെ നാഥന്റെ കല്‍പനയില്ലാതെ ഞങ്ങള്‍ ഇറങ്ങിവരാറില്ല. നമ്മുടെ മുന്നിലും പിന്നിലും അവയ്ക്കിടയിലുമുള്ളതെല്ലാം അവന്റേതാണ്. നിന്റെ നാഥനൊന്നും മറക്കുന്നവനല്ല.”
\end{malayalam}}
\flushright{\begin{Arabic}
\quranayah[19][65]
\end{Arabic}}
\flushleft{\begin{malayalam}
അവന്‍ ആകാശഭൂമികളുടെ സംരക്ഷകനാണ്. അവയ്ക്കിടയിലുള്ളവയുടെയും. അതിനാല്‍ അവന്നു മാത്രം വഴിപ്പെടുക. അവനെ അനുസരിച്ച് കഴിയുന്നതില്‍ ക്ഷമയോടെ ഉറച്ചുനില്‍ക്കുക. അവനോട് പേരൊത്ത ആരെയെങ്കിലും നിനക്കറിയാമോ?
\end{malayalam}}
\flushright{\begin{Arabic}
\quranayah[19][66]
\end{Arabic}}
\flushleft{\begin{malayalam}
മനുഷ്യന്‍ ചോദിക്കുന്നു: "ഞാന്‍ മരിച്ചുകഴിഞ്ഞാല്‍ പിന്നെ വീണ്ടും എന്നെ ജീവനോടെ പുറത്തുകൊണ്ടുവരുമെന്നോ!”
\end{malayalam}}
\flushright{\begin{Arabic}
\quranayah[19][67]
\end{Arabic}}
\flushleft{\begin{malayalam}
മനുഷ്യന്‍ ഒന്നുമല്ലാതിരുന്ന അവസ്ഥയില്‍ നിന്ന് നാം അവനെ സൃഷ്ടിച്ചുണ്ടാക്കിയ കാര്യം അവനൊന്നോര്‍ത്തുകൂടേ?
\end{malayalam}}
\flushright{\begin{Arabic}
\quranayah[19][68]
\end{Arabic}}
\flushleft{\begin{malayalam}
നിന്റെ നാഥന്‍ തന്നെ സത്യം! തീര്‍ച്ചയായും അവരെയും പിശാചുക്കളെയും നാം ഒരുമിച്ചുകൂട്ടും. പിന്നെ നാമവരെ മുട്ടിലിഴയുന്നവരായി നരകത്തിനു ചുറ്റും കൊണ്ടുവരും.
\end{malayalam}}
\flushright{\begin{Arabic}
\quranayah[19][69]
\end{Arabic}}
\flushleft{\begin{malayalam}
പിന്നീട് ഓരോ വിഭാഗത്തില്‍നിന്നും പരമകാരുണികനായ അല്ലാഹുവോട് ഏറ്റം കൂടുതല്‍ ധിക്കാരം കാണിച്ചവരെ നാം വേര്‍തിരിച്ചെടുക്കും.
\end{malayalam}}
\flushright{\begin{Arabic}
\quranayah[19][70]
\end{Arabic}}
\flushleft{\begin{malayalam}
അവരില്‍ നരകത്തീയിലെരിയാന്‍ ഏറ്റവും അര്‍ഹര്‍ ആരെന്ന് നമുക്ക് നന്നായറിയാം.
\end{malayalam}}
\flushright{\begin{Arabic}
\quranayah[19][71]
\end{Arabic}}
\flushleft{\begin{malayalam}
നിങ്ങളിലാരും തന്നെ നരകത്തീയിനടുത്ത് എത്താതിരിക്കില്ല. നിന്റെ നാഥന്റെ ഖണ്ഡിതവും നിര്‍ബന്ധപൂര്‍വം നടപ്പാക്കപ്പെടുന്നതുമായ തീരുമാനമാണിത്.
\end{malayalam}}
\flushright{\begin{Arabic}
\quranayah[19][72]
\end{Arabic}}
\flushleft{\begin{malayalam}
പിന്നെ, ഭക്തന്മാരായിരുന്നവരെ നാം രക്ഷപ്പെടുത്തും. അതിക്രമികളെ മുട്ടിലിഴയുന്നവരായി നരകത്തീയില്‍ ഉപേക്ഷിക്കുകയും ചെയ്യും.
\end{malayalam}}
\flushright{\begin{Arabic}
\quranayah[19][73]
\end{Arabic}}
\flushleft{\begin{malayalam}
നമ്മുടെ സുവ്യക്തമായ വചനങ്ങള്‍ ഈ ജനത്തെ വായിച്ചുകേള്‍പ്പിക്കും. അപ്പോള്‍ സത്യനിഷേധികള്‍ സത്യവിശ്വാസികളോടു ചോദിക്കുന്നു: "അല്ല, പറയൂ: നാം ഇരുകൂട്ടരില്‍ ആരാണ് ഉയര്‍ന്ന പദവിയുള്ളവര്‍? ആരുടെ സംഘമാണ് ഏറെ ഗംഭീരം?”
\end{malayalam}}
\flushright{\begin{Arabic}
\quranayah[19][74]
\end{Arabic}}
\flushleft{\begin{malayalam}
എന്നാല്‍ സാധന സാമഗ്രികളിലും ബാഹ്യപ്രതാപത്തിലും ഇവരേക്കാളേറെ മികച്ച എത്രയെത്ര തലമുറകളെയാണ് നാം ഇവര്‍ക്കു മുമ്പേ നശിപ്പിച്ചിട്ടുള്ളത്!
\end{malayalam}}
\flushright{\begin{Arabic}
\quranayah[19][75]
\end{Arabic}}
\flushleft{\begin{malayalam}
പറയുക: ദുര്‍മാര്‍ഗികളെ പരമകാരുണികനായ അല്ലാഹു അയച്ചുവിടുന്നതാണ്. അങ്ങനെ അവരോട് വാഗ്ദാനം ചെയ്യുന്ന കാര്യം, അഥവാ ഒന്നുകില്‍ ദൈവശിക്ഷ, അല്ലെങ്കില്‍ അന്ത്യദിനം, നേരില്‍ കാണുമ്പോള്‍ അവരറിയുകതന്നെ ചെയ്യും; ആരാണ് മോശമായ അവസ്ഥയിലുള്ളതെന്ന്. ആരുടെ സൈന്യമാണ് ദുര്‍ബലമെന്നും.
\end{malayalam}}
\flushright{\begin{Arabic}
\quranayah[19][76]
\end{Arabic}}
\flushleft{\begin{malayalam}
നേര്‍വഴി സ്വീകരിച്ചവര്‍ക്ക് അല്ലാഹു സന്മാര്‍ഗനിഷ്ഠ വര്‍ധിപ്പിച്ചുകൊടുക്കുന്നു. നശിക്കാതെ ബാക്കിനില്‍ക്കുന്ന സല്‍ക്കര്‍മങ്ങള്‍ക്കാണ് നിന്റെ നാഥന്റെ അടുത്ത് ഉത്തമമായ പ്രതിഫലമുള്ളത്. മെച്ചപ്പെട്ട പരിണതിയും അവയ്ക്കുതന്നെ.
\end{malayalam}}
\flushright{\begin{Arabic}
\quranayah[19][77]
\end{Arabic}}
\flushleft{\begin{malayalam}
നമ്മുടെ വചനങ്ങളെ നിഷേധിച്ചു തള്ളുകയും എന്നിട്ട് എനിക്കാണ് കൂടുതല്‍ സമ്പത്തും സന്താനങ്ങളും നല്‍കപ്പെടുകയെന്ന് വീമ്പു പറയുകയും ചെയ്യുന്നവനെ നീ കണ്ടിട്ടുണ്ടോ?
\end{malayalam}}
\flushright{\begin{Arabic}
\quranayah[19][78]
\end{Arabic}}
\flushleft{\begin{malayalam}
അവന്‍ വല്ല അഭൌതിക കാര്യവും കണ്ടറിഞ്ഞിട്ടുണ്ടോ? അല്ലെങ്കില്‍ പരമകാരുണികനായ അല്ലാഹുവില്‍നിന്ന് വല്ല കരാറും അവന്‍ വാങ്ങിയിട്ടുണ്ടോ?
\end{malayalam}}
\flushright{\begin{Arabic}
\quranayah[19][79]
\end{Arabic}}
\flushleft{\begin{malayalam}
ഒരിക്കലുമില്ല. അവന്‍ പറയുന്നതൊക്കെ നാം രേഖപ്പെടുത്തുന്നുണ്ട്. അവന്നു നാം ശിക്ഷയുടെ കാഠിന്യം വര്‍ധിപ്പിക്കുകതന്നെ ചെയ്യും.
\end{malayalam}}
\flushright{\begin{Arabic}
\quranayah[19][80]
\end{Arabic}}
\flushleft{\begin{malayalam}
അവന്‍ തന്റേതായി എടുത്തുപറയുന്ന സാധനസാമഗ്രികളെല്ലാം നമ്മുടെ വരുതിയിലായിത്തീരും. പിന്നെ അവന്‍ ഏകനായി നമ്മുടെ അടുത്തുവരും.
\end{malayalam}}
\flushright{\begin{Arabic}
\quranayah[19][81]
\end{Arabic}}
\flushleft{\begin{malayalam}
അവര്‍ അല്ലാഹുവെക്കൂടാതെ നിരവധി മൂര്‍ത്തികളെ സങ്കല്‍പിച്ചുവെച്ചിരിക്കുന്നു. അവ തങ്ങള്‍ക്ക് താങ്ങായിത്തീരുമെന്ന് കരുതിയാണത്.
\end{malayalam}}
\flushright{\begin{Arabic}
\quranayah[19][82]
\end{Arabic}}
\flushleft{\begin{malayalam}
എന്നാല്‍ അവയെല്ലാം ഇക്കൂട്ടരുടെ ആരാധനയെ തള്ളിപ്പറയും. ആ ആരാധ്യര്‍ ഇവരുടെ വിരോധികളായിത്തീരുകയും ചെയ്യും.
\end{malayalam}}
\flushright{\begin{Arabic}
\quranayah[19][83]
\end{Arabic}}
\flushleft{\begin{malayalam}
നാം സത്യനിഷേധികളുടെയിടയിലേക്ക് പിശാചുക്കളെ വിട്ടയച്ചത് നീ കണ്ടിട്ടില്ലേ? ആ പിശാചുക്കള്‍ അവരെ വളരെയേറെ ഉത്തേജിപ്പിച്ചുകൊണ്ടിരിക്കുന്നു.
\end{malayalam}}
\flushright{\begin{Arabic}
\quranayah[19][84]
\end{Arabic}}
\flushleft{\begin{malayalam}
അതിനാല്‍ അവരുടെ കാര്യത്തില്‍ നീ ധൃതികാണിക്കേണ്ട. നാം അവരുടെ നാളുകളെണ്ണിക്കൊണ്ടിരിക്കുകയാണ്.
\end{malayalam}}
\flushright{\begin{Arabic}
\quranayah[19][85]
\end{Arabic}}
\flushleft{\begin{malayalam}
ഭക്തജനങ്ങളെ പരമകാരുണികനായ അല്ലാഹുവിന്റെ അടുത്ത് ഒരുമിച്ചുകൂട്ടുന്നദിനം.
\end{malayalam}}
\flushright{\begin{Arabic}
\quranayah[19][86]
\end{Arabic}}
\flushleft{\begin{malayalam}
അന്ന് കുറ്റവാളികളെ ദാഹാര്‍ത്തരായി നരകത്തീയിലേക്ക് തെളിച്ചുകൊണ്ടുപോകും.
\end{malayalam}}
\flushright{\begin{Arabic}
\quranayah[19][87]
\end{Arabic}}
\flushleft{\begin{malayalam}
അന്ന് ആര്‍ക്കും ശിപാര്‍ശക്കധികാരമില്ല; പരമ കാരുണികനായ അല്ലാഹുവുമായി കരാറുണ്ടാക്കിയവര്‍ക്കൊഴികെ.
\end{malayalam}}
\flushright{\begin{Arabic}
\quranayah[19][88]
\end{Arabic}}
\flushleft{\begin{malayalam}
പരമകാരുണികനായ അല്ലാഹു പുത്രനെ സ്വീകരിച്ചിരിക്കുന്നുവെന്ന് അവര്‍ പറഞ്ഞുണ്ടാക്കിയിരിക്കുന്നു.
\end{malayalam}}
\flushright{\begin{Arabic}
\quranayah[19][89]
\end{Arabic}}
\flushleft{\begin{malayalam}
ഏറെ ഗുരുതരമായ കാര്യമാണ് നിങ്ങളാരോപിച്ചിരിക്കുന്നത്.
\end{malayalam}}
\flushright{\begin{Arabic}
\quranayah[19][90]
\end{Arabic}}
\flushleft{\begin{malayalam}
ആകാശങ്ങള്‍ പൊട്ടിപ്പിളരാനും ഭൂമി വിണ്ടുകീറാനും പര്‍വതങ്ങള്‍ തകര്‍ന്നുവീഴാനും പോന്നകാര്യം.
\end{malayalam}}
\flushright{\begin{Arabic}
\quranayah[19][91]
\end{Arabic}}
\flushleft{\begin{malayalam}
പരമകാരുണികനായ അല്ലാഹുവിന് പുത്രനുണ്ടെന്ന് അവര്‍ വാദിച്ചല്ലോ.
\end{malayalam}}
\flushright{\begin{Arabic}
\quranayah[19][92]
\end{Arabic}}
\flushleft{\begin{malayalam}
ആരെയെങ്കിലും പുത്രനായി സ്വീകരിക്കുകയെന്നത് പരമകാരുണികനായ അല്ലാഹുവിന് ചേര്‍ന്നതല്ല.
\end{malayalam}}
\flushright{\begin{Arabic}
\quranayah[19][93]
\end{Arabic}}
\flushleft{\begin{malayalam}
ആകാശഭൂമികളിലുള്ളവരെല്ലാം ആ പരമകാരുണികന്റെ മുന്നില്‍ കേവലം ദാസന്മാരായി വന്നെത്തുന്നവരാണ്.
\end{malayalam}}
\flushright{\begin{Arabic}
\quranayah[19][94]
\end{Arabic}}
\flushleft{\begin{malayalam}
തീര്‍ച്ചയായും അവന്‍ അവരെ തിട്ടപ്പെടുത്തിയിട്ടുണ്ട്. എണ്ണിക്കണക്കാക്കുകയും ചെയ്തിരിക്കുന്നു.
\end{malayalam}}
\flushright{\begin{Arabic}
\quranayah[19][95]
\end{Arabic}}
\flushleft{\begin{malayalam}
ഉയിര്‍ത്തെഴുന്നേല്‍പുനാളില്‍ അവരെല്ലാം ഒറ്റയ്ക്കൊറ്റയ്ക്ക് അവന്റെ അടുത്ത് വന്നെത്തും.
\end{malayalam}}
\flushright{\begin{Arabic}
\quranayah[19][96]
\end{Arabic}}
\flushleft{\begin{malayalam}
സത്യവിശ്വാസം സ്വീകരിക്കുകയും സല്‍ക്കര്‍മങ്ങള്‍ പ്രവര്‍ത്തിക്കുകയും ചെയ്തവരുമായി പരമകാരുണികനായ അല്ലാഹു സ്നേഹബന്ധമുണ്ടാക്കും.
\end{malayalam}}
\flushright{\begin{Arabic}
\quranayah[19][97]
\end{Arabic}}
\flushleft{\begin{malayalam}
നാം ഈ വചനങ്ങളെ നിന്റെ ഭാഷയില്‍ വളരെ ലളിതവും സരളവുമാക്കിയിരിക്കുന്നു. നീ ഭക്തജനങ്ങളെ ശുഭവാര്‍ത്ത അറിയിക്കാനാണിത്. താര്‍ക്കികരായ ജനത്തെ താക്കീത് ചെയ്യാനും.
\end{malayalam}}
\flushright{\begin{Arabic}
\quranayah[19][98]
\end{Arabic}}
\flushleft{\begin{malayalam}
ഇവര്‍ക്കു മുമ്പ് എത്ര തലമുറകളെ നാം നശിപ്പിച്ചു! എന്നിട്ട് അവരിലാരെയെങ്കിലും നീയിപ്പോള്‍ കാണുന്നുണ്ടോ? അല്ലെങ്കില്‍ അവരുടെ നേര്‍ത്ത ശബ്ദമെങ്കിലും കേള്‍ക്കുന്നുണ്ടോ?
\end{malayalam}}
\chapter{\textmalayalam{ത്വാഹാ}}
\begin{Arabic}
\Huge{\centerline{\basmalah}}\end{Arabic}
\flushright{\begin{Arabic}
\quranayah[20][1]
\end{Arabic}}
\flushleft{\begin{malayalam}
ത്വാഹാ.
\end{malayalam}}
\flushright{\begin{Arabic}
\quranayah[20][2]
\end{Arabic}}
\flushleft{\begin{malayalam}
നിനക്കു നാം ഈ ഖുര്‍ആന്‍ ഇറക്കിയത് നീ കഷ്ടപ്പെടാന്‍ വേണ്ടിയല്ല.
\end{malayalam}}
\flushright{\begin{Arabic}
\quranayah[20][3]
\end{Arabic}}
\flushleft{\begin{malayalam}
ഭയഭക്തിയുള്ളവര്‍ക്ക് ഉദ്ബോധനമായാണ്.
\end{malayalam}}
\flushright{\begin{Arabic}
\quranayah[20][4]
\end{Arabic}}
\flushleft{\begin{malayalam}
ഭൂമിയും അത്യുന്നതമായ ആകാശങ്ങളും സൃഷ്ടിച്ചവനില്‍ നിന്ന് ഇറക്കപ്പെട്ടതാണിത്.
\end{malayalam}}
\flushright{\begin{Arabic}
\quranayah[20][5]
\end{Arabic}}
\flushleft{\begin{malayalam}
ആ പരമകാരുണികനായ അല്ലാഹു സിംഹാസനസ്ഥനായിരിക്കുന്നു.
\end{malayalam}}
\flushright{\begin{Arabic}
\quranayah[20][6]
\end{Arabic}}
\flushleft{\begin{malayalam}
ആകാശങ്ങളിലും ഭൂമിയിലും അവയ്ക്കിടയിലുമുള്ളതെല്ലാം അവന്റേതാണ്. മണ്ണിനടിയിലുമുള്ളതും.
\end{malayalam}}
\flushright{\begin{Arabic}
\quranayah[20][7]
\end{Arabic}}
\flushleft{\begin{malayalam}
നിനക്കു വേണമെങ്കില്‍ വാക്ക് ഉറക്കെ പറയാം. എന്നാല്‍ അല്ലാഹു രഹസ്യമായതും പരമ നിഗൂഢമായതുമെല്ലാം നന്നായറിയുന്നവനാണ്.
\end{malayalam}}
\flushright{\begin{Arabic}
\quranayah[20][8]
\end{Arabic}}
\flushleft{\begin{malayalam}
അല്ലാഹു. അവനല്ലാതെ ദൈവമില്ല. ഉല്‍കൃഷ്ട നാമങ്ങളെല്ലാം അവന്നുള്ളതാണ്.
\end{malayalam}}
\flushright{\begin{Arabic}
\quranayah[20][9]
\end{Arabic}}
\flushleft{\begin{malayalam}
മൂസയുടെ കഥ നിനക്കു വന്നെത്തിയോ?
\end{malayalam}}
\flushright{\begin{Arabic}
\quranayah[20][10]
\end{Arabic}}
\flushleft{\begin{malayalam}
അദ്ദേഹം തീ കണ്ട സന്ദര്‍ഭം: അപ്പോള്‍ അദ്ദേഹം തന്റെ കുടുംബത്തോടു പറഞ്ഞു: "ഇവിടെ നില്‍ക്കൂ. ഞാനിതാ തീ കാണുന്നു. അതില്‍നിന്ന് ഞാനല്‍പം തീയെടുത്ത് നിങ്ങള്‍ക്കായി കൊണ്ടുവരാം. അല്ലെങ്കില്‍ അവിടെ വല്ല വഴികാട്ടിയെയും ഞാന്‍ കണ്ടെത്തിയേക്കാം.”
\end{malayalam}}
\flushright{\begin{Arabic}
\quranayah[20][11]
\end{Arabic}}
\flushleft{\begin{malayalam}
അങ്ങനെ അദ്ദേഹം അവിടെയെത്തിയപ്പോള്‍ ഒരു വിളി കേട്ടു: "മൂസാ,
\end{malayalam}}
\flushright{\begin{Arabic}
\quranayah[20][12]
\end{Arabic}}
\flushleft{\begin{malayalam}
"നിശ്ചയം; ഞാന്‍ നിന്റെ നാഥനാണ്. അതിനാല്‍ നീ നിന്റെ ചെരിപ്പ് അഴിച്ചുവെക്കുക. തീര്‍ച്ചയായും നീയിപ്പോള്‍ വിശുദ്ധമായ ത്വുവാ താഴ്വരയിലാണ്.
\end{malayalam}}
\flushright{\begin{Arabic}
\quranayah[20][13]
\end{Arabic}}
\flushleft{\begin{malayalam}
"ഞാന്‍ നിന്നെ തെരഞ്ഞെടുത്തിരിക്കുന്നു. അതിനാല്‍ ബോധനമായി കിട്ടുന്നത് നന്നായി കേട്ടുമനസ്സിലാക്കുക.
\end{malayalam}}
\flushright{\begin{Arabic}
\quranayah[20][14]
\end{Arabic}}
\flushleft{\begin{malayalam}
"തീര്‍ച്ചയായും ഞാന്‍ തന്നെ അല്ലാഹു. ഞാനല്ലാതെ ദൈവമില്ല. അതിനാല്‍ എനിക്കു വഴിപ്പെടുക. എന്നെ ഓര്‍ക്കാനായി നമസ്കാരം നിഷ്ഠയോടെ നിര്‍വഹിക്കുക.
\end{malayalam}}
\flushright{\begin{Arabic}
\quranayah[20][15]
\end{Arabic}}
\flushleft{\begin{malayalam}
"തീര്‍ച്ചയായും അന്ത്യനാള്‍ വന്നെത്തുക തന്നെ ചെയ്യും. അതെപ്പോഴെന്നത് ഞാന്‍ മറച്ചുവെച്ചിരിക്കുകയാണ്. ഓരോ വ്യക്തിക്കും തന്റെ അധ്വാനഫലം കൃത്യമായി ലഭിക്കാന്‍ വേണ്ടിയാണിത്.
\end{malayalam}}
\flushright{\begin{Arabic}
\quranayah[20][16]
\end{Arabic}}
\flushleft{\begin{malayalam}
"അതിനാല്‍ അന്ത്യദിനത്തില്‍ വിശ്വസിക്കാതിരിക്കുകയും തന്നിഷ്ടങ്ങളെ പിന്‍പറ്റുകയും ചെയ്യുന്നവര്‍ നിന്നെ വിശ്വാസത്തിന്റെ വഴിയില്‍നിന്ന് തെറ്റിച്ചു കളയാതിരിക്കട്ടെ. അങ്ങനെ സംഭവിച്ചാല്‍ നീയും നാശത്തിലകപ്പെടും.
\end{malayalam}}
\flushright{\begin{Arabic}
\quranayah[20][17]
\end{Arabic}}
\flushleft{\begin{malayalam}
"മൂസാ, നിന്റെ വലതു കയ്യിലെന്താണ്?”
\end{malayalam}}
\flushright{\begin{Arabic}
\quranayah[20][18]
\end{Arabic}}
\flushleft{\begin{malayalam}
മൂസ പറഞ്ഞു: "ഇതെന്റെ വടിയാണ്. ഞാനിതിന്മേല്‍ ഊന്നി നടക്കുന്നു. ഞാനിതുകൊണ്ട് എന്റെ ആടുകള്‍ക്ക് ഇല വീഴ്ത്തിക്കൊടുക്കുന്നു. ഇതുകൊണ്ട് എനിക്ക് വേറെയും ചില ആവശ്യങ്ങളുണ്ട്.”
\end{malayalam}}
\flushright{\begin{Arabic}
\quranayah[20][19]
\end{Arabic}}
\flushleft{\begin{malayalam}
അല്ലാഹു കല്‍പിച്ചു: "മൂസാ, നീ ആ വടി താഴെയിടൂ.”
\end{malayalam}}
\flushright{\begin{Arabic}
\quranayah[20][20]
\end{Arabic}}
\flushleft{\begin{malayalam}
അദ്ദേഹം അതു താഴെയിട്ടു. പെട്ടെന്നതാ, അതൊരിഴയുന്ന പാമ്പായിത്തീരുന്നു.
\end{malayalam}}
\flushright{\begin{Arabic}
\quranayah[20][21]
\end{Arabic}}
\flushleft{\begin{malayalam}
അല്ലാഹു പറഞ്ഞു: "അതിനെ പിടിക്കൂ. പേടിക്കേണ്ട. നാം അതിനെ പൂര്‍വസ്ഥിതിയിലേക്കു തന്നെ തിരിച്ചുകൊണ്ടുവരും.
\end{malayalam}}
\flushright{\begin{Arabic}
\quranayah[20][22]
\end{Arabic}}
\flushleft{\begin{malayalam}
"നിന്റെ കൈ നീ കക്ഷത്തു ചേര്‍ത്തുവെക്കുക. അപ്പോഴതു ദോഷമേതുമില്ലാതെ വെളുത്തു തിളങ്ങുന്നതായി പുറത്തുവരും. ഇത് മറ്റൊരു ദൃഷ്ടാന്തമാണ്.
\end{malayalam}}
\flushright{\begin{Arabic}
\quranayah[20][23]
\end{Arabic}}
\flushleft{\begin{malayalam}
"നമ്മുടെ മഹത്തായ ചില ദൃഷ്ടാന്തങ്ങള്‍ നിന്നെ കാണിക്കാന്‍ വേണ്ടിയാണിത്.
\end{malayalam}}
\flushright{\begin{Arabic}
\quranayah[20][24]
\end{Arabic}}
\flushleft{\begin{malayalam}
"നീയിനി ഫറവോന്റെ അടുത്തേക്ക് പോകൂ. അവന്‍ കടുത്ത ധിക്കാരിയായിത്തീര്‍ന്നിരിക്കുന്നു.”
\end{malayalam}}
\flushright{\begin{Arabic}
\quranayah[20][25]
\end{Arabic}}
\flushleft{\begin{malayalam}
മൂസ പറഞ്ഞു: "എന്റെ നാഥാ! എനിക്കു നീ ഹൃദയവിശാലത നല്‍കേണമേ.
\end{malayalam}}
\flushright{\begin{Arabic}
\quranayah[20][26]
\end{Arabic}}
\flushleft{\begin{malayalam}
"എന്റെ കാര്യം എനിക്കു നീ എളുപ്പമാക്കിത്തരേണമേ!
\end{malayalam}}
\flushright{\begin{Arabic}
\quranayah[20][27]
\end{Arabic}}
\flushleft{\begin{malayalam}
"എന്റെ നാവിന്റെ കുരുക്കഴിച്ചു തരേണമേ!
\end{malayalam}}
\flushright{\begin{Arabic}
\quranayah[20][28]
\end{Arabic}}
\flushleft{\begin{malayalam}
"എന്റെ സംസാരം ജനം മനസ്സിലാക്കാനാവും വിധമാക്കേണമേ!
\end{malayalam}}
\flushright{\begin{Arabic}
\quranayah[20][29]
\end{Arabic}}
\flushleft{\begin{malayalam}
"എന്റെ കുടുംബത്തില്‍ നിന്ന് എനിക്കൊരു സഹായിയെ ഏര്‍പ്പെടുത്തിത്തരേണമേ?”
\end{malayalam}}
\flushright{\begin{Arabic}
\quranayah[20][30]
\end{Arabic}}
\flushleft{\begin{malayalam}
"എന്റെ സഹോദരന്‍ ഹാറൂനെ തന്നെ.
\end{malayalam}}
\flushright{\begin{Arabic}
\quranayah[20][31]
\end{Arabic}}
\flushleft{\begin{malayalam}
"അവനിലൂടെ എന്റെ കഴിവിന് മികവ് വരുത്തേണമേ.
\end{malayalam}}
\flushright{\begin{Arabic}
\quranayah[20][32]
\end{Arabic}}
\flushleft{\begin{malayalam}
"എന്റെ ദൌത്യത്തില്‍ അവനെ നീ പങ്കാളിയാക്കേണമേ.
\end{malayalam}}
\flushright{\begin{Arabic}
\quranayah[20][33]
\end{Arabic}}
\flushleft{\begin{malayalam}
"ഞങ്ങള്‍ നിന്റെ വിശുദ്ധി ധാരാളമായി വാഴ്ത്താനാണിത്.
\end{malayalam}}
\flushright{\begin{Arabic}
\quranayah[20][34]
\end{Arabic}}
\flushleft{\begin{malayalam}
"നിന്നെ നന്നായി ഓര്‍ക്കാനും.
\end{malayalam}}
\flushright{\begin{Arabic}
\quranayah[20][35]
\end{Arabic}}
\flushleft{\begin{malayalam}
"തീര്‍ച്ചയായും നീ ഞങ്ങളെ സദാ കണ്ടുകൊണ്ടിരിക്കുന്നവനാണല്ലോ.”
\end{malayalam}}
\flushright{\begin{Arabic}
\quranayah[20][36]
\end{Arabic}}
\flushleft{\begin{malayalam}
അല്ലാഹു അറിയിച്ചു: "മൂസാ, നീ ചോദിച്ചതൊക്കെ നിനക്കിതാ നല്‍കിക്കഴിഞ്ഞിരിക്കുന്നു.
\end{malayalam}}
\flushright{\begin{Arabic}
\quranayah[20][37]
\end{Arabic}}
\flushleft{\begin{malayalam}
"മറ്റൊരിക്കലും നിന്നോട് നാം ഔദാര്യം കാണിച്ചിട്ടുണ്ട്.
\end{malayalam}}
\flushright{\begin{Arabic}
\quranayah[20][38]
\end{Arabic}}
\flushleft{\begin{malayalam}
"ദിവ്യബോധനത്തിലൂടെ നല്‍കപ്പെടുന്ന കാര്യം നാം നിന്റെ മാതാവിന് ബോധനം നല്‍കിയപ്പോഴാണത്.”
\end{malayalam}}
\flushright{\begin{Arabic}
\quranayah[20][39]
\end{Arabic}}
\flushleft{\begin{malayalam}
"അതിതായിരുന്നു: “നീ ആ ശിശുവെ പെട്ടിയിലടക്കം ചെയ്യുക. എന്നിട്ട് പെട്ടി നദിയിലൊഴുക്കുക. നദി അതിനെ കരയിലെത്തിക്കും. എന്റെയും ആ ശിശുവിന്റെയും ശത്രു അവനെ എടുക്കും. മൂസാ, ഞാന്‍ എന്നില്‍ നിന്നുള്ള സ്നേഹം നിന്റെമേല്‍ വര്‍ഷിച്ചു. നീ എന്റെ മേല്‍നോട്ടത്തില്‍ വളര്‍ത്തപ്പെടാന്‍ വേണ്ടി.
\end{malayalam}}
\flushright{\begin{Arabic}
\quranayah[20][40]
\end{Arabic}}
\flushleft{\begin{malayalam}
"നിന്റെ സഹോദരി നടന്നുപോവുകയായിരുന്നു. അവളവിടെ ചെന്നിങ്ങനെ പറഞ്ഞു: “ഈ കുഞ്ഞിനെ നന്നായി പോറ്റാന്‍ പറ്റുന്ന ഒരാളെപ്പറ്റി ഞാന്‍ നിങ്ങള്‍ക്ക് പറഞ്ഞു തരട്ടെയോ?” അങ്ങനെ നിന്നെ നാം നിന്റെ മാതാവിന്റെ അടുത്തുതന്നെ തിരിച്ചെത്തിച്ചു. അവളുടെ കണ്‍കുളിര്‍ക്കാന്‍. അവള്‍ ദുഃഖിക്കാതിരിക്കാനും. നീ ഒരാളെ കൊന്നിരുന്നുവല്ലോ. എന്നാല്‍ അതിന്റെ മനഃപ്രയാസത്തില്‍നിന്ന് നിന്നെ നാം രക്ഷിച്ചു. പല തരത്തിലും നിന്നെ നാം പരീക്ഷിച്ചു. പിന്നീട് കൊല്ലങ്ങളോളം നീ മദ്യന്‍കാരുടെ കൂടെ താമസിച്ചു. അനന്തരം അല്ലയോ മൂസാ; ഇതാ ഇപ്പോള്‍ ദൈവ നിശ്ചയമനുസരിച്ച് നീ ഇവിടെ വന്നിരിക്കുന്നു.
\end{malayalam}}
\flushright{\begin{Arabic}
\quranayah[20][41]
\end{Arabic}}
\flushleft{\begin{malayalam}
"ഞാനിതാ നിന്നെ എനിക്കുവേണ്ടി വളര്‍ത്തിയെടുത്തിരിക്കുന്നു.
\end{malayalam}}
\flushright{\begin{Arabic}
\quranayah[20][42]
\end{Arabic}}
\flushleft{\begin{malayalam}
"എന്റെ തെളിവുകളുമായി നീയും നിന്റെ സഹോദരനും പോവുക. എന്നെ സ്മരിക്കുന്നതില്‍ നിങ്ങള്‍ വീഴ്ചവരുത്തരുത്.
\end{malayalam}}
\flushright{\begin{Arabic}
\quranayah[20][43]
\end{Arabic}}
\flushleft{\begin{malayalam}
"നിങ്ങളിരുവരും ഫറവോന്റെ അടുത്തേക്ക് പോവുക. നിശ്ചയമായും അവന്‍ അതിക്രമിയായിരിക്കുന്നു.
\end{malayalam}}
\flushright{\begin{Arabic}
\quranayah[20][44]
\end{Arabic}}
\flushleft{\begin{malayalam}
"നിങ്ങളവനോട് സൌമ്യമായി സംസാരിക്കുക. ഒരുവേള അവന്‍ ചിന്തിച്ചു മനസ്സിലാക്കിയെങ്കിലോ? അല്ലെങ്കില്‍ ഭയന്ന് അനുസരിച്ചെങ്കിലോ?”
\end{malayalam}}
\flushright{\begin{Arabic}
\quranayah[20][45]
\end{Arabic}}
\flushleft{\begin{malayalam}
അവരിരുവരും പറഞ്ഞു: "ഞങ്ങളുടെ നാഥാ! ഫറവോന്‍ ഞങ്ങളോട് അവിവേകമോ അതിക്രമമോ കാണിക്കുമെന്ന് ഞങ്ങള്‍ ഭയപ്പെടുന്നു.”
\end{malayalam}}
\flushright{\begin{Arabic}
\quranayah[20][46]
\end{Arabic}}
\flushleft{\begin{malayalam}
അല്ലാഹു പറഞ്ഞു: "നിങ്ങള്‍ പേടിക്കേണ്ട. ഞാന്‍ നിങ്ങളോടൊപ്പമുണ്ട്. ഞാന്‍ എല്ലാം കേള്‍ക്കുകയും കാണുകയും ചെയ്യുന്നുണ്ട്.”
\end{malayalam}}
\flushright{\begin{Arabic}
\quranayah[20][47]
\end{Arabic}}
\flushleft{\begin{malayalam}
"അതിനാല്‍ നിങ്ങളിരുവരും അവന്റെയടുത്ത് ചെന്ന് പറയുക: “തീര്‍ച്ചയായും ഞങ്ങള്‍ നിന്റെ നാഥന്റെ ദൂതന്മാരാണ്. അതിനാല്‍ ഇസ്രയേല്‍ മക്കളെ നീ ഞങ്ങളോടൊപ്പമയക്കുക. അവരെ പീഡിപ്പിക്കരുത്. നിന്റെ അടുത്ത് ഞങ്ങള്‍ വന്നത് നിന്റെ നാഥനില്‍നിന്നുള്ള വ്യക്തമായ തെളിവുകളുമായാണ്. നേര്‍വഴിയില്‍ നടക്കുന്നവര്‍ക്കാണ് സമാധാനമുണ്ടാവുക.
\end{malayalam}}
\flushright{\begin{Arabic}
\quranayah[20][48]
\end{Arabic}}
\flushleft{\begin{malayalam}
“സത്യത്തെ തള്ളിപ്പറയുകയും അതില്‍നിന്ന് പിന്തിരിഞ്ഞു പോവുകയും ചെയ്യുന്നവര്‍ക്ക് കടുത്ത ശിക്ഷയാണുണ്ടാവുകയെന്ന് തീര്‍ച്ചയായും ഞങ്ങള്‍ക്ക് ദിവ്യബോധനം ലഭിച്ചിരിക്കുന്നു.”
\end{malayalam}}
\flushright{\begin{Arabic}
\quranayah[20][49]
\end{Arabic}}
\flushleft{\begin{malayalam}
ഫറവോന്‍ ചോദിച്ചു: "മൂസാ, അപ്പോള്‍ ആരാണ് നിങ്ങളുടെ ഈ രക്ഷിതാവ്?”
\end{malayalam}}
\flushright{\begin{Arabic}
\quranayah[20][50]
\end{Arabic}}
\flushleft{\begin{malayalam}
മൂസ പറഞ്ഞു: "എല്ലാ ഓരോ വസ്തുവിനും അതിന്റെ പ്രകൃതം നല്‍കുകയും പിന്നെ അവയ്ക്ക് വഴി കാണിക്കുകയും ചെയ്തവനാണ് ഞങ്ങളുടെ രക്ഷിതാവ്.
\end{malayalam}}
\flushright{\begin{Arabic}
\quranayah[20][51]
\end{Arabic}}
\flushleft{\begin{malayalam}
അയാള്‍ ചോദിച്ചു: "അപ്പോള്‍ നേരത്തെ കഴിഞ്ഞുപോയ തലമുറകളുടെ സ്ഥിതിയോ?”
\end{malayalam}}
\flushright{\begin{Arabic}
\quranayah[20][52]
\end{Arabic}}
\flushleft{\begin{malayalam}
മൂസ പറഞ്ഞു: "അതേക്കുറിച്ചുള്ള എല്ലാ വിവരവും എന്റെ നാഥന്റെ അടുക്കല്‍ ഒരു പ്രമാണത്തിലുണ്ട്. എന്റെ നാഥന്‍ ഒട്ടും പിഴവു പറ്റാത്തവനാണ്. തീരെ മറവിയില്ലാത്തവനും.”
\end{malayalam}}
\flushright{\begin{Arabic}
\quranayah[20][53]
\end{Arabic}}
\flushleft{\begin{malayalam}
നിങ്ങള്‍ക്കായി ഭൂമിയെ തൊട്ടിലാക്കിത്തന്നത് അവനാണ്. അതില്‍ നിങ്ങള്‍ക്ക് നിരവധി വഴികളൊരുക്കിത്തന്നതും മാനത്തുനിന്നു മഴ വീഴ്ത്തിത്തന്നതും അവന്‍ തന്നെ. അങ്ങനെ ആ മഴമൂലം വിവിധയിനം സസ്യങ്ങളിലെ ഇണകളെ നാം ഉല്‍പാദിപ്പിച്ചു.
\end{malayalam}}
\flushright{\begin{Arabic}
\quranayah[20][54]
\end{Arabic}}
\flushleft{\begin{malayalam}
നിങ്ങള്‍ തിന്നുകൊള്ളുക. നിങ്ങളുടെ കന്നുകാലികളെ മേയ്ക്കുകയും ചെയ്യുക. വിചാരശീലര്‍ക്ക് ഇതിലെല്ലാം ധാരാളം തെളിവുകളുണ്ട്.
\end{malayalam}}
\flushright{\begin{Arabic}
\quranayah[20][55]
\end{Arabic}}
\flushleft{\begin{malayalam}
ഇതേ മണ്ണില്‍നിന്നാണ് നിങ്ങളെ നാം സൃഷ്ടിച്ചത്. അതിലേക്കു തന്നെ നിങ്ങളെ നാം തിരിച്ചുകൊണ്ടുപോകും. അതില്‍നിന്നു തന്നെ നിങ്ങളെ നാം മറ്റൊരിക്കല്‍ പുറത്തുകൊണ്ടുവരികയും ചെയ്യും.
\end{malayalam}}
\flushright{\begin{Arabic}
\quranayah[20][56]
\end{Arabic}}
\flushleft{\begin{malayalam}
ഫറവോന് നാം നമ്മുടെ തെളിവുകളൊക്കെയും കാണിച്ചുകൊടുത്തു. എന്നിട്ടും അയാള്‍ അവയെ തള്ളിപ്പറഞ്ഞു. സത്യത്തെ നിരാകരിച്ചു.
\end{malayalam}}
\flushright{\begin{Arabic}
\quranayah[20][57]
\end{Arabic}}
\flushleft{\begin{malayalam}
അയാള്‍ ചോദിച്ചു: "ഓ മൂസാ, നിന്റെ ജാലവിദ്യകൊണ്ട് ഞങ്ങളെ ഞങ്ങളുടെ നാട്ടില്‍ നിന്ന് പുറത്താക്കാനാണോ നീ ഞങ്ങളുടെ അടുത്തു വന്നിരിക്കുന്നത്?
\end{malayalam}}
\flushright{\begin{Arabic}
\quranayah[20][58]
\end{Arabic}}
\flushleft{\begin{malayalam}
"എന്നാല്‍ ഇതുപോലുള്ള ജാലവിദ്യ നിന്റെ മുന്നില്‍ ഞങ്ങളും അവതരിപ്പിക്കാം. അതിനാല്‍ ഞങ്ങള്‍ക്കും നിനക്കുമിടയില്‍ ഒരു സമയം നിശ്ചയിക്കുക. നീയോ ഞങ്ങളോ അത് ലംഘിക്കരുത്. ഇരുകൂട്ടര്‍ക്കും സൌകര്യമുള്ള തുറന്ന മൈതാനിയില്‍വെച്ചാകാം മത്സരം.”
\end{malayalam}}
\flushright{\begin{Arabic}
\quranayah[20][59]
\end{Arabic}}
\flushleft{\begin{malayalam}
മൂസ പറഞ്ഞു: "അതൊരു ഉല്‍സവ ദിനമാകട്ടെ. അന്ന് പൂര്‍വാഹ്നത്തില്‍ ജനം ഒരുമിച്ചുകൂടട്ടെ.”
\end{malayalam}}
\flushright{\begin{Arabic}
\quranayah[20][60]
\end{Arabic}}
\flushleft{\begin{malayalam}
പിന്നീട് ഫറവോന്‍ അവിടെനിന്ന് പിന്മാറി. തന്റെ തന്ത്രങ്ങളെല്ലാം ഒരുക്കൂട്ടി തിരികെ വന്നു.
\end{malayalam}}
\flushright{\begin{Arabic}
\quranayah[20][61]
\end{Arabic}}
\flushleft{\begin{malayalam}
മൂസ അവരോടു പറഞ്ഞു: "നിങ്ങള്‍ക്കു നാശം? നിങ്ങള്‍ അല്ലാഹുവിന്റെ പേരില്‍ കള്ളം കെട്ടിച്ചമയ്ക്കരുത്. അങ്ങനെ ചെയ്താല്‍ കൊടിയ ശിക്ഷയാല്‍ അവന്‍ നിങ്ങളെ ഉന്മൂലനം ചെയ്യും. കള്ളം കെട്ടിച്ചമയ്ക്കുന്നവന്‍ തുലഞ്ഞതുതന്നെ; തീര്‍ച്ച.”
\end{malayalam}}
\flushright{\begin{Arabic}
\quranayah[20][62]
\end{Arabic}}
\flushleft{\begin{malayalam}
ഇതുകേട്ട് അവര്‍ക്കിടയില്‍ അഭിപ്രായ ഭിന്നതയുണ്ടായി. അവര്‍ രഹസ്യമായി കൂടിയാലോചിക്കാന്‍ തുടങ്ങി.
\end{malayalam}}
\flushright{\begin{Arabic}
\quranayah[20][63]
\end{Arabic}}
\flushleft{\begin{malayalam}
അതിനുശേഷം അവര്‍ പറഞ്ഞു: "ഇവരിരുവരും തനി ജാലവിദ്യക്കാരാണ്. ഇവരുടെ ജാലവിദ്യയിലൂടെ നിങ്ങളെ നിങ്ങളുടെ നാട്ടില്‍നിന്ന് പുറന്തള്ളാനും നിങ്ങളുടെ ചിട്ടയൊത്ത ജീവിതരീതി തകര്‍ക്കാനുമാണ് ഇവരുദ്ദേശിക്കുന്നത്.
\end{malayalam}}
\flushright{\begin{Arabic}
\quranayah[20][64]
\end{Arabic}}
\flushleft{\begin{malayalam}
"അതിനാല്‍ നിങ്ങള്‍ നിങ്ങളുടെ തന്ത്രങ്ങളൊക്കെയും ഒരുക്കൂട്ടി വെക്കുക. അങ്ങനെ വലിയ സംഘടിതശക്തിയായി രംഗത്തുവരിക. ഓര്‍ക്കുക: ആര്‍ എതിരാളിയെ തോല്‍പിക്കുന്നുവോ അവരിന്ന് വിജയം വരിച്ചതുതന്നെ.”
\end{malayalam}}
\flushright{\begin{Arabic}
\quranayah[20][65]
\end{Arabic}}
\flushleft{\begin{malayalam}
ജാലവിദ്യക്കാര്‍ പറഞ്ഞു: "മൂസാ, ഒന്നുകില്‍ നീ വടിയെറിയുക; അല്ലെങ്കില്‍ ആദ്യം ഞങ്ങളെറിയാം.”
\end{malayalam}}
\flushright{\begin{Arabic}
\quranayah[20][66]
\end{Arabic}}
\flushleft{\begin{malayalam}
മൂസ പറഞ്ഞു: "ഇല്ല. നിങ്ങള്‍ തന്നെ എറിഞ്ഞുകൊള്ളുക.” അപ്പോഴതാ അവരുടെ ജാലവിദ്യയാല്‍ കയറുകളും വടികളും ഇഴഞ്ഞുനീങ്ങുന്നതായി മൂസാക്കു തോന്നിത്തുടങ്ങി.
\end{malayalam}}
\flushright{\begin{Arabic}
\quranayah[20][67]
\end{Arabic}}
\flushleft{\begin{malayalam}
മൂസാക്ക് മനസ്സില്‍ പേടിതോന്നി.
\end{malayalam}}
\flushright{\begin{Arabic}
\quranayah[20][68]
\end{Arabic}}
\flushleft{\begin{malayalam}
നാം പറഞ്ഞു: "പേടിക്കേണ്ട. ഉറപ്പായും നീ തന്നെയാണ് അതിജയിക്കുക.
\end{malayalam}}
\flushright{\begin{Arabic}
\quranayah[20][69]
\end{Arabic}}
\flushleft{\begin{malayalam}
"നീ നിന്റെ വലതു കയ്യിലുള്ളത് നിലത്തിടുക. അവരുണ്ടാക്കിയ ജാലവിദ്യയൊക്കെയും അതു വിഴുങ്ങിക്കൊള്ളും.” അവരുണ്ടാക്കിയത് ജാലവിദ്യക്കാരുടെ തന്ത്രം മാത്രമാണ്. ജാലവിദ്യക്കാര്‍ എവിടെച്ചെന്നാലും വിജയിക്കുകയില്ല.
\end{malayalam}}
\flushright{\begin{Arabic}
\quranayah[20][70]
\end{Arabic}}
\flushleft{\begin{malayalam}
അവസാനം ജാലവിദ്യക്കാരെല്ലാം സാഷ്ടാംഗം പ്രണമിച്ചു. അവര്‍ പ്രഖ്യാപിച്ചു: "ഞങ്ങള്‍ ഹാറൂന്റെയും മൂസായുടെയും നാഥനില്‍ വിശ്വസിച്ചിരിക്കുന്നു.”
\end{malayalam}}
\flushright{\begin{Arabic}
\quranayah[20][71]
\end{Arabic}}
\flushleft{\begin{malayalam}
ഫറവോന്‍ പറഞ്ഞു: "ഞാന്‍ അനുമതി തരുംമുമ്പെ നിങ്ങളവനില്‍ വിശ്വസിച്ചുവെന്നോ? തീര്‍ച്ചയായും നിങ്ങളെ ജാലവിദ്യ പഠിപ്പിച്ച നിങ്ങളുടെ നേതാവാണവന്‍. നിങ്ങളുടെ കൈകാലുകള്‍ എതിര്‍വശങ്ങളില്‍ നിന്നായി ഞാന്‍ കൊത്തിമുറിക്കും. ഈന്തപ്പനത്തടികളില്‍ നിങ്ങളെ ക്രൂശിക്കും. നമ്മിലാരാണ് ഏറ്റവും കഠിനവും നീണ്ടുനില്‍ക്കുന്നതുമായ ശിക്ഷ നടപ്പാക്കുന്നവരെന്ന് അപ്പോള്‍ നിങ്ങളറിയും; തീര്‍ച്ച.”
\end{malayalam}}
\flushright{\begin{Arabic}
\quranayah[20][72]
\end{Arabic}}
\flushleft{\begin{malayalam}
അവര്‍ പറഞ്ഞു: "ഞങ്ങള്‍ക്കു വന്നെത്തിയ വ്യക്തമായ തെളിവുകളേക്കാളും ഞങ്ങളെ സൃഷ്ടിച്ചവനെക്കാളും ഞങ്ങളൊരിക്കലും നിനക്ക് പ്രാധാന്യം കല്‍പിക്കുകയില്ല. അതിനാല്‍ നീ വിധിക്കുന്നതെന്തോ അത് വിധിച്ചുകൊള്ളുക. ഈ ഐഹികജീവിതത്തില്‍ മാത്രമേ നിന്റെ വിധി നടക്കുകയുള്ളൂ.
\end{malayalam}}
\flushright{\begin{Arabic}
\quranayah[20][73]
\end{Arabic}}
\flushleft{\begin{malayalam}
"ഞങ്ങള്‍ ഞങ്ങളുടെ നാഥനില്‍ പൂര്‍ണമായും വിശ്വസിച്ചിരിക്കുന്നു. അവന്‍ ഞങ്ങളുടെ പാപങ്ങള്‍ പൊറുത്തുതന്നേക്കാം. നീ ഞങ്ങളെ നിര്‍ബന്ധിച്ച് ചെയ്യിച്ച ഈ ജാലവിദ്യയുടെ കുറ്റവും മാപ്പാക്കിയേക്കാം. അല്ലാഹുവാണ് ഏറ്റവും നല്ലവന്‍. എന്നെന്നും നിലനില്‍ക്കുന്നവനും അവന്‍ തന്നെ.”
\end{malayalam}}
\flushright{\begin{Arabic}
\quranayah[20][74]
\end{Arabic}}
\flushleft{\begin{malayalam}
എന്നാല്‍ കുറ്റവാളിയായി തന്റെ നാഥന്റെ അടുത്തെത്തുന്നവന്നുണ്ടാവുക നരകത്തീയാണ്. അതിലവന്‍ മരിക്കുകയില്ല. ജീവിക്കുകയുമില്ല.
\end{malayalam}}
\flushright{\begin{Arabic}
\quranayah[20][75]
\end{Arabic}}
\flushleft{\begin{malayalam}
അതോടൊപ്പം സത്യവിശ്വാസം സ്വീകരിച്ച് സല്‍പ്രവര്‍ത്തനങ്ങള്‍ ചെയ്ത് അവന്റെ അടുത്തെത്തുന്നവര്‍ക്ക് ഉന്നതമായ പദവികളുണ്ട്.
\end{malayalam}}
\flushright{\begin{Arabic}
\quranayah[20][76]
\end{Arabic}}
\flushleft{\begin{malayalam}
സ്ഥിരവാസത്തിനുള്ള സ്വര്‍ഗീയാരാമങ്ങള്‍. അതിന്റെ താഴ്ഭാഗത്തൂടെ ആറുകളൊഴുകിക്കൊണ്ടിരിക്കും. അവരതില്‍ നിത്യവാസികളായിരിക്കും. വിശുദ്ധിവരിച്ചവര്‍ക്കുള്ള പ്രതിഫലമിതാണ്.
\end{malayalam}}
\flushright{\begin{Arabic}
\quranayah[20][77]
\end{Arabic}}
\flushleft{\begin{malayalam}
മൂസാക്കു നാം ഇങ്ങനെ ബോധനം നല്‍കി: എന്റെ ദാസന്മാരെയും കൂട്ടി നീ രാത്രി പുറപ്പെടുക. എന്നിട്ട് അവര്‍ക്കായി കടലില്‍ വെള്ളം വറ്റി ഉണങ്ങിയ വഴി ഒരുക്കിക്കൊടുക്കുക. ആരും നിന്നെ പിടികൂടുമെന്ന് പേടിക്കേണ്ട. ഒട്ടും പരിഭ്രമിക്കുകയും വേണ്ട.
\end{malayalam}}
\flushright{\begin{Arabic}
\quranayah[20][78]
\end{Arabic}}
\flushleft{\begin{malayalam}
അപ്പോള്‍ ഫറവോന്‍ തന്റെ സൈന്യത്തെയും കൂട്ടി അവരെ പിന്തുടര്‍ന്നു. എന്നിട്ടോ കടല്‍ അവരെ മുക്കേണ്ട മട്ടിലങ്ങ് മുക്കി.
\end{malayalam}}
\flushright{\begin{Arabic}
\quranayah[20][79]
\end{Arabic}}
\flushleft{\begin{malayalam}
ഫറവോന്‍ തന്റെ ജനതയെ വഴികേടിലാക്കി. അവന്‍ അവരെ നേര്‍വഴിയില്‍ നയിച്ചില്ല.
\end{malayalam}}
\flushright{\begin{Arabic}
\quranayah[20][80]
\end{Arabic}}
\flushleft{\begin{malayalam}
ഇസ്രയേല്‍ മക്കളേ; നാം നിങ്ങളെ നിങ്ങളുടെ ശത്രുവില്‍നിന്ന് മോചിപ്പിച്ചു. ത്വൂര്‍ പര്‍വതത്തിന്റെ വലതുഭാഗത്ത് നിങ്ങള്‍ വന്നെത്തേണ്ടതെപ്പോഴെന്ന് നാം നിശ്ചയിച്ചറിയിച്ചുതന്നു. നിങ്ങള്‍ക്ക് മന്നും സല്‍വായും ഇറക്കിത്തന്നു.
\end{malayalam}}
\flushright{\begin{Arabic}
\quranayah[20][81]
\end{Arabic}}
\flushleft{\begin{malayalam}
നാം നിങ്ങള്‍ക്കു നല്‍കിയ വിശിഷ്ട വിഭവങ്ങളില്‍നിന്ന് ആഹരിച്ചുകൊള്ളുക. എന്നാല്‍ നിങ്ങളതില്‍ അതിരുകവിയരുത്. അങ്ങനെ സംഭവിച്ചാല്‍ എന്റെ കോപം നിങ്ങളിലുണ്ടാകും. എന്റെ കോപത്തിനിരയാകുന്നവന്‍ തുലഞ്ഞതു തന്നെ.
\end{malayalam}}
\flushright{\begin{Arabic}
\quranayah[20][82]
\end{Arabic}}
\flushleft{\begin{malayalam}
പശ്ചാത്തപിക്കുകയും സത്യവിശ്വാസം സ്വീകരിക്കുകയും സല്‍ക്കര്‍മങ്ങള്‍ പ്രവര്‍ത്തിക്കുകയും അങ്ങനെ നേര്‍വഴിയില്‍ നിലകൊള്ളുകയും ചെയ്യുന്നവര്‍ക്കു നാം അവരുടെ പാപങ്ങള്‍ പൂര്‍ണമായും പൊറുത്തുകൊടുക്കും.
\end{malayalam}}
\flushright{\begin{Arabic}
\quranayah[20][83]
\end{Arabic}}
\flushleft{\begin{malayalam}
അല്ലാഹു ചോദിച്ചു: മൂസാ, നീ നിന്റെ ജനത്തെ വിട്ടേച്ച് ധൃതിപ്പെട്ട് ഇവിടെ വരാന്‍ കാരണം?
\end{malayalam}}
\flushright{\begin{Arabic}
\quranayah[20][84]
\end{Arabic}}
\flushleft{\begin{malayalam}
അദ്ദേഹം പറഞ്ഞു: "അവരിതാ എന്റെ പിറകില്‍ത്തന്നെയുണ്ട്. ഞാന്‍ നിന്റെ അടുത്ത് ധൃതിപ്പെട്ടുവന്നത് നാഥാ, നീയെന്നെ തൃപ്തിപ്പെടാന്‍ വേണ്ടി മാത്രമാണ്.”
\end{malayalam}}
\flushright{\begin{Arabic}
\quranayah[20][85]
\end{Arabic}}
\flushleft{\begin{malayalam}
അല്ലാഹു പറഞ്ഞു: "എന്നാല്‍ അറിയുക: നീ പോന്നശേഷം നിന്റെ ജനതയെ നാം പരീക്ഷണ വിധേയരാക്കി. സാമിരി അവരെ വഴിപിഴപ്പിച്ചിരിക്കുന്നു.”
\end{malayalam}}
\flushright{\begin{Arabic}
\quranayah[20][86]
\end{Arabic}}
\flushleft{\begin{malayalam}
മൂസ അത്യന്തം കോപിതനും ദുഃഖിതനുമായി തന്റെ ജനതയിലേക്ക് മടങ്ങിച്ചെന്നു. അദ്ദേഹം പറഞ്ഞു: "എന്റെ ജനമേ, നിങ്ങളുടെ നാഥന്‍ നിങ്ങള്‍ക്ക് നല്ല വാഗ്ദാനം നല്‍കിയിരുന്നില്ലേ? എന്നിട്ട് കാലമേറെ നീണ്ടുപോയോ? അതല്ലെങ്കില്‍ നിങ്ങളുടെ നാഥന്റെ കോപം നിങ്ങളില്‍ വന്നുപതിക്കണമെന്ന് നിങ്ങളാഗ്രഹിച്ചോ? അതുകൊണ്ടാണോ നിങ്ങളെന്നോടുള്ള വാഗ്ദാനം ലംഘിച്ചത്?”
\end{malayalam}}
\flushright{\begin{Arabic}
\quranayah[20][87]
\end{Arabic}}
\flushleft{\begin{malayalam}
അവര്‍ പറഞ്ഞു: "അങ്ങയോടുള്ള വാഗ്ദാനം ഞങ്ങള്‍ സ്വയമാഗ്രഹിച്ച് ലംഘിച്ചതല്ല. എന്നാല്‍ വന്നുഭവിച്ചതങ്ങനെയാണ്. ഈ ജനതയുടെ ആഭരണങ്ങളുടെ ചുമടുകള്‍ ഞങ്ങള്‍ വഹിക്കേണ്ടിവന്നിരുന്നുവല്ലോ. ഞങ്ങളത് തീയിലെറിഞ്ഞു. അപ്പോള്‍ അതേപ്രകാരം സാമിരിയും അത് തീയിലിട്ടു.
\end{malayalam}}
\flushright{\begin{Arabic}
\quranayah[20][88]
\end{Arabic}}
\flushleft{\begin{malayalam}
സാമിരി അവര്‍ക്ക് അതുകൊണ്ട് മുക്രയിടുന്ന ഒരു കാളക്കിടാവിന്റെ രൂപമുണ്ടാക്കിക്കൊടുത്തു. അപ്പോള്‍ അവരന്യോന്യം പറഞ്ഞു: "ഇതാകുന്നു നിങ്ങളുടെ ദൈവം. മൂസയുടെ ദൈവവും ഇതുതന്നെ. മൂസയിതു മറന്നുപോയതാണ്.”
\end{malayalam}}
\flushright{\begin{Arabic}
\quranayah[20][89]
\end{Arabic}}
\flushleft{\begin{malayalam}
എന്നാല്‍ ആ കാളക്കിടാവ് ഒരു വാക്കുപോലും ശബ്ദിക്കുന്നില്ലെന്നും അവര്‍ക്കൊരുവിധ ഉപദ്രവമോ ഉപകാരമോ ചെയ്യാനതിനു കഴിയില്ലെന്നും അവര്‍ക്കെന്തുകൊണ്ട് കാണാന്‍ കഴിയുന്നില്ല?
\end{malayalam}}
\flushright{\begin{Arabic}
\quranayah[20][90]
\end{Arabic}}
\flushleft{\begin{malayalam}
ഹാറൂന്‍ നേരത്തെ തന്നെ അവരോടിങ്ങനെ പറഞ്ഞിരുന്നു: "എന്റെ ജനമേ ഈ കാളക്കിടാവ് വഴി നിങ്ങള്‍ പരീക്ഷിക്കപ്പെടുകയാണ്. നിങ്ങളുടെ നാഥന്‍ പരമകാരുണികനാണ്. അതിനാല്‍ നിങ്ങളെന്നെ പിന്‍പറ്റുക. എന്റെ കല്‍പനയനുസരിക്കുക.”
\end{malayalam}}
\flushright{\begin{Arabic}
\quranayah[20][91]
\end{Arabic}}
\flushleft{\begin{malayalam}
അവര്‍ പറഞ്ഞു: "മൂസ ഞങ്ങളുടെ അടുത്ത് മടങ്ങിയെത്തുംവരെ ഞങ്ങളിതിനെത്തന്നെ പൂജിച്ചുകൊണ്ടേയിരിക്കും.”
\end{malayalam}}
\flushright{\begin{Arabic}
\quranayah[20][92]
\end{Arabic}}
\flushleft{\begin{malayalam}
മൂസ ചോദിച്ചു: "ഹാറൂനേ, ഇവര്‍ പിഴച്ചുപോകുന്നതു കണ്ടപ്പോള്‍ നിന്നെ തടഞ്ഞതെന്ത്?
\end{malayalam}}
\flushright{\begin{Arabic}
\quranayah[20][93]
\end{Arabic}}
\flushleft{\begin{malayalam}
എന്നെ പിന്തുടരുന്നതില്‍നിന്ന്; നീ എന്റെ കല്‍പന ധിക്കരിക്കുകയായിരുന്നോ?”
\end{malayalam}}
\flushright{\begin{Arabic}
\quranayah[20][94]
\end{Arabic}}
\flushleft{\begin{malayalam}
ഹാറൂന്‍ പറഞ്ഞു: "എന്റെ മാതാവിന്റെ മകനേ, നീയെന്റെ താടിയും തലമുടിയും പിടിച്ചുവലിക്കല്ലേ? “നീ ഇസ്രയേല്‍ മക്കള്‍ക്കിടയില്‍ ഭിന്നിപ്പുണ്ടാക്കി. എന്റെ വാക്കിനു കാത്തിരുന്നില്ല” എന്ന് നീ പറയുമെന്ന് ഞാന്‍ ഭയപ്പെട്ടു.”
\end{malayalam}}
\flushright{\begin{Arabic}
\quranayah[20][95]
\end{Arabic}}
\flushleft{\begin{malayalam}
മൂസ ചോദിച്ചു: "സാമിരി, നിന്റെ നിലപാടെന്താണ്?”
\end{malayalam}}
\flushright{\begin{Arabic}
\quranayah[20][96]
\end{Arabic}}
\flushleft{\begin{malayalam}
സാമിരി പറഞ്ഞു: "ഇവര്‍ കാണാത്ത ചിലത് ഞാന്‍ കണ്ടു. അങ്ങനെ ദൈവദൂതന്റെ കാല്‍ച്ചുവട്ടില്‍നിന്ന് ഞാനൊരു പിടി മണ്ണെടുത്തു. എന്നിട്ട് ഞാനത് താഴെയിട്ടു. അങ്ങനെ ചെയ്യാനാണ് എന്റെ മനസ്സെന്നോട് മന്ത്രിച്ചത്.”
\end{malayalam}}
\flushright{\begin{Arabic}
\quranayah[20][97]
\end{Arabic}}
\flushleft{\begin{malayalam}
മൂസ പറഞ്ഞു: എങ്കില്‍ നിനക്കു പോകാം. ഇനി ജീവിതകാലം മുഴുവന്‍ നീ “എന്നെ തൊടരുതേ” എന്ന് വിലപിച്ചു കഴിയേണ്ടിവരും. ഉറപ്പായും നിനക്കൊരു നിശ്ചിത അവധിയുണ്ട്. അതൊരിക്കലും ലംഘിക്കപ്പെടുകയില്ല. നീ പൂജിച്ചുകൊണ്ടിരുന്ന ആ ദൈവത്തെ നോക്കൂ. നിശ്ചയമായും നാം അതിനെ ചുട്ടുകരിക്കുക തന്നെ ചെയ്യും. പിന്നെ നാമതിനെ ചാരമാക്കി കടലില്‍ വിതറും.
\end{malayalam}}
\flushright{\begin{Arabic}
\quranayah[20][98]
\end{Arabic}}
\flushleft{\begin{malayalam}
നിങ്ങളുടെ ദൈവം അല്ലാഹു മാത്രമാണ്. അവനല്ലാതെ ദൈവമില്ല. അവന്റെ അറിവ് സകലതിനെയും ഉള്‍ക്കൊള്ളുംവിധം വിശാലമാണ്.
\end{malayalam}}
\flushright{\begin{Arabic}
\quranayah[20][99]
\end{Arabic}}
\flushleft{\begin{malayalam}
ഇങ്ങനെ മുമ്പു കഴിഞ്ഞുപോയ സംഭവങ്ങളുടെ വിവരങ്ങളൊക്കെ നാം നിനക്ക് വിശദീകരിച്ചുതരുന്നു. തീര്‍ച്ചയായും നാം നിനക്ക് നമ്മില്‍നിന്നുള്ള ഈ ഖുര്‍ആനാകുന്ന ഉദ്ബോധനം നല്‍കിയിരിക്കുന്നു.
\end{malayalam}}
\flushright{\begin{Arabic}
\quranayah[20][100]
\end{Arabic}}
\flushleft{\begin{malayalam}
അതിനെ അവഗണിക്കുന്നവന്‍ ഉറപ്പായും ഉയിര്‍ത്തെഴുന്നേല്‍പുനാളില്‍ പാപഭാരം പേറേണ്ടിവരും.
\end{malayalam}}
\flushright{\begin{Arabic}
\quranayah[20][101]
\end{Arabic}}
\flushleft{\begin{malayalam}
അവര്‍ അതുമായി എന്നെന്നും കഴിയേണ്ടിവരും. ഉയിര്‍ത്തെഴുന്നേല്‍പുനാളില്‍ ആ ഭാരം അവര്‍ക്ക് ഏറെ ദുസ്സഹമായിരിക്കും.
\end{malayalam}}
\flushright{\begin{Arabic}
\quranayah[20][102]
\end{Arabic}}
\flushleft{\begin{malayalam}
കാഹളംവിളി മുഴങ്ങുന്നദിനം നാം കുറ്റവാളികളെ കണ്ണു നീലിച്ചവരായി ഒരുമിച്ചുകൂട്ടും.
\end{malayalam}}
\flushright{\begin{Arabic}
\quranayah[20][103]
\end{Arabic}}
\flushleft{\begin{malayalam}
അന്ന് അവര്‍ അന്യോന്യം പിറുപിറുക്കും: “ഭൂമിയില്‍ നിങ്ങള്‍ പത്തുനാളല്ലാതെ കഴിഞ്ഞുകാണില്ല.”
\end{malayalam}}
\flushright{\begin{Arabic}
\quranayah[20][104]
\end{Arabic}}
\flushleft{\begin{malayalam}
അവരെന്താണ് പിറുപിറുത്തുകൊണ്ടിരിക്കുന്നതെന്ന് നന്നായറിയുന്നവന്‍ നാമാണ്. അവരിലെ ഏറ്റം ന്യായമായ നിലപാടുകാരന്‍ പറയും: "നിങ്ങള്‍ ഒരു ദിവസമേ അവിടെ താമസിച്ചിട്ടുള്ളൂ.” അതും നാമറിയുന്നു.
\end{malayalam}}
\flushright{\begin{Arabic}
\quranayah[20][105]
\end{Arabic}}
\flushleft{\begin{malayalam}
അന്നെന്തായിരിക്കും പര്‍വതങ്ങളുടെ സ്ഥിതിയെന്ന് അവര്‍ നിന്നോട് ചോദിക്കുന്നു: പറയുക: "എന്റെ നാഥന്‍ അവയെ പൊടിയാക്കി പറത്തിക്കളയും.”
\end{malayalam}}
\flushright{\begin{Arabic}
\quranayah[20][106]
\end{Arabic}}
\flushleft{\begin{malayalam}
അങ്ങനെ അവന്‍ അതിനെ നിരന്ന മൈതാനിയാക്കും.
\end{malayalam}}
\flushright{\begin{Arabic}
\quranayah[20][107]
\end{Arabic}}
\flushleft{\begin{malayalam}
അന്ന് അവിടെ നിനക്കു കയറ്റിറക്കങ്ങള്‍ കാണാനാവില്ല.
\end{malayalam}}
\flushright{\begin{Arabic}
\quranayah[20][108]
\end{Arabic}}
\flushleft{\begin{malayalam}
അന്ന് അവര്‍ ഒരു വിളിയാളനെ ഒട്ടും സങ്കോചമില്ലാതെ പിന്തുടരും. സകല ശബ്ദവും പരമകാരുണികനായ അല്ലാഹുവിന് കീഴൊതുങ്ങും. അതിനാല്‍ നേര്‍ത്ത ശബ്ദമല്ലാതൊന്നും നീ കേള്‍ക്കുകയില്ല.
\end{malayalam}}
\flushright{\begin{Arabic}
\quranayah[20][109]
\end{Arabic}}
\flushleft{\begin{malayalam}
അന്ന് ശിപാര്‍ശ ഉപകരിക്കുകയില്ല. പരമകാരുണികനായ അല്ലാഹു ആര്‍ക്കുവേണ്ടി അതിനനുമതി നല്‍കുകയും ആരുടെ വാക്ക് തൃപ്തിപ്പെടുകയും ചെയ്യുന്നുവോ അവര്‍ക്കല്ലാതെ.
\end{malayalam}}
\flushright{\begin{Arabic}
\quranayah[20][110]
\end{Arabic}}
\flushleft{\begin{malayalam}
അവരുടെ കഴിഞ്ഞതും വരാനിരിക്കുന്നതുമായ എല്ലാ കാര്യങ്ങളും അവനറിയുന്നു. അവരോ, അതേക്കുറിച്ച് ഒന്നും അറിയുന്നില്ല.
\end{malayalam}}
\flushright{\begin{Arabic}
\quranayah[20][111]
\end{Arabic}}
\flushleft{\begin{malayalam}
എന്നെന്നും ജീവിച്ചിരിക്കുന്നവനും എല്ലാം നോക്കിനടത്തുന്നവനുമായ അല്ലാഹുവിന് സകല മനുഷ്യരും അന്ന് കീഴൊതുങ്ങും. അക്രമത്തിന്റെ പാപഭാരം പേറിവന്നവര്‍ അന്ന് തുലഞ്ഞതുതന്നെ.
\end{malayalam}}
\flushright{\begin{Arabic}
\quranayah[20][112]
\end{Arabic}}
\flushleft{\begin{malayalam}
എന്നാല്‍ സത്യവിശ്വാസിയായി സല്‍ക്കര്‍മങ്ങള്‍ പ്രവര്‍ത്തിക്കുന്നവന്‍ അക്രമത്തെയോ അനീതിയെയോ അല്‍പവും ഭയപ്പെടേണ്ടിവരില്ല.
\end{malayalam}}
\flushright{\begin{Arabic}
\quranayah[20][113]
\end{Arabic}}
\flushleft{\begin{malayalam}
ഇങ്ങനെ നാമിതിനെ അറബി ഭാഷയിലുള്ള ഖുര്‍ആന്‍ ആയി ഇറക്കിത്തന്നിരിക്കുന്നു. നാം ഇതില്‍ പലതരം താക്കീതുകള്‍ നല്‍കിയിരിക്കുന്നു. ഒരുവേള ഇക്കൂട്ടര്‍ ഭക്തരായെങ്കിലോ; അല്ലെങ്കില്‍ ഇവര്‍ കാര്യബോധമുള്ളവരായെങ്കിലോ!
\end{malayalam}}
\flushright{\begin{Arabic}
\quranayah[20][114]
\end{Arabic}}
\flushleft{\begin{malayalam}
സാക്ഷാല്‍ അധിപതിയായ അല്ലാഹു അത്യുന്നതനാണ്. ഖുര്‍ആന്‍ നിനക്കു ബോധനം നല്‍കിക്കഴിയും മുമ്പെ നീയതു വായിക്കാന്‍ ധൃതികാണിക്കരുത്. നീയിങ്ങനെ പ്രാര്‍ഥിച്ചുകൊണ്ടിരിക്കുക: "എന്റെ നാഥാ! എനിക്കു നീ വിജ്ഞാനം വര്‍ധിപ്പിച്ചു തരേണമേ.”
\end{malayalam}}
\flushright{\begin{Arabic}
\quranayah[20][115]
\end{Arabic}}
\flushleft{\begin{malayalam}
നാം ഇതിനു മുമ്പ് ആദമിനോടും കരാര്‍ ചെയ്തിരുന്നു. പക്ഷേ, അദ്ദേഹമത് മറന്നു. അദ്ദേഹത്തെ നാം ഇച്ഛാശക്തിയുള്ളവനായി കണ്ടില്ല.
\end{malayalam}}
\flushright{\begin{Arabic}
\quranayah[20][116]
\end{Arabic}}
\flushleft{\begin{malayalam}
നാം മലക്കുകളോട് പറഞ്ഞതോര്‍ക്കുക: "നിങ്ങള്‍ ആദമിന് സാഷ്ടാംഗം പ്രണമിക്കുക.” അപ്പോള്‍ അവരെല്ലാം പ്രണമിച്ചു; ഇബ്ലീസൊഴികെ. അവന്‍ വിസമ്മതിച്ചു.
\end{malayalam}}
\flushright{\begin{Arabic}
\quranayah[20][117]
\end{Arabic}}
\flushleft{\begin{malayalam}
അപ്പോള്‍ നാം പറഞ്ഞു: "ആദമേ, തീര്‍ച്ചയായും അവന്‍ നിന്റെയും നിന്റെ ഇണയുടെയും ശത്രുവാണ്. അതിനാല്‍ അവന്‍ നിങ്ങളിരുവരെയും സ്വര്‍ഗത്തില്‍നിന്ന് പുറത്താക്കാന്‍ ഇടവരാതിരിക്കട്ടെ. അങ്ങനെ സംഭവിച്ചാല്‍ നീ ഏറെ നിര്‍ഭാഗ്യവാനായിത്തീരും.
\end{malayalam}}
\flushright{\begin{Arabic}
\quranayah[20][118]
\end{Arabic}}
\flushleft{\begin{malayalam}
"തീര്‍ച്ചയായും നിനക്കിവിടെ വിശപ്പറിയാതെയും നഗ്നനാകാതെയും കഴിയാനുള്ള സൌകര്യമുണ്ട്.
\end{malayalam}}
\flushright{\begin{Arabic}
\quranayah[20][119]
\end{Arabic}}
\flushleft{\begin{malayalam}
"ദാഹമനുഭവിക്കാതെയും ചൂടേല്‍ക്കാതെയും ജീവിക്കാം.”
\end{malayalam}}
\flushright{\begin{Arabic}
\quranayah[20][120]
\end{Arabic}}
\flushleft{\begin{malayalam}
എന്നാല്‍ പിശാച് അദ്ദേഹത്തിന് ഇങ്ങനെ ദുര്‍ബോധനം നല്‍കി: "ആദമേ, താങ്കള്‍ക്ക് നിത്യജീവിതവും അന്യൂനമായ ആധിപത്യവും നല്‍കുന്ന ഒരു വൃക്ഷം കാണിച്ചുതരട്ടെയോ?”
\end{malayalam}}
\flushright{\begin{Arabic}
\quranayah[20][121]
\end{Arabic}}
\flushleft{\begin{malayalam}
അങ്ങനെ അവരിരുവരും ആ വൃക്ഷത്തില്‍നിന്ന് ഭക്ഷിച്ചു. അതോടെ അവര്‍ക്കിരുവര്‍ക്കും തങ്ങളുടെ നഗ്നത വെളിവായി. ഇരുവരും സ്വര്‍ഗത്തിലെ ഇലകള്‍കൊണ്ട് തങ്ങളെ പൊതിയാന്‍ തുടങ്ങി. ആദം തന്റെ നാഥനെ ധിക്കരിച്ചു. അങ്ങനെ പിഴച്ചുപോയി.
\end{malayalam}}
\flushright{\begin{Arabic}
\quranayah[20][122]
\end{Arabic}}
\flushleft{\begin{malayalam}
പിന്നീട് തന്റെ നാഥന്‍ അദ്ദേഹത്തെ തെരഞ്ഞെടുത്തു. അദ്ദേഹത്തിന്റെ പശ്ചാത്താപം സ്വീകരിച്ചു. അദ്ദേഹത്തെ നേര്‍വഴിയില്‍ നയിച്ചു.
\end{malayalam}}
\flushright{\begin{Arabic}
\quranayah[20][123]
\end{Arabic}}
\flushleft{\begin{malayalam}
അല്ലാഹു ആജ്ഞാപിച്ചു: നിങ്ങളിരുകൂട്ടരും ഒന്നിച്ച് ഇവിടെ നിന്നിറങ്ങിപ്പോകണം. നിങ്ങള്‍ പരസ്പരം ശത്രുക്കളായിരിക്കും. എന്നാല്‍ എന്നില്‍നിന്നുള്ള മാര്‍ഗദര്‍ശനം നിങ്ങള്‍ക്ക് വന്നെത്തുമ്പോള്‍ ആരത് പിന്‍പറ്റുന്നുവോ അവന്‍ വഴിപിഴക്കുകയില്ല. ഭാഗ്യംകെട്ടവനാവുകയില്ല.
\end{malayalam}}
\flushright{\begin{Arabic}
\quranayah[20][124]
\end{Arabic}}
\flushleft{\begin{malayalam}
എന്റെ ഉദ്ബോധനത്തെ അവഗണിക്കുന്നവന്ന് ഈ ലോകത്ത് ഇടുങ്ങിയ ജീവിതമാണുണ്ടാവുക. പുനരുത്ഥാനനാളില്‍ നാമവനെ കണ്ണുപൊട്ടനായാണ് ഉയിര്‍ത്തെഴുന്നേല്‍പിക്കുക.
\end{malayalam}}
\flushright{\begin{Arabic}
\quranayah[20][125]
\end{Arabic}}
\flushleft{\begin{malayalam}
അപ്പോള്‍ അവന്‍ പറയും: "എന്റെ നാഥാ; നീയെന്തിനാണെന്നെ കണ്ണുപൊട്ടനാക്കി ഉയിര്‍ത്തെഴുന്നേല്‍പിച്ചത്? ഞാന്‍ കാഴ്ചയുള്ളവനായിരുന്നുവല്ലോ.”
\end{malayalam}}
\flushright{\begin{Arabic}
\quranayah[20][126]
\end{Arabic}}
\flushleft{\begin{malayalam}
അല്ലാഹു പറയും: "ശരിയാണ്. നമ്മുടെ പ്രമാണങ്ങള്‍ നിനക്കു വന്നെത്തിയിരുന്നു. അപ്പോള്‍ നീ അവയെ വിസ്മരിച്ചു. അവ്വിധം ഇന്ന് നീയും വിസ്മരിക്കപ്പെടുകയാണ്.”
\end{malayalam}}
\flushright{\begin{Arabic}
\quranayah[20][127]
\end{Arabic}}
\flushleft{\begin{malayalam}
അതിരു കവിയുകയും തന്റെ നാഥന്റെ വചനങ്ങളില്‍ വിശ്വസിക്കാതിരിക്കുകയും ചെയ്തവര്‍ക്ക് നാം ഇവ്വിധമാണ് പ്രതിഫലം നല്‍കുക. പരലോകശിക്ഷ കൂടുതല്‍ കഠിനവും ദീര്‍ഘവുമാണ്.
\end{malayalam}}
\flushright{\begin{Arabic}
\quranayah[20][128]
\end{Arabic}}
\flushleft{\begin{malayalam}
ഇവര്‍ക്കുമുമ്പ് എത്രയോ തലമുറകളെ നാം നിശ്ശേഷം നശിപ്പിച്ചിട്ടുണ്ട്. അവരുടെ വാസസ്ഥലങ്ങളിലൂടെയാണ് ഇവരിന്ന് സഞ്ചരിച്ചുകൊണ്ടിരിക്കുന്നത്. എന്നിട്ടും ഇതൊന്നും ഇവര്‍ക്ക് മാര്‍ഗദര്‍ശകമാവുന്നില്ലേ? തീര്‍ച്ചയായും വിചാരമതികള്‍ക്ക് ഇതില്‍ ധാരാളം ദൃഷ്ടാന്തങ്ങളുണ്ട്.
\end{malayalam}}
\flushright{\begin{Arabic}
\quranayah[20][129]
\end{Arabic}}
\flushleft{\begin{malayalam}
നിന്റെ നാഥനില്‍നിന്നുള്ള തീരുമാന വിളംബരം നേരത്തെ ഉണ്ടാവുകയും അതിനു കാലാവധി നിശ്ചയിക്കുകയും ചെയ്തിട്ടുണ്ടായിരുന്നില്ലെങ്കില്‍ ഇവര്‍ക്കും ശിക്ഷ അനിവാര്യമാകുമായിരുന്നു.
\end{malayalam}}
\flushright{\begin{Arabic}
\quranayah[20][130]
\end{Arabic}}
\flushleft{\begin{malayalam}
അതിനാല്‍ ഇവര്‍ പറയുന്നതൊക്കെ ക്ഷമിക്കുക. സൂര്യോദയത്തിനും അസ്തമയത്തിനും മുമ്പ് നിന്റെ നാഥനെ കീര്‍ത്തിച്ച് അവന്റെ വിശുദ്ധി വാഴ്ത്തുക. രാവിന്റെ ചില യാമങ്ങളിലും പകലിന്റെ രണ്ടറ്റങ്ങളിലും അവന്റെ പരിശുദ്ധിയെ പ്രകീര്‍ത്തിക്കുക. നിനക്കു സംതൃപ്തി ലഭിച്ചേക്കാം.
\end{malayalam}}
\flushright{\begin{Arabic}
\quranayah[20][131]
\end{Arabic}}
\flushleft{\begin{malayalam}
മനുഷ്യരില്‍ വിവിധ വിഭാഗങ്ങള്‍ക്കു നാം നല്‍കിയ ഐഹിക സുഖാഢംബരങ്ങളില്‍ നീ കണ്ണുവെക്കരുത്. അതിലൂടെ നാമവരെ പരീക്ഷിക്കുകയാണ്. നിന്റെ നാഥന്റെ ഉപജീവനമാണ് ഉല്‍കൃഷ്ടം. നിലനില്‍ക്കുന്നതും അതുതന്നെ.
\end{malayalam}}
\flushright{\begin{Arabic}
\quranayah[20][132]
\end{Arabic}}
\flushleft{\begin{malayalam}
നിന്റെ കുടുംബത്തോടു നീ നമസ്കരിക്കാന്‍ കല്‍പിക്കുക. നീയതില്‍ ക്ഷമയോടെ ഉറച്ചുനില്‍ക്കുകയും ചെയ്യുക. നാം നിന്നോട് ജീവിതവിഭവമൊന്നും ആവശ്യപ്പെടുന്നില്ല. മറിച്ച് നിനക്ക് ജീവിതവിഭവം നല്‍കുന്നത് നാമാണ്. ഭക്തിക്കാണ് ശുഭാന്ത്യം.
\end{malayalam}}
\flushright{\begin{Arabic}
\quranayah[20][133]
\end{Arabic}}
\flushleft{\begin{malayalam}
അവര്‍ ചോദിക്കുന്നു: "ഇയാള്‍ തന്റെ നാഥനില്‍നിന്ന് ദൈവികമായ അടയാളമൊന്നും കൊണ്ടുവരാത്തതെന്ത്?” പൂര്‍വവേദങ്ങളിലെ വ്യക്തമായ തെളിവുകളൊന്നും അവര്‍ക്കു വന്നുകിട്ടിയിട്ടില്ലേ?
\end{malayalam}}
\flushright{\begin{Arabic}
\quranayah[20][134]
\end{Arabic}}
\flushleft{\begin{malayalam}
ഇതിനു മുമ്പ് വല്ല കടുത്ത ശിക്ഷയും നല്‍കി നാം ഇവരെ നശിപ്പിച്ചിരുന്നുവെങ്കില്‍ ഇവര്‍ തന്നെ പറയുമായിരുന്നു: "ഞങ്ങളുടെ നാഥാ! നീ എന്തുകൊണ്ട് ഞങ്ങള്‍ക്കൊരു ദൂതനെ അയച്ചുതന്നില്ല? എങ്കില്‍ ഞങ്ങള്‍ അപമാനിതരും പറ്റെ നിന്ദ്യരും ആകും മുമ്പെ നിന്റെ വചനങ്ങളെ പിന്‍പറ്റുമായിരുന്നുവല്ലോ.”
\end{malayalam}}
\flushright{\begin{Arabic}
\quranayah[20][135]
\end{Arabic}}
\flushleft{\begin{malayalam}
പറയുക: എല്ലാവരും അന്തിമമായ തീരുമാനം കാത്തിരിക്കുന്നവരാണ്. നിങ്ങളും കാത്തിരിക്കുക. നേര്‍വഴിയില്‍ നീങ്ങുന്നവര്‍ ആരെന്നും സന്മാര്‍ഗം പ്രാപിച്ചവര്‍ ആരെന്നും ഏറെ വൈകാതെ നിങ്ങളറിയുക തന്നെ ചെയ്യും.
\end{malayalam}}
\chapter{\textmalayalam{അന്‍ബിയാഅ് ( പ്രവാചകന്മാര്‍ )}}
\begin{Arabic}
\Huge{\centerline{\basmalah}}\end{Arabic}
\flushright{\begin{Arabic}
\quranayah[21][1]
\end{Arabic}}
\flushleft{\begin{malayalam}
ജനത്തിന് അവരുടെ വിചാരണാ വേള വളരെ അടുത്തെത്തിയിരിക്കുന്നു. എന്നിട്ടും അവര്‍ അതേക്കുറിച്ച് തീര്‍ത്തും അശ്രദ്ധരാണ്. അതിനെ അപ്പാടെ അവഗണിക്കുന്നവരും.
\end{malayalam}}
\flushright{\begin{Arabic}
\quranayah[21][2]
\end{Arabic}}
\flushleft{\begin{malayalam}
തങ്ങളുടെ നാഥനില്‍നിന്ന് പുതുതായി ഏതു ഉദ്ബോധനം വന്നെത്തുമ്പോഴും അവരത് കേള്‍ക്കുന്നതുതന്നെ കളിതമാശകളില്‍ മുഴുകുന്നവരായാണ്;
\end{malayalam}}
\flushright{\begin{Arabic}
\quranayah[21][3]
\end{Arabic}}
\flushleft{\begin{malayalam}
അശ്രദ്ധമായ മനസ്സോടെയും. ആ അതിക്രമികള്‍ അന്യോന്യം ഇങ്ങനെ അടക്കം പറയുന്നു: "ഇയാള്‍ നിങ്ങളെപ്പോലുള്ള ഒരു മനുഷ്യന്‍ മാത്രമല്ലേ? എന്നിട്ടും നിങ്ങളെന്തിനാണ് ബോധപൂര്‍വം ഈ ജാലവിദ്യയില്‍ ചെന്നുവീഴുന്നത്?”
\end{malayalam}}
\flushright{\begin{Arabic}
\quranayah[21][4]
\end{Arabic}}
\flushleft{\begin{malayalam}
പ്രവാചകന്‍ പറഞ്ഞു: "ആകാശത്തും ഭൂമിയിലും ആരെന്തു പറഞ്ഞാലും അതൊക്കെയും എന്റെ നാഥന്‍ അറിയുന്നു. അവന്‍ എല്ലാം കേള്‍ക്കുന്നവനും അറിയുന്നവനുമാണ്.”
\end{malayalam}}
\flushright{\begin{Arabic}
\quranayah[21][5]
\end{Arabic}}
\flushleft{\begin{malayalam}
അവര്‍ പറയുന്നു: "ഇതൊക്കെ വെറും പൊയ്ക്കിനാവുകളാണ്. അല്ല; ഇവനിത് സ്വയം കെട്ടിച്ചമച്ചതാണ്. ഇയാളൊരു കവിയാണ്. അല്ലെങ്കില്‍ ഇയാള്‍ ഒരു ദൃഷ്ടാന്തം കൊണ്ടുവന്ന് നമ്മെ കാണിക്കട്ടെ. പൂര്‍വപ്രവാചകന്മാര്‍ ചെയ്ത പോലെ.”
\end{malayalam}}
\flushright{\begin{Arabic}
\quranayah[21][6]
\end{Arabic}}
\flushleft{\begin{malayalam}
എന്നാല്‍ ഇവര്‍ക്കു മുമ്പ് നാം നിശ്ശേഷം നശിപ്പിച്ച ഒരു നാടും വിശ്വസിച്ചിട്ടില്ല. ഇനിയിപ്പോള്‍ ഇവരാണോ വിശ്വസിക്കാന്‍ പോകുന്നത്?
\end{malayalam}}
\flushright{\begin{Arabic}
\quranayah[21][7]
\end{Arabic}}
\flushleft{\begin{malayalam}
നിനക്കു മുമ്പും മനുഷ്യരെത്തന്നെയാണ് നാം ദൂതന്മാരായി നിയോഗിച്ചത്. നാം അവര്‍ക്കു ബോധനം നല്‍കുകയായിരുന്നു. നിങ്ങള്‍ക്കിത് അറിയില്ലെങ്കില്‍ വേദക്കാരോട് ചോദിച്ചുനോക്കുക.
\end{malayalam}}
\flushright{\begin{Arabic}
\quranayah[21][8]
\end{Arabic}}
\flushleft{\begin{malayalam}
ദൈവദൂതന്മാര്‍ക്കു നാം അന്നം തിന്നാത്ത ശരീരം നല്‍കിയിട്ടില്ല. അവരിവിടെ സ്ഥിരവാസികളുമായിരുന്നില്ല.
\end{malayalam}}
\flushright{\begin{Arabic}
\quranayah[21][9]
\end{Arabic}}
\flushleft{\begin{malayalam}
പിന്നീട് അവരോടുള്ള വാഗ്ദാനം നാം പാലിച്ചു. അങ്ങനെ നാമവരെ രക്ഷിച്ചു; നാം ഉദ്ദേശിച്ച മറ്റുള്ളവരെയും. അതിരു കടന്നവരെ നശിപ്പിക്കുകയും ചെയ്തു.
\end{malayalam}}
\flushright{\begin{Arabic}
\quranayah[21][10]
\end{Arabic}}
\flushleft{\begin{malayalam}
നിങ്ങള്‍ക്ക് നാം വേദപുസ്തകം ഇറക്കിത്തന്നിരിക്കുന്നു. അതില്‍ നിങ്ങള്‍ക്കുള്ള ഉദ്ബോധനമുണ്ട്. എന്നിട്ടും നിങ്ങള്‍ അതേക്കുറിച്ചൊന്നും ചിന്തിക്കുന്നില്ലേ?
\end{malayalam}}
\flushright{\begin{Arabic}
\quranayah[21][11]
\end{Arabic}}
\flushleft{\begin{malayalam}
അതിക്രമത്തിലേര്‍പ്പെട്ട എത്രയെത്ര നാടുകളെയാണ് നാം നിശ്ശേഷം നശിപ്പിച്ചത്! അവര്‍ക്കു ശേഷം നാം മറ്റു ജനവിഭാഗങ്ങളെ വളര്‍ത്തിക്കൊണ്ടുവന്നു.
\end{malayalam}}
\flushright{\begin{Arabic}
\quranayah[21][12]
\end{Arabic}}
\flushleft{\begin{malayalam}
നമ്മുടെ ശിക്ഷ അനുഭവിച്ചുതുടങ്ങിയപ്പോള്‍ അവരതാ അവിടെ നിന്ന് ഓടിരക്ഷപ്പെടാന്‍ ശ്രമിക്കുന്നു.
\end{malayalam}}
\flushright{\begin{Arabic}
\quranayah[21][13]
\end{Arabic}}
\flushleft{\begin{malayalam}
അപ്പോഴവരോടു പറയും: "ഓടേണ്ട. നിങ്ങളനുഭവിച്ചുകൊണ്ടിരുന്ന സുഖസൌകര്യങ്ങളിലേക്കും നിങ്ങളുടെ വസതികളിലേക്കും തന്നെ തിരികെ ചെല്ലുക. നിങ്ങളെ ചോദ്യം ചെയ്തേക്കാം.”
\end{malayalam}}
\flushright{\begin{Arabic}
\quranayah[21][14]
\end{Arabic}}
\flushleft{\begin{malayalam}
അവര്‍ പറഞ്ഞു: "അയ്യോ, നമ്മുടെ നാശം! സംശയമില്ല; ഞങ്ങള്‍ അതിക്രമികളായിപ്പോയി.”
\end{malayalam}}
\flushright{\begin{Arabic}
\quranayah[21][15]
\end{Arabic}}
\flushleft{\begin{malayalam}
അവരുടെ ഈ വിലാപം തുടര്‍ന്നുകൊണ്ടേയിരിക്കും. നാമവരെ കൊയ്തിട്ട വൈക്കോല്‍തുരുമ്പ്പോലെ ആക്കുംവരെ.
\end{malayalam}}
\flushright{\begin{Arabic}
\quranayah[21][16]
\end{Arabic}}
\flushleft{\begin{malayalam}
ഈ ആകാശവും ഭൂമിയും അവയ്ക്കിടയിലുള്ളതും നാം കുട്ടിക്കളിയായി ഉണ്ടാക്കിയതല്ല.
\end{malayalam}}
\flushright{\begin{Arabic}
\quranayah[21][17]
\end{Arabic}}
\flushleft{\begin{malayalam}
നാം ഒരു വിനോദമുണ്ടാക്കാനുദ്ദേശിച്ചിരുന്നെങ്കില്‍ നാം സ്വയം തന്നെ അതു ചെയ്യുമായിരുന്നു. എന്നാല്‍ നാമങ്ങനെ ചെയ്തിട്ടില്ല.
\end{malayalam}}
\flushright{\begin{Arabic}
\quranayah[21][18]
\end{Arabic}}
\flushleft{\begin{malayalam}
നാം സത്യംകൊണ്ട് അസത്യത്തെ ഇടിക്കുന്നു. അങ്ങനെ അത് അസത്യത്തെ ഉടയ്ക്കുന്നു. അതോടെ അസത്യം അപ്രത്യക്ഷമാകുന്നു. നിങ്ങള്‍ സങ്കല്‍പിച്ചു പറയുന്നതു കാരണം നിങ്ങള്‍ക്കു നാശം.
\end{malayalam}}
\flushright{\begin{Arabic}
\quranayah[21][19]
\end{Arabic}}
\flushleft{\begin{malayalam}
ആകാശഭൂമികളിലുള്ള സകലതും അല്ലാഹുവിന്റേതാണ്. അവന്റെ അടുത്തുള്ളവര്‍ അവന്ന് വഴിപ്പെടുന്നതിലൊട്ടും അഹങ്കരിക്കുന്നില്ല. അവര്‍ ക്ഷീണിക്കുന്നുമില്ല.
\end{malayalam}}
\flushright{\begin{Arabic}
\quranayah[21][20]
\end{Arabic}}
\flushleft{\begin{malayalam}
ഇടവേളകളില്ലാതെ രാവും പകലും അവനെ അവര്‍ വാഴ്ത്തിക്കൊണ്ടേയിരിക്കുന്നു.
\end{malayalam}}
\flushright{\begin{Arabic}
\quranayah[21][21]
\end{Arabic}}
\flushleft{\begin{malayalam}
ഈ ഭൂമിയില്‍ അവര്‍ സങ്കല്‍പിച്ചുവെച്ച ദൈവങ്ങള്‍ക്ക് മരിച്ചവരെ ജീവിപ്പിക്കാനാവുമോ?
\end{malayalam}}
\flushright{\begin{Arabic}
\quranayah[21][22]
\end{Arabic}}
\flushleft{\begin{malayalam}
ആകാശഭൂമികളില്‍ അല്ലാഹുവല്ലാത്ത വല്ല ദൈവങ്ങളുമുണ്ടായിരുന്നുവെങ്കില്‍ അവ രണ്ടും താറുമാറാകുമായിരുന്നു. ഇക്കൂട്ടര്‍ പറഞ്ഞുപരത്തുന്നതില്‍ നിന്നെല്ലാം എത്രയോ പരിശുദ്ധനാണ് അല്ലാഹു. സിംഹാസനത്തിന്ന് അധിപനാണവന്‍.
\end{malayalam}}
\flushright{\begin{Arabic}
\quranayah[21][23]
\end{Arabic}}
\flushleft{\begin{malayalam}
അവന്‍ പ്രവര്‍ത്തിക്കുന്നതിനെപ്പറ്റി ആരും ചോദ്യംചെയ്യുകയില്ല. എന്നാല്‍ ഉറപ്പായും അവര്‍ ചോദ്യം ചെയ്യപ്പെടും.
\end{malayalam}}
\flushright{\begin{Arabic}
\quranayah[21][24]
\end{Arabic}}
\flushleft{\begin{malayalam}
അതല്ല, അവര്‍ അവനെക്കൂടാതെ മറ്റു ദൈവങ്ങളെ സ്വീകരിച്ചിരിക്കയാണോ? പറയുക: "നിങ്ങള്‍ക്കുള്ള തെളിവ് കൊണ്ടുവരൂ. എന്റെ കൂടെയുള്ളവര്‍ക്കുള്ള ഉദ്ബോധനമാണിത്. എന്റെ മുമ്പുള്ളവര്‍ക്കുള്ള ഉദ്ബോധനവും ഇതു തന്നെയായിരുന്നു.” എന്നാല്‍ അവരിലേറെ പേരും സത്യമറിയുന്നില്ല. അതിനാലവര്‍ പിന്തിരിഞ്ഞുകളയുകയാണ്.
\end{malayalam}}
\flushright{\begin{Arabic}
\quranayah[21][25]
\end{Arabic}}
\flushleft{\begin{malayalam}
“ഞാനല്ലാതെ ദൈവമില്ല. അതിനാല്‍ നിങ്ങള്‍ എനിക്കു വഴിപ്പെടുക” എന്ന സന്ദേശം നല്‍കിക്കൊണ്ടല്ലാതെ നിനക്കു മുമ്പ് ഒരു ദൂതനെയും നാം അയച്ചിട്ടില്ല.
\end{malayalam}}
\flushright{\begin{Arabic}
\quranayah[21][26]
\end{Arabic}}
\flushleft{\begin{malayalam}
അവര്‍ പറയുന്നു: "പരമ കാരുണികനായ ദൈവം പുത്രനെ സ്വീകരിച്ചിരിക്കുന്നു.” എന്നാല്‍ അവനെത്ര പരിശുദ്ധന്‍! അവര്‍ അവന്റെ ആദരണീയരായ അടിമകള്‍ മാത്രമാണ്.
\end{malayalam}}
\flushright{\begin{Arabic}
\quranayah[21][27]
\end{Arabic}}
\flushleft{\begin{malayalam}
അവര്‍ അവനെ മറികടന്നു സംസാരിക്കുകയില്ല. അവന്റെ കല്‍പനയനുസരിച്ചാണ് അവര്‍ പ്രവര്‍ത്തിക്കുന്നത്.
\end{malayalam}}
\flushright{\begin{Arabic}
\quranayah[21][28]
\end{Arabic}}
\flushleft{\begin{malayalam}
അവരുടെ മുന്നിലും പിന്നിലുമുള്ള സകലതും അവനറിയുന്നു. അവരുടെ നാഥന്‍ തൃപ്തിപ്പെട്ടവര്‍ക്കു വേണ്ടിയല്ലാതെ ആര്‍ക്കുമവര്‍ ശുപാര്‍ശ ചെയ്യുകയില്ല. അവരോ, അവനോടുള്ള ഭയത്താല്‍ നടുക്കമനുഭവിക്കുന്നവരാണ്.
\end{malayalam}}
\flushright{\begin{Arabic}
\quranayah[21][29]
\end{Arabic}}
\flushleft{\begin{malayalam}
അവരിലാരെങ്കിലും അല്ലാഹുവെക്കൂടാതെ താനും ദൈവമാണെന്ന് വാദിച്ചാല്‍ പ്രതിഫലമായി നാമവന്ന് നരകശിക്ഷ നല്‍കും. അവ്വിധമാണ് നാം അതിക്രമികള്‍ക്ക് പ്രതിഫലം നല്‍കുക.
\end{malayalam}}
\flushright{\begin{Arabic}
\quranayah[21][30]
\end{Arabic}}
\flushleft{\begin{malayalam}
ആകാശങ്ങളും ഭൂമിയും പരസ്പരം ഒട്ടിച്ചേര്‍ന്നവയായിരുന്നു. എന്നിട്ട് നാമവയെ വേര്‍പെടുത്തി. വെള്ളത്തില്‍നിന്ന് ജീവനുള്ള എല്ലാ വസ്തുക്കളെയും സൃഷ്ടിച്ചു. സത്യനിഷേധികള്‍ ഇതൊന്നും കാണുന്നില്ലേ? അങ്ങനെ അവര്‍ വിശ്വസിക്കുന്നില്ലേ?
\end{malayalam}}
\flushright{\begin{Arabic}
\quranayah[21][31]
\end{Arabic}}
\flushleft{\begin{malayalam}
ഭൂമിയില്‍ നാം പര്‍വതങ്ങളെ ഉറപ്പിച്ചുനിര്‍ത്തി. ഭൂമി അവരെയുംകൊണ്ട് ഉലഞ്ഞുപോകാതിരിക്കാന്‍. നാമതില്‍ സൌകര്യപ്രദവും വിശാലവുമായ വഴികളുണ്ടാക്കി. അവര്‍ക്ക് നേര്‍വഴിയറിയാന്‍.
\end{malayalam}}
\flushright{\begin{Arabic}
\quranayah[21][32]
\end{Arabic}}
\flushleft{\begin{malayalam}
മാനത്തെ നാം സുരക്ഷിതമായ മേല്‍പ്പുരയാക്കി. എന്നിട്ടും അവരതിലെ ദൃഷ്ടാന്തങ്ങളെ അവഗണിക്കുകയാണ്.
\end{malayalam}}
\flushright{\begin{Arabic}
\quranayah[21][33]
\end{Arabic}}
\flushleft{\begin{malayalam}
രാപ്പകലുകള്‍ സൃഷ്ടിച്ചത് അവനാണ്. സൂര്യചന്ദ്രന്മാരെ പടച്ചതും അവന്‍തന്നെ. അവയൊക്കെയും ഓരോ സഞ്ചാരപഥത്തില്‍ ചരിച്ചുകൊണ്ടിരിക്കുകയാണ്.
\end{malayalam}}
\flushright{\begin{Arabic}
\quranayah[21][34]
\end{Arabic}}
\flushleft{\begin{malayalam}
നിനക്ക് മുമ്പ് നാം ഒരു മനുഷ്യന്നും നിത്യത നല്‍കിയിട്ടില്ല. എന്നിരിക്കെ നീ മരിച്ചെന്നു വരികില്‍ അതില്‍ അസാധാരണമായി എന്തുണ്ട്? ഇക്കൂട്ടര്‍ എന്നെന്നും ജീവിച്ചിരിക്കുന്നവരാണോ?
\end{malayalam}}
\flushright{\begin{Arabic}
\quranayah[21][35]
\end{Arabic}}
\flushleft{\begin{malayalam}
എല്ലാ ജീവികളും മരണം രുചിക്കുകതന്നെ ചെയ്യും. ഗുണദോഷങ്ങള്‍ നല്‍കി നിങ്ങളെ നാം പരീക്ഷിച്ചുകൊണ്ടിരിക്കുകയാണ്. നിങ്ങളുടെയൊക്കെ മടക്കം നമ്മുടെയടുത്തേക്കാണ്.
\end{malayalam}}
\flushright{\begin{Arabic}
\quranayah[21][36]
\end{Arabic}}
\flushleft{\begin{malayalam}
നിന്നെ കാണുമ്പോള്‍ പരിഹസിക്കലല്ലാതെ സത്യനിഷേധികള്‍ക്ക് വേറെ പണിയൊന്നുമില്ല. അവര്‍ പുച്ഛത്തോടെ പറയുന്നു: "ഇവനാണോ നിങ്ങളുടെ ദൈവങ്ങളെ ചോദ്യം ചെയ്യുന്നവന്‍?” എന്നാല്‍, അവര്‍ തന്നെയാണ് പരമകാരുണികനായ അല്ലാഹുവിന്റെ ഉദ്ബോധനത്തെ കള്ളമാക്കി തള്ളുന്നവര്‍.
\end{malayalam}}
\flushright{\begin{Arabic}
\quranayah[21][37]
\end{Arabic}}
\flushleft{\begin{malayalam}
ധൃതി കാട്ടുന്നവനായാണ് മനുഷ്യന്‍ സൃഷ്ടിക്കപ്പെട്ടത്. വൈകാതെ തന്നെ ഞാനെന്റെ ദൃഷ്ടാന്തങ്ങള്‍ നിങ്ങള്‍ക്കു കാട്ടിത്തരും. അതിനാല്‍ നിങ്ങളെന്നോട് ധൃതികൂട്ടേണ്ടതില്ല.
\end{malayalam}}
\flushright{\begin{Arabic}
\quranayah[21][38]
\end{Arabic}}
\flushleft{\begin{malayalam}
അവര്‍ ചോദിക്കുന്നു: "നിങ്ങളുടെ ഈ വാഗ്ദാനം എപ്പോഴാണ് പുലരുക? നിങ്ങള്‍ സത്യസന്ധരെങ്കില്‍.”
\end{malayalam}}
\flushright{\begin{Arabic}
\quranayah[21][39]
\end{Arabic}}
\flushleft{\begin{malayalam}
തങ്ങളുടെ മുഖങ്ങളെയും മുതുകുകളെയും നരകത്തീയില്‍ നിന്ന് തടുക്കാനാവാത്ത, എങ്ങുനിന്നും ഒരു സഹായവും കിട്ടാത്ത അവസ്ഥയെ സംബന്ധിച്ച് സത്യനിഷേധികള്‍ അറിഞ്ഞിരുന്നെങ്കില്‍!
\end{malayalam}}
\flushright{\begin{Arabic}
\quranayah[21][40]
\end{Arabic}}
\flushleft{\begin{malayalam}
എന്നാല്‍ വളരെ പെട്ടെന്നായിരിക്കും അതവരില്‍ വന്നെത്തുക. അപ്പോള്‍ അതവരെ അമ്പരപ്പിക്കും. അതിനെ തടുക്കാനവര്‍ക്കാവില്ല. അവര്‍ക്കൊട്ടും അവസരംനല്‍കുകയുമില്ല.
\end{malayalam}}
\flushright{\begin{Arabic}
\quranayah[21][41]
\end{Arabic}}
\flushleft{\begin{malayalam}
നിനക്കു മുമ്പും പല പ്രവാചകന്മാരും ഇവ്വിധം പരിഹസിക്കപ്പെട്ടിട്ടുണ്ട്. എന്നിട്ടോ, തങ്ങള്‍ ഏതൊന്നിനെപ്പറ്റിയാണോ പരിഹസിച്ചുകൊണ്ടിരുന്നത് ആ ശിക്ഷ പരിഹസിച്ചിരുന്നവരെ പിടികൂടുക തന്നെ ചെയ്തു.
\end{malayalam}}
\flushright{\begin{Arabic}
\quranayah[21][42]
\end{Arabic}}
\flushleft{\begin{malayalam}
ചോദിക്കുക: രാവിലാവട്ടെ പകലിലാവട്ടെ, പരമ ദയാലുവായ അല്ലാഹുവിന്റെ പിടുത്തത്തില്‍നിന്ന് നിങ്ങളെ രക്ഷിക്കാന്‍ കഴിയുന്ന ആരുണ്ട്? എന്നിട്ടും അവര്‍ തങ്ങളുടെ നാഥന്റെ ഉദ്ബോധനം അവഗണിച്ചുതള്ളുകയാണ്!
\end{malayalam}}
\flushright{\begin{Arabic}
\quranayah[21][43]
\end{Arabic}}
\flushleft{\begin{malayalam}
അതല്ല; നമ്മെക്കൂടാതെ അവരെ സംരക്ഷിക്കുന്ന വല്ലദൈവങ്ങളും അവര്‍ക്കുണ്ടോ? എന്നാല്‍ ആ ദൈവങ്ങള്‍ക്ക് തങ്ങളെത്തന്നെ രക്ഷിക്കാനാവില്ലെന്നതാണ് സത്യം. നമ്മുടെ സഹായം അവര്‍ക്കൊട്ടും കിട്ടുകയുമില്ല.
\end{malayalam}}
\flushright{\begin{Arabic}
\quranayah[21][44]
\end{Arabic}}
\flushleft{\begin{malayalam}
നാം അവര്‍ക്കും അവരുടെ പിതാക്കള്‍ക്കും ജീവിതസുഖം നല്‍കിക്കൊണ്ടിരുന്നു. അങ്ങനെ അവരുടെ കാലം ഏറെ നീണ്ടുപോയി. നാം ഈ ഭൂമിയെ അതിന്റെ ചുറ്റു നിന്നും ചുരുക്കിക്കൊണ്ടുവരുന്നത് ഇക്കൂട്ടര്‍ കാണുന്നില്ലേ? എന്നിട്ടും അവര്‍ തന്നെ വിജയം വരിക്കുമെന്നോ?
\end{malayalam}}
\flushright{\begin{Arabic}
\quranayah[21][45]
\end{Arabic}}
\flushleft{\begin{malayalam}
പറയുക: "ദിവ്യ സന്ദേശമനുസരിച്ച് മാത്രമാണ് ഞാന്‍ നിങ്ങള്‍ക്ക് മുന്നറിയിപ്പ് നല്‍കുന്നത്.” പക്ഷേ, മുന്നറിയിപ്പ് നല്‍കുമ്പോള്‍ കാതുപൊട്ടന്മാര്‍ ആ വിളി കേള്‍ക്കുകയില്ല.
\end{malayalam}}
\flushright{\begin{Arabic}
\quranayah[21][46]
\end{Arabic}}
\flushleft{\begin{malayalam}
നിന്റെ നാഥന്റെ ശിക്ഷയില്‍നിന്ന് ഒരു നേരിയ കാറ്റ് അവരെ സ്പര്‍ശിച്ചാല്‍ അവരിങ്ങനെ വിലപിക്കും: "ഞങ്ങളുടെ ഭാഗ്യദോഷം! ഉറപ്പായും ഞങ്ങള്‍ അതിക്രമികളായിപ്പോയല്ലോ.”
\end{malayalam}}
\flushright{\begin{Arabic}
\quranayah[21][47]
\end{Arabic}}
\flushleft{\begin{malayalam}
ഉയിര്‍ത്തെഴുന്നേല്‍പുനാളില്‍ നാം കൃത്യതയുള്ള തുലാസ്സുകള്‍ സ്ഥാപിക്കും. പിന്നെ ആരോടും അല്‍പവും അനീതി കാണിക്കുകയില്ല. കര്‍മം ഒരു കടുകുമണിത്തൂക്കമായാല്‍ പോലും നാമത് വിലയിരുത്തും. കണക്കുനോക്കാന്‍ നാം തന്നെ മതി.
\end{malayalam}}
\flushright{\begin{Arabic}
\quranayah[21][48]
\end{Arabic}}
\flushleft{\begin{malayalam}
മൂസാക്കും ഹാറൂന്നും നാം ശരി തെറ്റുകള്‍ വേര്‍തിരിച്ചുകാണിക്കുന്ന പ്രമാണം നല്‍കി. പ്രകാശവും ദൈവഭക്തര്‍ക്കുള്ള ഉദ്ബോധനവും സമ്മാനിച്ചു.
\end{malayalam}}
\flushright{\begin{Arabic}
\quranayah[21][49]
\end{Arabic}}
\flushleft{\begin{malayalam}
അവര്‍ തങ്ങളുടെ നാഥനെ കാണാതെ തന്നെ അവനെ ഭയപ്പെടുന്നവരാണ്. അന്ത്യനാളിനെ പേടിയോടെ ഓര്‍ക്കുന്നവരും.
\end{malayalam}}
\flushright{\begin{Arabic}
\quranayah[21][50]
\end{Arabic}}
\flushleft{\begin{malayalam}
നാം ഇറക്കിത്തന്ന അനുഗൃഹീതമായ ഉദ്ബോധനമാണ് ഈ ഖുര്‍ആന്‍. എന്നിട്ടും നിങ്ങളിതിനെ തള്ളിക്കളയുകയോ?
\end{malayalam}}
\flushright{\begin{Arabic}
\quranayah[21][51]
\end{Arabic}}
\flushleft{\begin{malayalam}
നേരത്തെ നാം ഇബ്റാഹീമിന് തന്റേതായ വിവേകം നല്‍കിയിരുന്നു. നമുക്കദ്ദേഹത്തെ നന്നായറിയാമായിരുന്നു.
\end{malayalam}}
\flushright{\begin{Arabic}
\quranayah[21][52]
\end{Arabic}}
\flushleft{\begin{malayalam}
അദ്ദേഹം തന്റെ പിതാവിനോടും ജനത്തോടും ചോദിച്ചതോര്‍ക്കുക: "നിങ്ങള്‍ പൂജിക്കുന്ന ഈ പ്രതിഷ്ഠകള്‍ എന്താണ്?”
\end{malayalam}}
\flushright{\begin{Arabic}
\quranayah[21][53]
\end{Arabic}}
\flushleft{\begin{malayalam}
അവര്‍ പറഞ്ഞു: "ഞങ്ങളുടെ പിതാക്കള്‍ ഇവയെ പൂജിക്കുന്നത് ഞങ്ങള്‍ കണ്ടിട്ടുണ്ട്.”
\end{malayalam}}
\flushright{\begin{Arabic}
\quranayah[21][54]
\end{Arabic}}
\flushleft{\begin{malayalam}
അദ്ദേഹം പറഞ്ഞു: "തീര്‍ച്ചയായും നിങ്ങളും നിങ്ങളുടെ പിതാക്കളും വ്യക്തമായ വഴികേടിലാണ്.”
\end{malayalam}}
\flushright{\begin{Arabic}
\quranayah[21][55]
\end{Arabic}}
\flushleft{\begin{malayalam}
അവര്‍ ചോദിച്ചു: "അല്ല; നീ കാര്യമായിത്തന്നെയാണോ ഞങ്ങളോടിപ്പറയുന്നത്; അതോ കളിതമാശ പറയുകയോ?”
\end{malayalam}}
\flushright{\begin{Arabic}
\quranayah[21][56]
\end{Arabic}}
\flushleft{\begin{malayalam}
അദ്ദേഹം പറഞ്ഞു: "അല്ല, യഥാര്‍ഥത്തില്‍ നിങ്ങളുടെ നാഥന്‍ ആകാശഭൂമികളുടെ സംരക്ഷകനാണ്. അവയെ സൃഷ്ടിച്ചുണ്ടാക്കിയവന്‍. ഇതു സത്യംതന്നെ എന്ന് ഞാന്‍ നിങ്ങള്‍ക്കു മുന്നില്‍ സാക്ഷ്യം വഹിക്കുന്നു.
\end{malayalam}}
\flushright{\begin{Arabic}
\quranayah[21][57]
\end{Arabic}}
\flushleft{\begin{malayalam}
"അല്ലാഹു തന്നെ സത്യം! നിങ്ങള്‍ പിരിഞ്ഞുപോയശേഷം നിങ്ങളുടെ ഈ വിഗ്രഹങ്ങളുടെ കാര്യത്തില്‍ ഞാനൊരു തന്ത്രം പ്രയോഗിക്കും.”
\end{malayalam}}
\flushright{\begin{Arabic}
\quranayah[21][58]
\end{Arabic}}
\flushleft{\begin{malayalam}
അദ്ദേഹം അവയെ തുണ്ടം തുണ്ടമാക്കി. വലിയ ഒന്നിനെയൊഴികെ. ഒരുവേള അവര്‍ സത്യത്തിലേക്ക് തിരിച്ചുവന്നെങ്കിലോ?
\end{malayalam}}
\flushright{\begin{Arabic}
\quranayah[21][59]
\end{Arabic}}
\flushleft{\begin{malayalam}
അവര്‍ ചോദിച്ചു: "നമ്മളുടെ ദൈവങ്ങളോട് ഇവ്വിധം ചെയ്തവനാര്? ആരായാലും അവന്‍ അക്രമി തന്നെ.”
\end{malayalam}}
\flushright{\begin{Arabic}
\quranayah[21][60]
\end{Arabic}}
\flushleft{\begin{malayalam}
ചിലര്‍ പറഞ്ഞു: "ഇബ്റാഹീം എന്നു പേരുള്ള ഒരു ചെറുപ്പക്കാരന്‍ ആ ദൈവങ്ങളെപ്പറ്റി സംസാരിക്കുന്നത് ഞങ്ങള്‍ കേട്ടിട്ടുണ്ട്.”
\end{malayalam}}
\flushright{\begin{Arabic}
\quranayah[21][61]
\end{Arabic}}
\flushleft{\begin{malayalam}
അവര്‍ പറഞ്ഞു: "എങ്കില്‍ നിങ്ങളവനെ ജനങ്ങളുടെ കണ്‍മുന്നില്‍ കൊണ്ടുവരിക. അവര്‍ സാക്ഷി പറയട്ടെ.”
\end{malayalam}}
\flushright{\begin{Arabic}
\quranayah[21][62]
\end{Arabic}}
\flushleft{\begin{malayalam}
അവര്‍ ചോദിച്ചു: "ഇബ്റാഹീമേ, നീയാണോ ഞങ്ങളുടെ ദൈവങ്ങളെ ഇവ്വിധം ചെയ്തത്?”
\end{malayalam}}
\flushright{\begin{Arabic}
\quranayah[21][63]
\end{Arabic}}
\flushleft{\begin{malayalam}
അദ്ദേഹം പറഞ്ഞു: "അല്ല; ആ ദൈവങ്ങളിലെ ഈ വലിയവനാണിതു ചെയ്തത്. നിങ്ങളവരോടു ചോദിച്ചു നോക്കൂ; അവര്‍ സംസാരിക്കുമെങ്കില്‍!
\end{malayalam}}
\flushright{\begin{Arabic}
\quranayah[21][64]
\end{Arabic}}
\flushleft{\begin{malayalam}
അപ്പോള്‍ അവര്‍ തങ്ങളുടെ ബോധതലത്തിലേക്കൊന്നു തിരിഞ്ഞു. അങ്ങനെ അവരന്യോന്യം പറഞ്ഞു: "നിങ്ങള്‍ തന്നെയാണ് അതിക്രമികള്‍.”
\end{malayalam}}
\flushright{\begin{Arabic}
\quranayah[21][65]
\end{Arabic}}
\flushleft{\begin{malayalam}
പിന്നെയവര്‍ തല തിരിഞ്ഞു. അവര്‍ പറഞ്ഞു: "ഇവര്‍ സംസാരിക്കുകയില്ലെന്ന് നിനക്കറിയാമല്ലോ.”
\end{malayalam}}
\flushright{\begin{Arabic}
\quranayah[21][66]
\end{Arabic}}
\flushleft{\begin{malayalam}
"അപ്പോള്‍ നിങ്ങള്‍ അല്ലാഹുവെക്കൂടാതെ പൂജിക്കുന്നത് നിങ്ങള്‍ക്ക് എന്തെങ്കിലും ഉപകാരമോ ഉപദ്രവമോ ചെയ്യാത്ത വസ്തുക്കളെയാണോ?
\end{malayalam}}
\flushright{\begin{Arabic}
\quranayah[21][67]
\end{Arabic}}
\flushleft{\begin{malayalam}
"നിങ്ങളുടെയും, അല്ലാഹുവെക്കൂടാതെ നിങ്ങള്‍ പൂജിക്കുന്നവയുടെയും കാര്യം അത്യന്തം അപമാനകരം തന്നെ. നിങ്ങളൊട്ടും ചിന്തിക്കുന്നില്ലേ?”
\end{malayalam}}
\flushright{\begin{Arabic}
\quranayah[21][68]
\end{Arabic}}
\flushleft{\begin{malayalam}
അവര്‍ പറഞ്ഞു: "നിങ്ങളിവനെ ചുട്ടെരിക്കുക. അങ്ങനെ നിങ്ങളുടെ ദൈവങ്ങളെ തുണക്കുക. നിങ്ങള്‍ വല്ലതും ചെയ്യാനുദ്ദേശിക്കുന്നുവെങ്കില്‍ അതാണ് വേണ്ടത്.”
\end{malayalam}}
\flushright{\begin{Arabic}
\quranayah[21][69]
\end{Arabic}}
\flushleft{\begin{malayalam}
നാം പറഞ്ഞു: "തീയേ, തണുക്കൂ; ഇബ്റാഹീമിന് രക്ഷാകവചമാകൂ.”
\end{malayalam}}
\flushright{\begin{Arabic}
\quranayah[21][70]
\end{Arabic}}
\flushleft{\begin{malayalam}
അദ്ദേഹത്തിനെതിരെ അവര്‍ തന്ത്രമൊരുക്കി. എന്നാല്‍ നാമവരെ എല്ലാം നഷ്ടപ്പെട്ടവരാക്കി.
\end{malayalam}}
\flushright{\begin{Arabic}
\quranayah[21][71]
\end{Arabic}}
\flushleft{\begin{malayalam}
മുഴുലോകര്‍ക്കും നാം അനുഗ്രഹങ്ങള്‍ ഒരുക്കിവെച്ച നാട്ടിലേക്ക് അദ്ദേഹത്തെയും ലൂത്വിനെയും രക്ഷപ്പെടുത്തി.
\end{malayalam}}
\flushright{\begin{Arabic}
\quranayah[21][72]
\end{Arabic}}
\flushleft{\begin{malayalam}
അദ്ദേഹത്തിനു നാം ഇസ്ഹാഖിനെ സമ്മാനിച്ചു. അതിനു പുറമെ യഅ്ഖൂബിനെയും. അവരെയൊക്കെ നാം സച്ചരിതരാക്കുകയും ചെയ്തു.
\end{malayalam}}
\flushright{\begin{Arabic}
\quranayah[21][73]
\end{Arabic}}
\flushleft{\begin{malayalam}
അവരെ നാം നമ്മുടെ നിര്‍ദേശാനുസരണം നേര്‍വഴി കാണിച്ചുകൊടുക്കുന്ന നേതാക്കന്മാരാക്കി. നാമവര്‍ക്ക് നല്ല കാര്യങ്ങള്‍ ചെയ്യാനും നമസ്കാരം നിഷ്ഠയോടെ നിര്‍വഹിക്കാനും സകാത്ത് നല്‍കാനും നിര്‍ദേശം നല്‍കി. അവരൊക്കെ നമുക്ക് വഴിപ്പെട്ട് ജീവിക്കുന്നവരായിരുന്നു.
\end{malayalam}}
\flushright{\begin{Arabic}
\quranayah[21][74]
\end{Arabic}}
\flushleft{\begin{malayalam}
ലൂത്വിനു നാം തത്ത്വബോധവും അറിവും നല്‍കി. ആഭാസം നടന്നിരുന്ന നാട്ടില്‍ നിന്ന് നാം അദ്ദേഹത്തെ രക്ഷപ്പെടുത്തി. അന്നാട്ടുകാര്‍ ദുഷിച്ച തെമ്മാടികളായ ജനമായിരുന്നു.
\end{malayalam}}
\flushright{\begin{Arabic}
\quranayah[21][75]
\end{Arabic}}
\flushleft{\begin{malayalam}
ലൂത്വിനെ നാം നമ്മുടെ കാരുണ്യവലയത്തിലുള്‍പ്പെടുത്തി. തീര്‍ച്ച; അദ്ദേഹം സച്ചരിതനായിരുന്നു.
\end{malayalam}}
\flushright{\begin{Arabic}
\quranayah[21][76]
\end{Arabic}}
\flushleft{\begin{malayalam}
നൂഹിന്റെ കാര്യവും ഓര്‍ക്കുക: ഇവര്‍ക്കെല്ലാം മുമ്പെ അദ്ദേഹം നമ്മെ വിളിച്ചുപ്രാര്‍ഥിച്ച കാര്യം. അങ്ങനെ നാം അദ്ദേഹത്തിന്ഉത്തരം നല്‍കി. അദ്ദേഹത്തെയും കുടുംബത്തെയും കൊടുംദുരിതത്തില്‍ നിന്ന് രക്ഷപ്പെടുത്തി.
\end{malayalam}}
\flushright{\begin{Arabic}
\quranayah[21][77]
\end{Arabic}}
\flushleft{\begin{malayalam}
നമ്മുടെ വചനങ്ങളെ തള്ളിപ്പറഞ്ഞ ജനത്തിനെതിരെ നാം അദ്ദേഹത്തെ തുണച്ചു. തീര്‍ച്ചയായും അവര്‍ പറ്റെ ദുഷിച്ച ജനതയായിരുന്നു. അതിനാല്‍ അവരെ ഒന്നടങ്കം നാം മുക്കിയൊടുക്കി.
\end{malayalam}}
\flushright{\begin{Arabic}
\quranayah[21][78]
\end{Arabic}}
\flushleft{\begin{malayalam}
ദാവൂദിന്റെയും സുലൈമാന്റെയും കാര്യം ഓര്‍ക്കുക: അവരിരുവരും ഒരു കൃഷിയിടത്തിന്റെ പ്രശ്നത്തില്‍ തീര്‍പ്പുകല്‍പിച്ച കാര്യം. ഒരു കൂട്ടരുടെ ആടുകള്‍ കൃഷിയിടത്തില്‍ കടന്നു വിള തിന്നു. അവരുടെ വിധിക്കു നാം സാക്ഷിയായിരുന്നു.
\end{malayalam}}
\flushright{\begin{Arabic}
\quranayah[21][79]
\end{Arabic}}
\flushleft{\begin{malayalam}
അന്നേരം സുലൈമാന്ന് നാം കാര്യത്തിന്റെ നിജസ്ഥിതി മനസ്സിലാക്കിക്കൊടുത്തു. അവരിരുവര്‍ക്കും നാം തത്ത്വബോധവും അറിവും നല്‍കി. ദാവൂദിനോടൊപ്പം, ദൈവത്തെ കീര്‍ത്തനം ചെയ്യുന്ന പര്‍വതങ്ങളെയും പറവകളെയും നാം അധീനപ്പെടുത്തിക്കൊടുത്തു. നാമാണിതൊക്കെ ചെയ്തുകൊണ്ടിരുന്നത്.
\end{malayalam}}
\flushright{\begin{Arabic}
\quranayah[21][80]
\end{Arabic}}
\flushleft{\begin{malayalam}
നിങ്ങള്‍ക്കുവേണ്ടി നാം അദ്ദേഹത്തിന് പടയങ്കിനിര്‍മാണം പഠിപ്പിച്ചുകൊടുത്തു. നിങ്ങളെ യുദ്ധവിപത്തുകളില്‍നിന്ന് രക്ഷിക്കാനാണത്. എന്നിട്ട് നിങ്ങള്‍ നന്ദിയുള്ളവരാണോ?
\end{malayalam}}
\flushright{\begin{Arabic}
\quranayah[21][81]
\end{Arabic}}
\flushleft{\begin{malayalam}
സുലൈമാന്ന് നാം ആഞ്ഞുവീശുന്ന കാറ്റിനെയും അധീനപ്പെടുത്തിക്കൊടുത്തു. അദ്ദേഹത്തിന്റെ കല്‍പന പ്രകാരം, നാം അനുഗ്രഹങ്ങളൊരുക്കിവെച്ച നാട്ടിലേക്ക് അത് സഞ്ചരിച്ചുകൊണ്ടിരുന്നു. എല്ലാ കാര്യത്തെപ്പറ്റിയും നന്നായറിയുന്നവനാണ് നാം.
\end{malayalam}}
\flushright{\begin{Arabic}
\quranayah[21][82]
\end{Arabic}}
\flushleft{\begin{malayalam}
പിശാചുക്കളില്‍നിന്ന് ചിലരെയും നാം അദ്ദേഹത്തിനു കീഴ്പെടുത്തിക്കൊടുത്തു. അവര്‍ അദ്ദേഹത്തിനുവേണ്ടി വെള്ളത്തില്‍മുങ്ങുമായിരുന്നു. കൂടാതെ മറ്റു പല ജോലികളും ചെയ്യുന്നവരുമായിരുന്നു. നാമാണവര്‍ക്ക് മേല്‍നോട്ടംവഹിച്ചിരുന്നത്.
\end{malayalam}}
\flushright{\begin{Arabic}
\quranayah[21][83]
\end{Arabic}}
\flushleft{\begin{malayalam}
അയ്യൂബ് തന്റെ നാഥനെ വിളിച്ച് പ്രാര്‍ഥിച്ച കാര്യം ഓര്‍ക്കുക: "എന്നെ ദുരിതം ബാധിച്ചിരിക്കുന്നു. നീ കരുണയുള്ളവരിലേറ്റവും കരുണയുള്ളവനാണല്ലോ.”
\end{malayalam}}
\flushright{\begin{Arabic}
\quranayah[21][84]
\end{Arabic}}
\flushleft{\begin{malayalam}
അപ്പോള്‍ അദ്ദേഹത്തിനു നാം ഉത്തരമേകി. അദ്ദേഹത്തിനുണ്ടായിരുന്ന ദുരിതം ദൂരീകരിച്ചുകൊടുത്തു. അദ്ദേഹത്തിനു നാം തന്റെ കുടുംബത്തെ നല്‍കി. അവരോടൊപ്പം അത്രയും പേരെ വേറെയും കൊടുത്തു. നമ്മുടെ ഭാഗത്തുനിന്നുള്ള അനുഗ്രഹമായാണത്. ആരാധനയില്‍ മുഴുകുന്നവര്‍ക്ക് ഒരോര്‍മപ്പെടുത്തലും.
\end{malayalam}}
\flushright{\begin{Arabic}
\quranayah[21][85]
\end{Arabic}}
\flushleft{\begin{malayalam}
ഇസ്മാഈലിന്റെയും ഇദ്രീസിന്റെയും ദുല്‍കിഫ്ലിന്റെയും കാര്യവും ഓര്‍ക്കുക. അവരൊക്കെ ഏറെ ക്ഷമാലുക്കളായിരുന്നു.
\end{malayalam}}
\flushright{\begin{Arabic}
\quranayah[21][86]
\end{Arabic}}
\flushleft{\begin{malayalam}
അവരെയെല്ലാം നാം നമ്മുടെ അനുഗ്രഹത്തിലുള്‍പ്പെടുത്തി. അവരെല്ലാവരും സച്ചരിതരായിരുന്നു.
\end{malayalam}}
\flushright{\begin{Arabic}
\quranayah[21][87]
\end{Arabic}}
\flushleft{\begin{malayalam}
ദുന്നൂന്‍ കുപിതനായി പോയ കാര്യം ഓര്‍ക്കുക: നാം പിടികൂടുകയില്ലെന്ന് അദ്ദേഹം കരുതി. അതിനാല്‍ കൂരിരുളുകളില്‍ വെച്ച് അദ്ദേഹം കേണപേക്ഷിച്ചു: "നീയല്ലാതെ ദൈവമില്ല. നീയെത്ര പരിശുദ്ധന്‍! സംശയമില്ല; ഞാന്‍ അതിക്രമിയായിരിക്കുന്നു.”
\end{malayalam}}
\flushright{\begin{Arabic}
\quranayah[21][88]
\end{Arabic}}
\flushleft{\begin{malayalam}
അന്നേരം നാം അദ്ദേഹത്തിന് ഉത്തരമേകി. അദ്ദേഹത്തെ ദുഃഖത്തില്‍നിന്നു മോചിപ്പിച്ചു. ഇവ്വിധം നാം സത്യവിശ്വാസികളെ രക്ഷപ്പെടുത്തുന്നു.
\end{malayalam}}
\flushright{\begin{Arabic}
\quranayah[21][89]
\end{Arabic}}
\flushleft{\begin{malayalam}
സകരിയ്യാ തന്റെ നാഥനെ വിളിച്ചുപ്രാര്‍ഥിച്ച കാര്യം ഓര്‍ക്കുക: "എന്റെ നാഥാ, നീയെന്നെ ഒറ്റയാനായി വിടരുതേ. നീയാണല്ലോ അനന്തരമെടുക്കുന്നവരില്‍ അത്യുത്തമന്‍.”
\end{malayalam}}
\flushright{\begin{Arabic}
\quranayah[21][90]
\end{Arabic}}
\flushleft{\begin{malayalam}
അപ്പോള്‍ നാം അദ്ദേഹത്തിനുത്തരം നല്‍കി. യഹ്യായെ സമ്മാനമായി കൊടുത്തു. അദ്ദേഹത്തിന്റെ പത്നിയെ നാമതിന് പ്രാപ്തയാക്കി. തീര്‍ച്ചയായും അവര്‍ നല്ല കാര്യങ്ങളില്‍ ആവേശം കാണിക്കുന്നവരായിരുന്നു. പേടിയോടെയും പ്രതീക്ഷയോടെയും നമ്മോട് പ്രാര്‍ഥിക്കുന്നവരും താഴ്മ കാണിക്കുന്നവരുമായിരുന്നു.
\end{malayalam}}
\flushright{\begin{Arabic}
\quranayah[21][91]
\end{Arabic}}
\flushleft{\begin{malayalam}
തന്റെ പാതിവ്രത്യം സൂക്ഷിച്ചവളുടെ കാര്യം ഓര്‍ക്കുക: അങ്ങനെ നാമവളില്‍ നമ്മുടെ ആത്മാവില്‍നിന്ന് ഊതി. അവളെയും അവളുടെ മകനെയും ലോകര്‍ക്ക് തെളിഞ്ഞ അടയാളമാക്കുകയും ചെയ്തു.
\end{malayalam}}
\flushright{\begin{Arabic}
\quranayah[21][92]
\end{Arabic}}
\flushleft{\begin{malayalam}
നിങ്ങളുടെ ഈ സമുദായം സത്യത്തില്‍ ഒരൊറ്റ സമുദായമാണ്. ഞാന്‍ നിങ്ങളുടെ നാഥനും. അതിനാല്‍ നിങ്ങളെനിക്കു മാത്രം വഴിപ്പെടുക.
\end{malayalam}}
\flushright{\begin{Arabic}
\quranayah[21][93]
\end{Arabic}}
\flushleft{\begin{malayalam}
എന്നാല്‍ അവര്‍ തങ്ങളുടെ മതകാര്യത്തില്‍ നിരവധി ചേരികളായി. അറിയുക: എല്ലാവരും നമ്മിലേക്ക് തിരിച്ചുവരേണ്ടവരാണ്.
\end{malayalam}}
\flushright{\begin{Arabic}
\quranayah[21][94]
\end{Arabic}}
\flushleft{\begin{malayalam}
സത്യവിശ്വാസിയായി സല്‍ക്കര്‍മങ്ങള്‍ പ്രവര്‍ത്തിക്കുന്നവന്റെ കര്‍മഫലം പാഴാവുകയില്ല. ഉറപ്പായും നാമത് രേഖപ്പെടുത്തുന്നുണ്ട്.
\end{malayalam}}
\flushright{\begin{Arabic}
\quranayah[21][95]
\end{Arabic}}
\flushleft{\begin{malayalam}
നാമൊരു നാടിനെ നശിപ്പിച്ചാല്‍ അവര്‍ പിന്നെയൊരിക്കലും അവിടേക്ക് തിരിച്ചുവരില്ല;
\end{malayalam}}
\flushright{\begin{Arabic}
\quranayah[21][96]
\end{Arabic}}
\flushleft{\begin{malayalam}
യഅ്ജൂജ്- മഅ്ജൂജ് ജനവിഭാഗങ്ങള്‍ക്ക് ഒരു വഴി തുറന്നുകിട്ടുംവരെ; അങ്ങനെ അവര്‍ എല്ലാ കുന്നിന്‍പുറങ്ങളില്‍നിന്നും കുതിച്ചിറങ്ങി വരും വരെയും;
\end{malayalam}}
\flushright{\begin{Arabic}
\quranayah[21][97]
\end{Arabic}}
\flushleft{\begin{malayalam}
ആ സത്യവാഗ്ദാനം അടുത്തു വരുന്നതു വരെയും. അപ്പോള്‍ സത്യനിഷേധികളുടെ കണ്ണുകള്‍ തുറിച്ചുനില്‍ക്കും. അവരിങ്ങനെ വിലപിക്കും. "ഞങ്ങളുടെ ഭാഗ്യദോഷം. ഞങ്ങള്‍ ഇതേക്കുറിച്ച് തീര്‍ത്തും അശ്രദ്ധരായിരുന്നു. ഞങ്ങള്‍ അതിക്രമികളായിപ്പോയല്ലോ.”
\end{malayalam}}
\flushright{\begin{Arabic}
\quranayah[21][98]
\end{Arabic}}
\flushleft{\begin{malayalam}
തീര്‍ച്ചയായും നിങ്ങളും അല്ലാഹുവെക്കൂടാതെ നിങ്ങള്‍ പൂജിക്കുന്നവരും നരകത്തീയിലെ വിറകാണ്. നിങ്ങളെല്ലാം അവിടെ എത്തിച്ചേരുക തന്നെ ചെയ്യും.
\end{malayalam}}
\flushright{\begin{Arabic}
\quranayah[21][99]
\end{Arabic}}
\flushleft{\begin{malayalam}
യഥാര്‍ഥത്തില്‍ അവര്‍ ദൈവങ്ങളായിരുന്നെങ്കില്‍ ഈ നരകത്തീയില്‍ വന്നെത്തുമായിരുന്നില്ല. എന്നാല്‍ ഓര്‍ക്കുക: അവരെല്ലാം അവിടെ നിത്യവാസികളായിരിക്കും.
\end{malayalam}}
\flushright{\begin{Arabic}
\quranayah[21][100]
\end{Arabic}}
\flushleft{\begin{malayalam}
അവര്‍ക്കവിടെ നെടുനിശ്വാസമാണുണ്ടാവുക. അതല്ലാതൊന്നും കേള്‍ക്കാനാവില്ല.
\end{malayalam}}
\flushright{\begin{Arabic}
\quranayah[21][101]
\end{Arabic}}
\flushleft{\begin{malayalam}
എന്നാല്‍ നേരത്തെ തന്നെ നമ്മില്‍ നിന്ന് നന്മ ലഭിച്ചവര്‍ അതില്‍നിന്ന് മാറ്റിനിര്‍ത്തപ്പെടും.
\end{malayalam}}
\flushright{\begin{Arabic}
\quranayah[21][102]
\end{Arabic}}
\flushleft{\begin{malayalam}
അവരതിന്റെ നേരിയ ശബ്ദംപോലും കേള്‍ക്കുകയില്ല. അവര്‍ എന്നെന്നും തങ്ങളുടെ മനസ്സിഷ്ടപ്പെടുന്ന സുഖാസ്വാദ്യതകളിലായിരിക്കും.
\end{malayalam}}
\flushright{\begin{Arabic}
\quranayah[21][103]
\end{Arabic}}
\flushleft{\begin{malayalam}
ആ മഹാ സംഭ്രമം അവരെ ഒട്ടും ആകുലരാക്കുകയില്ല. മലക്കുകള്‍ അവരെ സ്വീകരിച്ച് എതിരേല്‍ക്കും. മലക്കുകള്‍ അവര്‍ക്കിങ്ങനെ സ്വാഗതമോതുകയും ചെയ്യും:"നിങ്ങളോടു വാഗ്ദാനം ചെയ്ത ആ മോഹന ദിനമാണിത്.”
\end{malayalam}}
\flushright{\begin{Arabic}
\quranayah[21][104]
\end{Arabic}}
\flushleft{\begin{malayalam}
പുസ്തകത്താളുകള്‍ ചുരുട്ടുംപോലെ ആകാശത്തെ നാം ചുരുട്ടിക്കൂട്ടുന്ന ദിനമാണത്. നാം സൃഷ്ടി ആദ്യമാരംഭിച്ചപോലെ തന്നെ അതാവര്‍ത്തിക്കും. വാഗ്ദാനം വഴി ഇത് നമ്മുടെ ബാധ്യതയായിരിക്കുന്നു. നാമത് നടപ്പാക്കുകതന്നെ ചെയ്യും.
\end{malayalam}}
\flushright{\begin{Arabic}
\quranayah[21][105]
\end{Arabic}}
\flushleft{\begin{malayalam}
സബൂറില്‍ ഉദ്ബോധനത്തിനുശേഷം നാമിങ്ങനെ രേഖപ്പെടുത്തിയിട്ടുണ്ട്: "ഭൂമിയുടെ പിന്തുടര്‍ച്ചാവകാശം സച്ചരിതരായ എന്റെ ദാസന്മാര്‍ക്കായിരിക്കും.”
\end{malayalam}}
\flushright{\begin{Arabic}
\quranayah[21][106]
\end{Arabic}}
\flushleft{\begin{malayalam}
തീര്‍ച്ചയായും ഇതില്‍ അല്ലാഹുവെ വഴിപ്പെടുന്ന ജനത്തിന് മഹത്തായ സന്ദേശമുണ്ട്.
\end{malayalam}}
\flushright{\begin{Arabic}
\quranayah[21][107]
\end{Arabic}}
\flushleft{\begin{malayalam}
ലോകര്‍ക്കാകെ അനുഗ്രഹമായല്ലാതെ നിന്നെ നാം അയച്ചിട്ടില്ല.
\end{malayalam}}
\flushright{\begin{Arabic}
\quranayah[21][108]
\end{Arabic}}
\flushleft{\begin{malayalam}
പറയുക: എനിക്കു ബോധനമായി കിട്ടിയതിതാണ്: നിങ്ങളുടെ ദൈവം ഏകനായ അല്ലാഹു മാത്രമാണ്. എന്നിട്ടും നിങ്ങള്‍ മുസ്ലിംകളാവുന്നില്ലേ?
\end{malayalam}}
\flushright{\begin{Arabic}
\quranayah[21][109]
\end{Arabic}}
\flushleft{\begin{malayalam}
ഇനിയും അവര്‍ പിന്തിരിയുകയാണെങ്കില്‍ പറയുക: "ഞാന്‍ നിങ്ങള്‍ക്കെല്ലാം ഒരേപോലെ അറിയിപ്പ് നല്‍കിക്കഴിഞ്ഞു. നിങ്ങളോട് വാഗ്ദാനം ചെയ്യുന്ന കാര്യം അടുത്തോ അകലെയോ എന്നെനിക്കറിയുകയില്ല.
\end{malayalam}}
\flushright{\begin{Arabic}
\quranayah[21][110]
\end{Arabic}}
\flushleft{\begin{malayalam}
"എന്നാല്‍ നിങ്ങള്‍ ഉറക്കെ പറയുന്നതും മറച്ചുവെക്കുന്നതും തീര്‍ച്ചയായും അല്ലാഹു അറിയും.
\end{malayalam}}
\flushright{\begin{Arabic}
\quranayah[21][111]
\end{Arabic}}
\flushleft{\begin{malayalam}
"എനിക്കറിഞ്ഞുകൂടാ, ഒരു വേള ഇത് നിങ്ങള്‍ക്കൊരു പരീക്ഷണമായേക്കാം; ഒരു നിശ്ചിതകാലം വരെ നിങ്ങള്‍ക്ക് സുഖാസ്വാദനത്തിനുള്ള അവസരം നല്‍കിയതുമാവാം.”
\end{malayalam}}
\flushright{\begin{Arabic}
\quranayah[21][112]
\end{Arabic}}
\flushleft{\begin{malayalam}
പ്രവാചകന്‍ പറഞ്ഞു: "എന്റെ നാഥാ; നീ സത്യംപോലെ വിധി കല്‍പിക്കുക. ഞങ്ങളുടെ നാഥന്‍ പരമകാരുണികനാണ്. നിങ്ങള്‍ പറഞ്ഞുപരത്തുന്നതിനെതിരെ ഞങ്ങള്‍ക്ക് സഹായത്തിന് ആശ്രയിക്കാവുന്നവനും.”
\end{malayalam}}
\chapter{\textmalayalam{ഹജ്ജ് ( തീര്‍ത്ഥാടനം )}}
\begin{Arabic}
\Huge{\centerline{\basmalah}}\end{Arabic}
\flushright{\begin{Arabic}
\quranayah[22][1]
\end{Arabic}}
\flushleft{\begin{malayalam}
മനുഷ്യരേ, നിങ്ങള്‍ നിങ്ങളുടെ നാഥനോട് ഭക്തിയുള്ളവരാവുക. ഉറപ്പായും അന്ത്യനാളിന്റെ പ്രകമ്പനം അതിഭയങ്കരം തന്നെ.
\end{malayalam}}
\flushright{\begin{Arabic}
\quranayah[22][2]
\end{Arabic}}
\flushleft{\begin{malayalam}
നിങ്ങളതു കാണുംനാളിലെ അവസ്ഥയോ; മുലയൂട്ടുന്ന മാതാക്കള്‍ തങ്ങളുടെ കുഞ്ഞുങ്ങളെ മറക്കും. ഗര്‍ഭിണികള്‍ പ്രസവിച്ചുപോകും. ജനങ്ങളെ ലഹരിബാധിതരെപ്പോലെ നിനക്കന്ന് കാണാം. യഥാര്‍ഥത്തിലവര്‍ ലഹരിബാധിതരല്ല. എന്നാല്‍ അല്ലാഹുവിന്റെ ശിക്ഷ അത്രമാത്രം ഘോരമായിരിക്കും.
\end{malayalam}}
\flushright{\begin{Arabic}
\quranayah[22][3]
\end{Arabic}}
\flushleft{\begin{malayalam}
ഒന്നുമറിയാതെ അല്ലാഹുവിന്റെ കാര്യത്തില്‍ തര്‍ക്കിച്ചുകൊണ്ടിരിക്കുന്ന ചിലരുണ്ട്. ധിക്കാരിയായ ഏതു ചെകുത്താനെയുമവര്‍ പിന്‍പറ്റുന്നു.
\end{malayalam}}
\flushright{\begin{Arabic}
\quranayah[22][4]
\end{Arabic}}
\flushleft{\begin{malayalam}
ചെകുത്താന്റെ കാര്യത്തില്‍ രേഖപ്പെടുത്തിയതിങ്ങനെയാണ്: ആര്‍ ചെകുത്താനെ മിത്രമായി സ്വീകരിക്കുന്നുവോ അയാളെ അവന്‍ പിഴപ്പിക്കും. നരകശിക്ഷയിലേക്ക് നയിക്കുകയും ചെയ്യും.
\end{malayalam}}
\flushright{\begin{Arabic}
\quranayah[22][5]
\end{Arabic}}
\flushleft{\begin{malayalam}
മനുഷ്യരേ, ഉയിര്‍ത്തെഴുന്നേല്‍പിനെപ്പറ്റി നിങ്ങള്‍ സംശയത്തിലാണെങ്കില്‍ ഒന്നോര്‍ത്തുനോക്കൂ: തീര്‍ച്ചയായും ആദിയില്‍ നാം നിങ്ങളെ സൃഷ്ടിച്ചത് മണ്ണില്‍നിന്നാണ്. പിന്നെ ബീജത്തില്‍നിന്ന്; പിന്നെ ഭ്രൂണത്തില്‍ നിന്ന്; പിന്നെ രൂപമണിഞ്ഞതും അല്ലാത്തതുമായ മാംസപിണ്ഡത്തില്‍നിന്ന്. നാമിതു വിവരിക്കുന്നത് നിങ്ങള്‍ക്ക് കാര്യം വ്യക്തമാക്കിത്തരാനാണ്. നാം ഇച്ഛിക്കുന്നതിനെ ഒരു നിശ്ചിത അവധിവരെ ഗര്‍ഭാശയത്തില്‍ സൂക്ഷിക്കുന്നു. പിന്നെ നിങ്ങളെ നാം ശിശുക്കളായി പുറത്തുകൊണ്ടുവരുന്നു. പിന്നീട് നിങ്ങള്‍ യൌവനം പ്രാപിക്കുംവരെ നിങ്ങളെ വളര്‍ത്തുന്നു. നിങ്ങളില്‍ ചിലരെ നേരത്തെ തന്നെ തിരിച്ചുവിളിക്കുന്നു. എല്ലാം അറിയാവുന്ന അവസ്ഥക്കുശേഷം ഒന്നും അറിയാത്ത സ്ഥിതിയിലെത്തുമാറ് അവശമായ പ്രായാധിക്യത്തിലേക്ക് തള്ളപ്പെടുന്നവരും നിങ്ങളിലുണ്ട്. ഭൂമി വരണ്ട് ചത്ത് കിടക്കുന്നതു നിനക്കുകാണാം. പിന്നെ നാമതില്‍ മഴവീഴ്ത്തിയാല്‍ അത് തുടിക്കുകയും വികസിക്കുകയും ചെയ്യുന്നു. കൌതുകമുണര്‍ത്തുന്ന സകലയിനം ചെടികളെയും മുളപ്പിക്കുന്നു.
\end{malayalam}}
\flushright{\begin{Arabic}
\quranayah[22][6]
\end{Arabic}}
\flushleft{\begin{malayalam}
അല്ലാഹു തന്നെയാണ് പരമ സത്യമെന്നതാണ് ഇതെല്ലാം തെളിയിക്കുന്നത്. അവന്‍ മരിച്ചവരെ ജീവിപ്പിക്കും. എല്ലാ കാര്യത്തിനും കഴിവുറ്റവനാണവന്‍.
\end{malayalam}}
\flushright{\begin{Arabic}
\quranayah[22][7]
\end{Arabic}}
\flushleft{\begin{malayalam}
അന്ത്യസമയം വന്നെത്തുക തന്നെ ചെയ്യും; അതില്‍ സംശയം വേണ്ട. കുഴിമാടങ്ങളിലുള്ളവരെയെല്ലാം അല്ലാഹു ഉയിര്‍ത്തെഴുന്നേല്‍പിക്കുകതന്നെ ചെയ്യും.
\end{malayalam}}
\flushright{\begin{Arabic}
\quranayah[22][8]
\end{Arabic}}
\flushleft{\begin{malayalam}
എന്തെങ്കിലും അറിവോ വഴികാട്ടിയോ വെളിച്ചം നല്‍കുന്ന വേദപുസ്തകമോ ഇല്ലാതെ അല്ലാഹുവിന്റെ കാര്യത്തില്‍ വെറുതെ തര്‍ക്കിച്ചുകൊണ്ടിരിക്കുന്ന ചിലരുണ്ട്.
\end{malayalam}}
\flushright{\begin{Arabic}
\quranayah[22][9]
\end{Arabic}}
\flushleft{\begin{malayalam}
പിരടി ചെരിച്ച് ഹുങ്കുകാട്ടുന്നവനാണവന്‍.അല്ലാഹുവിന്റെ മാര്‍ഗത്തില്‍നിന്ന് ആളുകളെ തെറ്റിക്കാനാണ് അവനിങ്ങനെ ചെയ്യുന്നത്. ഉറപ്പായും അവന് ഇഹലോകത്ത് നിന്ദ്യതയാണുണ്ടാവുക. ഉയിര്‍ത്തെഴുന്നേല്‍പുനാളില്‍ നാമവനെ ചുട്ടെരിക്കുന്ന ശിക്ഷ ആസ്വദിപ്പിക്കും.
\end{malayalam}}
\flushright{\begin{Arabic}
\quranayah[22][10]
\end{Arabic}}
\flushleft{\begin{malayalam}
നിന്റെ കൈകള്‍ നേരത്തെ നേടിവെച്ചതിന്റെ ഫലമാണിത്. അല്ലാഹു തന്റെ ദാസന്മാരോട് അനീതി കാട്ടുന്നവനല്ല.
\end{malayalam}}
\flushright{\begin{Arabic}
\quranayah[22][11]
\end{Arabic}}
\flushleft{\begin{malayalam}
ഓരത്ത്നിന്ന് അല്ലാഹുവിന് വഴിപ്പെടുന്ന ചിലരുണ്ട്. നേട്ടം വല്ലതും കിട്ടുകയാണെങ്കില്‍ അതിലവന്‍ സമാധാനമടയും. വല്ല വിപത്തും വന്നാലോ, അപ്പോഴവന്‍ തിരിഞ്ഞുകളയും. അവന് ഇഹവും പരവും നഷ്ടപ്പെട്ടതുതന്നെ. പ്രകടമായ നഷ്ടവും ഇതത്രെ.
\end{malayalam}}
\flushright{\begin{Arabic}
\quranayah[22][12]
\end{Arabic}}
\flushleft{\begin{malayalam}
അല്ലാഹുവെക്കൂടാതെ തനിക്ക് ഉപകാരമോ ഉപദ്രവമോ ചെയ്യാനാവാത്ത വസ്തുക്കളെയവന്‍ വിളിച്ചുപ്രാര്‍ഥിക്കുന്നു. ഇതുതന്നെയാണ് പരമമായ വഴികേട്.
\end{malayalam}}
\flushright{\begin{Arabic}
\quranayah[22][13]
\end{Arabic}}
\flushleft{\begin{malayalam}
ആരുടെ ഉപദ്രവം അവന്റെ ഉപകാരത്തെക്കാള്‍ അടുത്തതാണോ അവരെയാണവന്‍ വിളിച്ചുപ്രാര്‍ഥിക്കുന്നത്. അവന്റെ രക്ഷകന്‍ എത്ര ചീത്ത! എത്ര വിലകെട്ട കൂട്ടുകാരന്‍!
\end{malayalam}}
\flushright{\begin{Arabic}
\quranayah[22][14]
\end{Arabic}}
\flushleft{\begin{malayalam}
സത്യവിശ്വാസം സ്വീകരിക്കുകയും സല്‍ക്കര്‍മങ്ങള്‍ പ്രവര്‍ത്തിക്കുകയും ചെയ്യുന്നവരെ അല്ലാഹു, താഴ്ഭാഗത്തൂടെ ആറുകളൊഴുകുന്ന സ്വര്‍ഗീയാരാമങ്ങളില്‍ പ്രവേശിപ്പിക്കും. അല്ലാഹു അവനിച്ഛിക്കുന്നത് ചെയ്യുന്നു.
\end{malayalam}}
\flushright{\begin{Arabic}
\quranayah[22][15]
\end{Arabic}}
\flushleft{\begin{malayalam}
ഇഹത്തിലും പരത്തിലും പ്രവാചകനെ അല്ലാഹു സഹായിക്കാന്‍ പോകുന്നില്ലെന്ന് കരുതുന്നവന്‍, ആകാശത്തേക്ക് ഒരു കയര്‍ നീട്ടിക്കെട്ടിയിട്ട് ആ സഹായം മുറിച്ചുകളയട്ടെ. എന്നിട്ട് തന്നെ വെറുപ്പ് പിടിപ്പിക്കുന്ന അക്കാര്യം ഇല്ലാതാക്കാന്‍ തന്റെ തന്ത്രം കൊണ്ട് സാധിക്കുമോയെന്ന് അവനൊന്ന് നോക്കട്ടെ.
\end{malayalam}}
\flushright{\begin{Arabic}
\quranayah[22][16]
\end{Arabic}}
\flushleft{\begin{malayalam}
ഇവ്വിധം പ്രകടമായ തെളിവുകളുമായി നാം ഈ ഖുര്‍ആന്‍ ഇറക്കിത്തന്നിരിക്കുന്നു. അല്ലാഹു അവനിച്ഛിക്കുന്നവരെ നേര്‍വഴിയില്‍ നയിക്കുന്നു.
\end{malayalam}}
\flushright{\begin{Arabic}
\quranayah[22][17]
\end{Arabic}}
\flushleft{\begin{malayalam}
സത്യവിശ്വാസികള്‍, യഹൂദര്‍, സാബികള്‍, ക്രിസ്ത്യാനികള്‍, മജൂസികള്‍, ബഹുദൈവവിശ്വാസികള്‍ എന്നിവര്‍ക്കിടയില്‍ ഉയിര്‍ത്തെഴുന്നേല്‍പുനാളില്‍ അല്ലാഹു തീര്‍പ്പുകല്‍പിക്കുക തന്നെ ചെയ്യും. അല്ലാഹു സകലസംഗതികള്‍ക്കും സാക്ഷിയാകുന്നു.
\end{malayalam}}
\flushright{\begin{Arabic}
\quranayah[22][18]
\end{Arabic}}
\flushleft{\begin{malayalam}
ആകാശങ്ങളിലുള്ളവര്‍, ഭൂമിയിലുള്ളവര്‍, സൂര്യന്‍, ചന്ദ്രന്‍, നക്ഷത്രങ്ങള്‍, മലകള്‍, മരങ്ങള്‍, ജീവജാലങ്ങള്‍, എണ്ണമറ്റ മനുഷ്യര്‍, എല്ലാം അല്ലാഹുവിന് പ്രണാമമര്‍പ്പിച്ചുകൊണ്ടിരിക്കുന്നത് നീ കാണുന്നില്ലേ? കുറേപേര്‍ ദൈവശിക്ഷക്ക് അര്‍ഹരായിരിക്കുന്നു. അല്ലാഹു ആരെയെങ്കിലും അപമാനിതനാക്കുകയാണെങ്കില്‍ അയാളെ ആദരണീയനാക്കാന്‍ ആര്‍ക്കുമാവില്ല. സംശയം വേണ്ട; അല്ലാഹു അവനിച്ഛിക്കുന്നതു ചെയ്യുന്നു.
\end{malayalam}}
\flushright{\begin{Arabic}
\quranayah[22][19]
\end{Arabic}}
\flushleft{\begin{malayalam}
തങ്ങളുടെ നാഥന്റെ കാര്യത്തില്‍ തര്‍ക്കത്തിലേര്‍പ്പെട്ട രണ്ടു കക്ഷികളാണിത്. എന്നാല്‍ സത്യത്തെ തള്ളിപ്പറഞ്ഞവര്‍ക്ക് തീയാലുള്ള തുണി മുറിച്ചുകൊടുക്കുന്നതാണ്. അവരുടെ തലയ്ക്കുമീതെ തിളച്ചവെള്ളം ഒഴിക്കും.
\end{malayalam}}
\flushright{\begin{Arabic}
\quranayah[22][20]
\end{Arabic}}
\flushleft{\begin{malayalam}
അതുവഴി അവരുടെ വയറ്റിലുള്ളതും തൊലിയും ഉരുകിപ്പോകും.
\end{malayalam}}
\flushright{\begin{Arabic}
\quranayah[22][21]
\end{Arabic}}
\flushleft{\begin{malayalam}
അവര്‍ക്കെതിരെ ഇരുമ്പുദണ്ഡുകള്‍ പ്രയോഗിക്കും.
\end{malayalam}}
\flushright{\begin{Arabic}
\quranayah[22][22]
\end{Arabic}}
\flushleft{\begin{malayalam}
അവര്‍ ആ നരകത്തീയില്‍നിന്ന് കൊടുംക്ളേശം കാരണം പുറത്തുപോകാന്‍ ഉദ്ദേശിക്കുമ്പോഴൊക്കെ അവരെ അതിലേക്കുതന്നെ തിരിച്ചയക്കും. കരിച്ചുകളയുന്ന ശിക്ഷ നിങ്ങളനുഭവിച്ചുകൊള്ളുക.
\end{malayalam}}
\flushright{\begin{Arabic}
\quranayah[22][23]
\end{Arabic}}
\flushleft{\begin{malayalam}
സത്യവിശ്വാസം സ്വീകരിക്കുകയും സല്‍ക്കര്‍മങ്ങള്‍ പ്രവര്‍ത്തിക്കുകയും ചെയ്തവരെ അല്ലാഹു താഴ്ഭാഗത്തൂടെ ആറുകളൊഴുകുന്ന സ്വര്‍ഗീയാരാമങ്ങളില്‍ പ്രവേശിപ്പിക്കുകതന്നെ ചെയ്യും. അവരെയവിടെ സ്വര്‍ണവളകളും മുത്തും അണിയിക്കും. അവരുടെ വസ്ത്രങ്ങള്‍ മിനുത്ത പട്ടുകൊണ്ടുള്ളവയായിരിക്കും.
\end{malayalam}}
\flushright{\begin{Arabic}
\quranayah[22][24]
\end{Arabic}}
\flushleft{\begin{malayalam}
ഏറ്റം ഉല്‍കൃഷ്ടമായ വചനത്തിലേക്കാണവര്‍ നയിക്കപ്പെട്ടത്. സ്തുത്യര്‍ഹനായ അല്ലാഹുവിന്റെ മാര്‍ഗത്തിലേക്കാണവര്‍ ആനയിക്കപ്പെട്ടത്.
\end{malayalam}}
\flushright{\begin{Arabic}
\quranayah[22][25]
\end{Arabic}}
\flushleft{\begin{malayalam}
സത്യത്തെ തള്ളിപ്പറയുകയും അല്ലാഹുവിന്റെ മാര്‍ഗത്തില്‍ നിന്ന് ജനത്തെ തടയുകയും ചെയ്തവര്‍ ശിക്ഷാര്‍ഹരാണ്. നാം സര്‍വ ജനത്തിനുമായി നിര്‍മിച്ചുവെച്ചതും തദ്ദേശീയര്‍ക്കും പരദേശികള്‍ക്കും തുല്യാവകാശമുള്ളതുമായ മസ്ജിദുല്‍ ഹറാമിലേക്കുള്ള പ്രവേശനത്തിന് വിലക്കേര്‍പ്പെടുത്തിയവരും ശിക്ഷാര്‍ഹര്‍ തന്നെ. അവിടെവെച്ച് അന്യായമായി അധര്‍മം കാട്ടാനുദ്ദേശിക്കുന്നവരെ നാം നോവേറിയശിക്ഷ ആസ്വദിപ്പിക്കുകതന്നെ ചെയ്യും.
\end{malayalam}}
\flushright{\begin{Arabic}
\quranayah[22][26]
\end{Arabic}}
\flushleft{\begin{malayalam}
ഇബ്റാഹീമിനു നാം ആ മന്ദിരത്തിന്റെ സ്ഥാനം നിര്‍ണയിച്ചുകൊടുത്ത സന്ദര്‍ഭം: ഒന്നിനെയും എന്റെ പങ്കാളിയാക്കരുതെന്ന് നാം നിര്‍ദേശിച്ചു; ത്വവാഫ് ചെയ്യുന്നവര്‍ക്കും നിന്നു നമസ്കരിക്കുന്നവര്‍ക്കും നമിക്കുന്നവര്‍ക്കും സാഷ്ടാംഗം പ്രണമിക്കുന്നവര്‍ക്കും വേണ്ടി എന്റെ ആ മന്ദിരം ശുദ്ധമാക്കിവെക്കണമെന്നും.
\end{malayalam}}
\flushright{\begin{Arabic}
\quranayah[22][27]
\end{Arabic}}
\flushleft{\begin{malayalam}
തീര്‍ഥാടനത്തിനായി നീ ജനങ്ങള്‍ക്കിടയില്‍ പൊതുവിളംബരം നടത്തുക. ദൂരദിക്കുകളില്‍ നിന്നുപോലും ആളുകള്‍ കാല്‍നടയായും മെലിഞ്ഞ ഒട്ടകങ്ങളുടെ പുറത്തും നിന്റെയടുത്ത് വന്നെത്തും.
\end{malayalam}}
\flushright{\begin{Arabic}
\quranayah[22][28]
\end{Arabic}}
\flushleft{\begin{malayalam}
അവിടെ അവര്‍ തങ്ങള്‍ക്കുപകരിക്കുന്ന രംഗങ്ങളില്‍ സന്നിഹിതരാകും. അല്ലാഹു അവര്‍ക്കേകിയ മൃഗങ്ങളെ ചില നിര്‍ണിത ദിവസങ്ങളില്‍ അവന്റെ പേരുച്ചരിച്ച് ബലിയര്‍പ്പിക്കും. ആ ബലിമാംസം നിങ്ങള്‍ തിന്നുക. പ്രയാസക്കാര്‍ക്കും പാവങ്ങള്‍ക്കും തിന്നാന്‍ കൊടുക്കുക.
\end{malayalam}}
\flushright{\begin{Arabic}
\quranayah[22][29]
\end{Arabic}}
\flushleft{\begin{malayalam}
പിന്നീടവര്‍ തങ്ങളുടെ അഴുക്കുകള്‍ നീക്കിക്കളയട്ടെ. നേര്‍ച്ചകള്‍ നിറവേറ്റട്ടെ. ആ പുരാതനമന്ദിരത്തെ ചുറ്റട്ടെ.
\end{malayalam}}
\flushright{\begin{Arabic}
\quranayah[22][30]
\end{Arabic}}
\flushleft{\begin{malayalam}
അല്ലാഹുവിന്റെ കല്‍പനയാണിത്. അല്ലാഹു ആദരണീയമാക്കിയവയെ അംഗീകരിച്ചാദരിക്കുന്നവന് തന്റെ നാഥന്റെ അടുക്കലത് ഏറെ ഗുണകരമായിരിക്കും. നിങ്ങള്‍ക്ക് ഖുര്‍ആനിലൂടെ വിവരിച്ചുതന്നതൊഴികെയുള്ള നാല്‍ക്കാലികള്‍ നിങ്ങള്‍ക്ക് അനുവദനീയമാണ്. അതിനാല്‍ വിഗ്രഹങ്ങളാകുന്ന മാലിന്യങ്ങളില്‍നിന്ന് അകന്നുനില്‍ക്കുക. വ്യാജവാക്കുകള്‍ വര്‍ജിക്കുക.
\end{malayalam}}
\flushright{\begin{Arabic}
\quranayah[22][31]
\end{Arabic}}
\flushleft{\begin{malayalam}
അല്ലാഹുവില്‍ ഒന്നിനെയും പങ്കുചേര്‍ക്കാതെ ഉറച്ചമനസ്സോടെ അവനിലേക്കു തിരിയുക. അല്ലാഹുവിന് പങ്കാളികളെ കല്‍പിക്കുന്നവന്‍ ആകാശത്തുനിന്ന് വീണവനെപ്പോലെയാണ്. അങ്ങനെ പക്ഷികള്‍ അവനെ റാഞ്ചിയെടുക്കുന്നു. അല്ലെങ്കില്‍ കാറ്റ് അവനെ ഏതെങ്കിലും വിദൂരദിക്കില്‍ കൊണ്ടുപോയിത്തള്ളുന്നു.
\end{malayalam}}
\flushright{\begin{Arabic}
\quranayah[22][32]
\end{Arabic}}
\flushleft{\begin{malayalam}
കാര്യമിതാണ്. ആരെങ്കിലും അല്ലാഹു നിശ്ചയിച്ച ചിഹ്നങ്ങളെ ആദരിക്കുന്നുവെങ്കില്‍ അത് ആത്മാര്‍ഥമായ ഹൃദയഭക്തിയില്‍ നിന്നുണ്ടാവുന്നതാണ്.
\end{malayalam}}
\flushright{\begin{Arabic}
\quranayah[22][33]
\end{Arabic}}
\flushleft{\begin{malayalam}
ഒരു നിശ്ചിത അവധിവരെ ആ ബലിമൃഗങ്ങളെ നിങ്ങള്‍ക്കുപയോഗിക്കാം. പിന്നീട് അതിന്റെ ബലിസ്ഥലം ആ പുണ്യപുരാതന മന്ദിരത്തിങ്കലാണ്.
\end{malayalam}}
\flushright{\begin{Arabic}
\quranayah[22][34]
\end{Arabic}}
\flushleft{\begin{malayalam}
ഓരോ സമുദായത്തിനും നാം ഓരോ ബലിനിയമം നിശ്ചയിച്ചിട്ടുണ്ട്. അല്ലാഹു അവര്‍ക്കേകിയ കന്നുകാലികളില്‍ അവന്റെ നാമമുച്ചരിച്ച് അറുക്കാന്‍വേണ്ടിയാണിത്. നിങ്ങളുടെ ദൈവം ഏകദൈവമാകുന്നു. അതിനാല്‍ നിങ്ങളവനുമാത്രം വഴിപ്പെടുക. വിനയം കാണിക്കുന്നവരെ ശുഭവാര്‍ത്ത യറിയിക്കുക.
\end{malayalam}}
\flushright{\begin{Arabic}
\quranayah[22][35]
\end{Arabic}}
\flushleft{\begin{malayalam}
അല്ലാഹുവെക്കുറിച്ച് കേള്‍ക്കുമ്പോള്‍ ഹൃദയങ്ങള്‍ ഭയചകിതരാകുന്നവരാണവര്‍. ഏതു വിപത്വേളകളിലും ക്ഷമയവലംബിക്കുന്നവരും. നമസ്കാരം നിഷ്ഠയോടെ നിര്‍വഹിക്കുന്നവരും നാം നല്‍കിയതില്‍നിന്ന് ചെലവഴിക്കുന്നവരുമാണ്.
\end{malayalam}}
\flushright{\begin{Arabic}
\quranayah[22][36]
\end{Arabic}}
\flushleft{\begin{malayalam}
ബലിയൊട്ടകങ്ങളെ നാം നിങ്ങള്‍ക്കുള്ള ദൈവിക ചിഹ്നങ്ങളിലുള്‍പ്പെടുത്തിയിരിക്കുന്നു. നിശ്ചയമായും നിങ്ങള്‍ക്കവയില്‍ നന്മയുണ്ട്. അതിനാല്‍ നിങ്ങളവയെ അണിയായിനിര്‍ത്തി അല്ലാഹുവിന്റെ പേരുച്ചരിച്ച് ബലിയര്‍പ്പിക്കുക. അങ്ങനെ പാര്‍ശ്വങ്ങളിലേക്ക് അവ വീണുകഴിഞ്ഞാല്‍ നിങ്ങളവയുടെ മാംസം ഭക്ഷിക്കുക. ഉള്ളതുകൊണ്ട് തൃപ്തരായി കഴിയുന്നവരെയും ചോദിച്ചുവരുന്നവരെയും തീറ്റിക്കുക. അവയെ നാം നിങ്ങള്‍ക്ക് ഇവ്വിധം അധീനപ്പെടുത്തിത്തന്നിരിക്കുന്നു. നിങ്ങള്‍ നന്ദി കാണിക്കാനാണിത്.
\end{malayalam}}
\flushright{\begin{Arabic}
\quranayah[22][37]
\end{Arabic}}
\flushleft{\begin{malayalam}
അവയുടെ മാംസമോ രക്തമോ അല്ലാഹുവെ പ്രാപിക്കുന്നില്ല. മറിച്ച് അല്ലാഹുവിലെത്തിച്ചേരുന്നത് നിങ്ങളുടെ ഭക്തിയാണ്. അവന്‍ നിങ്ങള്‍ക്കവയെ അവ്വിധം അധീനപ്പെടുത്തിത്തന്നിരിക്കുന്നു. അല്ലാഹു നിങ്ങളെ നേര്‍വഴിയിലാക്കിയതിന് നിങ്ങളവന്റെ മഹത്വം കീര്‍ത്തിക്കാന്‍ വേണ്ടിയാണത്. സച്ചരിതരെ നീ ശുഭവാര്‍ത്ത അറിയിക്കുക.
\end{malayalam}}
\flushright{\begin{Arabic}
\quranayah[22][38]
\end{Arabic}}
\flushleft{\begin{malayalam}
സത്യവിശ്വാസികളെ അല്ലാഹു സംരക്ഷിച്ചുനിര്‍ത്തുകതന്നെ ചെയ്യും. തീര്‍ച്ചയായും നന്ദികെട്ട ചതിയന്മാരെ അവന്‍ ഇഷ്ടപ്പെടുന്നില്ല.
\end{malayalam}}
\flushright{\begin{Arabic}
\quranayah[22][39]
\end{Arabic}}
\flushleft{\begin{malayalam}
യുദ്ധത്തിനിരയായവര്‍ക്ക് തിരിച്ചടിക്കാന്‍ അനുവാദം നല്‍കിയിരിക്കുന്നു. കാരണം അവര്‍ മര്‍ദിതരാണ്. ഉറപ്പായും അല്ലാഹു അവരെ സഹായിക്കാന്‍ പോന്നവന്‍ തന്നെ.
\end{malayalam}}
\flushright{\begin{Arabic}
\quranayah[22][40]
\end{Arabic}}
\flushleft{\begin{malayalam}
സ്വന്തം വീടുകളില്‍നിന്ന് അന്യായമായി ഇറക്കപ്പെട്ടവരാണവര്‍. “ഞങ്ങളുടെ നാഥന്‍ അല്ലാഹുവാണ്” എന്നു പ്രഖ്യാപിച്ചതല്ലാതെ ഒരു തെറ്റുമവര്‍ ചെയ്തിട്ടില്ല. അല്ലാഹു ജനങ്ങളില്‍ ചിലരെ മറ്റുചിലരെക്കൊണ്ട് പ്രതിരോധിക്കുന്നില്ലായെങ്കില്‍ ദൈവനാമം ധാരാളമായി സ്മരിക്കപ്പെടുന്ന സന്യാസിമഠങ്ങളും ചര്‍ച്ചുകളും സെനഗോഗുകളും മുസ്ലിംപള്ളികളും തകര്‍ക്കപ്പെടുമായിരുന്നു. തന്നെ സഹായിക്കുന്നവരെ ഉറപ്പായും അല്ലാഹു സഹായിക്കും. അല്ലാഹു സര്‍വശക്തനും ഏറെ പ്രതാപിയും തന്നെ.
\end{malayalam}}
\flushright{\begin{Arabic}
\quranayah[22][41]
\end{Arabic}}
\flushleft{\begin{malayalam}
അവയുടെ മാംസമോ രക്തമോ അല്ലാഹുവെ പ്രാപിക്കുന്നില്ല. മറിച്ച് അല്ലാഹുവിലെത്തിച്ചേരുന്നത് നിങ്ങളുടെ ഭക്തിയാണ്. അവന്‍ നിങ്ങള്‍ക്കവയെ അവ്വിധം അധീനപ്പെടുത്തിത്തന്നിരിക്കുന്നു. അല്ലാഹു നിങ്ങളെ നേര്‍വഴിയിലാക്കിയതിന് നിങ്ങളവന്റെ മഹത്വം കീര്‍ത്തിക്കാന്‍ വേണ്ടിയാണത്. സച്ചരിതരെ നീ ശുഭവാര്‍ത്ത അറിയിക്കുക.
\end{malayalam}}
\flushright{\begin{Arabic}
\quranayah[22][42]
\end{Arabic}}
\flushleft{\begin{malayalam}
സത്യവിശ്വാസികളെ അല്ലാഹു സംരക്ഷിച്ചുനിര്‍ത്തുകതന്നെ ചെയ്യും. തീര്‍ച്ചയായും നന്ദികെട്ട ചതിയന്മാരെ അവന്‍ ഇഷ്ടപ്പെടുന്നില്ല.
\end{malayalam}}
\flushright{\begin{Arabic}
\quranayah[22][43]
\end{Arabic}}
\flushleft{\begin{malayalam}
യുദ്ധത്തിനിരയായവര്‍ക്ക് തിരിച്ചടിക്കാന്‍ അനുവാദം നല്‍കിയിരിക്കുന്നു. കാരണം അവര്‍ മര്‍ദിതരാണ്. ഉറപ്പായും അല്ലാഹു അവരെ സഹായിക്കാന്‍ പോന്നവന്‍ തന്നെ.
\end{malayalam}}
\flushright{\begin{Arabic}
\quranayah[22][44]
\end{Arabic}}
\flushleft{\begin{malayalam}
സ്വന്തം വീടുകളില്‍നിന്ന് അന്യായമായി ഇറക്കപ്പെട്ടവരാണവര്‍. “ഞങ്ങളുടെ നാഥന്‍ അല്ലാഹുവാണ്” എന്നു പ്രഖ്യാപിച്ചതല്ലാതെ ഒരു തെറ്റുമവര്‍ ചെയ്തിട്ടില്ല. അല്ലാഹു ജനങ്ങളില്‍ ചിലരെ മറ്റുചിലരെക്കൊണ്ട് പ്രതിരോധിക്കുന്നില്ലായെങ്കില്‍ ദൈവനാമം ധാരാളമായി സ്മരിക്കപ്പെടുന്ന സന്യാസിമഠങ്ങളും ചര്‍ച്ചുകളും സെനഗോഗുകളും മുസ്ലിംപള്ളികളും തകര്‍ക്കപ്പെടുമായിരുന്നു. തന്നെ സഹായിക്കുന്നവരെ ഉറപ്പായും അല്ലാഹു സഹായിക്കും. അല്ലാഹു സര്‍വശക്തനും ഏറെ പ്രതാപിയും തന്നെ.
\end{malayalam}}
\flushright{\begin{Arabic}
\quranayah[22][45]
\end{Arabic}}
\flushleft{\begin{malayalam}
എത്രയെത്ര നാടുകളെയാണ് നാം നശിപ്പിച്ചത്. അന്നാട്ടുകാര്‍ കൊടിയ അതിക്രമികളായിരുന്നു. അവര്‍ തങ്ങളുടെ വീടുകളുടെ മേല്‍പ്പുരകളോടെ തകര്‍ന്നടിഞ്ഞു. എത്രയെത്ര കിണറുകളാണ് ഉപയോഗശൂന്യമായിത്തീര്‍ന്നത്! എത്രയേറെ കൂറ്റന്‍ കോട്ടകളാണ് നിലംപൊത്തിയത്.
\end{malayalam}}
\flushright{\begin{Arabic}
\quranayah[22][46]
\end{Arabic}}
\flushleft{\begin{malayalam}
അവര്‍ ഈ ഭൂമിയില്‍ സഞ്ചരിക്കാറില്ലേ? എങ്കിലവര്‍ക്ക് ചിന്തിക്കുന്ന മനസ്സുകളും കേള്‍ക്കുന്ന കാതുകളുമുണ്ടാകുമായിരുന്നു. സത്യത്തില്‍ അന്ധത ബാധിക്കുന്നത് കണ്ണുകളെയല്ല, നെഞ്ചകങ്ങളിലെ മനസ്സുകളെയാണ്.
\end{malayalam}}
\flushright{\begin{Arabic}
\quranayah[22][47]
\end{Arabic}}
\flushleft{\begin{malayalam}
അവര്‍ നിന്നോട് ശിക്ഷ വന്നുകിട്ടാന്‍ ധൃതികൂട്ടുന്നു. അല്ലാഹു തന്റെ വാഗ്ദാനം തെറ്റിക്കുകയില്ല. നിന്റെ നാഥന്റെയടുത്ത് ഒരു നാളെന്നത് നിങ്ങളെണ്ണും പോലുള്ള ആയിരംകൊല്ലങ്ങള്‍ക്കു തുല്യമാണ്.
\end{malayalam}}
\flushright{\begin{Arabic}
\quranayah[22][48]
\end{Arabic}}
\flushleft{\begin{malayalam}
അക്രമത്തിലാണ്ടുപോയിട്ടും എത്രയോ നാടുകള്‍ക്കു നാം സമയം നീട്ടിക്കൊടുത്തു. പിന്നെ നാം അവയെ പിടികൂടി. എല്ലാം തിരിച്ചുവരുന്നത് നമ്മുടെയടുത്തേക്കുതന്നെ.
\end{malayalam}}
\flushright{\begin{Arabic}
\quranayah[22][49]
\end{Arabic}}
\flushleft{\begin{malayalam}
പറയുക: "ജനങ്ങളേ; നിങ്ങള്‍ക്ക് വ്യക്തമായ മുന്നറിയിപ്പ് നല്‍കാന്‍ വന്നവന്‍ മാത്രമാണ് ഞാന്‍.”
\end{malayalam}}
\flushright{\begin{Arabic}
\quranayah[22][50]
\end{Arabic}}
\flushleft{\begin{malayalam}
സത്യവിശ്വാസം സ്വീകരിക്കുകയും സല്‍ക്കര്‍മങ്ങള്‍ പ്രവര്‍ത്തിക്കുകയും ചെയ്യുന്നവര്‍ക്ക് പാപമോചനമുണ്ട്. മാന്യമായ ഉപജീവനവും.
\end{malayalam}}
\flushright{\begin{Arabic}
\quranayah[22][51]
\end{Arabic}}
\flushleft{\begin{malayalam}
നമ്മുടെ വചനങ്ങളെ തോല്‍പിക്കാന്‍ ശ്രമിക്കുന്നവര്‍ തന്നെയാണ് നരകാവകാശികള്‍.
\end{malayalam}}
\flushright{\begin{Arabic}
\quranayah[22][52]
\end{Arabic}}
\flushleft{\begin{malayalam}
നിനക്കുമുമ്പ് നാമൊരു ദൂതനെയും പ്രവാചകനെയും അയച്ചിട്ടില്ല; അദ്ദേഹത്തിന്റെ പാരായണവേളയില്‍ പിശാച് അതില്‍ ഇടപെടാന്‍ ശ്രമിച്ചിട്ടല്ലാതെ. എന്നാല്‍ അല്ലാഹു പിശാചിന്റെ എല്ലാ ഇടപെടലുകളെയും തുടച്ചുമാറ്റുന്നു. അങ്ങനെ തന്റെ വചനങ്ങളെ ഭദ്രമാക്കുന്നു. അല്ലാഹു എല്ലാം അറിയുന്നവനും യുക്തിജ്ഞനുമാണ്.
\end{malayalam}}
\flushright{\begin{Arabic}
\quranayah[22][53]
\end{Arabic}}
\flushleft{\begin{malayalam}
മനസ്സില്‍ ദീനം ബാധിച്ചവര്‍ക്കും ഹൃദയങ്ങള്‍ കടുത്തുപോയവര്‍ക്കും പിശാചിന്റെ ഇടപെടലിനെ അല്ലാഹു ഒരു പരീക്ഷണമാക്കുകയാണ്. ഉറപ്പായും അതിക്രമികള്‍ ധിക്കാരപരമായ മാത്സര്യത്തില്‍ ഏറെദൂരം പിന്നിട്ടിരിക്കുന്നു.
\end{malayalam}}
\flushright{\begin{Arabic}
\quranayah[22][54]
\end{Arabic}}
\flushleft{\begin{malayalam}
അതോടൊപ്പം ജ്ഞാനം ലഭിച്ചവര്‍ അത് നിന്റെ നാഥനില്‍ നിന്നുള്ള സത്യമാണെന്ന് മനസ്സിലാക്കാനാണ് അങ്ങനെ ചെയ്യുന്നത്. അതുവഴി അവരതില്‍ വിശ്വസിക്കാനും തങ്ങളുടെ ഹൃദയങ്ങളെ അതിനു കീഴ്പ്പെടുത്താനുമാണ്. തീര്‍ച്ചയായും അല്ലാഹു സത്യവിശ്വാസികളെ നേര്‍വഴിക്ക് നയിക്കുന്നവനാണ്.
\end{malayalam}}
\flushright{\begin{Arabic}
\quranayah[22][55]
\end{Arabic}}
\flushleft{\begin{malayalam}
എന്നാല്‍ സത്യനിഷേധികള്‍ അതേക്കുറിച്ച് സദാ സംശയത്തിലായിരിക്കും. പെട്ടെന്ന് അന്ത്യസമയം ആസന്നമാകുംവരെ അതു തുടരും. അല്ലെങ്കില്‍ ഒരു ദുര്‍ദിനത്തിലെ ശിക്ഷ അവര്‍ക്കു വന്നെത്തുംവരെ.
\end{malayalam}}
\flushright{\begin{Arabic}
\quranayah[22][56]
\end{Arabic}}
\flushleft{\begin{malayalam}
അന്നാളില്‍ അധികാരമൊക്കെ അല്ലാഹുവിനായിരിക്കും. അവന്‍ അവര്‍ക്കിടയില്‍ തീര്‍പ്പുകല്‍പിക്കും. അപ്പോള്‍, സത്യവിശ്വാസം സ്വീകരിക്കുകയും സല്‍ക്കര്‍മങ്ങള്‍ പ്രവര്‍ത്തിക്കുകയും ചെയ്തവര്‍ അനുഗ്രഹപൂര്‍ണമായ സ്വര്‍ഗീയാരാമങ്ങളിലായിരിക്കും.
\end{malayalam}}
\flushright{\begin{Arabic}
\quranayah[22][57]
\end{Arabic}}
\flushleft{\begin{malayalam}
സത്യത്തെ നിഷേധിക്കുകയും നമ്മുടെ വചനങ്ങളെ കള്ളമാക്കി തള്ളുകയും ചെയ്തവര്‍ക്കുതന്നെയാണ് ഏറെ നിന്ദ്യവും ഹീനവുമായ ശിക്ഷയുണ്ടാവുക.
\end{malayalam}}
\flushright{\begin{Arabic}
\quranayah[22][58]
\end{Arabic}}
\flushleft{\begin{malayalam}
അല്ലാഹുവിന്റെ മാര്‍ഗത്തില്‍ നാടുവിടേണ്ടിവന്നശേഷം വധിക്കപ്പെടുകയോ മരണമടയുകയോ ചെയ്തവര്‍ക്ക് ഉറപ്പായും അല്ലാഹു ഉത്തമമായ ഉപജീവനം നല്‍കും. തീര്‍ച്ചയായും അല്ലാഹു തന്നെയാണ് ഉപജീവനം നല്‍കുന്നവരില്‍ അത്യുത്തമന്‍.
\end{malayalam}}
\flushright{\begin{Arabic}
\quranayah[22][59]
\end{Arabic}}
\flushleft{\begin{malayalam}
അവരാഗ്രഹിക്കുകയും തൃപ്തിപ്പെടുകയും ചെയ്യുന്നിടത്തേക്ക് അല്ലാഹു അവരെ കൊണ്ടെത്തിക്കും. അല്ലാഹു എല്ലാം അറിയുന്നവനും ഏറെ സഹനമുള്ളവനുമാണ്.
\end{malayalam}}
\flushright{\begin{Arabic}
\quranayah[22][60]
\end{Arabic}}
\flushleft{\begin{malayalam}
അത് അങ്ങനെയാണ്. ഒരാള്‍ താന്‍ ദ്രോഹിക്കപ്പെട്ടതുപോലെ പകരമങ്ങോട്ടും ചെയ്തശേഷം വീണ്ടും പീഡനത്തിനിരയാവുകയാണെങ്കില്‍ ഉറപ്പായും അല്ലാഹു അവനെ സഹായിക്കും. അല്ലാഹു ഏറെ പൊറുക്കുന്നവനും മാപ്പേകുന്നവനുമാണ്.
\end{malayalam}}
\flushright{\begin{Arabic}
\quranayah[22][61]
\end{Arabic}}
\flushleft{\begin{malayalam}
ഇതെന്തുകൊണ്ടെന്നാല്‍, തീര്‍ച്ചയായും അല്ലാഹുവാണ് രാവിനെ പകലിലേക്ക് കടത്തിവിടുന്നത്. പകലിനെ രാവില്‍ പ്രവേശിപ്പിക്കുന്നതും അവന്‍ തന്നെ. തീര്‍ച്ചയായും അല്ലാഹു എല്ലാം കേള്‍ക്കുന്നവനും കാണുന്നവനുമാകുന്നു.
\end{malayalam}}
\flushright{\begin{Arabic}
\quranayah[22][62]
\end{Arabic}}
\flushleft{\begin{malayalam}
അല്ലാഹു തന്നെയാണ് നിത്യസത്യം. അവനെക്കൂടാതെ അവര്‍ വിളിച്ചുപ്രാര്‍ഥിക്കുന്നവയൊക്കെയും കേവലം മിഥ്യയാണ്. അല്ലാഹു തന്നെയാണ് അത്യുന്നതനും മഹാനും.
\end{malayalam}}
\flushright{\begin{Arabic}
\quranayah[22][63]
\end{Arabic}}
\flushleft{\begin{malayalam}
നീ കാണുന്നില്ലേ; അല്ലാഹു മാനത്തുനിന്നു മഴ വീഴ്ത്തുന്നത്? അതുവഴി ഭൂമി പച്ചപ്പുള്ളതായിത്തീരുന്നു. അല്ലാഹു എല്ലാം സൂക്ഷ്മമായി അറിയുന്നവനും തിരിച്ചറിവുള്ളവനുമാണ്.
\end{malayalam}}
\flushright{\begin{Arabic}
\quranayah[22][64]
\end{Arabic}}
\flushleft{\begin{malayalam}
ആകാശഭൂമികളിലുള്ളതെല്ലാം അവന്റെതാണ്. അല്ലാഹു അന്യാശ്രയം ആവശ്യമില്ലാത്തവനാണ്. സ്തുത്യര്‍ഹനും.
\end{malayalam}}
\flushright{\begin{Arabic}
\quranayah[22][65]
\end{Arabic}}
\flushleft{\begin{malayalam}
നീ കാണുന്നില്ലേ; അല്ലാഹു നിങ്ങള്‍ക്ക് ഈ ഭൂമിയിലുള്ളതൊക്കെയും അധീനപ്പെടുത്തിത്തന്നിരിക്കുന്നു; അവന്റെ ഹിതമനുസരിച്ച് കടലില്‍ സഞ്ചരിക്കുന്ന കപ്പലും. തന്റെ അനുമതിയില്ലാതെ ഭൂമിക്കുമേല്‍ വീണുപോകാത്തവിധം വാനലോകത്തെ പിടിച്ചുനിര്‍ത്തുന്നതും അവനാണ്. തീര്‍ച്ചയായും അല്ലാഹു മനുഷ്യരോട് ഏറെ കൃപയുള്ളവനും പരമ കാരുണികനുമാണ്.
\end{malayalam}}
\flushright{\begin{Arabic}
\quranayah[22][66]
\end{Arabic}}
\flushleft{\begin{malayalam}
അവനാണ് നിങ്ങളെ ജീവിപ്പിച്ചത്. ഇനിയവന്‍ നിങ്ങളെ മരിപ്പിക്കും. പിന്നെ വീണ്ടും ജീവിപ്പിക്കും. തീര്‍ച്ചയായും മനുഷ്യന്‍ വളരെയേറെ നന്ദികെട്ടവനാണ്.
\end{malayalam}}
\flushright{\begin{Arabic}
\quranayah[22][67]
\end{Arabic}}
\flushleft{\begin{malayalam}
എല്ലാ ഓരോ സമുദായത്തിനും നാം ഓരോതരം ആരാധനാരീതി നിശ്ചയിച്ചിട്ടുണ്ട്. അവരതനുഷ്ഠിച്ചുപോരുന്നു. അതിനാല്‍ ഇക്കാര്യത്തില്‍ അവരാരും നിന്നോട് കലഹിക്കാതിരിക്കട്ടെ. നീയവരെ നിന്റെ നാഥങ്കലേക്ക് ക്ഷണിച്ചുകൊള്ളുക. തീര്‍ച്ചയായും നീ വളവൊട്ടുമില്ലാത്ത നേര്‍വഴിയില്‍ തന്നെയാണ്.
\end{malayalam}}
\flushright{\begin{Arabic}
\quranayah[22][68]
\end{Arabic}}
\flushleft{\begin{malayalam}
അവര്‍ നിന്നോട് തര്‍ക്കിക്കുന്നുവെങ്കില്‍ പറയുക: നിങ്ങള്‍ പ്രവര്‍ത്തിക്കുന്നതിനെപ്പറ്റിയെല്ലാം നന്നായറിയുന്നവനാണ് അല്ലാഹു.
\end{malayalam}}
\flushright{\begin{Arabic}
\quranayah[22][69]
\end{Arabic}}
\flushleft{\begin{malayalam}
നിങ്ങള്‍ ഭിന്നിച്ചകന്നുകൊണ്ടിരിക്കുന്ന വിഷയങ്ങളിലെല്ലാം ഉയിര്‍ത്തെഴുന്നേല്‍പുനാളില്‍ അല്ലാഹു നിങ്ങള്‍ക്കിടയില്‍ തീര്‍പ്പുകല്‍പിക്കും.
\end{malayalam}}
\flushright{\begin{Arabic}
\quranayah[22][70]
\end{Arabic}}
\flushleft{\begin{malayalam}
നിനക്കറിഞ്ഞുകൂടേ; ആകാശഭൂമികളിലുള്ളതെല്ലാം അല്ലാഹുവിന്ന് നന്നായറിയാമെന്ന്. തീര്‍ച്ചയായും അതൊക്കെയും ഒരു മൂല പ്രമാണത്തിലുണ്ട്. അതെല്ലാം അല്ലാഹുവിന് ഏറെ എളുപ്പമാണ്.
\end{malayalam}}
\flushright{\begin{Arabic}
\quranayah[22][71]
\end{Arabic}}
\flushleft{\begin{malayalam}
അല്ലാഹു ഒരു തെളിവും അവതരിപ്പിച്ചിട്ടില്ലാത്തവയെ അവന്റെ പങ്കുകാരായി സങ്കല്‍പിച്ച് അവര്‍ പൂജിച്ചുകൊണ്ടിരിക്കുന്നു. അവര്‍ക്ക് അതേക്കുറിച്ച് ഒന്നുമറിയില്ല. അക്രമികള്‍ക്ക് ഒരു സഹായിയുമുണ്ടാവുകയില്ല.
\end{malayalam}}
\flushright{\begin{Arabic}
\quranayah[22][72]
\end{Arabic}}
\flushleft{\begin{malayalam}
നമ്മുടെ സുവ്യക്തമായ വചനങ്ങള്‍ അവരെ ഓതിക്കേള്‍പ്പിക്കുകയാണെങ്കില്‍ സത്യനിഷേധികളുടെ മുഖങ്ങളില്‍ വെറുപ്പ് പ്രകടമാകുന്നത് നിനക്കു മനസ്സിലാകും. നമ്മുടെ വചനങ്ങള്‍ വായിച്ചുകേള്‍പ്പിക്കുന്നവരെ കയ്യേറ്റം ചെയ്യാന്‍പോലും അവര്‍ മുതിര്‍ന്നേക്കാം. പറയുക: അതിനെക്കാളെല്ലാം ദോഷകരമായ കാര്യം ഏതെന്ന് ഞാന്‍ നിങ്ങള്‍ക്കറിയിച്ചുതരട്ടെയോ? നരകത്തീയാണത്. സത്യത്തെ തള്ളിപ്പറഞ്ഞവര്‍ക്ക് അതാണ് അല്ലാഹു വാഗ്ദാനം ചെയ്തിട്ടുള്ളത്. അതെത്ര ചീത്ത സങ്കേതം!
\end{malayalam}}
\flushright{\begin{Arabic}
\quranayah[22][73]
\end{Arabic}}
\flushleft{\begin{malayalam}
മനുഷ്യരേ, ഒരുദാഹരണമിങ്ങനെ വിശദീകരിക്കാം. നിങ്ങളിത് ശ്രദ്ധയോടെ കേള്‍ക്കുക: അല്ലാഹുവെക്കൂടാതെ നിങ്ങള്‍ വിളിച്ചുപ്രാര്‍ഥിച്ചുകൊണ്ടിരിക്കുന്നവരെല്ലാം ഒരുമിച്ചുചേര്‍ന്ന് ശ്രമിച്ചാലും ഒരീച്ചയെപ്പോലും സൃഷ്ടിക്കാന്‍ അവര്‍ക്കാവില്ല. എന്നല്ല; ഈച്ച അവരുടെ പക്കല്‍നിന്നെന്തെങ്കിലും തട്ടിയെടുത്താല്‍ അത് മോചിപ്പിച്ചെടുക്കാന്‍പോലും അവര്‍ക്ക് സാധ്യമല്ല. സഹായം തേടുന്നവനും തേടപ്പെടുന്നവനും ഏറെ ദുര്‍ബലര്‍ തന്നെ.
\end{malayalam}}
\flushright{\begin{Arabic}
\quranayah[22][74]
\end{Arabic}}
\flushleft{\begin{malayalam}
അല്ലാഹുവെ അവനര്‍ഹിക്കുംവിധം നിങ്ങള്‍ പരിഗണിച്ചിട്ടില്ല. തീര്‍ച്ചയായും അല്ലാഹു കരുത്തനും പ്രതാപിയുമാണ്.
\end{malayalam}}
\flushright{\begin{Arabic}
\quranayah[22][75]
\end{Arabic}}
\flushleft{\begin{malayalam}
മലക്കുകളില്‍നിന്നും മനുഷ്യരില്‍നിന്നും അല്ലാഹു ചില സന്ദേശവാഹകരെ തിരഞ്ഞെടുക്കുന്നു. അല്ലാഹു എല്ലാം കേള്‍ക്കുന്നവനും കാണുന്നവനുമാണ്.
\end{malayalam}}
\flushright{\begin{Arabic}
\quranayah[22][76]
\end{Arabic}}
\flushleft{\begin{malayalam}
അവരുടെ ഭാവിയും ഭൂതവും അവനറിയുന്നു. കാര്യങ്ങളെല്ലാം വിധിത്തീര്‍പ്പിനായി മടക്കപ്പെടുന്നത് അവങ്കലേക്കാണ്.
\end{malayalam}}
\flushright{\begin{Arabic}
\quranayah[22][77]
\end{Arabic}}
\flushleft{\begin{malayalam}
വിശ്വസിച്ചവരേ, നിങ്ങള്‍ നമിക്കുക. സാഷ്ടാംഗം പ്രണമിക്കുക. നിങ്ങളുടെ നാഥന്ന് വഴിപ്പെടുക. നന്മ ചെയ്യുക. നിങ്ങള്‍ വിജയംവരിച്ചേക്കാം.
\end{malayalam}}
\flushright{\begin{Arabic}
\quranayah[22][78]
\end{Arabic}}
\flushleft{\begin{malayalam}
അല്ലാഹുവിന്റെ മാര്‍ഗത്തില്‍ പൊരുതേണ്ടവിധം പൊരുതുക. അവന്‍ നിങ്ങളെ പ്രത്യേകം തെരഞ്ഞെടുത്തിരിക്കുന്നു. മതകാര്യത്തില്‍ ഒരു വിഷമവും അവന്‍ നിങ്ങള്‍ക്കുണ്ടാക്കിവെച്ചിട്ടില്ല. നിങ്ങളുടെ പിതാവായ ഇബ്റാഹീമിന്റെ പാത പിന്തുടരുക. പണ്ടേതന്നെ അല്ലാഹു നിങ്ങളെ മുസ്ലിംകളെന്ന് വിളിച്ചിരിക്കുന്നു. ഈ ഖുര്‍ആനിലും അതുതന്നെയാണ് വിളിപ്പേര്. ദൈവദൂതന്‍ നിങ്ങള്‍ക്ക് സാക്ഷിയാകാനാണിത്. നിങ്ങള്‍ ജനങ്ങള്‍ക്ക് സാക്ഷികളാകാനും. അതിനാല്‍ നിങ്ങള്‍ നമസ്കാരം നിഷ്ഠയോടെ നിര്‍വഹിക്കുക. സകാത്ത് നല്‍കുക. അല്ലാഹുവിനെ മുറുകെ പിടിക്കുക. അവനാണ് നിങ്ങളുടെ രക്ഷകന്‍. എത്ര നല്ല രക്ഷകന്‍! എത്ര നല്ല സഹായി!
\end{malayalam}}
\chapter{\textmalayalam{അല്‍ മുഅ്മിനൂന്‍ ( സത്യവിശ്വാസികള്‍ )}}
\begin{Arabic}
\Huge{\centerline{\basmalah}}\end{Arabic}
\flushright{\begin{Arabic}
\quranayah[23][1]
\end{Arabic}}
\flushleft{\begin{malayalam}
നിശ്ചയമായും സത്യവിശ്വാസികള്‍ വിജയിച്ചിരിക്കുന്നു.
\end{malayalam}}
\flushright{\begin{Arabic}
\quranayah[23][2]
\end{Arabic}}
\flushleft{\begin{malayalam}
അവര്‍ തങ്ങളുടെ നമസ്കാരത്തില്‍ ഭക്തി പുലര്‍ത്തുന്നവരാണ്.
\end{malayalam}}
\flushright{\begin{Arabic}
\quranayah[23][3]
\end{Arabic}}
\flushleft{\begin{malayalam}
അനാവശ്യങ്ങളില്‍നിന്ന് അകന്നുനില്‍ക്കുന്നവരാണ്;
\end{malayalam}}
\flushright{\begin{Arabic}
\quranayah[23][4]
\end{Arabic}}
\flushleft{\begin{malayalam}
സകാത്ത് നല്‍കുന്നവരും.
\end{malayalam}}
\flushright{\begin{Arabic}
\quranayah[23][5]
\end{Arabic}}
\flushleft{\begin{malayalam}
തങ്ങളുടെ ലൈംഗികവിശുദ്ധി കാത്തുസൂക്ഷിക്കുന്നവരുമാണ്.
\end{malayalam}}
\flushright{\begin{Arabic}
\quranayah[23][6]
\end{Arabic}}
\flushleft{\begin{malayalam}
തങ്ങളുടെ ഇണകളും അധീനതയിലുള്ള സ്ത്രീകളുമായി മാത്രമേ അവര്‍ വേഴ്ചകളിലേര്‍പ്പെടുകയുള്ളൂ. അവരുമായുള്ള ബന്ധം ആക്ഷേപാര്‍ഹമല്ല.
\end{malayalam}}
\flushright{\begin{Arabic}
\quranayah[23][7]
\end{Arabic}}
\flushleft{\begin{malayalam}
എന്നാല്‍ അതിനപ്പുറം ആഗ്രഹിക്കുന്നവര്‍ അതിക്രമകാരികളാണ്.
\end{malayalam}}
\flushright{\begin{Arabic}
\quranayah[23][8]
\end{Arabic}}
\flushleft{\begin{malayalam}
ആ സത്യവിശ്വാസികള്‍ തങ്ങളുടെ ബാധ്യതകളും കരാറുകളും പൂര്‍ത്തീകരിക്കുന്നവരാണ്.
\end{malayalam}}
\flushright{\begin{Arabic}
\quranayah[23][9]
\end{Arabic}}
\flushleft{\begin{malayalam}
അവര്‍ തങ്ങളുടെ നമസ്കാരങ്ങള്‍ നിഷ്ഠയോടെ നിര്‍വഹിക്കുന്നവരാണ്.
\end{malayalam}}
\flushright{\begin{Arabic}
\quranayah[23][10]
\end{Arabic}}
\flushleft{\begin{malayalam}
അവര്‍ തന്നെയാണ് അനന്തരാവകാശികള്‍.
\end{malayalam}}
\flushright{\begin{Arabic}
\quranayah[23][11]
\end{Arabic}}
\flushleft{\begin{malayalam}
പറുദീസ അനന്തരമെടുക്കുന്നവര്‍. അവരതില്‍ നിത്യവാസികളായിരിക്കും.
\end{malayalam}}
\flushright{\begin{Arabic}
\quranayah[23][12]
\end{Arabic}}
\flushleft{\begin{malayalam}
മനുഷ്യനെ നാം കളിമണ്ണിന്റെ സത്തില്‍നിന്ന് സൃഷ്ടിച്ചു.
\end{malayalam}}
\flushright{\begin{Arabic}
\quranayah[23][13]
\end{Arabic}}
\flushleft{\begin{malayalam}
പിന്നെ നാമവനെ ബീജകണമാക്കി ഭദ്രമായ ഒരിടത്ത് സ്ഥാപിച്ചു.
\end{malayalam}}
\flushright{\begin{Arabic}
\quranayah[23][14]
\end{Arabic}}
\flushleft{\begin{malayalam}
അനന്തരം നാം ആ ബീജത്തെ ഭ്രൂണമാക്കി മാറ്റി. പിന്നീട് ഭ്രൂണത്തെ മാംസക്കട്ടയാക്കി. അതിനുശേഷം മാംസത്തെ എല്ലുകളാക്കി. എല്ലുകളെ മാംസംകൊണ്ട് പൊതിഞ്ഞു. പിന്നീട് നാമതിനെ തീര്‍ത്തും വ്യത്യസ്തമായ ഒരു സൃഷ്ടിയായി വളര്‍ത്തിയെടുത്തു. ഏറ്റം നല്ല സൃഷ്ടികര്‍ത്താവായ അല്ലാഹു അനുഗ്രഹപൂര്‍ണന്‍ തന്നെ.
\end{malayalam}}
\flushright{\begin{Arabic}
\quranayah[23][15]
\end{Arabic}}
\flushleft{\begin{malayalam}
പിന്നെ, ഇനി ഉറപ്പായും നിങ്ങള്‍ മരിക്കേണ്ടവരാണ്.
\end{malayalam}}
\flushright{\begin{Arabic}
\quranayah[23][16]
\end{Arabic}}
\flushleft{\begin{malayalam}
പിന്നീട് പുനരുത്ഥാനനാളില്‍ തീര്‍ച്ചയായും നിങ്ങള്‍ ഉയിര്‍ത്തെഴുന്നേല്‍പിക്കപ്പെടുകതന്നെ ചെയ്യും.
\end{malayalam}}
\flushright{\begin{Arabic}
\quranayah[23][17]
\end{Arabic}}
\flushleft{\begin{malayalam}
നിങ്ങള്‍ക്കുമീതെ നാം ഏഴു സഞ്ചാരപഥങ്ങള്‍ സൃഷ്ടിച്ചിട്ടുണ്ട്. സൃഷ്ടിയെ സംബന്ധിച്ച് നാമൊട്ടും അശ്രദ്ധനായിട്ടില്ല.
\end{malayalam}}
\flushright{\begin{Arabic}
\quranayah[23][18]
\end{Arabic}}
\flushleft{\begin{malayalam}
നാം മാനത്തുനിന്ന് നിശ്ചിത തോതില്‍ വെള്ളം വീഴ്ത്തി. അതിനെ ഭൂമിയില്‍ തങ്ങിനില്‍ക്കുന്നതാക്കി. അതുവറ്റിച്ചുകളയാനും നമുക്കു കഴിയും.
\end{malayalam}}
\flushright{\begin{Arabic}
\quranayah[23][19]
\end{Arabic}}
\flushleft{\begin{malayalam}
അങ്ങനെ ആ വെള്ളംവഴി നിങ്ങള്‍ക്ക് ഈത്തപ്പനകളുടെയും മുന്തിരിവള്ളിയുടെയും തോട്ടങ്ങള്‍ വളര്‍ത്തിത്തന്നു. നിങ്ങള്‍ക്കവയില്‍ ഒരുപാട് പഴങ്ങളുണ്ട്. നിങ്ങള്‍ അവയില്‍നിന്ന് ആഹരിച്ചുകൊണ്ടിരിക്കുന്നു.
\end{malayalam}}
\flushright{\begin{Arabic}
\quranayah[23][20]
\end{Arabic}}
\flushleft{\begin{malayalam}
സീനാമലയില്‍ മുളച്ചുവരുന്ന ഒരു മരവും നാമുണ്ടാക്കി. അത് എണ്ണയും ആഹരിക്കുന്നവര്‍ക്ക് കറിയും ഉല്‍പാദിപ്പിക്കുന്നു.
\end{malayalam}}
\flushright{\begin{Arabic}
\quranayah[23][21]
\end{Arabic}}
\flushleft{\begin{malayalam}
തീര്‍ച്ചയായും കന്നുകാലികളില്‍ നിങ്ങള്‍ക്ക് ഗുണപാഠമുണ്ട്. അവയുടെ ഉദരത്തിലുള്ളവയില്‍നിന്ന് നിങ്ങളെ നാം കുടിപ്പിക്കുന്നു. നിങ്ങള്‍ക്കവയില്‍ ധാരാളം പ്രയോജനങ്ങളുണ്ട്. നിങ്ങളവയെ ഭക്ഷിക്കുകയും ചെയ്യുന്നു.
\end{malayalam}}
\flushright{\begin{Arabic}
\quranayah[23][22]
\end{Arabic}}
\flushleft{\begin{malayalam}
അവയുടെ പുറത്ത് നിങ്ങള്‍ യാത്രചെയ്യുന്നു. കപ്പലുകളിലും.
\end{malayalam}}
\flushright{\begin{Arabic}
\quranayah[23][23]
\end{Arabic}}
\flushleft{\begin{malayalam}
നൂഹിനെ നാം തന്റെ ജനതയിലേക്ക് ദൂതനായി അയച്ചു. അദ്ദേഹം പറഞ്ഞു: "എന്റെ ജനമേ, നിങ്ങള്‍ അല്ലാഹുവിന് വഴിപ്പെടുക. അവനല്ലാതെ നിങ്ങള്‍ക്കു ദൈവമില്ല. എന്നിട്ടും നിങ്ങള്‍ ഭക്തരാവുന്നില്ലേ?”
\end{malayalam}}
\flushright{\begin{Arabic}
\quranayah[23][24]
\end{Arabic}}
\flushleft{\begin{malayalam}
അപ്പോള്‍ അദ്ദേഹത്തിന്റെ ജനതയിലെ സത്യനിഷേധികളായ പ്രമാണിമാര്‍ പറഞ്ഞു: "ഇയാള്‍ നിങ്ങളെപ്പോലുള്ള ഒരു മനുഷ്യന്‍ മാത്രമാണ്. നിങ്ങളെക്കാള്‍ വലുപ്പം നേടാന്‍ നോക്കുകയാണ് ഇവന്‍. സത്യത്തില്‍ ദൈവം ഇച്ഛിച്ചിരുന്നെങ്കില്‍ അവന്‍ മലക്കുകളെ ഇറക്കിത്തരുമായിരുന്നു. ഞങ്ങളുടെ പൂര്‍വപിതാക്കള്‍ക്കിടയിലൊന്നും ഇങ്ങനെയൊന്ന് ഞങ്ങള്‍ കേട്ടിട്ടേയില്ല.
\end{malayalam}}
\flushright{\begin{Arabic}
\quranayah[23][25]
\end{Arabic}}
\flushleft{\begin{malayalam}
"ഇയാള്‍ ഭ്രാന്തുബാധിച്ച ഒരാള്‍ മാത്രമാണ്. അതിനാല്‍ ഇയാളുടെ കാര്യത്തില്‍ നിങ്ങള്‍ ഇത്തിരികാലം കാത്തിരിക്കുക.”
\end{malayalam}}
\flushright{\begin{Arabic}
\quranayah[23][26]
\end{Arabic}}
\flushleft{\begin{malayalam}
നൂഹ് പ്രാര്‍ഥിച്ചു: "എന്റെ നാഥാ, ഈ ജനം എന്നെ തള്ളിപ്പറഞ്ഞിരിക്കുന്നു. അതിനാല്‍ നീയെനിക്കു തുണയായുണ്ടാകേണമേ.”
\end{malayalam}}
\flushright{\begin{Arabic}
\quranayah[23][27]
\end{Arabic}}
\flushleft{\begin{malayalam}
അപ്പോള്‍ നാമദ്ദേഹത്തിന് ഇങ്ങനെ ബോധനംനല്‍കി: "നമ്മുടെ മേല്‍നോട്ടത്തിലും നമ്മുടെ നിര്‍ദേശമനുസരിച്ചും നീയൊരു കപ്പലുണ്ടാക്കുക. പിന്നെ നമ്മുടെ കല്‍പനവരും. അപ്പോള്‍ അടുപ്പില്‍നിന്ന് ഉറവ പൊട്ടും. അന്നേരം എല്ലാ വസ്തുക്കളില്‍നിന്നും ഈരണ്ട് ഇണകളെയും കൂട്ടി അതില്‍ കയറുക. നിന്റെ കുടുംബത്തെയും അതില്‍ കയറ്റുക. അവരില്‍ ചിലര്‍ക്കെതിരെ നേരത്തെ വിധി വന്നുകഴിഞ്ഞിട്ടുണ്ട്. അവരെ ഒഴിവാക്കുക. അക്രമികളുടെ കാര്യം എന്നോട് പറഞ്ഞുപോകരുത്. ഉറപ്പായും അവര്‍ മുങ്ങിയൊടുങ്ങാന്‍ പോവുകയാണ്.
\end{malayalam}}
\flushright{\begin{Arabic}
\quranayah[23][28]
\end{Arabic}}
\flushleft{\begin{malayalam}
"അങ്ങനെ നീയും നിന്നോടൊപ്പമുള്ളവരും കപ്പലില്‍ കയറിക്കഴിഞ്ഞാല്‍ നീ പറയുക: “അക്രമികളില്‍ നിന്ന് ഞങ്ങളെ രക്ഷിച്ച അല്ലാഹുവിന് സ്തുതി.”
\end{malayalam}}
\flushright{\begin{Arabic}
\quranayah[23][29]
\end{Arabic}}
\flushleft{\begin{malayalam}
നീ വീണ്ടും പറയുക: “എന്റെ നാഥാ, അനുഗൃഹീതമായ ഒരിടത്ത് നീയെന്നെ ഇറക്കിത്തരേണമേ. ഇറക്കിത്തരുന്നവരില്‍ ഏറ്റവും ഉത്തമന്‍ നീയാണല്ലോ.”
\end{malayalam}}
\flushright{\begin{Arabic}
\quranayah[23][30]
\end{Arabic}}
\flushleft{\begin{malayalam}
തീര്‍ച്ചയായും ആ സംഭവത്തില്‍ ധാരാളം ദൃഷ്ടാന്തങ്ങളുണ്ട്. സംശയമില്ല; നാം പരീക്ഷണം നടത്തുന്നവന്‍ തന്നെ.
\end{malayalam}}
\flushright{\begin{Arabic}
\quranayah[23][31]
\end{Arabic}}
\flushleft{\begin{malayalam}
പിന്നീട് അവര്‍ക്കുപിറകെ നാം മറ്റൊരു തലമുറയെ വളര്‍ത്തിക്കൊണ്ടുവന്നു.
\end{malayalam}}
\flushright{\begin{Arabic}
\quranayah[23][32]
\end{Arabic}}
\flushleft{\begin{malayalam}
അങ്ങനെ അവരില്‍നിന്നു തന്നെയുള്ള ഒരു ദൂതനെ നാം അവരിലേക്കയച്ചു. അദ്ദേഹം പറഞ്ഞു: "നിങ്ങള്‍ അല്ലാഹുവിന് വഴിപ്പെടുക. അവനല്ലാതെ നിങ്ങള്‍ക്ക് ദൈവമില്ല. എന്നിട്ടും നിങ്ങള്‍ ഭക്തരാവുന്നില്ലേ?”
\end{malayalam}}
\flushright{\begin{Arabic}
\quranayah[23][33]
\end{Arabic}}
\flushleft{\begin{malayalam}
അദ്ദേഹത്തിന്റെ ജനതയിലെ സത്യനിഷേധികളും പരലോകത്തെ കണ്ടുമുട്ടുന്നതിനെ തള്ളിപ്പറഞ്ഞവരും ഐഹികജീവിതത്തില്‍ നാം സുഖാഡംബരങ്ങള്‍ ഒരുക്കിക്കൊടുത്തവരുമായ പ്രമാണിമാര്‍ പറഞ്ഞു: "ഇവന്‍ നിങ്ങളെപ്പോലുള്ള ഒരു മനുഷ്യന്‍ മാത്രമാണ്. ഇവനും നിങ്ങള്‍ തിന്നുന്നതു തിന്നുന്നു. നിങ്ങള്‍ കുടിക്കുന്നതു കുടിക്കുന്നു.
\end{malayalam}}
\flushright{\begin{Arabic}
\quranayah[23][34]
\end{Arabic}}
\flushleft{\begin{malayalam}
"നിങ്ങളെപ്പോലുള്ള ഒരു മനുഷ്യനെത്തന്നെ നിങ്ങള്‍ അനുസരിക്കുകയാണെങ്കില്‍, സംശയമില്ല; നിങ്ങള്‍ തീര്‍ത്തും നഷ്ടപ്പെട്ടവര്‍ തന്നെ.
\end{malayalam}}
\flushright{\begin{Arabic}
\quranayah[23][35]
\end{Arabic}}
\flushleft{\begin{malayalam}
"നിങ്ങള്‍ മരിക്കുകയും എല്ലും മണ്ണുമായി മാറുകയും ചെയ്താല്‍ പിന്നെയും നിങ്ങള്‍ പുറത്തുകൊണ്ടുവരപ്പെടുമെന്നാണോ ഇവന്‍ നിങ്ങളോടു വാഗ്ദാനം ചെയ്യുന്നത്?
\end{malayalam}}
\flushright{\begin{Arabic}
\quranayah[23][36]
\end{Arabic}}
\flushleft{\begin{malayalam}
"നിങ്ങള്‍ക്കു നല്‍കുന്ന ആ വാഗ്ദാനം വളരെ വളരെ വിദൂരം തന്നെ.
\end{malayalam}}
\flushright{\begin{Arabic}
\quranayah[23][37]
\end{Arabic}}
\flushleft{\begin{malayalam}
"നമ്മുടെ ഈ ഐഹികജീവിതമല്ലാതെ വേറെ ജീവിതമില്ല. നാം ജീവിക്കുന്നു; മരിക്കുന്നു. നാമൊരിക്കലും ഉയിര്‍ത്തെഴുന്നേല്‍പിക്കപ്പെടുന്നവരല്ല.
\end{malayalam}}
\flushright{\begin{Arabic}
\quranayah[23][38]
\end{Arabic}}
\flushleft{\begin{malayalam}
"ദൈവത്തിന്റെ പേരില്‍ കള്ളം കെട്ടിച്ചമച്ച ഒരുത്തന്‍ മാത്രമാണിവന്‍. ഞങ്ങളൊരിക്കലും ഇവനില്‍ വിശ്വസിക്കുന്നവരല്ല.”
\end{malayalam}}
\flushright{\begin{Arabic}
\quranayah[23][39]
\end{Arabic}}
\flushleft{\begin{malayalam}
അദ്ദേഹം പറഞ്ഞു: "എന്റെ നാഥാ, ഇവരെന്നെ തള്ളിപ്പറഞ്ഞിരിക്കുന്നു. അതിനാല്‍ നീയെന്നെ സഹായിക്കേണമേ.”
\end{malayalam}}
\flushright{\begin{Arabic}
\quranayah[23][40]
\end{Arabic}}
\flushleft{\begin{malayalam}
അല്ലാഹു അറിയിച്ചു: "അടുത്തുതന്നെ അവര്‍ കൊടുംഖേദത്തിനിരയാകും.”
\end{malayalam}}
\flushright{\begin{Arabic}
\quranayah[23][41]
\end{Arabic}}
\flushleft{\begin{malayalam}
അവസാനം തീര്‍ത്തും ന്യായമായ നിലയില്‍ ഒരു ഘോരഗര്‍ജനം അവരെ പിടികൂടി. അങ്ങനെ നാമവരെ ചവറുകളാക്കി. അക്രമികളായ ജനത്തിനു നാശം!
\end{malayalam}}
\flushright{\begin{Arabic}
\quranayah[23][42]
\end{Arabic}}
\flushleft{\begin{malayalam}
പിന്നെ അവര്‍ക്കുശേഷം നാം മറ്റു തലമുറകളെ വളര്‍ത്തിക്കൊണ്ടുവന്നു.
\end{malayalam}}
\flushright{\begin{Arabic}
\quranayah[23][43]
\end{Arabic}}
\flushleft{\begin{malayalam}
ഒരു സമുദായവും അതിന്റെ നിശ്ചിത അവധിക്കുമുമ്പ് നശിക്കുകയോ അവധിക്കുശേഷം നിലനില്‍ക്കുകയോ ഇല്ല.
\end{malayalam}}
\flushright{\begin{Arabic}
\quranayah[23][44]
\end{Arabic}}
\flushleft{\begin{malayalam}
പിന്നീട് നാം തുടര്‍ച്ചയായി നമ്മുടെ ദൂതന്മാരെ അയച്ചുകൊണ്ടിരുന്നു. ഓരോ സമുദായത്തിലും അതിന്റെ ദൂതന്‍ ആഗതമായപ്പോഴെല്ലാം അവരദ്ദേഹത്തെ തള്ളിപ്പറഞ്ഞു. അപ്പോഴെല്ലാം നാമവരെ ഒന്നിനുപിറകെ മറ്റൊന്നായി നശിപ്പിച്ചുകൊണ്ടിരുന്നു. അങ്ങനെ അവരെ നാം കഥാവശേഷരാക്കി. വിശ്വസിക്കാത്ത ജനതക്ക് സര്‍വനാശം!
\end{malayalam}}
\flushright{\begin{Arabic}
\quranayah[23][45]
\end{Arabic}}
\flushleft{\begin{malayalam}
പിന്നീട് മൂസായെയും അദ്ദേഹത്തിന്റെ സഹോദരന്‍ ഹാറൂനെയും നാം നമ്മുടെ തെളിവുകളോടെയും വ്യക്തമായ പ്രമാണങ്ങളോടെയും അയച്ചു.
\end{malayalam}}
\flushright{\begin{Arabic}
\quranayah[23][46]
\end{Arabic}}
\flushleft{\begin{malayalam}
ഫറവോന്റെയും അവന്റെ പ്രമാണിപ്പരിഷകളുടെയും അടുത്തേക്ക്. അപ്പോഴവര്‍ അഹങ്കരിച്ചു. ഔദ്ധത്യം നടിക്കുന്ന ജനതയായിരുന്നു അവര്‍.
\end{malayalam}}
\flushright{\begin{Arabic}
\quranayah[23][47]
\end{Arabic}}
\flushleft{\begin{malayalam}
അതിനാലവര്‍ പറഞ്ഞു: "ഞങ്ങള്‍ ഞങ്ങളെപ്പോലെത്തന്നെയുള്ള രണ്ടു മനുഷ്യരില്‍ വിശ്വസിക്കുകയോ? അവരുടെ ആളുകളാണെങ്കില്‍ നമുക്ക് അടിമപ്പണി ചെയ്യുന്നവരും!”
\end{malayalam}}
\flushright{\begin{Arabic}
\quranayah[23][48]
\end{Arabic}}
\flushleft{\begin{malayalam}
അങ്ങനെ അവര്‍ ആ രണ്ടുപേരെയും തള്ളിപ്പറഞ്ഞു. അതിനാലവര്‍ നാശത്തിനിരയായി.
\end{malayalam}}
\flushright{\begin{Arabic}
\quranayah[23][49]
\end{Arabic}}
\flushleft{\begin{malayalam}
മൂസാക്കു നാം വേദം നല്‍കി. അതിലൂടെ അവര്‍ നേര്‍വഴി പ്രാപിക്കാന്‍.
\end{malayalam}}
\flushright{\begin{Arabic}
\quranayah[23][50]
\end{Arabic}}
\flushleft{\begin{malayalam}
മര്‍യമിന്റെ പുത്രനെയും അവന്റെ മാതാവിനെയും നാമൊരു ദൃഷ്ടാന്തമാക്കി. അവരിരുവര്‍ക്കും നാം സൌകര്യപ്രദവും ഉറവകളുള്ളതുമായ ഒരുയര്‍ന്ന പ്രദേശത്ത് അഭയം നല്‍കി.
\end{malayalam}}
\flushright{\begin{Arabic}
\quranayah[23][51]
\end{Arabic}}
\flushleft{\begin{malayalam}
അല്ലാഹുവിന്റെ ദൂതന്മാരേ, നല്ല ആഹാരപദാര്‍ഥങ്ങള്‍ ഭക്ഷിക്കുക. സല്‍ക്കര്‍മങ്ങള്‍ ചെയ്യുക. തീര്‍ച്ചയായും നിങ്ങള്‍ പ്രവര്‍ത്തിക്കുന്നതിനെപ്പറ്റിയെല്ലാം നന്നായറിയുന്നവനാണ് നാം.
\end{malayalam}}
\flushright{\begin{Arabic}
\quranayah[23][52]
\end{Arabic}}
\flushleft{\begin{malayalam}
നിശ്ചയമായും ഇതാണ് നിങ്ങളുടെ സമുദായം; ഏകസമുദായം. ഞാനാണ് നിങ്ങളുടെ നാഥന്‍. അതിനാല്‍ എന്നോട് ഭക്തിയുള്ളവരാവുക.
\end{malayalam}}
\flushright{\begin{Arabic}
\quranayah[23][53]
\end{Arabic}}
\flushleft{\begin{malayalam}
പക്ഷേ, പിന്നീടവര്‍ കക്ഷികളായിപ്പിരിഞ്ഞ് തങ്ങളുടെ മതത്തെ തുണ്ടംതുണ്ടമാക്കി. ഓരോ കക്ഷിയും തങ്ങളുടെ വശമുള്ളതില്‍ തൃപ്തിയടയുന്നവരാണ്.
\end{malayalam}}
\flushright{\begin{Arabic}
\quranayah[23][54]
\end{Arabic}}
\flushleft{\begin{malayalam}
അതിനാല്‍ ഒരു നിശ്ചിതകാലംവരെ അവരെ തങ്ങളുടെ “ബോധംകെട്ട” അവസ്ഥയില്‍ തുടരാന്‍ വിട്ടേക്കുക.
\end{malayalam}}
\flushright{\begin{Arabic}
\quranayah[23][55]
\end{Arabic}}
\flushleft{\begin{malayalam}
അവര്‍ വിചാരിക്കുന്നോ, സമ്പത്തും സന്താനങ്ങളും നല്‍കി നാമവരെ സഹായിച്ചുകൊണ്ടിരിക്കുന്നത്-
\end{malayalam}}
\flushright{\begin{Arabic}
\quranayah[23][56]
\end{Arabic}}
\flushleft{\begin{malayalam}
നാമവര്‍ക്ക് നന്മവരുത്താന്‍ തിടുക്കം കൂട്ടുന്നതിനാലാണെന്ന്? അല്ല; അവര്‍ സത്യാവസ്ഥ തിരിച്ചറിയുന്നില്ല.
\end{malayalam}}
\flushright{\begin{Arabic}
\quranayah[23][57]
\end{Arabic}}
\flushleft{\begin{malayalam}
തീര്‍ച്ചയായും തങ്ങളുടെ നാഥനെ ഭയന്നു നടുങ്ങുന്നവര്‍;
\end{malayalam}}
\flushright{\begin{Arabic}
\quranayah[23][58]
\end{Arabic}}
\flushleft{\begin{malayalam}
തങ്ങളുടെ നാഥന്റെ വചനങ്ങളില്‍ വിശ്വസിക്കുന്നവര്‍;
\end{malayalam}}
\flushright{\begin{Arabic}
\quranayah[23][59]
\end{Arabic}}
\flushleft{\begin{malayalam}
തങ്ങളുടെ നാഥന്ന് പങ്കാളികളെ കല്‍പിക്കാത്തവര്‍;
\end{malayalam}}
\flushright{\begin{Arabic}
\quranayah[23][60]
\end{Arabic}}
\flushleft{\begin{malayalam}
തങ്ങളുടെ നാഥങ്കലേക്ക് തിരിച്ചുചെല്ലേണ്ടവരാണല്ലോ എന്ന വിചാരത്താല്‍ ദാനംചെയ്യുമ്പോള്‍ ഹൃദയം വിറപൂണ്ട് ദാനം നല്‍കുന്നവര്‍;
\end{malayalam}}
\flushright{\begin{Arabic}
\quranayah[23][61]
\end{Arabic}}
\flushleft{\begin{malayalam}
ഇവരൊക്കെയാണ് നന്മ ചെയ്യാന്‍ തിടുക്കം കൂട്ടുന്നവര്‍. അവയില്‍ ആദ്യം ചെന്നെത്തുന്നവരും അവര്‍ തന്നെ.
\end{malayalam}}
\flushright{\begin{Arabic}
\quranayah[23][62]
\end{Arabic}}
\flushleft{\begin{malayalam}
ആരെയും അവരുടെ കഴിവിനതീതമായതിന് നാം നിര്‍ബന്ധിക്കുന്നില്ല. സത്യം കൃത്യമായി പ്രതിപാദിക്കുന്ന ഒരു രേഖ നമ്മുടെ വശമുണ്ട്. ആരും ഒരിക്കലും ഒട്ടും അനീതിക്കിരയാവില്ല.
\end{malayalam}}
\flushright{\begin{Arabic}
\quranayah[23][63]
\end{Arabic}}
\flushleft{\begin{malayalam}
എന്നാല്‍, അവരുടെ ഹൃദയങ്ങള്‍ ഇക്കാര്യത്തെപ്പറ്റി തീരെ അശ്രദ്ധമാണ്. അവര്‍ക്ക് അതല്ലാത്ത മറ്റുചില പണികളാണുള്ളത്. അവരതു ചെയ്തുകൊണ്ടേയിരിക്കുന്നു.
\end{malayalam}}
\flushright{\begin{Arabic}
\quranayah[23][64]
\end{Arabic}}
\flushleft{\begin{malayalam}
അങ്ങനെ, അവരിലെ സുഖലോലുപരെ ശിക്ഷയാല്‍ നാം പിടികൂടും. അപ്പോഴവര്‍ വിലപിക്കാന്‍ തുടങ്ങും.
\end{malayalam}}
\flushright{\begin{Arabic}
\quranayah[23][65]
\end{Arabic}}
\flushleft{\begin{malayalam}
നിങ്ങളിന്നു വിലപിക്കേണ്ടതില്ല. നിങ്ങള്‍ക്കിന്ന് നമ്മുടെ ഭാഗത്തുനിന്ന് ഒരു സഹായവും ലഭിക്കുകയില്ല.
\end{malayalam}}
\flushright{\begin{Arabic}
\quranayah[23][66]
\end{Arabic}}
\flushleft{\begin{malayalam}
നമ്മുടെ വചനങ്ങള്‍ നിങ്ങളെ വ്യക്തമായി ഓതിക്കേള്‍പ്പിച്ചിരുന്നല്ലോ. അപ്പോള്‍ നിങ്ങള്‍ പിന്തിരിഞ്ഞുപോവുകയായിരുന്നു;
\end{malayalam}}
\flushright{\begin{Arabic}
\quranayah[23][67]
\end{Arabic}}
\flushleft{\begin{malayalam}
പൊങ്ങച്ചം നടിക്കുന്നവരായി. രാക്കഥാ കഥനങ്ങളില്‍ നിങ്ങള്‍ അതേപ്പറ്റി അസംബന്ധം പുലമ്പുകയായിരുന്നു.
\end{malayalam}}
\flushright{\begin{Arabic}
\quranayah[23][68]
\end{Arabic}}
\flushleft{\begin{malayalam}
അവര്‍ ഈ വചനത്തെപ്പറ്റി തെല്ലും ചിന്തിച്ചുനോക്കിയിട്ടില്ലേ? അതല്ല; അവരുടെ പൂര്‍വ പിതാക്കള്‍ക്ക് വന്നെത്തിയിട്ടില്ലാത്ത ഒന്നാണോ ഇവര്‍ക്ക് വന്നുകിട്ടിയിരിക്കുന്നത്?
\end{malayalam}}
\flushright{\begin{Arabic}
\quranayah[23][69]
\end{Arabic}}
\flushleft{\begin{malayalam}
അതല്ല; തങ്ങളുടെ ദൂതനെ പരിചയമില്ലാത്തതിനാലാണോ അവരദ്ദേഹത്തെ തള്ളിപ്പറയുന്നത്?
\end{malayalam}}
\flushright{\begin{Arabic}
\quranayah[23][70]
\end{Arabic}}
\flushleft{\begin{malayalam}
അതുമല്ലെങ്കില്‍ അദ്ദേഹത്തിന് ഭ്രാന്തുണ്ടെന്നാണോ അവര്‍ പറയുന്നത്? എന്നാല്‍ അറിയുക. സത്യസന്ദേശവുമായാണ് അദ്ദേഹം അവരുടെയടുത്ത് വന്നെത്തിയത്. എന്നാല്‍ അവരിലേറെപ്പേരും സത്യത്തെ വെറുക്കുന്നവരാണ്.
\end{malayalam}}
\flushright{\begin{Arabic}
\quranayah[23][71]
\end{Arabic}}
\flushleft{\begin{malayalam}
സത്യം അവരുടെ തന്നിഷ്ടങ്ങളെ പിന്‍പറ്റിയിരുന്നുവെങ്കില്‍ ആകാശഭൂമികളും അവയിലെല്ലാമുള്ളവരും കുഴപ്പത്തിലാകുമായിരുന്നു. എന്നാല്‍, നാം അവര്‍ക്കുള്ള ഉദ്ബോധനവുമായാണ് അവരെ സമീപിച്ചത്. എന്നിട്ടും അവര്‍ തങ്ങള്‍ക്കുള്ള ഉദ്ബോധനം അവഗണിക്കുകയാണുണ്ടായത്.
\end{malayalam}}
\flushright{\begin{Arabic}
\quranayah[23][72]
\end{Arabic}}
\flushleft{\begin{malayalam}
അല്ല; നീ അവരോടു വല്ല പ്രതിഫലവും ആവശ്യപ്പെടുന്നുണ്ടോ? എന്നാല്‍ ഓര്‍ക്കുക: നിന്റെ നാഥന്റെ പ്രതിഫലമാണ് മഹത്തരം. അവന്‍ അന്നദാതാക്കളില്‍ അത്യുത്തമന്‍ തന്നെ.
\end{malayalam}}
\flushright{\begin{Arabic}
\quranayah[23][73]
\end{Arabic}}
\flushleft{\begin{malayalam}
തീര്‍ച്ചയായും നീയവരെ നേര്‍വഴിയിലേക്കാണ് വിളിച്ചുകൊണ്ടിരിക്കുന്നത്.
\end{malayalam}}
\flushright{\begin{Arabic}
\quranayah[23][74]
\end{Arabic}}
\flushleft{\begin{malayalam}
എന്നാല്‍, പരലോക വിശ്വാസമില്ലാത്തവര്‍ ആ നേര്‍വഴിയില്‍ നിന്ന് തെറ്റിപ്പോകുന്നവരാണ്.
\end{malayalam}}
\flushright{\begin{Arabic}
\quranayah[23][75]
\end{Arabic}}
\flushleft{\begin{malayalam}
നാം അവരോട് കരുണകാണിക്കുകയും അവരെ ബാധിച്ച വിപത്ത് ഒഴിവാക്കിക്കൊടുക്കുകയുമാണെങ്കില്‍ അവര്‍ തങ്ങളുടെ ധിക്കാരത്തില്‍ വാശിയോടെ വിഹരിക്കുമായിരുന്നു.
\end{malayalam}}
\flushright{\begin{Arabic}
\quranayah[23][76]
\end{Arabic}}
\flushleft{\begin{malayalam}
നാം അവരെ ശിക്ഷയാല്‍ പിടികൂടി. എന്നിട്ടും അവര്‍ തങ്ങളുടെ നാഥന്ന് കീഴൊതുങ്ങുന്നവരായില്ല. അവര്‍ താഴ്മ കാണിച്ചതുമില്ല.
\end{malayalam}}
\flushright{\begin{Arabic}
\quranayah[23][77]
\end{Arabic}}
\flushleft{\begin{malayalam}
അതിനാല്‍ നാം അവരുടെ നേരെ കൊടുംശിക്ഷയുടെ കവാടം തുറന്നു. അതോടെയവര്‍ അങ്ങേയറ്റം നിരാശരായി.
\end{malayalam}}
\flushright{\begin{Arabic}
\quranayah[23][78]
\end{Arabic}}
\flushleft{\begin{malayalam}
അവനാണ് നിങ്ങള്‍ക്ക് കേള്‍വിയും കാഴ്ചയും ഹൃദയങ്ങളും ഉണ്ടാക്കിത്തന്നത്. പക്ഷേ, നന്നെക്കുറച്ചു മാത്രമേ നിങ്ങള്‍ നന്ദി കാണിക്കുന്നുള്ളൂ.
\end{malayalam}}
\flushright{\begin{Arabic}
\quranayah[23][79]
\end{Arabic}}
\flushleft{\begin{malayalam}
അവനാണ് ഭൂമിയില്‍ നിങ്ങളെ വ്യാപിപ്പിച്ചവന്‍. നിങ്ങളെല്ലാം ഒരുമിച്ചുകൂട്ടപ്പെടുന്നതും അവനിലേക്കുതന്നെ.
\end{malayalam}}
\flushright{\begin{Arabic}
\quranayah[23][80]
\end{Arabic}}
\flushleft{\begin{malayalam}
അവനാണ് ജീവിപ്പിക്കുന്നവനും മരിപ്പിക്കുന്നവനും. രാപ്പകലുകള്‍ മാറിമാറി വരുന്നതും അവന്റെ നിയമമനുസരിച്ചാണ്. നിങ്ങള്‍ ചിന്തിക്കുന്നില്ലേ?
\end{malayalam}}
\flushright{\begin{Arabic}
\quranayah[23][81]
\end{Arabic}}
\flushleft{\begin{malayalam}
എന്നാല്‍ ഇക്കൂട്ടര്‍ അവരുടെ പൂര്‍വികര്‍ പറഞ്ഞിരുന്നതുപോലെത്തന്നെ പറയുകയാണ്.
\end{malayalam}}
\flushright{\begin{Arabic}
\quranayah[23][82]
\end{Arabic}}
\flushleft{\begin{malayalam}
അവര്‍ പറഞ്ഞു: "ഞങ്ങള്‍ മരിച്ച് മണ്ണും എല്ലുമായി മാറിയാല്‍ വീണ്ടും ഉയിര്‍ത്തെഴുന്നേല്‍പിക്കപ്പെടുമെന്നോ!
\end{malayalam}}
\flushright{\begin{Arabic}
\quranayah[23][83]
\end{Arabic}}
\flushleft{\begin{malayalam}
"ഞങ്ങള്‍ക്ക് ഈ വാഗ്ദാനം നല്‍കപ്പെട്ടിരുന്നു. ഇതിനുമുമ്പ് ഞങ്ങളുടെ പിതാക്കള്‍ക്കും ഇവ്വിധം വാഗ്ദാനം നല്‍കിയിരുന്നു. എന്നാലിത് പൂര്‍വികരുടെ കെട്ടുകഥകളല്ലാതൊന്നുമല്ല.”
\end{malayalam}}
\flushright{\begin{Arabic}
\quranayah[23][84]
\end{Arabic}}
\flushleft{\begin{malayalam}
ചോദിക്കുക: "ഭൂമിയും അതിലുള്ളതും ആരുടേതാണ്? നിങ്ങള്‍ക്ക് അറിയുമെങ്കില്‍ പറയൂ.”
\end{malayalam}}
\flushright{\begin{Arabic}
\quranayah[23][85]
\end{Arabic}}
\flushleft{\begin{malayalam}
അവര്‍ പറയും: "അല്ലാഹുവിന്റേതാണ്.” ചോദിക്കുക: "നിങ്ങള്‍ ആലോചിച്ചു നോക്കുന്നില്ലേ?”
\end{malayalam}}
\flushright{\begin{Arabic}
\quranayah[23][86]
\end{Arabic}}
\flushleft{\begin{malayalam}
ചോദിക്കുക: "ആരാണ് ഏഴാകാശങ്ങളുടെയും അതിമഹത്തായ സിംഹാസനത്തിന്റെയും അധിപന്‍.”
\end{malayalam}}
\flushright{\begin{Arabic}
\quranayah[23][87]
\end{Arabic}}
\flushleft{\begin{malayalam}
അവര്‍ പറയും:"അല്ലാഹു.” ചോദിക്കുക: "എന്നിട്ടും നിങ്ങള്‍ സൂക്ഷ്മത പാലിക്കുന്നില്ലേ?”
\end{malayalam}}
\flushright{\begin{Arabic}
\quranayah[23][88]
\end{Arabic}}
\flushleft{\begin{malayalam}
ചോദിക്കുക: "ആരുടെ വശമാണ് സകല വസ്തുക്കളുടെയും ആധിപത്യം? അഭയമേകുന്നവനും തനിക്കെതിരെ ഒരിടത്തുനിന്നും അഭയം ലഭിക്കാത്തവനും ആരാണ്? പറയൂ; നിങ്ങള്‍ക്ക് അറിയുമെങ്കില്‍!”
\end{malayalam}}
\flushright{\begin{Arabic}
\quranayah[23][89]
\end{Arabic}}
\flushleft{\begin{malayalam}
അവര്‍ പറയും: "എല്ലാം അല്ലാഹുവാണ്.” ചോദിക്കുക: "എന്നിട്ടും നിങ്ങള്‍ എങ്ങനെ മായാവലയത്തില്‍ പെട്ടുപോകുന്നു?”
\end{malayalam}}
\flushright{\begin{Arabic}
\quranayah[23][90]
\end{Arabic}}
\flushleft{\begin{malayalam}
അറിയുക; നാം അവരുടെ അടുത്തേക്ക് അയച്ചത് സത്യസന്ദേശമാണ്. അവരോ; കള്ളംപറയുന്നവരും.
\end{malayalam}}
\flushright{\begin{Arabic}
\quranayah[23][91]
\end{Arabic}}
\flushleft{\begin{malayalam}
അല്ലാഹു ആരെയും പുത്രനാക്കി വെച്ചിട്ടില്ല. അവനോടൊപ്പം വേറെ ദൈവമില്ല. ഉണ്ടായിരുന്നെങ്കില്‍ ഓരോ ദൈവവും താന്‍ സൃഷ്ടിച്ചതുമായി പോയിക്കളയുമായിരുന്നു. അവര്‍ പരസ്പരം കീഴ്പെടുത്തുമായിരുന്നു. അവര്‍ പറഞ്ഞുപരത്തുന്നതില്‍നിന്നെല്ലാം എത്രയോ പരിശുദ്ധനാണ് അല്ലാഹു.
\end{malayalam}}
\flushright{\begin{Arabic}
\quranayah[23][92]
\end{Arabic}}
\flushleft{\begin{malayalam}
കണ്ണുകൊണ്ട് കാണാനാവുന്നതും കാണാനാവാത്തതും അറിയുന്നവനാണ് അവന്‍. അവര്‍ പങ്കുചേര്‍ക്കുന്നവയില്‍ നിന്നെല്ലാം അതീതനും.
\end{malayalam}}
\flushright{\begin{Arabic}
\quranayah[23][93]
\end{Arabic}}
\flushleft{\begin{malayalam}
പറയുക: "നാഥാ, ഇവരെ താക്കീതു ചെയ്തുകൊണ്ടിരിക്കുന്ന ശിക്ഷ കാണേണ്ട അവസ്ഥ എനിക്കുണ്ടാവുകയാണെങ്കില്‍,
\end{malayalam}}
\flushright{\begin{Arabic}
\quranayah[23][94]
\end{Arabic}}
\flushleft{\begin{malayalam}
"എന്റെ നാഥാ, നീ എന്നെ അക്രമികളായ ജനത്തില്‍ പെടുത്തരുതേ.”
\end{malayalam}}
\flushright{\begin{Arabic}
\quranayah[23][95]
\end{Arabic}}
\flushleft{\begin{malayalam}
അവരെ താക്കീതു ചെയ്തുകൊണ്ടിരിക്കുന്ന ശിക്ഷ നിനക്കു കാണിച്ചുതരാന്‍ തീര്‍ച്ചയായും കഴിവുറ്റവന്‍ തന്നെ നാം.
\end{malayalam}}
\flushright{\begin{Arabic}
\quranayah[23][96]
\end{Arabic}}
\flushleft{\begin{malayalam}
ഏറ്റവും നല്ലതുകൊണ്ട് നീ തിന്മയെ തടയുക. അവര്‍ പറഞ്ഞുപരത്തുന്നതിനെപ്പറ്റി നന്നായറിയുന്നവനാണ് നാം.
\end{malayalam}}
\flushright{\begin{Arabic}
\quranayah[23][97]
\end{Arabic}}
\flushleft{\begin{malayalam}
പറയുക: "എന്റെ നാഥാ, പിശാചിന്റെ പ്രലോഭനങ്ങളില്‍നിന്ന് ഞാനിതാ നിന്നിലഭയം തേടുന്നു.
\end{malayalam}}
\flushright{\begin{Arabic}
\quranayah[23][98]
\end{Arabic}}
\flushleft{\begin{malayalam}
"എന്റെ നാഥാ, പിശാചുക്കള്‍ എന്റെയടുത്ത് വരുന്നതില്‍ നിന്നും ഞാനിതാ നിന്നോട് രക്ഷതേടുന്നു.”
\end{malayalam}}
\flushright{\begin{Arabic}
\quranayah[23][99]
\end{Arabic}}
\flushleft{\begin{malayalam}
അങ്ങനെ അവരിലൊരുവന്ന് മരണം വന്നെത്തുമ്പോള്‍ അവന്‍ കേണുപറയും: "എന്റെ നാഥാ, നീ എന്നെയൊന്ന് ഭൂമിയിലേക്ക് തിരിച്ചയക്കേണമേ.
\end{malayalam}}
\flushright{\begin{Arabic}
\quranayah[23][100]
\end{Arabic}}
\flushleft{\begin{malayalam}
"ഞാന്‍ ഉപേക്ഷ വരുത്തിയ കാര്യത്തില്‍ ഞാന്‍ നല്ല നിലയില്‍ പ്രവര്‍ത്തിക്കുന്നവനായേക്കാം.” ഒരിക്കലുമില്ല. അതൊരു വെറും വാക്കാണ്. അവനതങ്ങനെ പറഞ്ഞുകൊണ്ടേയിരിക്കും. അവരുടെ പിന്നില്‍ ഒരു മറയുണ്ടായിരിക്കും. അവരെ ഉയിര്‍ത്തെഴുന്നേല്‍പിക്കുംവരെ.
\end{malayalam}}
\flushright{\begin{Arabic}
\quranayah[23][101]
\end{Arabic}}
\flushleft{\begin{malayalam}
പിന്നെ കാഹളം ഊതപ്പെടും. അന്നാളില്‍ അവര്‍ക്കിടയില്‍ ഒ രുവിധ ബന്ധവുമുണ്ടായിരിക്കുകയില്ല. അവരന്യോന്യം അന്വേഷിക്കുകയുമില്ല.
\end{malayalam}}
\flushright{\begin{Arabic}
\quranayah[23][102]
\end{Arabic}}
\flushleft{\begin{malayalam}
അന്ന് ആരുടെ തുലാസിന്‍തട്ട് ഭാരം തൂങ്ങുന്നുവോ അവരാണ് വിജയംവരിച്ചവര്‍.
\end{malayalam}}
\flushright{\begin{Arabic}
\quranayah[23][103]
\end{Arabic}}
\flushleft{\begin{malayalam}
ആരുടെ തുലാസിന്‍തട്ട് ഭാരം കുറയുന്നുവോ അവര്‍ സ്വയം നഷ്ടം വരുത്തിവെച്ചവരാണ്. അവര്‍ നരകത്തീയില്‍ സ്ഥിരവാസികളായിരിക്കും.
\end{malayalam}}
\flushright{\begin{Arabic}
\quranayah[23][104]
\end{Arabic}}
\flushleft{\begin{malayalam}
നരകത്തീ അവരുടെ മുഖം കരിച്ചുകളയും. അവരതില്‍ മോണകാട്ടിയിളിക്കുന്നവരായിരിക്കും.
\end{malayalam}}
\flushright{\begin{Arabic}
\quranayah[23][105]
\end{Arabic}}
\flushleft{\begin{malayalam}
അന്ന് അവരോടു പറയും: "എന്റെ വചനങ്ങള്‍ നിങ്ങളെ ഓതിക്കേള്‍പ്പിച്ചിരുന്നില്ലേ? അപ്പോള്‍ നിങ്ങളവയെ തള്ളിപ്പറയുകയായിരുന്നില്ലേ.”
\end{malayalam}}
\flushright{\begin{Arabic}
\quranayah[23][106]
\end{Arabic}}
\flushleft{\begin{malayalam}
അവര്‍ പറയും: "ഞങ്ങളുടെ നാഥാ! ഞങ്ങളുടെ ഭാഗ്യദോഷം ഞങ്ങളെ കീഴ്പെടുത്തി. ഞങ്ങള്‍പിഴച്ച ജനതയായിപ്പോയി.
\end{malayalam}}
\flushright{\begin{Arabic}
\quranayah[23][107]
\end{Arabic}}
\flushleft{\begin{malayalam}
"ഞങ്ങളുടെ നാഥാ! ഞങ്ങളെ നീ ഇവിടെനിന്ന് പുറത്തേക്കെടുക്കേണമേ! ഇനിയും ഞങ്ങള്‍ വഴികേടിലേക്ക് തിരിച്ചുപോവുകയാണെങ്കില്‍ തീര്‍ച്ചയായും ഞങ്ങള്‍ അതിക്രമികള്‍ തന്നെയായിരിക്കും.”
\end{malayalam}}
\flushright{\begin{Arabic}
\quranayah[23][108]
\end{Arabic}}
\flushleft{\begin{malayalam}
അല്ലാഹു പറയും: "നിങ്ങളവിടെത്തന്നെ അപമാനിതരായി കഴിയുക. എന്നോടു മിണ്ടരുത്.”
\end{malayalam}}
\flushright{\begin{Arabic}
\quranayah[23][109]
\end{Arabic}}
\flushleft{\begin{malayalam}
എന്റെ ദാസന്മാരിലൊരു വിഭാഗം ഇവ്വിധം പറയാറുണ്ടായിരുന്നു: "ഞങ്ങളുടെ നാഥാ; ഞങ്ങള്‍ വിശ്വസിച്ചിരിക്കുന്നു. അതിനാല്‍ ഞങ്ങള്‍ക്കു നീ പൊറുത്തുതരേണമേ. ഞങ്ങളോടു കരുണ കാണിക്കേണമേ. നീ കരുണ കാണിക്കുന്നവരില്‍ അത്യുത്തമനാണല്ലോ.”
\end{malayalam}}
\flushright{\begin{Arabic}
\quranayah[23][110]
\end{Arabic}}
\flushleft{\begin{malayalam}
നിങ്ങളവരെ പരിഹസിച്ചുകൊണ്ടിരിക്കുകയായിരുന്നു. അതിനിടയില്‍ നിങ്ങള്‍ക്ക് എന്നെ ഓര്‍ക്കാന്‍പോലും കഴിയാതെപോയി. നിങ്ങള്‍ അവരെ പുച്ഛിച്ചു ചിരിക്കുകയായിരുന്നു.
\end{malayalam}}
\flushright{\begin{Arabic}
\quranayah[23][111]
\end{Arabic}}
\flushleft{\begin{malayalam}
അവര്‍ നന്നായി ക്ഷമിച്ചു. അതിനാല്‍ നാമിതാ ഇന്ന് അവര്‍ക്ക് പ്രതിഫലം നല്‍കിയിരിക്കുന്നു. തീര്‍ച്ചയായും അവര്‍ തന്നെയാണ് വിജയംവരിച്ചവര്‍.
\end{malayalam}}
\flushright{\begin{Arabic}
\quranayah[23][112]
\end{Arabic}}
\flushleft{\begin{malayalam}
അല്ലാഹു ചോദിക്കും: "നിങ്ങള്‍ ഭൂമിയില്‍ എത്രകൊല്ലം താമസിച്ചു?”
\end{malayalam}}
\flushright{\begin{Arabic}
\quranayah[23][113]
\end{Arabic}}
\flushleft{\begin{malayalam}
അവര്‍ പറയും: "ഞങ്ങള്‍ ഒരു ദിവസം താമസിച്ചുകാണും. അല്ലെങ്കില്‍ ഒരു ദിവസത്തിന്റെ അല്‍പഭാഗം. എണ്ണിക്കണക്കാക്കുന്നവരോട് നീയൊന്ന് ചോദിച്ചുനോക്കൂ.”
\end{malayalam}}
\flushright{\begin{Arabic}
\quranayah[23][114]
\end{Arabic}}
\flushleft{\begin{malayalam}
അല്ലാഹു പറയും: "സത്യത്തില്‍ നിങ്ങള്‍ അല്‍പകാലം മാത്രമേ താമസിച്ചിട്ടുള്ളൂ. ഇക്കാര്യം നിങ്ങള്‍ അന്ന് മനസ്സിലാക്കിയിരുന്നെങ്കില്‍.
\end{malayalam}}
\flushright{\begin{Arabic}
\quranayah[23][115]
\end{Arabic}}
\flushleft{\begin{malayalam}
"നിങ്ങളെ നാം വെറുതെ സൃഷ്ടിച്ചതാണെന്നും നിങ്ങള്‍ നമ്മുടെയടുത്തേക്ക് മടക്കപ്പെടുകയില്ലെന്നുമാണോ നിങ്ങള്‍ കരുതിയിരുന്നത്?”
\end{malayalam}}
\flushright{\begin{Arabic}
\quranayah[23][116]
\end{Arabic}}
\flushleft{\begin{malayalam}
എന്നാല്‍ അല്ലാഹു അത്യുന്നതനാണ്. അവനാണ് യഥാര്‍ഥ രാജാവ്. അവനല്ലാതെ ദൈവമില്ല. മഹത്തായ സിംഹാസനത്തിന്നുടമയാണവന്‍.
\end{malayalam}}
\flushright{\begin{Arabic}
\quranayah[23][117]
\end{Arabic}}
\flushleft{\begin{malayalam}
ഒരുവിധ തെളിവോ ന്യായമോ ഇല്ലാതെ ആരെങ്കിലും അല്ലാഹുവോടൊപ്പം വേറെ ഏതെങ്കിലും ദൈവത്തെ വിളിച്ചുപ്രാര്‍ഥിക്കുന്നുവെങ്കില്‍ അവന്റെ വിചാരണ തന്റെ നാഥന്റെ അടുത്തുവെച്ചുതന്നെയായിരിക്കും. തീര്‍ച്ചയായും സത്യനിഷേധികള്‍ വിജയം വരിക്കുകയില്ല.
\end{malayalam}}
\flushright{\begin{Arabic}
\quranayah[23][118]
\end{Arabic}}
\flushleft{\begin{malayalam}
പറയുക: എന്റെ നാഥാ, എനിക്കു നീ പൊറുത്തുതരേണമേ. എന്നോട് കരുണകാണിക്കേണമേ! നീ കരുണകാണിക്കുന്നവരില്‍ അത്യുത്തമനാണല്ലോ.
\end{malayalam}}
\chapter{\textmalayalam{നൂര്‍ ( പ്രകാശം )}}
\begin{Arabic}
\Huge{\centerline{\basmalah}}\end{Arabic}
\flushright{\begin{Arabic}
\quranayah[24][1]
\end{Arabic}}
\flushleft{\begin{malayalam}
ഇതൊരധ്യായമാണ്. നാം ഇതിറക്കിത്തന്നിരിക്കുന്നു. ഇതിനെ നിയമമാക്കി നിശ്ചയിച്ചിരിക്കുന്നു. നാം ഇതില്‍ വ്യക്തമായ തെളിവുകള്‍ അവതരിപ്പിച്ചിട്ടുണ്ട്. നിങ്ങള്‍ ചിന്തിച്ചുമനസ്സിലാക്കാന്‍.
\end{malayalam}}
\flushright{\begin{Arabic}
\quranayah[24][2]
\end{Arabic}}
\flushleft{\begin{malayalam}
വ്യഭിചാരിണിയെയും വ്യഭിചാരിയെയും നൂറടിവീതം അടിക്കുക. അല്ലാഹുവിന്റെ നിയമവ്യവസ്ഥ നടപ്പാക്കുന്നകാര്യത്തില്‍ അവരോടുള്ള ദയ നിങ്ങളെ പിടികൂടാതിരിക്കട്ടെ- നിങ്ങള്‍ അല്ലാഹുവിലും അന്ത്യദിനത്തിലും വിശ്വസിക്കുന്നവരെങ്കില്‍. അവരെ ശിക്ഷിക്കുന്നതിന് സത്യവിശ്വാസികളിലൊരുസംഘം സാക്ഷ്യംവഹിക്കുകയും ചെയ്യട്ടെ.
\end{malayalam}}
\flushright{\begin{Arabic}
\quranayah[24][3]
\end{Arabic}}
\flushleft{\begin{malayalam}
വ്യഭിചാരി വ്യഭിചാരിണിയെയോ ബഹുദൈവവിശ്വാസിനിയെയോ അല്ലാതെ വിവാഹം കഴിക്കുകയില്ല. വ്യഭിചാരിണിയെ വ്യഭിചാരിയോ ബഹുദൈവ വിശ്വാസിയോ അല്ലാതെ വിവാഹംചെയ്യുകയുമില്ല. സത്യവിശ്വാസികള്‍ക്ക് അത് നിഷിദ്ധമാക്കിയിരിക്കുന്നു.
\end{malayalam}}
\flushright{\begin{Arabic}
\quranayah[24][4]
\end{Arabic}}
\flushleft{\begin{malayalam}
നാലു സാക്ഷികളെ ഹാജറാക്കാതെ ചാരിത്രവതികളുടെമേല്‍ കുറ്റമാരോപിക്കുന്നവരെ നിങ്ങള്‍ എണ്‍പത് അടിവീതം അടിക്കുക. അവരുടെ സാക്ഷ്യം പിന്നീടൊരിക്കലും സ്വീകരിക്കരുത്. അവര്‍തന്നെയാണ് തെമ്മാടികള്‍.
\end{malayalam}}
\flushright{\begin{Arabic}
\quranayah[24][5]
\end{Arabic}}
\flushleft{\begin{malayalam}
അതിനുശേഷം പശ്ചാത്തപിക്കുകയും വിശുദ്ധിവരിക്കുകയും ചെയ്തവരൊഴികെ. അല്ലാഹു ഏറെ പൊറുക്കുന്നവനും പരമകാരുണികനുമാണ്.
\end{malayalam}}
\flushright{\begin{Arabic}
\quranayah[24][6]
\end{Arabic}}
\flushleft{\begin{malayalam}
തങ്ങളുടെ ഭാര്യമാരുടെമേല്‍ കുറ്റമാരോപിക്കുകയും അതിനു തങ്ങളല്ലാതെ മറ്റു സാക്ഷികളില്ലാതിരിക്കുകയുമാണെങ്കില്‍, അവരിലൊരാളുടെ സാക്ഷ്യം “താന്‍ തീര്‍ച്ചയായും സത്യവാനാണെ”ന്ന് അല്ലാഹുവിന്റെപേരില്‍ നാലുതവണ ആണയിട്ട് പറയലാണ്.
\end{malayalam}}
\flushright{\begin{Arabic}
\quranayah[24][7]
\end{Arabic}}
\flushleft{\begin{malayalam}
അഞ്ചാം തവണ, താന്‍ കള്ളം പറയുന്നവനാണെങ്കില്‍ ദൈവശാപം തന്റെമേല്‍ പതിക്കട്ടെ എന്നും പറയണം.
\end{malayalam}}
\flushright{\begin{Arabic}
\quranayah[24][8]
\end{Arabic}}
\flushleft{\begin{malayalam}
“തീര്‍ച്ചയായും അയാള്‍ കള്ളം പറയുന്നവനാണെ”ന്ന് അല്ലാഹുവിന്റെ പേരില്‍ അവള്‍ നാലു തവണ ആണയിട്ടു സാക്ഷ്യപ്പെടുത്തിയാല്‍ അത് അവളെ ശിക്ഷയില്‍നിന്നൊഴിവാക്കുന്നതാണ്.
\end{malayalam}}
\flushright{\begin{Arabic}
\quranayah[24][9]
\end{Arabic}}
\flushleft{\begin{malayalam}
അഞ്ചാം തവണ, അവന്‍ സത്യവാനെങ്കില്‍ അല്ലാഹുവിന്റെ ശാപം തന്റെമേല്‍ പതിക്കട്ടെ എന്നും പറയണം.
\end{malayalam}}
\flushright{\begin{Arabic}
\quranayah[24][10]
\end{Arabic}}
\flushleft{\begin{malayalam}
അല്ലാഹുവിന്റെ അനുഗ്രഹവും കാരുണ്യവും നിങ്ങള്‍ക്കില്ലാതിരിക്കുകയും അല്ലാഹു ഏറെ പശ്ചാത്താപം സ്വീകരിക്കുന്നവനും യുക്തിമാനും അല്ലാതിരിക്കുകയുമാണെങ്കില്‍ നിങ്ങളുടെ അവസ്ഥ എന്താകുമായിരുന്നു.
\end{malayalam}}
\flushright{\begin{Arabic}
\quranayah[24][11]
\end{Arabic}}
\flushleft{\begin{malayalam}
തീര്‍ച്ചയായും ഈ അപവാദം പറഞ്ഞുപരത്തിയവര്‍ നിങ്ങളില്‍ നിന്നുതന്നെയുള്ള ഒരു വിഭാഗമാണ്. അത് നിങ്ങള്‍ക്ക് ദോഷകരമാണെന്ന് നിങ്ങള്‍ കരുതേണ്ട. മറിച്ച് അത് നിങ്ങള്‍ക്കു ഗുണകരമാണ്. അവരിലോരോരുത്തര്‍ക്കും താന്‍ സമ്പാദിച്ച പാപത്തിന്റെ ഫലമുണ്ട്. അതോടൊപ്പം അതിനു നേതൃത്വം നല്‍കിയവന് കടുത്ത ശിക്ഷയുമുണ്ട്.
\end{malayalam}}
\flushright{\begin{Arabic}
\quranayah[24][12]
\end{Arabic}}
\flushleft{\begin{malayalam}
ആ വാര്‍ത്ത കേട്ടപ്പോള്‍തന്നെ സത്യവിശ്വാസികളായ സ്ത്രീ പുരുഷന്മാര്‍ക്ക് സ്വന്തം ആളുകളെപ്പറ്റി നല്ലതു വിചാരിക്കാമായിരുന്നില്ലേ? “ഇതു തികഞ്ഞ അപവാദമാണെ”ന്ന് അവര്‍ പറയാതിരുന്നതെന്തുകൊണ്ട്?
\end{malayalam}}
\flushright{\begin{Arabic}
\quranayah[24][13]
\end{Arabic}}
\flushleft{\begin{malayalam}
അവരെന്തുകൊണ്ട് അതിനു നാലു സാക്ഷികളെ ഹാജരാക്കിയില്ല? അവര്‍ സാക്ഷികളെ ഹാജരാക്കാത്തതിനാല്‍ അവര്‍ തന്നെയാണ് അല്ലാഹുവിങ്കല്‍ അസത്യവാദികള്‍.
\end{malayalam}}
\flushright{\begin{Arabic}
\quranayah[24][14]
\end{Arabic}}
\flushleft{\begin{malayalam}
ഇഹത്തിലും പരത്തിലും അല്ലാഹുവിന്റെ അനുഗ്രഹവും കാരുണ്യവും നിങ്ങള്‍ക്കുണ്ടായിരുന്നില്ലെങ്കില്‍, ഈ അപവാദവാര്‍ത്തകളില്‍ മുഴുകിക്കഴിഞ്ഞതിന്റെ പേരില്‍ നിങ്ങളെ കഠിനമായ ശിക്ഷ ബാധിക്കുമായിരുന്നു.
\end{malayalam}}
\flushright{\begin{Arabic}
\quranayah[24][15]
\end{Arabic}}
\flushleft{\begin{malayalam}
നിങ്ങള്‍ ഈ അപവാദം നിങ്ങളുടെ നാവുകൊണ്ട് ഏറ്റുപറഞ്ഞു. നിങ്ങള്‍ക്കറിയാത്ത കാര്യങ്ങള്‍ നിങ്ങളുടെ വായകൊണ്ടു പറഞ്ഞുപരത്തി. അപ്പോള്‍ നിങ്ങളത് നന്നെ നിസ്സാരമാണെന്നുകരുതി. എന്നാല്‍ അല്ലാഹുവിങ്കലത് അത്യന്തം ഗുരുതരമായ കാര്യമാണ്.
\end{malayalam}}
\flushright{\begin{Arabic}
\quranayah[24][16]
\end{Arabic}}
\flushleft{\begin{malayalam}
അതുകേട്ട ഉടനെ നിങ്ങളെന്തുകൊണ്ടിങ്ങനെ പറഞ്ഞില്ല: "നമുക്ക് ഇത്തരം കാര്യങ്ങളെ സംബന്ധിച്ച് സംസാരിക്കാന്‍ പാടില്ല. അല്ലാഹുവേ നീയെത്ര പരിശുദ്ധന്‍! ഇത് അതിഗുരുതരമായ അപവാദം തന്നെ.”
\end{malayalam}}
\flushright{\begin{Arabic}
\quranayah[24][17]
\end{Arabic}}
\flushleft{\begin{malayalam}
അല്ലാഹു നിങ്ങളെയിതാ ഉപദേശിക്കുന്നു: "നിങ്ങളൊരിക്കലും ഇതുപോലുള്ളത് ആവര്‍ത്തിക്കരുത്. നിങ്ങള്‍ സത്യവിശ്വാസികളെങ്കില്‍!”
\end{malayalam}}
\flushright{\begin{Arabic}
\quranayah[24][18]
\end{Arabic}}
\flushleft{\begin{malayalam}
അല്ലാഹു നിങ്ങള്‍ക്ക് അവന്റെ പ്രമാണങ്ങള്‍ വിവരിച്ചുതരുന്നു. അല്ലാഹു എല്ലാം അറിയുന്നവനും യുക്തിമാനുമാണ്.
\end{malayalam}}
\flushright{\begin{Arabic}
\quranayah[24][19]
\end{Arabic}}
\flushleft{\begin{malayalam}
സത്യവിശ്വാസികള്‍ക്കിടയില്‍ അശ്ളീലം പ്രചരിക്കുന്നതില്‍ കൌതുകം കാട്ടുന്നവര്‍ക്ക് ഇഹത്തിലും പരത്തിലും നോവുറ്റ ശിക്ഷയുണ്ട്. അല്ലാഹു എല്ലാം അറിയുന്നു. നിങ്ങളോ അറിയുന്നുമില്ല.
\end{malayalam}}
\flushright{\begin{Arabic}
\quranayah[24][20]
\end{Arabic}}
\flushleft{\begin{malayalam}
നിങ്ങള്‍ക്ക് അല്ലാഹുവിന്റെ അനുഗ്രഹവും കാരുണ്യവുമില്ലാതിരിക്കുകയും അല്ലാഹു കൃപയും കാരുണ്യവുമില്ലാത്തവനാവുകയുമാണെങ്കില്‍ നിങ്ങളുടെ അവസ്ഥ എന്തായിരിക്കും?
\end{malayalam}}
\flushright{\begin{Arabic}
\quranayah[24][21]
\end{Arabic}}
\flushleft{\begin{malayalam}
വിശ്വസിച്ചവരേ, നിങ്ങള്‍ പിശാചിന്റെ കാല്‍പ്പാടുകള്‍ പിന്‍പറ്റരുത്. ആരെങ്കിലും പിശാചിന്റെ കാല്‍പ്പാടുകള്‍ പിന്‍പറ്റുകയാണെങ്കില്‍ അറിയുക: നീചവും നിഷിദ്ധവും ചെയ്യാനായിരിക്കും പിശാച് കല്‍പിക്കുക. നിങ്ങള്‍ക്ക് അല്ലാഹുവിന്റെ കാരുണ്യവും അനുഗ്രഹവും ഇല്ലായിരുന്നെങ്കില്‍ നിങ്ങളിലാരും ഒരിക്കലും വിശുദ്ധിവരിക്കുമായിരുന്നില്ല. എന്നാല്‍ അല്ലാഹു അവനിച്ഛിക്കുന്നവരെ ശുദ്ധീകരിക്കുന്നു. അല്ലാഹു എല്ലാം കേള്‍ക്കുന്നവനും അറിയുന്നവനുമാണ്.
\end{malayalam}}
\flushright{\begin{Arabic}
\quranayah[24][22]
\end{Arabic}}
\flushleft{\begin{malayalam}
നിങ്ങളില്‍ ദൈവാനുഗ്രഹവും സാമ്പത്തിക കഴിവുമുള്ളവര്‍, തങ്ങളുടെ കുടുംബക്കാര്‍ക്കും അഗതികള്‍ക്കും അല്ലാഹുവിന്റെ മാര്‍ഗത്തില്‍ നാടുവെടിഞ്ഞ് പലായനം ചെയ്തെത്തിയവര്‍ക്കും സഹായം കൊടുക്കുകയില്ലെന്ന് ശപഥം ചെയ്യരുത്. അവര്‍ മാപ്പുനല്‍കുകയും വിട്ടുവീഴ്ച കാണിക്കുകയും ചെയ്യട്ടെ. അല്ലാഹു നിങ്ങള്‍ക്ക് പൊറുത്തുതരണമെന്ന് നിങ്ങളാഗ്രഹിക്കുന്നില്ലേ? അല്ലാഹു ഏറെ പൊറുക്കുന്നവനും പരമകാരുണികനുമാണ്.
\end{malayalam}}
\flushright{\begin{Arabic}
\quranayah[24][23]
\end{Arabic}}
\flushleft{\begin{malayalam}
പതിവ്രതകളും ദുര്‍ന്നടപടിയെക്കുറിച്ചാലോചിക്കുകപോലും ചെയ്യാത്തവരുമായ സത്യവിശ്വാസിനികളെസംബന്ധിച്ച് ദുരാരോപണമുന്നയിക്കുന്നവര്‍ ഇഹത്തിലും പരത്തിലും ശപിക്കപ്പെട്ടിരിക്കുന്നു. അവര്‍ക്ക് കഠിനമായ ശിക്ഷയുണ്ട്.
\end{malayalam}}
\flushright{\begin{Arabic}
\quranayah[24][24]
\end{Arabic}}
\flushleft{\begin{malayalam}
അവര്‍ പ്രവര്‍ത്തിച്ചുകൊണ്ടിരിക്കുന്നതിനെപ്പറ്റി അവരുടെതന്നെ നാവുകളും കൈകാലുകളും സാക്ഷിനില്‍ക്കുന്ന നാളിലാണ് അതുണ്ടാവുക.
\end{malayalam}}
\flushright{\begin{Arabic}
\quranayah[24][25]
\end{Arabic}}
\flushleft{\begin{malayalam}
അന്ന് അല്ലാഹു അവര്‍ക്ക് അവരര്‍ഹിക്കുന്ന പ്രതിഫലം പൂര്‍ണമായി നല്‍കും. അല്ലാഹു തന്നെയാണ് പ്രത്യക്ഷമായ സത്യമെന്ന് അവര്‍ അന്നറിയും.
\end{malayalam}}
\flushright{\begin{Arabic}
\quranayah[24][26]
\end{Arabic}}
\flushleft{\begin{malayalam}
ദുഷിച്ച സ്ത്രീകള്‍ ദുഷിച്ച പുരുഷന്മാര്‍ക്കുള്ളവരാണ്. ദുഷിച്ച പുരുഷന്മാര്‍ ദുഷിച്ച സ്ത്രീകള്‍ക്കും. പരിശുദ്ധകളായ സ്ത്രീകള്‍ പരിശുദ്ധരായ പുരുഷന്മാര്‍ക്കുള്ളതാണ്. പരിശുദ്ധരായ പുരുഷന്മാര്‍ പരിശുദ്ധകളായ സ്ത്രീകള്‍ക്കും. ആളുകള്‍ ആരോപിക്കുന്ന കാര്യത്തില്‍ അവര്‍ നിരപരാധരാണ്. അവര്‍ക്ക് പാപമോചനമുണ്ട്. മാന്യമായ ജീവിതവിഭവങ്ങളും.
\end{malayalam}}
\flushright{\begin{Arabic}
\quranayah[24][27]
\end{Arabic}}
\flushleft{\begin{malayalam}
വിശ്വസിച്ചവരേ, നിങ്ങളുടെതല്ലാത്ത വീടുകളില്‍ നിങ്ങള്‍ പ്രവേശിക്കരുത്; ആ വീട്ടുകാരോട് നിങ്ങള്‍ അനുവാദംതേടുകയും അവര്‍ക്ക് സലാംപറയുകയും ചെയ്യുംവരെ. അതാണ് നിങ്ങള്‍ക്കുത്തമം. നിങ്ങളിതു ചിന്തിച്ചുമനസ്സിലാക്കുമല്ലോ.
\end{malayalam}}
\flushright{\begin{Arabic}
\quranayah[24][28]
\end{Arabic}}
\flushleft{\begin{malayalam}
അഥവാ, നിങ്ങള്‍ അവിടെ ആരെയും കണ്ടില്ലെങ്കില്‍ നിങ്ങള്‍ക്ക് അനുവാദം കിട്ടുംവരെ അകത്തുകടക്കരുത്. നിങ്ങളോട് തിരിച്ചുപോകാനാണ് ആവശ്യപ്പെട്ടതെങ്കില്‍ നിങ്ങള്‍ മടങ്ങിപ്പോവണം. അതാണ് നിങ്ങള്‍ക്കേറെ പവിത്രമായ നിലപാട്. അല്ലാഹു നിങ്ങള്‍ പ്രവര്‍ത്തിക്കുന്നതെന്തും നന്നായറിയുന്നവനാണ്.
\end{malayalam}}
\flushright{\begin{Arabic}
\quranayah[24][29]
\end{Arabic}}
\flushleft{\begin{malayalam}
എന്നാല്‍ ആള്‍പാര്‍പ്പില്ലാത്തതും നിങ്ങള്‍ക്കാവശ്യമായ വസ്തുക്കളുള്ളതുമായ വീടുകളില്‍ നിങ്ങള്‍ പ്രവേശിക്കുന്നതില്‍ തെറ്റില്ല. നിങ്ങള്‍ വെളിപ്പെടുത്തുന്നതും മറച്ചുവെക്കുന്നതും അല്ലാഹു അറിയുന്നു.
\end{malayalam}}
\flushright{\begin{Arabic}
\quranayah[24][30]
\end{Arabic}}
\flushleft{\begin{malayalam}
നീ സത്യവിശ്വാസികളോട് പറയുക: അവര്‍ തങ്ങളുടെ ദൃഷ്ടികള്‍ നിയന്ത്രിക്കട്ടെ. ഗുഹ്യഭാഗങ്ങള്‍ സൂക്ഷിക്കുകയും ചെയ്യട്ടെ. അതാണ് അവരുടെ പരിശുദ്ധിക്ക് ഏറ്റം പറ്റിയത്. സംശയം വേണ്ട; അല്ലാഹു അവരുടെ പ്രവര്‍ത്തനങ്ങളെപ്പറ്റിയെല്ലാം നന്നായി അറിയുന്നവനാണ്.
\end{malayalam}}
\flushright{\begin{Arabic}
\quranayah[24][31]
\end{Arabic}}
\flushleft{\begin{malayalam}
നീ സത്യവിശ്വാസിനികളോട് പറയുക: അവരും തങ്ങളുടെ ദൃഷ്ടികള്‍ നിയന്ത്രിക്കണം. ഗുഹ്യഭാഗങ്ങള്‍ കാത്തുസൂക്ഷിക്കണം; തങ്ങളുടെ ശരീരസൌന്ദര്യം വെളിപ്പെടുത്തരുത്; സ്വയം വെളിവായതൊഴികെ. ശിരോവസ്ത്രം മാറിടത്തിനുമീതെ താഴ്ത്തിയിടണം. തങ്ങളുടെ ഭര്‍ത്താക്കന്മാര്‍, പിതാക്കള്‍, ഭര്‍ത്തൃപിതാക്കള്‍, പുത്രന്മാര്‍, ഭര്‍ത്തൃപുത്രന്മാര്‍, സഹോദരങ്ങള്‍, സഹോദരപുത്രന്മാര്‍, സഹോദരീപുത്രന്മാര്‍, തങ്ങളുമായി ഇടപഴകുന്ന സ്ത്രീകള്‍, വലംകൈ ഉടമപ്പെടുത്തിയവര്‍, ലൈംഗികാസക്തിയില്ലാത്ത പുരുഷപരിചാരകര്‍, സ്ത്രൈണ രഹസ്യങ്ങളറിഞ്ഞിട്ടില്ലാത്ത കുട്ടികള്‍ എന്നിവരുടെ മുന്നിലൊഴികെ അവര്‍ തങ്ങളുടെ ശരീരഭംഗി വെളിവാക്കരുത്. മറച്ചുവെക്കുന്ന അലങ്കാരത്തിലേക്ക് ശ്രദ്ധ തിരിക്കാനായി കാലുകള്‍ നിലത്തടിച്ച് നടക്കരുത്. സത്യവിശ്വാസികളേ; നിങ്ങളെല്ലാവരും ഒന്നായി അല്ലാഹുവിങ്കലേക്ക് പശ്ചാത്തപിച്ചു മടങ്ങുക. നിങ്ങള്‍ വിജയം വരിച്ചേക്കാം.
\end{malayalam}}
\flushright{\begin{Arabic}
\quranayah[24][32]
\end{Arabic}}
\flushleft{\begin{malayalam}
നിങ്ങളിലെ ഇണയില്ലാത്തവരെയും നിങ്ങളുടെ അടിമകളായ സ്ത്രീപുരുഷന്മാരില്‍ നല്ലവരെയും നിങ്ങള്‍ വിവാഹം കഴിപ്പിക്കുക. അവരിപ്പോള്‍ ദരിദ്രരാണെങ്കില്‍ അല്ലാഹു തന്റെ ഔദാര്യത്താല്‍ അവര്‍ക്ക് ഐശ്വര്യമേകും. അല്ലാഹു ഏറെ ഉദാരനും എല്ലാം അറിയുന്നവനുമാണ്.
\end{malayalam}}
\flushright{\begin{Arabic}
\quranayah[24][33]
\end{Arabic}}
\flushleft{\begin{malayalam}
വിവാഹം കഴിക്കാന്‍ കഴിവില്ലാത്തവര്‍ അല്ലാഹു തന്റെ ഔദാര്യത്താല്‍ അവരെ സ്വന്തം കാലില്‍ നില്‍ക്കാന്‍ കരുത്തുറ്റവരാക്കുംവരെ സദാചാരനിഷ്ഠ പാലിക്കണം. നിങ്ങളുടെ അടിമകളില്‍ മോചനക്കരാറിലേര്‍പ്പെടാന്‍ ആഗ്രഹിക്കുന്നവരുമായി നിങ്ങള്‍ മോചനക്കരാറുണ്ടാക്കുക. അവരില്‍ നന്മയുള്ളതായി നിങ്ങള്‍ക്കു ബോധ്യമുണ്ടെങ്കില്‍! അല്ലാഹു നിങ്ങള്‍ക്കേകിയ അവന്റെ ധനത്തില്‍നിന്ന് അവര്‍ക്ക് കൊടുക്കുകയും ചെയ്യുക. ഭൌതികനേട്ടം കൊതിച്ച്, നിങ്ങളുടെ അടിമസ്ത്രീകളെ- അവര്‍ ചാരിത്രവതികളായി ജീവിക്കാനാഗ്രഹിക്കുമ്പോള്‍- നിങ്ങള്‍ വേശ്യാവൃത്തിക്ക് നിര്‍ബന്ധിക്കരുത്. ആരെങ്കിലുമവരെ അതിനു നിര്‍ബന്ധിക്കുകയാണെങ്കില്‍ ആ നിര്‍ബന്ധിതരോട് അല്ലാഹു ഏറെ പൊറുക്കുന്നവനും പരമകാരുണികനുമല്ലോ.
\end{malayalam}}
\flushright{\begin{Arabic}
\quranayah[24][34]
\end{Arabic}}
\flushleft{\begin{malayalam}
നിങ്ങള്‍ക്കു നാം കാര്യങ്ങള്‍ വ്യക്തമാക്കുന്ന വചനങ്ങളിറക്കിത്തന്നിരിക്കുന്നു. നിങ്ങള്‍ക്കുമുമ്പ് കഴിഞ്ഞുപോയ സമൂഹങ്ങളുടെ ഉദാഹരണങ്ങളും സൂക്ഷ്മശാലികള്‍ക്കുള്ള സദുപദേശങ്ങളും നല്‍കിയിരിക്കുന്നു.
\end{malayalam}}
\flushright{\begin{Arabic}
\quranayah[24][35]
\end{Arabic}}
\flushleft{\begin{malayalam}
അല്ലാഹു ആകാശഭൂമികളുടെ വെളിച്ചമാണ്. അവന്റെ വെളിച്ചത്തിന്റെ ഉപമയിതാ: ഒരു വിളക്കുമാടം; അതിലൊരു വിളക്ക്. വിളക്ക് ഒരു സ്ഫടികക്കൂട്ടിലാണ്. സ്ഫടികക്കൂട് വെട്ടിത്തിളങ്ങുന്ന ആകാശനക്ഷത്രം പോലെയും. അനുഗൃഹീതമായ ഒരു വൃക്ഷത്തില്‍ നിന്നുള്ള എണ്ണ കൊണ്ടാണത് കത്തുന്നത്. അഥവാ, കിഴക്കനോ പടിഞ്ഞാറനോ അല്ലാത്ത ഒലീവ് വൃക്ഷത്തില്‍നിന്ന്. അതിന്റെ എണ്ണ തീ കൊളുത്തിയില്ലെങ്കില്‍പോലും സ്വയം പ്രകാശിക്കുമാറാകും. വെളിച്ചത്തിനുമേല്‍ വെളിച്ചം. അല്ലാഹു തന്റെ വെളിച്ചത്തിലേക്ക് താനിച്ഛിക്കുന്നവരെ നയിക്കുന്നു. അവന്‍ സര്‍വ ജനത്തിനുമായി ഉദാഹരണങ്ങള്‍ വിശദീകരിക്കുന്നു. അല്ലാഹു സകല സംഗതികളും നന്നായറിയുന്നവനാണ്.
\end{malayalam}}
\flushright{\begin{Arabic}
\quranayah[24][36]
\end{Arabic}}
\flushleft{\begin{malayalam}
ആ വെളിച്ചം ലഭിച്ചവരുണ്ടാവുക ചില മന്ദിരങ്ങളിലാണ്. അവ പടുത്തുയര്‍ത്താനും അവിടെ തന്റെ നാമം ഉരുവിടാനും അല്ലാഹു ഉത്തരവ് നല്‍കിയിരിക്കുന്നു. രാവിലെയും വൈകുന്നേരവും അവിടെ അവന്റെ വിശുദ്ധി വാഴ്ത്തിക്കൊണ്ടിരിക്കുന്നു.
\end{malayalam}}
\flushright{\begin{Arabic}
\quranayah[24][37]
\end{Arabic}}
\flushleft{\begin{malayalam}
കച്ചവടമോ കൊള്ളക്കൊടുക്കകളോ അല്ലാഹുവെ സ്മരിക്കുന്നതിനും നമസ്കാരം നിലനിര്‍ത്തുന്നതിനും സകാത്ത് നല്‍കുന്നതിനും തടസ്സമാകാത്ത ചില വിശുദ്ധന്മാരാണ് അങ്ങനെ ചെയ്തുകൊണ്ടിരിക്കുന്നത്. മനസ്സുകള്‍ താളംതെറ്റുകയും കണ്ണുകള്‍ ഇളകിമറിയുകയും ചെയ്യുന്ന അന്ത്യനാളിനെ ഭയപ്പെടുന്നവരാണവര്‍.
\end{malayalam}}
\flushright{\begin{Arabic}
\quranayah[24][38]
\end{Arabic}}
\flushleft{\begin{malayalam}
അല്ലാഹു അവര്‍ക്ക് തങ്ങള്‍ ചെയ്ത ഏറ്റം നല്ല പ്രവര്‍ത്തനങ്ങള്‍ക്ക് അര്‍ഹമായ പ്രതിഫലം നല്‍കാനാണത്. അവര്‍ക്ക് തന്റെ അനുഗ്രഹം കൂടുതലായി കൊടുക്കാനും. അല്ലാഹു താനിച്ഛിക്കുന്നവര്‍ക്ക് കണക്കില്ലാതെ കൊടുക്കുന്നു.
\end{malayalam}}
\flushright{\begin{Arabic}
\quranayah[24][39]
\end{Arabic}}
\flushleft{\begin{malayalam}
സത്യത്തെ തള്ളിപ്പറഞ്ഞവരുടെ സ്ഥിതിയോ, അവരുടെ പ്രവര്‍ത്തനങ്ങള്‍ മരുപ്പറമ്പിലെ മരീചികപോലെയാണ്. ദാഹിച്ചുവലഞ്ഞവന്‍ അത് വെള്ളമാണെന്നു കരുതുന്നു. അങ്ങനെ അവനതിന്റെ അടുത്തുചെന്നാല്‍ അവിടെയൊന്നുംതന്നെ കാണുകയില്ല. എന്നാല്‍ അവനവിടെ കണ്ടെത്തുക അല്ലാഹുവെയാണ്. അല്ലാഹു അവന്ന് തന്റെ കണക്ക് തീര്‍ത്തുകൊടുക്കുന്നു. അല്ലാഹു അതിവേഗം കണക്കു തീര്‍ക്കുന്നവനാണ്.
\end{malayalam}}
\flushright{\begin{Arabic}
\quranayah[24][40]
\end{Arabic}}
\flushleft{\begin{malayalam}
അല്ലെങ്കില്‍ അവരുടെ ഉപമ ഇങ്ങനെയാണ്: ആഴക്കടലിലെ ഘനാന്ധകാരം; അതിനെ തിരമാല മൂടിയിരിക്കുന്നു. അതിനുമീതെ വേറെയും തിരമാല. അതിനു മീതെ കാര്‍മേഘവും. ഇരുളിനുമേല്‍ ഇരുള്‍-ഒട്ടേറെ ഇരുട്ടുകള്‍. സ്വന്തം കൈ പുറത്തേക്കു നീട്ടിയാല്‍ അതുപോലും കാണാനാവാത്ത കൂരിരുട്ട്! അല്ലാഹു വെളിച്ചം നല്‍കാത്തവര്‍ക്ക് പിന്നെ വെളിച്ചമേയില്ല.
\end{malayalam}}
\flushright{\begin{Arabic}
\quranayah[24][41]
\end{Arabic}}
\flushleft{\begin{malayalam}
ആകാശഭൂമികളിലുള്ളവര്‍; ചിറകുവിരുത്തിപ്പറക്കുന്ന പക്ഷികള്‍; എല്ലാം അല്ലാഹുവിന്റെ വിശുദ്ധി വാഴ്ത്തിക്കൊണ്ടിരിക്കുന്നത് നീ കണ്ടിട്ടില്ലേ? തന്റെ പ്രാര്‍ഥനയും കീര്‍ത്തനവും എങ്ങനെയെന്ന് ഓരോന്നിനും നന്നായറിയാം. അവര്‍ പ്രവര്‍ത്തിക്കുന്നതിനെപ്പറ്റി സൂക്ഷ്മമായി അറിയുന്നവനാണ് അല്ലാഹു.
\end{malayalam}}
\flushright{\begin{Arabic}
\quranayah[24][42]
\end{Arabic}}
\flushleft{\begin{malayalam}
ആകാശഭൂമികളുടെ ആധിപത്യം അല്ലാഹുവിനാണ്. മടക്കവും അല്ലാഹുവിങ്കലേക്കുതന്നെ.
\end{malayalam}}
\flushright{\begin{Arabic}
\quranayah[24][43]
\end{Arabic}}
\flushleft{\begin{malayalam}
അല്ലാഹു കാര്‍മേഘത്തെ മന്ദംമന്ദം തെളിച്ചുകൊണ്ടുവരുന്നതും പിന്നീടവയെ ഒരുമിച്ചു ചേര്‍ക്കുന്നതും എന്നിട്ടതിനെ അട്ടിയാക്കിവെച്ച് കട്ടപിടിച്ചതാക്കുന്നതും നീ കണ്ടിട്ടില്ലേ? അങ്ങനെ അവയ്ക്കിടയില്‍ നിന്ന് മഴത്തുള്ളികള്‍ പുറപ്പെടുന്നത് നിനക്കു കാണാം. മാനത്തെ മലകള്‍പോലുള്ള മേഘക്കൂട്ടങ്ങളില്‍നിന്ന് അവന്‍ ആലിപ്പഴം വീഴ്ത്തുന്നു. എന്നിട്ട് താനിച്ഛിക്കുന്നവര്‍ക്ക് അതിന്റെ വിപത്ത് വരുത്തുന്നു. താനിച്ഛിക്കുന്നവരില്‍നിന്നത് തിരിച്ചുവിടുകയും ചെയ്യുന്നു. അതിന്റെ മിന്നല്‍വെളിച്ചം കാഴ്ചകളെ ഇല്ലാതാക്കാന്‍ പോന്നതാണ്.
\end{malayalam}}
\flushright{\begin{Arabic}
\quranayah[24][44]
\end{Arabic}}
\flushleft{\begin{malayalam}
അല്ലാഹു രാപ്പകലുകളെ മാറ്റിമറിച്ചുകൊണ്ടിരിക്കുന്നു. തീര്‍ച്ചയായും അതില്‍ കണ്ണുള്ളവര്‍ക്ക് ഗുണപാഠമുണ്ട്.
\end{malayalam}}
\flushright{\begin{Arabic}
\quranayah[24][45]
\end{Arabic}}
\flushleft{\begin{malayalam}
അല്ലാഹു എല്ലാ ജീവജാലങ്ങളെയും വെള്ളത്തില്‍നിന്ന് സൃഷ്ടിച്ചു. അവയില്‍ ഉദരത്തിന്മേല്‍ ഇഴയുന്നവയുണ്ട്. ഇരുകാലില്‍ നടക്കുന്നവയുണ്ട്. നാലുകാലില്‍ ചരിക്കുന്നവയുമുണ്ട്. അല്ലാഹു അവനിച്ഛിക്കുന്നത് സൃഷ്ടിക്കുന്നു. അവന്‍ എല്ലാ കാര്യത്തിനും കഴിവുറ്റവനാണ്.
\end{malayalam}}
\flushright{\begin{Arabic}
\quranayah[24][46]
\end{Arabic}}
\flushleft{\begin{malayalam}
നാം നിയമങ്ങള്‍ വ്യക്തമാക്കുന്ന വചനങ്ങള്‍ ഇറക്കിത്തന്നിരിക്കുന്നു. അല്ലാഹു അവനിച്ഛിക്കുന്നവരെ നേര്‍വഴിക്ക് നയിക്കുന്നു.
\end{malayalam}}
\flushright{\begin{Arabic}
\quranayah[24][47]
\end{Arabic}}
\flushleft{\begin{malayalam}
അവര്‍ പറയുന്നു: "ഞങ്ങള്‍ അല്ലാഹുവിലും അവന്റെ ദൂതനിലും വിശ്വസിച്ചിരിക്കുന്നു. അവരെ അനുസരിക്കുകയും ചെയ്തിരിക്കുന്നു.” എന്നാല്‍ അതിനുശേഷം അവരിലൊരുവിഭാഗം പിന്തിരിഞ്ഞുപോകുന്നു. അവര്‍ വിശ്വാസികളേയല്ല.
\end{malayalam}}
\flushright{\begin{Arabic}
\quranayah[24][48]
\end{Arabic}}
\flushleft{\begin{malayalam}
അവര്‍ക്കിടയില്‍ വിധിത്തീര്‍പ്പ് കല്‍പിക്കാനായി അവരെ അല്ലാഹുവിങ്കലേക്കും അവന്റെ ദൂതനിലേക്കും വിളിച്ചാല്‍ അവരിലൊരു വിഭാഗം ഒഴിഞ്ഞുമാറുന്നു.
\end{malayalam}}
\flushright{\begin{Arabic}
\quranayah[24][49]
\end{Arabic}}
\flushleft{\begin{malayalam}
അഥവാ ന്യായം അവര്‍ക്കനുകൂലമാണെങ്കിലോ അവര്‍ ദൈവദൂതന്റെ അടുത്തേക്ക് വിധേയത്വഭാവത്തോടെ വരികയും ചെയ്യുന്നു.
\end{malayalam}}
\flushright{\begin{Arabic}
\quranayah[24][50]
\end{Arabic}}
\flushleft{\begin{malayalam}
അവരുടെ ഹൃദയങ്ങളില്‍ കാപട്യത്തിന്റെ ദീനമുണ്ടോ? അല്ലെങ്കിലവര്‍ സംശയത്തിലകപ്പെട്ടതാണോ? അതുമല്ലെങ്കില്‍ അല്ലാഹുവും അവന്റെ ദൂതനും അവരോട് അനീതി കാണിച്ചേക്കുമെന്ന് അവര്‍ ഭയപ്പെടുകയാണോ? എന്നാല്‍ കാര്യം ഇതൊന്നുമല്ല; അവര്‍ തന്നെയാണ് ധിക്കാരികള്‍.
\end{malayalam}}
\flushright{\begin{Arabic}
\quranayah[24][51]
\end{Arabic}}
\flushleft{\begin{malayalam}
എന്നാല്‍ തങ്ങള്‍ക്കിടയില്‍ വിധിത്തീര്‍പ്പ് കല്‍പിക്കാനായി അല്ലാഹുവിലേക്കും അവന്റെ ദൂതനിലേക്കും ക്ഷണിച്ചാല്‍ സത്യവിശ്വാസികള്‍ പറയുക ഇതുമാത്രമായിരിക്കും: "ഞങ്ങള്‍ കേട്ടിരിക്കുന്നു. അനുസരിച്ചിരിക്കുന്നു.” അവര്‍ തന്നെയാണ് വിജയികള്‍.
\end{malayalam}}
\flushright{\begin{Arabic}
\quranayah[24][52]
\end{Arabic}}
\flushleft{\begin{malayalam}
അല്ലാഹുവെയും അവന്റെ ദൂതനെയും അനുസരിക്കുകയും അല്ലാഹുവെ ഭയപ്പെടുകയും അവനോട് ഭക്തിപുലര്‍ത്തുകയും ചെയ്യുന്നവരാണ് വിജയംവരിക്കുന്നവര്‍.
\end{malayalam}}
\flushright{\begin{Arabic}
\quranayah[24][53]
\end{Arabic}}
\flushleft{\begin{malayalam}
അവര്‍ തങ്ങളാലാവുംവിധമൊക്കെ അല്ലാഹുവിന്റെ പേരില്‍ ആണയിട്ടുപറയുന്നു, നീ അവരോട് കല്‍പിക്കുകയാണെങ്കില്‍ അവര്‍ പുറപ്പെടുകതന്നെ ചെയ്യുമെന്ന്. പറയുക: "നിങ്ങള്‍ ആണയിടേണ്ടതില്ല. ആത്മാര്‍ഥമായ അനുസരണമാണാവശ്യം. തീര്‍ച്ചയായും നിങ്ങള്‍ ചെയ്യുന്നതൊക്കെയും സൂക്ഷ്മമായി അറിയുന്നവനാണ് അല്ലാഹു.”
\end{malayalam}}
\flushright{\begin{Arabic}
\quranayah[24][54]
\end{Arabic}}
\flushleft{\begin{malayalam}
പറയുക: നിങ്ങള്‍ അല്ലാഹുവെ അനുസരിക്കുക. അവന്റെ ദൂതനെയും അനുസരിക്കുക. അഥവാ, നിങ്ങള്‍ പുറംതിരിഞ്ഞുപോവുകയാണെങ്കില്‍ അറിയുക: ദൈവദൂതന് ബാധ്യതയുള്ളത് അദ്ദേഹം ഭരമേല്‍പിക്കപ്പെട്ട കാര്യത്തില്‍ മാത്രമാണ്. നിങ്ങള്‍ക്കുള്ള ബാധ്യത നിങ്ങള്‍ ഭരമേല്‍പിക്കപ്പെട്ട കാര്യത്തിലും. നിങ്ങള്‍ അദ്ദേഹത്തെ അനുസരിക്കുന്നുവെങ്കില്‍ നിങ്ങള്‍ക്കു നേര്‍വഴി നേടാം. ദൈവദൂതന്റെ ബാധ്യത, സന്ദേശം തെളിമയോടെ എത്തിക്കല്‍ മാത്രമാണ്.
\end{malayalam}}
\flushright{\begin{Arabic}
\quranayah[24][55]
\end{Arabic}}
\flushleft{\begin{malayalam}
നിങ്ങളില്‍ നിന്ന് സത്യവിശ്വാസം സ്വീകരിക്കുകയും സല്‍ക്കര്‍മങ്ങള്‍ പ്രവര്‍ത്തിക്കുകയും ചെയ്തവരോട് അല്ലാഹു വാഗ്ദാനം ചെയ്തിരിക്കുന്നു: "അവന്‍ അവരെ ഭൂമിയിലെ പ്രതിനിധികളാക്കും. അവരുടെ മുമ്പുള്ളവരെ പ്രതിനിധികളാക്കിയപോലെത്തന്നെ. അവര്‍ക്കായി അല്ലാഹു തൃപ്തിപ്പെട്ടേകിയ അവരുടെ ജീവിത വ്യവസ്ഥ സ്ഥാപിച്ചുകൊടുക്കും. നിലവിലുള്ള അവരുടെ ഭയാവസ്ഥക്കുപകരം നിര്‍ഭയാവസ്ഥ ഉണ്ടാക്കിക്കൊടുക്കും.” അവര്‍ എനിക്കു മാത്രമാണ് വഴിപ്പെടുക. എന്നിലൊന്നിനെയും പങ്കുചേര്‍ക്കുകയില്ല. അതിനുശേഷം ആരെങ്കിലും സത്യത്തെ നിഷേധിക്കുന്നുവെങ്കില്‍ അവര്‍ തന്നെയാണ് ധിക്കാരികള്‍.
\end{malayalam}}
\flushright{\begin{Arabic}
\quranayah[24][56]
\end{Arabic}}
\flushleft{\begin{malayalam}
നിങ്ങള്‍ നമസ്കാരം നിഷ്ഠയോടെ നിര്‍വഹിക്കുക. സകാത്ത് നല്‍കുക. ദൈവദൂതനെ അനുസരിക്കുക. നിങ്ങള്‍ക്ക് ദിവ്യാനുഗ്രഹം ലഭിച്ചേക്കാം.
\end{malayalam}}
\flushright{\begin{Arabic}
\quranayah[24][57]
\end{Arabic}}
\flushleft{\begin{malayalam}
സത്യനിഷേധികള്‍, ഇവിടെ ഭൂമിയില്‍ അല്ലാഹുവെ തോല്‍പിച്ചുകളയുമെന്ന് നീ കരുതരുത്. അവരുടെ താവളം നരകത്തീയാണ്. അതെത്ര ചീത്ത സങ്കേതം!
\end{malayalam}}
\flushright{\begin{Arabic}
\quranayah[24][58]
\end{Arabic}}
\flushleft{\begin{malayalam}
വിശ്വസിച്ചവരേ, നിങ്ങളുടെ അടിമകളും നിങ്ങളിലെ പ്രായപൂര്‍ത്തിയെത്താത്തവരും മൂന്നു പ്രത്യേക സമയങ്ങളില്‍ അനുവാദം വാങ്ങിയശേഷമേ നിങ്ങളുടെയടുത്തു വരാന്‍ പാടുള്ളൂ. പ്രഭാത നമസ്കാരത്തിനു മുമ്പും ഉച്ചയുറക്കിന് നിങ്ങള്‍ വസ്ത്രമഴിച്ചുവെക്കുന്ന നേരത്തും ഇശാ നമസ്കാരത്തിനുശേഷവുമാണത്. ഇതുമൂന്നും നിങ്ങളുടെ സ്വകാര്യ സമയങ്ങളാണ്. മറ്റുസമയങ്ങളില്‍ അനുവാദമാരായാതെ നിങ്ങളുടെ അടുത്തുവരുന്നതില്‍ നിങ്ങള്‍ക്കോ അവര്‍ക്കോ കുറ്റമില്ല. അവര്‍ നിങ്ങളെ ചുറ്റിപ്പറ്റിക്കഴിയുന്നവരാണല്ലോ. നിങ്ങള്‍അന്യോന്യം ഇടകലര്‍ന്ന് ജീവിക്കുന്നവരുമാണ്. ഇവ്വിധം അല്ലാഹു നിങ്ങള്‍ക്ക് അവന്റെ നിയമങ്ങള്‍ വിവരിച്ചുതരുന്നു. അല്ലാഹു എല്ലാം അറിയുന്നവനാണ്. യുക്തിമാനും.
\end{malayalam}}
\flushright{\begin{Arabic}
\quranayah[24][59]
\end{Arabic}}
\flushleft{\begin{malayalam}
നിങ്ങളിലെ കുട്ടികള്‍ക്ക് പ്രായപൂര്‍ത്തിയെത്തിയാല്‍ അവരും അനുവാദം തേടണം; മറ്റുള്ളവര്‍ അനുവാദം തേടുന്നപോലെത്തന്നെ. ഇവ്വിധം അല്ലാഹു നിങ്ങള്‍ക്ക് അവന്റെ നിയമങ്ങള്‍ വിശദീകരിച്ചുതരുന്നു. അല്ലാഹു എല്ലാം അറിയുന്നവനും യുക്തിമാനുമാണ്.
\end{malayalam}}
\flushright{\begin{Arabic}
\quranayah[24][60]
\end{Arabic}}
\flushleft{\begin{malayalam}
വിവാഹജീവിതം കൊതിക്കാത്ത കിഴവികള്‍ തങ്ങളുടെ മേല്‍വസ്ത്രങ്ങള്‍ അഴിച്ചുവെക്കുന്നതില്‍ തെറ്റില്ല. എന്നാല്‍ അവര്‍ തങ്ങളുടെ ശരീരസൌന്ദര്യം പ്രദര്‍ശിപ്പിക്കുന്നവരാകരുത്. മാന്യത പുലര്‍ത്തുന്നതുതന്നെയാണ് അവര്‍ക്കും നല്ലത്. അല്ലാഹു എല്ലാം കേള്‍ക്കുന്നവനും അറിയുന്നവനുമാണ്.
\end{malayalam}}
\flushright{\begin{Arabic}
\quranayah[24][61]
\end{Arabic}}
\flushleft{\begin{malayalam}
നിങ്ങളുടെയോ നിങ്ങളുടെ പിതാക്കള്‍, മാതാക്കള്‍, സഹോദരന്മാര്‍, സഹോദരിമാര്‍, പിതൃവ്യന്മാര്‍, അമ്മായിമാര്‍, അമ്മാവന്മാര്‍, മാതൃസഹോദരിമാര്‍ എന്നിവരുടെയോ വീടുകളില്‍നിന്ന് ഭക്ഷണം കഴിക്കുന്നതില്‍ കുരുടന്നും മുടന്തന്നും രോഗിക്കും നിങ്ങള്‍ക്കും കുറ്റമില്ല. ഏതു വീടിന്റെ താക്കോലുകള്‍ നിങ്ങളുടെ വശമാണോ ആ വീടുകളില്‍നിന്നും നിങ്ങളുടെ കൂട്ടുകാരന്റെ വീട്ടില്‍നിന്നും ആഹാരംകഴിക്കുന്നതിലും തെറ്റില്ല. നിങ്ങള്‍ക്ക് ഒറ്റക്കോ കൂട്ടായോ ആഹാരം കഴിക്കാവുന്നതാണ്. എന്നാല്‍ നിങ്ങള്‍ വീടുകളില്‍ കടന്നുചെല്ലുകയാണെങ്കില്‍ അല്ലാഹുവില്‍നിന്നുള്ള അനുഗൃഹീതവും പവിത്രവുമായ അഭിവാദ്യമെന്നനിലയില്‍ നിങ്ങളന്യോന്യം സലാം പറയണം. ഇവ്വിധം അല്ലാഹു നിങ്ങള്‍ക്ക് അവന്റെ വചനങ്ങള്‍ വിശദീകരിച്ചുതരുന്നു. നിങ്ങള്‍ ചിന്തിച്ചുമനസ്സിലാക്കാന്‍.
\end{malayalam}}
\flushright{\begin{Arabic}
\quranayah[24][62]
\end{Arabic}}
\flushleft{\begin{malayalam}
അല്ലാഹുവിലും അവന്റെ ദൂതനിലും ആത്മാര്‍ഥമായി വിശ്വസിക്കുന്നവര്‍ മാത്രമാണ് സത്യവിശ്വാസികള്‍. പ്രവാചകനോടൊപ്പം ഏതെങ്കിലും പൊതുകാര്യത്തിലായിരിക്കെ, അദ്ദേഹത്തോട് അനുവാദം ചോദിക്കാതെ അവര്‍ പിരിഞ്ഞുപോവുകയില്ല. നിന്നോട് അനുവാദം ചോദിക്കുന്നവര്‍ ഉറപ്പായും അല്ലാഹുവിലും അവന്റെ ദൂതനിലും വിശ്വസിക്കുന്നവരാണ്. അതിനാല്‍ അവര്‍ തങ്ങളുടെ എന്തെങ്കിലും ആവശ്യനിര്‍വഹണത്തിന് നിന്നോട് അനുവാദം തേടിയാല്‍ നീ ഉദ്ദേശിക്കുന്നവര്‍ക്ക് അനുവാദം നല്‍കുക. അവര്‍ക്കുവേണ്ടി അല്ലാഹുവോട് പാപമോചനം തേടുക. അല്ലാഹു ഏറെ പൊറുക്കുന്നവനും പരമകാരുണികനുമാണ്.
\end{malayalam}}
\flushright{\begin{Arabic}
\quranayah[24][63]
\end{Arabic}}
\flushleft{\begin{malayalam}
നിങ്ങളോടുള്ള ദൈവദൂതന്റെ വിളി നിങ്ങള്‍ അന്യോന്യം വിളിക്കുംവിധംകരുതി അവഗണിക്കരുത്. മറ്റുള്ളവരെ മറയാക്കി നിങ്ങളില്‍നിന്ന് ഊരിച്ചാടുന്നവരെ അല്ലാഹു നന്നായറിയുന്നുണ്ട്. അതിനാല്‍ അദ്ദേഹത്തിന്റെ കല്‍പന ലംഘിക്കുന്നവര്‍ തങ്ങളെ വല്ലവിപത്തും ബാധിക്കുമെന്നോ നോവേറിയ ശിക്ഷ പിടികൂടുമെന്നോ തീര്‍ച്ചയായും ഭയപ്പെട്ടുകൊള്ളട്ടെ.
\end{malayalam}}
\flushright{\begin{Arabic}
\quranayah[24][64]
\end{Arabic}}
\flushleft{\begin{malayalam}
അറിയുക: ആകാശഭൂമികളിലുള്ളതൊക്കെയും അല്ലാഹുവിന്റേതാണ്. നിങ്ങള്‍ എന്തു നിലപാടാണെടുക്കുന്നതെന്ന് അവനു നന്നായറിയാം. അവങ്കലേക്ക് എല്ലാവരും തിരിച്ചുചെല്ലുന്ന നാളിനെക്കുറിച്ചും അവന്‍ നന്നായറിയുന്നു. അപ്പോള്‍ അവര്‍ പ്രവര്‍ത്തിച്ചുകൊണ്ടിരുന്നതിനെപ്പറ്റി അവന്‍ അവര്‍ക്ക് വിവരിച്ചുകൊടുക്കും. അല്ലാഹു സകല സംഗതികളും നന്നായറിയുന്നവനാണ്.
\end{malayalam}}
\chapter{\textmalayalam{ഫുര്‍ഖാന്‍ ( സത്യാസത്യ വിവേചനം )}}
\begin{Arabic}
\Huge{\centerline{\basmalah}}\end{Arabic}
\flushright{\begin{Arabic}
\quranayah[25][1]
\end{Arabic}}
\flushleft{\begin{malayalam}
തന്റെ ദാസന് ശരിതെറ്റുകളെ വേര്‍തിരിച്ചുകാണിക്കുന്ന ഈ പ്രമാണം ഇറക്കിക്കൊടുത്ത അല്ലാഹു അളവറ്റ അനുഗ്രഹമുള്ളവനാണ്. അദ്ദേഹം ലോകര്‍ക്കാകെ മുന്നറിയിപ്പു നല്‍കുന്നവനാകാന്‍ വേണ്ടിയാണിത്.
\end{malayalam}}
\flushright{\begin{Arabic}
\quranayah[25][2]
\end{Arabic}}
\flushleft{\begin{malayalam}
ആകാശഭൂമികളുടെ ആധിപത്യത്തിനുടമയാണവന്‍. അവനാരെയും പുത്രനായി സ്വീകരിച്ചിട്ടില്ല. ആധിപത്യത്തില്‍ അവന് ഒരു പങ്കാളിയുമില്ല. അവന്‍ സകല വസ്തുക്കളെയും സൃഷ്ടിച്ചു. കൃത്യമായി അവയെ ക്രമീകരിക്കുകയും ചെയ്തു.
\end{malayalam}}
\flushright{\begin{Arabic}
\quranayah[25][3]
\end{Arabic}}
\flushleft{\begin{malayalam}
എന്നിട്ടും ഈ ജനം അവനെക്കൂടാതെ പല ദൈവങ്ങളെയും സങ്കല്‍പിച്ചുണ്ടാക്കി. എന്നാല്‍ അവര്‍ ഒന്നിനെയും സൃഷ്ടിക്കുന്നില്ല. എന്നല്ല, അവര്‍തന്നെ സൃഷ്ടിക്കപ്പെട്ടവരാണ്. തങ്ങള്‍ക്കുതന്നെ എന്തെങ്കിലും ഉപകാരമോ ഉപദ്രവമോ ചെയ്യാനുള്ള കഴിവുപോലും അവര്‍ക്കില്ല. മരിപ്പിക്കാനോ ജീവിപ്പിക്കാനോ ഉയിര്‍ത്തെഴുന്നേല്‍പിക്കാനോ അവര്‍ക്കാവില്ല.
\end{malayalam}}
\flushright{\begin{Arabic}
\quranayah[25][4]
\end{Arabic}}
\flushleft{\begin{malayalam}
സത്യനിഷേധികള്‍ പറയുന്നു: "ഈ ഖുര്‍ആന്‍ ഇയാള്‍ കെട്ടിച്ചമച്ച കള്ളക്കഥയാണ്. അതിലയാളെ ആരൊക്കെയോ സഹായിച്ചിട്ടുണ്ട്.” എന്നാല്‍ അറിയുക: അവരെത്തിപ്പെട്ടത് കടുത്ത അതിക്രമത്തിലാണ്. പറഞ്ഞത് പച്ചക്കള്ളവും.
\end{malayalam}}
\flushright{\begin{Arabic}
\quranayah[25][5]
\end{Arabic}}
\flushleft{\begin{malayalam}
അവര്‍ പറയുന്നു: "ഇത് പൂര്‍വികരുടെ കെട്ടുകഥകളാണ്. ഇയാളിത് പകര്‍ത്തിയെഴുതിയതാണ്. രാവിലെയും വൈകുന്നേരവും ആരോ അതിയാള്‍ക്ക് വായിച്ചുകൊടുക്കുകയാണ്.”
\end{malayalam}}
\flushright{\begin{Arabic}
\quranayah[25][6]
\end{Arabic}}
\flushleft{\begin{malayalam}
പറയുക: "ആകാശഭൂമികളിലെ പരമരഹസ്യങ്ങള്‍പോലും അറിയുന്നവനാണ് ഇത് ഇറക്കിത്തന്നത്.” തീര്‍ച്ചയായും അവന്‍ ഏറെ പൊറുക്കുന്നവനും പരമകാരുണികനുമാണ്.
\end{malayalam}}
\flushright{\begin{Arabic}
\quranayah[25][7]
\end{Arabic}}
\flushleft{\begin{malayalam}
അവര്‍ പറയുന്നു: "ഇതെന്ത് ദൈവദൂതന്‍? ഇയാള്‍ അന്നം തിന്നുന്നു. അങ്ങാടിയിലൂടെ നടക്കുന്നു. ഇയാളോടൊപ്പം മുന്നറിയിപ്പുകാരനായി ഒരു മലക്കിനെ ഇറക്കിക്കൊടുക്കാത്തതെന്ത്?
\end{malayalam}}
\flushright{\begin{Arabic}
\quranayah[25][8]
\end{Arabic}}
\flushleft{\begin{malayalam}
"അല്ലെങ്കില്‍ എന്തുകൊണ്ട് ഇയാള്‍ക്കൊരു നിധി ഇങ്ങ് ഇട്ടുകൊടുക്കുന്നില്ല? അതുമല്ലെങ്കില്‍ എന്തും തിന്നാന്‍കിട്ടുന്ന ഒരു തോട്ടമെങ്കിലും ഇയാള്‍ക്ക് ഉണ്ടാക്കിക്കൊടുത്തുകൂടേ?” ആ അക്രമികള്‍ പറയുന്നു: "മാരണം ബാധിച്ച ഒരുത്തനെയാണ് നിങ്ങള്‍ പിന്‍പറ്റുന്നത്.”
\end{malayalam}}
\flushright{\begin{Arabic}
\quranayah[25][9]
\end{Arabic}}
\flushleft{\begin{malayalam}
നോക്കൂ: എങ്ങനെയൊക്കെയാണ് അവര്‍ നിന്നെ ചിത്രീകരിച്ചുകൊണ്ടിരിക്കുന്നത്? അങ്ങനെ അവര്‍ തീര്‍ത്തും വഴികേടിലായിരിക്കുന്നു. ഒരു വഴിയും കണ്ടെത്താനവര്‍ക്കു കഴിയുന്നില്ല.
\end{malayalam}}
\flushright{\begin{Arabic}
\quranayah[25][10]
\end{Arabic}}
\flushleft{\begin{malayalam}
താനുദ്ദേശിക്കുന്നുവെങ്കില്‍ അവരാവശ്യപ്പെട്ടതിനെക്കാളെല്ലാം മെച്ചപ്പെട്ട പലതും അഥവാ, താഴ്ഭാഗത്തൂടെ ആറുകളൊഴുകുന്ന അനേകം ആരാമങ്ങളും നിരവധി കൊട്ടാരങ്ങളും നിനക്കു നല്‍കാന്‍ കഴിവുറ്റവനാണ് അല്ലാഹു. അവന്‍ അളവറ്റ അനുഗ്രഹങ്ങളുള്ളവനാണ്.
\end{malayalam}}
\flushright{\begin{Arabic}
\quranayah[25][11]
\end{Arabic}}
\flushleft{\begin{malayalam}
എന്നാല്‍ കാര്യമിതാണ്: അന്ത്യസമയത്തെ അവര്‍ തള്ളിപ്പറഞ്ഞിരിക്കുന്നു. അന്ത്യദിനത്തെ തള്ളിപ്പറയുന്നവര്‍ക്ക് നാം കത്തിക്കാളുന്ന നരകത്തീ ഒരുക്കിവെച്ചിരിക്കുന്നു.
\end{malayalam}}
\flushright{\begin{Arabic}
\quranayah[25][12]
\end{Arabic}}
\flushleft{\begin{malayalam}
ദൂരത്തുനിന്നു അതവരെ കാണുമ്പോള്‍ തന്നെ അതിന്റെ ക്ഷോഭവും ഇരമ്പലും അവര്‍ക്ക് കേള്‍ക്കാനാവും.
\end{malayalam}}
\flushright{\begin{Arabic}
\quranayah[25][13]
\end{Arabic}}
\flushleft{\begin{malayalam}
ചങ്ങലകളില്‍ ബന്ധിതരായി നരകത്തിലെ ഇടുങ്ങിയ ഇടത്തേക്ക് എറിയപ്പെട്ടാല്‍ അവരവിടെവച്ച് നശിച്ചുകിട്ടുന്നതിനായി മുറവിളി കൂട്ടും.
\end{malayalam}}
\flushright{\begin{Arabic}
\quranayah[25][14]
\end{Arabic}}
\flushleft{\begin{malayalam}
അപ്പോള്‍ അവരോടു പറയും: "നിങ്ങളിന്ന് ഒരു നാശത്തെയല്ല അനേകം നാശത്തെ വിളിച്ചു കൊള്ളുക.”
\end{malayalam}}
\flushright{\begin{Arabic}
\quranayah[25][15]
\end{Arabic}}
\flushleft{\begin{malayalam}
ചോദിക്കുക: ഇതാണോ നല്ലത്, അതോ ശാശ്വത സ്വര്‍ഗമോ? ഭക്തന്മാര്‍ക്ക് വാഗ്ദാനമായി നല്‍കിയത് അതാണ്. അവര്‍ക്കുള്ള പ്രതിഫലമാണത്. അന്ത്യസങ്കേതവും അതുതന്നെ.
\end{malayalam}}
\flushright{\begin{Arabic}
\quranayah[25][16]
\end{Arabic}}
\flushleft{\begin{malayalam}
അവിടെ അവര്‍ക്ക് അവരാഗ്രഹിക്കുന്നതൊക്കെ കിട്ടും. അവരവിടെ നിത്യവാസികളായിരിക്കും. പൂര്‍ത്തീകരണം തന്റെ ബാധ്യതയായി നിശ്ചയിച്ച് നിന്റെ നാഥന്‍ നല്‍കിയ വാഗ്ദാനമാണിത്.
\end{malayalam}}
\flushright{\begin{Arabic}
\quranayah[25][17]
\end{Arabic}}
\flushleft{\begin{malayalam}
അവരെയും അല്ലാഹുവെക്കൂടാതെ അവര്‍ പൂജിച്ചുവരുന്നവയെയും ഒരുമിച്ചുകൂട്ടുന്ന ദിവസം അല്ലാഹു അവരോട് ചോദിക്കും: "നിങ്ങളാണോ എന്റെ ഈ അടിമകളെ വഴിപിഴപ്പിച്ചത്. അതല്ല; അവര്‍ സ്വയം പിഴച്ചുപോയതോ?”
\end{malayalam}}
\flushright{\begin{Arabic}
\quranayah[25][18]
\end{Arabic}}
\flushleft{\begin{malayalam}
അവര്‍ പറയും: "നീയെത്ര പരിശുദ്ധന്‍! നിന്നെക്കൂടാതെ ഏതെങ്കിലും രക്ഷാധികാരികളെ സ്വീകരിക്കുകയെന്നത് ഞങ്ങള്‍ക്കു ചേര്‍ന്നതല്ല. എന്നാല്‍ നീ അവര്‍ക്കും അവരുടെ പിതാക്കള്‍ക്കും ജീവിതസുഖം നല്‍കി. അങ്ങനെ അവര്‍ ഈ ഉദ്ബോധനം മറന്നുകളഞ്ഞു. അതുവഴി അവരൊരു നശിച്ച ജനതയായി.”
\end{malayalam}}
\flushright{\begin{Arabic}
\quranayah[25][19]
\end{Arabic}}
\flushleft{\begin{malayalam}
അല്ലാഹു പറയും: "നിങ്ങള്‍ പറയുന്നതൊക്കെ അവര്‍ നിഷേധിച്ചു തള്ളിയിരിക്കുന്നു. ഇനി ശിക്ഷയെ തട്ടിമാറ്റാനോ എന്തെങ്കിലും സഹായം നേടാനോ നിങ്ങള്‍ക്കു സാധ്യമല്ല. അതിനാല്‍ നിങ്ങളില്‍നിന്ന് അക്രമം കാണിച്ചവരെ നാം കഠിനശിക്ഷക്കു വിധേയമാക്കും.”
\end{malayalam}}
\flushright{\begin{Arabic}
\quranayah[25][20]
\end{Arabic}}
\flushleft{\begin{malayalam}
ആഹാരംകഴിക്കുകയും അങ്ങാടിയിലൂടെ നടക്കുകയും ചെയ്യുന്നവരല്ലാത്ത ആരെയും നിനക്കുമുമ്പും നാം ദൂതന്മാരായി അയച്ചിട്ടില്ല. യഥാര്‍ഥത്തില്‍ നിങ്ങളില്‍ ചിലരെ മറ്റുചിലര്‍ക്കു നാം പരീക്ഷണമാക്കിയിരിക്കുന്നു. നിങ്ങള്‍ ക്ഷമിക്കുമോ എന്നറിയാന്‍. നിന്റെ നാഥന്‍ എല്ലാം കണ്ടറിയുന്നവനാണ്.
\end{malayalam}}
\flushright{\begin{Arabic}
\quranayah[25][21]
\end{Arabic}}
\flushleft{\begin{malayalam}
നാമുമായി കണ്ടുമുട്ടാന്‍ ആഗ്രഹിക്കാത്തവര്‍ പറഞ്ഞു: "നമുക്ക് മലക്കുകള്‍ ഇറക്കപ്പെടാത്തതെന്ത്? അല്ലെങ്കില്‍ നമ്മുടെ നാഥനെ നാം നേരില്‍ കാണാത്തതെന്ത്?” അവര്‍ സ്വയം പൊങ്ങച്ചം നടിക്കുകയും കടുത്തധിക്കാരം കാട്ടുകയും ചെയ്തിരിക്കുന്നു.
\end{malayalam}}
\flushright{\begin{Arabic}
\quranayah[25][22]
\end{Arabic}}
\flushleft{\begin{malayalam}
മലക്കുകളെ അവര്‍ കാണുംദിനം. അന്ന് കുറ്റവാളികള്‍ക്ക് ശുഭവാര്‍ത്തയൊന്നുമില്ല. അവരിങ്ങനെ പറയും: "കാക്കണേ; തടുക്കണേ!”
\end{malayalam}}
\flushright{\begin{Arabic}
\quranayah[25][23]
\end{Arabic}}
\flushleft{\begin{malayalam}
അപ്പോള്‍ അവര്‍ പ്രവര്‍ത്തിച്ചിരുന്ന കര്‍മങ്ങളുടെ നേരെ നാം തിരിയും. അങ്ങനെ നാമവയെ ചിതറിയ പൊടിപടലങ്ങളാക്കും.
\end{malayalam}}
\flushright{\begin{Arabic}
\quranayah[25][24]
\end{Arabic}}
\flushleft{\begin{malayalam}
സ്വര്‍ഗാവകാശികള്‍ നല്ല വാസസ്ഥലവും ഉത്തമമായ വിശ്രമകേന്ദ്രവും ഉള്ളവരായിരിക്കും.
\end{malayalam}}
\flushright{\begin{Arabic}
\quranayah[25][25]
\end{Arabic}}
\flushleft{\begin{malayalam}
ആകാശം പൊട്ടിപ്പിളര്‍ന്ന് മേഘപടലം പുറത്തുവരികയും മലക്കുകളെ കൂട്ടംകൂട്ടമായി ഇറക്കുകയും ചെയ്യുന്ന ദിവസം.
\end{malayalam}}
\flushright{\begin{Arabic}
\quranayah[25][26]
\end{Arabic}}
\flushleft{\begin{malayalam}
അന്ന് യഥാര്‍ഥ ആധിപത്യം പരമകാരുണികനായ അല്ലാഹുവിനായിരിക്കും. സത്യനിഷേധികള്‍ക്കാണെങ്കില്‍ അത് ഏറെ ക്ളേശകരമായ ദിനമായിരിക്കും.
\end{malayalam}}
\flushright{\begin{Arabic}
\quranayah[25][27]
\end{Arabic}}
\flushleft{\begin{malayalam}
അക്രമിയായ മനുഷ്യന്‍ ഖേദത്താല്‍ കൈ കടിക്കുന്ന ദിനമാണത്. അന്ന് അയാള്‍ പറയും: "ഹാ കഷ്ടം! ഞാന്‍ ദൈവദൂതനോടൊപ്പം അദ്ദേഹത്തിന്റെ മാര്‍ഗമവലംബിച്ചിരുന്നെങ്കില്‍ എത്ര നന്നായേനെ.
\end{malayalam}}
\flushright{\begin{Arabic}
\quranayah[25][28]
\end{Arabic}}
\flushleft{\begin{malayalam}
"എന്റെ നിര്‍ഭാഗ്യം! ഞാന്‍ ഇന്നയാളെ കൂട്ടുകാരനാക്കിയിരുന്നില്ലെങ്കില്‍!
\end{malayalam}}
\flushright{\begin{Arabic}
\quranayah[25][29]
\end{Arabic}}
\flushleft{\begin{malayalam}
"എനിക്ക് ഉദ്ബോധനം വന്നെത്തിയശേഷം അവനെന്നെ അതില്‍നിന്ന് തെറ്റിച്ചുകളഞ്ഞല്ലോ. പിശാച് മനുഷ്യനെ സംബന്ധിച്ചിടത്തോളം കൊടിയ വഞ്ചകന്‍ തന്നെ.”
\end{malayalam}}
\flushright{\begin{Arabic}
\quranayah[25][30]
\end{Arabic}}
\flushleft{\begin{malayalam}
ദൈവദൂതന്‍ അന്ന് പറയും: "നാഥാ, എന്റെ ജനം ഈ ഖുര്‍ആനെ തീര്‍ത്തും നിരാകരിച്ചു.”
\end{malayalam}}
\flushright{\begin{Arabic}
\quranayah[25][31]
\end{Arabic}}
\flushleft{\begin{malayalam}
അവ്വിധം എല്ലാ പ്രവാചകന്മാര്‍ക്കും കുറ്റവാളികളായ ചില ശത്രുക്കളെ നാം ഉണ്ടാക്കിയിരിക്കുന്നു. വഴികാട്ടിയായും സഹായിയായും നിന്റെ നാഥന്‍ തന്നെ മതി.
\end{malayalam}}
\flushright{\begin{Arabic}
\quranayah[25][32]
\end{Arabic}}
\flushleft{\begin{malayalam}
സത്യനിഷേധികള്‍ ചോദിക്കുന്നു: "ഇയാള്‍ക്ക് ഈ ഖുര്‍ആന്‍ മുഴുവനും ഒന്നിച്ച് ഒരേസമയം ഇറക്കിക്കൊടുക്കാത്തതെന്ത്?” എന്നാല്‍ അത് ഇങ്ങനെത്തന്നെയാണ് വേണ്ടത്. നിന്റെ ഹൃദയത്തെ ഉറപ്പിച്ചുനിര്‍ത്താനാണിത്. നാമിത് ഇടവിട്ട് ഇടവിട്ട് പലതവണയായി ഓതിക്കേള്‍പ്പിക്കുന്നു.
\end{malayalam}}
\flushright{\begin{Arabic}
\quranayah[25][33]
\end{Arabic}}
\flushleft{\begin{malayalam}
അവര്‍ ഏതൊരു പ്രശ്നവുമായി നിന്നെ സമീപിക്കുകയാണെങ്കിലും അവയ്ക്കെല്ലാം ശക്തമായ ന്യായവും വ്യക്തമായ വിശദീകരണവും നിനക്കു നാമെത്തിച്ചുതരാതിരിക്കില്ല.
\end{malayalam}}
\flushright{\begin{Arabic}
\quranayah[25][34]
\end{Arabic}}
\flushleft{\begin{malayalam}
മുഖംകുത്തി നരകത്തീയിലേക്കു തള്ളപ്പെടുന്നവരാണ് ഏറ്റം നീചാവസ്ഥയിലുള്ളവര്‍. അങ്ങേയറ്റം പിഴച്ചവരും അവര്‍ തന്നെ.
\end{malayalam}}
\flushright{\begin{Arabic}
\quranayah[25][35]
\end{Arabic}}
\flushleft{\begin{malayalam}
മൂസാക്കു നാം വേദപുസ്തകം നല്‍കി. അദ്ദേഹത്തോടൊപ്പം സഹോദരന്‍ ഹാറൂനെ സഹായിയായി നിശ്ചയിച്ചു.
\end{malayalam}}
\flushright{\begin{Arabic}
\quranayah[25][36]
\end{Arabic}}
\flushleft{\begin{malayalam}
എന്നിട്ടു നാം പറഞ്ഞു: "നിങ്ങളിരുവരും പോകൂ. നമ്മുടെ വചനങ്ങളെ തള്ളിപ്പറഞ്ഞ ജനത്തിന്റെ അടുത്തേക്ക്.” അങ്ങനെ നാം അവരെ തകര്‍ത്തുതരിപ്പണമാക്കി.
\end{malayalam}}
\flushright{\begin{Arabic}
\quranayah[25][37]
\end{Arabic}}
\flushleft{\begin{malayalam}
നൂഹിന്റെ ജനതയെയും നാം അതുതന്നെ ചെയ്തു. അവര്‍ ദൈവദൂതന്മാരെ കള്ളമാക്കി തള്ളി. അപ്പോള്‍ നാം അവരെ മുക്കിക്കൊന്നു. അങ്ങനെ നാം അവരെ ജനങ്ങള്‍ക്ക് ഒരു ദൃഷ്ടാന്തമാക്കി. അക്രമികള്‍ക്കു നാം നോവേറിയശിക്ഷ ഒരുക്കിവെച്ചിട്ടുണ്ട്.
\end{malayalam}}
\flushright{\begin{Arabic}
\quranayah[25][38]
\end{Arabic}}
\flushleft{\begin{malayalam}
ആദ്, സമൂദ്, റസ്സുകാര്‍, അതിനിടയിലെ നിരവധി തലമുറകള്‍, എല്ലാവരെയും നാം നശിപ്പിച്ചു.
\end{malayalam}}
\flushright{\begin{Arabic}
\quranayah[25][39]
\end{Arabic}}
\flushleft{\begin{malayalam}
എല്ലാവര്‍ക്കും നാം മുന്‍ഗാമികളുടെ ജീവിതാനുഭവങ്ങള്‍ വിവരിച്ചുകൊടുത്തിരുന്നു. അവസാനം അവരെയൊക്കെ നാം തകര്‍ത്ത് തരിപ്പണമാക്കി.
\end{malayalam}}
\flushright{\begin{Arabic}
\quranayah[25][40]
\end{Arabic}}
\flushleft{\begin{malayalam}
വിപത്തിന്റെ മഴ പെയ്തിറങ്ങിയ ആ നാട്ടിലൂടെയും ഇവര്‍ കടന്നുപോയിട്ടുണ്ട്. എന്നിട്ടും ഇവരിതൊന്നും കണ്ടിട്ടില്ലേ? യഥാര്‍ഥത്തിലിവര്‍ ഉയിര്‍ത്തെഴുന്നേല്‍പ് ഒട്ടും പ്രതീക്ഷിക്കാത്തവരായിരുന്നു.
\end{malayalam}}
\flushright{\begin{Arabic}
\quranayah[25][41]
\end{Arabic}}
\flushleft{\begin{malayalam}
നിന്നെ കാണുമ്പോഴെല്ലാം ഇക്കൂട്ടര്‍ നിന്നെ പുച്ഛിക്കുകയാണല്ലോ. അവര്‍ ചോദിക്കുന്നു: "ഇയാളെയാണോ ദൈവം തന്റെ ദൂതനായി അയച്ചത്?
\end{malayalam}}
\flushright{\begin{Arabic}
\quranayah[25][42]
\end{Arabic}}
\flushleft{\begin{malayalam}
"നമ്മുടെ ദൈവങ്ങളിലെ വിശ്വാസത്തില്‍ നാം ക്ഷമയോടെ ഉറച്ചുനിന്നിരുന്നില്ലെങ്കില്‍ അവയില്‍നിന്ന് ഇവന്‍ നമ്മെ തെറ്റിച്ചുകളയുമായിരുന്നു.” എന്നാല്‍ ശിക്ഷ നേരില്‍ കാണുംനേരം അവര്‍ തിരിച്ചറിയും, ഏറ്റം വഴിപിഴച്ചവര്‍ ആരെന്ന്.
\end{malayalam}}
\flushright{\begin{Arabic}
\quranayah[25][43]
\end{Arabic}}
\flushleft{\begin{malayalam}
തന്റെ ദേഹേച്ഛയെ ദൈവമാക്കിയവനെ നീ കണ്ടോ? എന്നിട്ടും അവനെ നേര്‍വഴിയിലാക്കുന്ന ബാധ്യത നീ ഏല്‍ക്കുകയോ?
\end{malayalam}}
\flushright{\begin{Arabic}
\quranayah[25][44]
\end{Arabic}}
\flushleft{\begin{malayalam}
അല്ല, നീ കരുതുന്നുണ്ടോ; അവരിലേറെപ്പേരും കേള്‍ക്കുകയും ചിന്തിക്കുകയും ചെയ്യുന്നുവെന്ന്. എന്നാലവര്‍ കന്നുകാലികളെപ്പോലെയാണ്. അല്ല; അവയെക്കാളും പിഴച്ചവരാണ്.
\end{malayalam}}
\flushright{\begin{Arabic}
\quranayah[25][45]
\end{Arabic}}
\flushleft{\begin{malayalam}
നിന്റെ നാഥനെക്കുറിച്ച് നീ ആലോചിച്ചുനോക്കിയിട്ടില്ലേ? എങ്ങനെയാണവന്‍ നിഴലിനെ നീട്ടിയിട്ടുകൊണ്ടിരിക്കുന്നതെന്ന്. അവനിച്ഛിച്ചിരുന്നെങ്കില്‍ അതിനെ അവന്‍ ഒരേ സ്ഥലത്തുതന്നെ നിശ്ചലമാക്കുമായിരുന്നു. സൂര്യനെ നാം നിഴലിന് വഴികാട്ടിയാക്കി.
\end{malayalam}}
\flushright{\begin{Arabic}
\quranayah[25][46]
\end{Arabic}}
\flushleft{\begin{malayalam}
പിന്നെ നാം ആ നിഴലിനെ അല്‍പാല്‍പമായി നമ്മുടെ അടുത്തേക്ക് ചുരുക്കിക്കൊണ്ടുവരുന്നു.
\end{malayalam}}
\flushright{\begin{Arabic}
\quranayah[25][47]
\end{Arabic}}
\flushleft{\begin{malayalam}
അവനാണ് നിങ്ങള്‍ക്ക് രാവിനെ വസ്ത്രമാക്കിയത്. ഉറക്കത്തെ വിശ്രമാവസരവും പകലിനെ ഉണര്‍വുവേളയുമാക്കിയതും അവന്‍ തന്നെ.
\end{malayalam}}
\flushright{\begin{Arabic}
\quranayah[25][48]
\end{Arabic}}
\flushleft{\begin{malayalam}
തന്റെ അനുഗ്രഹത്തിന്റെ മുന്നോടിയായി ശുഭസൂചനയോടെ കാറ്റുകളെ അയച്ചവനും അവനാണ്. മാനത്തുനിന്നു നാം ശുദ്ധമായ വെള്ളം വീഴ്ത്തി.
\end{malayalam}}
\flushright{\begin{Arabic}
\quranayah[25][49]
\end{Arabic}}
\flushleft{\begin{malayalam}
അതുവഴി ചത്ത നാടിനെ ചൈതന്യമുള്ളതാക്കാന്‍. നാം സൃഷ്ടിച്ച ഒട്ടുവളരെ കന്നുകാലികളെയും മനുഷ്യരെയും കുടിപ്പിക്കാനും.
\end{malayalam}}
\flushright{\begin{Arabic}
\quranayah[25][50]
\end{Arabic}}
\flushleft{\begin{malayalam}
നാം ആ മഴവെള്ളത്തെ അവര്‍ക്കിടയില്‍ വിതരണം ചെയ്തു. അവര്‍ ചിന്തിച്ചറിയാന്‍. എന്നാല്‍ ജനങ്ങളിലേറെപ്പേരും നന്ദികേടു കാട്ടാനാണ് ഒരുമ്പെട്ടത്.
\end{malayalam}}
\flushright{\begin{Arabic}
\quranayah[25][51]
\end{Arabic}}
\flushleft{\begin{malayalam}
നാം ഇച്ഛിച്ചിരുന്നുവെങ്കില്‍ എല്ലാ ഓരോ നാട്ടിലും നാം ഓരോ താക്കീതുകാരനെ നിയോഗിക്കുമായിരുന്നു.
\end{malayalam}}
\flushright{\begin{Arabic}
\quranayah[25][52]
\end{Arabic}}
\flushleft{\begin{malayalam}
അതിനാല്‍ നീ സത്യനിഷേധികളെ അനുസരിക്കരുത്. ഈ ഖുര്‍ആനുപയോഗിച്ച് നീ അവരോട് ശക്തമായി സമരം ചെയ്യുക.
\end{malayalam}}
\flushright{\begin{Arabic}
\quranayah[25][53]
\end{Arabic}}
\flushleft{\begin{malayalam}
രണ്ടു സമുദ്രങ്ങളെ സംയോജിപ്പിച്ചതും അവനാണ്. ഒന്നില്‍ ശുദ്ധമായ തെളിനീരാണ്. രണ്ടാമത്തേതില്‍ ചവര്‍പ്പുള്ള ഉപ്പുവെള്ളവും. അവ രണ്ടിനുമിടയില്‍ അവനൊരു മറയുണ്ടാക്കിയിരിക്കുന്നു. ശക്തമായ തടസ്സവും.
\end{malayalam}}
\flushright{\begin{Arabic}
\quranayah[25][54]
\end{Arabic}}
\flushleft{\begin{malayalam}
വെള്ളത്തില്‍നിന്ന് മനുഷ്യനെ സൃഷ്ടിച്ചവനും അവനാണ്. അങ്ങനെ അവനെ രക്തബന്ധവും വിവാഹബന്ധവുമുള്ളവനാക്കി. നിന്റെ നാഥന്‍ എല്ലാറ്റിനും കഴിവുറ്റവനാണ്.
\end{malayalam}}
\flushright{\begin{Arabic}
\quranayah[25][55]
\end{Arabic}}
\flushleft{\begin{malayalam}
അല്ലാഹുവെക്കൂടാതെ, തങ്ങള്‍ക്ക് ഗുണമോ ദോഷമോ ചെയ്യാത്ത പലതിനെയും അവര്‍ പൂജിച്ചുകൊണ്ടിരിക്കുന്നു. സത്യനിഷേധി തന്റെ നാഥനെതിരെ എല്ലാ ദുശ്ശക്തികളെയും സഹായിക്കുന്നവനാണ്.
\end{malayalam}}
\flushright{\begin{Arabic}
\quranayah[25][56]
\end{Arabic}}
\flushleft{\begin{malayalam}
ശുഭവാര്‍ത്ത അറിയിക്കുന്നവനും മുന്നറിയിപ്പുനല്‍കുന്നവനുമായല്ലാതെ നിന്നെ നാം അയച്ചിട്ടില്ല.
\end{malayalam}}
\flushright{\begin{Arabic}
\quranayah[25][57]
\end{Arabic}}
\flushleft{\begin{malayalam}
പറയുക: "ഇതിന്റെ പേരില്‍ ഞാന്‍ നിങ്ങളോട് ഒരു പ്രതിഫലവും ആവശ്യപ്പെടുന്നില്ല. ആരെങ്കിലും തന്റെ നാഥനിലേക്കുള്ള വഴിയവലംബിക്കാനുദ്ദേശിക്കുന്നുവെങ്കില്‍ അങ്ങനെ ചെയ്തുകൊള്ളട്ടെയെന്നുമാത്രം.”
\end{malayalam}}
\flushright{\begin{Arabic}
\quranayah[25][58]
\end{Arabic}}
\flushleft{\begin{malayalam}
എന്നെന്നും ജീവിച്ചിരിക്കുന്നവനാണ് അല്ലാഹു. ഒരിക്കലും മരിക്കാത്തവനും. അവനില്‍ ഭരമേല്‍പിക്കുക. അവന്റെ വിശുദ്ധി വാഴ്ത്തുക. അവനെ കീര്‍ത്തിക്കുക. തന്റെ ദാസന്മാരുടെ പാപങ്ങളെപ്പറ്റി സൂക്ഷ്മമായി അറിയുന്നവനായി അവന്‍ തന്നെ മതി.
\end{malayalam}}
\flushright{\begin{Arabic}
\quranayah[25][59]
\end{Arabic}}
\flushleft{\begin{malayalam}
ആകാശഭൂമികളെയും അവയ്ക്കിടയിലുള്ളവയെയും ആറുദിനംകൊണ്ട് സൃഷ്ടിച്ചവനാണവന്‍. പിന്നെയവന്‍ സിംഹാസനസ്ഥനായി. പരമകാരുണികനാണവന്‍. അവനെപ്പറ്റി സൂക്ഷ്മജ്ഞാനമുള്ളവരോടു ചോദിച്ചുനോക്കൂ.
\end{malayalam}}
\flushright{\begin{Arabic}
\quranayah[25][60]
\end{Arabic}}
\flushleft{\begin{malayalam}
ആ പരമകാരുണികനെ സാഷ്ടാംഗം പ്രണമിക്കൂ എന്ന് അവരോട് പറഞ്ഞാല്‍ അവര്‍ ചോദിക്കും: "എന്താണീ പരമകാരുണികനെന്നു പറഞ്ഞാല്‍? നീ പറയുന്നവരെയൊക്കെ ഞങ്ങള്‍ സാഷ്ടാംഗം പ്രണമിക്കണമെന്നോ?” അങ്ങനെ സത്യപ്രബോധനം അവരുടെ അകല്‍ച്ചയും വെറുപ്പും വര്‍ധിപ്പിക്കുകയാണുണ്ടായത്.
\end{malayalam}}
\flushright{\begin{Arabic}
\quranayah[25][61]
\end{Arabic}}
\flushleft{\begin{malayalam}
ആകാശത്ത് നക്ഷത്രപഥങ്ങളുണ്ടാക്കിയവന്‍ ഏറെ അനുഗ്രഹമുള്ളവന്‍ തന്നെ. അതിലവന്‍ ജ്വലിക്കുന്ന വിളക്ക് സ്ഥാപിച്ചിരിക്കുന്നു. പ്രകാശിക്കുന്ന ചന്ദ്രനും.
\end{malayalam}}
\flushright{\begin{Arabic}
\quranayah[25][62]
\end{Arabic}}
\flushleft{\begin{malayalam}
രാപകലുകള്‍ മാറിമാറിവരുംവിധമാക്കിയവനും അവന്‍ തന്നെ. ചിന്തിച്ചറിയാനോ നന്ദി കാണിക്കാനോ ആഗ്രഹിക്കുന്നവര്‍ക്കുവേണ്ടിയാണ് ഇതൊക്കെയും ഒരുക്കിയത്.
\end{malayalam}}
\flushright{\begin{Arabic}
\quranayah[25][63]
\end{Arabic}}
\flushleft{\begin{malayalam}
പരമകാരുണികനായ അല്ലാഹുവിന്റെ ദാസന്മാര്‍ ഭൂമിയില്‍ വിനയത്തോടെ നടക്കുന്നവരാണ്. അവിവേകികള്‍ വാദകോലാഹലത്തിനുവന്നാല്‍ “നിങ്ങള്‍ക്കു സമാധാനം” എന്നുമാത്രം പറഞ്ഞൊഴിയുന്നവരാണവര്‍;
\end{malayalam}}
\flushright{\begin{Arabic}
\quranayah[25][64]
\end{Arabic}}
\flushleft{\begin{malayalam}
സാഷ്ടാംഗം പ്രണമിച്ചും നിന്ന് പ്രാര്‍ഥിച്ചും തങ്ങളുടെ നാഥന്റെ മുമ്പില്‍ രാത്രി കഴിച്ചുകൂട്ടുന്നവരും.
\end{malayalam}}
\flushright{\begin{Arabic}
\quranayah[25][65]
\end{Arabic}}
\flushleft{\begin{malayalam}
അവരിങ്ങനെ പ്രാര്‍ഥിക്കുന്നു: "ഞങ്ങളുടെ നാഥാ, ഞങ്ങളില്‍നിന്ന് നീ നരകശിക്ഷയെ തട്ടിനീക്കേണമേ തീര്‍ച്ചയായും അതിന്റെ ശിക്ഷ വിട്ടൊഴിയാത്തതുതന്നെ.”
\end{malayalam}}
\flushright{\begin{Arabic}
\quranayah[25][66]
\end{Arabic}}
\flushleft{\begin{malayalam}
അത് ഏറ്റം ചീത്തയായ താവളവും മോശമായ പാര്‍പ്പിടവുമത്രെ.
\end{malayalam}}
\flushright{\begin{Arabic}
\quranayah[25][67]
\end{Arabic}}
\flushleft{\begin{malayalam}
ചെലവഴിക്കുമ്പോള്‍ അവര്‍ പരിധിവിടുകയില്ല. പിശുക്കുകാട്ടുകയുമില്ല. രണ്ടിനുമിടയ്ക്ക് മിതമാര്‍ഗം സ്വീകരിക്കുന്നവരാണവര്‍.
\end{malayalam}}
\flushright{\begin{Arabic}
\quranayah[25][68]
\end{Arabic}}
\flushleft{\begin{malayalam}
അല്ലാഹുവോടൊപ്പം മറ്റു ദൈവങ്ങളെ വിളിച്ചുപ്രാര്‍ഥിക്കാത്തവരുമാണവര്‍. അല്ലാഹു ആദരണീയമാക്കിയ ജീവനെ ന്യായമായ കാരണത്താലല്ലാതെ ഹനിക്കാത്തവരും. വ്യഭിചരിക്കാത്തവരുമാണ്. ഇക്കാര്യങ്ങള്‍ ആരെങ്കിലും ചെയ്യുകയാണെങ്കില്‍ അവന്‍ അതിന്റെ പാപഫലം അനുഭവിക്കുകതന്നെ ചെയ്യും.
\end{malayalam}}
\flushright{\begin{Arabic}
\quranayah[25][69]
\end{Arabic}}
\flushleft{\begin{malayalam}
ഉയിര്‍ത്തെഴുന്നേല്‍പുനാളില്‍ അവന് ഇരട്ടി ശിക്ഷ കിട്ടും. അവനതില്‍ നിന്ദിതനായി എന്നെന്നും കഴിയേണ്ടിവരും.
\end{malayalam}}
\flushright{\begin{Arabic}
\quranayah[25][70]
\end{Arabic}}
\flushleft{\begin{malayalam}
പശ്ചാത്തപിക്കുകയും സത്യവിശ്വാസം സ്വീകരിക്കുകയും സല്‍ക്കര്‍മം പ്രവര്‍ത്തിക്കുകയും ചെയ്തവരൊഴികെ. അത്തരക്കാരുടെ തിന്മകള്‍ അല്ലാഹു നന്മകളാക്കി മാറ്റും. അല്ലാഹു ഏറെ പൊറുക്കുന്നവനും പരമകാരുണികനുമാണ്.
\end{malayalam}}
\flushright{\begin{Arabic}
\quranayah[25][71]
\end{Arabic}}
\flushleft{\begin{malayalam}
ആരെങ്കിലും പശ്ചാത്തപിക്കുകയും സല്‍ക്കര്‍മം പ്രവര്‍ത്തിക്കുകയുമാണെങ്കില്‍ അവന്‍ അല്ലാഹുവിങ്കലേക്ക് യഥാവിധി മടങ്ങിച്ചെല്ലുകയാണ് ചെയ്യുന്നത്.
\end{malayalam}}
\flushright{\begin{Arabic}
\quranayah[25][72]
\end{Arabic}}
\flushleft{\begin{malayalam}
കള്ളസാക്ഷ്യം പറയാത്തവരാണവര്‍. അനാവശ്യം നടക്കുന്നിടത്തൂടെ പോകേണ്ടിവന്നാല്‍ അതിലിടപെടാതെ മാന്യമായി കടന്നുപോകുന്നവരും
\end{malayalam}}
\flushright{\begin{Arabic}
\quranayah[25][73]
\end{Arabic}}
\flushleft{\begin{malayalam}
തങ്ങളുടെ നാഥന്റെ വചനങ്ങളിലൂടെ ഉദ്ബോധനം നല്‍കിയാല്‍ ബധിരരും അന്ധരുമായി അതിന്മേല്‍ വീഴാത്തവരും.
\end{malayalam}}
\flushright{\begin{Arabic}
\quranayah[25][74]
\end{Arabic}}
\flushleft{\begin{malayalam}
അവരിങ്ങനെ പ്രാര്‍ഥിക്കുന്നവരുമാണ്: "ഞങ്ങളുടെ നാഥാ, ഞങ്ങളുടെ ഇണകളില്‍നിന്നും സന്തതികളില്‍നിന്നും ഞങ്ങള്‍ക്കു നീ കണ്‍കുളിര്‍മ നല്‍കേണമേ. ഭക്തിപുലര്‍ത്തുന്നവര്‍ക്ക് ഞങ്ങളെ നീ മാതൃകയാക്കേണമേ.”
\end{malayalam}}
\flushright{\begin{Arabic}
\quranayah[25][75]
\end{Arabic}}
\flushleft{\begin{malayalam}
അത്തരക്കാര്‍ക്ക് തങ്ങള്‍ ക്ഷമിച്ചതിന്റെ പേരില്‍ സ്വര്‍ഗത്തിലെ ഉന്നതസ്ഥാനങ്ങള്‍ പ്രതിഫലമായി നല്‍കും. അഭിവാദ്യത്തോടെയും സമാധാനാശംസകളോടെയുമാണ് അവരെയവിടെ സ്വീകരിക്കുക.
\end{malayalam}}
\flushright{\begin{Arabic}
\quranayah[25][76]
\end{Arabic}}
\flushleft{\begin{malayalam}
അവരവിടെ നിത്യവാസികളായിരിക്കും. എത്ര നല്ല താവളം! എത്ര നല്ല വാസസ്ഥലം!
\end{malayalam}}
\flushright{\begin{Arabic}
\quranayah[25][77]
\end{Arabic}}
\flushleft{\begin{malayalam}
പറയുക: നിങ്ങളുടെ പ്രാര്‍ഥനയില്ലെങ്കില്‍ എന്റെ നാഥന്‍ നിങ്ങളെ ഒട്ടും പരിഗണിക്കുകയില്ല. നിങ്ങള്‍ അവനെ നിഷേധിച്ചുതള്ളിയിരിക്കയാണല്ലോ. അതിനാല്‍ അതിനുള്ള ശിക്ഷ അടുത്തുതന്നെ അനിവാര്യമായും ഉണ്ടാകും.
\end{malayalam}}
\chapter{\textmalayalam{ശുഅറാ ( കവികള്‍ )}}
\begin{Arabic}
\Huge{\centerline{\basmalah}}\end{Arabic}
\flushright{\begin{Arabic}
\quranayah[26][1]
\end{Arabic}}
\flushleft{\begin{malayalam}
ത്വാ-സീന്‍-മീം.
\end{malayalam}}
\flushright{\begin{Arabic}
\quranayah[26][2]
\end{Arabic}}
\flushleft{\begin{malayalam}
ഇത് സുവ്യക്തമായ വേദപുസ്തകത്തിലെ വചനങ്ങളാണ്.
\end{malayalam}}
\flushright{\begin{Arabic}
\quranayah[26][3]
\end{Arabic}}
\flushleft{\begin{malayalam}
അവര്‍ വിശ്വാസികളായില്ലല്ലോ എന്നോര്‍ത്ത് ദുഃഖിതനായി നീ നിന്റെ ജീവനൊടുക്കിയേക്കാം.
\end{malayalam}}
\flushright{\begin{Arabic}
\quranayah[26][4]
\end{Arabic}}
\flushleft{\begin{malayalam}
നാം ഇച്ഛിക്കുകയാണെങ്കില്‍ അവര്‍ക്കു നാം മാനത്തുനിന്ന് ഒരു ദൃഷ്ടാന്തം ഇറക്കിക്കൊടുക്കും. അപ്പോള്‍ അവരുടെ പിരടികള്‍ അതിന് വിധേയമായിത്തീരും.
\end{malayalam}}
\flushright{\begin{Arabic}
\quranayah[26][5]
\end{Arabic}}
\flushleft{\begin{malayalam}
പരമകാരുണികനായ അല്ലാഹുവില്‍നിന്ന് പുതുതായി ഏതൊരു ഉദ്ബോധനം വന്നെത്തുമ്പോഴും അവരതിനെ അപ്പാടെ അവഗണിക്കുകയാണ്.
\end{malayalam}}
\flushright{\begin{Arabic}
\quranayah[26][6]
\end{Arabic}}
\flushleft{\begin{malayalam}
ഇപ്പോഴവര്‍ തള്ളിപ്പറഞ്ഞിരിക്കുന്നു. എന്നാല്‍ അവര്‍ പുച്ഛിച്ചുതള്ളിക്കളയുന്നതിന്റെ നിജസ്ഥിതി വൈകാതെ തന്നെ അവര്‍ക്കു വന്നെത്തും.
\end{malayalam}}
\flushright{\begin{Arabic}
\quranayah[26][7]
\end{Arabic}}
\flushleft{\begin{malayalam}
അവര്‍ ഭൂമിയിലേക്കു നോക്കുന്നില്ലേ? എത്രയേറെ വൈവിധ്യപൂര്‍ണമായ നല്ലയിനം സസ്യങ്ങളെയാണ് നാമതില്‍ മുളപ്പിച്ചിരിക്കുന്നത്.
\end{malayalam}}
\flushright{\begin{Arabic}
\quranayah[26][8]
\end{Arabic}}
\flushleft{\begin{malayalam}
തീര്‍ച്ചയായും അതിലൊരു ദൃഷ്ടാന്തമുണ്ട്. എന്നിട്ടും അവരിലേറെ പേരും വിശ്വാസികളായില്ല.
\end{malayalam}}
\flushright{\begin{Arabic}
\quranayah[26][9]
\end{Arabic}}
\flushleft{\begin{malayalam}
നിന്റെ നാഥന്‍ തന്നെയാണ് പ്രതാപിയും പരമകാരുണികനും.
\end{malayalam}}
\flushright{\begin{Arabic}
\quranayah[26][10]
\end{Arabic}}
\flushleft{\begin{malayalam}
നിന്റെ നാഥന്‍ മൂസായെ വിളിച്ചുപറഞ്ഞ സന്ദര്‍ഭം: "നീ അക്രമികളായ ആ ജനങ്ങളിലേക്കു പോവുക.
\end{malayalam}}
\flushright{\begin{Arabic}
\quranayah[26][11]
\end{Arabic}}
\flushleft{\begin{malayalam}
"ഫറവോന്റെ ജനത്തിലേക്ക്; എന്നിട്ട് ചോദിക്കൂ, അവര്‍ ഭക്തിപുലര്‍ത്തുന്നില്ലേയെന്ന്.”
\end{malayalam}}
\flushright{\begin{Arabic}
\quranayah[26][12]
\end{Arabic}}
\flushleft{\begin{malayalam}
മൂസ പറഞ്ഞു: "എന്റെ നാഥാ, അവരെന്നെ തള്ളിപ്പറയുമെന്ന് ഞാന്‍ ഭയപ്പെടുന്നു.
\end{malayalam}}
\flushright{\begin{Arabic}
\quranayah[26][13]
\end{Arabic}}
\flushleft{\begin{malayalam}
"എന്റെ ഹൃദയം ഞെരുങ്ങിപ്പോകുന്നു. എന്റെ നാവിന് സംസാരവൈഭവമില്ല. അതിനാല്‍ നീ ഹാറൂന്ന് സന്ദേശമയച്ചാലും.
\end{malayalam}}
\flushright{\begin{Arabic}
\quranayah[26][14]
\end{Arabic}}
\flushleft{\begin{malayalam}
"അവര്‍ക്കാണെങ്കില്‍ എന്റെ പേരില്‍ ഒരു കുറ്റാരോപണവുമുണ്ട്. അതിനാലവരെന്നെ കൊന്നുകളയുമോയെന്ന് ഞാന്‍ ഭയപ്പെടുന്നു.”
\end{malayalam}}
\flushright{\begin{Arabic}
\quranayah[26][15]
\end{Arabic}}
\flushleft{\begin{malayalam}
അല്ലാഹു പറഞ്ഞു: "ഒരിക്കലുമില്ല. അതിനാല്‍ നിങ്ങളിരുവരും നമ്മുടെ ദൃഷ്ടാന്തങ്ങളുമായി പോവുക. തീര്‍ച്ചയായും നിങ്ങളോടൊപ്പം എല്ലാം കേള്‍ക്കുന്നവനായി നാമുണ്ട്.
\end{malayalam}}
\flushright{\begin{Arabic}
\quranayah[26][16]
\end{Arabic}}
\flushleft{\begin{malayalam}
"അങ്ങനെ നിങ്ങളിരുവരും ഫറവോന്റെ അടുത്തുചെന്ന് പറയുക: “തീര്‍ച്ചയായും ഞങ്ങള്‍ പ്രപഞ്ചനാഥന്റെ ദൂതന്മാരാണ്.
\end{malayalam}}
\flushright{\begin{Arabic}
\quranayah[26][17]
\end{Arabic}}
\flushleft{\begin{malayalam}
“ഇസ്രയേല്‍ മക്കളെ ഞങ്ങളോടൊപ്പമയക്കണമെന്നതാണ് ദൈവശാസന.”
\end{malayalam}}
\flushright{\begin{Arabic}
\quranayah[26][18]
\end{Arabic}}
\flushleft{\begin{malayalam}
ഫറവോന്‍ പറഞ്ഞു: "കുട്ടിയായിരിക്കെ ഞങ്ങള്‍ നിന്നെ ഞങ്ങളോടൊപ്പം വളര്‍ത്തിയില്ലേ? നിന്റെ ആയുസ്സില്‍ കുറേകാലം ഞങ്ങളോടൊപ്പമാണല്ലോ നീ കഴിച്ചുകൂട്ടിയത്.
\end{malayalam}}
\flushright{\begin{Arabic}
\quranayah[26][19]
\end{Arabic}}
\flushleft{\begin{malayalam}
"പിന്നെ നീ ചെയ്ത ആ കൃത്യം നീ ചെയ്തിട്ടുമുണ്ട്. നീ തീരേ നന്ദികെട്ടവന്‍ തന്നെ.”
\end{malayalam}}
\flushright{\begin{Arabic}
\quranayah[26][20]
\end{Arabic}}
\flushleft{\begin{malayalam}
മൂസ പറഞ്ഞു: "അന്ന് ഞാനതു അറിവില്ലായ്മയാല്‍ ചെയ്തതായിരുന്നു.
\end{malayalam}}
\flushright{\begin{Arabic}
\quranayah[26][21]
\end{Arabic}}
\flushleft{\begin{malayalam}
"അങ്ങനെ നിങ്ങളെപ്പറ്റി പേടി തോന്നിയപ്പോള്‍ ഞാനിവിടെ നിന്ന് ഒളിച്ചോടി. പിന്നീട് എന്റെ നാഥന്‍ എനിക്ക് തത്ത്വജ്ഞാനം നല്‍കി. അവനെന്നെ തന്റെ ദൂതന്മാരിലൊരുവനാക്കി.
\end{malayalam}}
\flushright{\begin{Arabic}
\quranayah[26][22]
\end{Arabic}}
\flushleft{\begin{malayalam}
"എനിക്കു ചെയ്തുതന്നതായി നീ എടുത്തുകാണിച്ച ആ അനുഗ്രഹം ഇസ്രയേല്‍ മക്കളെ നീ അടിമകളാക്കിവെച്ചതിനാല്‍ സംഭവിച്ചതാണ്.”
\end{malayalam}}
\flushright{\begin{Arabic}
\quranayah[26][23]
\end{Arabic}}
\flushleft{\begin{malayalam}
ഫറവോന്‍ ചോദിച്ചു: "എന്താണ് ഈ ലോകരക്ഷിതാവെന്നത്?”
\end{malayalam}}
\flushright{\begin{Arabic}
\quranayah[26][24]
\end{Arabic}}
\flushleft{\begin{malayalam}
മൂസ പറഞ്ഞു: "ആകാശഭൂമികളുടെയും അവയ്ക്കിടയിലുള്ളവയുടെയും സംരക്ഷകന്‍ തന്നെ. നിങ്ങള്‍ കാര്യം മനസ്സിലാകുന്നവരാണെങ്കില്‍ ഇതുബോധ്യമാകും.”
\end{malayalam}}
\flushright{\begin{Arabic}
\quranayah[26][25]
\end{Arabic}}
\flushleft{\begin{malayalam}
ഫറവോന്‍ തന്റെ ചുറ്റുമുള്ളവരോട് ചോദിച്ചു: "നിങ്ങള്‍ കേള്‍ക്കുന്നില്ലേ?”
\end{malayalam}}
\flushright{\begin{Arabic}
\quranayah[26][26]
\end{Arabic}}
\flushleft{\begin{malayalam}
മൂസ പറഞ്ഞു: "നിങ്ങളുടെ രക്ഷിതാവാണത്. നിങ്ങളുടെ പൂര്‍വപിതാക്കളുടെയും രക്ഷിതാവ്.”
\end{malayalam}}
\flushright{\begin{Arabic}
\quranayah[26][27]
\end{Arabic}}
\flushleft{\begin{malayalam}
ഫറവോന്‍ പറഞ്ഞു: "നിങ്ങളിലേക്ക് അയക്കപ്പെട്ട നിങ്ങളുടെ ഈ ദൈവദൂതന്‍ ഒരു മുഴുഭ്രാന്തന്‍ തന്നെ; സംശയം വേണ്ടാ.”
\end{malayalam}}
\flushright{\begin{Arabic}
\quranayah[26][28]
\end{Arabic}}
\flushleft{\begin{malayalam}
മൂസ പറഞ്ഞു: "ഉദയ സ്ഥാനത്തിന്റെയും അസ്തമയ സ്ഥാനത്തിന്റെയും അവയ്ക്കിടയിലുള്ളവയുടെയും രക്ഷിതാവാണവന്‍. നിങ്ങള്‍ ചിന്തിച്ചറിയുന്നവരെങ്കില്‍ ഇത് മനസ്സിലാകും.”
\end{malayalam}}
\flushright{\begin{Arabic}
\quranayah[26][29]
\end{Arabic}}
\flushleft{\begin{malayalam}
ഫറവോന്‍ പറഞ്ഞു: "ഞാനല്ലാത്ത ഒരു ദൈവത്തെ നീ സ്വീകരിക്കുകയാണെങ്കില്‍ നിശ്ചയമായും നിന്നെ ഞാന്‍ ജയിലിലടക്കും.”
\end{malayalam}}
\flushright{\begin{Arabic}
\quranayah[26][30]
\end{Arabic}}
\flushleft{\begin{malayalam}
മൂസ ചോദിച്ചു: "ഞാന്‍ താങ്കളുടെയടുത്ത് വ്യക്തമായ വല്ല തെളിവും കൊണ്ടുവന്നാലും?”
\end{malayalam}}
\flushright{\begin{Arabic}
\quranayah[26][31]
\end{Arabic}}
\flushleft{\begin{malayalam}
ഫറവോന്‍ പറഞ്ഞു: "എങ്കില്‍ നീ അതിങ്ങുകൊണ്ടുവരിക. നീ സത്യവാനെങ്കില്‍!”
\end{malayalam}}
\flushright{\begin{Arabic}
\quranayah[26][32]
\end{Arabic}}
\flushleft{\begin{malayalam}
അപ്പോള്‍ മൂസ തന്റെ വടി താഴെയിട്ടു. ഉടനെയതാ അത് ശരിക്കുമൊരു പാമ്പായി മാറുന്നു.
\end{malayalam}}
\flushright{\begin{Arabic}
\quranayah[26][33]
\end{Arabic}}
\flushleft{\begin{malayalam}
അദ്ദേഹം തന്റെ കൈ കക്ഷത്തുനിന്ന് പുറത്തെടുത്തു. അപ്പോഴതാ അത് കാണികള്‍ക്കൊക്കെ തിളങ്ങുന്നതായിത്തീരുന്നു.
\end{malayalam}}
\flushright{\begin{Arabic}
\quranayah[26][34]
\end{Arabic}}
\flushleft{\begin{malayalam}
ഫറവോന്‍ തന്റെ ചുറ്റുമുള്ള പ്രമാണിമാരോടു പറഞ്ഞു: "സംശയമില്ല; ഇവനൊരു പഠിച്ച ജാലവിദ്യക്കാരന്‍ തന്നെ.
\end{malayalam}}
\flushright{\begin{Arabic}
\quranayah[26][35]
\end{Arabic}}
\flushleft{\begin{malayalam}
"തന്റെ ജാലവിദ്യയിലൂടെ നിങ്ങളെ നിങ്ങളുടെ നാട്ടില്‍നിന്ന് പുറന്തള്ളാനാണിവനുദ്ദേശിക്കുന്നത്. അതിനാല്‍ നിങ്ങള്‍ക്കെന്തു നിര്‍ദേശമാണ് നല്‍കാനുള്ളത്?”
\end{malayalam}}
\flushright{\begin{Arabic}
\quranayah[26][36]
\end{Arabic}}
\flushleft{\begin{malayalam}
അവര്‍ പറഞ്ഞു: "ഇവന്റെയും ഇവന്റെ സഹോദരന്റെയും കാര്യം ഇത്തിരി നീട്ടിവെക്കുക. എന്നിട്ട് ആളുകളെ വിളിച്ചുകൂട്ടാന്‍ നഗരങ്ങളിലേക്ക് ദൂതന്മാരെ നിയോഗിക്കുക.
\end{malayalam}}
\flushright{\begin{Arabic}
\quranayah[26][37]
\end{Arabic}}
\flushleft{\begin{malayalam}
"സമര്‍ഥരായ സകല ജാലവിദ്യക്കാരെയും അവര്‍ അങ്ങയുടെ അടുത്ത് വിളിച്ചുചേര്‍ക്കട്ടെ.”
\end{malayalam}}
\flushright{\begin{Arabic}
\quranayah[26][38]
\end{Arabic}}
\flushleft{\begin{malayalam}
അങ്ങനെ ഒരു നിശ്ചിതദിവസം നിശ്ചിതസമയത്ത് ജാലവിദ്യക്കാരെയൊക്കെ ഒരുമിച്ചുകൂട്ടി.
\end{malayalam}}
\flushright{\begin{Arabic}
\quranayah[26][39]
\end{Arabic}}
\flushleft{\begin{malayalam}
ഫറവോന്‍ ജനങ്ങളോടു പറഞ്ഞു: "നിങ്ങളെല്ലാം ഇവിടെ ഒത്തുകൂടുന്നുണ്ടല്ലോ.
\end{malayalam}}
\flushright{\begin{Arabic}
\quranayah[26][40]
\end{Arabic}}
\flushleft{\begin{malayalam}
"ജാലവിദ്യക്കാരാണ് വിജയിക്കുന്നതെങ്കില്‍ നമുക്കവരോടൊപ്പം ചേരാം.”
\end{malayalam}}
\flushright{\begin{Arabic}
\quranayah[26][41]
\end{Arabic}}
\flushleft{\begin{malayalam}
അങ്ങനെ ജാലവിദ്യക്കാര്‍ വന്നു. അവര്‍ ഫറവോനോട് ചോദിച്ചു: "ഞങ്ങളാണ് വിജയിക്കുന്നതെങ്കില്‍ ഉറപ്പായും ഞങ്ങള്‍ക്ക് നല്ല പ്രതിഫലമുണ്ടാവില്ലേ!”
\end{malayalam}}
\flushright{\begin{Arabic}
\quranayah[26][42]
\end{Arabic}}
\flushleft{\begin{malayalam}
ഫറവോന്‍ പറഞ്ഞു: "അതെ. ഉറപ്പായും നിങ്ങളപ്പോള്‍ നമ്മുടെ അടുത്ത ആളുകളായിരിക്കും.”
\end{malayalam}}
\flushright{\begin{Arabic}
\quranayah[26][43]
\end{Arabic}}
\flushleft{\begin{malayalam}
മൂസ അവരോടു പറഞ്ഞു: "നിങ്ങള്‍ക്ക് എറിയാനുള്ളത് എറിഞ്ഞുകൊള്ളുക.”
\end{malayalam}}
\flushright{\begin{Arabic}
\quranayah[26][44]
\end{Arabic}}
\flushleft{\begin{malayalam}
അവര്‍ തങ്ങളുടെ കയ്യിലുണ്ടായിരുന്ന കയറുകളും വടികളും നിലത്തിട്ടു. എന്നിട്ടിങ്ങനെ പറഞ്ഞു: "ഫറവോന്റെ പ്രതാപത്താല്‍ തീര്‍ച്ചയായും ഞങ്ങള്‍ തന്നെയായിരിക്കും വിജയികള്‍.”
\end{malayalam}}
\flushright{\begin{Arabic}
\quranayah[26][45]
\end{Arabic}}
\flushleft{\begin{malayalam}
പിന്നെ മൂസ തന്റെ വടി നിലത്തിട്ടു. ഉടനെയതാ അത് അവരുടെ വ്യാജനിര്‍മിതികളെയൊക്കെ വിഴുങ്ങിക്കളഞ്ഞു.
\end{malayalam}}
\flushright{\begin{Arabic}
\quranayah[26][46]
\end{Arabic}}
\flushleft{\begin{malayalam}
അതോടെ ജാലവിദ്യക്കാരെല്ലാം സാഷ്ടാംഗം പ്രണമിച്ചു നിലത്തുവീണു.
\end{malayalam}}
\flushright{\begin{Arabic}
\quranayah[26][47]
\end{Arabic}}
\flushleft{\begin{malayalam}
അവര്‍ പറഞ്ഞു: "ഞങ്ങള്‍ പ്രപഞ്ചനാഥനില്‍ വിശ്വസിച്ചിരിക്കുന്നു.
\end{malayalam}}
\flushright{\begin{Arabic}
\quranayah[26][48]
\end{Arabic}}
\flushleft{\begin{malayalam}
"മൂസായുടെയും ഹാറൂന്റെയും നാഥനില്‍.”
\end{malayalam}}
\flushright{\begin{Arabic}
\quranayah[26][49]
\end{Arabic}}
\flushleft{\begin{malayalam}
ഫറവോന്‍ പറഞ്ഞു: "ഞാന്‍ അനുവാദം തരുംമുമ്പെ നിങ്ങളവനില്‍ വിശ്വസിച്ചുവെന്നോ? തീര്‍ച്ചയായും നിങ്ങളെ ജാലവിദ്യ പഠിപ്പിച്ച നിങ്ങളുടെ തലവനാണിവന്‍. ഇതിന്റെ ഫലം ഇപ്പോള്‍തന്നെ നിങ്ങളറിയും. ഞാന്‍ നിങ്ങളുടെ കൈകാലുകള്‍ എതിര്‍വശങ്ങളില്‍ നിന്നായി മുറിച്ചുകളയും; തീര്‍ച്ച. നിങ്ങളെയൊക്കെ ഞാന്‍ കുരിശില്‍ തറക്കും.
\end{malayalam}}
\flushright{\begin{Arabic}
\quranayah[26][50]
\end{Arabic}}
\flushleft{\begin{malayalam}
അവര്‍ പറഞ്ഞു: "വിരോധമില്ല. ഞങ്ങള്‍ ഞങ്ങളുടെ നാഥനിലേക്ക് തിരിച്ചുപോകുന്നവരാണ്.
\end{malayalam}}
\flushright{\begin{Arabic}
\quranayah[26][51]
\end{Arabic}}
\flushleft{\begin{malayalam}
"ഫറവോന്റെ അനുയായികളില്‍ ആദ്യം വിശ്വസിക്കുന്നവര്‍ ഞങ്ങളാണ്. അതിനാല്‍ ഞങ്ങളുടെ നാഥന്‍ ഞങ്ങളുടെ പാപങ്ങളൊക്കെ പൊറുത്തുതരണമെന്ന് ഞങ്ങള്‍ ആഗ്രഹിക്കുന്നു.”
\end{malayalam}}
\flushright{\begin{Arabic}
\quranayah[26][52]
\end{Arabic}}
\flushleft{\begin{malayalam}
മൂസാക്കു നാം ബോധനം നല്‍കി: "എന്റെ ദാസന്മാരെയും കൂട്ടി രാത്രിതന്നെ പുറപ്പെട്ടുകൊള്ളുക. തീര്‍ച്ചയായും അവര്‍ നിങ്ങളെ പിന്തുടരും.”
\end{malayalam}}
\flushright{\begin{Arabic}
\quranayah[26][53]
\end{Arabic}}
\flushleft{\begin{malayalam}
അപ്പോള്‍ ഫറവോന്‍ ആളുകളെ ഒരുമിച്ചുകൂട്ടാന്‍ പട്ടണങ്ങളിലേക്ക് ദൂതന്മാരെ അയച്ചു.
\end{malayalam}}
\flushright{\begin{Arabic}
\quranayah[26][54]
\end{Arabic}}
\flushleft{\begin{malayalam}
ഫറവോന്‍ പറഞ്ഞു: "തീര്‍ച്ചയായും ഇവര്‍ ഏതാനും പേരുടെ ഒരു ചെറുസംഘമാണ്.
\end{malayalam}}
\flushright{\begin{Arabic}
\quranayah[26][55]
\end{Arabic}}
\flushleft{\begin{malayalam}
"അവര്‍ നമ്മെ വല്ലാതെ കോപാകുലരാക്കിയിരിക്കുന്നു.
\end{malayalam}}
\flushright{\begin{Arabic}
\quranayah[26][56]
\end{Arabic}}
\flushleft{\begin{malayalam}
"തീര്‍ച്ചയായും നാം സംഘടിതരാണ്. ഏറെ ജാഗ്രത പുലര്‍ത്തുന്നവരും.”
\end{malayalam}}
\flushright{\begin{Arabic}
\quranayah[26][57]
\end{Arabic}}
\flushleft{\begin{malayalam}
അങ്ങനെ നാമവരെ തോട്ടങ്ങളില്‍നിന്നും നീരുറവകളില്‍ നിന്നും പുറത്തിറക്കി.
\end{malayalam}}
\flushright{\begin{Arabic}
\quranayah[26][58]
\end{Arabic}}
\flushleft{\begin{malayalam}
ഖജനാവുകളില്‍ നിന്നും മാന്യമായ പാര്‍പ്പിടങ്ങളില്‍ നിന്നും.
\end{malayalam}}
\flushright{\begin{Arabic}
\quranayah[26][59]
\end{Arabic}}
\flushleft{\begin{malayalam}
അങ്ങനെയാണ് നാം ചെയ്യുക. അവയൊക്കെ ഇസ്രയേല്‍ മക്കള്‍ക്കു നാം അവകാശപ്പെടുത്തിക്കൊടുക്കുകയും ചെയ്തു.
\end{malayalam}}
\flushright{\begin{Arabic}
\quranayah[26][60]
\end{Arabic}}
\flushleft{\begin{malayalam}
പിന്നീട് പ്രഭാതവേളയില്‍ ആ ജനം ഇവരെ പിന്തുടര്‍ന്നു.
\end{malayalam}}
\flushright{\begin{Arabic}
\quranayah[26][61]
\end{Arabic}}
\flushleft{\begin{malayalam}
ഇരുസംഘവും പരസ്പരം കണ്ടുമുട്ടിയപ്പോള്‍ മൂസായുടെ അനുയായികള്‍ പറഞ്ഞു: "ഉറപ്പായും നാമിതാ പിടികൂടപ്പെടാന്‍ പോവുകയാണ്.”
\end{malayalam}}
\flushright{\begin{Arabic}
\quranayah[26][62]
\end{Arabic}}
\flushleft{\begin{malayalam}
മൂസ പറഞ്ഞു: "ഒരിക്കലുമില്ല. എന്നോടൊപ്പം എന്റെ നാഥനുണ്ട്. അവന്‍ എനിക്കു രക്ഷാമാര്‍ഗം കാണിച്ചുതരികതന്നെ ചെയ്യും.”
\end{malayalam}}
\flushright{\begin{Arabic}
\quranayah[26][63]
\end{Arabic}}
\flushleft{\begin{malayalam}
അപ്പോള്‍ മൂസാക്കു നാം ബോധനം നല്‍കി: "നീ നിന്റെ വടികൊണ്ട് കടലിനെ അടിക്കുക.” അതോടെ കടല്‍ പിളര്‍ന്നു. ഇരുപുറവും പടുകൂറ്റന്‍ പര്‍വതംപോലെയായി.
\end{malayalam}}
\flushright{\begin{Arabic}
\quranayah[26][64]
\end{Arabic}}
\flushleft{\begin{malayalam}
ഫറവോനെയും സംഘത്തെയും നാം അതിന്റെ അടുത്തെത്തിച്ചു.
\end{malayalam}}
\flushright{\begin{Arabic}
\quranayah[26][65]
\end{Arabic}}
\flushleft{\begin{malayalam}
മൂസായെയും അദ്ദേഹത്തോടൊപ്പമുണ്ടായിരുന്ന എല്ലാവരെയും നാം രക്ഷപ്പെടുത്തി.
\end{malayalam}}
\flushright{\begin{Arabic}
\quranayah[26][66]
\end{Arabic}}
\flushleft{\begin{malayalam}
പിന്നെ മറ്റുള്ളവരെ വെള്ളത്തിലാഴ്ത്തിക്കൊന്നു.
\end{malayalam}}
\flushright{\begin{Arabic}
\quranayah[26][67]
\end{Arabic}}
\flushleft{\begin{malayalam}
തീര്‍ച്ചയായും ഇതില്‍ വലിയ ഗുണപാഠമുണ്ട്. എന്നിട്ടും അവരിലേറെ പേരും വിശ്വസിക്കുന്നവരായില്ല.
\end{malayalam}}
\flushright{\begin{Arabic}
\quranayah[26][68]
\end{Arabic}}
\flushleft{\begin{malayalam}
തീര്‍ച്ചയായും നിന്റെ നാഥന്‍ ഏറെ പ്രതാപിയും പരമകാരുണികനുമാണ്.
\end{malayalam}}
\flushright{\begin{Arabic}
\quranayah[26][69]
\end{Arabic}}
\flushleft{\begin{malayalam}
ഇബ്റാഹീമിന്റെ കഥ ഇവരെ വായിച്ചുകേള്‍പ്പിക്കുക:
\end{malayalam}}
\flushright{\begin{Arabic}
\quranayah[26][70]
\end{Arabic}}
\flushleft{\begin{malayalam}
അദ്ദേഹം തന്റെ പിതാവിനോടും ജനതയോടും ചോദിച്ച സന്ദര്‍ഭം: "നിങ്ങള്‍ എന്തിനെയാണ് പൂജിച്ചുകൊണ്ടിരിക്കുന്നത്?”
\end{malayalam}}
\flushright{\begin{Arabic}
\quranayah[26][71]
\end{Arabic}}
\flushleft{\begin{malayalam}
അവര്‍ പറഞ്ഞു: "ഞങ്ങള്‍ ചില വിഗ്രഹങ്ങളെ പൂജിക്കുന്നു. അവയ്ക്ക് ഭജനമിരിക്കുകയും ചെയ്യുന്നു.”
\end{malayalam}}
\flushright{\begin{Arabic}
\quranayah[26][72]
\end{Arabic}}
\flushleft{\begin{malayalam}
അദ്ദേഹം ചോദിച്ചു: "നിങ്ങള്‍ പ്രാര്‍ഥിക്കുമ്പോള്‍ അവയത് കേള്‍ക്കുമോ?
\end{malayalam}}
\flushright{\begin{Arabic}
\quranayah[26][73]
\end{Arabic}}
\flushleft{\begin{malayalam}
"അല്ലെങ്കില്‍ നിങ്ങള്‍ക്ക് അവ വല്ല ഉപകാരമോ ഉപദ്രവമോ വരുത്തുമോ?”
\end{malayalam}}
\flushright{\begin{Arabic}
\quranayah[26][74]
\end{Arabic}}
\flushleft{\begin{malayalam}
അവര്‍ പറഞ്ഞു: "ഇല്ല. എന്നാല്‍ ഞങ്ങളുടെ പിതാക്കള്‍ അവയെ പൂജിക്കുന്നതായി ഞങ്ങള്‍ കണ്ടിട്ടുണ്ട്.”
\end{malayalam}}
\flushright{\begin{Arabic}
\quranayah[26][75]
\end{Arabic}}
\flushleft{\begin{malayalam}
അദ്ദേഹം ചോദിച്ചു: " നിങ്ങള്‍ പൂജിച്ചുകൊണ്ടിരിക്കുന്നത് എന്തിനെയാണെന്ന് നിങ്ങള്‍ ആലോചിച്ചുനോക്കിയിട്ടുണ്ടോ?
\end{malayalam}}
\flushright{\begin{Arabic}
\quranayah[26][76]
\end{Arabic}}
\flushleft{\begin{malayalam}
"നിങ്ങളും നിങ്ങളുടെ പൂര്‍വപിതാക്കളും
\end{malayalam}}
\flushright{\begin{Arabic}
\quranayah[26][77]
\end{Arabic}}
\flushleft{\begin{malayalam}
"അറിയുക: അവരൊക്കെയും എന്റെ എതിരാളികളാണ്. പ്രപഞ്ചനാഥനൊഴികെ.
\end{malayalam}}
\flushright{\begin{Arabic}
\quranayah[26][78]
\end{Arabic}}
\flushleft{\begin{malayalam}
"എന്നെ സൃഷ്ടിച്ചവനാണവന്‍. എന്നെ നേര്‍വഴിയിലാക്കുന്നതും അവന്‍തന്നെ.
\end{malayalam}}
\flushright{\begin{Arabic}
\quranayah[26][79]
\end{Arabic}}
\flushleft{\begin{malayalam}
"എനിക്ക് അന്നം നല്‍കുന്നതും കുടിനീര്‍ തരുന്നതും അവനാണ്.
\end{malayalam}}
\flushright{\begin{Arabic}
\quranayah[26][80]
\end{Arabic}}
\flushleft{\begin{malayalam}
"രോഗംബാധിച്ചാല്‍ സുഖപ്പെടുത്തുന്നതും അവന്‍ തന്നെ.
\end{malayalam}}
\flushright{\begin{Arabic}
\quranayah[26][81]
\end{Arabic}}
\flushleft{\begin{malayalam}
"എന്നെ മരിപ്പിക്കുന്നതും പിന്നെ ജീവിപ്പിക്കുന്നതും അവനാണ്.
\end{malayalam}}
\flushright{\begin{Arabic}
\quranayah[26][82]
\end{Arabic}}
\flushleft{\begin{malayalam}
"പ്രതിഫലനാളില്‍ എന്റെ പാപങ്ങള്‍ പൊറുത്തുതരുമെന്ന് ഞാന്‍ പ്രതീക്ഷയര്‍പ്പിക്കുന്നത് അവനിലാണ്.
\end{malayalam}}
\flushright{\begin{Arabic}
\quranayah[26][83]
\end{Arabic}}
\flushleft{\begin{malayalam}
"എന്റെ നാഥാ, എനിക്കു നീ യുക്തിജ്ഞാനം നല്‍കേണമേ. എന്നെ നീ സജ്ജനങ്ങളില്‍പെടുത്തേണമേ.
\end{malayalam}}
\flushright{\begin{Arabic}
\quranayah[26][84]
\end{Arabic}}
\flushleft{\begin{malayalam}
"പിന്‍മുറക്കാരില്‍ എനിക്കു നീ സല്‍പ്പേരുണ്ടാക്കേണമേ.
\end{malayalam}}
\flushright{\begin{Arabic}
\quranayah[26][85]
\end{Arabic}}
\flushleft{\begin{malayalam}
"എന്നെ നീ അനുഗൃഹീതമായ സ്വര്‍ഗത്തിന്റെ അവകാശികളില്‍ പെടുത്തേണമേ.
\end{malayalam}}
\flushright{\begin{Arabic}
\quranayah[26][86]
\end{Arabic}}
\flushleft{\begin{malayalam}
"എന്റെ പിതാവിനു നീ പൊറുത്തുകൊടുക്കേണമേ. സംശയമില്ല; അദ്ദേഹം വഴിപിഴച്ചവന്‍തന്നെ.
\end{malayalam}}
\flushright{\begin{Arabic}
\quranayah[26][87]
\end{Arabic}}
\flushleft{\begin{malayalam}
"ജനം ഉയിര്‍ത്തെഴുന്നേല്‍ക്കുന്ന നാളില്‍ നീയെന്നെ അപമാനിതനാക്കരുതേ.
\end{malayalam}}
\flushright{\begin{Arabic}
\quranayah[26][88]
\end{Arabic}}
\flushleft{\begin{malayalam}
"സമ്പത്തോ സന്താനങ്ങളോ ഒട്ടും ഉപകരിക്കാത്ത ദിനമാണത്.
\end{malayalam}}
\flushright{\begin{Arabic}
\quranayah[26][89]
\end{Arabic}}
\flushleft{\begin{malayalam}
"കുറ്റമറ്റ മനസ്സുമായി അല്ലാഹുവിന്റെ സന്നിധിയില്‍ ചെന്നെത്തിയവര്‍ക്കൊഴികെ.”
\end{malayalam}}
\flushright{\begin{Arabic}
\quranayah[26][90]
\end{Arabic}}
\flushleft{\begin{malayalam}
അന്ന് ഭക്തന്മാര്‍ക്ക് സ്വര്‍ഗം വളരെ അടുത്തായിരിക്കും.
\end{malayalam}}
\flushright{\begin{Arabic}
\quranayah[26][91]
\end{Arabic}}
\flushleft{\begin{malayalam}
വഴിപിഴച്ചവരുടെ മുന്നില്‍ നരകം വെളിപ്പെടുത്തുകയും ചെയ്യും.
\end{malayalam}}
\flushright{\begin{Arabic}
\quranayah[26][92]
\end{Arabic}}
\flushleft{\begin{malayalam}
അന്ന് അവരോടു ചോദിക്കും: "നിങ്ങള്‍ പൂജിച്ചിരുന്നവയെല്ലാം എവിടെപ്പോയി?
\end{malayalam}}
\flushright{\begin{Arabic}
\quranayah[26][93]
\end{Arabic}}
\flushleft{\begin{malayalam}
അല്ലാഹുവെക്കൂടാതെ; അവ നിങ്ങളെ സഹായിക്കുന്നുണ്ടോ? എന്നല്ല; അവയ്ക്ക് സ്വയം രക്ഷപ്പെടാനെങ്കിലും കഴിയുന്നുണ്ടോ?”
\end{malayalam}}
\flushright{\begin{Arabic}
\quranayah[26][94]
\end{Arabic}}
\flushleft{\begin{malayalam}
അങ്ങനെ ആ ദുര്‍മാര്‍ഗികളെയും അവരുടെ ആരാധ്യരെയും അതില്‍ മുഖംകുത്തി വീഴ്ത്തും.
\end{malayalam}}
\flushright{\begin{Arabic}
\quranayah[26][95]
\end{Arabic}}
\flushleft{\begin{malayalam}
ഇബ്ലീസിന്റെ മുഴുവന്‍ പടയെയും.
\end{malayalam}}
\flushright{\begin{Arabic}
\quranayah[26][96]
\end{Arabic}}
\flushleft{\begin{malayalam}
അവിടെ അവരന്യോന്യം ശണ്ഠകൂടിക്കൊണ്ടിരിക്കും. അന്നേരം ആ ദുര്‍മാര്‍ഗികള്‍ പറയും:
\end{malayalam}}
\flushright{\begin{Arabic}
\quranayah[26][97]
\end{Arabic}}
\flushleft{\begin{malayalam}
"അല്ലാഹുവാണ് സത്യം? ഞങ്ങള്‍ വ്യക്തമായ വഴികേടില്‍ തന്നെയായിരുന്നു.
\end{malayalam}}
\flushright{\begin{Arabic}
\quranayah[26][98]
\end{Arabic}}
\flushleft{\begin{malayalam}
"ഞങ്ങള്‍ നിങ്ങളെ പ്രപഞ്ചനാഥന്ന് തുല്യരാക്കിയപ്പോള്‍.
\end{malayalam}}
\flushright{\begin{Arabic}
\quranayah[26][99]
\end{Arabic}}
\flushleft{\begin{malayalam}
"ഞങ്ങളെ വഴിതെറ്റിച്ചത് ആ കുറ്റവാളികളല്ലാതാരുമല്ല.
\end{malayalam}}
\flushright{\begin{Arabic}
\quranayah[26][100]
\end{Arabic}}
\flushleft{\begin{malayalam}
-"ഇപ്പോള്‍ ഞങ്ങള്‍ക്ക് ശിപാര്‍ശകരായി ആരുമില്ല.
\end{malayalam}}
\flushright{\begin{Arabic}
\quranayah[26][101]
\end{Arabic}}
\flushleft{\begin{malayalam}
-"ഉറ്റമിത്രവുമില്ല.
\end{malayalam}}
\flushright{\begin{Arabic}
\quranayah[26][102]
\end{Arabic}}
\flushleft{\begin{malayalam}
"അതിനാല്‍ ഞങ്ങള്‍ക്കൊന്ന് തിരിച്ചുപോകാന്‍ കഴിഞ്ഞിരുന്നെങ്കില്‍! അപ്പോള്‍ ഉറപ്പായും ഞങ്ങള്‍ സത്യവിശ്വാസികളിലുള്‍പ്പെടുമായിരുന്നു!”
\end{malayalam}}
\flushright{\begin{Arabic}
\quranayah[26][103]
\end{Arabic}}
\flushleft{\begin{malayalam}
തീര്‍ച്ചയായും ഇതില്‍ ജനങ്ങള്‍ക്ക് ഗുണപാഠമുണ്ട്. എന്നിട്ടും അവരിലേറെ പേരും വിശ്വസിക്കുന്നവരായില്ല.
\end{malayalam}}
\flushright{\begin{Arabic}
\quranayah[26][104]
\end{Arabic}}
\flushleft{\begin{malayalam}
തീര്‍ച്ചയായും നിന്റെ നാഥന്‍ തന്നെയാണ് പ്രതാപിയും പരമ കാരുണികനും.
\end{malayalam}}
\flushright{\begin{Arabic}
\quranayah[26][105]
\end{Arabic}}
\flushleft{\begin{malayalam}
നൂഹിന്റെ ജനത ദൈവദൂതന്മാരെ തള്ളിപ്പറഞ്ഞു.
\end{malayalam}}
\flushright{\begin{Arabic}
\quranayah[26][106]
\end{Arabic}}
\flushleft{\begin{malayalam}
അവരുടെ സഹോദരന്‍ നൂഹ് അവരോടിങ്ങനെ പറഞ്ഞ സന്ദര്‍ഭം: "നിങ്ങള്‍ ഭക്തിപുലര്‍ത്തുന്നില്ലേ?
\end{malayalam}}
\flushright{\begin{Arabic}
\quranayah[26][107]
\end{Arabic}}
\flushleft{\begin{malayalam}
"സംശയം വേണ്ട; ഞാന്‍ നിങ്ങളിലേക്കുള്ള വിശ്വസ്തനായ ദൈവദൂതനാണ്.
\end{malayalam}}
\flushright{\begin{Arabic}
\quranayah[26][108]
\end{Arabic}}
\flushleft{\begin{malayalam}
"അതിനാല്‍ അല്ലാഹുവോട് ഭക്തിയുള്ളവരാവുക. എന്നെ അനുസരിക്കുക.
\end{malayalam}}
\flushright{\begin{Arabic}
\quranayah[26][109]
\end{Arabic}}
\flushleft{\begin{malayalam}
"ഇതിന്റെ പേരില്‍ ഞാന്‍ നിങ്ങളോടൊരു പ്രതിഫലവും ചോദിക്കുന്നില്ല. എനിക്കുള്ള പ്രതിഫലം പ്രപഞ്ചനാഥന്റെ വശമാണുള്ളത്.
\end{malayalam}}
\flushright{\begin{Arabic}
\quranayah[26][110]
\end{Arabic}}
\flushleft{\begin{malayalam}
"അതിനാല്‍ നിങ്ങള്‍ അല്ലാഹുവോട് ഭക്തിയുള്ളവരാവുക. എന്നെ അനുസരിക്കുക.”
\end{malayalam}}
\flushright{\begin{Arabic}
\quranayah[26][111]
\end{Arabic}}
\flushleft{\begin{malayalam}
അവര്‍ പറഞ്ഞു: "നിന്നെ പിന്‍പറ്റിയവരൊക്കെ താണവിഭാഗത്തില്‍ പെട്ടവരാണല്ലോ. പിന്നെ ഞങ്ങള്‍ക്കെങ്ങനെ നിന്നില്‍ വിശ്വസിക്കാനാകും?”
\end{malayalam}}
\flushright{\begin{Arabic}
\quranayah[26][112]
\end{Arabic}}
\flushleft{\begin{malayalam}
അദ്ദേഹം പറഞ്ഞു: "അവര്‍ ചെയ്തുകൊണ്ടിരിക്കുന്നതിനെപ്പറ്റി എനിക്കെന്തറിയാം?
\end{malayalam}}
\flushright{\begin{Arabic}
\quranayah[26][113]
\end{Arabic}}
\flushleft{\begin{malayalam}
"അവരുടെ വിചാരണ എന്റെ നാഥന്റെ മാത്രം ചുമതലയത്രെ. നിങ്ങള്‍ ബോധമുള്ളവരെങ്കില്‍ അതോര്‍ക്കുക.
\end{malayalam}}
\flushright{\begin{Arabic}
\quranayah[26][114]
\end{Arabic}}
\flushleft{\begin{malayalam}
"സത്യവിശ്വാസികളെ ഞാനെന്തായാലും ആട്ടിയകറ്റുകയില്ല.
\end{malayalam}}
\flushright{\begin{Arabic}
\quranayah[26][115]
\end{Arabic}}
\flushleft{\begin{malayalam}
"ഞാന്‍ വ്യക്തമായ മുന്നറിയിപ്പുകാരന്‍ മാത്രമാണ്.”
\end{malayalam}}
\flushright{\begin{Arabic}
\quranayah[26][116]
\end{Arabic}}
\flushleft{\begin{malayalam}
അവര്‍ പറഞ്ഞു: "നൂഹേ, നീയിതു നിര്‍ത്തുന്നില്ലെങ്കില്‍ നിശ്ചയമായും നിന്നെ എറിഞ്ഞുകൊല്ലുകതന്നെ ചെയ്യും.”
\end{malayalam}}
\flushright{\begin{Arabic}
\quranayah[26][117]
\end{Arabic}}
\flushleft{\begin{malayalam}
നൂഹ് പറഞ്ഞു: "എന്റെ നാഥാ, തീര്‍ച്ചയായും എന്റെ ജനത എന്നെ തള്ളിപ്പറഞ്ഞിരിക്കുന്നു.
\end{malayalam}}
\flushright{\begin{Arabic}
\quranayah[26][118]
\end{Arabic}}
\flushleft{\begin{malayalam}
"അതിനാല്‍ എനിക്കും അവര്‍ക്കുമിടയില്‍ നീയൊരു നിര്‍ണായക തീരുമാനമെടുക്കേണമേ. എന്നെയും എന്റെ കൂടെയുള്ള സത്യവിശ്വാസികളെയും രക്ഷപ്പെടുത്തേണമേ.”
\end{malayalam}}
\flushright{\begin{Arabic}
\quranayah[26][119]
\end{Arabic}}
\flushleft{\begin{malayalam}
അപ്പോള്‍ അദ്ദേഹത്തെയും അദ്ദേഹത്തോടൊപ്പമുള്ളവരെയും തിങ്ങിനിറഞ്ഞ ഒരു കപ്പലില്‍ നാം രക്ഷപ്പെടുത്തി.
\end{malayalam}}
\flushright{\begin{Arabic}
\quranayah[26][120]
\end{Arabic}}
\flushleft{\begin{malayalam}
പിന്നെ അതിനുശേഷം ബാക്കിയുള്ളവരെയൊക്കെ വെള്ളത്തില്‍ മുക്കിയാഴ്ത്തി.
\end{malayalam}}
\flushright{\begin{Arabic}
\quranayah[26][121]
\end{Arabic}}
\flushleft{\begin{malayalam}
തീര്‍ച്ചയായും അതില്‍ ജനങ്ങള്‍ക്കൊരു ദൃഷ്ടാന്തമുണ്ട്. എന്നിട്ടും അവരിലേറെ പേരും വിശ്വസിക്കുന്നവരായില്ല.
\end{malayalam}}
\flushright{\begin{Arabic}
\quranayah[26][122]
\end{Arabic}}
\flushleft{\begin{malayalam}
നിശ്ചയം നിന്റെ നാഥന്‍ തന്നെയാണ് പ്രതാപവാനും പരമകാരുണികനും.
\end{malayalam}}
\flushright{\begin{Arabic}
\quranayah[26][123]
\end{Arabic}}
\flushleft{\begin{malayalam}
ആദ് സമുദായം ദൈവദൂതന്മാരെ തള്ളിപ്പറഞ്ഞു.
\end{malayalam}}
\flushright{\begin{Arabic}
\quranayah[26][124]
\end{Arabic}}
\flushleft{\begin{malayalam}
അവരുടെ സഹോദരന്‍ ഹൂദ് അവരോടു പറഞ്ഞതോര്‍ക്കുക: "നിങ്ങള്‍ ഭക്തരാവുന്നില്ലേ?
\end{malayalam}}
\flushright{\begin{Arabic}
\quranayah[26][125]
\end{Arabic}}
\flushleft{\begin{malayalam}
"തീര്‍ച്ചയായും ഞാന്‍ നിങ്ങളിലേക്കുള്ള വിശ്വസ്തനായ ദൈവദൂതനാണ്.
\end{malayalam}}
\flushright{\begin{Arabic}
\quranayah[26][126]
\end{Arabic}}
\flushleft{\begin{malayalam}
"അതിനാല്‍ നിങ്ങള്‍ അല്ലാഹുവോട് ഭക്തിപുലര്‍ത്തുക. എന്നെ അനുസരിക്കുക.
\end{malayalam}}
\flushright{\begin{Arabic}
\quranayah[26][127]
\end{Arabic}}
\flushleft{\begin{malayalam}
"ഇതിന്റെ പേരില്‍ ഞാന്‍ നിങ്ങളോടൊരു പ്രതിഫലവും ചോദിക്കുന്നില്ല. എനിക്കുള്ള പ്രതിഫലം പ്രപഞ്ചനാഥന്റെ വശമാണുള്ളത്.
\end{malayalam}}
\flushright{\begin{Arabic}
\quranayah[26][128]
\end{Arabic}}
\flushleft{\begin{malayalam}
"വെറുതെ പൊങ്ങച്ചം കാട്ടാനായി നിങ്ങള്‍ എല്ലാ കുന്നിന്‍മുകളിലും സ്മാരകസൌധങ്ങള്‍ കെട്ടിപ്പൊക്കുകയാണോ?
\end{malayalam}}
\flushright{\begin{Arabic}
\quranayah[26][129]
\end{Arabic}}
\flushleft{\begin{malayalam}
"നിങ്ങള്‍ക്ക് എക്കാലവും പാര്‍ക്കാനെന്നപോലെ പടുകൂറ്റന്‍ കൊട്ടാരങ്ങള്‍ പടുത്തുയര്‍ത്തുകയാണോ?
\end{malayalam}}
\flushright{\begin{Arabic}
\quranayah[26][130]
\end{Arabic}}
\flushleft{\begin{malayalam}
"നിങ്ങള്‍ ആരെയെങ്കിലും പിടികൂടിയാല്‍ വളരെ ക്രൂരമായാണ് ബലപ്രയോഗം നടത്തുന്നത്.
\end{malayalam}}
\flushright{\begin{Arabic}
\quranayah[26][131]
\end{Arabic}}
\flushleft{\begin{malayalam}
"അതിനാല്‍ നിങ്ങള്‍ അല്ലാഹുവോട് ഭക്തിയുള്ളവരാവുക. എന്നെ അനുസരിക്കുക.
\end{malayalam}}
\flushright{\begin{Arabic}
\quranayah[26][132]
\end{Arabic}}
\flushleft{\begin{malayalam}
"അല്ലാഹു നിങ്ങളെ സഹായിച്ചതെങ്ങനെയെല്ലാമെന്ന് നിങ്ങള്‍ക്കു നന്നായറിയാമല്ലോ. അതിനാല്‍ നിങ്ങള്‍ അവനോട് ഭക്തിയുള്ളവരാവുക.
\end{malayalam}}
\flushright{\begin{Arabic}
\quranayah[26][133]
\end{Arabic}}
\flushleft{\begin{malayalam}
"കന്നുകാലികളെയും മക്കളെയും നല്‍കി അവന്‍ നിങ്ങളെ സഹായിച്ചു.
\end{malayalam}}
\flushright{\begin{Arabic}
\quranayah[26][134]
\end{Arabic}}
\flushleft{\begin{malayalam}
"തോട്ടങ്ങളും അരുവികളും തന്നു.
\end{malayalam}}
\flushright{\begin{Arabic}
\quranayah[26][135]
\end{Arabic}}
\flushleft{\begin{malayalam}
"ഭയങ്കരമായ ഒരുനാളിലെ ശിക്ഷ നിങ്ങള്‍ക്കു വന്നെത്തുമെന്ന് ഞാന്‍ ഭയപ്പെടുന്നു.”
\end{malayalam}}
\flushright{\begin{Arabic}
\quranayah[26][136]
\end{Arabic}}
\flushleft{\begin{malayalam}
അവര്‍ പറഞ്ഞു: "നീ ഉപദേശിക്കുന്നതും ഉപദേശിക്കാതിരിക്കുന്നതും ഞങ്ങള്‍ക്ക് ഒരേപോലെയാണ്.
\end{malayalam}}
\flushright{\begin{Arabic}
\quranayah[26][137]
\end{Arabic}}
\flushleft{\begin{malayalam}
"ഞങ്ങള്‍ ഈ ചെയ്യുന്നതൊക്കെ പൂര്‍വികരുടെ പതിവില്‍ പെട്ടതാണ്.
\end{malayalam}}
\flushright{\begin{Arabic}
\quranayah[26][138]
\end{Arabic}}
\flushleft{\begin{malayalam}
"ഞങ്ങള്‍ ശിക്ഷിക്കപ്പെടുന്നവരല്ല.”
\end{malayalam}}
\flushright{\begin{Arabic}
\quranayah[26][139]
\end{Arabic}}
\flushleft{\begin{malayalam}
അങ്ങനെ അവരദ്ദേഹത്തെ തള്ളിപ്പറഞ്ഞു. അതിനാല്‍ നാമവരെ നശിപ്പിച്ചു. തീര്‍ച്ചയായും അതിലൊരു ദൃഷ്ടാന്തമുണ്ട്. എന്നിട്ടും അവരിലേറെപ്പേരും വിശ്വാസികളായില്ല.
\end{malayalam}}
\flushright{\begin{Arabic}
\quranayah[26][140]
\end{Arabic}}
\flushleft{\begin{malayalam}
നിശ്ചയം നിന്റെ നാഥന്‍ ഏറെ പ്രതാപിയും പരമദയാലുവുമാണ്.
\end{malayalam}}
\flushright{\begin{Arabic}
\quranayah[26][141]
\end{Arabic}}
\flushleft{\begin{malayalam}
സമൂദ് സമുദായം ദൈവദൂതന്മാരെ തള്ളിപ്പറഞ്ഞു.
\end{malayalam}}
\flushright{\begin{Arabic}
\quranayah[26][142]
\end{Arabic}}
\flushleft{\begin{malayalam}
അവരുടെ സഹോദരന്‍ സ്വാലിഹ് അവരോട് ചോദിച്ച സന്ദര്‍ഭം: "നിങ്ങള്‍ ഭക്തരാവുന്നില്ലേ?
\end{malayalam}}
\flushright{\begin{Arabic}
\quranayah[26][143]
\end{Arabic}}
\flushleft{\begin{malayalam}
"തീര്‍ച്ചയായും ഞാന്‍ നിങ്ങളിലേക്കയക്കപ്പെട്ട വിശ്വസ്തനായ ദൈവദൂതനാണ്.
\end{malayalam}}
\flushright{\begin{Arabic}
\quranayah[26][144]
\end{Arabic}}
\flushleft{\begin{malayalam}
"അതിനാല്‍ നിങ്ങള്‍ ദൈവഭക്തരാവുക. എന്നെ അനുസരിക്കുക.
\end{malayalam}}
\flushright{\begin{Arabic}
\quranayah[26][145]
\end{Arabic}}
\flushleft{\begin{malayalam}
"ഇതിന്റെ പേരില്‍ ഞാന്‍ നിങ്ങളോടൊരു പ്രതിഫലവും ചോദിക്കുന്നില്ല. എനിക്കുള്ള പ്രതിഫലം പ്രപഞ്ചനാഥനില്‍ നിന്ന് മാത്രമാണ്.
\end{malayalam}}
\flushright{\begin{Arabic}
\quranayah[26][146]
\end{Arabic}}
\flushleft{\begin{malayalam}
"അല്ലാ, ഇവിടെ ഇക്കാണുന്നതിലൊക്കെ നിര്‍ഭയമായി യഥേഷ്ടം വിഹരിക്കാന്‍ നിങ്ങളെ വിട്ടേക്കുമെന്നാണോ നിങ്ങള്‍ കരുതുന്നത്?
\end{malayalam}}
\flushright{\begin{Arabic}
\quranayah[26][147]
\end{Arabic}}
\flushleft{\begin{malayalam}
"അതായത് ഈ തോട്ടങ്ങളിലും അരുവികളിലും?
\end{malayalam}}
\flushright{\begin{Arabic}
\quranayah[26][148]
\end{Arabic}}
\flushleft{\begin{malayalam}
"വയലുകളിലും പാകമായ പഴക്കുലകള്‍ നിറഞ്ഞ ഈന്തപ്പനത്തോപ്പുകളിലും?
\end{malayalam}}
\flushright{\begin{Arabic}
\quranayah[26][149]
\end{Arabic}}
\flushleft{\begin{malayalam}
"നിങ്ങള്‍ ആര്‍ഭാടപ്രിയരായി പര്‍വതങ്ങള്‍ തുരന്ന് വീടുകളുണ്ടാക്കുന്നു.
\end{malayalam}}
\flushright{\begin{Arabic}
\quranayah[26][150]
\end{Arabic}}
\flushleft{\begin{malayalam}
"നിങ്ങള്‍ ദൈവഭക്തരാവുക. എന്നെ അനുസരിക്കുക.
\end{malayalam}}
\flushright{\begin{Arabic}
\quranayah[26][151]
\end{Arabic}}
\flushleft{\begin{malayalam}
"അതിക്രമികളുടെ ആജ്ഞകള്‍ അനുസരിക്കരുത്.
\end{malayalam}}
\flushright{\begin{Arabic}
\quranayah[26][152]
\end{Arabic}}
\flushleft{\begin{malayalam}
"ഭൂമിയില്‍ കുഴപ്പമുണ്ടാക്കുന്നവരാണവര്‍. ഒരുവിധ സംസ്കരണവും വരുത്താത്തവരും.”
\end{malayalam}}
\flushright{\begin{Arabic}
\quranayah[26][153]
\end{Arabic}}
\flushleft{\begin{malayalam}
അവര്‍ പറഞ്ഞു: "നീ മാരണം ബാധിച്ചവന്‍ തന്നെ.
\end{malayalam}}
\flushright{\begin{Arabic}
\quranayah[26][154]
\end{Arabic}}
\flushleft{\begin{malayalam}
"നീ ഞങ്ങളെപ്പോലുള്ള ഒരു മനുഷ്യനല്ലാതാരുമല്ല. അതിനാല്‍ നീ എന്തെങ്കിലും അടയാളം കൊണ്ടുവരിക. നീ സത്യവാദിയെങ്കില്‍!”
\end{malayalam}}
\flushright{\begin{Arabic}
\quranayah[26][155]
\end{Arabic}}
\flushleft{\begin{malayalam}
അദ്ദേഹം പറഞ്ഞു: "ഇതാ ഒരൊട്ടകം. നിശ്ചിത ദിവസം അതിനു വെള്ളം കുടിക്കാന്‍ അവസരമുണ്ട്. നിങ്ങള്‍ക്കും ഒരവസരമുണ്ട്.
\end{malayalam}}
\flushright{\begin{Arabic}
\quranayah[26][156]
\end{Arabic}}
\flushleft{\begin{malayalam}
"നിങ്ങള്‍ അതിനെ ഒരു നിലക്കും ദ്രോഹിക്കരുത്. അങ്ങനെ ചെയ്താല്‍ ഒരു ഭയങ്കര നാളിലെ ശിക്ഷ നിങ്ങളെ പിടികൂടും.”
\end{malayalam}}
\flushright{\begin{Arabic}
\quranayah[26][157]
\end{Arabic}}
\flushleft{\begin{malayalam}
എന്നാല്‍ അവരതിനെ കശാപ്പ് ചെയ്തു. അങ്ങനെ അവര്‍ കടുത്ത ദുഃഖത്തിനിരയായി.
\end{malayalam}}
\flushright{\begin{Arabic}
\quranayah[26][158]
\end{Arabic}}
\flushleft{\begin{malayalam}
അതോടെ മുന്നറിയിപ്പ് നല്‍കപ്പെട്ട ശിക്ഷ അവരെ പിടികൂടി. തീര്‍ച്ചയായും അതിലൊരു ദൃഷ്ടാന്തമുണ്ട്. എന്നിട്ടും അവരിലേറെ പ്പേരും വിശ്വാസികളായില്ല.
\end{malayalam}}
\flushright{\begin{Arabic}
\quranayah[26][159]
\end{Arabic}}
\flushleft{\begin{malayalam}
നിശ്ചയം നിന്റെ നാഥന്‍ ഏറെ പ്രതാപിയും പരമകാരുണികനുമാണ്.
\end{malayalam}}
\flushright{\begin{Arabic}
\quranayah[26][160]
\end{Arabic}}
\flushleft{\begin{malayalam}
ലൂത്വിന്റെ ജനത ദൈവദൂതന്മാരെ തള്ളിപ്പറഞ്ഞു.
\end{malayalam}}
\flushright{\begin{Arabic}
\quranayah[26][161]
\end{Arabic}}
\flushleft{\begin{malayalam}
അവരുടെ സഹോദരന്‍ ലൂത്വ് അവരോടു ചോദിച്ച സന്ദര്‍ഭം: "നിങ്ങള്‍ ഭക്തരാവുന്നില്ലേ?
\end{malayalam}}
\flushright{\begin{Arabic}
\quranayah[26][162]
\end{Arabic}}
\flushleft{\begin{malayalam}
"തീര്‍ച്ചയായും ഞാന്‍ നിങ്ങള്‍ക്കുള്ള വിശ്വസ്തനായ ദൈവദൂതനാണ്.
\end{malayalam}}
\flushright{\begin{Arabic}
\quranayah[26][163]
\end{Arabic}}
\flushleft{\begin{malayalam}
"അതിനാല്‍ നിങ്ങള്‍ ദൈവഭക്തരാവുക. എന്നെ അനുസരിക്കുക.
\end{malayalam}}
\flushright{\begin{Arabic}
\quranayah[26][164]
\end{Arabic}}
\flushleft{\begin{malayalam}
"ഇതിന്റെ പേരില്‍ ഞാന്‍ നിങ്ങളോടൊരു പ്രതിഫലവും ചോദിക്കുന്നില്ല. എനിക്കുള്ള പ്രതിഫലം പ്രപഞ്ചനാഥന്റെ വശമാണുള്ളത്.
\end{malayalam}}
\flushright{\begin{Arabic}
\quranayah[26][165]
\end{Arabic}}
\flushleft{\begin{malayalam}
"മാലോകരില്‍ കാമശമനത്തിന് വേണ്ടി നിങ്ങള്‍ പുരുഷന്മാരെ സമീപിക്കുകയാണോ?
\end{malayalam}}
\flushright{\begin{Arabic}
\quranayah[26][166]
\end{Arabic}}
\flushleft{\begin{malayalam}
"നിങ്ങളുടെ നാഥന്‍ നിങ്ങള്‍ക്കായി സൃഷ്ടിച്ചുതന്ന നിങ്ങളുടെ ഇണകളെ ഉപേക്ഷിക്കുകയും? നിങ്ങള്‍ പരിധിവിട്ട ജനംതന്നെ.”
\end{malayalam}}
\flushright{\begin{Arabic}
\quranayah[26][167]
\end{Arabic}}
\flushleft{\begin{malayalam}
അവര്‍ പറഞ്ഞു: "ലൂത്വേ, നീ ഇത് നിര്‍ത്തുന്നില്ലെങ്കില്‍ ഞങ്ങളുടെ നാട്ടില്‍നിന്ന് പുറത്താക്കപ്പെടുന്നവരില്‍ നീയുമുണ്ടാകും.”
\end{malayalam}}
\flushright{\begin{Arabic}
\quranayah[26][168]
\end{Arabic}}
\flushleft{\begin{malayalam}
അദ്ദേഹം പറഞ്ഞു: "ഞാന്‍ നിങ്ങളുടെ ഇത്തരം ചെയ്തികളെ വെറുക്കുന്ന കൂട്ടത്തിലാണ്.
\end{malayalam}}
\flushright{\begin{Arabic}
\quranayah[26][169]
\end{Arabic}}
\flushleft{\begin{malayalam}
"എന്റെ നാഥാ! നീ എന്നെയും എന്റെ കുടുംബത്തെയും ഇവര്‍ ഈ ചെയ്തുകൊണ്ടിരിക്കുന്നതില്‍നിന്ന് രക്ഷിക്കേണമേ.”
\end{malayalam}}
\flushright{\begin{Arabic}
\quranayah[26][170]
\end{Arabic}}
\flushleft{\begin{malayalam}
അവസാനം അദ്ദേഹത്തെയും അദ്ദേഹത്തിന്റെ കുടുംബത്തെയും നാം രക്ഷിച്ചു.
\end{malayalam}}
\flushright{\begin{Arabic}
\quranayah[26][171]
\end{Arabic}}
\flushleft{\begin{malayalam}
പിന്തിനിന്ന ഒരു കിഴവിയെ ഒഴികെ.
\end{malayalam}}
\flushright{\begin{Arabic}
\quranayah[26][172]
\end{Arabic}}
\flushleft{\begin{malayalam}
പിന്നീട് മറ്റുള്ളവരെ നാം തകര്‍ത്ത് നാമാവശേഷമാക്കി.
\end{malayalam}}
\flushright{\begin{Arabic}
\quranayah[26][173]
\end{Arabic}}
\flushleft{\begin{malayalam}
അവരുടെമേല്‍ നാം ഒരുതരം മഴ വീഴ്ത്തി. താക്കീത് നല്‍കപ്പെട്ടവര്‍ക്ക് കിട്ടിയ ആ മഴ എത്ര ഭീകരം!
\end{malayalam}}
\flushright{\begin{Arabic}
\quranayah[26][174]
\end{Arabic}}
\flushleft{\begin{malayalam}
തീര്‍ച്ചയായും അതിലൊരു ദൃഷ്ടാന്തമുണ്ട്. എന്നിട്ടും അവരിലേറെപ്പേരും വിശ്വാസികളായില്ല.
\end{malayalam}}
\flushright{\begin{Arabic}
\quranayah[26][175]
\end{Arabic}}
\flushleft{\begin{malayalam}
സംശയമില്ല, നിന്റെ നാഥന്‍ ഏറെ പ്രതാപിയും പരമദയാലുവും തന്നെ.
\end{malayalam}}
\flushright{\begin{Arabic}
\quranayah[26][176]
\end{Arabic}}
\flushleft{\begin{malayalam}
“ഐക്ക” നിവാസികള്‍ ദൈവദൂതന്മാരെ തള്ളിപ്പറഞ്ഞു.
\end{malayalam}}
\flushright{\begin{Arabic}
\quranayah[26][177]
\end{Arabic}}
\flushleft{\begin{malayalam}
ശുഐബ് അവരോടു ചോദിച്ച സന്ദര്‍ഭം: "നിങ്ങള്‍ ദൈവത്തോടു ഭക്തിപുലര്‍ത്തുന്നില്ലേ?
\end{malayalam}}
\flushright{\begin{Arabic}
\quranayah[26][178]
\end{Arabic}}
\flushleft{\begin{malayalam}
"ഞാന്‍ നിങ്ങളിലേക്ക് നിയോഗിതനായ വിശ്വസ്തനായ ദൈവദൂതനാണ്.
\end{malayalam}}
\flushright{\begin{Arabic}
\quranayah[26][179]
\end{Arabic}}
\flushleft{\begin{malayalam}
"അതിനാല്‍ നിങ്ങള്‍ ദൈവഭക്തരാവുക. എന്നെ അനുസരിക്കുക.
\end{malayalam}}
\flushright{\begin{Arabic}
\quranayah[26][180]
\end{Arabic}}
\flushleft{\begin{malayalam}
"ഇതിന്റെ പേരില്‍ ഞാന്‍ നിങ്ങളോട് ഒരു പ്രതിഫലവും ആവശ്യപ്പെടുന്നില്ല. എനിക്കുള്ള പ്രതിഫലം പ്രപഞ്ചനാഥന്റെ വശം മാത്രമാണ്.
\end{malayalam}}
\flushright{\begin{Arabic}
\quranayah[26][181]
\end{Arabic}}
\flushleft{\begin{malayalam}
"നിങ്ങള്‍ അളവില്‍ തികവുവരുത്തുക. അളവില്‍ കുറവുവരുത്തുന്നവരില്‍ പെട്ടുപോകരുത്.
\end{malayalam}}
\flushright{\begin{Arabic}
\quranayah[26][182]
\end{Arabic}}
\flushleft{\begin{malayalam}
"കൃത്യതയുള്ള തുലാസുകളില്‍ തൂക്കുക.
\end{malayalam}}
\flushright{\begin{Arabic}
\quranayah[26][183]
\end{Arabic}}
\flushleft{\begin{malayalam}
"ജനങ്ങള്‍ക്ക് അവരുടെ ചരക്കുകളില്‍ കുറവുവരുത്തരുത്. നാട്ടില്‍ കുഴപ്പക്കാരായി വിഹരിക്കരുത്.
\end{malayalam}}
\flushright{\begin{Arabic}
\quranayah[26][184]
\end{Arabic}}
\flushleft{\begin{malayalam}
"നിങ്ങളെയും മുന്‍തലമുറകളെയും സൃഷ്ടിച്ച അല്ലാഹുവോട് ഭക്തിയുള്ളവരാവുക.”
\end{malayalam}}
\flushright{\begin{Arabic}
\quranayah[26][185]
\end{Arabic}}
\flushleft{\begin{malayalam}
അവര്‍ പറഞ്ഞു: "നീ മാരണം ബാധിച്ച ഒരുത്തന്‍ മാത്രമാണ്.
\end{malayalam}}
\flushright{\begin{Arabic}
\quranayah[26][186]
\end{Arabic}}
\flushleft{\begin{malayalam}
"നീ ഞങ്ങളെപ്പോലുള്ള ഒരു മനുഷ്യനല്ലാതാരുമല്ല. കള്ളം പറയുന്നവനായാണ് നിന്നെ ഞങ്ങള്‍ കരുതുന്നത്.
\end{malayalam}}
\flushright{\begin{Arabic}
\quranayah[26][187]
\end{Arabic}}
\flushleft{\begin{malayalam}
"ആകാശത്തിന്റെ ചില കഷണങ്ങള്‍ ഞങ്ങള്‍ക്കുമേല്‍ വീഴ്ത്തുക, നീ സത്യവാദിയെങ്കില്‍.”
\end{malayalam}}
\flushright{\begin{Arabic}
\quranayah[26][188]
\end{Arabic}}
\flushleft{\begin{malayalam}
അദ്ദേഹം പറഞ്ഞു: "നിങ്ങള്‍ ചെയ്യുന്നതിനെപ്പറ്റി നന്നായറിയുന്നവനാണ് എന്റെ നാഥന്‍.”
\end{malayalam}}
\flushright{\begin{Arabic}
\quranayah[26][189]
\end{Arabic}}
\flushleft{\begin{malayalam}
അങ്ങനെ അവരദ്ദേഹത്തെ തള്ളിപ്പറഞ്ഞു. അതിനാല്‍ കാര്‍മേഘം കുടപിടിച്ച നാളിന്റെ ശിക്ഷ അവരെ പിടികൂടി. ഭയങ്കരമായ ഒരു നാളിന്റെ ശിക്ഷ തന്നെയായിരുന്നു അത്.
\end{malayalam}}
\flushright{\begin{Arabic}
\quranayah[26][190]
\end{Arabic}}
\flushleft{\begin{malayalam}
തീര്‍ച്ചയായും അതിലൊരു ദൃഷ്ടാന്തമുണ്ട്. എന്നിട്ടും അവരിലേറെ പേരും വിശ്വസിക്കുന്നവരായില്ല.
\end{malayalam}}
\flushright{\begin{Arabic}
\quranayah[26][191]
\end{Arabic}}
\flushleft{\begin{malayalam}
നിശ്ചയം, നിന്റെ നാഥന്‍ ഏറെ പ്രതാപിയും പരമദയാലുവുമാണ്.
\end{malayalam}}
\flushright{\begin{Arabic}
\quranayah[26][192]
\end{Arabic}}
\flushleft{\begin{malayalam}
തീര്‍ച്ചയായും ഇത് പ്രപഞ്ചനാഥനില്‍ നിന്ന് അവതരിച്ചുകിട്ടിയതാണ്.
\end{malayalam}}
\flushright{\begin{Arabic}
\quranayah[26][193]
\end{Arabic}}
\flushleft{\begin{malayalam}
വിശ്വസ്തനായ ആത്മാവാണ് അതുമായി ഇറങ്ങിയത്.
\end{malayalam}}
\flushright{\begin{Arabic}
\quranayah[26][194]
\end{Arabic}}
\flushleft{\begin{malayalam}
നിന്റെ ഹൃദയത്തിലാണിതിറക്കിത്തന്നത്. നീ താക്കീതു നല്‍കുന്നവരിലുള്‍പ്പെടാന്‍.
\end{malayalam}}
\flushright{\begin{Arabic}
\quranayah[26][195]
\end{Arabic}}
\flushleft{\begin{malayalam}
തെളിഞ്ഞ അറബിഭാഷയിലാണിത്.
\end{malayalam}}
\flushright{\begin{Arabic}
\quranayah[26][196]
\end{Arabic}}
\flushleft{\begin{malayalam}
പൂര്‍വികരുടെ വേദപുസ്തകങ്ങളിലും ഇതുണ്ട്.
\end{malayalam}}
\flushright{\begin{Arabic}
\quranayah[26][197]
\end{Arabic}}
\flushleft{\begin{malayalam}
ഇസ്രയേല്‍ മക്കളിലെ പണ്ഡിതന്മാര്‍ക്ക് അതറിയാം എന്നതുതന്നെ ഇവര്‍ക്ക് ഒരു ദൃഷ്ടാന്തമല്ലേ?
\end{malayalam}}
\flushright{\begin{Arabic}
\quranayah[26][198]
\end{Arabic}}
\flushleft{\begin{malayalam}
നാമിത് അനറബികളില്‍ ആര്‍ക്കെങ്കിലുമാണ് ഇറക്കിക്കൊടുത്തതെന്ന് കരുതുക.
\end{malayalam}}
\flushright{\begin{Arabic}
\quranayah[26][199]
\end{Arabic}}
\flushleft{\begin{malayalam}
അങ്ങനെ അയാളത് അവരെ വായിച്ചുകേള്‍പ്പിച്ചുവെന്നും; എന്നാലും അവരിതില്‍ വിശ്വസിക്കുമായിരുന്നില്ല.
\end{malayalam}}
\flushright{\begin{Arabic}
\quranayah[26][200]
\end{Arabic}}
\flushleft{\begin{malayalam}
അവ്വിധം നാമിതിനെ കുറ്റവാളികളുടെ ഹൃദയങ്ങളില്‍ കടത്തിവിട്ടിരിക്കുന്നു.
\end{malayalam}}
\flushright{\begin{Arabic}
\quranayah[26][201]
\end{Arabic}}
\flushleft{\begin{malayalam}
നോവേറിയശിക്ഷ കാണുംവരെ അവരിതില്‍ വിശ്വസിക്കുകയില്ല.
\end{malayalam}}
\flushright{\begin{Arabic}
\quranayah[26][202]
\end{Arabic}}
\flushleft{\begin{malayalam}
അവരറിയാത്ത നേരത്ത് വളരെ പെട്ടെന്നായിരിക്കും അതവരില്‍ വന്നെത്തുക.
\end{malayalam}}
\flushright{\begin{Arabic}
\quranayah[26][203]
\end{Arabic}}
\flushleft{\begin{malayalam}
അപ്പോഴവര്‍ പറയും: "ഞങ്ങള്‍ക്കൊരിത്തിരി അവധി കിട്ടുമോ?”
\end{malayalam}}
\flushright{\begin{Arabic}
\quranayah[26][204]
\end{Arabic}}
\flushleft{\begin{malayalam}
എന്നിട്ടും ഇക്കൂട്ടര്‍ നമ്മുടെ ശിക്ഷ കിട്ടാനാണോ ഇത്ര ധൃതികൂട്ടുന്നത്?
\end{malayalam}}
\flushright{\begin{Arabic}
\quranayah[26][205]
\end{Arabic}}
\flushleft{\begin{malayalam}
നീ ചിന്തിച്ചുനോക്കിയോ: നാം അവര്‍ക്ക് കൊല്ലങ്ങളോളം കഴിയാന്‍ ആവശ്യമായ സുഖസൌകര്യങ്ങള്‍ നല്‍കിയെന്നുവെക്കുക;
\end{malayalam}}
\flushright{\begin{Arabic}
\quranayah[26][206]
\end{Arabic}}
\flushleft{\begin{malayalam}
പിന്നീട് അവര്‍ക്ക് മുന്നറിയിപ്പ് നല്‍കിക്കൊണ്ടിരിക്കുന്ന ശിക്ഷ അവരില്‍ വന്നെത്തിയെന്നും.
\end{malayalam}}
\flushright{\begin{Arabic}
\quranayah[26][207]
\end{Arabic}}
\flushleft{\begin{malayalam}
എന്നാലും അവര്‍ക്കു സുഖിച്ചുകഴിയാന്‍ നല്‍കിയ സൌകര്യം അവര്‍ക്കൊട്ടും ഉപകരിക്കുമായിരുന്നില്ല.
\end{malayalam}}
\flushright{\begin{Arabic}
\quranayah[26][208]
\end{Arabic}}
\flushleft{\begin{malayalam}
മുന്നറിയിപ്പുകാരനെ അയച്ചിട്ടല്ലാതെ ഒരു നാടിനെയും നാം നശിപ്പിച്ചിട്ടില്ല.
\end{malayalam}}
\flushright{\begin{Arabic}
\quranayah[26][209]
\end{Arabic}}
\flushleft{\begin{malayalam}
അവരെ ഉദ്ബോധിപ്പിക്കാനാണിത്. നാം ആരോടും ഒട്ടും അതിക്രമം കാണിക്കുന്നവനല്ല.
\end{malayalam}}
\flushright{\begin{Arabic}
\quranayah[26][210]
\end{Arabic}}
\flushleft{\begin{malayalam}
ഈ ഖുര്‍ആന്‍ ഇറക്കിക്കൊണ്ടുവന്നത് പിശാചുക്കളല്ല.
\end{malayalam}}
\flushright{\begin{Arabic}
\quranayah[26][211]
\end{Arabic}}
\flushleft{\begin{malayalam}
അതവര്‍ക്കു ചേര്‍ന്നതല്ല. അവര്‍ക്കതൊട്ടു സാധ്യവുമല്ല.
\end{malayalam}}
\flushright{\begin{Arabic}
\quranayah[26][212]
\end{Arabic}}
\flushleft{\begin{malayalam}
അവരിത് കേള്‍ക്കുന്നതില്‍ നിന്നുപോലും അകറ്റിനിര്‍ത്തപ്പെട്ടവരാണ്.
\end{malayalam}}
\flushright{\begin{Arabic}
\quranayah[26][213]
\end{Arabic}}
\flushleft{\begin{malayalam}
അതിനാല്‍ അല്ലാഹുവോടൊപ്പം മറ്റൊരു ദൈവത്തെയും നീ വിളിച്ചുപ്രാര്‍ഥിക്കരുത്. അങ്ങനെ ചെയ്താല്‍ നീയും ശിക്ഷാര്‍ഹരില്‍പെടും.
\end{malayalam}}
\flushright{\begin{Arabic}
\quranayah[26][214]
\end{Arabic}}
\flushleft{\begin{malayalam}
നീ നിന്റെ അടുത്തബന്ധുക്കള്‍ക്ക് മുന്നറിയിപ്പ് നല്‍കുക.
\end{malayalam}}
\flushright{\begin{Arabic}
\quranayah[26][215]
\end{Arabic}}
\flushleft{\begin{malayalam}
നിന്നെ പിന്‍പറ്റിയ സത്യവിശ്വാസികള്‍ക്ക് നിന്റെ ചിറക് താഴ്ത്തിക്കൊടുക്കുകയും ചെയ്യുക.
\end{malayalam}}
\flushright{\begin{Arabic}
\quranayah[26][216]
\end{Arabic}}
\flushleft{\begin{malayalam}
അഥവാ, അവര്‍ നിന്നെ ധിക്കരിക്കുകയാണെങ്കില്‍ പറയുക: "നിങ്ങള്‍ ചെയ്യുന്നതിനൊന്നും ഞാനുത്തരവാദിയല്ല.”
\end{malayalam}}
\flushright{\begin{Arabic}
\quranayah[26][217]
\end{Arabic}}
\flushleft{\begin{malayalam}
പ്രതാപിയും ദയാപരനുമായ അല്ലാഹുവില്‍ ഭരമേല്‍പിക്കുക.
\end{malayalam}}
\flushright{\begin{Arabic}
\quranayah[26][218]
\end{Arabic}}
\flushleft{\begin{malayalam}
നീ നിന്നു പ്രാര്‍ഥിക്കുംനേരത്ത് നിന്നെ കാണുന്നവനാണവന്‍.
\end{malayalam}}
\flushright{\begin{Arabic}
\quranayah[26][219]
\end{Arabic}}
\flushleft{\begin{malayalam}
സാഷ്ടാംഗം പ്രണമിക്കുന്നവരില്‍ നിന്റെ ചലനങ്ങള്‍ കാണുന്നവനും.
\end{malayalam}}
\flushright{\begin{Arabic}
\quranayah[26][220]
\end{Arabic}}
\flushleft{\begin{malayalam}
തീര്‍ച്ചയായും അവന്‍ എല്ലാം കേള്‍ക്കുന്നവനും അറിയുന്നവനുമാണ്.
\end{malayalam}}
\flushright{\begin{Arabic}
\quranayah[26][221]
\end{Arabic}}
\flushleft{\begin{malayalam}
പിശാചുക്കള്‍ വന്നിറങ്ങുന്നത് ആരിലാണെന്ന് നാം നിങ്ങളെ അറിയിച്ചുതരട്ടെയോ?
\end{malayalam}}
\flushright{\begin{Arabic}
\quranayah[26][222]
\end{Arabic}}
\flushleft{\begin{malayalam}
തനി നുണയന്മാരും കുറ്റവാളികളുമായ എല്ലാവരിലുമാണ് പിശാച് വന്നിറങ്ങുന്നത്.
\end{malayalam}}
\flushright{\begin{Arabic}
\quranayah[26][223]
\end{Arabic}}
\flushleft{\begin{malayalam}
അവര്‍ പിശാചുക്കളുടെ വാക്കുകള്‍ കാതോര്‍ത്ത് കേള്‍ക്കുന്നു. അവരിലേറെപ്പേരും കള്ളംപറയുന്നവരാണ്.
\end{malayalam}}
\flushright{\begin{Arabic}
\quranayah[26][224]
\end{Arabic}}
\flushleft{\begin{malayalam}
വഴിപിഴച്ചവരാണ് കവികളെ പിന്‍പറ്റുന്നത്.
\end{malayalam}}
\flushright{\begin{Arabic}
\quranayah[26][225]
\end{Arabic}}
\flushleft{\begin{malayalam}
നീ കാണുന്നില്ലേ; അവര്‍ സകല താഴ്വരകളിലും അലഞ്ഞുതിരിയുന്നത്;
\end{malayalam}}
\flushright{\begin{Arabic}
\quranayah[26][226]
\end{Arabic}}
\flushleft{\begin{malayalam}
തങ്ങള്‍ ചെയ്യാത്തത് പറയുന്നതും.
\end{malayalam}}
\flushright{\begin{Arabic}
\quranayah[26][227]
\end{Arabic}}
\flushleft{\begin{malayalam}
സത്യവിശ്വാസം സ്വീകരിക്കുകയും സല്‍ക്കര്‍മങ്ങള്‍ പ്രവര്‍ത്തിക്കുകയും ദൈവത്തെ ധാരാളമായി സ്മരിക്കുകയും തങ്ങള്‍ അക്രമിക്കപ്പെട്ടശേഷം അതിനെ നേരിടുക മാത്രം ചെയ്തവരുമൊഴികെ. അതിക്രമികള്‍ അടുത്തുതന്നെ അറിയും, തങ്ങള്‍ മാറിമറിഞ്ഞ് ഏതൊരു പരിണതിയിലാണ് എത്തുകയെന്ന്.
\end{malayalam}}
\chapter{\textmalayalam{നംല്‍ ( ഉറുമ്പ് )}}
\begin{Arabic}
\Huge{\centerline{\basmalah}}\end{Arabic}
\flushright{\begin{Arabic}
\quranayah[27][1]
\end{Arabic}}
\flushleft{\begin{malayalam}
ത്വാ-സീന്‍. ഇത് ഖുര്‍ആന്റെയും സുവ്യക്തമായ വേദപുസ്തകത്തിന്റെയും വചനങ്ങളാണ്.
\end{malayalam}}
\flushright{\begin{Arabic}
\quranayah[27][2]
\end{Arabic}}
\flushleft{\begin{malayalam}
സത്യവിശ്വാസികള്‍ക്ക് നേര്‍വഴി കാണിക്കുന്നതും ശുഭവാര്‍ത്ത അറിയിക്കുന്നതുമാണ്.
\end{malayalam}}
\flushright{\begin{Arabic}
\quranayah[27][3]
\end{Arabic}}
\flushleft{\begin{malayalam}
അവര്‍ നമസ്കാരം നിഷ്ഠയോടെ നിര്‍വഹിക്കുന്നവരും സകാത്ത് നല്‍കുന്നവരുമാണ്. പരലോകത്തില്‍ അടിയുറച്ചുവിശ്വസിക്കുന്നവരും.
\end{malayalam}}
\flushright{\begin{Arabic}
\quranayah[27][4]
\end{Arabic}}
\flushleft{\begin{malayalam}
പരലോകത്തില്‍ വിശ്വസിക്കാത്തവര്‍ക്കു നാം അവരുടെ ചെയ്തികള്‍ ചേതോഹരമായി തോന്നിപ്പിക്കുന്നു. അങ്ങനെ അവര്‍ വിഭ്രാന്തരായി ഉഴറിനടക്കുന്നു.
\end{malayalam}}
\flushright{\begin{Arabic}
\quranayah[27][5]
\end{Arabic}}
\flushleft{\begin{malayalam}
അവര്‍ക്കാണ് കൊടിയ ശിക്ഷയുള്ളത്. പരലോകത്ത് പറ്റെ പരാജയപ്പെടുന്നവരും അവര്‍ തന്നെ.
\end{malayalam}}
\flushright{\begin{Arabic}
\quranayah[27][6]
\end{Arabic}}
\flushleft{\begin{malayalam}
നിശ്ചയം; യുക്തിജ്ഞനും സര്‍വജ്ഞനുമായവനില്‍ നിന്നാണ് നീ ഈ ഖുര്‍ആന്‍ സ്വീകരിച്ചുകൊണ്ടിരിക്കുന്നത്.
\end{malayalam}}
\flushright{\begin{Arabic}
\quranayah[27][7]
\end{Arabic}}
\flushleft{\begin{malayalam}
മൂസ തന്റെ കുടുംബത്തോടുപറഞ്ഞ സന്ദര്‍ഭം: "തീര്‍ച്ചയായും ഞാന്‍ തീ കാണുന്നുണ്ട്. ഞാനവിടെനിന്ന് വല്ല വിവരവുമായി വരാം. അല്ലെങ്കില്‍ തീനാളം കൊളുത്തി നിങ്ങള്‍ക്കെത്തിച്ചുതരാം. നിങ്ങള്‍ക്ക് തീക്കായാമല്ലോ.”
\end{malayalam}}
\flushright{\begin{Arabic}
\quranayah[27][8]
\end{Arabic}}
\flushleft{\begin{malayalam}
അങ്ങനെ അദ്ദേഹം അതിനടുത്തുചെന്നു. അപ്പോള്‍ ഇങ്ങനെയൊരു വിളംബരം കേട്ടു: "തീയിലുള്ളവരും അതിന്റെ ചുറ്റുമുള്ളവരും ഏറെ അനുഗൃഹീതരാണ്. പ്രപഞ്ചനാഥനായ അല്ലാഹു എത്ര പരിശുദ്ധന്‍.
\end{malayalam}}
\flushright{\begin{Arabic}
\quranayah[27][9]
\end{Arabic}}
\flushleft{\begin{malayalam}
"ഓ മൂസാ; നിശ്ചയം, ഞാന്‍ അല്ലാഹുവാണ്. പ്രതാപിയും യുക്തിജ്ഞനും.
\end{malayalam}}
\flushright{\begin{Arabic}
\quranayah[27][10]
\end{Arabic}}
\flushleft{\begin{malayalam}
"നിന്റെ വടി താഴെയിടൂ.” അങ്ങനെ അതൊരു പാമ്പിനെപ്പോലെ പുളയാന്‍ തുടങ്ങി. ഇതുകണ്ടപ്പോള്‍ മൂസ പിന്തിരിഞ്ഞോടി. തിരിഞ്ഞുനോക്കിയതുപോലുമില്ല. അല്ലാഹു പറഞ്ഞു: "മൂസാ, പേടിക്കേണ്ട. എന്റെ അടുത്ത് ദൈവദൂതന്മാര്‍ ഭയപ്പെടാറില്ല;
\end{malayalam}}
\flushright{\begin{Arabic}
\quranayah[27][11]
\end{Arabic}}
\flushleft{\begin{malayalam}
"അതിക്രമം പ്രവര്‍ത്തിച്ചവരൊഴികെ. പിന്നെ തിന്മക്കു പിറകെ പകരം നന്മ കൊണ്ടുവരികയാണെങ്കില്‍; ഞാന്‍ ഏറെ പൊറുക്കുന്നവനും പരമദയാലുവും തന്നെ.
\end{malayalam}}
\flushright{\begin{Arabic}
\quranayah[27][12]
\end{Arabic}}
\flushleft{\begin{malayalam}
"നീ നിന്റെ കൈ കുപ്പായത്തിന്റെ മാറിനുള്ളില്‍ തിരുകിവെക്കുക. എന്നാല്‍ ന്യൂനതയൊട്ടുമില്ലാത്തവിധം തിളക്കമുള്ളതായി അതു പുറത്തുവരും. ഫറവോന്റെയും അവന്റെ ജനതയുടെയും അടുക്കലേക്കുള്ള ഒമ്പതു ദൃഷ്ടാന്തങ്ങളില്‍ പെട്ടതാണിവ. തീര്‍ച്ചയായും അവര്‍ തെമ്മാടികളായ ജനമാണ്.”
\end{malayalam}}
\flushright{\begin{Arabic}
\quranayah[27][13]
\end{Arabic}}
\flushleft{\begin{malayalam}
അങ്ങനെ കണ്ണു തുറപ്പിക്കാന്‍പോന്ന നമ്മുടെ ദൃഷ്ടാന്തങ്ങള്‍ അവര്‍ക്കു വന്നെത്തിയപ്പോള്‍ അവര്‍ പറഞ്ഞു: "ഇതു വളരെ പ്രകടമായ ജാലവിദ്യ തന്നെ.”
\end{malayalam}}
\flushright{\begin{Arabic}
\quranayah[27][14]
\end{Arabic}}
\flushleft{\begin{malayalam}
അവരുടെ മനസ്സുകള്‍ക്ക് ആ ദൃഷ്ടാന്തങ്ങള്‍ നന്നായി ബോധ്യമായിരുന്നു. എന്നിട്ടും അക്രമവും അഹങ്കാരവും കാരണം അവര്‍അവയെ തള്ളിപ്പറഞ്ഞു. നോക്കൂ; ആ നാശകാരികളുടെ ഒടുക്കം എവ്വിധമായിരുന്നുവെന്ന്.
\end{malayalam}}
\flushright{\begin{Arabic}
\quranayah[27][15]
\end{Arabic}}
\flushleft{\begin{malayalam}
ദാവൂദിനും സുലൈമാന്നും നാം ജ്ഞാനം നല്‍കി. അവരിരുവരും പറഞ്ഞു: "വിശ്വാസികളായ തന്റെ ദാസന്മാരില്‍ മറ്റുപലരെക്കാളും ഞങ്ങള്‍ക്കു ശ്രേഷ്ഠത നല്‍കിയ അല്ലാഹുവിനാണ് സര്‍വസ്തുതിയും.
\end{malayalam}}
\flushright{\begin{Arabic}
\quranayah[27][16]
\end{Arabic}}
\flushleft{\begin{malayalam}
സുലൈമാന്‍ ദാവൂദിന്റെ അനന്തരാവകാശിയായി. അദ്ദേഹം പറഞ്ഞു: "ജനങ്ങളേ, പക്ഷികളുടെ ഭാഷ നമ്മെ പഠിപ്പിച്ചിരിക്കുന്നു. ആവശ്യമായ എല്ലാം നമുക്ക് നല്‍കിയിരിക്കുന്നു. ഇതുതന്നെയാണ് പ്രത്യക്ഷമായ ദിവ്യാനുഗ്രഹം.”
\end{malayalam}}
\flushright{\begin{Arabic}
\quranayah[27][17]
\end{Arabic}}
\flushleft{\begin{malayalam}
സുലൈമാന്നുവേണ്ടി മനുഷ്യരിലെയും ജിന്നുകളിലെയും പക്ഷികളിലെയും തന്റെ സൈന്യങ്ങളെ സംഘടിപ്പിച്ചു. എന്നിട്ടവയെ യഥാവിധി ക്രമീകരിച്ചു.
\end{malayalam}}
\flushright{\begin{Arabic}
\quranayah[27][18]
\end{Arabic}}
\flushleft{\begin{malayalam}
അങ്ങനെ അവരെല്ലാം ഉറുമ്പുകളുടെ താഴ്വരയിലെത്തി. അപ്പോള്‍ ഒരുറുമ്പ് പറഞ്ഞു: "ഹേ, ഉറുമ്പുകളേ, നിങ്ങള്‍ നിങ്ങളുടെ മാളങ്ങളില്‍ പ്രവേശിച്ചുകൊള്ളുക. സുലൈമാനും സൈന്യവും അവരറിയാതെ നിങ്ങളെ ചവുട്ടിത്തേച്ചുകളയാനിടവരാതിരിക്കട്ടെ.”
\end{malayalam}}
\flushright{\begin{Arabic}
\quranayah[27][19]
\end{Arabic}}
\flushleft{\begin{malayalam}
അതിന്റെ വാക്കുകേട്ട് സുലൈമാന്‍ മന്ദഹസിച്ചു. അദ്ദേഹം പറഞ്ഞു: "എന്റെ നാഥാ, എനിക്കും എന്റെ മാതാപിതാക്കള്‍ക്കും നീ ചെയ്തുതന്ന അനുഗ്രഹങ്ങള്‍ക്ക് നന്ദി കാണിക്കാനും നിനക്കിഷ്ടപ്പെട്ട സല്‍ക്കര്‍മങ്ങള്‍ പ്രവര്‍ത്തിക്കാനും എനിക്കു നീ അവസരമേകേണമേ. നിന്റെ അനുഗ്രഹത്താല്‍ സച്ചരിതരായ നിന്റെ ദാസന്മാരില്‍ എനിക്കും നീ ഇടം നല്‍കേണമേ.”
\end{malayalam}}
\flushright{\begin{Arabic}
\quranayah[27][20]
\end{Arabic}}
\flushleft{\begin{malayalam}
സുലൈമാന്‍ പക്ഷികളെ പരിശോധിച്ചു. അപ്പോള്‍ അദ്ദേഹം പറഞ്ഞു: "ഇതെന്തുപറ്റി? ആ മരംകൊത്തിയെ ഞാന്‍ കാണുന്നില്ലല്ലോ. അത് എവിടെയെങ്കിലും അപ്രത്യക്ഷമായോ?
\end{malayalam}}
\flushright{\begin{Arabic}
\quranayah[27][21]
\end{Arabic}}
\flushleft{\begin{malayalam}
"അതിനെ ഞാന്‍ കഠിനമായി ശിക്ഷിക്കും. അല്ലെങ്കില്‍ അറുത്തുകളയും. അതുമല്ലെങ്കില്‍ വ്യക്തമായ വല്ല ന്യായവും അതെനിക്കു സമര്‍പ്പിക്കണം.”
\end{malayalam}}
\flushright{\begin{Arabic}
\quranayah[27][22]
\end{Arabic}}
\flushleft{\begin{malayalam}
എന്നാല്‍ ഏറെക്കഴിയുംമുമ്പെ അതെത്തിച്ചേര്‍ന്നു. അപ്പോള്‍ അതു പറഞ്ഞു: "അങ്ങയ്ക്കറിയാത്ത ചില കാര്യങ്ങള്‍ ഞാന്‍ സൂക്ഷ്മമായി മനസ്സിലാക്കിയിരിക്കുന്നു. "സബഇല്‍ നിന്ന് ഉറപ്പുള്ള ചില വാര്‍ത്തകളുമായാണ് ഞാന്‍ വന്നിരിക്കുന്നത്.
\end{malayalam}}
\flushright{\begin{Arabic}
\quranayah[27][23]
\end{Arabic}}
\flushleft{\begin{malayalam}
"ഞാന്‍ അവിടെ ഒരു സ്ത്രീയെ കണ്ടു. അവരാണ് അന്നാട്ടുകാരെ ഭരിക്കുന്നത്. അവര്‍ക്ക് സകല സൌകര്യങ്ങളും അവിടെയുണ്ട്. ഗംഭീരമായ ഒരു സിംഹാസനവും.
\end{malayalam}}
\flushright{\begin{Arabic}
\quranayah[27][24]
\end{Arabic}}
\flushleft{\begin{malayalam}
"അവരും അവരുടെ ജനതയും അല്ലാഹുവിനു പുറമെ സൂര്യനെ സാഷ്ടാംഗം പ്രണമിക്കുന്നതായി ഞാന്‍ കണ്ടു.” പിശാച് അവര്‍ക്ക് തങ്ങളുടെ ചെയ്തികളാകെ ചേതോഹരങ്ങളായി തോന്നിപ്പിച്ചിരിക്കുന്നു. അവന്‍ അവരെ നേര്‍വഴിയില്‍ നിന്ന് തടഞ്ഞു. അതിനാലവര്‍ നേര്‍വഴി പ്രാപിക്കുന്നില്ല.
\end{malayalam}}
\flushright{\begin{Arabic}
\quranayah[27][25]
\end{Arabic}}
\flushleft{\begin{malayalam}
ആകാശഭൂമികളില്‍ മറഞ്ഞുകിടക്കുന്നവയെ പുറത്തുകൊണ്ടുവരികയും നിങ്ങള്‍ മറച്ചുവെക്കുന്നതും വെളിപ്പെടുത്തുന്നതുമായ എല്ലാം അറിയുകയും ചെയ്യുന്ന അല്ലാഹുവിന് സാഷ്ടാംഗം പ്രണമിക്കാതിരിക്കാനാണ് പിശാച് അത് ചെയ്തത്.
\end{malayalam}}
\flushright{\begin{Arabic}
\quranayah[27][26]
\end{Arabic}}
\flushleft{\begin{malayalam}
അല്ലാഹു, അവനല്ലാതെ ദൈവമില്ല. അതിമഹത്തായ സിംഹാസനത്തിന്റെ അധിപനാണവന്‍.
\end{malayalam}}
\flushright{\begin{Arabic}
\quranayah[27][27]
\end{Arabic}}
\flushleft{\begin{malayalam}
സുലൈമാന്‍ പറഞ്ഞു: "നാമൊന്നു നോക്കട്ടെ; നീ പറഞ്ഞത് സത്യമാണോ; അതല്ല നീ കള്ളം പറയുന്നവരില്‍ പെട്ടവനാണോ എന്ന്.
\end{malayalam}}
\flushright{\begin{Arabic}
\quranayah[27][28]
\end{Arabic}}
\flushleft{\begin{malayalam}
"നീ എന്റെ ഈ എഴുത്തുകൊണ്ടുപോയി അവര്‍ക്കിട്ടുകൊടുക്കുക. പിന്നെ അവരില്‍നിന്ന് മാറിനില്‍ക്കുക. എന്നിട്ട് അവരെന്തു മറുപടിയാണ് തരുന്നതെന്ന് നോക്കുക.”
\end{malayalam}}
\flushright{\begin{Arabic}
\quranayah[27][29]
\end{Arabic}}
\flushleft{\begin{malayalam}
ആ രാജ്ഞി പറഞ്ഞു: "അല്ലയോ നേതാക്കളേ, മാന്യമായ ഒരെഴുത്ത് എനിക്കിതാ വന്നെത്തിയിരിക്കുന്നു.
\end{malayalam}}
\flushright{\begin{Arabic}
\quranayah[27][30]
\end{Arabic}}
\flushleft{\begin{malayalam}
"അത് സുലൈമാനില്‍ നിന്നുള്ളതാണ്. പരമകാരുണികനും ദയാപരനുമായ അല്ലാഹുവിന്റെ നാമത്തില്‍ ആരംഭിക്കുന്നതും.
\end{malayalam}}
\flushright{\begin{Arabic}
\quranayah[27][31]
\end{Arabic}}
\flushleft{\begin{malayalam}
"അതിലുള്ളതിതാണ്: നിങ്ങള്‍ എനിക്കെതിരെ ധിക്കാരം കാണിക്കരുത്. മുസ്ലിംകളായി എന്റെ അടുത്തുവരികയും വേണം.”
\end{malayalam}}
\flushright{\begin{Arabic}
\quranayah[27][32]
\end{Arabic}}
\flushleft{\begin{malayalam}
രാജ്ഞി പറഞ്ഞു: "അല്ലയോ നേതാക്കളേ, ഇക്കാര്യത്തില്‍ നിങ്ങളെനിക്ക് ആവശ്യമായ നിര്‍ദേശം തരിക. നിങ്ങളെക്കൂടാതെ ഒരു കാര്യവും ഖണ്ഡിതമായി തീരുമാനിക്കുന്നവളല്ലല്ലോ ഞാന്‍.”
\end{malayalam}}
\flushright{\begin{Arabic}
\quranayah[27][33]
\end{Arabic}}
\flushleft{\begin{malayalam}
അവര്‍ പറഞ്ഞു: "നാമിപ്പോള്‍ പ്രബലരും പരാക്രമശാലികളുമാണല്ലോ. ഇനി തീരുമാനം അങ്ങയുടേതുതന്നെ. അതിനാല്‍ എന്തു കല്‍പിക്കണമെന്ന് അങ്ങുതന്നെ ആലോചിച്ചുനോക്കുക.”
\end{malayalam}}
\flushright{\begin{Arabic}
\quranayah[27][34]
\end{Arabic}}
\flushleft{\begin{malayalam}
രാജ്ഞി പറഞ്ഞു: "രാജാക്കന്മാര്‍ ഒരു നാട്ടില്‍ പ്രവേശിച്ചാല്‍ അവരവിടം നശിപ്പിക്കും. അവിടത്തുകാരിലെ അന്തസ്സുള്ളവരെ അപമാനിതരാക്കും. അങ്ങനെയാണ് അവര്‍ ചെയ്യാറുള്ളത്.
\end{malayalam}}
\flushright{\begin{Arabic}
\quranayah[27][35]
\end{Arabic}}
\flushleft{\begin{malayalam}
"ഞാന്‍ അവര്‍ക്ക് ഒരു പാരിതോഷികം കൊടുത്തയക്കട്ടെ. എന്നിട്ട് നമ്മുടെ ദൂതന്മാര്‍ എന്തു മറുപടിയുമായാണ് മടങ്ങിവരുന്നതെന്ന് നോക്കാം.”
\end{malayalam}}
\flushright{\begin{Arabic}
\quranayah[27][36]
\end{Arabic}}
\flushleft{\begin{malayalam}
അങ്ങനെ അവരുടെ ദൂതന്‍ സുലൈമാന്റെ അടുത്തുചെന്നപ്പോള്‍ അദ്ദേഹം പറഞ്ഞു: "നിങ്ങളെന്നെ സമ്പത്ത് തന്ന് സഹായിച്ചുകളയാമെന്നാണോ കരുതുന്നത്? എന്നാല്‍ അല്ലാഹു എനിക്കു തന്നത് നിങ്ങള്‍ക്ക് അവന്‍ തന്നതിനെക്കാള്‍ എത്രയോ മികച്ചതാണ്. എന്നിട്ടും നിങ്ങള്‍ നിങ്ങളുടെ പാരിതോഷികത്തില്‍ ഊറ്റംകൊള്ളുകയാണ്.
\end{malayalam}}
\flushright{\begin{Arabic}
\quranayah[27][37]
\end{Arabic}}
\flushleft{\begin{malayalam}
"നീ അവരിലേക്കുതന്നെ തിരിച്ചുപോവുക. നാം പട്ടാളത്തെ കൂട്ടി അവരുടെ അടുത്തെത്തും; തീര്‍ച്ച. അതിനെ നേരിടാന്‍ അവര്‍ക്കാവില്ല. അവരെ നാം അന്നാട്ടില്‍നിന്ന് അപമാനിതരും നിന്ദ്യരുമാക്കി പുറന്തള്ളും.”
\end{malayalam}}
\flushright{\begin{Arabic}
\quranayah[27][38]
\end{Arabic}}
\flushleft{\begin{malayalam}
സുലൈമാന്‍ പറഞ്ഞു: "അല്ലയോ പ്രധാനികളേ; നിങ്ങളിലാര് അവരുടെ സിംഹാസനം എനിക്കു കൊണ്ടുവന്നുതരും? അവര്‍ വിധേയത്വത്തോടെ എന്റെ അടുത്തുവരുംമുമ്പെ.”
\end{malayalam}}
\flushright{\begin{Arabic}
\quranayah[27][39]
\end{Arabic}}
\flushleft{\begin{malayalam}
ജിന്നുകളിലെ ഒരു മഹാമല്ലന്‍ പറഞ്ഞു: "ഞാനത് അങ്ങയ്ക്ക് കൊണ്ടുവന്നുതരാം. അങ്ങ് ഇരുന്ന ഇരിപ്പില്‍നിന്ന് എഴുന്നേല്‍ക്കും മുമ്പെ. സംശയം വേണ്ട; ഞാനതിനു കഴിവുറ്റവനാണ്. വിശ്വസ്തനും.
\end{malayalam}}
\flushright{\begin{Arabic}
\quranayah[27][40]
\end{Arabic}}
\flushleft{\begin{malayalam}
അപ്പോള്‍ വേദവിജ്ഞാനം കൈമുതലായുണ്ടായിരുന്ന ഒരാള്‍ പറഞ്ഞു: "അങ്ങ് കണ്ണുചിമ്മി തുറക്കും മുമ്പായി ഞാനത് ഇവിടെ എത്തിക്കാം.” അങ്ങനെ അത് തന്റെ അടുത്ത് കൊണ്ടുവന്ന് സ്ഥാപിച്ചതായി കണ്ടപ്പോള്‍ അദ്ദേഹം പറഞ്ഞു: "ഇത് എന്റെ നാഥന്റെ അനുഗ്രഹം കൊണ്ടാണ്. എന്നെ പരീക്ഷിക്കാനാണിത്. ഞാന്‍ നന്ദി കാണിക്കുമോ അതല്ല നന്ദികേട് കാണിക്കുമോയെന്ന് അറിയാന്‍. നന്ദി കാണിക്കുന്നവര്‍ സ്വന്തം നന്മക്കുവേണ്ടിത്തന്നെയാണ് നന്ദി കാണിക്കുന്നത്. എന്നാല്‍ ആരെങ്കിലും നന്ദികേടു കാണിക്കുന്നുവെങ്കില്‍ സംശയംവേണ്ട; എന്റെ നാഥന്‍ അന്യാശ്രയമില്ലാത്തവനാണ്. അത്യുല്‍കൃഷ്ടനും.”
\end{malayalam}}
\flushright{\begin{Arabic}
\quranayah[27][41]
\end{Arabic}}
\flushleft{\begin{malayalam}
സുലൈമാന്‍ പറഞ്ഞു: "നിങ്ങള്‍ അവളുടെ സിംഹാസനം അവള്‍ക്കു തിരിച്ചറിയാനാവാത്തവിധം രൂപമാറ്റം വരുത്തുക. നമുക്കു നോക്കാമല്ലോ, അവള്‍ വസ്തുത മനസ്സിലാക്കുമോ; അതല്ല നേര്‍വഴി കണ്ടെത്താത്തവരില്‍ പെട്ടവളാകുമോയെന്ന്.”
\end{malayalam}}
\flushright{\begin{Arabic}
\quranayah[27][42]
\end{Arabic}}
\flushleft{\begin{malayalam}
അങ്ങനെ രാജ്ഞി വന്നപ്പോള്‍ അവരോട് ചോദിച്ചു: "നിങ്ങളുടെ സിംഹാസനം ഇതുപോലെത്തന്നെയാണോ?” അവര്‍ പറഞ്ഞു: "ഇത് അതുപോലെത്തന്നെയാണല്ലോ. ഇതിനുമുമ്പുതന്നെ ഞങ്ങള്‍ക്കു വിവരം കിട്ടിയിരുന്നു. ഞങ്ങള്‍ മുസ്ലിംകളാവുകയും ചെയ്തിരുന്നു.”
\end{malayalam}}
\flushright{\begin{Arabic}
\quranayah[27][43]
\end{Arabic}}
\flushleft{\begin{malayalam}
അല്ലാഹുവെക്കൂടാതെ അവര്‍ പൂജിച്ചിരുന്ന വസ്തുക്കളാണ് മുസ്ലിമാകുന്നതില്‍നിന്ന് അവരെ തടഞ്ഞിരുന്നത്. തീര്‍ച്ചയായും അവര്‍ സത്യനിഷേധികളായ ജനമായിരുന്നു.
\end{malayalam}}
\flushright{\begin{Arabic}
\quranayah[27][44]
\end{Arabic}}
\flushleft{\begin{malayalam}
അവളോടു പറഞ്ഞു: "കൊട്ടാരത്തില്‍ പ്രവേശിക്കുക.” എന്നാല്‍ അവളതു കണ്ടപ്പോള്‍ തെളിനീര്‍ തടാകമാണെന്നു തോന്നി. തന്റെ കണങ്കാലില്‍നിന്ന് പുടവ പൊക്കുകയും ചെയ്തു. സുലൈമാന്‍ പറഞ്ഞു: "ഇത് സ്ഫടികക്കഷ്ണങ്ങള്‍ പതിച്ചുണ്ടാക്കിയ കൊട്ടാരമാണ്.” അവള്‍ പറഞ്ഞു: "എന്റെ നാഥാ, ഞാന്‍ എന്നോടുതന്നെ അന്യായം ചെയ്തിരിക്കുന്നു. ഞാനിതാ സുലൈമാനോടൊപ്പം പ്രപഞ്ചനാഥനായ അല്ലാഹുവിന് പൂര്‍ണമായും വിധേയയായിരിക്കുന്നു.”
\end{malayalam}}
\flushright{\begin{Arabic}
\quranayah[27][45]
\end{Arabic}}
\flushleft{\begin{malayalam}
സമൂദ് സമുദായത്തിലേക്ക് നാം അവരുടെ സഹോദരന്‍ സ്വാലിഹിനെ അയച്ചു. “നിങ്ങള്‍ അല്ലാഹുവിനുമാത്രം വഴിപ്പെടുക” എന്നതായിരുന്നു അദ്ദേഹത്തിലൂടെ നല്‍കിയ സന്ദേശം. അതോടെ അവര്‍ പരസ്പരം കയര്‍ക്കുന്ന രണ്ട് കക്ഷികളായിപിരിഞ്ഞു.
\end{malayalam}}
\flushright{\begin{Arabic}
\quranayah[27][46]
\end{Arabic}}
\flushleft{\begin{malayalam}
സ്വാലിഹ് പറഞ്ഞു: "എന്റെ ജനമേ; നിങ്ങളെന്തിനു നന്മക്ക് മുമ്പേ തിന്മക്കുവേണ്ടി തിടുക്കം കൂട്ടുന്നു? നിങ്ങള്‍ക്ക് അല്ലാഹുവോട് മാപ്പിരന്നുകൂടേ? അങ്ങനെ ചെയ്താല്‍ നിങ്ങള്‍ക്ക് കാരുണ്യം കിട്ടിയേക്കാം.”
\end{malayalam}}
\flushright{\begin{Arabic}
\quranayah[27][47]
\end{Arabic}}
\flushleft{\begin{malayalam}
അവര്‍ പറഞ്ഞു: "ഞങ്ങള്‍ നിന്നെയും നിന്നോടൊപ്പമുള്ളവരെയും ദുശ്ശകുനമായാണ് കാണുന്നത്.” സ്വാലിഹ് പറഞ്ഞു: "നിങ്ങളുടെ ശകുനം അല്ലാഹുവിന്റെ അടുത്താണ്. പക്ഷേ, നിങ്ങള്‍ പരീക്ഷിക്കപ്പെട്ടുകൊണ്ടിരിക്കുന്ന ജനതയാണ്.”
\end{malayalam}}
\flushright{\begin{Arabic}
\quranayah[27][48]
\end{Arabic}}
\flushleft{\begin{malayalam}
ആ പട്ടണത്തില്‍ ഒമ്പതു പേരുണ്ടായിരുന്നു. അവര്‍ നാട്ടില്‍ കുഴപ്പമുണ്ടാക്കുന്നവരായിരുന്നു. സംസ്കരണം നിര്‍വഹിക്കാത്തവരും.
\end{malayalam}}
\flushright{\begin{Arabic}
\quranayah[27][49]
\end{Arabic}}
\flushleft{\begin{malayalam}
അവരന്യോന്യം പറഞ്ഞു: "നിങ്ങള്‍ ദൈവത്തിന്റെ പേരില്‍ സത്യം ചെയ്യുക, “സ്വാലിഹിനെയും കുടുംബത്തെയും നാം രാത്രി കൊന്നുകളയു”മെന്ന്. എന്നിട്ട് അവന്റെ അവകാശിയോട് തന്റെ ആള്‍ക്കാരുടെ നാശത്തിന് ഞങ്ങള്‍ സാക്ഷികളായിട്ടില്ലെന്നു ബോധിപ്പിക്കണം. തീര്‍ച്ചയായും ഞങ്ങള്‍ സത്യം പറയുന്നവരാണെന്നും.”
\end{malayalam}}
\flushright{\begin{Arabic}
\quranayah[27][50]
\end{Arabic}}
\flushleft{\begin{malayalam}
അവര്‍ ഒരു തന്ത്രം പ്രയോഗിച്ചു. നാമും ഒരു തന്ത്രം പ്രയോഗിച്ചു. അവരത് അറിയുന്നുണ്ടായിരുന്നില്ല.
\end{malayalam}}
\flushright{\begin{Arabic}
\quranayah[27][51]
\end{Arabic}}
\flushleft{\begin{malayalam}
നോക്കൂ; അവരുടെ തന്ത്രത്തിന്റെ ഒടുക്കം എവ്വിധമായിരുന്നുവെന്ന്. സംശയമില്ല; അവരെയും അവരുടെ ജനതയെയും ഒന്നാകെ നാം നശിപ്പിച്ച് നാമാവശേഷമാക്കി.
\end{malayalam}}
\flushright{\begin{Arabic}
\quranayah[27][52]
\end{Arabic}}
\flushleft{\begin{malayalam}
അവരുടെ വീടുകളതാ തകര്‍ന്നു വിജനമായി കിടക്കുന്നു. അവര്‍ അതിക്രമം പ്രവര്‍ത്തിച്ചതിനാലാണത്. ഉറപ്പായും അതില്‍ കാര്യം മനസ്സിലാക്കുന്ന ജനത്തിന് ദൃഷ്ടാന്തമുണ്ട്.
\end{malayalam}}
\flushright{\begin{Arabic}
\quranayah[27][53]
\end{Arabic}}
\flushleft{\begin{malayalam}
സത്യവിശ്വാസം സ്വീകരിക്കുകയും ഭക്തിപുലര്‍ത്തുകയും ചെയ്തവരെ നാം രക്ഷപ്പെടുത്തി.
\end{malayalam}}
\flushright{\begin{Arabic}
\quranayah[27][54]
\end{Arabic}}
\flushleft{\begin{malayalam}
ലൂത്വിനെയും നാം നിയോഗിച്ചു. അദ്ദേഹം തന്റെ ജനതയോടു ചോദിച്ച സന്ദര്‍ഭം: "നിങ്ങള്‍ കണ്ടറിഞ്ഞുകൊണ്ടുതന്നെ വഷളത്തം പ്രവര്‍ത്തിക്കുകയാണോ?
\end{malayalam}}
\flushright{\begin{Arabic}
\quranayah[27][55]
\end{Arabic}}
\flushleft{\begin{malayalam}
"നിങ്ങള്‍ സ്ത്രീകളെ വെടിഞ്ഞ് വികാരശമനത്തിന് പുരുഷന്മാരെ സമീപിക്കുകയാണോ? അല്ല; നിങ്ങള്‍ തീര്‍ത്തും അവിവേകികളായ ജനത തന്നെ.”
\end{malayalam}}
\flushright{\begin{Arabic}
\quranayah[27][56]
\end{Arabic}}
\flushleft{\begin{malayalam}
അദ്ദേഹത്തിന്റെ ജനതയുടെ മറുപടി അവരുടെ ഈ പറച്ചില്‍ മാത്രമായിരുന്നു: "ലൂത്വിന്റെ ആള്‍ക്കാരെ നിങ്ങളുടെ നാട്ടില്‍ നിന്നും പുറത്താക്കുക. അവര്‍ വലിയ വിശുദ്ധി ചമയുകയാണ്.”
\end{malayalam}}
\flushright{\begin{Arabic}
\quranayah[27][57]
\end{Arabic}}
\flushleft{\begin{malayalam}
അവസാനം അദ്ദേഹത്തെയും അദ്ദേഹത്തിന്റെ ആളുകളെയും നാം രക്ഷപ്പെടുത്തി. അദ്ദേഹത്തിന്റെ ഭാര്യയെ ഒഴികെ. പിന്മാറി നിന്നവരിലായിരിക്കും അവളെന്ന് നേരത്തെതന്നെ നാം കണക്കാക്കിയിരുന്നു.
\end{malayalam}}
\flushright{\begin{Arabic}
\quranayah[27][58]
\end{Arabic}}
\flushleft{\begin{malayalam}
അവരുടെ മേല്‍ നാമൊരു മഴ വീഴ്ത്തി. മുന്നറിയിപ്പു നല്‍കപ്പെട്ടജനത്തിനു കിട്ടിയ ആ മഴ എത്ര ചീത്ത!
\end{malayalam}}
\flushright{\begin{Arabic}
\quranayah[27][59]
\end{Arabic}}
\flushleft{\begin{malayalam}
പറയുക: അല്ലാഹുവിന് സ്തുതി. അവന്‍ പ്രത്യേകം തെരഞ്ഞെടുത്ത തന്റെ ദാസന്മാര്‍ക്കു സമാധാനം. അല്ലാഹുവാണോ ഉത്തമന്‍; അതോ ഇവര്‍ സങ്കല്‍പിച്ചുണ്ടാക്കുന്ന പങ്കാളികളോ?
\end{malayalam}}
\flushright{\begin{Arabic}
\quranayah[27][60]
\end{Arabic}}
\flushleft{\begin{malayalam}
ആകാശഭൂമികളെ സൃഷ്ടിക്കുകയും നിങ്ങള്‍ക്കു മാനത്തുനിന്ന് മഴവെള്ളം വീഴ്ത്തിത്തരികയും ചെയ്തവനാരാണ്? അതുവഴി നാം ചേതോഹരമായ തോട്ടങ്ങള്‍ വളര്‍ത്തിയെടുത്തു. അതിലെ മരങ്ങള്‍ മുളപ്പിക്കാന്‍ നിങ്ങള്‍ക്കാവില്ലല്ലോ. ഇതിലൊക്കെ അല്ലാഹുവോടൊപ്പം വേറെ വല്ല ദൈവവുമുണ്ടോ? ഇല്ല; എന്നാല്‍ അവര്‍ വഴിതെറ്റിപ്പോയ ജനത തന്നെ.
\end{malayalam}}
\flushright{\begin{Arabic}
\quranayah[27][61]
\end{Arabic}}
\flushleft{\begin{malayalam}
ഭൂമിയെ പാര്‍ക്കാന്‍ പറ്റിയതാക്കുകയും അതില്‍ അങ്ങിങ്ങ് നദികളുണ്ടാക്കുകയും നങ്കൂരമിട്ടുറപ്പിച്ചപോലുള്ള പര്‍വതങ്ങളുണ്ടാക്കുകയും രണ്ടിനം ജലാശയങ്ങള്‍ക്കിടയില്‍ മറയുണ്ടാക്കുകയും ചെയ്തവന്‍ ആരാണ്? ഇതിലെല്ലാം അല്ലാഹുവോടൊപ്പം വേറെ വല്ല ദൈവവുമുണ്ടോ? ഇല്ല; എന്നാല്‍ അവരിലേറെ പേരും അറിവില്ലാത്തവരാണ്.
\end{malayalam}}
\flushright{\begin{Arabic}
\quranayah[27][62]
\end{Arabic}}
\flushleft{\begin{malayalam}
പ്രയാസമനുഭവിക്കുന്നവന്‍ പ്രാര്‍ഥിക്കുമ്പോള്‍ അതിനുത്തരം നല്‍കുകയും ദുരിതങ്ങളകറ്റുകയും നിങ്ങളെ ഭൂമിയിലെ പ്രതിനിധികളാക്കുകയും ചെയ്തവന്‍ ആരാണ്? ഇതിലൊക്കെ അല്ലാഹുവോടൊപ്പം വേറെ വല്ല ദൈവവുമുണ്ടോ? അല്‍പം മാത്രമേ നിങ്ങള്‍ ചിന്തിച്ചറിയുന്നുള്ളൂ.
\end{malayalam}}
\flushright{\begin{Arabic}
\quranayah[27][63]
\end{Arabic}}
\flushleft{\begin{malayalam}
കരയിലെയും കടലിലെയും കൂരിരുളില്‍ നിങ്ങള്‍ക്കു വഴികാണിക്കുന്നത് ആരാണ്? തന്റെ അനുഗ്രഹത്തിനു മുന്നോടിയായി ശുഭവാര്‍ത്തയുമായി കാറ്റിനെ അയക്കുന്നത് ആരാണ്? അല്ലാഹുവോടൊപ്പം വേറെ വല്ല ദൈവവുമുണ്ടോ? അവര്‍ പങ്കുചേര്‍ക്കുന്നതില്‍ നിന്നെല്ലാം അതീതനാണ് അല്ലാഹു.
\end{malayalam}}
\flushright{\begin{Arabic}
\quranayah[27][64]
\end{Arabic}}
\flushleft{\begin{malayalam}
സൃഷ്ടി ആരംഭിക്കുകയും പിന്നീട് അതാവര്‍ത്തിക്കുകയും ചെയ്യുന്നതാരാണ്? മാനത്തു നിന്നും മണ്ണില്‍ നിന്നും നിങ്ങള്‍ക്ക് അന്നം തരുന്നതാരാണ്? അല്ലാഹുവോടൊപ്പം വേറെ വല്ല ദൈവവുമുണ്ടോ? പറയുക: "നിങ്ങള്‍ നിങ്ങളുടെ തെളിവ് കൊണ്ടുവരിക. നിങ്ങള്‍ സത്യവാന്മാരെങ്കില്‍!”
\end{malayalam}}
\flushright{\begin{Arabic}
\quranayah[27][65]
\end{Arabic}}
\flushleft{\begin{malayalam}
പറയുക: അല്ലാഹുവിനല്ലാതെ ആകാശഭൂമികളിലാര്‍ക്കുംതന്നെ അഭൌതിക കാര്യങ്ങളറിയുകയില്ല. തങ്ങള്‍ എന്നാണ് ഉയിര്‍ത്തെഴുന്നേല്‍പിക്കപ്പെടുകയെന്നും അവര്‍ക്കറിയില്ല.
\end{malayalam}}
\flushright{\begin{Arabic}
\quranayah[27][66]
\end{Arabic}}
\flushleft{\begin{malayalam}
എന്നല്ല; പരലോകത്തെപ്പറ്റിയുള്ള അറിവേ അവര്‍ക്കില്ല. അവരിപ്പോഴും അതേക്കുറിച്ച് സംശയത്തിലാണ്. അല്ല, അവര്‍ അതേപ്പറ്റി തികഞ്ഞ അന്ധതയിലാണ്.
\end{malayalam}}
\flushright{\begin{Arabic}
\quranayah[27][67]
\end{Arabic}}
\flushleft{\begin{malayalam}
സത്യനിഷേധികള്‍ ചോദിക്കുന്നു: "നാമും നമ്മുടെ പിതാക്കന്മാരും മണ്ണായി മാറിയശേഷം ശവകുടീരങ്ങളില്‍നിന്ന് വീണ്ടും നമ്മെ പുറത്തുകൊണ്ടുവരുമെന്നോ?
\end{malayalam}}
\flushright{\begin{Arabic}
\quranayah[27][68]
\end{Arabic}}
\flushleft{\begin{malayalam}
"ഞങ്ങളോടും ഞങ്ങളുടെ പിതാക്കളോടും മുമ്പുതന്നെ ഇവ്വിധം വാഗ്ദാനം ചെയ്തിരുന്നു. പൂര്‍വികരുടെ പഴമ്പുരാണങ്ങള്‍ മാത്രമാണിത്.”
\end{malayalam}}
\flushright{\begin{Arabic}
\quranayah[27][69]
\end{Arabic}}
\flushleft{\begin{malayalam}
പറയുക: "നിങ്ങള്‍ ഭൂമിയിലൊന്നു സഞ്ചരിച്ചുനോക്കൂ. കുറ്റവാളികളുടെ അന്ത്യം എവ്വിധമായിരുന്നുവെന്ന്.”
\end{malayalam}}
\flushright{\begin{Arabic}
\quranayah[27][70]
\end{Arabic}}
\flushleft{\begin{malayalam}
നീ അവരെയോര്‍ത്ത് ദുഃഖിക്കേണ്ട. അവരുടെ കുതന്ത്രങ്ങളോര്‍ത്ത് മനസ്സു തിടുങ്ങേണ്ട.
\end{malayalam}}
\flushright{\begin{Arabic}
\quranayah[27][71]
\end{Arabic}}
\flushleft{\begin{malayalam}
അവര്‍ ചോദിക്കുന്നു: "ഈ വാഗ്ദാനം എപ്പോഴാണ് പുലരുക? നിങ്ങള്‍ സത്യവാദികളെങ്കില്‍!”
\end{malayalam}}
\flushright{\begin{Arabic}
\quranayah[27][72]
\end{Arabic}}
\flushleft{\begin{malayalam}
പറയുക: "ഏതൊരു ശിക്ഷക്കുവേണ്ടിയാണോ നിങ്ങള്‍ തിടുക്കം കൂട്ടുന്നത് അതിന്റെ ഒരു ഭാഗം ഒരുവേള നിങ്ങളുടെ തൊട്ടുപിന്നില്‍ എത്തിക്കഴിഞ്ഞിട്ടുണ്ടാവാം.”
\end{malayalam}}
\flushright{\begin{Arabic}
\quranayah[27][73]
\end{Arabic}}
\flushleft{\begin{malayalam}
സംശയമില്ല; നിന്റെ നാഥന്‍ ജനങ്ങളോട് അത്യുദാരനാണ്. എന്നാല്‍ അവരിലേറെ പേരും നന്ദി കാണിക്കുന്നവരല്ല.
\end{malayalam}}
\flushright{\begin{Arabic}
\quranayah[27][74]
\end{Arabic}}
\flushleft{\begin{malayalam}
അവരുടെ മനസ്സുകള്‍ മറച്ചുവെക്കുന്നതും അവര്‍ പരസ്യമാക്കുന്നതുമെല്ലാം നിന്റെ നാഥന്‍ നന്നായറിയുന്നുണ്ട്.
\end{malayalam}}
\flushright{\begin{Arabic}
\quranayah[27][75]
\end{Arabic}}
\flushleft{\begin{malayalam}
സ്പഷ്ടമായ മൂലപ്രമാണത്തില്‍ രേഖപ്പെടുത്താത്ത ഒന്നും ആകാശഭൂമികളില്‍ ഒളിഞ്ഞുകിടക്കുന്നില്ല.
\end{malayalam}}
\flushright{\begin{Arabic}
\quranayah[27][76]
\end{Arabic}}
\flushleft{\begin{malayalam}
ഇസ്രയേല്‍ മക്കള്‍ ഭിന്നത പുലര്‍ത്തുന്ന മിക്ക കാര്യങ്ങളുടെയും നിജസ്ഥിതി ഈ ഖുര്‍ആന്‍ അവര്‍ക്ക് വിശദീകരിച്ചുകൊടുക്കുന്നു.
\end{malayalam}}
\flushright{\begin{Arabic}
\quranayah[27][77]
\end{Arabic}}
\flushleft{\begin{malayalam}
തീര്‍ച്ചയായും സത്യവിശ്വാസികള്‍ക്കിത് നല്ലൊരു വഴികാട്ടിയാണ്. മഹത്തായ അനുഗ്രഹവും.
\end{malayalam}}
\flushright{\begin{Arabic}
\quranayah[27][78]
\end{Arabic}}
\flushleft{\begin{malayalam}
സംശയമില്ല; നിന്റെ നാഥന്‍ തന്റെ വിധിയിലൂടെ അവര്‍ക്കിടയില്‍ തീര്‍പ്പുകല്‍പിക്കും. അവന്‍ പ്രതാപിയാണ്. എല്ലാം അറിയുന്നവനും.
\end{malayalam}}
\flushright{\begin{Arabic}
\quranayah[27][79]
\end{Arabic}}
\flushleft{\begin{malayalam}
അതിനാല്‍ നീ അല്ലാഹുവില്‍ ഭരമേല്‍പിക്കുക. ഉറപ്പായും നീ സുവ്യക്തമായ സത്യത്തില്‍ തന്നെയാണ്.
\end{malayalam}}
\flushright{\begin{Arabic}
\quranayah[27][80]
\end{Arabic}}
\flushleft{\begin{malayalam}
മരിച്ചവരെയും കാതുപൊട്ടന്മാരെയും കേള്‍പ്പിക്കാന്‍ നിനക്കാവില്ല. അവര്‍ പിന്തിരിഞ്ഞു പോയാല്‍.
\end{malayalam}}
\flushright{\begin{Arabic}
\quranayah[27][81]
\end{Arabic}}
\flushleft{\begin{malayalam}
കണ്ണുപൊട്ടന്മാരെ അവരകപ്പെട്ട ദുര്‍മാര്‍ഗത്തില്‍നിന്ന് നേര്‍വഴിയിലേക്കു നയിക്കാനും നിനക്കാവില്ല. നമ്മുടെ വചനങ്ങളില്‍ വിശ്വസിക്കുകയും അങ്ങനെ അനുസരണമുള്ളവരാവുകയും ചെയ്യുന്നവരെ മാത്രമേ നിനക്കു കേള്‍പ്പിക്കാന്‍ കഴിയുകയുള്ളൂ.
\end{malayalam}}
\flushright{\begin{Arabic}
\quranayah[27][82]
\end{Arabic}}
\flushleft{\begin{malayalam}
നമ്മുടെ വചനം അവരില്‍ പുലര്‍ന്നാല്‍, നാം അവര്‍ക്കായി ഭൂമിയില്‍നിന്ന് ഒരു ജന്തുവെ പുറപ്പെടുവിക്കും. ജനം നമ്മുടെ വചനങ്ങളില്‍ ദൃഢവിശ്വാസം ഉള്ളവരായില്ല എന്ന കാര്യം അതവരോടു പറയും.
\end{malayalam}}
\flushright{\begin{Arabic}
\quranayah[27][83]
\end{Arabic}}
\flushleft{\begin{malayalam}
നാം എല്ലാ സമുദായങ്ങളിലെയും നമ്മുടെ വചനങ്ങളെ തള്ളിപ്പറഞ്ഞ ഓരോ സംഘത്തെ ഒരുമിച്ചുകൂട്ടുന്ന നാളിനെ ഒന്ന് സങ്കല്‍പിച്ചു നോക്കുക. അപ്പോള്‍, അവരെ ക്രമത്തില്‍ നിര്‍ത്തും.
\end{malayalam}}
\flushright{\begin{Arabic}
\quranayah[27][84]
\end{Arabic}}
\flushleft{\begin{malayalam}
അങ്ങനെ അവരെല്ലാം വന്നെത്തിയാല്‍ അല്ലാഹു ചോദിക്കും: "എന്റെ വചനങ്ങള്‍ ശരിക്കും മനസ്സിലാക്കാതെ നിങ്ങളവയെ തള്ളിപ്പറഞ്ഞുവോ? അല്ലെങ്കില്‍ പിന്നെ നിങ്ങളെന്താണ് ചെയ്തുകൊണ്ടിരുന്നത്?”
\end{malayalam}}
\flushright{\begin{Arabic}
\quranayah[27][85]
\end{Arabic}}
\flushleft{\begin{malayalam}
അവര്‍ അതിക്രമം കാണിച്ചതിനാല്‍ നിശ്ചയമായും ശിക്ഷാവിധി അവരില്‍ വന്നെത്തും. അപ്പോഴവര്‍ക്കൊന്നും പറയാനാവില്ല.
\end{malayalam}}
\flushright{\begin{Arabic}
\quranayah[27][86]
\end{Arabic}}
\flushleft{\begin{malayalam}
അവര്‍ കാണുന്നില്ലേ? അവര്‍ക്ക് ശാന്തി കൈവരിക്കാന്‍ നാം രാവിനെ ഉണ്ടാക്കിയത്. പകലിനെ പ്രഭാപൂരിതമാക്കിയതും. സംശയമില്ല; വിശ്വസിക്കുന്ന ജനത്തിന് അതില്‍ ധാരാളം തെളിവുകളുണ്ട്.
\end{malayalam}}
\flushright{\begin{Arabic}
\quranayah[27][87]
\end{Arabic}}
\flushleft{\begin{malayalam}
കാഹളം ഊതപ്പെടുന്ന ദിനം. അന്ന് ആകാശഭൂമികളിലുള്ളവരെല്ലാം പേടിച്ചരണ്ടുപോകും. അല്ലാഹു ഉദ്ദേശിക്കുന്നവരൊഴികെ. എല്ലാവരും ഏറെ എളിമയോടെ അവന്റെ അടുത്ത് വന്നെത്തും.
\end{malayalam}}
\flushright{\begin{Arabic}
\quranayah[27][88]
\end{Arabic}}
\flushleft{\begin{malayalam}
നീയിപ്പോള്‍ മലകളെ കാണുന്നു. അവ ഊന്നിയുറച്ചവയാണെന്ന് നിനക്ക് തോന്നും എന്നാല്‍ അവ മേഘങ്ങള്‍ പോലെ ഇളകി നീങ്ങിക്കൊണ്ടിരിക്കും. അല്ലാഹുവിന്റെ പ്രവൃത്തിയാണത്. എല്ലാ കാര്യങ്ങളും കുറ്റമറ്റതാക്കിയവനാണല്ലോ അവന്‍. നിശ്ചയമായും നിങ്ങള്‍ പ്രവര്‍ത്തിക്കുന്നതിനെപ്പറ്റിയെല്ലാം സൂക്ഷ്മമായി അറിയുന്നവനാണവന്‍.
\end{malayalam}}
\flushright{\begin{Arabic}
\quranayah[27][89]
\end{Arabic}}
\flushleft{\begin{malayalam}
അന്ന് നന്മയുമായി വന്നെത്തുന്നവന് അതിലും മെച്ചപ്പെട്ട പ്രതിഫലം കിട്ടും. അന്നത്തെ കൊടുംപേടിയില്‍ നിന്ന് അവര്‍ തീര്‍ത്തും മുക്തരായിരിക്കും.
\end{malayalam}}
\flushright{\begin{Arabic}
\quranayah[27][90]
\end{Arabic}}
\flushleft{\begin{malayalam}
തിന്മയുമായി വന്നെത്തുന്നവരെ നരകത്തീയില്‍ മുഖം കുത്തിവീഴ്ത്തും. നിങ്ങള്‍ പ്രവര്‍ത്തിച്ചുകൊണ്ടിരുന്നതിനല്ലാതെ നിങ്ങള്‍ക്ക് പ്രതിഫലം കിട്ടുമോ?
\end{malayalam}}
\flushright{\begin{Arabic}
\quranayah[27][91]
\end{Arabic}}
\flushleft{\begin{malayalam}
പറയുക: എന്നോടു കല്‍പിച്ചത് ഈ നാടിന്റെ നാഥന്ന് വഴിപ്പെടാന്‍ മാത്രമാണ്. അതിനെ ആദരണീയമാക്കിയത് അവനാണ്. എല്ലാ വസ്തുക്കളുടെയും ഉടമയും അവന്‍തന്നെ. ഞാന്‍ മുസ്ലിംകളിലുള്‍പ്പെടണമെന്നും അവനെന്നോട് ആജ്ഞാപിച്ചിരിക്കുന്നു.
\end{malayalam}}
\flushright{\begin{Arabic}
\quranayah[27][92]
\end{Arabic}}
\flushleft{\begin{malayalam}
ഈ ഖുര്‍ആന്‍ ഓതിക്കേള്‍പിക്കണമെന്നും എന്നോടു കല്‍പിച്ചിരിക്കുന്നു. അതിനാല്‍ ആരെങ്കിലും നേര്‍വഴി സ്വീകരിക്കുന്നുവെങ്കില്‍ അത് അവന്റെ തന്നെ ഗുണത്തിനുവേണ്ടിയാണ്. ആരെങ്കിലും വഴികേടിലാവുന്നുവെങ്കില്‍ നീ പറയുക: "ഞാനൊരു മുന്നറിയിപ്പുകാരന്‍
\end{malayalam}}
\flushright{\begin{Arabic}
\quranayah[27][93]
\end{Arabic}}
\flushleft{\begin{malayalam}
പറയുക: സര്‍വസ്തുതിയും അല്ലാഹുവിനാണ്. വൈകാതെ തന്നെ അവന്‍ തന്റെ ദൃഷ്ടാന്തങ്ങള്‍ നിങ്ങള്‍ക്കു കാണിച്ചുതരും. അപ്പോള്‍ നിങ്ങള്‍ക്കത് ബോധ്യമാവും. നിങ്ങള്‍ പ്രവര്‍ത്തിച്ചുകൊണ്ടിരിക്കുന്നതിനെപ്പറ്റി നിന്റെ നാഥന്‍ ഒട്ടും അശ്രദ്ധനല്ല.
\end{malayalam}}
\chapter{\textmalayalam{ഖസസ് ( കഥാകഥനം‍ )}}
\begin{Arabic}
\Huge{\centerline{\basmalah}}\end{Arabic}
\flushright{\begin{Arabic}
\quranayah[28][1]
\end{Arabic}}
\flushleft{\begin{malayalam}
ത്വാ-സീന്‍-മീം.
\end{malayalam}}
\flushright{\begin{Arabic}
\quranayah[28][2]
\end{Arabic}}
\flushleft{\begin{malayalam}
സുവ്യക്തമായ വേദപുസ്തകത്തിലെ വചനങ്ങളാണിത്.
\end{malayalam}}
\flushright{\begin{Arabic}
\quranayah[28][3]
\end{Arabic}}
\flushleft{\begin{malayalam}
മൂസായുടെയും ഫറവോന്റെയും ചില വൃത്താന്തങ്ങള്‍ നാം നിന്നെ വസ്തുനിഷ്ഠമായി ഓതിക്കേള്‍പ്പിക്കാം. വിശ്വസിക്കുന്ന ജനത്തിനുവേണ്ടിയാണിത്.
\end{malayalam}}
\flushright{\begin{Arabic}
\quranayah[28][4]
\end{Arabic}}
\flushleft{\begin{malayalam}
ഫറവോന്‍ നാട്ടില്‍ അഹങ്കരിച്ചുനടന്നു. അന്നാട്ടുകാരെ വിവിധ വിഭാഗങ്ങളാക്കി. അവരിലൊരു വിഭാഗത്തെ പറ്റെ ദുര്‍ബലമാക്കി. അവരിലെ ആണ്‍കുട്ടികളെ അറുകൊല ചെയ്തു. പെണ്‍മക്കളെ ജീവിക്കാന്‍ വിട്ടു. അവന്‍ നാശകാരികളില്‍ പെട്ടവനായിരുന്നു; തീര്‍ച്ച.
\end{malayalam}}
\flushright{\begin{Arabic}
\quranayah[28][5]
\end{Arabic}}
\flushleft{\begin{malayalam}
എന്നാല്‍ ഭൂമിയില്‍ മര്‍ദിച്ചൊതുക്കപ്പെട്ടവരോട് ഔദാര്യം കാണിക്കണമെന്ന് നാം ആഗ്രഹിച്ചു. അവരെ നേതാക്കളും ഭൂമിയുടെ അവകാശികളുമാക്കണമെന്നും.
\end{malayalam}}
\flushright{\begin{Arabic}
\quranayah[28][6]
\end{Arabic}}
\flushleft{\begin{malayalam}
അവര്‍ക്ക് ഭൂമിയില്‍ അധികാരം നല്‍കണമെന്നും അങ്ങനെ ഫറവോന്നും ഹാമാന്നും അവരുടെ സൈന്യത്തിനും അവര്‍ ആശങ്കിച്ചുകൊണ്ടിരുന്നതെന്തോ അതു കാണിച്ചുകൊടുക്കണമെന്നും.
\end{malayalam}}
\flushright{\begin{Arabic}
\quranayah[28][7]
\end{Arabic}}
\flushleft{\begin{malayalam}
മൂസായുടെ മാതാവിനു നാം സന്ദേശം നല്‍കി: "അവനെ മുലയൂട്ടുക. അഥവാ, അവന്റെ കാര്യത്തില്‍ നിനക്ക് ആശങ്ക തോന്നുന്നുവെങ്കില്‍ അവനെ നീ പുഴയിലെറിയുക. പേടിക്കേണ്ട. ദുഃഖിക്കുകയും വേണ്ട. തീര്‍ച്ചയായും നാമവനെ നിന്റെയടുത്ത് തിരിച്ചെത്തിക്കും. അവനെ ദൈവദൂതന്മാരിലൊരുവനാക്കുകയും ചെയ്യും."
\end{malayalam}}
\flushright{\begin{Arabic}
\quranayah[28][8]
\end{Arabic}}
\flushleft{\begin{malayalam}
അങ്ങനെ ഫറവോന്റെ ആള്‍ക്കാര്‍ ആ കുട്ടിയെ കണ്ടെടുത്തു. അവസാനം അവന്‍ അവരുടെ ശത്രുവും ദുഃഖകാരണവുമാകാന്‍. സംശയമില്ല; ഫറവോനും ഹാമാനും അവരുടെ പട്ടാളക്കാരും തീര്‍ത്തും വഴികേടിലായിരുന്നു.
\end{malayalam}}
\flushright{\begin{Arabic}
\quranayah[28][9]
\end{Arabic}}
\flushleft{\begin{malayalam}
ഫറവോന്റെ പത്നി പറഞ്ഞു: "എന്റെയും നിങ്ങളുടെയും കണ്ണിനു കുളിര്‍മയാണിവന്‍. അതിനാല്‍ നിങ്ങളിവനെ കൊല്ലരുത്. നമുക്ക് ഇവന്‍ ഉപകരിച്ചേക്കാം. അല്ലെങ്കില്‍ നമുക്കിവനെ നമ്മുടെ മകനാക്കാമല്ലോ." അവര്‍ ആ കുട്ടിയെസംബന്ധിച്ച നിജസ്ഥിതി അറിഞ്ഞിരുന്നില്ല.
\end{malayalam}}
\flushright{\begin{Arabic}
\quranayah[28][10]
\end{Arabic}}
\flushleft{\begin{malayalam}
മൂസായുടെ മാതാവിന്റെ മനസ്സ് അസ്വസ്ഥമായി. അവളുടെ മനസ്സിനെ നാം ഉറപ്പിച്ചുനിര്‍ത്തിയില്ലായിരുന്നുവെങ്കില്‍ അവന്റെ കാര്യം അവള്‍ വെളിപ്പെടുത്തുമായിരുന്നു. അവള്‍ സത്യവിശ്വാസികളില്‍ പെട്ടവളാകാനാണ് നാമങ്ങനെ ചെയ്തത്.
\end{malayalam}}
\flushright{\begin{Arabic}
\quranayah[28][11]
\end{Arabic}}
\flushleft{\begin{malayalam}
അവള്‍ ആ കുട്ടിയുടെ സഹോദരിയോടു പറഞ്ഞു: "നീ അവന്റെ പിറകെ പോയി അന്വേഷിച്ചുനോക്കുക." അങ്ങനെ അവള്‍ അകലെനിന്ന് അവനെ വീക്ഷിച്ചു. ഇതൊന്നും അവരറിയുന്നുണ്ടായിരുന്നില്ല.
\end{malayalam}}
\flushright{\begin{Arabic}
\quranayah[28][12]
\end{Arabic}}
\flushleft{\begin{malayalam}
ആ കുട്ടിക്ക് മുലയൂട്ടുകാരികള്‍ മുലകൊടുക്കുന്നത് നാം മുമ്പേ വിലക്കിയിട്ടുണ്ടായിരുന്നു. അപ്പോള്‍ മൂസായുടെ സഹോദരി പറഞ്ഞു: "നിങ്ങള്‍ക്ക് ഞാനൊരു വീട്ടുകാരെ പരിചയപ്പെടുത്തി തരട്ടെയോ? നിങ്ങള്‍ക്കുവേണ്ടി അവര്‍ ഈ കുട്ടിയെ നന്നായി സംരക്ഷിച്ചുകൊള്ളും. അവര്‍ കുട്ടിയോടു ഗുണകാംക്ഷ പുലര്‍ത്തുകയും ചെയ്യും."
\end{malayalam}}
\flushright{\begin{Arabic}
\quranayah[28][13]
\end{Arabic}}
\flushleft{\begin{malayalam}
ഇങ്ങനെ നാം മൂസായെ അവന്റെ മാതാവിന് തിരിച്ചേല്‍പിച്ചു. അവളുടെ കണ്ണു കുളിര്‍ക്കാന്‍. അവള്‍ ദുഃഖിക്കാതിരിക്കാനും അല്ലാഹുവിന്റെ വാഗ്ദാനം സത്യമാണെന്ന് അവളറിയാനും. എന്നാല്‍ അവരിലേറെ പേരും കാര്യം മനസ്സിലാക്കുന്നവരല്ല.
\end{malayalam}}
\flushright{\begin{Arabic}
\quranayah[28][14]
\end{Arabic}}
\flushleft{\begin{malayalam}
അങ്ങനെ മൂസ കരുത്തു നേടുകയും പക്വത പ്രാപിക്കുകയും ചെയ്തപ്പോള്‍ നാം അവന്ന് തീരുമാനശക്തിയും വിജ്ഞാനവും നല്‍കി. അവ്വിധമാണ് സച്ചരിതര്‍ക്കു നാം പ്രതിഫലം നല്‍കുക.
\end{malayalam}}
\flushright{\begin{Arabic}
\quranayah[28][15]
\end{Arabic}}
\flushleft{\begin{malayalam}
നഗരവാസികള്‍ അശ്രദ്ധരായിരിക്കെ മൂസ അവിടെ കടന്നുചെന്നു. അപ്പോള്‍ രണ്ടുപേര്‍ തമ്മില്‍ തല്ലുകൂടുന്നത് അദ്ദേഹം കണ്ടു. ഒരാള്‍ തന്റെ കക്ഷിയില്‍ പെട്ടവനാണ്. അപരന്‍ ശത്രുവിഭാഗത്തിലുള്ളവനും. തന്റെ കക്ഷിയില്‍ പെട്ടവന്‍ ശത്രുവിഭാഗത്തിലുള്ളവനെതിരെ മൂസായോട് സഹായം തേടി. അപ്പോള്‍ മൂസ അയാളെ ഇടിച്ചു. അതവന്റെ കഥ കഴിച്ചു. മൂസ പറഞ്ഞു: "ഇതു പിശാചിന്റെ ചെയ്തികളില്‍പെട്ടതാണ്. സംശയമില്ല; അവന്‍ പ്രത്യക്ഷ ശത്രുവാണ്. വഴിപിഴപ്പിക്കുന്നവനും."
\end{malayalam}}
\flushright{\begin{Arabic}
\quranayah[28][16]
\end{Arabic}}
\flushleft{\begin{malayalam}
അദ്ദേഹം പറഞ്ഞു: "എന്റെ നാഥാ, തീര്‍ച്ചയായും ഞാനെന്നോടു തന്നെ അതിക്രമം കാണിച്ചിരിക്കുന്നു. അതിനാല്‍ നീയെനിക്കു പൊറുത്തുതരേണമേ." അപ്പോള്‍ അല്ലാഹു അദ്ദേഹത്തിനു പൊറുത്തുകൊടുത്തു. തീര്‍ച്ചയായും അവന്‍ ഏറെ പൊറുക്കുന്നവനാണ്. പരമദയാലുവും.
\end{malayalam}}
\flushright{\begin{Arabic}
\quranayah[28][17]
\end{Arabic}}
\flushleft{\begin{malayalam}
അദ്ദേഹം പറഞ്ഞു: "എന്റെ നാഥാ, നീയെനിക്ക് ധാരാളം അനുഗ്രഹം തന്നല്ലോ. അതിനാല്‍ ഞാനിനിയൊരിക്കലും കുറ്റവാളികള്‍ക്ക് തുണയാവുകയില്ല."
\end{malayalam}}
\flushright{\begin{Arabic}
\quranayah[28][18]
\end{Arabic}}
\flushleft{\begin{malayalam}
അടുത്ത പ്രഭാതത്തില്‍ പേടിയോടെ പാത്തും പതുങ്ങിയും മൂസ പട്ടണത്തില്‍ പ്രവേശിച്ചു. അപ്പോഴതാ തലേന്നാള്‍ തന്നോടു സഹായം തേടിയ അതേയാള്‍ അന്നും സഹായത്തിനായി മുറവിളികൂട്ടുന്നു. മൂസ അയാളോട് പറഞ്ഞു: "നീ വ്യക്തമായും ദുര്‍മാര്‍ഗി തന്നെ."
\end{malayalam}}
\flushright{\begin{Arabic}
\quranayah[28][19]
\end{Arabic}}
\flushleft{\begin{malayalam}
അങ്ങനെ അദ്ദേഹം അവരിരുവരുടെയും ശത്രുവായ ആളെ പിടികൂടാന്‍ തുനിഞ്ഞപ്പോള്‍ അവന്‍ പറഞ്ഞു: "ഇന്നലെ നീയൊരുവനെ കൊന്നപോലെ ഇന്ന് നീയെന്നെയും കൊല്ലാനുദ്ദേശിക്കുകയാണോ? ഇന്നാട്ടിലെ ഒരു മേലാളനാകാന്‍ മാത്രമാണ് നീ ആഗ്രഹിക്കുന്നത്. നന്മ വരുത്തുന്ന നല്ലവനാകാനല്ല."
\end{malayalam}}
\flushright{\begin{Arabic}
\quranayah[28][20]
\end{Arabic}}
\flushleft{\begin{malayalam}
അപ്പോള്‍ പട്ടണത്തിന്റെ മറ്റേ അറ്റത്തുനിന്ന് ഒരാള്‍ ഓടിവന്നു. അയാള്‍ പറഞ്ഞു: "ഓ, മൂസാ, താങ്കളെ കൊല്ലാന്‍ നാട്ടിലെ പ്രധാനികള്‍ ആലോചിക്കുന്നുണ്ട്. അതിനാല്‍ ഒട്ടും വൈകാതെ താങ്കളിവിടെനിന്ന് പുറത്തുപോയി രക്ഷപ്പെട്ടുകൊള്ളുക. തീര്‍ച്ചയായും ഞാന്‍ താങ്കളുടെ ഗുണകാംക്ഷികളിലൊരാളാണ്."
\end{malayalam}}
\flushright{\begin{Arabic}
\quranayah[28][21]
\end{Arabic}}
\flushleft{\begin{malayalam}
അങ്ങനെ മൂസ പേടിയോടും കരുതലോടും കൂടി അവിടെനിന്ന് പുറപ്പെട്ടു. അദ്ദേഹം ഇങ്ങനെ പ്രാര്‍ഥിച്ചു: "എന്റെ നാഥാ, അക്രമികളായ ഈ ജനതയില്‍ നിന്ന് നീയെന്നെ രക്ഷപ്പെടുത്തേണമേ."
\end{malayalam}}
\flushright{\begin{Arabic}
\quranayah[28][22]
\end{Arabic}}
\flushleft{\begin{malayalam}
മദ്യന്റെ നേരെ യാത്ര തിരിച്ചപ്പോള്‍ അദ്ദേഹം പറഞ്ഞു: "എന്റെ നാഥന്‍ എന്നെ ശരിയായ വഴിയിലൂടെ നയിച്ചേക്കാം."
\end{malayalam}}
\flushright{\begin{Arabic}
\quranayah[28][23]
\end{Arabic}}
\flushleft{\begin{malayalam}
മദ്യനിലെ ജലാശയത്തിനടുത്തെത്തിയപ്പോള്‍ അവിടെ ഒരു കൂട്ടം ആളുകള്‍ തങ്ങളുടെ ആടുകളെ വെള്ളം കുടിപ്പിക്കുന്നതുകണ്ടു. അവരില്‍ നിന്ന് വിട്ടുമാറി രണ്ടു സ്ത്രീകള്‍ ആടുകളെ തടഞ്ഞുനിര്‍ത്തുന്നതായും. അതിനാല്‍ അദ്ദേഹം ചോദിച്ചു: "നിങ്ങളുടെ പ്രശ്നമെന്താണ്?" അവരിരുവരും പറഞ്ഞു: "ആ ഇടയന്മാര്‍ അവരുടെ ആടുകളെ തിരിച്ചുകൊണ്ടുപോകുംവരെ ഞങ്ങള്‍ക്ക് വെള്ളം കുടിപ്പിക്കാനാവില്ല. ഞങ്ങളുടെ പിതാവാണെങ്കില്‍ അവശനായ ഒരു വൃദ്ധനാണ്."
\end{malayalam}}
\flushright{\begin{Arabic}
\quranayah[28][24]
\end{Arabic}}
\flushleft{\begin{malayalam}
അപ്പോള്‍ അദ്ദേഹം അവര്‍ക്കുവേണ്ടി ആടുകളെ വെള്ളം കുടിപ്പിച്ചു. പിന്നീട് ഒരു തണലില്‍ ചെന്നിരുന്ന് ഇങ്ങനെ പ്രാര്‍ഥിച്ചു: "എന്റെ നാഥാ, നീയെനിക്കിറക്കിത്തന്ന ഏതൊരുനന്മയ്ക്കും ഏറെ ആവശ്യമുള്ളവനാണ് ഞാന്‍."
\end{malayalam}}
\flushright{\begin{Arabic}
\quranayah[28][25]
\end{Arabic}}
\flushleft{\begin{malayalam}
അപ്പോള്‍ ആ രണ്ടു സ്ത്രീകളിലൊരുവള്‍ ലജ്ജയോടെ അദ്ദേഹത്തെ സമീപിച്ച് ഇങ്ങനെ പറഞ്ഞു: "താങ്കള്‍ ഞങ്ങള്‍ക്കുവേണ്ടി ആടുകളെ വെള്ളം കുടിപ്പിച്ചു. അതിനുള്ള പ്രതിഫലം തരാനായി താങ്കളെ എന്റെ പിതാവ് വിളിക്കുന്നുണ്ട്." അങ്ങനെ മൂസ അദ്ദേഹത്തിന്റെ അടുത്തെത്തി, തന്റെ കഥകളൊക്കെയും വിവരിച്ചുകൊടുത്തു. അതുകേട്ട് ആ വൃദ്ധന്‍ പറഞ്ഞു: "പേടിക്കേണ്ട. അക്രമികളില്‍നിന്ന് താങ്കള്‍ രക്ഷപ്പെട്ടുകഴിഞ്ഞു."
\end{malayalam}}
\flushright{\begin{Arabic}
\quranayah[28][26]
\end{Arabic}}
\flushleft{\begin{malayalam}
ആ രണ്ടു സ്ത്രീകളിലൊരുവള്‍ പറഞ്ഞു: "പിതാവേ, അങ്ങ് ഇദ്ദേഹത്തെ നമ്മുടെ കൂലിക്കാരനാക്കിയാലും. തീര്‍ച്ചയായും അങ്ങ്കൂലിക്കാരായി നിശ്ചയിച്ചിരിക്കുന്നവരില്‍ ഏറ്റവും നല്ലവന്‍ ശക്തനും വിശ്വസ്തനുമായിട്ടുള്ളവനാണ്."
\end{malayalam}}
\flushright{\begin{Arabic}
\quranayah[28][27]
\end{Arabic}}
\flushleft{\begin{malayalam}
വൃദ്ധന്‍ പറഞ്ഞു: "എന്റെ ഈ രണ്ടു പെണ്‍മക്കളില്‍ ഒരുവളെ നിനക്കു വിവാഹം ചെയ്തുതരാന്‍ ഞാന്‍ ഉദ്ദേശിക്കുന്നു. അതിനുള്ള വ്യവസ്ഥയിതാണ്: എട്ടു കൊല്ലം നീയെനിക്ക് കൂലിപ്പണിയെടുക്കണം. അഥവാ പത്തുകൊല്ലം പൂര്‍ത്തിയാക്കുകയാണെങ്കില്‍ അതു നിന്റെയിഷ്ടം. ഞാന്‍ നിന്നെ ഒട്ടും കഷ്ടപ്പെടുത്താനുദ്ദേശിക്കുന്നില്ല. ഞാന്‍ നല്ലവനാണെന്ന് നിനക്കു കണ്ടറിയാം. അല്ലാഹു അനുഗ്രഹിച്ചെങ്കില്‍!"
\end{malayalam}}
\flushright{\begin{Arabic}
\quranayah[28][28]
\end{Arabic}}
\flushleft{\begin{malayalam}
മൂസ പറഞ്ഞു: "നമുക്കിടയിലുള്ള വ്യവസ്ഥ അതുതന്നെ. രണ്ട് അവധികളില്‍ ഏതു പൂര്‍ത്തീകരിച്ചാലും പിന്നെ എന്നോട് വിഷമം തോന്നരുത്. നാം ഇപ്പറയുന്നതിന് അല്ലാഹു സാക്ഷി."
\end{malayalam}}
\flushright{\begin{Arabic}
\quranayah[28][29]
\end{Arabic}}
\flushleft{\begin{malayalam}
അങ്ങനെ മൂസ ആ അവധി പൂര്‍ത്തിയാക്കി. പിന്നെ തന്റെ കുടുംബത്തെയും കൂട്ടി യാത്ര തിരിച്ചു. അപ്പോള്‍ ആ മലയുടെ ഭാഗത്തുനിന്ന് അദ്ദേഹം തീ കണ്ടു. മൂസ തന്റെ കുടുംബത്തോടു പറഞ്ഞു: "നില്‍ക്കൂ. ഞാന്‍ തീ കാണുന്നുണ്ട്. അവിടെ നിന്നു വല്ല വിവരവുമായി വരാം. അല്ലെങ്കില്‍ നിങ്ങള്‍ക്കൊരു തീക്കൊള്ളി കൊണ്ടുവന്നുതരാം. നിങ്ങള്‍ക്കു തീ കായാമല്ലോ."
\end{malayalam}}
\flushright{\begin{Arabic}
\quranayah[28][30]
\end{Arabic}}
\flushleft{\begin{malayalam}
അങ്ങനെ അദ്ദേഹം അതിനടുത്തെത്തി. അപ്പോള്‍ അനുഗൃഹീതമായ ആ പ്രദേശത്തെ താഴ്വരയുടെ വലതുവശത്തെ വൃക്ഷത്തില്‍നിന്ന് ഒരശരീരിയുണ്ടായി. "മൂസാ, സംശയം വേണ്ട; ഞാനാണ് അല്ലാഹു. സര്‍വലോകസംരക്ഷകന്‍.
\end{malayalam}}
\flushright{\begin{Arabic}
\quranayah[28][31]
\end{Arabic}}
\flushleft{\begin{malayalam}
"നിന്റെ വടി താഴെയിടൂ." അതോടെ അത് പാമ്പിനെപ്പോലെ ഇഴയാന്‍ തുടങ്ങി. ഇതുകണ്ട് അദ്ദേഹം പേടിച്ച് പിന്തിരിഞ്ഞോടി. തിരിഞ്ഞുനോക്കിയതുപോലുമില്ല. അല്ലാഹു പറഞ്ഞു: "മൂസാ, തിരിച്ചുവരിക. പേടിക്കേണ്ട. നീ തികച്ചും സുരക്ഷിതനാണ്.
\end{malayalam}}
\flushright{\begin{Arabic}
\quranayah[28][32]
\end{Arabic}}
\flushleft{\begin{malayalam}
"നീ നിന്റെ കൈ കുപ്പായത്തിന്റെ മാറിലേക്ക് കടത്തിവെക്കുക. ന്യൂനതയൊന്നുമില്ലാതെ വെളുത്തുതിളങ്ങുന്നതായി അതു പുറത്തുവരും. പേടി വിട്ടുപോകാന്‍ നിന്റെ കൈ ശരീരത്തോടു ചേര്‍ത്ത് പിടിക്കുക. ഫറവോന്റെയും അവന്റെ പ്രമാണിമാരുടെയും അടുത്തേക്ക്, നിന്റെ നാഥനില്‍ നിന്നുള്ള തെളിവുകളാണ് ഇവ രണ്ടും. അവര്‍ ഏറെ ധിക്കാരികളായ ജനം തന്നെ."
\end{malayalam}}
\flushright{\begin{Arabic}
\quranayah[28][33]
\end{Arabic}}
\flushleft{\begin{malayalam}
മൂസ പറഞ്ഞു: "എന്റെ നാഥാ, അവരിലൊരുവനെ ഞാന്‍ കൊന്നിട്ടുണ്ട്. അതിനാല്‍ അവരെന്നെ കൊന്നുകളയുമെന്ന് ഞാന്‍ ഭയപ്പെടുന്നു.
\end{malayalam}}
\flushright{\begin{Arabic}
\quranayah[28][34]
\end{Arabic}}
\flushleft{\begin{malayalam}
"എന്റെ സഹോദര്‍ ഹാറൂന്‍ എന്നെക്കാള്‍ സ്ഫുടമായി സംസാരിക്കാന്‍ കഴിയുന്നവനാണ്. അതിനാല്‍ അവനെ എന്നോടൊപ്പം എനിക്കൊരു സഹായിയായി അയച്ചുതരിക. അവന്‍ എന്റെ സത്യത ബോധ്യപ്പെടുത്തിക്കൊള്ളും. അവരെന്നെ തള്ളിപ്പറയുമോ എന്നു ഞാന്‍ ആശങ്കിക്കുന്നു."
\end{malayalam}}
\flushright{\begin{Arabic}
\quranayah[28][35]
\end{Arabic}}
\flushleft{\begin{malayalam}
അല്ലാഹു പറഞ്ഞു: "നിന്റെ സഹോദരനിലൂടെ നിന്റെ കൈക്കു നാം കരുത്തേകും. നിങ്ങള്‍ക്കിരുവര്‍ക്കും നാം സ്വാധീനമുണ്ടാക്കും. അതിനാല്‍ അവര്‍ക്കു നിങ്ങളെ ദ്രോഹിക്കാനാവില്ല. നമ്മുടെ ദൃഷ്ടാന്തങ്ങള്‍ കാരണം നിങ്ങളും നിങ്ങളെ പിന്തുടര്‍ന്നവരും തന്നെയായിരിക്കും വിജയികള്‍."
\end{malayalam}}
\flushright{\begin{Arabic}
\quranayah[28][36]
\end{Arabic}}
\flushleft{\begin{malayalam}
അങ്ങനെ നമ്മുടെ വളരെ പ്രകടമായ അടയാളങ്ങളുമായി മൂസ അവരുടെ അടുത്തെത്തി. അവര്‍ പറഞ്ഞു: "ഇതു കെട്ടിച്ചമച്ച ജാലവിദ്യയല്ലാതൊന്നുമല്ല. നമ്മുടെ പൂര്‍വപിതാക്കളില്‍ ഇങ്ങനെയൊന്ന് നാം കേട്ടിട്ടേയില്ലല്ലോ."
\end{malayalam}}
\flushright{\begin{Arabic}
\quranayah[28][37]
\end{Arabic}}
\flushleft{\begin{malayalam}
മൂസ പറഞ്ഞു: "എന്റെ നാഥന് നന്നായറിയാം; അവന്റെ അടുത്തുനിന്ന് നേര്‍വഴിയുമായി വന്നത് ആരാണെന്ന്. ഈ ലോകത്തിന്റെ അന്ത്യം ആര്‍ക്കനുകൂലമാകുമെന്നും. തീര്‍ച്ചയായും അതിക്രമികള്‍ വിജയിക്കുകയില്ല."
\end{malayalam}}
\flushright{\begin{Arabic}
\quranayah[28][38]
\end{Arabic}}
\flushleft{\begin{malayalam}
ഫറവോന്‍ പറഞ്ഞു: "അല്ലയോ പ്രമാണിമാരേ, ഞാനല്ലാതെ നിങ്ങള്‍ക്കൊരു ദൈവമുള്ളതായി എനിക്കറിയില്ല. അതിനാല്‍ ഹാമാനേ, എനിക്കുവേണ്ടി കളിമണ്ണ് ചുട്ട് അത്യുന്നതമായ ഒരു ഗോപുരമുണ്ടാക്കുക. മൂസയുടെ ദൈവത്തെ ഞാനൊന്ന് എത്തിനോക്കട്ടെ. ഉറപ്പായും അവന്‍ കള്ളം പറയുന്നവനാണെന്ന് ഞാന്‍ കരുതുന്നു."
\end{malayalam}}
\flushright{\begin{Arabic}
\quranayah[28][39]
\end{Arabic}}
\flushleft{\begin{malayalam}
അവനും അവന്റെ പടയാളികളും ഭൂമിയില്‍ അന്യായമായി അഹങ്കരിച്ചു. നമ്മിലേക്ക് മടങ്ങിവരില്ലെന്നാണവര്‍ വിചാരിച്ചത്.
\end{malayalam}}
\flushright{\begin{Arabic}
\quranayah[28][40]
\end{Arabic}}
\flushleft{\begin{malayalam}
അതിനാല്‍ അവനെയും അവന്റെ പടയാളികളെയും നാം പിടികൂടി കടലിലെറിഞ്ഞു. നോക്കൂ; ആ അക്രമികളുടെ അന്ത്യം എവ്വിധമായിരുന്നുവെന്ന്.
\end{malayalam}}
\flushright{\begin{Arabic}
\quranayah[28][41]
\end{Arabic}}
\flushleft{\begin{malayalam}
അവരെ നാം നരകത്തിലേക്കു വിളിക്കുന്ന നായകന്മാരാക്കി. ഒന്നുറപ്പ്; ഉയിര്‍ത്തെഴുന്നേല്‍പുനാളില്‍ അവര്‍ക്കൊരു സഹായവും ലഭിക്കുകയില്ല.
\end{malayalam}}
\flushright{\begin{Arabic}
\quranayah[28][42]
\end{Arabic}}
\flushleft{\begin{malayalam}
ഈ ലോകത്ത് ശാപം അവരെ പിന്തുടരുന്ന അവസ്ഥ നാം ഉണ്ടാക്കി. ഉയിര്‍ത്തെഴുന്നേല്‍പുനാളില്‍ ഉറപ്പായും അവര്‍ തന്നെയായിരിക്കും അങ്ങേയറ്റം നീചന്മാര്‍.
\end{malayalam}}
\flushright{\begin{Arabic}
\quranayah[28][43]
\end{Arabic}}
\flushleft{\begin{malayalam}
മൂസാക്കു നാം വേദപുസ്തകം നല്‍കി. മുന്‍തലമുറകളെ നശിപ്പിച്ചശേഷമാണത്. ജനങ്ങള്‍ക്ക് ഉള്‍ക്കാഴ്ചയും നേര്‍വഴിയും അനുഗ്രഹവുമായാണത്. ഒരു വേള, അവര്‍ ചിന്തിച്ചു മനസ്സിലാക്കിയെങ്കിലോ.
\end{malayalam}}
\flushright{\begin{Arabic}
\quranayah[28][44]
\end{Arabic}}
\flushleft{\begin{malayalam}
മൂസാക്കു നാം നിയമ പ്രമാണം നല്‍കിയപ്പോള്‍ ആ പശ്ചിമ ദിക്കില്‍ നീ ഉണ്ടായിരുന്നില്ല. അതിനു സാക്ഷിയായവരിലും നീയുണ്ടായിരുന്നില്ല.
\end{malayalam}}
\flushright{\begin{Arabic}
\quranayah[28][45]
\end{Arabic}}
\flushleft{\begin{malayalam}
എന്നല്ല; പിന്നീട് പല തലമുറകളെയും നാം കരുപ്പിടിപ്പിച്ചു. അവരിലൂടെ കുറേകാലം കടന്നുപോയി. നമ്മുടെ വചനങ്ങള്‍ ഓതിക്കേള്‍പ്പിച്ചുകൊണ്ട് മദ്യന്‍കാരിലും നീ ഉണ്ടായിരുന്നില്ല. എങ്കിലും നാം നിനക്കു സന്ദേശവാഹകരെ അയക്കുകയായിരുന്നു.
\end{malayalam}}
\flushright{\begin{Arabic}
\quranayah[28][46]
\end{Arabic}}
\flushleft{\begin{malayalam}
നാം മൂസയെ വിളിച്ചപ്പോള്‍ ആമലയുടെ ഭാഗത്തും നീയുണ്ടായിരുന്നില്ല. എന്നാല്‍, നിന്റെ നാഥന്റെ അനുഗ്രഹത്താല്‍ ഇതൊക്കെ നിനക്കറിയിച്ചുതരികയാണ്. ഒരു ജനതക്ക് മുന്നറിയിപ്പ് നല്‍കാനാണിത്. നിനക്കുമുമ്പ് ഒരു മുന്നറിയിപ്പുകാരനും അവരില്‍ വന്നെത്തിയിട്ടില്ല. അവര്‍ ചിന്തിച്ചു മനസ്സിലാക്കിയേക്കാം.
\end{malayalam}}
\flushright{\begin{Arabic}
\quranayah[28][47]
\end{Arabic}}
\flushleft{\begin{malayalam}
തങ്ങളുടെ തന്നെ കൈകള്‍ നേരത്തെ പ്രവര്‍ത്തിച്ചതിന്റെ ഫലമായി വല്ല വിപത്തും അവരെ ബാധിച്ചാല്‍ അവര്‍ ഇങ്ങനെ പറയാതിരിക്കാനാണ് നാം നിന്നെ അയച്ചത്: "ഞങ്ങളുടെ നാഥാ, ഞങ്ങളിലേക്ക് ഒരു ദൂതനെ നിനക്ക് നിയോഗിച്ചുകൂടായിരുന്നോ? എങ്കില്‍ ഞങ്ങള്‍ നിന്റെ കല്‍പനകള്‍ പിന്‍പറ്റുകയും സത്യവിശ്വാസികളിലുള്‍പ്പെടുകയും ചെയ്യുമായിരുന്നല്ലോ."
\end{malayalam}}
\flushright{\begin{Arabic}
\quranayah[28][48]
\end{Arabic}}
\flushleft{\begin{malayalam}
എന്നാല്‍ നമ്മില്‍ നിന്നുള്ള സത്യം വന്നെത്തിയപ്പോള്‍ അവര്‍ പറഞ്ഞു: "മൂസാക്കു ലഭിച്ചതുപോലുള്ള ദൃഷ്ടാന്തം ഇവനു കിട്ടാത്തതെന്ത്?" എന്നാല്‍ മൂസാക്കു ദൃഷ്ടാന്തം കിട്ടിയിട്ടും ജനം അദ്ദേഹത്തെ തള്ളിപ്പറയുകയല്ലേ ചെയ്തത്? അവര്‍ പറഞ്ഞു: "പരസ്പരം പിന്തുണക്കുന്ന രണ്ടു ജാലവിദ്യക്കാര്‍!" അവര്‍ ഇത്രകൂടി പറഞ്ഞു: "ഞങ്ങളിതാ ഇതിനെയൊക്കെ തള്ളിപ്പറയുന്നു."
\end{malayalam}}
\flushright{\begin{Arabic}
\quranayah[28][49]
\end{Arabic}}
\flushleft{\begin{malayalam}
പറയുക: "ഇവ രണ്ടിനെക്കാളും നേര്‍വഴി കാണിക്കുന്ന ഒരു ഗ്രന്ഥം അല്ലാഹുവിങ്കല്‍ നിന്നിങ്ങ് കൊണ്ടുവരൂ. ഞാനത് പിന്‍പറ്റാം. നിങ്ങള്‍ സത്യവാദികളെങ്കില്‍!"
\end{malayalam}}
\flushright{\begin{Arabic}
\quranayah[28][50]
\end{Arabic}}
\flushleft{\begin{malayalam}
അഥവാ, അവര്‍ നിനക്ക് ഉത്തരം നല്‍കുന്നില്ലെങ്കില്‍ അറിയുക: തങ്ങളുടെ തന്നിഷ്ടങ്ങളെ മാത്രമാണ് അവര്‍ പിന്‍പറ്റുന്നത്. അല്ലാഹുവില്‍ നിന്നുള്ള മാര്‍ഗദര്‍ശനമൊന്നുമില്ലാതെ തന്നിഷ്ടങ്ങളെ പിന്‍പറ്റുന്നവനെക്കാള്‍ വഴിപിഴച്ചവനായി ആരുമില്ല. സംശയമില്ല; അല്ലാഹു അക്രമികളായ ജനത്തെ നേര്‍വഴിയിലാക്കുകയില്ല.
\end{malayalam}}
\flushright{\begin{Arabic}
\quranayah[28][51]
\end{Arabic}}
\flushleft{\begin{malayalam}
നാമവര്‍ക്ക് നമ്മുടെ വചനം അടിക്കടി എത്തിച്ചുകൊടുത്തിട്ടുണ്ട്. അവര്‍ ചിന്തിച്ചു മനസ്സിലാക്കിയെങ്കിലോ.
\end{malayalam}}
\flushright{\begin{Arabic}
\quranayah[28][52]
\end{Arabic}}
\flushleft{\begin{malayalam}
ഇതിനുമുമ്പ് നാം വേദപുസ്തകം നല്‍കിയവര്‍ ഇതില്‍ വിശ്വസിക്കുന്നു.
\end{malayalam}}
\flushright{\begin{Arabic}
\quranayah[28][53]
\end{Arabic}}
\flushleft{\begin{malayalam}
ഇത് അവരെ ഓതിക്കേള്‍പ്പിച്ചാല്‍ അവര്‍ പറയും: "ഞങ്ങളിതില്‍ വിശ്വസിച്ചിരിക്കുന്നു. സംശയമില്ല; ഇതു ഞങ്ങളുടെ നാഥനില്‍ നിന്നുള്ള സത്യം തന്നെ. തീര്‍ച്ചയായും ഇതിനു മുമ്പുതന്നെ ഞങ്ങള്‍ മുസ്ലിംകളായിരുന്നുവല്ലോ."
\end{malayalam}}
\flushright{\begin{Arabic}
\quranayah[28][54]
\end{Arabic}}
\flushleft{\begin{malayalam}
അവര്‍ നന്നായി ക്ഷമിച്ചു. അതിനാല്‍ അവര്‍ക്ക് ഇരട്ടി പ്രതിഫലമുണ്ട്. അവര്‍ തിന്മയെ നന്മകൊണ്ടു നേരിടുന്നവരാണ്. നാം അവര്‍ക്കു നല്‍കിയതില്‍നിന്ന് ചെലവഴിക്കുന്നവരും.
\end{malayalam}}
\flushright{\begin{Arabic}
\quranayah[28][55]
\end{Arabic}}
\flushleft{\begin{malayalam}
പാഴ്മൊഴികള്‍ കേട്ടാല്‍ അവരതില്‍ നിന്ന് വിട്ടകലും. എന്നിട്ടിങ്ങനെ പറയും: "ഞങ്ങളുടെ കര്‍മങ്ങള്‍ ഞങ്ങള്‍ക്ക്; നിങ്ങള്‍ക്ക് നിങ്ങളുടെ കര്‍മങ്ങളും. അവിവേകികളുടെ കൂട്ട് ഞങ്ങള്‍ക്കുവേണ്ട. നിങ്ങള്‍ക്കു സലാം."
\end{malayalam}}
\flushright{\begin{Arabic}
\quranayah[28][56]
\end{Arabic}}
\flushleft{\begin{malayalam}
സംശയമില്ല; നിനക്കിഷ്ടപ്പെട്ടവരെ നേര്‍വഴിയിലാക്കാന്‍ നിനക്കാവില്ല. എന്നാല്‍ അല്ലാഹു അവനിച്ഛിക്കുന്നവരെ നേര്‍വഴിയിലാക്കുന്നു. നേര്‍വഴി നേടുന്നവരെപ്പറ്റി നന്നായറിയുന്നവനാണവന്‍.
\end{malayalam}}
\flushright{\begin{Arabic}
\quranayah[28][57]
\end{Arabic}}
\flushleft{\begin{malayalam}
അവര്‍ പറയുന്നു: "ഞങ്ങള്‍ നിന്നോടൊപ്പം നീ നിര്‍ദേശിക്കുംവിധം നേര്‍വഴി സ്വീകരിച്ചാല്‍ ഞങ്ങളെ ഞങ്ങളുടെ നാട്ടില്‍നിന്ന് പിഴുതെറിയും." എന്നാല്‍ നിര്‍ഭയമായ ഹറം നാം അവര്‍ക്ക് വാസസ്ഥലമായി ഒരുക്കിക്കൊടുത്തിട്ടില്ലേ? എല്ലായിനം പഴങ്ങളും ശേഖരിച്ച് നാമവിടെ കൊണ്ടെത്തിക്കുന്നു. നമ്മുടെ പക്കല്‍ നിന്നുള്ള ഉപജീവനമാണത്. പക്ഷേ, അവരിലേറെ പേരും കാര്യം മനസ്സിലാക്കുന്നില്ല.
\end{malayalam}}
\flushright{\begin{Arabic}
\quranayah[28][58]
\end{Arabic}}
\flushleft{\begin{malayalam}
എത്രയെത്ര നാടുകളെയാണ് നാം നശിപ്പിച്ചത്. അവിടത്തുകാര്‍ ജീവിതാസ്വാദനത്തില്‍ മതിമറന്ന് അഹങ്കരിക്കുന്നവരായിരുന്നു. അതാ അവരുടെ പാര്‍പ്പിടങ്ങള്‍! അവര്‍ക്കുശേഷം അല്‍പംചിലരല്ലാതെ അവിടെ താമസിച്ചിട്ടില്ല. അവസാനം അവയുടെ അവകാശി നാം തന്നെയായി.
\end{malayalam}}
\flushright{\begin{Arabic}
\quranayah[28][59]
\end{Arabic}}
\flushleft{\begin{malayalam}
നിന്റെ നാഥന്‍ ഒരു നാടിനെയും നശിപ്പിക്കുകയില്ല. ജനങ്ങള്‍ക്ക് നമ്മുടെ വചനങ്ങള്‍ വായിച്ചുകേള്‍പ്പിക്കുന്ന ദൂതനെ നാടിന്റെ കേന്ദ്രത്തിലേക്ക് നിയോഗിച്ചിട്ടല്ലാതെ. നാട്ടുകാര്‍ അതിക്രമികളായിരിക്കെയല്ലാതെ ഒരു നാടിനെയും നാം നശിപ്പിച്ചിട്ടില്ല.
\end{malayalam}}
\flushright{\begin{Arabic}
\quranayah[28][60]
\end{Arabic}}
\flushleft{\begin{malayalam}
നിങ്ങള്‍ക്ക് കൈവന്നതെല്ലാം കേവലം ഐഹികജീവിതവിഭവങ്ങളും അതിന്റെ അലങ്കാരവസ്തുക്കളുമാണ്. അല്ലാഹുവിന്റെ അടുത്തുള്ളതാണ് അത്യുത്തമം. അനശ്വരമായിട്ടുള്ളതും അതുതന്നെ. എന്നിട്ടും നിങ്ങളെന്തുകൊണ്ട് ചിന്തിക്കുന്നില്ല?
\end{malayalam}}
\flushright{\begin{Arabic}
\quranayah[28][61]
\end{Arabic}}
\flushleft{\begin{malayalam}
നാം ഒരാള്‍ക്ക് നല്ലൊരു വാഗ്ദാനം നല്‍കി. ആ വാഗ്ദാനം അയാള്‍ക്ക് സഫലമാകും. മറ്റൊരാളെ നാം ഐഹികജീവിതവിഭവങ്ങള്‍ ആസ്വദിപ്പിച്ചു. പിന്നീട് അയാളെ ഉയിര്‍ത്തെഴുന്നേല്‍പുനാളില്‍ നോവേറിയ ശിക്ഷക്കായി ഹാജരാക്കും. ഇരുവരും ഒരേപോലെയാണോ?
\end{malayalam}}
\flushright{\begin{Arabic}
\quranayah[28][62]
\end{Arabic}}
\flushleft{\begin{malayalam}
അല്ലാഹു അവരെ വിളിക്കുന്ന ദിവസം. എന്നിട്ടിങ്ങനെ ചോദിക്കും: "എനിക്കു നിങ്ങള്‍ സങ്കല്‍പിച്ചുവെച്ചിരുന്ന ആ പങ്കാളികളെവിടെ?"
\end{malayalam}}
\flushright{\begin{Arabic}
\quranayah[28][63]
\end{Arabic}}
\flushleft{\begin{malayalam}
ശിക്ഷാവചനം ബാധകമായത് ആരിലാണോ അവര്‍ അന്ന് പറയും: "ഞങ്ങളുടെ നാഥാ, ഇവരെയാണ് ഞങ്ങള്‍ വഴിപിഴപ്പിച്ചത്. ഞങ്ങള്‍ വഴിപിഴച്ചപോലെ ഞങ്ങളിവരെയും പിഴപ്പിച്ചു. ഞങ്ങളിതാ നിന്റെ മുന്നില്‍ ഉത്തരവാദിത്തമൊഴിയുന്നു. ഞങ്ങളെയല്ല ഇവര്‍ പൂജിച്ചുകൊണ്ടിരുന്നത്."
\end{malayalam}}
\flushright{\begin{Arabic}
\quranayah[28][64]
\end{Arabic}}
\flushleft{\begin{malayalam}
അന്ന് ഇവരോടിങ്ങനെ പറയും: "നിങ്ങള്‍ നിങ്ങളുടെ പങ്കാളികളെ വിളിക്കൂ." അപ്പോഴിവര്‍ അവരെ വിളിച്ചുനോക്കും. എന്നാല്‍ അവര്‍ ഇവര്‍ക്ക് ഉത്തരം നല്‍കുകയില്ല. ഇവരോ ശിക്ഷ നേരില്‍ കാണുകയും ചെയ്യും. ഇവര്‍ നേര്‍വഴിയിലായിരുന്നെങ്കില്‍!
\end{malayalam}}
\flushright{\begin{Arabic}
\quranayah[28][65]
\end{Arabic}}
\flushleft{\begin{malayalam}
അല്ലാഹു അവരെ വിളിക്കുന്ന ദിവസത്തെ ഓര്‍ക്കുക: അന്ന് അവന്‍ ചോദിക്കും: "ദൈവദൂതന്മാര്‍ക്ക് എന്ത് ഉത്തരമാണ് നിങ്ങള്‍ നല്‍കിയത്?"
\end{malayalam}}
\flushright{\begin{Arabic}
\quranayah[28][66]
\end{Arabic}}
\flushleft{\begin{malayalam}
അന്നാളില്‍ വര്‍ത്തമാനമൊന്നും പറയാന്‍ അവര്‍ക്കാവില്ല. അവര്‍ക്കൊന്നും പരസ്പരം ചോദിക്കാന്‍പോലും കഴിയില്ല.
\end{malayalam}}
\flushright{\begin{Arabic}
\quranayah[28][67]
\end{Arabic}}
\flushleft{\begin{malayalam}
എന്നാല്‍ പശ്ചാത്തപിച്ചു മടങ്ങുകയും സത്യവിശ്വാസം സ്വീകരിക്കുകയും സല്‍ക്കര്‍മങ്ങള്‍ പ്രവര്‍ത്തിക്കുകയും ചെയ്തവന്‍ വിജയികളിലുള്‍പ്പെട്ടേക്കാം.
\end{malayalam}}
\flushright{\begin{Arabic}
\quranayah[28][68]
\end{Arabic}}
\flushleft{\begin{malayalam}
നിന്റെ നാഥന്‍ താനിച്ഛിക്കുന്നത് സൃഷ്ടിക്കുന്നു. താനിച്ഛിക്കുന്നവരെ തെരഞ്ഞെടുക്കുന്നു. മനുഷ്യര്‍ക്ക് ഈ തെരഞ്ഞെടുപ്പിലൊരു പങ്കുമില്ല. അല്ലാഹു ഏറെ പരിശുദ്ധനാണ്. അവര്‍ പങ്കുചേര്‍ക്കുന്നവയ്ക്കെല്ലാം അതീതനും.
\end{malayalam}}
\flushright{\begin{Arabic}
\quranayah[28][69]
\end{Arabic}}
\flushleft{\begin{malayalam}
അവരുടെ നെഞ്ചകം ഒളിപ്പിച്ചുവെക്കുന്നതും അവര്‍ വെളിപ്പെടുത്തുന്നതുമെല്ലാം നിന്റെ നാഥന്‍ നന്നായറിയുന്നു.
\end{malayalam}}
\flushright{\begin{Arabic}
\quranayah[28][70]
\end{Arabic}}
\flushleft{\begin{malayalam}
അവനാണ് അല്ലാഹു. അവനല്ലാതെ ദൈവമില്ല. ഈ ലോകത്തും പരലോകത്തും സ്തുതിയൊക്കെയും അവനാണ്. കല്‍പനാധികാരവും അവനുതന്നെ. നിങ്ങളൊക്കെ മടങ്ങിച്ചെല്ലുക അവങ്കലേക്കാണ്.
\end{malayalam}}
\flushright{\begin{Arabic}
\quranayah[28][71]
\end{Arabic}}
\flushleft{\begin{malayalam}
പറയുക: നിങ്ങളെപ്പോഴെങ്കിലും ആലോചിച്ചുനോക്കിയിട്ടുണ്ടോ? ഉയിര്‍ത്തെഴുന്നേല്‍പുനാള്‍വരെ അല്ലാഹു നിങ്ങളില്‍ രാവിനെ സ്ഥിരമായി നിലനിര്‍ത്തുന്നുവെന്ന് കരുതുക; എങ്കില്‍ അല്ലാഹുഅല്ലാതെ നിങ്ങള്‍ക്കു വെളിച്ചമെത്തിച്ചുതരാന്‍ മറ്റേതു ദൈവമാണുള്ളത്? നിങ്ങള്‍ കേള്‍ക്കുന്നില്ലേ?
\end{malayalam}}
\flushright{\begin{Arabic}
\quranayah[28][72]
\end{Arabic}}
\flushleft{\begin{malayalam}
പറയുക: നിങ്ങള്‍ എപ്പോഴെങ്കിലും ആലോചിച്ചുനോക്കിയിട്ടുണ്ടോ? ഉയിര്‍ത്തെഴുന്നേല്‍പുനാള്‍ വരെ അല്ലാഹു നിങ്ങളില്‍ പകലിനെ സ്ഥിരമായി നിലനിര്‍ത്തുന്നുവെന്ന് കരുതുക; എങ്കില്‍ നിങ്ങള്‍ക്കു വിശ്രമത്തിനു രാവിനെ കൊണ്ടുവന്നുതരാന്‍ അല്ലാഹുവെക്കൂടാതെ മറ്റേതു ദൈവമാണുള്ളത്? നിങ്ങള്‍ കണ്ടറിയുന്നില്ലേ?
\end{malayalam}}
\flushright{\begin{Arabic}
\quranayah[28][73]
\end{Arabic}}
\flushleft{\begin{malayalam}
അവന്റെ അനുഗ്രഹത്താല്‍ അവന്‍ നിങ്ങള്‍ക്ക് രാപ്പകലുകള്‍ നിശ്ചയിച്ചുതന്നു. നിങ്ങള്‍ക്ക് വിശ്രമിക്കാനും അവന്റെ അനുഗ്രഹങ്ങള്‍ തേടാനുമാണിത്. നിങ്ങള്‍ നന്ദിയുള്ളവരായെങ്കിലോ?
\end{malayalam}}
\flushright{\begin{Arabic}
\quranayah[28][74]
\end{Arabic}}
\flushleft{\begin{malayalam}
ഒരു ദിനം വരും. അന്ന് അല്ലാഹു അവരെ വിളിക്കും. എന്നിട്ടിങ്ങനെ ചോദിക്കും: "നിങ്ങള്‍ സങ്കല്‍പിച്ചുവെച്ചിരുന്ന ആ പങ്കാളികളെവിടെ?"
\end{malayalam}}
\flushright{\begin{Arabic}
\quranayah[28][75]
\end{Arabic}}
\flushleft{\begin{malayalam}
ഓരോ സമുദായത്തില്‍ നിന്നും ഓരോ സാക്ഷിയെ നാം അന്ന് രംഗത്ത് വരുത്തും. എന്നിട്ട് നാം അവരോടു പറയും: "നിങ്ങള്‍ നിങ്ങളുടെ തെളിവുകൊണ്ടുവരൂ!" സത്യം അല്ലാഹുവിന്റേതാണെന്ന് അപ്പോള്‍ അവരറിയും. അവര്‍ കെട്ടിച്ചമച്ചിരുന്നതൊക്കെയും അവരില്‍നിന്ന് തെന്നിമാറുകയും ചെയ്യും.
\end{malayalam}}
\flushright{\begin{Arabic}
\quranayah[28][76]
\end{Arabic}}
\flushleft{\begin{malayalam}
ഖാറൂന്‍ മൂസയുടെ ജനതയില്‍ പെട്ടവനായിരുന്നു. അവന്‍ അവര്‍ക്കെതിരെ അതിക്രമം കാണിച്ചു. നാം അവന്ന് ധാരാളം ഖജനാവുകള്‍ നല്‍കി. ഒരുകൂട്ടം മല്ലന്മാര്‍പോലും അവയുടെ താക്കോല്‍കൂട്ടം ചുമക്കാന്‍ ഏറെ പ്രയാസപ്പെട്ടിരുന്നു. അയാളുടെ ജനത ഇങ്ങനെ പറഞ്ഞ സന്ദര്‍ഭം: "നീ അഹങ്കരിക്കരുത്. അഹങ്കരിക്കുന്നവരെ അല്ലാഹു ഇഷ്ടപ്പെടുകയില്ല.
\end{malayalam}}
\flushright{\begin{Arabic}
\quranayah[28][77]
\end{Arabic}}
\flushleft{\begin{malayalam}
"അല്ലാഹു നിനക്കു തന്നതിലൂടെ നീ പരലോകവിജയം തേടുക. എന്നാല്‍ ഇവിടെ ഇഹലോക ജീവിതത്തില്‍ നിനക്കുള്ള വിഹിതം മറക്കാതിരിക്കുക. അല്ലാഹു നിനക്കു നന്മ ചെയ്തപോലെ നീയും നന്മ ചെയ്യുക. നാട്ടില്‍ നാശം വരുത്താന്‍ തുനിയരുത്. നാശകാരികളെ അല്ലാഹു ഇഷ്ടപ്പെടുകയില്ല."
\end{malayalam}}
\flushright{\begin{Arabic}
\quranayah[28][78]
\end{Arabic}}
\flushleft{\begin{malayalam}
ഖാറൂന്‍ പറഞ്ഞു: "എനിക്കിതൊക്കെ കിട്ടിയത് എന്റെ വശമുള്ള വിദ്യകൊണ്ടാണ്." അവനറിഞ്ഞിട്ടില്ലേ; അവനു മുമ്പ് അവനെക്കാള്‍ കരുത്തും സംഘബലവുമുണ്ടായിരുന്ന അനേകം തലമുറകളെ അല്ലാഹു നശിപ്പിച്ചിട്ടുണ്ടെന്ന്. കുറ്റവാളികളോട് അവരുടെ കുറ്റങ്ങളെക്കുറിച്ച് ചോദിക്കുകപോലുമില്ല.
\end{malayalam}}
\flushright{\begin{Arabic}
\quranayah[28][79]
\end{Arabic}}
\flushleft{\begin{malayalam}
അങ്ങനെ അവന്‍ എല്ലാവിധ ആര്‍ഭാടങ്ങളോടുംകൂടി ജനത്തിനിടയിലേക്ക് ഇറങ്ങിത്തിരിച്ചു. അതുകണ്ട് ഐഹികജീവിതസുഖം കൊതിക്കുന്നവര്‍ പറഞ്ഞു: "ഖാറൂന് കിട്ടിയതുപോലുള്ളത് ഞങ്ങള്‍ക്കും കിട്ടിയിരുന്നെങ്കില്‍! ഖാറൂന്‍ മഹാ ഭാഗ്യവാന്‍ തന്നെ."
\end{malayalam}}
\flushright{\begin{Arabic}
\quranayah[28][80]
\end{Arabic}}
\flushleft{\begin{malayalam}
എന്നാല്‍ അറിവുള്ളവര്‍ പറഞ്ഞതിങ്ങനെയാണ്: "നിങ്ങള്‍ക്കു നാശം! സത്യവിശ്വാസം സ്വീകരിക്കുകയും സല്‍ക്കര്‍മങ്ങള്‍ പ്രവര്‍ത്തിക്കുകയും ചെയ്യുന്നവന്ന് അല്ലാഹുവിന്റെ പ്രതിഫലമാണ് ഏറ്റം നല്ലത്. എന്നാല്‍ ക്ഷമാശീലര്‍ക്കല്ലാതെ അതു ലഭ്യമല്ല."
\end{malayalam}}
\flushright{\begin{Arabic}
\quranayah[28][81]
\end{Arabic}}
\flushleft{\begin{malayalam}
അങ്ങനെ അവനെയും അവന്റെ ഭവനത്തെയും നാം ഭൂമിയില്‍ ആഴ്ത്തി. അപ്പോള്‍ അല്ലാഹുവെക്കൂടാതെ അവനെ സഹായിക്കാന്‍ അവന്റെ കക്ഷികളാരുമുണ്ടായില്ല. സ്വന്തത്തിന് സഹായിയാകാന്‍ അവനു സാധിച്ചതുമില്ല.
\end{malayalam}}
\flushright{\begin{Arabic}
\quranayah[28][82]
\end{Arabic}}
\flushleft{\begin{malayalam}
അതോടെ ഇന്നലെ അവന്റെ സ്ഥാനം മോഹിച്ചിരുന്ന അതേ ആളുകള്‍ പറഞ്ഞു: "കഷ്ടം! അല്ലാഹു തന്റെ ദാസന്മാരില്‍ അവനിച്ഛിക്കുന്നവര്‍ക്ക് ഉപജീവനം ഉദാരമായി നല്‍കുന്നു. അവനിച്ഛിക്കുന്നവര്‍ക്ക് ഇടുക്കം വരുത്തുകയും ചെയ്യുന്നു. അല്ലാഹു നമ്മോട് ഔദാര്യം കാണിച്ചില്ലായിരുന്നുവെങ്കില്‍ നമ്മെയും അവന്‍ ഭൂമിയില്‍ ആഴ്ത്തിക്കളയുമായിരുന്നു. കഷ്ടം! സത്യനിഷേധികള്‍ വിജയം വരിക്കുകയില്ല."
\end{malayalam}}
\flushright{\begin{Arabic}
\quranayah[28][83]
\end{Arabic}}
\flushleft{\begin{malayalam}
ആ പരലോകഭവനം നാം ഏര്‍പ്പെടുത്തിയത് ഭൂമിയില്‍ ധിക്കാരമോ കുഴപ്പമോ ആഗ്രഹിക്കാത്തവര്‍ക്കാണ്. ഒന്നുറപ്പ്; അന്തിമവിജയം ദൈവഭക്തന്മാര്‍ക്കു മാത്രമാണ്.
\end{malayalam}}
\flushright{\begin{Arabic}
\quranayah[28][84]
\end{Arabic}}
\flushleft{\begin{malayalam}
നന്മയുമായി വരുന്നവന് അതിനെക്കാള്‍ മെച്ചമായതു പ്രതിഫലമായി കിട്ടും. എന്നാല്‍ ആരെങ്കിലും തിന്മയുമായി വരുന്നുവെങ്കില്‍ അവര്‍ പ്രവര്‍ത്തിച്ചതിനനുസരിച്ച പ്രതിഫലമേ അവര്‍ക്കുണ്ടാവുകയുള്ളൂ.
\end{malayalam}}
\flushright{\begin{Arabic}
\quranayah[28][85]
\end{Arabic}}
\flushleft{\begin{malayalam}
നിശ്ചയമായും നിനക്ക് ഈ ഖുര്‍ആന്‍ ജീവിതക്രമമായി നിശ്ചയിച്ചവന്‍ നിന്നെ മഹത്തായ ഒരു പരിണതിയിലേക്കു നയിക്കുക തന്നെ ചെയ്യും. പറയുക: എന്റെ നാഥന് നന്നായറിയാം; നേര്‍വഴിയുമായി വന്നവനാരെന്ന്. വ്യക്തമായ വഴികേടിലകപ്പെട്ടവനാരെന്നും.
\end{malayalam}}
\flushright{\begin{Arabic}
\quranayah[28][86]
\end{Arabic}}
\flushleft{\begin{malayalam}
നിനക്ക് ഈ വേദപുസ്തകം ഇറക്കപ്പെടുമെന്ന് നീയൊരിക്കലും പ്രതീക്ഷിച്ചിരുന്നില്ല. നിന്റെ നാഥനില്‍ നിന്നുള്ള കാരുണ്യമാണിത്. അതിനാല്‍ നീ സത്യനിഷേധികള്‍ക്ക് തുണയാകരുത്.
\end{malayalam}}
\flushright{\begin{Arabic}
\quranayah[28][87]
\end{Arabic}}
\flushleft{\begin{malayalam}
അല്ലാഹുവിന്റെ വചനങ്ങള്‍ നിനക്കിറക്കിക്കിട്ടിയശേഷം സത്യനിഷേധികള്‍ നിന്നെ അതില്‍നിന്ന് തെറ്റിക്കാതിരിക്കട്ടെ. നീ ജനങ്ങളെ നിന്റെ നാഥനിലേക്കു ക്ഷണിക്കുക. ഒരിക്കലും ബഹുദൈവ വിശ്വാസികളില്‍ പെട്ടുപോകരുത്.
\end{malayalam}}
\flushright{\begin{Arabic}
\quranayah[28][88]
\end{Arabic}}
\flushleft{\begin{malayalam}
അല്ലാഹുവോടൊപ്പം മറ്റു ദൈവങ്ങളെ വിളിച്ചു പ്രാര്‍ഥിക്കരുത്. അവനല്ലാതെ ദൈവമില്ല. സകല വസ്തുക്കളും നശിക്കും. അവന്റെ സത്തയൊഴികെ. അവനു മാത്രമേ കല്‍പനാധികാരമുള്ളൂ. നിങ്ങളെല്ലാവരും അവങ്കലേക്കു തിരിച്ചുചെല്ലുന്നവരാണ്.
\end{malayalam}}
\chapter{\textmalayalam{അങ്കബൂത് ( എട്ടുകാലി )}}
\begin{Arabic}
\Huge{\centerline{\basmalah}}\end{Arabic}
\flushright{\begin{Arabic}
\quranayah[29][1]
\end{Arabic}}
\flushleft{\begin{malayalam}
അലിഫ്-ലാം-മീം.
\end{malayalam}}
\flushright{\begin{Arabic}
\quranayah[29][2]
\end{Arabic}}
\flushleft{\begin{malayalam}
ജനങ്ങള്‍ വിചാരിക്കുന്നുണ്ടോ; “ഞങ്ങള്‍ വിശ്വസിച്ചിരിക്കുന്നു”വെന്ന് പറയുന്നതുകൊണ്ടുമാത്രം അവരെ വെറുതെ വിട്ടേക്കുമെന്ന്. അവര്‍ പരീക്ഷണ വിധേയരാവാതെ.
\end{malayalam}}
\flushright{\begin{Arabic}
\quranayah[29][3]
\end{Arabic}}
\flushleft{\begin{malayalam}
നിശ്ചയം, അവര്‍ക്കു മുമ്പുണ്ടായിരുന്നവരെ നാം പരീക്ഷിച്ചിട്ടുണ്ട്. അപ്പോള്‍ സത്യവാന്മാര്‍ ആരെന്ന് അല്ലാഹു തിരിച്ചറിയുകതന്നെ ചെയ്യും. കള്ളന്മാരാരെന്നും.
\end{malayalam}}
\flushright{\begin{Arabic}
\quranayah[29][4]
\end{Arabic}}
\flushleft{\begin{malayalam}
തിന്മ ചെയ്തുകൊണ്ടിരിക്കുന്നവര്‍ കരുതുന്നുണ്ടോ; നമ്മെ മറികടന്നുകളയാമെന്ന്. അവരുടെ വിധിത്തീര്‍പ്പ് വളരെ ചീത്ത തന്നെ.
\end{malayalam}}
\flushright{\begin{Arabic}
\quranayah[29][5]
\end{Arabic}}
\flushleft{\begin{malayalam}
അല്ലാഹുവുമായി കണ്ടുമുട്ടണമെന്നാഗ്രഹിക്കുന്നവര്‍ അറിയട്ടെ: അല്ലാഹുവിന്റെ നിശ്ചിത അവധി വന്നെത്തും. അവന്‍ എല്ലാം കേള്‍ക്കുന്നവനും അറിയുന്നവനുമാണ്.
\end{malayalam}}
\flushright{\begin{Arabic}
\quranayah[29][6]
\end{Arabic}}
\flushleft{\begin{malayalam}
ആരെങ്കിലും അല്ലാഹുവിന്റെ മാര്‍ഗത്തില്‍ പൊരുതുന്നുവെങ്കില്‍ തന്റെ തന്നെ നന്മക്കുവേണ്ടിയാണ് അവനതു ചെയ്യുന്നത്. സംശയമില്ല; അല്ലാഹു ലോകരിലാരുടെയും ആശ്രയമാവശ്യമില്ലാത്തവനാണ്.
\end{malayalam}}
\flushright{\begin{Arabic}
\quranayah[29][7]
\end{Arabic}}
\flushleft{\begin{malayalam}
സത്യവിശ്വാസം സ്വീകരിക്കുകയും സല്‍ക്കര്‍മങ്ങള്‍ പ്രവര്‍ത്തിക്കുകയും ചെയ്തവരുടെ തിന്മകള്‍ നാം മായ്ച്ചുകളയും. അവരുടെ ഉത്തമവൃത്തികള്‍ക്കെല്ലാം പ്രതിഫലം നല്‍കുകയും ചെയ്യും.
\end{malayalam}}
\flushright{\begin{Arabic}
\quranayah[29][8]
\end{Arabic}}
\flushleft{\begin{malayalam}
മാതാപിതാക്കളോട് നന്മ ചെയ്യണമെന്ന് നാം മനുഷ്യനെ ഉപദേശിച്ചിരിക്കുന്നു. നിനക്കറിയാത്ത വല്ലതിനെയും എന്റെ പങ്കാളിയാക്കാന്‍ അവര്‍ നിന്നെ നിര്‍ബന്ധിക്കുകയാണെങ്കില്‍ നീ അവരെ അനുസരിക്കരുത്. എന്റെ അടുത്തേക്കാണ് നിങ്ങളുടെയൊക്കെ മടക്കം. അപ്പോള്‍ നിങ്ങള്‍ പ്രവര്‍ത്തിച്ചിരുന്നതിനെപ്പറ്റിയെല്ലാം നിങ്ങളെ നാം വിശദമായി വിവരമറിയിക്കും.
\end{malayalam}}
\flushright{\begin{Arabic}
\quranayah[29][9]
\end{Arabic}}
\flushleft{\begin{malayalam}
അല്ലാഹുവുമായി കണ്ടുമുട്ടണമെന്നാഗ്രഹിക്കുന്നവര്‍ അറിയട്ടെ: അല്ലാഹുവിന്റെ നിശ്ചിത അവധി വന്നെത്തും. അവന്‍ എല്ലാം കേള്‍ക്കുന്നവനും അറിയുന്നവനുമാണ്.
\end{malayalam}}
\flushright{\begin{Arabic}
\quranayah[29][10]
\end{Arabic}}
\flushleft{\begin{malayalam}
രെങ്കിലും അല്ലാഹുവിന്റെ മാര്‍ഗത്തില്‍ പൊരുതുന്നുവെങ്കില്‍ തന്റെ തന്നെ നന്മക്കുവേണ്ടിയാണ് അവനതു ചെയ്യുന്നത്. സംശയമില്ല; അല്ലാഹു ലോകരിലാരുടെയും ആശ്രയമാവശ്യമില്ലാത്തവനാണ്.
\end{malayalam}}
\flushright{\begin{Arabic}
\quranayah[29][11]
\end{Arabic}}
\flushleft{\begin{malayalam}
ത്യവിശ്വാസം സ്വീകരിക്കുകയും സല്‍ക്കര്‍മങ്ങള്‍ പ്രവര്‍ത്തിക്കുകയും ചെയ്തവരുടെ തിന്മകള്‍ നാം മായ്ച്ചുകളയും. അവരുടെ ഉത്തമവൃത്തികള്‍ക്കെല്ലാം പ്രതിഫലം നല്‍കുകയും ചെയ്യും.
\end{malayalam}}
\flushright{\begin{Arabic}
\quranayah[29][12]
\end{Arabic}}
\flushleft{\begin{malayalam}
മാതാപിതാക്കളോട് നന്മ ചെയ്യണമെന്ന് നാം മനുഷ്യനെ ഉപദേശിച്ചിരിക്കുന്നു. നിനക്കറിയാത്ത വല്ലതിനെയും എന്റെ പങ്കാളിയാക്കാന്‍ അവര്‍ നിന്നെ നിര്‍ബന്ധിക്കുകയാണെങ്കില്‍ നീ അവരെ അനുസരിക്കരുത്. എന്റെ അടുത്തേക്കാണ് നിങ്ങളുടെയൊക്കെ മടക്കം. അപ്പോള്‍ നിങ്ങള്‍ പ്രവര്‍ത്തിച്ചിരുന്നതിനെപ്പറ്റിയെല്ലാം നിങ്ങളെ നാം വിശദമായി വിവരമറിയിക്കും.
\end{malayalam}}
\flushright{\begin{Arabic}
\quranayah[29][13]
\end{Arabic}}
\flushleft{\begin{malayalam}
തങ്ങളുടെ പാപഭാരങ്ങള്‍ ഇവര്‍ വഹിക്കും. തങ്ങളുടെ പാപഭാരങ്ങളോടൊപ്പം വേറെയും പാപഭാരങ്ങളും അവര്‍ ചുമക്കേണ്ടിവരും. അവര്‍ കെട്ടിച്ചമച്ചതിനെപ്പറ്റി ഉയിര്‍ത്തെഴുന്നേല്‍പുനാളില്‍ തീര്‍ച്ചയായും അവരെ ചോദ്യംചെയ്യും.
\end{malayalam}}
\flushright{\begin{Arabic}
\quranayah[29][14]
\end{Arabic}}
\flushleft{\begin{malayalam}
നൂഹിനെ നാം അദ്ദേഹത്തിന്റെ ജനതയിലേക്കയച്ചു. തൊള്ളായിരത്തി അമ്പതുകൊല്ലം അദ്ദേഹം അവര്‍ക്കിടയില്‍ കഴിച്ചുകൂട്ടി. അവസാനം അവര്‍ അക്രമികളായിരിക്കെ ജലപ്രളയം അവരെ പിടികൂടി.
\end{malayalam}}
\flushright{\begin{Arabic}
\quranayah[29][15]
\end{Arabic}}
\flushleft{\begin{malayalam}
അപ്പോള്‍ നാം അദ്ദേഹത്തെയും കപ്പലിലെ മറ്റുള്ളവരെയും രക്ഷപ്പെടുത്തി. അങ്ങനെയതിനെ ലോകര്‍ക്ക് ഒരു ദൃഷ്ടാന്തമാക്കി.
\end{malayalam}}
\flushright{\begin{Arabic}
\quranayah[29][16]
\end{Arabic}}
\flushleft{\begin{malayalam}
ഇബ്റാഹീമിനെയും നാം നമ്മുടെ ദൂതനായി നിയോഗിച്ചു. അദ്ദേഹം തന്റെ ജനതയോട് ഇങ്ങനെ പറഞ്ഞ സന്ദര്‍ഭം: "നിങ്ങള്‍ അല്ലാഹുവെ മാത്രം ആരാധിക്കുക. അവനോടു ഭക്തിയുള്ളവരാവുക. അതാണ് നിങ്ങള്‍ക്കുത്തമം. നിങ്ങള്‍ യാഥാര്‍ഥ്യം അറിയുന്നവരെങ്കില്‍!
\end{malayalam}}
\flushright{\begin{Arabic}
\quranayah[29][17]
\end{Arabic}}
\flushleft{\begin{malayalam}
"അല്ലാഹുവെക്കൂടാതെ നിങ്ങള്‍ പൂജിച്ചുകൊണ്ടിരിക്കുന്നത് ചില വിഗ്രഹങ്ങളെയാണ്. നിങ്ങള്‍ കള്ളം കെട്ടിയുണ്ടാക്കുകയാണ്. അല്ലാഹുവെക്കൂടാതെ നിങ്ങള്‍ പൂജിച്ചുകൊണ്ടിരിക്കുന്ന വിഗ്രഹങ്ങള്‍ക്ക് നിങ്ങള്‍ക്കാവശ്യമായ ഉപജീവനം തരാന്‍പോലും കഴിയില്ല. അതിനാല്‍ നിങ്ങള്‍ അല്ലാഹുവോട് ഉപജീവനം തേടുക. അവനെ മാത്രം ആരാധിക്കുക. അവനോടു നന്ദികാണിക്കുക. നിങ്ങളൊക്കെ മടങ്ങിയെത്തുക അവന്റെ അടുത്തേക്കാണ്.
\end{malayalam}}
\flushright{\begin{Arabic}
\quranayah[29][18]
\end{Arabic}}
\flushleft{\begin{malayalam}
"നിങ്ങളിത് കള്ളമാക്കി തള്ളുകയാണെങ്കില്‍ നിങ്ങള്‍ക്കുമുമ്പുള്ള പല സമുദായങ്ങളും അങ്ങനെ ചെയ്തിട്ടുണ്ട്. സന്ദേശം വ്യക്തമായി എത്തിച്ചുതരിക എന്നതല്ലാതെ മറ്റു ബാധ്യതയൊന്നും ദൈവദൂതനില്ല.
\end{malayalam}}
\flushright{\begin{Arabic}
\quranayah[29][19]
\end{Arabic}}
\flushleft{\begin{malayalam}
അവര്‍ ചിന്തിച്ചുനോക്കിയിട്ടില്ലേ? അല്ലാഹു എങ്ങനെ സൃഷ്ടികര്‍മം തുടങ്ങുന്നുവെന്ന്. പിന്നീട് എങ്ങനെ അതാവര്‍ത്തിക്കുന്നുവെന്നും. തീര്‍ച്ചയായും അല്ലാഹുവിന് അത് ഒട്ടും പ്രയാസകരമല്ല.
\end{malayalam}}
\flushright{\begin{Arabic}
\quranayah[29][20]
\end{Arabic}}
\flushleft{\begin{malayalam}
പറയുക: നിങ്ങള്‍ ഭൂമിയിലൂടെ സഞ്ചരിക്കൂ; എന്നിട്ട് അല്ലാഹു എങ്ങനെ സൃഷ്ടി ആരംഭിച്ചുവെന്ന് നോക്കൂ. പിന്നീട് അല്ലാഹു വീണ്ടുമൊരിക്കല്‍കൂടി സൃഷ്ടിക്കുന്നതെങ്ങനെയെന്നും. അല്ലാഹു എല്ലാ കാര്യങ്ങള്‍ക്കും കഴിവുറ്റവന്‍ തന്നെ; തീര്‍ച്ച.
\end{malayalam}}
\flushright{\begin{Arabic}
\quranayah[29][21]
\end{Arabic}}
\flushleft{\begin{malayalam}
അല്ലാഹു അവനിച്ഛിക്കുന്നവരെ ശിക്ഷിക്കുന്നു. അവനിച്ഛിക്കുന്നവരോട് കരുണകാണിക്കുന്നു. അവങ്കലേക്കാണ് നിങ്ങളൊക്കെ തിരിച്ചുചെല്ലുക.
\end{malayalam}}
\flushright{\begin{Arabic}
\quranayah[29][22]
\end{Arabic}}
\flushleft{\begin{malayalam}
നിങ്ങള്‍ക്ക് ഭൂമിയിലവനെ തോല്‍പിക്കാനാവില്ല. ആകാശത്തും സാധ്യമല്ല. അല്ലാഹുവെക്കൂടാതെ നിങ്ങള്‍ക്കൊരു രക്ഷകനില്ല. സഹായിയുമില്ല.
\end{malayalam}}
\flushright{\begin{Arabic}
\quranayah[29][23]
\end{Arabic}}
\flushleft{\begin{malayalam}
അല്ലാഹുവിന്റെ വചനങ്ങളെയും അവനെ കണ്ടുമുട്ടുമെന്നതിനെയും തള്ളിപ്പറയുന്നവര്‍ എന്റെ കാരുണ്യത്തെ സംബന്ധിച്ച് നിരാശരായിരിക്കുന്നു. അവര്‍ക്കുതന്നെയാണ് നോവേറിയ ശിക്ഷയുണ്ടാവുക.
\end{malayalam}}
\flushright{\begin{Arabic}
\quranayah[29][24]
\end{Arabic}}
\flushleft{\begin{malayalam}
അപ്പോള്‍ അദ്ദേഹത്തിന്റെ ജനതയുടെ പ്രതികരണം ഇത്രമാത്രമായിരുന്നു: "നിങ്ങളിവനെ കൊന്നുകളയുക. അല്ലെങ്കില്‍ ചുട്ടെരിക്കുക." എന്നാല്‍ അല്ലാഹു ഇബ്റാഹീമിനെ തിയ്യില്‍നിന്ന് രക്ഷിച്ചു. വിശ്വസിക്കുന്ന ജനത്തിന് ഇതില്‍ പല ദൃഷ്ടാന്തങ്ങളുമുണ്ട്.
\end{malayalam}}
\flushright{\begin{Arabic}
\quranayah[29][25]
\end{Arabic}}
\flushleft{\begin{malayalam}
ഇബ്റാഹീം പറഞ്ഞു: "അല്ലാഹുവിനു പുറമെ നിങ്ങള്‍ ചില വിഗ്രഹങ്ങളെ സ്വീകരിച്ചിരിക്കുന്നു. അത് ഇഹലോകജീവിതത്തില്‍ നിങ്ങള്‍ക്കിടയിലുള്ള സ്നേഹബന്ധം കാരണമായാണ്. എന്നാല്‍ ഉയിര്‍ത്തെഴുന്നേല്‍പുനാളില്‍ നിങ്ങളില്‍ ചിലര്‍ മറ്റുചിലരെ തള്ളിപ്പറയും. പരസ്പരം ശപിക്കും. ഒന്നുറപ്പ്; നിങ്ങളുടെ താവളം നരകത്തീയാണ്. നിങ്ങള്‍ക്കു സഹായികളായി ആരുമുണ്ടാവില്ല."
\end{malayalam}}
\flushright{\begin{Arabic}
\quranayah[29][26]
\end{Arabic}}
\flushleft{\begin{malayalam}
അപ്പോള്‍ ലൂത്വ് അദ്ദേഹത്തില്‍ വിശ്വസിച്ചു. ഇബ്റാഹീം പറഞ്ഞു: "ഞാന്‍ നാടുവിടുകയാണ്. എന്റെ നാഥന്റെ സന്നിധിയിലേക്കു പോവുകയാണ്. സംശയമില്ല; അവന്‍ തന്നെയാണ് പ്രതാപിയും യുക്തിമാനും."
\end{malayalam}}
\flushright{\begin{Arabic}
\quranayah[29][27]
\end{Arabic}}
\flushleft{\begin{malayalam}
അദ്ദേഹത്തിനു നാം ഇസ്ഹാഖിനെയും യഅ്ഖൂബിനെയും സമ്മാനിച്ചു. അദ്ദേഹത്തിന്റെ സന്താനപരമ്പരയില്‍ നാം പ്രവാചകത്വവും വേദവും പ്രദാനം ചെയ്തു. അദ്ദേഹത്തിന് നാം ഇഹലോകത്തുതന്നെ പ്രതിഫലം നല്‍കി. പരലോകത്തോ തീര്‍ച്ചയായും അദ്ദേഹം സച്ചരിതരിലായിരിക്കും.
\end{malayalam}}
\flushright{\begin{Arabic}
\quranayah[29][28]
\end{Arabic}}
\flushleft{\begin{malayalam}
ലൂത്വിനെയും നാം നിയോഗിച്ചു. അദ്ദേഹം തന്റെ ജനതയോടു പറഞ്ഞതോര്‍ക്കുക: "ലോകത്ത് നേരത്തെ ആരും ചെയ്തിട്ടില്ലാത്ത മ്ളേച്ഛവൃത്തിയാണല്ലോ നിങ്ങള്‍ ചെയ്തുകൊണ്ടിരിക്കുന്നത്.
\end{malayalam}}
\flushright{\begin{Arabic}
\quranayah[29][29]
\end{Arabic}}
\flushleft{\begin{malayalam}
"നിങ്ങള്‍ കാമശമനത്തിന് പുരുഷന്മാരെ സമീപിക്കുന്നു. നേരായവഴി കൈവെടിയുന്നു. സദസ്സുകളില്‍പോലും നീചകൃത്യങ്ങള്‍ ചെയ്തുകൂട്ടുന്നു." അപ്പോള്‍ അദ്ദേഹത്തിന്റെ ജനതയുടെ പ്രതികരണം ഇതുമാത്രമായിരുന്നു: "നീ ഞങ്ങള്‍ക്ക് അല്ലാഹുവിന്റെ ശിക്ഷയിങ്ങു കൊണ്ടുവരിക. നീ സത്യവാനെങ്കില്‍."
\end{malayalam}}
\flushright{\begin{Arabic}
\quranayah[29][30]
\end{Arabic}}
\flushleft{\begin{malayalam}
അദ്ദേഹം പ്രാര്‍ഥിച്ചു: "എന്റെ നാഥാ, നാശകാരികളായ ഈ ജനത്തിനെതിരെ നീയെന്നെ തുണക്കേണമേ."
\end{malayalam}}
\flushright{\begin{Arabic}
\quranayah[29][31]
\end{Arabic}}
\flushleft{\begin{malayalam}
നമ്മുടെ ദൂതന്മാര്‍ ഇബ്റാഹീമിന്റെ അടുത്ത് ശുഭവാര്‍ത്തയുമായെത്തി. അപ്പോള്‍ അവര്‍ പറഞ്ഞു: "തീര്‍ച്ചയായും ഞങ്ങള്‍ ഇന്നാട്ടുകാരെ നശിപ്പിക്കാന്‍ പോവുകയാണ്; ഉറപ്പായും ഇവിടത്തുകാര്‍ അതിക്രമികളായിരിക്കുന്നു."
\end{malayalam}}
\flushright{\begin{Arabic}
\quranayah[29][32]
\end{Arabic}}
\flushleft{\begin{malayalam}
ഇബ്റാഹീം പറഞ്ഞു: "അവിടെ ലൂത്വ് ഉണ്ടല്ലോ." അവര്‍ പറഞ്ഞു: "അവിടെ ആരൊക്കെയുണ്ടെന്ന് ഞങ്ങള്‍ക്കു നന്നായറിയാം. അദ്ദേഹത്തെയും കുടുംബത്തെയും ഞങ്ങള്‍ രക്ഷപ്പെടുത്തുക തന്നെ ചെയ്യും. അദ്ദേഹത്തിന്റെ ഭാര്യയെയൊഴികെ. അവള്‍ പിന്മാറി നിന്നവരില്‍ പെട്ടവളാണ്."
\end{malayalam}}
\flushright{\begin{Arabic}
\quranayah[29][33]
\end{Arabic}}
\flushleft{\begin{malayalam}
നമ്മുടെ ദൂതന്മാര്‍ ലൂത്വിന്റെ അടുത്തെത്തി. അപ്പോള്‍ അവരുടെ വരവില്‍ അദ്ദേഹം വല്ലാതെ വിഷമിച്ചു. ഏറെ പരിഭ്രമിക്കുകയും മനസ്സ് തിടുങ്ങുകയും ചെയ്തു. അവര്‍ പറഞ്ഞു: "പേടിക്കേണ്ട. ദുഃഖിക്കുകയും വേണ്ട. തീര്‍ച്ചയായും നിന്നെയും കുടുംബത്തെയും ഞങ്ങള്‍ രക്ഷപ്പെടുത്തും. നിന്റെ ഭാര്യയെയൊഴികെ. അവള്‍ പിന്മാറിനിന്നവരില്‍ പെട്ടവളാണ്."
\end{malayalam}}
\flushright{\begin{Arabic}
\quranayah[29][34]
\end{Arabic}}
\flushleft{\begin{malayalam}
ഇന്നാട്ടുകാരുടെമേല്‍ നാം ആകാശത്തുനിന്ന് ശിക്ഷയിറക്കുകതന്നെ ചെയ്യും. കാരണം അവര്‍ തെമ്മാടികളാണെന്നതുതന്നെ.
\end{malayalam}}
\flushright{\begin{Arabic}
\quranayah[29][35]
\end{Arabic}}
\flushleft{\begin{malayalam}
അങ്ങനെ നാമവിടെ വിചാരശീലരായ ജനത്തിന് വ്യക്തമായ ദൃഷ്ടാന്തം ബാക്കിവെച്ചു.
\end{malayalam}}
\flushright{\begin{Arabic}
\quranayah[29][36]
\end{Arabic}}
\flushleft{\begin{malayalam}
മദ്യനിലേക്ക് നാം അവരുടെ സഹോദരന്‍ ശുഐബിനെ അയച്ചു. അദ്ദേഹം പറഞ്ഞു: "എന്റെ ജനമേ, നിങ്ങള്‍ അല്ലാഹുവിനു വഴിപ്പെടുക. അന്ത്യദിനത്തെ പ്രതീക്ഷിക്കുക. നാട്ടില്‍ നാശകാരികളായി കുഴപ്പമുണ്ടാക്കരുത്."
\end{malayalam}}
\flushright{\begin{Arabic}
\quranayah[29][37]
\end{Arabic}}
\flushleft{\begin{malayalam}
അപ്പോള്‍ അവരദ്ദേഹത്തെ തള്ളിപ്പറഞ്ഞു. അതിനാല്‍ ഒരു ഭീകരപ്രകമ്പനം അവരെ പിടികൂടി. അതോടെ അവര്‍ തങ്ങളുടെ വീടുകളില്‍ വീണടിഞ്ഞവരായി.
\end{malayalam}}
\flushright{\begin{Arabic}
\quranayah[29][38]
\end{Arabic}}
\flushleft{\begin{malayalam}
ആദ്, സമൂദ് സമൂഹങ്ങളെയും നാം നശിപ്പിച്ചു. അവരുടെ പാര്‍പ്പിടങ്ങളില്‍ നിന്ന് നിങ്ങള്‍ക്കത് വ്യക്തമായി മനസ്സിലായിട്ടുണ്ടാകുമല്ലോ. അവരുടെ പ്രവര്‍ത്തനങ്ങളെ പിശാച് അവര്‍ക്ക് ഏറെ ചേതോഹരങ്ങളായി തോന്നിപ്പിച്ചു. സത്യമാര്‍ഗത്തില്‍ നിന്ന് പിശാച് അവരെ തടയുകയും ചെയ്തു. സത്യത്തിലവര്‍ ഉള്‍ക്കാഴ്ചയുള്ളവരായിരുന്നു.
\end{malayalam}}
\flushright{\begin{Arabic}
\quranayah[29][39]
\end{Arabic}}
\flushleft{\begin{malayalam}
ഖാറൂനെയും ഫറവോനെയും ഹാമാനെയും നാം നശിപ്പിച്ചു. വ്യക്തമായ തെളിവുകളുമായി മൂസ അവരുടെ അടുത്തു ചെന്നിട്ടുണ്ടായിരുന്നു. അപ്പോള്‍ ഭൂമിയില്‍ അവര്‍ അഹങ്കരിച്ചു. എന്നാല്‍ അവര്‍ക്ക് നമ്മെ മറികടക്കാന്‍ കഴിയുമായിരുന്നില്ല.
\end{malayalam}}
\flushright{\begin{Arabic}
\quranayah[29][40]
\end{Arabic}}
\flushleft{\begin{malayalam}
അങ്ങനെ അവരെയൊക്കെ തങ്ങളുടെ പാപങ്ങളുടെ പേരില്‍ നാം പിടികൂടി. അവരില്‍ ചിലരുടെമേല്‍ ചരല്‍ക്കാറ്റയച്ചു. മറ്റുചിലരെ ഘോരഗര്‍ജനം പിടികൂടി. വേറെ ചിലരെ ഭൂമിയില്‍ ആഴ്ത്തി. ഇനിയും ചിലരെ മുക്കിക്കൊന്നു. അല്ലാഹു അവരോടൊന്നും അക്രമം കാണിക്കുകയായിരുന്നില്ല. മറിച്ച് അവര്‍ തങ്ങളോടുതന്നെ അതിക്രമം പ്രവര്‍ത്തിക്കുകയായിരുന്നു.
\end{malayalam}}
\flushright{\begin{Arabic}
\quranayah[29][41]
\end{Arabic}}
\flushleft{\begin{malayalam}
അല്ലാഹുവെക്കൂടാതെ രക്ഷാധികാരികളെ സ്വീകരിക്കുന്നവരുടെ അവസ്ഥ എട്ടുകാലിയുടേതുപോലെയാണ്. അതൊരു വീടുണ്ടാക്കി. വീടുകളിലേറ്റം ദുര്‍ബലം എട്ടുകാലിയുടെ വീടാണ്. അവര്‍ കാര്യം ഗ്രഹിക്കുന്നവരെങ്കില്‍!
\end{malayalam}}
\flushright{\begin{Arabic}
\quranayah[29][42]
\end{Arabic}}
\flushleft{\begin{malayalam}
തന്നെ വെടിഞ്ഞ് അവര്‍ വിളിച്ചു പ്രാര്‍ഥിക്കുന്ന ഏതൊരു വസ്തുവെ സംബന്ധിച്ചും അല്ലാഹു സൂക്ഷ്മമായി അറിയുന്നുണ്ട്. അവന്‍ പ്രതാപിയും യുക്തിമാനുമാണ്.
\end{malayalam}}
\flushright{\begin{Arabic}
\quranayah[29][43]
\end{Arabic}}
\flushleft{\begin{malayalam}
മനുഷ്യര്‍ക്കുവേണ്ടിയാണ് നാമിങ്ങനെ ഉപമകള്‍ വിശദീകരിക്കുന്നത്. എന്നാല്‍ വിചാരമതികളല്ലാതെ അതേക്കുറിച്ച് ചിന്തിച്ച് മനസ്സിലാക്കുന്നില്ല.
\end{malayalam}}
\flushright{\begin{Arabic}
\quranayah[29][44]
\end{Arabic}}
\flushleft{\begin{malayalam}
അല്ലാഹു ആകാശഭൂമികളെ യാഥാര്‍ഥ്യത്തോടെ സൃഷ്ടിച്ചു. നിശ്ചയം; സത്യവിശ്വാസികള്‍ക്ക് അതില്‍ ഒരു ദൃഷ്ടാന്തമുണ്ട്.
\end{malayalam}}
\flushright{\begin{Arabic}
\quranayah[29][45]
\end{Arabic}}
\flushleft{\begin{malayalam}
ഈ വേദപുസ്തകത്തില്‍ നിനക്കു ബോധനമായി ലഭിച്ചവ നീ ഓതിക്കേള്‍പ്പിക്കുക. നമസ്കാരം നിഷ്ഠയോടെ നിര്‍വഹിക്കുക. നിശ്ചയമായും നമസ്കാരം നീചകൃത്യങ്ങളെയും നിഷിദ്ധകര്‍മങ്ങളെയും തടഞ്ഞുനിര്‍ത്തുന്നു. ദൈവസ്മരണയാണ് ഏറ്റവും മഹത്തരം. ഓര്‍ക്കുക: നിങ്ങള്‍ ചെയ്യുന്നതെന്തും അല്ലാഹു നന്നായി അറിയുന്നുണ്ട്.
\end{malayalam}}
\flushright{\begin{Arabic}
\quranayah[29][46]
\end{Arabic}}
\flushleft{\begin{malayalam}
ഏറ്റവും നല്ലരീതിയിലല്ലാതെ നിങ്ങള്‍ വേദക്കാരുമായി സംവാദത്തിലേര്‍പ്പെടരുത്; അവരിലെ അതിക്രമികളോടൊഴികെ. നിങ്ങള്‍ പറയൂ: "ഞങ്ങള്‍ക്ക് ഇറക്കിത്തന്നതിലും നിങ്ങള്‍ക്ക് ഇറക്കിത്തന്നതിലും ഞങ്ങള്‍ വിശ്വസിച്ചിരിക്കുന്നു. ഞങ്ങളുടെ ദൈവവും നിങ്ങളുടെ ദൈവവും ഒന്നുതന്നെ. ഞങ്ങള്‍ അവനെ മാത്രം അനുസരിക്കുന്നവരാണ്."
\end{malayalam}}
\flushright{\begin{Arabic}
\quranayah[29][47]
\end{Arabic}}
\flushleft{\begin{malayalam}
അവ്വിധം നിനക്കു നാം വേദപുസ്തകം ഇറക്കിത്തന്നിരിക്കുന്നു. നാം നേരത്തെ വേദം നല്‍കിയവര്‍ ഇതില്‍ വിശ്വസിക്കുന്നവരാണ്. ഇക്കൂട്ടരിലും ഇതില്‍ വിശ്വസിക്കുന്ന ചിലരുണ്ട്. സത്യനിഷേധികളല്ലാതെ നമ്മുടെ വചനങ്ങളെ തള്ളിപ്പറയുകയില്ല.
\end{malayalam}}
\flushright{\begin{Arabic}
\quranayah[29][48]
\end{Arabic}}
\flushleft{\begin{malayalam}
ഇതിനുമുമ്പ് നീ ഒരൊറ്റ പുസ്തകവും പാരായണം ചെയ്തിട്ടില്ല. നിന്റെ വലതുകൈകൊണ്ട് നീ അതെഴുതിയിട്ടുമില്ല. അങ്ങനെചെയ്തിരുന്നെങ്കില്‍ ഈ സത്യനിഷേധികള്‍ക്ക് സംശയിക്കാമായിരുന്നു.
\end{malayalam}}
\flushright{\begin{Arabic}
\quranayah[29][49]
\end{Arabic}}
\flushleft{\begin{malayalam}
യഥാര്‍ഥത്തില്‍ ജ്ഞാനം വന്നെത്തിയവരുടെ ഹൃദയങ്ങളില്‍ ഇത് സുവ്യക്തമായ വചനങ്ങള്‍ തന്നെയാണ്. അക്രമികളല്ലാതെ നമ്മുടെ വചനങ്ങളെ തള്ളിപ്പറയുകയില്ല.
\end{malayalam}}
\flushright{\begin{Arabic}
\quranayah[29][50]
\end{Arabic}}
\flushleft{\begin{malayalam}
അവര്‍ ചോദിക്കുന്നു: "ഇയാള്‍ക്ക് ഇയാളുടെ നാഥനില്‍നിന്ന് ദൃഷ്ടാന്തങ്ങള്‍ ഇറക്കിക്കൊടുക്കാത്തതെന്ത്?" പറയുക: "ദൃഷ്ടാന്തങ്ങള്‍ അല്ലാഹുവിന്റെ പക്കല്‍ മാത്രമാണുള്ളത്. ഞാനോ വ്യക്തമായ ഒരു മുന്നറിയിപ്പുകാരന്‍ മാത്രം."
\end{malayalam}}
\flushright{\begin{Arabic}
\quranayah[29][51]
\end{Arabic}}
\flushleft{\begin{malayalam}
നാം നിനക്ക് ഈ വേദപുസ്തകം ഇറക്കിത്തന്നു എന്നതുപോരേ അവര്‍ക്ക് തെളിവായി. അതവരെ ഓതിക്കേള്‍പ്പിക്കുന്നുമുണ്ട്. സംശയമില്ല; വിശ്വസിക്കുന്ന ജനത്തിന് അതില്‍ ധാരാളം അനുഗ്രഹമുണ്ട്. മതിയായ ഉദ്ബോധനവും.
\end{malayalam}}
\flushright{\begin{Arabic}
\quranayah[29][52]
\end{Arabic}}
\flushleft{\begin{malayalam}
പറയുക: "എനിക്കും നിങ്ങള്‍ക്കുമിടയില്‍ സാക്ഷിയായി അല്ലാഹുമതി. ആകാശഭൂമികളിലുള്ളതൊക്കെയും അവനറിയുന്നു. എന്നാല്‍ ഓര്‍ക്കുക; അസത്യത്തില്‍ വിശ്വസിക്കുകയും അല്ലാഹുവില്‍ അവിശ്വസിക്കുകയും ചെയ്യുന്നവര്‍ തന്നെയാണ് പരാജിതര്‍.
\end{malayalam}}
\flushright{\begin{Arabic}
\quranayah[29][53]
\end{Arabic}}
\flushleft{\begin{malayalam}
അവര്‍ നിന്നോട് ശിക്ഷക്കായി ധൃതി കൂട്ടുന്നു. കൃത്യമായ കാലാവധി നിശ്ചയിക്കപ്പെട്ടിട്ടില്ലായിരുന്നുവെങ്കില്‍ ശിക്ഷ അവര്‍ക്ക് ഇതിനകം വന്നെത്തിയിട്ടുണ്ടാകുമായിരുന്നു. അവരറിയാതെ പെട്ടെന്ന് അതവരില്‍ വന്നെത്തുകതന്നെ ചെയ്യും.
\end{malayalam}}
\flushright{\begin{Arabic}
\quranayah[29][54]
\end{Arabic}}
\flushleft{\begin{malayalam}
ശിക്ഷക്കായി അവര്‍ നിന്നോടു ധൃതി കൂട്ടുന്നു. സംശയംവേണ്ട; നരകം സത്യനിഷേധികളെ വലയംചെയ്തുനില്‍പുണ്ട്.
\end{malayalam}}
\flushright{\begin{Arabic}
\quranayah[29][55]
\end{Arabic}}
\flushleft{\begin{malayalam}
മുകളില്‍ നിന്നും കാലുകള്‍ക്കടിയില്‍ നിന്നും ശിക്ഷ അവരെ പൊതിയുന്ന ദിനം; അന്ന് അവരോടു പറയും: "നിങ്ങള്‍ പ്രവര്‍ത്തിച്ചിരുന്നതിന്റെ ഫലം അനുഭവിച്ചുകൊള്ളുക."
\end{malayalam}}
\flushright{\begin{Arabic}
\quranayah[29][56]
\end{Arabic}}
\flushleft{\begin{malayalam}
സത്യവിശ്വാസം സ്വീകരിച്ച എന്റെ ദാസന്മാരേ, എന്റെ ഭൂമി വിശാലമാണ്. അതിനാല്‍ നിങ്ങള്‍ എനിക്കുമാത്രം വഴിപ്പെടുക.
\end{malayalam}}
\flushright{\begin{Arabic}
\quranayah[29][57]
\end{Arabic}}
\flushleft{\begin{malayalam}
എല്ലാവരും മരണത്തിന്റെ രുചി അറിയും. പിന്നെ നിങ്ങളെയൊക്കെ നമ്മുടെ അടുത്തേക്ക് തിരിച്ചുകൊണ്ടുവരും.
\end{malayalam}}
\flushright{\begin{Arabic}
\quranayah[29][58]
\end{Arabic}}
\flushleft{\begin{malayalam}
സത്യവിശ്വാസം സ്വീകരിക്കുകയും സല്‍ക്കര്‍മങ്ങള്‍ പ്രവര്‍ത്തിക്കുകയും ചെയ്തവര്‍ക്ക് നാം സ്വര്‍ഗത്തില്‍ സമുന്നത സൌധങ്ങള്‍ ഒരുക്കിവെച്ചിരിക്കുന്നു. അതിന്റെ താഴ്ഭാഗത്തൂടെ ആറുകളൊഴുകിക്കൊണ്ടിരിക്കും. അവരതില്‍ സ്ഥിരവാസികളായിരിക്കും. നന്നായി പ്രവര്‍ത്തിക്കുന്നവര്‍ക്കുള്ള പ്രതിഫലം വളരെ വിശിഷ്ടം തന്നെ.
\end{malayalam}}
\flushright{\begin{Arabic}
\quranayah[29][59]
\end{Arabic}}
\flushleft{\begin{malayalam}
ക്ഷമ പാലിക്കുന്നവരാണവര്‍. തങ്ങളുടെ നാഥനില്‍ ഭരമേല്‍പിക്കുന്നവരും.
\end{malayalam}}
\flushright{\begin{Arabic}
\quranayah[29][60]
\end{Arabic}}
\flushleft{\begin{malayalam}
എത്രയെത്ര ജീവികളുണ്ട്. അവയൊന്നും തങ്ങളുടെ അന്നം ചുമന്നല്ല നടക്കുന്നത്. അല്ലാഹുവാണ് അവയ്ക്കും നിങ്ങള്‍ക്കും ആഹാരം നല്‍കുന്നത്. അവന്‍ എല്ലാം കേള്‍ക്കുന്നവനും അറിയുന്നവനുമാണ്.
\end{malayalam}}
\flushright{\begin{Arabic}
\quranayah[29][61]
\end{Arabic}}
\flushleft{\begin{malayalam}
ആകാശഭൂമികളെ സൃഷ്ടിച്ചതും സൂര്യചന്ദ്രന്മാരെ അധീനപ്പെടുത്തിത്തന്നതും ആരെന്ന് നീ ചോദിച്ചാല്‍ ഉറപ്പായും അവര്‍ പറയും “അല്ലാഹുവാണെ”ന്ന്. എന്നിട്ടും എങ്ങനെയാണ് അവര്‍ക്ക് വ്യതിയാനം സംഭവിക്കുന്നത്?
\end{malayalam}}
\flushright{\begin{Arabic}
\quranayah[29][62]
\end{Arabic}}
\flushleft{\begin{malayalam}
അല്ലാഹു തന്റെ ദാസന്മാരില്‍ അവനിച്ഛിക്കുന്നവര്‍ക്ക് ഉപജീവനത്തില്‍ വിശാലതവരുത്തുന്നു. അവനിച്ഛിക്കുന്നവര്‍ക്ക് അതില്‍ ഇടുക്കവും വരുത്തുന്നു. അല്ലാഹു എല്ലാ കാര്യങ്ങളെപ്പറ്റിയും നന്നായറിയുന്നവനാണ്.
\end{malayalam}}
\flushright{\begin{Arabic}
\quranayah[29][63]
\end{Arabic}}
\flushleft{\begin{malayalam}
മാനത്തുനിന്ന് വെള്ളം വീഴ്ത്തുന്നതും അതുവഴി ഭൂമിയെ അതിന്റെ നിര്‍ജീവതക്കുശേഷം ജീവസ്സുറ്റതാക്കുന്നതും ആരെന്ന് നീ ചോദിച്ചാല്‍ അവര്‍ പറയും “അല്ലാഹുവാണെ”ന്ന്. പറയുക: "സര്‍വ സ്തുതിയും അല്ലാഹുവിനാണ്." എന്നാല്‍ അവരിലേറെ പേരും ചിന്തിച്ചു മനസ്സിലാക്കുന്നില്ല.
\end{malayalam}}
\flushright{\begin{Arabic}
\quranayah[29][64]
\end{Arabic}}
\flushleft{\begin{malayalam}
ഈ ഇഹലോകജീവിതം കളിയും ഉല്ലാസവുമല്ലാതൊന്നുമല്ല. പരലോക ഭവനം തന്നെയാണ് യഥാര്‍ഥ ജീവിതം. അവര്‍ കാര്യം മനസ്സിലാക്കുന്നവരെങ്കില്‍!
\end{malayalam}}
\flushright{\begin{Arabic}
\quranayah[29][65]
\end{Arabic}}
\flushleft{\begin{malayalam}
എന്നാല്‍ അവര്‍ കപ്പലില്‍ കയറിയാല്‍ തങ്ങളുടെ വണക്കവും വഴക്കവുമൊക്കെ ആത്മാര്‍ഥമായും അല്ലാഹുവിനുമാത്രമാക്കി അവനോടു പ്രാര്‍ഥിക്കും. എന്നിട്ട്, അവന്‍ അവരെ രക്ഷപ്പെടുത്തി കരയിലെത്തിച്ചാലോ; അവരതാ അവന് പങ്കാളികളെ സങ്കല്‍പിക്കുന്നു.
\end{malayalam}}
\flushright{\begin{Arabic}
\quranayah[29][66]
\end{Arabic}}
\flushleft{\begin{malayalam}
അങ്ങനെ നാം അവര്‍ക്കു നല്‍കിയതിനോട് അവര്‍ നന്ദികേട് കാണിക്കുന്നു. ആസക്തിയിലാണ്ടുപോവുന്നു. എന്നാല്‍ അടുത്തുതന്നെ അവര്‍ എല്ലാം അറിഞ്ഞുകൊള്ളും.
\end{malayalam}}
\flushright{\begin{Arabic}
\quranayah[29][67]
\end{Arabic}}
\flushleft{\begin{malayalam}
അവര്‍ കാണുന്നില്ലേ; നാം നിര്‍ഭയമായ ഒരാദരണീയ സ്ഥലം ഏര്‍പ്പെടുത്തിയത്. അവരുടെ ചുറ്റുവട്ടത്തുനിന്ന് ആളുകള്‍ റാഞ്ചിയെടുക്കപ്പെട്ടുകൊണ്ടിരിക്കെയാണിത്. എന്നിട്ടും അസത്യത്തില്‍ അവര്‍ വിശ്വസിക്കുകയാണോ; അല്ലാഹുവിന്റെ അനുഗ്രഹങ്ങളെ തള്ളിപ്പറയുകയും.
\end{malayalam}}
\flushright{\begin{Arabic}
\quranayah[29][68]
\end{Arabic}}
\flushleft{\begin{malayalam}
അല്ലാഹുവിന്റെ പേരില്‍ കള്ളം കെട്ടിയുണ്ടാക്കുകയും സത്യം വന്നെത്തിയപ്പോള്‍ അതിനെ കള്ളമാക്കി തള്ളുകയും ചെയ്തവനെക്കാള്‍ കടുത്ത അക്രമി ആരുണ്ട്? ഇത്തരം സത്യനിഷേധികളുടെ വാസസ്ഥലം നരകം തന്നെയല്ലയോ?
\end{malayalam}}
\flushright{\begin{Arabic}
\quranayah[29][69]
\end{Arabic}}
\flushleft{\begin{malayalam}
നമ്മുടെ കാര്യത്തില്‍ സമരം ചെയ്യുന്നവരെ നാം നമ്മുടെ വഴികളിലൂടെ നയിക്കുക തന്നെ ചെയ്യും. സംശയമില്ല; അല്ലാഹു സച്ചരിതരോടൊപ്പമാണ്.
\end{malayalam}}
\chapter{\textmalayalam{റൂം ( റോമാക്കാര്‍ )}}
\begin{Arabic}
\Huge{\centerline{\basmalah}}\end{Arabic}
\flushright{\begin{Arabic}
\quranayah[30][1]
\end{Arabic}}
\flushleft{\begin{malayalam}
അലിഫ്-ലാം-മീം.
\end{malayalam}}
\flushright{\begin{Arabic}
\quranayah[30][2]
\end{Arabic}}
\flushleft{\begin{malayalam}
റോമക്കാര്‍ പരാജിതരായിരിക്കുന്നു.
\end{malayalam}}
\flushright{\begin{Arabic}
\quranayah[30][3]
\end{Arabic}}
\flushleft{\begin{malayalam}
അടുത്ത നാട്ടിലാണിതുണ്ടായത്. തങ്ങളുടെ പരാജയത്തിനുശേഷം അവര്‍ വിജയംവരിക്കും.
\end{malayalam}}
\flushright{\begin{Arabic}
\quranayah[30][4]
\end{Arabic}}
\flushleft{\begin{malayalam}
ഏതാനും കൊല്ലങ്ങള്‍ക്ക കമിതുണ്ടാകും. മുമ്പും പിമ്പും കാര്യങ്ങളുടെ നിയന്ത്രണം അല്ലാഹുവിന്റെ കരങ്ങളിലാണ്. അന്ന് സത്യവിശ്വാസികള്‍ സന്തോഷിക്കും.
\end{malayalam}}
\flushright{\begin{Arabic}
\quranayah[30][5]
\end{Arabic}}
\flushleft{\begin{malayalam}
അല്ലാഹുവിന്റെ സഹായത്താലാണിതുണ്ടാവുക. അവനിച്ഛിക്കുന്നവരെ അവന്‍ സഹായിക്കുന്നു. അവന്‍ പ്രതാപിയും പരമദയാലുവുമാണ്.
\end{malayalam}}
\flushright{\begin{Arabic}
\quranayah[30][6]
\end{Arabic}}
\flushleft{\begin{malayalam}
അല്ലാഹുവിന്റെ വാഗ്ദാനമാണിത്. അല്ലാഹു തന്റെ വാഗ്ദാനം ലംഘിക്കുകയില്ല. പക്ഷേ, മനുഷ്യരിലേറെ പേരും ഇതറിയുന്നില്ല.
\end{malayalam}}
\flushright{\begin{Arabic}
\quranayah[30][7]
\end{Arabic}}
\flushleft{\begin{malayalam}
ഐഹികജീവിതത്തിന്റെ ബാഹ്യവശമേ അവരറിയുന്നുള്ളൂ. പരലോകത്തെപ്പറ്റി അവര്‍ തീര്‍ത്തും അശ്രദ്ധരാണ്.
\end{malayalam}}
\flushright{\begin{Arabic}
\quranayah[30][8]
\end{Arabic}}
\flushleft{\begin{malayalam}
സ്വന്തത്തെ സംബന്ധിച്ച് അവര്‍ ചിന്തിച്ചിട്ടില്ലേ? ആകാശഭൂമികളെയും അവയ്ക്കിടയിലുള്ളവയെയും ശരിയായ ക്രമപ്രകാരവും കൃത്യമായ അവധി നിശ്ചയിച്ചുമല്ലാതെ അല്ലാഹു സൃഷ്ടിച്ചിട്ടില്ല. മനുഷ്യരിലേറെപ്പേരും തങ്ങളുടെ നാഥനെ കണ്ടുമുട്ടുമെന്നതിനെ തള്ളിപ്പറയുന്നവരാണ്.
\end{malayalam}}
\flushright{\begin{Arabic}
\quranayah[30][9]
\end{Arabic}}
\flushleft{\begin{malayalam}
അവര്‍ ഭൂമിയിലൂടെ സഞ്ചരിക്കുന്നില്ലേ? അങ്ങനെ അവര്‍ക്കുമുമ്പുള്ളവരുടെ അന്ത്യം എവ്വിധമായിരുന്നുവെന്ന് നോക്കിക്കാണുന്നില്ലേ? അവര്‍ ഇവരെക്കാളേറെ കരുത്തരായിരുന്നു. അവര്‍ ഭൂമിയെ നന്നായി കിളച്ചുമറിച്ചിരുന്നു. ഇവരതിനെ വാസയോഗ്യമാക്കിയതിനെക്കാള്‍ പാര്‍ക്കാന്‍ പറ്റുന്നതാക്കുകയും ചെയ്തിരുന്നു. അവര്‍ക്കുള്ള ദൂതന്മാര്‍ വ്യക്തമായ തെളിവുകളുമായി അവരെ സമീപിച്ചു. അല്ലാഹു അവരോട് അക്രമം കാണിക്കുകയായിരുന്നില്ല. മറിച്ച് അവര്‍ തങ്ങളോടുതന്നെ അതിക്രമം കാട്ടുകയായിരുന്നു.
\end{malayalam}}
\flushright{\begin{Arabic}
\quranayah[30][10]
\end{Arabic}}
\flushleft{\begin{malayalam}
പിന്നീട് തിന്മ ചെയ്തവരുടെ അന്ത്യം അങ്ങേയറ്റം ദുരന്തപൂര്‍ണമായിരുന്നു. അവര്‍ അല്ലാഹുവിന്റെ വചനങ്ങളെ തള്ളിപ്പറഞ്ഞതിനാലാണിത്. അവയെ അവഹേളിച്ചതിനാലും.
\end{malayalam}}
\flushright{\begin{Arabic}
\quranayah[30][11]
\end{Arabic}}
\flushleft{\begin{malayalam}
സൃഷ്ടി ആരംഭിക്കുന്നത് അല്ലാഹുവാണ്. പിന്നീട് അവനത് ആവര്‍ത്തിക്കുന്നു. അവസാനം നിങ്ങളെല്ലാം അവങ്കലേക്കുതന്നെ മടക്കപ്പെടും.
\end{malayalam}}
\flushright{\begin{Arabic}
\quranayah[30][12]
\end{Arabic}}
\flushleft{\begin{malayalam}
അന്ത്യസമയം വന്നെത്തുംനാളില്‍ കുറ്റവാളികള്‍ പറ്റെ നിരാശരായിത്തീരും.
\end{malayalam}}
\flushright{\begin{Arabic}
\quranayah[30][13]
\end{Arabic}}
\flushleft{\begin{malayalam}
അവര്‍ അല്ലാഹുവിന് കല്‍പിച്ചുവെച്ച പങ്കാളികളില്‍ അവര്‍ക്ക് ശിപാര്‍ശകരായി ആരുമുണ്ടാവില്ല. അവരുടെ പങ്കാളികളെത്തന്നെ അവര്‍ തള്ളിപ്പറയുന്നവരായിത്തീരും.
\end{malayalam}}
\flushright{\begin{Arabic}
\quranayah[30][14]
\end{Arabic}}
\flushleft{\begin{malayalam}
അന്ത്യസമയം വന്നെത്തുംനാളില്‍ അവര്‍ പല വിഭാഗങ്ങളായി പിരിയും.
\end{malayalam}}
\flushright{\begin{Arabic}
\quranayah[30][15]
\end{Arabic}}
\flushleft{\begin{malayalam}
സത്യവിശ്വാസം സ്വീകരിക്കുകയും സല്‍ക്കര്‍മങ്ങള്‍ പ്രവര്‍ത്തിക്കുകയും ചെയ്തവര്‍ പൂന്തോപ്പില്‍ ആനന്ദപുളകിതരായിരിക്കും.
\end{malayalam}}
\flushright{\begin{Arabic}
\quranayah[30][16]
\end{Arabic}}
\flushleft{\begin{malayalam}
എന്നാല്‍ സത്യത്തെ നിഷേധിക്കുകയും നമ്മുടെ വചനങ്ങളെയും പരലോകത്തിലെ നാമുമായുള്ള കണ്ടുമുട്ടലിനെയും തള്ളിപ്പറയുകയും ചെയ്തവര്‍ നോവേറിയ ശിക്ഷക്കായി ഹാജരാക്കപ്പെടും.
\end{malayalam}}
\flushright{\begin{Arabic}
\quranayah[30][17]
\end{Arabic}}
\flushleft{\begin{malayalam}
അതിനാല്‍ നിങ്ങള്‍ വൈകുന്നേരവും രാവിലെയും അല്ലാഹുവിന്റെ വിശുദ്ധിയെ വാഴ്ത്തുക.
\end{malayalam}}
\flushright{\begin{Arabic}
\quranayah[30][18]
\end{Arabic}}
\flushleft{\begin{malayalam}
ആകാശത്തും ഭൂമിയിലും അവനുതന്നെയാണ് സ്തുതി. വൈകുന്നേരവും ഉച്ചതിരിയുമ്പോഴും അവനെ വാഴ്ത്തുവിന്‍.
\end{malayalam}}
\flushright{\begin{Arabic}
\quranayah[30][19]
\end{Arabic}}
\flushleft{\begin{malayalam}
അവന്‍ ജീവനില്ലാത്തതില്‍ നിന്ന് ജീവനുള്ളതിനെ പുറത്തെടുക്കുന്നു. ജീവനുള്ളതില്‍ നിന്ന് ജീവനില്ലാത്തതിനെയും പുറപ്പെടുവിക്കുന്നു. ഭൂമിയെ അതിന്റെ മൃതാവസ്ഥക്കുശേഷം ജീവനുള്ളതാക്കുന്നു. അവ്വിധം നിങ്ങളെയും പുറത്തുകൊണ്ടുവരും.
\end{malayalam}}
\flushright{\begin{Arabic}
\quranayah[30][20]
\end{Arabic}}
\flushleft{\begin{malayalam}
നിങ്ങളെ അവന്‍ മണ്ണില്‍നിന്ന് സൃഷ്ടിച്ചു. എന്നിട്ട് നിങ്ങളിതാ മനുഷ്യരായി ലോകത്ത് വ്യാപരിച്ചുകൊണ്ടിരിക്കുന്നു. ഇത് അവന്റെ ദൃഷ്ടാന്തങ്ങളിലൊന്നാണ്.
\end{malayalam}}
\flushright{\begin{Arabic}
\quranayah[30][21]
\end{Arabic}}
\flushleft{\begin{malayalam}
അല്ലാഹു നിങ്ങളുടെ വര്‍ഗത്തില്‍ നിന്നുതന്നെ നിങ്ങള്‍ക്ക് ഇണകളെ സൃഷ്ടിച്ചുതന്നു. അവരിലൂടെ ശാന്തി തേടാന്‍. നിങ്ങള്‍ക്കിടയില്‍ സ്നേഹവും കാരുണ്യവും ഉണ്ടാക്കി. ഇതൊക്കെയും അല്ലാഹുവിന്റെ ദൃഷ്ടാന്തങ്ങളില്‍ പെട്ടവയാണ്. സംശയമില്ല; വിചാരശാലികളായ ജനത്തിന് ഇതിലെല്ലാം നിരവധി തെളിവുകളുണ്ട്.
\end{malayalam}}
\flushright{\begin{Arabic}
\quranayah[30][22]
\end{Arabic}}
\flushleft{\begin{malayalam}
ആകാശഭൂമികളുടെ സൃഷ്ടിപ്പ്, നിങ്ങളുടെ ഭാഷകളിലെയും വര്‍ണങ്ങളിലെയും വൈവിധ്യം; ഇവയും അവന്റെ അടയാളങ്ങളില്‍പെട്ടവയാണ്. ഇതിലൊക്കെയും അറിവുള്ളവര്‍ക്ക് ധാരാളം ദൃഷ്ടാന്തങ്ങളുണ്ട്.
\end{malayalam}}
\flushright{\begin{Arabic}
\quranayah[30][23]
\end{Arabic}}
\flushleft{\begin{malayalam}
രാപ്പകലുകളിലെ നിങ്ങളുടെ ഉറക്കവും നിങ്ങള്‍ അവന്റെ അനുഗ്രഹം തേടലും അവന്റെ ദൃഷ്ടാന്തങ്ങളില്‍ പെട്ടവയാണ്. കേട്ടുമനസ്സിലാക്കുന്ന ജനത്തിന് ഇതിലും നിരവധി തെളിവുകളുണ്ട്.
\end{malayalam}}
\flushright{\begin{Arabic}
\quranayah[30][24]
\end{Arabic}}
\flushleft{\begin{malayalam}
നിങ്ങള്‍ക്ക് പേടിയും പൂതിയുമുണര്‍ത്തുന്ന മിന്നല്‍പ്പിണര്‍ കാണിച്ചുതരുന്നതും മാനത്തുനിന്ന് മഴവീഴ്ത്തിത്തന്ന് അതിലൂടെ ഭൂമിയെ അതിന്റെ മൃതാവസ്ഥക്കുശേഷം ജീവനുള്ളതാക്കുന്നതും അവന്റെ ദൃഷ്ടാന്തങ്ങളില്‍ പെട്ടവയാണ്. ചിന്തിക്കുന്ന ജനത്തിന് തീര്‍ച്ചയായും ഇതില്‍ ഒട്ടേറെ തെളിവുകളുണ്ട്.
\end{malayalam}}
\flushright{\begin{Arabic}
\quranayah[30][25]
\end{Arabic}}
\flushleft{\begin{malayalam}
ആകാശഭൂമികള്‍ അവന്റെ ഹിതാനുസാരം നിലനില്‍ക്കുന്നുവെന്നതും അവന്റെ ദൃഷ്ടാന്തങ്ങളില്‍ പെട്ടതാണ്. പിന്നെ അവന്‍ ഭൂമിയില്‍നിന്ന് നിങ്ങളെയൊരു വിളിവിളിച്ചാല്‍ പെട്ടെന്നുതന്നെ നിങ്ങള്‍ പുറത്തുവരും.
\end{malayalam}}
\flushright{\begin{Arabic}
\quranayah[30][26]
\end{Arabic}}
\flushleft{\begin{malayalam}
ആകാശഭൂമികളിലുള്ളവയെല്ലാം അവന്റേതാണ്. എല്ലാം അവന് വിധേയവും.
\end{malayalam}}
\flushright{\begin{Arabic}
\quranayah[30][27]
\end{Arabic}}
\flushleft{\begin{malayalam}
സൃഷ്ടി ആരംഭിക്കുന്നത് അവനാണ്. പിന്നെ അവന്‍ തന്നെ അതാവര്‍ത്തിക്കുന്നു. അത് അവന് നന്നെ നിസ്സാരമത്രെ. ആകാശത്തും ഭൂമിയിലും അത്യുന്നതാവസ്ഥ അവന്നാണ്. അവന്‍ പ്രതാപിയും യുക്തിജ്ഞനുമാണ്.
\end{malayalam}}
\flushright{\begin{Arabic}
\quranayah[30][28]
\end{Arabic}}
\flushleft{\begin{malayalam}
നിങ്ങള്‍ക്ക് അവന്‍ നിങ്ങളില്‍ തന്നെയിതാ ഒരുപമ വിവരിച്ചുതരുന്നു: നിങ്ങളുടെ അധീനതയിലുള്ള അടിമകള്‍, നിങ്ങള്‍ക്കു നാം നല്‍കിയ സമ്പത്തില്‍ സമാവകാശികളായിക്കണ്ട് നിങ്ങളവരെ പങ്കാളികളാക്കുന്നുണ്ടോ? സ്വന്തക്കാരെ പേടിക്കുംപോലെ നിങ്ങളവരെ പേടിക്കുന്നുണ്ടോ? ആലോചിച്ചറിയുന്ന ജനത്തിനു നാം ഇവ്വിധം തെളിവുകള്‍ വിശദീകരിച്ചുകൊടുക്കുന്നു.
\end{malayalam}}
\flushright{\begin{Arabic}
\quranayah[30][29]
\end{Arabic}}
\flushleft{\begin{malayalam}
എന്നാല്‍ അതിക്രമം പ്രവര്‍ത്തിച്ചവര്‍ ഒരുവിധ വിവരവുമില്ലാതെ തങ്ങളുടെതന്നെ തന്നിഷ്ടങ്ങളെ പിന്‍പറ്റുകയാണ്. അല്ലാഹു വഴിതെറ്റിച്ചവനെ നേര്‍വഴിയിലാക്കുന്ന ആരുണ്ട്? അവര്‍ക്ക് സഹായികളായി ആരുമുണ്ടാവില്ല.
\end{malayalam}}
\flushright{\begin{Arabic}
\quranayah[30][30]
\end{Arabic}}
\flushleft{\begin{malayalam}
അതിനാല്‍ ശ്രദ്ധയോടെ നീ നിന്റെ മുഖം ഈ മതദര്‍ശനത്തിനുനേരെ ഉറപ്പിച്ചുനിര്‍ത്തുക. അല്ലാഹു മനുഷ്യരെ പടച്ചത് ഏതൊരു പ്രകൃതിയിലൂന്നിയാണോ ആ പ്രകൃതിതന്നെയാണ് ഇത്. അല്ലാഹുവിന്റെ സൃഷ്ടിഘടനക്ക് മാറ്റമില്ല. ഇതുതന്നെയാണ് ഏറ്റം ചൊവ്വായ മതം. പക്ഷേ; ജനങ്ങളിലേറെ പേരും അതറിയുന്നില്ല.
\end{malayalam}}
\flushright{\begin{Arabic}
\quranayah[30][31]
\end{Arabic}}
\flushleft{\begin{malayalam}
നിങ്ങള്‍ അല്ലാഹുവിങ്കലേക്ക് മടങ്ങുന്നവരായി നിലകൊള്ളുക. അവനോട് ഭക്തിപുലര്‍ത്തുക. നമസ്കാരം നിഷ്ഠയോടെ നിര്‍വഹിക്കുക. ബഹുദൈവവിശ്വാസികളില്‍ പെട്ടുപോകരുത്.
\end{malayalam}}
\flushright{\begin{Arabic}
\quranayah[30][32]
\end{Arabic}}
\flushleft{\begin{malayalam}
അഥവാ, തങ്ങളുടെ മതത്തെ ഛിന്നഭിന്നമാക്കുകയും പല കക്ഷികളായി പിരിയുകയും ചെയ്തവരുടെ കൂട്ടത്തില്‍ പെടാതിരിക്കുക. ഓരോ കക്ഷിയും തങ്ങളുടെ വശമുള്ളതില്‍ സന്തുഷ്ടരാണ്.
\end{malayalam}}
\flushright{\begin{Arabic}
\quranayah[30][33]
\end{Arabic}}
\flushleft{\begin{malayalam}
ജനങ്ങളെ വല്ല വിപത്തും ബാധിച്ചാല്‍ അവര്‍ തങ്ങളുടെ നാഥനിലേക്കുതിരിഞ്ഞ് അവനോട് പ്രാര്‍ഥിക്കും. പിന്നീട് അല്ലാഹു അവര്‍ക്ക് തന്റെ അനുഗ്രഹം അനുഭവിക്കാന്‍ അവസരം നല്‍കിയാല്‍ അവരിലൊരു വിഭാഗം തങ്ങളുടെ നാഥനില്‍ പങ്കുകാരെ സങ്കല്‍പിക്കുന്നു.
\end{malayalam}}
\flushright{\begin{Arabic}
\quranayah[30][34]
\end{Arabic}}
\flushleft{\begin{malayalam}
അങ്ങനെ അവര്‍ നാം നല്‍കിയതിനോട് നന്ദികേടു കാണിക്കുന്നു. ശരി, നിങ്ങള്‍ സുഖിച്ചോളൂ. അടുത്തുതന്നെ എല്ലാം നിങ്ങളറിയും.
\end{malayalam}}
\flushright{\begin{Arabic}
\quranayah[30][35]
\end{Arabic}}
\flushleft{\begin{malayalam}
അതല്ല; അവര്‍ അല്ലാഹുവോടു പങ്കുചേര്‍ത്തതിന് അനുകൂലമായി സംസാരിക്കുന്ന വല്ല തെളിവും നാം അവര്‍ക്ക് ഇറക്കിക്കൊടുത്തിട്ടുണ്ടോ?
\end{malayalam}}
\flushright{\begin{Arabic}
\quranayah[30][36]
\end{Arabic}}
\flushleft{\begin{malayalam}
മനുഷ്യര്‍ക്കു നാം വല്ല അനുഗ്രഹവും അനുഭവിക്കാന്‍ അവസരം നല്‍കിയാല്‍ അതിലവര്‍ മതിമറക്കുന്നു. തങ്ങളുടെ തന്നെ കൈകള്‍ നേരത്തെ ചെയ്തുവെച്ചതു കാരണം വല്ല വിപത്തും ബാധിച്ചാലോ; അതോടെ അവരതാ പറ്റെ നിരാശരായിത്തീരുന്നു.
\end{malayalam}}
\flushright{\begin{Arabic}
\quranayah[30][37]
\end{Arabic}}
\flushleft{\begin{malayalam}
അവര്‍ കാണുന്നില്ലേ; അല്ലാഹു അവനിച്ഛിക്കുന്നവര്‍ക്ക് ജീവിതവിഭവം വിപുലമാക്കുന്നത്? അവനിച്ഛിക്കുന്നവര്‍ക്ക് ഇടുക്കം വരുത്തുന്നതും. വിശ്വസിക്കുന്ന ജനത്തിന് തീര്‍ച്ചയായും അതില്‍ ധാരാളം ദൃഷ്ടാന്തങ്ങളുണ്ട്.
\end{malayalam}}
\flushright{\begin{Arabic}
\quranayah[30][38]
\end{Arabic}}
\flushleft{\begin{malayalam}
അതിനാല്‍ അടുത്തബന്ധുക്കള്‍ക്കും അഗതിക്കും വഴിപോക്കന്നും അവരുടെ അവകാശം നല്‍കുക. അല്ലാഹുവിന്റെ പ്രീതി കൊതിക്കുന്നവര്‍ക്ക് അതാണുത്തമം. വിജയം വരിക്കുന്നവരും അവര്‍തന്നെ.
\end{malayalam}}
\flushright{\begin{Arabic}
\quranayah[30][39]
\end{Arabic}}
\flushleft{\begin{malayalam}
ജനങ്ങളുടെ മുതലുകളില്‍ ചേര്‍ന്ന് വളരുന്നതിനുവേണ്ടി നിങ്ങള്‍ നല്‍കുന്ന പലിശയുണ്ടല്ലോ, അത് അല്ലാഹുവിന്റെ അടുത്ത് ഒട്ടും വളരുന്നില്ല. എന്നാല്‍ അല്ലാഹുവിന്റെ പ്രീതി പ്രതീക്ഷിച്ച് നിങ്ങള്‍ വല്ലതും സകാത്തായി നല്‍കുന്നുവെങ്കില്‍, അങ്ങനെ ചെയ്യുന്നവരാണ് അതിനെ ഇരട്ടിപ്പിച്ച് വളര്‍ത്തുന്നവര്‍.
\end{malayalam}}
\flushright{\begin{Arabic}
\quranayah[30][40]
\end{Arabic}}
\flushleft{\begin{malayalam}
അല്ലാഹുവാണ് നിങ്ങളെ സൃഷ്ടിച്ചത്. എന്നിട്ടവന്‍ നിങ്ങള്‍ക്ക് അന്നം തന്നു. പിന്നെ നിങ്ങളെ അവന്‍ മരിപ്പിക്കുന്നു. അതിനുശേഷം വീണ്ടും ജീവിപ്പിക്കും. ഇവയിലേതെങ്കിലും ഒരുകാര്യം ചെയ്യുന്ന ആരെങ്കിലും നിങ്ങള്‍ സങ്കല്‍പിച്ചുവെച്ച പങ്കാളികളിലുണ്ടോ? അവര്‍ സങ്കല്‍പിച്ചുണ്ടാക്കിയ പങ്കാളികളില്‍നിന്നെല്ലാം എത്രയോ പരിശുദ്ധനും അത്യുന്നതനുമാണ് അവന്‍.
\end{malayalam}}
\flushright{\begin{Arabic}
\quranayah[30][41]
\end{Arabic}}
\flushleft{\begin{malayalam}
മനുഷ്യകരങ്ങളുടെ പ്രവര്‍ത്തനഫലമായി കരയിലും കടലിലും കുഴപ്പം പ്രകടമായിരിക്കുന്നു. അവര്‍ ചെയ്തുകൂട്ടിയതില്‍ ചിലതിന്റെയെങ്കിലും ഫലം ഇവിടെ വെച്ചുതന്നെ ആസ്വദിപ്പിക്കാനാണത്. അവര്‍ ഒരുവേള നന്മയിലേക്കു മടങ്ങിയെങ്കിലോ?
\end{malayalam}}
\flushright{\begin{Arabic}
\quranayah[30][42]
\end{Arabic}}
\flushleft{\begin{malayalam}
പറയുക: നിങ്ങള്‍ ഭൂമിയിലൂടെ സഞ്ചരിക്കുക. എന്നിട്ട് മുമ്പുണ്ടായിരുന്നവരുടെ ഒടുക്കം എവ്വിധമായിരുന്നുവെന്ന് നോക്കുക. അവരിലേറെ പേരും ബഹുദൈവാരാധകരായിരുന്നു.
\end{malayalam}}
\flushright{\begin{Arabic}
\quranayah[30][43]
\end{Arabic}}
\flushleft{\begin{malayalam}
അതിനാല്‍ അല്ലാഹുവില്‍ നിന്ന് ആര്‍ക്കും തടുത്തുനിര്‍ത്താനാവാത്ത ഒരുനാള്‍ വന്നെത്തും മുമ്പെ നീ നിന്റെ മുഖത്തെ സത്യമതത്തിന്റെ നേരെ തിരിച്ചുനിര്‍ത്തുക. അന്നാളില്‍ ജനം പലവിഭാഗമായി പിരിയും.
\end{malayalam}}
\flushright{\begin{Arabic}
\quranayah[30][44]
\end{Arabic}}
\flushleft{\begin{malayalam}
ആര്‍ സത്യത്തെ തള്ളിപ്പറയുന്നുവോ ആ സത്യനിഷേധത്തിന്റെ ഫലം അവനുതന്നെയാണുണ്ടാവുക. വല്ലവരും സല്‍ക്കര്‍മം പ്രവര്‍ത്തിക്കുന്നുവെങ്കില്‍ അവര്‍ തങ്ങള്‍ക്കുവേണ്ടിത്തന്നെയാണ് സൌകര്യമൊരുക്കുന്നത്.
\end{malayalam}}
\flushright{\begin{Arabic}
\quranayah[30][45]
\end{Arabic}}
\flushleft{\begin{malayalam}
സത്യവിശ്വാസം സ്വീകരിക്കുകയും സല്‍ക്കര്‍മങ്ങള്‍ പ്രവര്‍ത്തിക്കുകയും ചെയ്തവര്‍ക്ക് അല്ലാഹു തന്റെ അനുഗ്രഹത്തില്‍ നിന്ന് പ്രതിഫലം നല്‍കാന്‍ വേണ്ടിയാണിത്. സംശയമില്ല; അല്ലാഹു സത്യനിഷേധികളെ ഇഷ്ടപ്പെടുന്നില്ല.
\end{malayalam}}
\flushright{\begin{Arabic}
\quranayah[30][46]
\end{Arabic}}
\flushleft{\begin{malayalam}
സന്തോഷസൂചകമായി കാറ്റുകളെ അയക്കുന്നത് അവന്റെ ദൃഷ്ടാന്തങ്ങളില്‍ പെട്ടതാണ്. അവന്റെ അനുഗ്രഹം നിങ്ങളെ ആസ്വദിപ്പിക്കുക; അവന്റെ ഹിതാനുസൃതം കപ്പല്‍ സഞ്ചരിക്കുക; അവന്റെ അനുഗ്രഹത്തില്‍നിന്ന് നിങ്ങള്‍ക്കു അന്നം തേടാനവസരമുണ്ടാവുക; അങ്ങനെ നിങ്ങള്‍ നന്ദിയുള്ളവരായിത്തീരുക; ഇതിനെല്ലാം വേണ്ടിയാണത്.
\end{malayalam}}
\flushright{\begin{Arabic}
\quranayah[30][47]
\end{Arabic}}
\flushleft{\begin{malayalam}
നിനക്കുമുമ്പു നാം നിരവധി ദൂതന്മാരെ അവരുടെ ജനതയിലേക്ക് അയച്ചിട്ടുണ്ട്. അവര്‍ വ്യക്തമായ തെളിവുകളുമായി അവരുടെ അടുത്തുചെന്നു. അപ്പോള്‍ പാപം പ്രവര്‍ത്തിച്ചവരോട് നാം പ്രതികാരം ചെയ്തു. സത്യവിശ്വാസികളെ സഹായിക്കുകയെന്നത് നമ്മുടെ ബാധ്യതയാണ്.
\end{malayalam}}
\flushright{\begin{Arabic}
\quranayah[30][48]
\end{Arabic}}
\flushleft{\begin{malayalam}
കാറ്റുകളെ അയക്കുന്നത് അല്ലാഹുവാണ്. അങ്ങനെ ആ കാറ്റുകള്‍ മേഘത്തെ ചലിപ്പിക്കുന്നു. അവനിച്ഛിക്കുംപോലെ ആ മേഘത്തെ ആകാശത്തു പരത്തുന്നു. അതിനെ പല കഷ്ണങ്ങളാക്കുന്നു. അപ്പോള്‍ അവയ്ക്കിടയില്‍നിന്ന് മഴത്തുള്ളികള്‍ പുറത്തുവരുന്നതായി നിനക്കുകാണാം. അങ്ങനെ അവന്‍ തന്റെ ദാസന്മാരില്‍ നിന്ന് താനിച്ഛിക്കുന്നവര്‍ക്ക് ആ മഴയെത്തിച്ചുകൊടുക്കുന്നു. അതോടെ അവര്‍ ആഹ്ളാദഭരിതരാകുന്നു.
\end{malayalam}}
\flushright{\begin{Arabic}
\quranayah[30][49]
\end{Arabic}}
\flushleft{\begin{malayalam}
അതിനുമുമ്പ്, അഥവാ ആ മഴ അവരുടെമേല്‍ പെയ്യും മുമ്പ് അവര്‍ പറ്റെ നിരാശരായിരുന്നു.
\end{malayalam}}
\flushright{\begin{Arabic}
\quranayah[30][50]
\end{Arabic}}
\flushleft{\begin{malayalam}
നോക്കൂ; ദിവ്യാനുഗ്രഹത്തിന്റെ വ്യക്തമായ അടയാളങ്ങള്‍. ഭൂമിയെ അതിന്റെ മൃതാവസ്ഥക്കുശേഷം അവനെങ്ങനെയാണ് ജീവനുള്ളതാക്കുന്നത്. സംശയമില്ല; അതുചെയ്യുന്നവന്‍ മരിച്ചവരെ ജീവിപ്പിക്കുക തന്നെ ചെയ്യും. അവന്‍ എല്ലാ കാര്യത്തിനും കഴിവുറ്റവനാണ്.
\end{malayalam}}
\flushright{\begin{Arabic}
\quranayah[30][51]
\end{Arabic}}
\flushleft{\begin{malayalam}
ഇനി നാം മറ്റൊരു കാറ്റിനെ അയക്കുന്നു. അതോടെ വിളകള്‍ വിളര്‍ത്ത് മഞ്ഞച്ചതായി അവര്‍ കാണുന്നു. അതിനുശേഷവും അവര്‍ നന്ദികെട്ടവരായിമാറുന്നു.
\end{malayalam}}
\flushright{\begin{Arabic}
\quranayah[30][52]
\end{Arabic}}
\flushleft{\begin{malayalam}
നിനക്കു മരിച്ചവരെ കേള്‍പ്പിക്കാനാവില്ല; തീര്‍ച്ച. പിന്തിരിഞ്ഞുപോകുന്ന കാതുപൊട്ടന്മാരെ വിളി കേള്‍പിക്കാനും നിനക്കു സാധ്യമല്ല.
\end{malayalam}}
\flushright{\begin{Arabic}
\quranayah[30][53]
\end{Arabic}}
\flushleft{\begin{malayalam}
കണ്ണുപൊട്ടന്മാരെ അവരുടെ വഴികേടില്‍ നിന്ന് നേര്‍വഴിയിലേക്കു നയിക്കാനും നിനക്കാവില്ല. നമ്മുടെ വചനങ്ങളില്‍ വിശ്വസിക്കുകയും അങ്ങനെ അനുസരണമുള്ളവരായിത്തീരുകയും ചെയ്തവരെ മാത്രമേ നിനക്കു കേള്‍പ്പിക്കാന്‍ കഴിയുകയുള്ളൂ.
\end{malayalam}}
\flushright{\begin{Arabic}
\quranayah[30][54]
\end{Arabic}}
\flushleft{\begin{malayalam}
നന്നെ ദുര്‍ബലാവസ്ഥയില്‍നിന്ന് നിങ്ങളെ സൃഷ്ടിച്ചുണ്ടാക്കിയത് അല്ലാഹുവാണ്. പിന്നീട് ആ ദുര്‍ബലാവസ്ഥക്കുശേഷം അവന്‍ നിങ്ങള്‍ക്ക് കരുത്തേകി. പിന്നെ ആ കരുത്തിനുശേഷം ദൌര്‍ബല്യവും നരയും ഉണ്ടാക്കി. അവന്‍ താനിച്ഛിക്കുന്നത് സൃഷ്ടിക്കുന്നു. അവന്‍ സകലതും അറിയുന്നവനാണ്. എല്ലാറ്റിനും കഴിവുറ്റവനും.
\end{malayalam}}
\flushright{\begin{Arabic}
\quranayah[30][55]
\end{Arabic}}
\flushleft{\begin{malayalam}
അന്ത്യനിമിഷം വന്നെത്തുംനാളില്‍ കുറ്റവാളികള്‍ ആണയിട്ടു പറയും: "തങ്ങള്‍ ഒരു നാഴിക നേരമല്ലാതെ ഭൂമിയില്‍ കഴിഞ്ഞിട്ടേയില്ല." ഇവ്വിധം തന്നെയാണ് അവര്‍ നേര്‍വഴിയില്‍നിന്ന് വ്യതിചലിച്ചിരുന്നത്.
\end{malayalam}}
\flushright{\begin{Arabic}
\quranayah[30][56]
\end{Arabic}}
\flushleft{\begin{malayalam}
വിജ്ഞാനവും വിശ്വാസവും കൈവന്നവര്‍ പറയും: "അല്ലാഹുവിന്റെ രേഖയനുസരിച്ചുള്ള ഉയിര്‍ത്തെഴുന്നേല്‍പുനാള്‍ വരെ നിങ്ങളവിടെ കഴിച്ചുകൂട്ടിയിട്ടുണ്ട്. ഇപ്പോഴിതാ ആ ഉയിര്‍ത്തെഴുന്നേല്‍പു നാളെത്തിയിരിക്കുന്നു. പക്ഷേ, നിങ്ങള്‍ അതേപ്പറ്റി അറിഞ്ഞിരുന്നില്ല."
\end{malayalam}}
\flushright{\begin{Arabic}
\quranayah[30][57]
\end{Arabic}}
\flushleft{\begin{malayalam}
അന്ന്, അതിക്രമം കാണിച്ചവര്‍ക്ക് തങ്ങളുടെ ഒഴികഴിവ് ഒട്ടും ഉപകരിക്കുകയില്ല. അവരോട് പശ്ചാത്താപത്തിന് ആവശ്യപ്പെടുകയുമില്ല.
\end{malayalam}}
\flushright{\begin{Arabic}
\quranayah[30][58]
\end{Arabic}}
\flushleft{\begin{malayalam}
ജനങ്ങള്‍ക്കായി ഈ ഖുര്‍ആനില്‍ നാം എല്ലാത്തരം ഉപമകളും സമര്‍പ്പിച്ചിട്ടുണ്ട്. എന്നാല്‍ നീ എന്തു തെളിവുമായി അവരുടെ അടുത്തുചെന്നാലും സത്യനിഷേധികള്‍ പറയും: "നിങ്ങള്‍ കേവലം അസത്യവാദികളല്ലാതാരുമല്ല."
\end{malayalam}}
\flushright{\begin{Arabic}
\quranayah[30][59]
\end{Arabic}}
\flushleft{\begin{malayalam}
കാര്യം ഗ്രഹിക്കാനൊരുക്കമില്ലാത്തവരുടെ ഹൃദയങ്ങള്‍ അല്ലാഹു ഇവ്വിധം അടച്ചുപൂട്ടി മുദ്രവെക്കുന്നു.
\end{malayalam}}
\flushright{\begin{Arabic}
\quranayah[30][60]
\end{Arabic}}
\flushleft{\begin{malayalam}
അതിനാല്‍ നീ ക്ഷമിക്കൂ. അല്ലാഹുവിന്റെ വാഗ്ദാനം തീര്‍ത്തും സത്യം തന്നെ. ദൃഢവിശ്വാസമില്ലാത്ത ജനം നിനക്കൊട്ടും ചാഞ്ചല്യം വരുത്താതിരിക്കട്ടെ.
\end{malayalam}}
\chapter{\textmalayalam{ലുഖ്മാന്‍}}
\begin{Arabic}
\Huge{\centerline{\basmalah}}\end{Arabic}
\flushright{\begin{Arabic}
\quranayah[31][1]
\end{Arabic}}
\flushleft{\begin{malayalam}
അലിഫ്-ലാം-മീം.
\end{malayalam}}
\flushright{\begin{Arabic}
\quranayah[31][2]
\end{Arabic}}
\flushleft{\begin{malayalam}
യുക്തിപൂര്‍ണമായ വേദപുസ്തകത്തിലെ വചനങ്ങളാണിത്.
\end{malayalam}}
\flushright{\begin{Arabic}
\quranayah[31][3]
\end{Arabic}}
\flushleft{\begin{malayalam}
സച്ചരിതര്‍ക്കിതൊരനുഗ്രഹമാണ്. വഴികാട്ടിയും.
\end{malayalam}}
\flushright{\begin{Arabic}
\quranayah[31][4]
\end{Arabic}}
\flushleft{\begin{malayalam}
അവര്‍ നമസ്കാരം നിഷ്ഠയോടെ നിര്‍വഹിക്കുന്നവരാണ്. സകാത്ത് നല്‍കുന്നവരാണ്. പരലോകത്തില്‍ അടിയുറച്ചു വിശ്വസിക്കുന്നവരും.
\end{malayalam}}
\flushright{\begin{Arabic}
\quranayah[31][5]
\end{Arabic}}
\flushleft{\begin{malayalam}
അവര്‍ തങ്ങളുടെ നാഥനില്‍ നിന്നുള്ള നേര്‍വഴിയിലാണ്. വിജയികളും അവര്‍ തന്നെ.
\end{malayalam}}
\flushright{\begin{Arabic}
\quranayah[31][6]
\end{Arabic}}
\flushleft{\begin{malayalam}
ജനങ്ങളില്‍ വിടുവാക്കുകള്‍ വിലയ്ക്കു വാങ്ങുന്ന ചിലരുണ്ട്. ഒരു വിവരവുമില്ലാതെ മനുഷ്യരെ ദൈവമാര്‍ഗത്തില്‍ നിന്ന് തെറ്റിച്ചുകളയാന്‍ വേണ്ടിയാണിത്. ദൈവമാര്‍ഗത്തെ പുച്ഛിച്ചുതള്ളാനും. അത്തരക്കാര്‍ക്കാണ് നന്നെ നിന്ദ്യമായ ശിക്ഷയുള്ളത്.
\end{malayalam}}
\flushright{\begin{Arabic}
\quranayah[31][7]
\end{Arabic}}
\flushleft{\begin{malayalam}
അവരിലൊരുവനെ നമ്മുടെ വചനങ്ങള്‍ ഓതിക്കേള്‍പ്പിച്ചാല്‍ അഹങ്കാരത്തോടെ തിരിഞ്ഞുനടക്കും. അങ്ങനെയൊന്നു കേട്ടിട്ടുപോലുമില്ലാത്ത വിധം. അവന്റെ ഇരു കാതുകളിലും അടപ്പുള്ളപോലെ. അതിനാലവനെ നോവേറിയ ശിക്ഷയെ സംബന്ധിച്ച “ശുഭവാര്‍ത്ത” അറിയിക്കുക.
\end{malayalam}}
\flushright{\begin{Arabic}
\quranayah[31][8]
\end{Arabic}}
\flushleft{\begin{malayalam}
സത്യവിശ്വാസം സ്വീകരിക്കുകയും സല്‍ക്കര്‍മങ്ങള്‍ പ്രവര്‍ത്തിക്കുകയും ചെയ്യുന്നവര്‍ക്ക് ഉറപ്പായും അനുഗ്രഹപൂര്‍ണമായ സ്വര്‍ഗീയാരാമങ്ങളുണ്ട്.
\end{malayalam}}
\flushright{\begin{Arabic}
\quranayah[31][9]
\end{Arabic}}
\flushleft{\begin{malayalam}
അവരവിടെ സ്ഥിരവാസികളായിരിക്കും. അല്ലാഹുവിന്റെ അലംഘനീയമായ വാഗ്ദാനമാണിത്. അവന്‍ ഏറെ പ്രതാപിയും യുക്തിമാനുമാണ്.
\end{malayalam}}
\flushright{\begin{Arabic}
\quranayah[31][10]
\end{Arabic}}
\flushleft{\begin{malayalam}
നിങ്ങള്‍ക്കു കാണാന്‍ കഴിയുന്ന തൂണുകളൊന്നുമില്ലാതെ അവന്‍ ആകാശങ്ങളെ സൃഷ്ടിച്ചു. ഭൂമിയില്‍ ഊന്നിയുറച്ച പര്‍വതങ്ങളുണ്ടാക്കി. ഭൂമി നിങ്ങളെയുംകൊണ്ട് ഉലഞ്ഞുപോകാതിരിക്കാന്‍. അതിലവന്‍ സകലയിനം ജീവജാലങ്ങളെയും വ്യാപിപ്പിച്ചു. മാനത്തുനിന്നു മഴ വീഴ്ത്തി. അതുവഴി ഭൂമിയില്‍ നാം സകലയിനം മികച്ച സസ്യങ്ങളേയും മുളപ്പിച്ചു.
\end{malayalam}}
\flushright{\begin{Arabic}
\quranayah[31][11]
\end{Arabic}}
\flushleft{\begin{malayalam}
ഇതൊക്കെയും അല്ലാഹുവിന്റെ സൃഷ്ടിയാണ്. എന്നാല്‍ അവനല്ലാത്തവര്‍ സൃഷ്ടിച്ചത് ഏതെന്ന് നിങ്ങളെനിക്കൊന്നു കാണിച്ചുതരൂ. അല്ല; അതിക്രമികള്‍ വ്യക്തമായ വഴികേടില്‍ തന്നെയാണ്.
\end{malayalam}}
\flushright{\begin{Arabic}
\quranayah[31][12]
\end{Arabic}}
\flushleft{\begin{malayalam}
ലുഖ്മാന്ന് നാം തത്ത്വജ്ഞാനം നല്‍കി. അദ്ദേഹത്തോട് നാം ആവശ്യപ്പെട്ടു: "നീ അല്ലാഹുവിനോടു നന്ദി കാണിക്കുക." ആരെങ്കിലും നന്ദി കാണിക്കുന്നുവെങ്കില്‍ സ്വന്തം നന്മക്കുവേണ്ടിത്തന്നെയാണ് അവനതു ചെയ്യുന്നത്. ആരെങ്കിലും നന്ദികേടു കാണിക്കുകയാണെങ്കിലോ, അറിയുക: തീര്‍ച്ചയായും അല്ലാഹു അന്യാശ്രയമില്ലാത്തവനും സ്തുത്യര്‍ഹനുമാണ്.
\end{malayalam}}
\flushright{\begin{Arabic}
\quranayah[31][13]
\end{Arabic}}
\flushleft{\begin{malayalam}
ലുഖ്മാന്‍ തന്റെ മകനെ ഉപദേശിക്കവെ ഇങ്ങനെ പറഞ്ഞതോര്‍ക്കുക: "എന്റെ കുഞ്ഞുമോനേ, നീ അല്ലാഹുവില്‍ പങ്കുചേര്‍ക്കരുത്. അങ്ങനെ പങ്കുചേര്‍ക്കുന്നത് കടുത്ത അക്രമമാണ്; തീര്‍ച്ച."
\end{malayalam}}
\flushright{\begin{Arabic}
\quranayah[31][14]
\end{Arabic}}
\flushleft{\begin{malayalam}
മാതാപിതാക്കളുടെ കാര്യത്തില്‍ മനുഷ്യനെ നാമുപദേശിച്ചിരിക്കുന്നു. അവന്റെ മാതാവ് മേല്‍ക്കുമേല്‍ ക്ഷീണം സഹിച്ചാണ് അവനെ ഗര്‍ഭം ചുമന്നത്. അവന്റെ മുലകുടി നിറുത്തലോ രണ്ട് കൊല്ലംകൊണ്ടുമാണ്. അതിനാല്‍ നീയെന്നോടു നന്ദി കാണിക്കുക. നിന്റെ മാതാപിതാക്കളോടും. എന്റെ അടുത്തേക്കാണ് നിന്റെ തിരിച്ചുവരവ്.
\end{malayalam}}
\flushright{\begin{Arabic}
\quranayah[31][15]
\end{Arabic}}
\flushleft{\begin{malayalam}
നിനക്കൊരറിവുമില്ലാത്ത വല്ലതിനെയും എന്റെ പങ്കാളിയാക്കാന്‍ അവരിരുവരും നിന്നെ നിര്‍ബന്ധിക്കുകയാണെങ്കില്‍ അക്കാര്യത്തില്‍ അവരെ നീ അനുസരിക്കരുത്. എന്നാലും ഇഹലോകത്ത് അവരോടു നല്ല നിലയില്‍ സഹവസിക്കുക. എന്നിലേക്കു പശ്ചാത്തപിച്ചു മടങ്ങിയവന്റെ പാത പിന്തുടരുക. അവസാനം നിങ്ങളുടെയൊക്കെ മടക്കം എന്നിലേക്കു തന്നെയാണ്. അപ്പോള്‍ നിങ്ങള്‍ ചെയ്തുകൊണ്ടിരുന്നതിനെപ്പറ്റി നിങ്ങളെ വിവരമറിയിക്കും.
\end{malayalam}}
\flushright{\begin{Arabic}
\quranayah[31][16]
\end{Arabic}}
\flushleft{\begin{malayalam}
"എന്റെ കുഞ്ഞുമോനേ, കര്‍മം കടുകുമണിത്തൂക്കത്തോളമാണെന്നു കരുതുക. എന്നിട്ട് അതൊരു പാറക്കല്ലിനുള്ളിലോ ആകാശഭൂമികളിലെവിടെയെങ്കിലുമോ ആണെന്നു വെക്കുക; എന്നാലും അല്ലാഹു അത് പുറത്തുകൊണ്ടുവരിക തന്നെ ചെയ്യും." നിശ്ചയമായും അല്ലാഹു സൂക്ഷ്മജ്ഞനും അഗാധജ്ഞനുമാണ്.
\end{malayalam}}
\flushright{\begin{Arabic}
\quranayah[31][17]
\end{Arabic}}
\flushleft{\begin{malayalam}
"എന്റെ കുഞ്ഞുമോനേ, നീ നമസ്കാരം നിഷ്ഠയോടെ നിര്‍വഹിക്കുക. നന്മ കല്‍പിക്കുക. തിന്മ വിലക്കുക. വിപത്തു വന്നാല്‍, ക്ഷമിക്കുക. ഇവയെല്ലാം ഉറപ്പായും ഊന്നിപ്പറയപ്പെട്ട കാര്യങ്ങളാണ്.
\end{malayalam}}
\flushright{\begin{Arabic}
\quranayah[31][18]
\end{Arabic}}
\flushleft{\begin{malayalam}
"നീ ജനങ്ങളുടെ നേരെ മുഖം കോട്ടരുത്. പൊങ്ങച്ചത്തോടെ ഭൂമിയില്‍ നടക്കരുത്. അഹന്ത നടിച്ചും പൊങ്ങച്ചം കാണിച്ചും നടക്കുന്ന ആരെയും അല്ലാഹു ഇഷ്ടപ്പെടുന്നില്ല; തീര്‍ച്ച.
\end{malayalam}}
\flushright{\begin{Arabic}
\quranayah[31][19]
\end{Arabic}}
\flushleft{\begin{malayalam}
"നീ നിന്റെ നടത്തത്തില്‍ മിതത്വം പുലര്‍ത്തുക. ശബ്ദത്തില്‍ ഒതുക്കം പാലിക്കുക. തീര്‍ച്ചയായും ഒച്ചകളിലേറ്റം അരോചകം കഴുതയുടെ ശബ്ദം തന്നെ!"
\end{malayalam}}
\flushright{\begin{Arabic}
\quranayah[31][20]
\end{Arabic}}
\flushleft{\begin{malayalam}
നിങ്ങള്‍ കാണുന്നില്ലേ; ആകാശഭൂമികളിലുള്ളതെല്ലാം അല്ലാഹു നിങ്ങള്‍ക്ക് അധീനപ്പെടുത്തിത്തന്നത്; ഒളിഞ്ഞതും തെളിഞ്ഞതുമായ അനുഗ്രഹങ്ങള്‍ നിങ്ങള്‍ക്ക്അവന്‍ നിറവേറ്റിത്തന്നതും. എന്നിട്ടും വല്ല വിവരമോ മാര്‍ഗദര്‍ശനമോ വെളിച്ചമേകുന്ന ഗ്രന്ഥമോ ഒന്നുമില്ലാതെ അല്ലാഹുവിന്റെ കാര്യത്തില്‍ തര്‍ക്കിച്ചുകൊണ്ടിരിക്കുന്ന ചിലര്‍ ജനങ്ങളിലുണ്ട്.
\end{malayalam}}
\flushright{\begin{Arabic}
\quranayah[31][21]
\end{Arabic}}
\flushleft{\begin{malayalam}
"അല്ലാഹു ഇറക്കിത്തന്നതിനെ പിന്‍പറ്റുക"യെന്ന് അവരോട് ആവശ്യപ്പെട്ടാല്‍ അവര്‍ പറയും: "അല്ല, ഞങ്ങളുടെ പൂര്‍വപിതാക്കള്‍ ഏതൊരു മാര്‍ഗത്തില്‍ നിലകൊള്ളുന്നതായാണോ ഞങ്ങള്‍ കണ്ടിട്ടുള്ളത് ആ മാര്‍ഗമാണ് ഞങ്ങള്‍ പിന്‍പറ്റുക." കത്തിക്കാളുന്ന നരകത്തീയിലേക്കാണ് പിശാച് അവരെ നയിക്കുന്നതെങ്കില്‍ അതുമവര്‍ പിന്‍പറ്റുമെന്നോ?
\end{malayalam}}
\flushright{\begin{Arabic}
\quranayah[31][22]
\end{Arabic}}
\flushleft{\begin{malayalam}
ആരെങ്കിലും സച്ചരിതനായി സ്വന്തത്തെ അല്ലാഹുവിന് സമര്‍പ്പിക്കുന്നുവെങ്കില്‍ തീര്‍ച്ചയായും അയാള്‍ മുറുകെപ്പിടിച്ചത് ഏറ്റം ഉറപ്പുള്ള പിടിവള്ളിയില്‍ തന്നെയാണ്. കാര്യങ്ങളുടെയൊക്കെ പര്യവസാനം അല്ലാഹുവിന്റെ സന്നിധിയിലാണ്.
\end{malayalam}}
\flushright{\begin{Arabic}
\quranayah[31][23]
\end{Arabic}}
\flushleft{\begin{malayalam}
ആരെങ്കിലും സത്യത്തെ തള്ളിപ്പറയുന്നുവെങ്കില്‍ അയാളുടെ സത്യനിഷേധം നിന്നെ ദുഃഖിപ്പിക്കാതിരിക്കട്ടെ. അവരുടെ മടക്കം നമ്മുടെ അടുത്തേക്കാണ്. അപ്പോള്‍ അവര്‍ പ്രവര്‍ത്തിച്ചുകൊണ്ടിരുന്നതിനെപ്പറ്റി നാമവരെ വിവരമറിയിക്കും. നെഞ്ചകത്തുള്ളതൊക്കെയും നന്നായറിയുന്നവനാണ് അല്ലാഹു.
\end{malayalam}}
\flushright{\begin{Arabic}
\quranayah[31][24]
\end{Arabic}}
\flushleft{\begin{malayalam}
അല്‍പകാലം നാമവരെ സുഖിപ്പിക്കുന്നു. പിന്നീട് നാമവരെ കൊടുംശിക്ഷയിലേക്ക് തള്ളിവിടും.
\end{malayalam}}
\flushright{\begin{Arabic}
\quranayah[31][25]
\end{Arabic}}
\flushleft{\begin{malayalam}
ആകാശഭൂമികളെ പടച്ചതാരെന്നു ചോദിച്ചാല്‍ അവര്‍ പറയും "അല്ലാഹു"വെന്ന്. പറയുക: "സര്‍വ സ്തുതിയും ആ അല്ലാഹുവിനാണ്." എന്നാല്‍ അവരിലേറെ പേരും അത് മനസ്സിലാക്കുന്നില്ല.
\end{malayalam}}
\flushright{\begin{Arabic}
\quranayah[31][26]
\end{Arabic}}
\flushleft{\begin{malayalam}
ആകാശഭൂമികളിലുള്ളതെല്ലാം അല്ലാഹുവിന്റെതാണ്. തീര്‍ച്ചയായും അല്ലാഹു സ്വയംപര്യാപ്തനാണ്. സ്തുത്യര്‍ഹനും.
\end{malayalam}}
\flushright{\begin{Arabic}
\quranayah[31][27]
\end{Arabic}}
\flushleft{\begin{malayalam}
ഭൂമിയിലുള്ള മരങ്ങളൊക്കെയും പേനയാവുക; സമുദ്രങ്ങളെല്ലാം മഷിയാവുക; വേറെയും ഏഴു സമുദ്രങ്ങള്‍ അതിനെ പോഷിപ്പിക്കുക; എന്നാലും അല്ലാഹുവിന്റെ വചനങ്ങള്‍ എഴുതിത്തീര്‍ക്കാനാവില്ല. അല്ലാഹു പ്രതാപിയും യുക്തിമാനും തന്നെ; തീര്‍ച്ച.
\end{malayalam}}
\flushright{\begin{Arabic}
\quranayah[31][28]
\end{Arabic}}
\flushleft{\begin{malayalam}
നിങ്ങളെ സൃഷ്ടിക്കലും ഉയിര്‍ത്തെഴുന്നേല്‍പിക്കലും ഒരൊറ്റയാളെ അങ്ങനെ ചെയ്യും പോലെത്തന്നെയാണ്. സംശയമില്ല; അല്ലാഹു എല്ലാം കേള്‍ക്കുന്നവനും കാണുന്നവനുമാണ്.
\end{malayalam}}
\flushright{\begin{Arabic}
\quranayah[31][29]
\end{Arabic}}
\flushleft{\begin{malayalam}
തീര്‍ച്ചയായും അല്ലാഹു രാവിനെ പകലില്‍ കടത്തിവിടുന്നു; പകലിനെ രാവിലും പ്രവേശിപ്പിക്കുന്നു. അവന്‍ സൂര്യചന്ദ്രന്മാരെ അധീനപ്പെടുത്തിയിരിക്കുന്നു. എല്ലാം ഒരു നിശ്ചിത അവധിവരെ ചരിച്ചുകൊണ്ടിരിക്കുന്നു. തീര്‍ച്ചയായും നിങ്ങള്‍ പ്രവര്‍ത്തിക്കുന്നതിനെപ്പറ്റിയൊക്കെ സൂക്ഷ്മമായി അറിയുന്നവനാണ് അല്ലാഹു. ഇതൊന്നും നിങ്ങള്‍ കണ്ടറിയുന്നില്ലേ?
\end{malayalam}}
\flushright{\begin{Arabic}
\quranayah[31][30]
\end{Arabic}}
\flushleft{\begin{malayalam}
അതിനൊക്കെ കാരണമിതാണ്. നിശ്ചയമായും അല്ലാഹു മാത്രമാണ് പരമമായ സത്യം. അവനെക്കൂടാതെ അവര്‍ വിളിച്ചുപ്രാര്‍ഥിക്കുന്നതെല്ലാം മിഥ്യയാണ്. അല്ലാഹുതന്നെയാണ് ഉന്നതനും വലിയവനും.
\end{malayalam}}
\flushright{\begin{Arabic}
\quranayah[31][31]
\end{Arabic}}
\flushleft{\begin{malayalam}
നീ കാണുന്നില്ലേ; കടലില്‍ കപ്പല്‍ സഞ്ചരിക്കുന്നത് അല്ലാഹുവിന്റെ അനുഗ്രഹത്താലാണെന്ന്. അവന്റെ ദൃഷ്ടാന്തങ്ങളില്‍ ചിലത് നിങ്ങളെ കാണിക്കാനാണിത്. നന്നായി ക്ഷമിക്കുകയും നന്ദി കാണിക്കുകയും ചെയ്യുന്ന ഏവര്‍ക്കും ഇതില്‍ ധാരാളം തെളിവുകളുണ്ട്.
\end{malayalam}}
\flushright{\begin{Arabic}
\quranayah[31][32]
\end{Arabic}}
\flushleft{\begin{malayalam}
മലകള്‍ പോലുള്ള തിരമാല അവരെ മൂടിയാല്‍ തങ്ങളുടെ വിധേയത്വം തീര്‍ത്തും അല്ലാഹുവിനു മാത്രം സമര്‍പ്പിച്ച് അവനോട് അവര്‍ പ്രാര്‍ഥിക്കുന്നു. എന്നാല്‍ അവരെയവന്‍ കരയിലേക്ക് രക്ഷപ്പെടുത്തിയാലോ, അവരില്‍ ചിലര്‍ മര്യാദ പുലര്‍ത്തുന്നവരായിരിക്കും. കൊടുംചതിയന്മാരും നന്ദികെട്ടവരുമല്ലാതെ നമ്മുടെ തെളിവുകളെ തള്ളിപ്പറയുകയില്ല.
\end{malayalam}}
\flushright{\begin{Arabic}
\quranayah[31][33]
\end{Arabic}}
\flushleft{\begin{malayalam}
മനുഷ്യരേ, നിങ്ങള്‍ നിങ്ങളുടെ നാഥനോട് ഭക്തിയുള്ളവരാവുക. ഒരു പിതാവിനും തന്റെ മകന് ഒരുപകാരവും ചെയ്യാനാവാത്ത, ഒരു മകന്നും തന്റെ പിതാവിന് ഒട്ടും പ്രയോജനപ്പെടാത്ത ഒരു നാളിനെ നിങ്ങള്‍ ഭയപ്പെടുക. നിശ്ചയമായും അല്ലാഹുവിന്റെ വാഗ്ദാനം സത്യമാണ്. അതിനാല്‍ ഐഹികജീവിതം നിങ്ങളെ വഞ്ചിക്കാതിരിക്കട്ടെ. കൊടും ചതിയനായ പിശാചും അല്ലാഹുവിന്റെ കാര്യത്തില്‍ നിങ്ങളെ വഞ്ചിക്കാതിരിക്കട്ടെ.
\end{malayalam}}
\flushright{\begin{Arabic}
\quranayah[31][34]
\end{Arabic}}
\flushleft{\begin{malayalam}
ആ അന്ത്യസമയം സംബന്ധിച്ച അറിവ് അല്ലാഹുവിങ്കല്‍ മാത്രമാണുള്ളത്. അവന്‍ മഴ വീഴ്ത്തുന്നു. ഗര്‍ഭാശയങ്ങളിലുള്ളതെന്തെന്ന് അറിയുന്നു. നാളെ താന്‍ എന്തു നേടുമെന്ന് ആര്‍ക്കും അറിയില്ല. ഏതു നാട്ടില്‍ വെച്ചാണ് മരിക്കുകയെന്നും അറിയില്ല. അല്ലാഹു എല്ലാം അറിയുന്നവനാണ്. സൂക്ഷ്മജ്ഞനും.
\end{malayalam}}
\chapter{\textmalayalam{സജദ ( സാഷ്ടാംഗം )}}
\begin{Arabic}
\Huge{\centerline{\basmalah}}\end{Arabic}
\flushright{\begin{Arabic}
\quranayah[32][1]
\end{Arabic}}
\flushleft{\begin{malayalam}
അലിഫ്-ലാം-മീം.
\end{malayalam}}
\flushright{\begin{Arabic}
\quranayah[32][2]
\end{Arabic}}
\flushleft{\begin{malayalam}
ഈ വേദപുസ്തകത്തിന്റെ അവതരണം പ്രപഞ്ചനാഥനില്‍ നിന്നാണ്. ഇതിലൊട്ടും സംശയമില്ല.
\end{malayalam}}
\flushright{\begin{Arabic}
\quranayah[32][3]
\end{Arabic}}
\flushleft{\begin{malayalam}
അതല്ല; ഇത് അദ്ദേഹം കെട്ടിച്ചമച്ചുവെന്നാണോ അവര്‍ പറയുന്നത്? എന്നാല്‍; ഇതു നിന്റെ നാഥനില്‍ നിന്നുള്ള സത്യമാണ്. നിനക്കു മുമ്പ് ഒരു മുന്നറിയിപ്പുകാരനും വന്നിട്ടില്ലാത്ത ജനതക്ക് മുന്നറിയിപ്പ് നല്‍കാനാണിത്. അവര്‍ നേര്‍വഴിയിലായേക്കാമല്ലോ.
\end{malayalam}}
\flushright{\begin{Arabic}
\quranayah[32][4]
\end{Arabic}}
\flushleft{\begin{malayalam}
ആറു നാളുകളിലായി ആകാശഭൂമികളെയും അവയ്ക്കിടയിലുള്ളവയെയും സൃഷ്ടിച്ചവനാണ് അല്ലാഹു. പിന്നെയവന്‍ സിംഹാസനസ്ഥനായി. അവനെക്കൂടാതെ നിങ്ങള്‍ക്കൊരു രക്ഷകനോ ശിപാര്‍ശകനോ ഇല്ല. നിങ്ങള്‍ ചിന്തിച്ചു മനസ്സിലാക്കുന്നില്ലേ?
\end{malayalam}}
\flushright{\begin{Arabic}
\quranayah[32][5]
\end{Arabic}}
\flushleft{\begin{malayalam}
ആകാശം മുതല്‍ ഭൂമിവരെയുള്ള സകല സംഗതികളെയും അവന്‍ നിയന്ത്രിക്കുന്നു. പിന്നീട് ഒരുനാള്‍ ഇക്കാര്യം അവങ്കലേക്കുയര്‍ന്നുപോകുന്നു. നിങ്ങള്‍ എണ്ണുന്ന ഒരായിരം കൊല്ലത്തിന്റെ ദൈര്‍ഘ്യമുണ്ട് ആ നാളിന്.
\end{malayalam}}
\flushright{\begin{Arabic}
\quranayah[32][6]
\end{Arabic}}
\flushleft{\begin{malayalam}
ഒളിഞ്ഞതും തെളിഞ്ഞതും അറിയുന്നവനാണവന്‍. പ്രതാപിയും പരമദയാലുവുമാണ്.
\end{malayalam}}
\flushright{\begin{Arabic}
\quranayah[32][7]
\end{Arabic}}
\flushleft{\begin{malayalam}
താന്‍ സൃഷ്ടിച്ച ഏതും ഏറെ നന്നാക്കി ക്രമീകരിച്ചവനാണവന്‍. അവന്‍ മനുഷ്യസൃഷ്ടി ആരംഭിച്ചത് കളിമണ്ണില്‍നിന്നാണ്.
\end{malayalam}}
\flushright{\begin{Arabic}
\quranayah[32][8]
\end{Arabic}}
\flushleft{\begin{malayalam}
പിന്നെ അവന്റെ വംശപരമ്പരയെ നന്നെ നിസ്സാരമായ ഒരു ദ്രാവകസത്തില്‍ നിന്നുണ്ടാക്കി.
\end{malayalam}}
\flushright{\begin{Arabic}
\quranayah[32][9]
\end{Arabic}}
\flushleft{\begin{malayalam}
പിന്നീട് അവനെ വേണ്ടവിധം ശരിപ്പെടുത്തി. എന്നിട്ട് തന്റെ ആത്മാവില്‍ നിന്ന് അതിലൂതി. നിങ്ങള്‍ക്കവന്‍ കേള്‍വിയും കാഴ്ചകളും ഹൃദയങ്ങളും ഉണ്ടാക്കിത്തന്നു. എന്നിട്ടും നന്നെ കുറച്ചേ നിങ്ങള്‍ നന്ദി കാണിക്കുന്നുള്ളൂ.
\end{malayalam}}
\flushright{\begin{Arabic}
\quranayah[32][10]
\end{Arabic}}
\flushleft{\begin{malayalam}
അവര്‍ ചോദിക്കുന്നു: "ഞങ്ങള്‍ മണ്ണില്‍ ലയിച്ചില്ലാതായാല്‍ പോലും പിന്നെയും പുതുതായി സൃഷ്ടിക്കപ്പെടുമെന്നോ?" അവര്‍ തങ്ങളുടെ നാഥനെ കണ്ടുമുട്ടുമെന്നതിനെ തള്ളിപ്പറയുന്നവരാണ്.
\end{malayalam}}
\flushright{\begin{Arabic}
\quranayah[32][11]
\end{Arabic}}
\flushleft{\begin{malayalam}
പറയുക: "നിങ്ങളുടെ കാര്യം ഏല്‍പിക്കപ്പെട്ട മരണത്തിന്റെ മലക്ക് നിങ്ങളുടെ ജീവനെടുക്കും. പിന്നീട് നിങ്ങള്‍ നിങ്ങളുടെ നാഥങ്കലേക്ക് മടക്കപ്പെടും."
\end{malayalam}}
\flushright{\begin{Arabic}
\quranayah[32][12]
\end{Arabic}}
\flushleft{\begin{malayalam}
കുറ്റവാളികള്‍ തങ്ങളുടെ നാഥന്റെ അടുത്ത് തലതാഴ്ത്തി നില്‍ക്കുന്നത് നീ കണ്ടിരുന്നെങ്കില്‍! അവര്‍ പറയും: "ഞങ്ങളുടെ നാഥാ, ഞങ്ങളിതാ എല്ലാം നേരില്‍ കണ്ടിരിക്കുന്നു. കേട്ടിരിക്കുന്നു. അതിനാല്‍ നീ ഞങ്ങളെ ഒന്നു തിരിച്ചയക്കേണമേ. ഞങ്ങള്‍ നല്ലതു ചെയ്തുകൊള്ളാം. ഇപ്പോള്‍ ഞങ്ങള്‍ക്കെല്ലാം നന്നായി ബോധ്യമായിരിക്കുന്നു."
\end{malayalam}}
\flushright{\begin{Arabic}
\quranayah[32][13]
\end{Arabic}}
\flushleft{\begin{malayalam}
നാം ഇച്ഛിച്ചിരുന്നെങ്കില്‍ നേരത്തെ തന്നെ എല്ലാ ഓരോരുത്തര്‍ക്കും നേര്‍വഴി കാണിച്ചുകൊടുക്കുമായിരുന്നു. എന്നാല്‍ നമ്മില്‍ നിന്നുണ്ടായ പ്രഖ്യാപനം യാഥാര്‍ഥ്യമായിത്തീര്‍ന്നിരിക്കുന്നു. “ജിന്നുകളാലും മനുഷ്യരാലും ഞാന്‍ നരകത്തെ നിറയ്ക്കുകതന്നെ ചെയ്യു”മെന്ന പ്രഖ്യാപനം.
\end{malayalam}}
\flushright{\begin{Arabic}
\quranayah[32][14]
\end{Arabic}}
\flushleft{\begin{malayalam}
നിങ്ങളുടെ ഈ നാളുമായുള്ള കണ്ടുമുട്ടല്‍ നിങ്ങള്‍ മറന്നുകളഞ്ഞതിനാല്‍ അതിന്റെ ശിക്ഷ ആസ്വദിച്ചുകൊള്ളുക. നിശ്ചയമായും നാം നിങ്ങളെയും മറന്നിരിക്കുന്നു. നിങ്ങള്‍ പ്രവര്‍ത്തിച്ചുകൊണ്ടിരുന്നതിന്റെ ഫലമായുള്ള ശാശ്വത ശിക്ഷ അനുഭവിച്ചുകൊള്ളുക.
\end{malayalam}}
\flushright{\begin{Arabic}
\quranayah[32][15]
\end{Arabic}}
\flushleft{\begin{malayalam}
നമ്മുടെ വചനങ്ങള്‍ വഴി ഉദ്ബോധനം നല്‍കിയാല്‍ സാഷ്ടാംഗ പ്രണാമമര്‍പ്പിക്കുന്നവരും തങ്ങളുടെ നാഥനെ വാഴ്ത്തുന്നവരും കീര്‍ത്തിക്കുന്നവരുംമാത്രമാണ് നമ്മുടെ വചനങ്ങളില്‍ വിശ്വസിക്കുന്നവര്‍. അവരൊട്ടും അഹങ്കരിക്കുകയില്ല.
\end{malayalam}}
\flushright{\begin{Arabic}
\quranayah[32][16]
\end{Arabic}}
\flushleft{\begin{malayalam}
പേടിയോടും പ്രത്യാശയോടും കൂടി തങ്ങളുടെ നാഥനോട് പ്രാര്‍ഥിക്കാനായി കിടപ്പിടങ്ങളില്‍ നിന്ന് അവരുടെ പാര്‍ശ്വങ്ങള്‍ ഉയര്‍ന്ന് അകന്നുപോകും. നാം അവര്‍ക്കു നല്‍കിയതില്‍ നിന്നവര്‍ ചെലവഴിക്കുകയും ചെയ്യും.
\end{malayalam}}
\flushright{\begin{Arabic}
\quranayah[32][17]
\end{Arabic}}
\flushleft{\begin{malayalam}
ആര്‍ക്കുമറിയില്ല; തങ്ങള്‍ക്കായി കണ്‍കുളിര്‍പ്പിക്കുന്ന എന്തൊക്കെയാണ് രഹസ്യമായി സൂക്ഷിച്ചുവെച്ചിരിക്കുന്നതെന്ന്. അവര്‍ പ്രവര്‍ത്തിച്ചതിനുള്ള പ്രതിഫലമാണ് അതെല്ലാം.
\end{malayalam}}
\flushright{\begin{Arabic}
\quranayah[32][18]
\end{Arabic}}
\flushleft{\begin{malayalam}
അല്ല; സത്യവിശ്വാസിയായ ഒരാള്‍ തെമ്മാടിയെപ്പോലെയാണെന്നോ? അവര്‍ ഒരുപോലെയാവുകയില്ല.
\end{malayalam}}
\flushright{\begin{Arabic}
\quranayah[32][19]
\end{Arabic}}
\flushleft{\begin{malayalam}
സത്യവിശ്വാസം സ്വീകരിക്കുകയും സല്‍ക്കര്‍മങ്ങള്‍ പ്രവര്‍ത്തിക്കുകയും ചെയ്തവര്‍ക്ക് പാര്‍ക്കാന്‍ സ്വര്‍ഗത്തോപ്പുകളുണ്ട്. അവരുടെ പ്രവര്‍ത്തനങ്ങളുടെ ഫലമായി വന്നെത്തിയ ആതിഥ്യമാണത്.
\end{malayalam}}
\flushright{\begin{Arabic}
\quranayah[32][20]
\end{Arabic}}
\flushleft{\begin{malayalam}
എന്നാല്‍ തെമ്മാടിത്തം കാണിച്ചവരുടെ താവളം നരകത്തീയാണ്. അവരതില്‍നിന്ന് പുറത്തുകടക്കാനാഗ്രഹിക്കുമ്പോഴെല്ലാം അവരെ അതിലേക്കുതന്നെ തിരിച്ചയക്കും. അവരോടിങ്ങനെ പറയും: "നിങ്ങള്‍ തള്ളിപ്പറഞ്ഞുകൊണ്ടിരുന്ന ആ നരകശിക്ഷ ആസ്വദിച്ചുകൊള്ളുക."
\end{malayalam}}
\flushright{\begin{Arabic}
\quranayah[32][21]
\end{Arabic}}
\flushleft{\begin{malayalam}
ഏറ്റവും വലിയ ആ ശിക്ഷ കൂടാതെ ഇഹലോകത്ത് ചില ചെറിയ ശിക്ഷകള്‍ നാമവരെ അനുഭവിപ്പിക്കും. ഒരുവേള അവര്‍ സത്യത്തിലേക്കു തിരിച്ചുവന്നെങ്കിലോ.
\end{malayalam}}
\flushright{\begin{Arabic}
\quranayah[32][22]
\end{Arabic}}
\flushleft{\begin{malayalam}
തന്റെ നാഥന്റെ വചനങ്ങളിലൂടെ ഉദ്ബോധനം ലഭിച്ചശേഷം അവയെ അവഗണിച്ചവനെക്കാള്‍ കടുത്ത അക്രമി ആരുണ്ട്? അത്തരം കുറ്റവാളികളോടു നാം പ്രതികാരം ചെയ്യും; തീര്‍ച്ച.
\end{malayalam}}
\flushright{\begin{Arabic}
\quranayah[32][23]
\end{Arabic}}
\flushleft{\begin{malayalam}
സംശയമില്ല; മൂസാക്കു നാം വേദം നല്‍കിയിട്ടുണ്ട്. അതിനാല്‍ ഇത്തരമൊന്ന് ലഭിക്കുന്നതില്‍ നീ ഒട്ടും സംശയിക്കേണ്ടതില്ല. ഇസ്രയേല്‍ മക്കള്‍ക്ക് നാമതിനെ വഴികാട്ടിയാക്കുകയും ചെയ്തു.
\end{malayalam}}
\flushright{\begin{Arabic}
\quranayah[32][24]
\end{Arabic}}
\flushleft{\begin{malayalam}
അവര്‍ ക്ഷമപാലിക്കുകയും നമ്മുടെ വചനങ്ങളില്‍ അടിയുറച്ചു വിശ്വസിക്കുകയും ചെയ്തപ്പോള്‍ അവരില്‍ നിന്നു നമ്മുടെ കല്‍പനയനുസരിച്ച് നേര്‍വഴി കാണിക്കുന്ന നേതാക്കന്മാരെ നാം ഉണ്ടാക്കി.
\end{malayalam}}
\flushright{\begin{Arabic}
\quranayah[32][25]
\end{Arabic}}
\flushleft{\begin{malayalam}
അവര്‍ പരസ്പരം ഭിന്നിച്ചുകൊണ്ടിരുന്ന കാര്യങ്ങളില്‍, ഉയിര്‍ത്തെഴുന്നേല്‍പുനാളില്‍ നിന്റെ നാഥന്‍ തീര്‍പ്പുകല്‍പിക്കും.
\end{malayalam}}
\flushright{\begin{Arabic}
\quranayah[32][26]
\end{Arabic}}
\flushleft{\begin{malayalam}
ഇവര്‍ക്കു മുമ്പ് എത്രയോ തലമുറകളെ നാം തകര്‍ത്തുകളഞ്ഞിട്ടുണ്ട്. അവരുടെ പാര്‍പ്പിടങ്ങളിലൂടെയാണ് ഇവര്‍ നടന്നുപോയിക്കൊണ്ടിരിക്കുന്നത്. എന്നിട്ടും ഇവര്‍ക്കത് നേര്‍വഴി കാണിക്കുന്ന ഗുണപാഠമാകുന്നില്ലേ? തീര്‍ച്ചയായും അതില്‍ ധാരാളം ദൃഷ്ടാന്തങ്ങളുണ്ട്. എന്നിട്ടും ഇവര്‍ കേട്ടറിയുന്നില്ലേ?
\end{malayalam}}
\flushright{\begin{Arabic}
\quranayah[32][27]
\end{Arabic}}
\flushleft{\begin{malayalam}
ഇവര്‍ കാണുന്നില്ലേ; വരണ്ട ഭൂമിയിലേക്കു നാം വെള്ളമെത്തിക്കുന്നു; അതുവഴി വിളവുല്‍പാദിപ്പിക്കുന്നു; അതില്‍നിന്ന് ഇവരുടെ കാലികള്‍ക്ക് തീറ്റ ലഭിക്കുന്നു. ഇവരും ആഹരിക്കുന്നു. എന്നിട്ടും ഇക്കൂട്ടര്‍ കണ്ടറിയുന്നില്ലേ?
\end{malayalam}}
\flushright{\begin{Arabic}
\quranayah[32][28]
\end{Arabic}}
\flushleft{\begin{malayalam}
ഇവര്‍ ചോദിക്കുന്നു: "ആ തീരുമാനം എപ്പോഴാണുണ്ടാവുക. നിങ്ങള്‍ സത്യവാന്മാരെങ്കില്‍?"
\end{malayalam}}
\flushright{\begin{Arabic}
\quranayah[32][29]
\end{Arabic}}
\flushleft{\begin{malayalam}
പറയുക: ആ തീരുമാനം നടപ്പില്‍വരുംനാള്‍, നിശ്ചയമായും സത്യനിഷേധികള്‍ക്ക് വിശ്വാസം കൊണ്ട് ഒരു പ്രയോജനവുമുണ്ടാവുകയില്ല. അവര്‍ക്ക് ഇനിയൊരവധി നീട്ടിക്കൊടുക്കുകയുമില്ല.
\end{malayalam}}
\flushright{\begin{Arabic}
\quranayah[32][30]
\end{Arabic}}
\flushleft{\begin{malayalam}
അതിനാല്‍ അവരെ നീ അവഗണിക്കുക. അവരുടെ പര്യവസാനത്തിനായി കാത്തിരിക്കുക. തീര്‍ച്ചയായും അവരും കാത്തിരിക്കുന്നവരാണ്.
\end{malayalam}}
\chapter{\textmalayalam{അഹ്സാബ് (സംഘടിത കക്ഷികള്‍ )}}
\begin{Arabic}
\Huge{\centerline{\basmalah}}\end{Arabic}
\flushright{\begin{Arabic}
\quranayah[33][1]
\end{Arabic}}
\flushleft{\begin{malayalam}
നബിയേ, ദൈവഭക്തനാവുക. സത്യനിഷേധികള്‍ക്കും കപടവിശ്വാസികള്‍ക്കും വഴിപ്പെടാതിരിക്കുക. അല്ലാഹു എല്ലാം അറിയുന്നവനാണ്. യുക്തിമാനും.
\end{malayalam}}
\flushright{\begin{Arabic}
\quranayah[33][2]
\end{Arabic}}
\flushleft{\begin{malayalam}
നിനക്ക് നിന്റെ നാഥനില്‍ നിന്ന് ബോധനമായി കിട്ടുന്ന സന്ദേശം പിന്‍പറ്റുക. നിങ്ങള്‍ പ്രവര്‍ത്തിക്കുന്നതിനെപ്പറ്റിയൊക്കെ നന്നായറിയുന്നവനാണ് അല്ലാഹു.
\end{malayalam}}
\flushright{\begin{Arabic}
\quranayah[33][3]
\end{Arabic}}
\flushleft{\begin{malayalam}
അല്ലാഹുവില്‍ ഭരമേല്‍പിക്കുക. കൈകാര്യ കര്‍ത്താവായി അല്ലാഹു തന്നെ മതി.
\end{malayalam}}
\flushright{\begin{Arabic}
\quranayah[33][4]
\end{Arabic}}
\flushleft{\begin{malayalam}
അല്ലാഹു ഒരു മനുഷ്യന്റെയും ഉള്ളില്‍ രണ്ട് ഹൃദയങ്ങളുണ്ടാക്കിയിട്ടില്ല. നിങ്ങള്‍ “ളിഹാര്‍” ചെയ്യുന്ന ഭാര്യമാരെ നിങ്ങളുടെ മാതാക്കളാക്കിയിട്ടില്ല. നിങ്ങളിലേക്കുചേര്‍ത്തുവിളിക്കുന്നവരെ നിങ്ങളുടെ പുത്രന്മാരാക്കിയിട്ടുമില്ല. അതൊക്കെ നിങ്ങളുടെ വായകൊണ്ടുള്ള വെറും വാക്കുകളാണ്. അല്ലാഹു സത്യം പറയുന്നു. അവന്‍ നേര്‍വഴിയില്‍ നയിക്കുകയും ചെയ്യുന്നു.
\end{malayalam}}
\flushright{\begin{Arabic}
\quranayah[33][5]
\end{Arabic}}
\flushleft{\begin{malayalam}
നിങ്ങള്‍ ദത്തുപുത്രന്മാരെ അവരുടെ പിതാക്കളിലേക്കു ചേര്‍ത്തുവിളിക്കുക. അതാണ് അല്ലാഹുവിന്റെയടുത്ത് ഏറെ നീതിപൂര്‍വകം. അഥവാ, അവരുടെ പിതാക്കളാരെന്ന് നിങ്ങള്‍ക്കറിയില്ലെങ്കില്‍ അവര്‍ നിങ്ങളുടെ ആദര്‍ശസഹോദരങ്ങളും മിത്രങ്ങളുമാകുന്നു. അബദ്ധത്തില്‍ നിങ്ങള്‍ പറഞ്ഞുപോയതിന്റെ പേരില്‍ നിങ്ങള്‍ക്കു കുറ്റമില്ല. എന്നാല്‍, നിങ്ങള്‍ മനഃപൂര്‍വം ചെയ്യുന്നത് കുറ്റം തന്നെ. അല്ലാഹു ഏറെ പൊറുക്കുന്നവനും പരമകാരുണികനുമാണ്.
\end{malayalam}}
\flushright{\begin{Arabic}
\quranayah[33][6]
\end{Arabic}}
\flushleft{\begin{malayalam}
പ്രവാചകന്‍ സത്യവിശ്വാസികള്‍ക്ക് സ്വന്തത്തെക്കാള്‍ ഉറ്റവനാണ്. അദ്ദേഹത്തിന്റെ പത്നിമാര്‍ അവരുടെ മാതാക്കളുമാണ്. അല്ലാഹുവിന്റെ ഗ്രന്ഥമനുസരിച്ച് രക്തബന്ധുക്കള്‍ പരസ്പരം മറ്റു വിശ്വാസികളെക്കാളും മുഹാജിറുകളെ ക്കാളും കൂടുതല്‍ അടുപ്പമുള്ളവരാണ്. എന്നാല്‍ നിങ്ങള്‍ സ്വന്തം ആത്മമിത്രങ്ങളോട് വല്ല നന്മയും ചെയ്യുന്നതിന് ഇതു തടസ്സമല്ല. ഈ വിധി വേദപുസ്തകത്തില്‍ രേഖപ്പെടുത്തിയതാണ്.
\end{malayalam}}
\flushright{\begin{Arabic}
\quranayah[33][7]
\end{Arabic}}
\flushleft{\begin{malayalam}
പ്രവാചകന്മാരില്‍ നിന്നു നാം വാങ്ങിയ കരാറിനെക്കുറിച്ചോര്‍ക്കുക. നിന്നില്‍ നിന്നും നൂഹ്, ഇബ്റാഹീം, മൂസാ, മര്‍യമിന്റെ മകന്‍ ഈസാ എന്നിവരില്‍ നിന്നും. അവരില്‍ നിന്നെല്ലാം നാം പ്രബലമായ കരാര്‍ വാങ്ങിയിട്ടുണ്ട്.
\end{malayalam}}
\flushright{\begin{Arabic}
\quranayah[33][8]
\end{Arabic}}
\flushleft{\begin{malayalam}
സത്യവാദികളോട് അവരുടെ സത്യതയെ സംബന്ധിച്ച് ചോദിക്കാനാണിത്. സത്യനിഷേധികള്‍ക്ക് നോവേറിയ ശിക്ഷ ഒരുക്കിവെച്ചിട്ടുണ്ട്.
\end{malayalam}}
\flushright{\begin{Arabic}
\quranayah[33][9]
\end{Arabic}}
\flushleft{\begin{malayalam}
വിശ്വസിച്ചവരേ; അല്ലാഹു നിങ്ങള്‍ക്കേകിയ അനുഗ്രഹം ഓര്‍ത്തുനോക്കൂ: നിങ്ങള്‍ക്കു നേരെ കുറേ പടയാളികള്‍ പാഞ്ഞടുത്തു. അപ്പോള്‍ അവര്‍ക്കെതിരെ നാം കൊടുങ്കാറ്റയച്ചു. നിങ്ങള്‍ക്കു കാണാനാവാത്ത സൈന്യത്തെയുമയച്ചു. നിങ്ങള്‍ ചെയ്യുന്നതൊക്കെയും കണ്ടറിയുന്നവനാണ് അല്ലാഹു.
\end{malayalam}}
\flushright{\begin{Arabic}
\quranayah[33][10]
\end{Arabic}}
\flushleft{\begin{malayalam}
ശത്രുസൈന്യം മുകള്‍ഭാഗത്തുനിന്നും താഴ്ഭാഗത്തുനിന്നും നിങ്ങളുടെ നേരെ വന്നടുത്ത സന്ദര്‍ഭം! ഭയം കാരണം ദൃഷ്ടികള്‍ പതറുകയും ഹൃദയങ്ങള്‍ തൊണ്ടകളിലെത്തുകയും നിങ്ങള്‍ അല്ലാഹുവെപ്പറ്റി പലതും കരുതിപ്പോവുകയും ചെയ്ത സന്ദര്‍ഭം.
\end{malayalam}}
\flushright{\begin{Arabic}
\quranayah[33][11]
\end{Arabic}}
\flushleft{\begin{malayalam}
അപ്പോള്‍ അവിടെവെച്ച് സത്യവിശ്വാസികള്‍ പരീക്ഷിക്കപ്പെട്ടു. കഠിനമായി വിറപ്പിക്കപ്പെടുകയും ചെയ്തു.
\end{malayalam}}
\flushright{\begin{Arabic}
\quranayah[33][12]
\end{Arabic}}
\flushleft{\begin{malayalam}
"അല്ലാഹുവും അവന്റെ ദൂതനും നമ്മോടു ചെയ്ത വാഗ്ദാനം വെറും വഞ്ചന മാത്രമാണെ"ന്ന് കപടവിശ്വാസികളും മനസ്സിന് ദീനം ബാധിച്ചവരും പറഞ്ഞുകൊണ്ടിരുന്നു.
\end{malayalam}}
\flushright{\begin{Arabic}
\quranayah[33][13]
\end{Arabic}}
\flushleft{\begin{malayalam}
അവരിലൊരു വിഭാഗം പറഞ്ഞതോര്‍ക്കുക: "യഥ്രിബുകാരേ, നിങ്ങള്‍ക്കിനി ഇവിടെ നില്‍ക്കാനാവില്ല. അതിനാല്‍ മടങ്ങിപ്പൊയിക്കോളൂ." മറ്റൊരു വിഭാഗം “ഞങ്ങളുടെ വീടുകള്‍ അപകടാവസ്ഥയിലാണെ”ന്ന് പറഞ്ഞ് പ്രവാചകനോടു യുദ്ധരംഗം വിടാന്‍ അനുവാദം തേടുകയായിരുന്നു. യഥാര്‍ഥത്തിലവയ്ക്ക് ഒരപകടാവസ്ഥയുമില്ല. അവര്‍ രംഗം വിട്ടോടാന്‍ വഴികളാരായുകയായിരുന്നുവെന്നുമാത്രം.
\end{malayalam}}
\flushright{\begin{Arabic}
\quranayah[33][14]
\end{Arabic}}
\flushleft{\begin{malayalam}
മദീനയുടെ നാനാഭാഗങ്ങളിലൂടെ ശത്രുക്കള്‍ കടന്നുചെല്ലുകയും അങ്ങനെ കലാപമുണ്ടാക്കാന്‍ അവരോട് ആവശ്യപ്പെടുകയും ചെയ്താല്‍ അവരത് നടപ്പാക്കുമായിരുന്നു. അവരക്കാര്യത്തില്‍ താമസം വരുത്തുകയുമില്ല; നന്നെ കുറച്ചല്ലാതെ.
\end{malayalam}}
\flushright{\begin{Arabic}
\quranayah[33][15]
\end{Arabic}}
\flushleft{\begin{malayalam}
തങ്ങള്‍ പിന്തിരിഞ്ഞോടുകയില്ലെന്ന് അവര്‍ നേരത്തെ അല്ലാഹുവോട് കരാര്‍ ചെയ്തിട്ടുണ്ടായിരുന്നു. അല്ലാഹുവോടുള്ള കരാറിനെക്കുറിച്ച് അവരോട് ചോദിക്കുകതന്നെചെയ്യും.
\end{malayalam}}
\flushright{\begin{Arabic}
\quranayah[33][16]
\end{Arabic}}
\flushleft{\begin{malayalam}
പറയുക: "നിങ്ങള്‍ മരണത്തെയോ കൊലയെയോ പേടിച്ചോടുകയാണെങ്കില്‍ ആ ഓട്ടം നിങ്ങള്‍ക്കൊട്ടും ഉപകരിക്കുകയില്ല. പിന്നെ ജീവിതമാസ്വദിക്കാന്‍ ഇത്തിരികാലമല്ലാതെ നിങ്ങള്‍ക്ക് കിട്ടുകയില്ല."
\end{malayalam}}
\flushright{\begin{Arabic}
\quranayah[33][17]
\end{Arabic}}
\flushleft{\begin{malayalam}
ചോദിക്കുക: "അല്ലാഹു നിങ്ങള്‍ക്കു വല്ല ദോഷവും വരുത്താനുദ്ദേശിച്ചാല്‍ അല്ലാഹുവില്‍ നിന്ന് നിങ്ങളെ രക്ഷിക്കാനാരുണ്ട്? അല്ലെങ്കില്‍ നിങ്ങള്‍ക്ക് വല്ല കാരുണ്യവുമുദ്ദേശിച്ചാല്‍ അത് തടയാനാരുണ്ട്?" അല്ലാഹുവെക്കൂടാതെ ഒരു രക്ഷകനെയും സഹായിയെയും അവര്‍ക്ക് കണ്ടെത്താനാവില്ല.
\end{malayalam}}
\flushright{\begin{Arabic}
\quranayah[33][18]
\end{Arabic}}
\flushleft{\begin{malayalam}
നിങ്ങളുടെ കൂട്ടത്തില്‍ തടസ്സം സൃഷ്ടിക്കുന്നതാരെന്ന് അല്ലാഹുവിനു നന്നായറിയാം. തങ്ങളുടെ സഹോദരന്മാരോട് “ഞങ്ങളോടൊപ്പം വരൂ” എന്നു പറയുന്നവരെയും. അപൂര്‍വമായല്ലാതെ അവര്‍ യുദ്ധത്തിന് പോവുകയില്ല.
\end{malayalam}}
\flushright{\begin{Arabic}
\quranayah[33][19]
\end{Arabic}}
\flushleft{\begin{malayalam}
നിങ്ങളോടൊപ്പം വരുന്നതില്‍ പിശുക്കു കാണിക്കുന്നവരാണവര്‍. ഭയാവസ്ഥ വന്നാല്‍ അവര്‍ നിന്നെ തുറിച്ചുനോക്കുന്നതു നിനക്കു കാണാം. ആസന്ന മരണനായവന്‍ ബോധം കെടുമ്പോഴെന്നപോലെ അവരുടെ കണ്ണുകള്‍ കറങ്ങിക്കൊണ്ടിരിക്കും. എന്നാല്‍ ഭയം വിട്ടകന്നാല്‍ സമ്പത്തില്‍ ആര്‍ത്തിപൂണ്ട് മൂര്‍ച്ചയേറിയ നാവുപയോഗിച്ച് അവര്‍ നിങ്ങളെ നേരിടുന്നു. യഥാര്‍ഥത്തിലവര്‍ സത്യവിശ്വാസം സ്വീകരിച്ചിട്ടില്ല. അതിനാല്‍ അല്ലാഹു അവരുടെ പ്രവര്‍ത്തനങ്ങള്‍ പാഴാക്കിയിരിക്കുന്നു. അല്ലാഹുവെ സംബന്ധിച്ചിടത്തോളം ഇതൊക്കെ നന്നെ നിസ്സാരമാണ്.
\end{malayalam}}
\flushright{\begin{Arabic}
\quranayah[33][20]
\end{Arabic}}
\flushleft{\begin{malayalam}
സഖ്യസേന ഇനിയും സ്ഥലം വിട്ടിട്ടില്ലെന്നാണവര്‍ കരുതുന്നത്. സഖ്യസേന ഇനിയും വരികയാണെങ്കില്‍ നിങ്ങളുടെ വിവരങ്ങള്‍ ചോദിച്ചറിഞ്ഞുകൊണ്ട് ഗ്രാമീണ അറബികളോടൊപ്പം മരുഭൂവാസികളായിക്കഴിയാനാണ് അവരിഷ്ടപ്പെടുക. അവര്‍ നിങ്ങളോടൊപ്പമുണ്ടായാലും വളരെ കുറച്ചേ യുദ്ധത്തില്‍ പങ്കാളികളാവുകയുള്ളൂ.
\end{malayalam}}
\flushright{\begin{Arabic}
\quranayah[33][21]
\end{Arabic}}
\flushleft{\begin{malayalam}
സംശയമില്ല; നിങ്ങള്‍ക്ക് അല്ലാഹുവിന്റെ ദൂതനില്‍ മികച്ച മാതൃകയുണ്ട്. അല്ലാഹുവിലും അന്ത്യദിനത്തിലും പ്രതീക്ഷയര്‍പ്പിച്ചവര്‍ക്കാണിത്. അല്ലാഹുവെ ധാരാളമായി ഓര്‍ക്കുന്നവര്‍ക്കും.
\end{malayalam}}
\flushright{\begin{Arabic}
\quranayah[33][22]
\end{Arabic}}
\flushleft{\begin{malayalam}
സത്യവിശ്വാസികള്‍ സഖ്യസേനയെ കണ്ടുമുട്ടിയപ്പോള്‍ പറഞ്ഞു: "ഇത് അല്ലാഹുവും അവന്റെ ദൂതനും ഞങ്ങളോട് വാഗ്ദാനം ചെയ്തതു തന്നെയാണ്. അല്ലാഹുവും അവന്റെ ദൂതനും പറഞ്ഞത് തീര്‍ത്തും സത്യമാണ്." ആ സംഭവം അവരുടെ വിശ്വാസവും സമര്‍പ്പണ സന്നദ്ധതയും വര്‍ധിപ്പിക്കുകയാണുണ്ടായത്.
\end{malayalam}}
\flushright{\begin{Arabic}
\quranayah[33][23]
\end{Arabic}}
\flushleft{\begin{malayalam}
സത്യവിശ്വാസികളില്‍ അല്ലാഹുവുമായി ചെയ്ത കരാറിന്റെ കാര്യത്തില്‍ സത്യസന്ധത പുലര്‍ത്തുന്ന ചിലരുണ്ട്. അങ്ങനെ തങ്ങളുടെ പ്രതിജ്ഞ പൂര്‍ത്തീകരിച്ചവര്‍ അവരിലുണ്ട്. അതിനായി അവസരം പാര്‍ത്തിരിക്കുന്നവരുമുണ്ട്. ആ കരാറിലൊരു മാറ്റവും അവര്‍ വരുത്തിയിട്ടില്ല.
\end{malayalam}}
\flushright{\begin{Arabic}
\quranayah[33][24]
\end{Arabic}}
\flushleft{\begin{malayalam}
സത്യസന്ധര്‍ക്ക് തങ്ങളുടെ സത്യതക്കുള്ള പ്രതിഫലം നല്‍കാനാണിത്. അല്ലാഹു ഇച്ഛിക്കുന്നുവെങ്കില്‍ കപടവിശ്വാസികളെ ശിക്ഷിക്കാനും. അല്ലെങ്കില്‍ അവരുടെ പശ്ചാത്താപം സ്വീകരിക്കാനും. തീര്‍ച്ചയായും അല്ലാഹു ഏറെ പൊറുക്കുന്നവനും പരമദയാലുവുമാണ്.
\end{malayalam}}
\flushright{\begin{Arabic}
\quranayah[33][25]
\end{Arabic}}
\flushleft{\begin{malayalam}
സത്യനിഷേധികളെ അവരുടെ കോപാഗ്നിയോടെത്തന്നെ യുദ്ധരംഗത്തുനിന്ന് അല്ലാഹു തിരിച്ചയച്ചു. അവര്‍ക്കൊട്ടും നേട്ടം കിട്ടിയില്ല. സത്യവിശ്വാസികള്‍ക്ക് വേണ്ടി പൊരുതാന്‍ അല്ലാഹു തന്നെ മതി. അല്ലാഹു ഏറെ കരുത്തനും പ്രതാപിയുമാണ്.
\end{malayalam}}
\flushright{\begin{Arabic}
\quranayah[33][26]
\end{Arabic}}
\flushleft{\begin{malayalam}
വേദക്കാരില്‍ ചിലര്‍ ശത്രുസൈന്യത്തെ സഹായിച്ചു. അല്ലാഹു അവരെ അവരുടെ കോട്ടകളില്‍ നിന്ന് ഇറക്കിവിട്ടു. അവരുടെ ഹൃദയങ്ങളില്‍ ഭയം കോരിയിടുകയും ചെയ്തു. അവരില്‍ ചിലരെ നിങ്ങള്‍ കൊന്നൊടുക്കുന്നു. മറ്റു ചിലരെ തടവിലാക്കുകയും ചെയ്യുന്നു.
\end{malayalam}}
\flushright{\begin{Arabic}
\quranayah[33][27]
\end{Arabic}}
\flushleft{\begin{malayalam}
അവന്‍ നിങ്ങളെ അവരുടെ ഭൂമിയുടെയും വീടുകളുടെയും സ്വത്തുക്കളുടെയും അവകാശികളാക്കി. നിങ്ങള്‍ മുമ്പൊരിക്കലും കാലുകുത്തിയിട്ടില്ലാത്ത സ്ഥലംപോലും അവന്‍ നിങ്ങള്‍ക്കു നല്‍കി. അല്ലാഹു എല്ലാ കാര്യത്തിനും കഴിവുറ്റവനാണ്.
\end{malayalam}}
\flushright{\begin{Arabic}
\quranayah[33][28]
\end{Arabic}}
\flushleft{\begin{malayalam}
നബിയേ, നീ നിന്റെ ഭാര്യമാരോടു പറയുക: "ഇഹലോക ജീവിതവും അതിലെ അലങ്കാരവുമാണ് നിങ്ങളാഗ്രഹിക്കുന്നതെങ്കില്‍ വരൂ! ഞാന്‍ നിങ്ങള്‍ക്കു ജീവിതവിഭവം നല്‍കാം. നല്ല നിലയില്‍ നിങ്ങളെ പിരിച്ചയക്കുകയും ചെയ്യാം.
\end{malayalam}}
\flushright{\begin{Arabic}
\quranayah[33][29]
\end{Arabic}}
\flushleft{\begin{malayalam}
"അല്ലാഹുവെയും അവന്റെ ദൂതനെയും പരലോകഭവനത്തെയുമാണ് നിങ്ങളാഗ്രഹിക്കുന്നതെങ്കില്‍ അറിയുക: നിങ്ങളിലെ സച്ചരിതകള്‍ക്ക് അല്ലാഹു അതിമഹത്തായ പ്രതിഫലം ഒരുക്കിവെച്ചിട്ടുണ്ട്."
\end{malayalam}}
\flushright{\begin{Arabic}
\quranayah[33][30]
\end{Arabic}}
\flushleft{\begin{malayalam}
പ്രവാചക പത്നിമാരേ, നിങ്ങളിലാരെങ്കിലും വ്യക്തമായ നീചവൃത്തിയിലേര്‍പ്പെടുകയാണെങ്കില്‍ അവള്‍ക്ക് രണ്ടിരട്ടി ശിക്ഷയുണ്ട്. അല്ലാഹുവിന് അത് വളരെ എളുപ്പമാണ്.
\end{malayalam}}
\flushright{\begin{Arabic}
\quranayah[33][31]
\end{Arabic}}
\flushleft{\begin{malayalam}
നിങ്ങളിലാരെങ്കിലും അല്ലാഹുവോടും അവന്റെ ദൂതനോടും വിനയം കാണിക്കുകയും സല്‍ക്കര്‍മം പ്രവര്‍ത്തിക്കുകയുമാണെങ്കില്‍ അവള്‍ക്ക് നാം രണ്ടിരട്ടി പ്രതിഫലം നല്‍കും. അവള്‍ക്കു നാം മാന്യമായ ജീവിതവിഭവം ഒരുക്കിവെച്ചിട്ടുമുണ്ട്.
\end{malayalam}}
\flushright{\begin{Arabic}
\quranayah[33][32]
\end{Arabic}}
\flushleft{\begin{malayalam}
പ്രവാചക പത്നിമാരേ, നിങ്ങള്‍ മറ്റു സ്ത്രീകളെപ്പോലെയല്ല. അതിനാല്‍ നിങ്ങള്‍ ദൈവഭക്തകളാണെങ്കില്‍ കൊഞ്ചിക്കുഴഞ്ഞ് സംസാരിക്കരുത്. അത് ദീനം പിടിച്ച മനസ്സുള്ളവരില്‍ മോഹമുണര്‍ത്തിയേക്കും. നിങ്ങള്‍ മാന്യമായി മാത്രം സംസാരിക്കുക.
\end{malayalam}}
\flushright{\begin{Arabic}
\quranayah[33][33]
\end{Arabic}}
\flushleft{\begin{malayalam}
നിങ്ങള്‍ നിങ്ങളുടെ വീടുകളില്‍ അടങ്ങിയൊതുങ്ങിക്കഴിയുക. പഴയ അനിസ്ലാമിക കാലത്തെപ്പോലെ സൌന്ദര്യം വെളിവാക്കി വിലസി നടക്കാതിരിക്കുക. നമസ്കാരം നിഷ്ഠയോടെ നിര്‍വഹിക്കുക, സകാത്ത് നല്‍കുക, അല്ലാഹുവെയും അവന്റെ ദൂതനേയും അനുസരിക്കുക. നബികുടുംബമേ, നിങ്ങളില്‍ നിന്നു മാലിന്യം നീക്കിക്കളയാനും നിങ്ങളെ പൂര്‍ണമായും ശുദ്ധീകരിക്കാനുമാണ് അല്ലാഹു ഉദ്ദേശിക്കുന്നത്.
\end{malayalam}}
\flushright{\begin{Arabic}
\quranayah[33][34]
\end{Arabic}}
\flushleft{\begin{malayalam}
നിങ്ങളുടെ വീടുകളില്‍ വെച്ച് ഓതിക്കേള്‍പിക്കുന്ന അല്ലാഹുവിന്റെ വചനങ്ങളും തത്ത്വജ്ഞാനങ്ങളും ഓര്‍മിക്കുക. അല്ലാഹു എല്ലാം നന്നായറിയുന്നവനും സൂക്ഷ്മജ്ഞനുമാണ്.
\end{malayalam}}
\flushright{\begin{Arabic}
\quranayah[33][35]
\end{Arabic}}
\flushleft{\begin{malayalam}
അല്ലാഹുവിലുള്ള സമര്‍പ്പണം, സത്യവിശ്വാസം, ഭയഭക്തി, സത്യസന്ധത, ക്ഷമാശീലം, വിനയം, ദാനശീലം, വ്രതാനുഷ്ഠാനം, ലൈംഗിക വിശുദ്ധി എന്നിവ ഉള്‍ക്കൊള്ളുന്നവരും അല്ലാഹുവെ ധാരാളമായി സ്മരിക്കുന്നവരുമായ സ്ത്രീപുരുഷന്മാര്‍ക്ക് അവന്‍ പാപമോചനവും മഹത്തായ പ്രതിഫലവും ഒരുക്കിവെച്ചിട്ടുണ്ട്.
\end{malayalam}}
\flushright{\begin{Arabic}
\quranayah[33][36]
\end{Arabic}}
\flushleft{\begin{malayalam}
അല്ലാഹുവും അവന്റെ ദൂതനും ഏതെങ്കിലും കാര്യത്തില്‍ വിധി പ്രഖ്യാപിച്ചുകഴിഞ്ഞാല്‍ സത്യവിശ്വാസിക്കോ വിശ്വാസിനിക്കോ അക്കാര്യത്തില്‍ മറിച്ചൊരു തീരുമാനമെടുക്കാന്‍ അവകാശമില്ല. ആരെങ്കിലും അല്ലാഹുവെയും അവന്റെ ദൂതനെയും ധിക്കരിക്കുകയാണെങ്കില്‍ അവന്‍ വ്യക്തമായ വഴികേടിലകപ്പെട്ടതുതന്നെ.
\end{malayalam}}
\flushright{\begin{Arabic}
\quranayah[33][37]
\end{Arabic}}
\flushleft{\begin{malayalam}
അല്ലാഹുവും നീയും ഔദാര്യം ചെയ്തുകൊടുത്ത ഒരാളോട് നീയിങ്ങനെ പറഞ്ഞ സന്ദര്‍ഭം: "നീ നിന്റെ ഭാര്യയെ നിന്നോടൊപ്പം നിര്‍ത്തുക; അല്ലാഹുവെ സൂക്ഷിക്കുക." അല്ലാഹു വെളിവാക്കാന്‍ പോകുന്ന ഒരു കാര്യം നീ മനസ്സിലൊളിപ്പിച്ചു വെക്കുകയായിരുന്നു. ജനങ്ങളെ പേടിക്കുകയും. എന്നാല്‍ നീ പേടിക്കേണ്ടത് അല്ലാഹുവിനെയാണ്. പിന്നീട് സൈദ് അവളില്‍ നിന്ന് തന്റെ ആവശ്യം നിറവേറ്റി കഴിഞ്ഞപ്പോള്‍ നാം അവളെ നിന്റെ ഭാര്യയാക്കിത്തന്നു. തങ്ങളുടെ ദത്തുപുത്രന്മാര്‍ അവരുടെ ഭാര്യമാരില്‍ നിന്നുള്ള ആവശ്യം നിറവേററിക്കഴിഞ്ഞാല്‍ അവരെ വിവാഹം ചെയ്യുന്ന കാര്യത്തില്‍ സത്യവിശ്വാസികള്‍ക്കൊട്ടും വിഷമമുണ്ടാവാതിരിക്കാനാണിത്. അല്ലാഹുവിന്റെ കല്‍പന നടപ്പാക്കപ്പെടുക തന്നെ ചെയ്യും.
\end{malayalam}}
\flushright{\begin{Arabic}
\quranayah[33][38]
\end{Arabic}}
\flushleft{\begin{malayalam}
അല്ലാഹു നിശ്ചയിച്ചുകൊടുത്ത ഇത്തരം കാര്യങ്ങളില്‍ പ്രവാചകന് ഒട്ടും പ്രയാസം തോന്നേണ്ടതില്ല. നേരത്തെ കഴിഞ്ഞുപോയവരുടെ കാര്യത്തില്‍ അല്ലാഹു നടപ്പാക്കിയ നടപടിക്രമം തന്നെയാണിത്. അല്ലാഹുവിന്റെ കല്‍പന കണിശമായും നടപ്പാക്കാനുള്ളതാണ്.
\end{malayalam}}
\flushright{\begin{Arabic}
\quranayah[33][39]
\end{Arabic}}
\flushleft{\begin{malayalam}
അഥവാ, അല്ലാഹുവിന്റെ സന്ദേശം മനുഷ്യര്‍ക്കു എത്തിച്ചുകൊടുക്കുന്നവരാണവര്‍. അവര്‍ അല്ലാഹുവെ പേടിക്കുന്നു. അവനല്ലാത്ത ആരെയും പേടിക്കുന്നുമില്ല. കണക്കുനോക്കാന്‍ അല്ലാഹു തന്നെ മതി.
\end{malayalam}}
\flushright{\begin{Arabic}
\quranayah[33][40]
\end{Arabic}}
\flushleft{\begin{malayalam}
മുഹമ്മദ് നിങ്ങളിലെ പുരുഷന്മാരിലാരുടെയും പിതാവല്ല. മറിച്ച്, അദ്ദേഹം അല്ലാഹുവിന്റെ ദൂതനാണ്. ദൈവദൂതന്മാരില്‍ അവസാനത്തെയാളും. എല്ലാ കാര്യങ്ങളെപ്പറ്റിയും നന്നായറിയുന്നവനാണ് അല്ലാഹു.
\end{malayalam}}
\flushright{\begin{Arabic}
\quranayah[33][41]
\end{Arabic}}
\flushleft{\begin{malayalam}
സത്യവിശ്വാസികളേ, നിങ്ങള്‍ അല്ലാഹുവെ ധാരാളമായി ഓര്‍ക്കുക.
\end{malayalam}}
\flushright{\begin{Arabic}
\quranayah[33][42]
\end{Arabic}}
\flushleft{\begin{malayalam}
കാലത്തും വൈകുന്നേരവും അവനെ കീര്‍ത്തിക്കുക.
\end{malayalam}}
\flushright{\begin{Arabic}
\quranayah[33][43]
\end{Arabic}}
\flushleft{\begin{malayalam}
അവനാണ് നിങ്ങള്‍ക്ക് കാരുണ്യമേകുന്നത്. അവന്റെ മലക്കുകള്‍ നിങ്ങള്‍ക്ക് കാരുണ്യത്തിനായി അര്‍ഥിക്കുന്നു. നിങ്ങളെ ഇരുളില്‍ നിന്ന് വെളിച്ചത്തിലേക്കു നയിക്കാനാണിത്. അല്ലാഹു സത്യവിശ്വാസികളോട് ഏറെ കരുണയുള്ളവനാണ്.
\end{malayalam}}
\flushright{\begin{Arabic}
\quranayah[33][44]
\end{Arabic}}
\flushleft{\begin{malayalam}
അവര്‍ അവനെ കണ്ടുമുട്ടുംനാള്‍ സലാം ചൊല്ലിയാണ് അവരെ അഭിവാദ്യം ചെയ്യുക. അവര്‍ക്കു മാന്യമായ പ്രതിഫലം ഒരുക്കിവെച്ചിട്ടുണ്ട്.
\end{malayalam}}
\flushright{\begin{Arabic}
\quranayah[33][45]
\end{Arabic}}
\flushleft{\begin{malayalam}
നബിയേ, നിശ്ചയമായും നാം നിന്നെ സാക്ഷിയും ശുഭവാര്‍ത്ത അറിയിക്കുന്നവനും മുന്നറിയിപ്പു നല്‍കുന്നവനുമായി അയച്ചിരിക്കുന്നു.
\end{malayalam}}
\flushright{\begin{Arabic}
\quranayah[33][46]
\end{Arabic}}
\flushleft{\begin{malayalam}
അല്ലാഹുവിന്റെ അനുമതി പ്രകാരം അവങ്കലേക്ക് ക്ഷണിക്കുന്നവനും പ്രകാശം പരത്തുന്ന വിളക്കുമായാണ് നിന്നെ അയച്ചത്.
\end{malayalam}}
\flushright{\begin{Arabic}
\quranayah[33][47]
\end{Arabic}}
\flushleft{\begin{malayalam}
സത്യവിശ്വാസികളെ ശുഭവാര്‍ത്ത അറിയിക്കുക, അല്ലാഹുവില്‍ നിന്ന് അവര്‍ക്ക് അതിമഹത്തായ അനുഗ്രഹങ്ങളുണ്ടെന്ന്.
\end{malayalam}}
\flushright{\begin{Arabic}
\quranayah[33][48]
\end{Arabic}}
\flushleft{\begin{malayalam}
സത്യനിഷേധികള്‍ക്കും കപടവിശ്വാസികള്‍ക്കും നീയൊട്ടും വഴങ്ങരുത്. അവരുടെ ദ്രോഹം അവഗണിക്കുക. അല്ലാഹുവില്‍ ഭരമേല്‍പിക്കുക. ഭരമേല്‍പിക്കാന്‍ അല്ലാഹു തന്നെ മതി.
\end{malayalam}}
\flushright{\begin{Arabic}
\quranayah[33][49]
\end{Arabic}}
\flushleft{\begin{malayalam}
വിശ്വസിച്ചവരേ, നിങ്ങള്‍ സത്യവിശ്വാസിനികളെ വിവാഹം കഴിക്കുകയും, പിന്നീട് അവരെ സ്പര്‍ശിക്കും മുമ്പായി വിവാഹമോചനം നടത്തുകയുമാണെങ്കില്‍ നിങ്ങള്‍ക്കായി ഇദ്ദ ആചരിക്കേണ്ട ബാധ്യത അവര്‍ക്കില്ല. എന്നാല്‍ നിങ്ങളവര്‍ക്ക് എന്തെങ്കിലും ജീവിതവിഭവം നല്‍കണം. നല്ല നിലയില്‍ അവരെ പിരിച്ചയക്കുകയും വേണം.
\end{malayalam}}
\flushright{\begin{Arabic}
\quranayah[33][50]
\end{Arabic}}
\flushleft{\begin{malayalam}
നബിയേ, നീ വിവാഹമൂല്യം നല്‍കിയ നിന്റെ പത്നിമാരെ നിനക്കു നാം അനുവദിച്ചുതന്നിരിക്കുന്നു. അല്ലാഹു നിനക്കു യുദ്ധത്തിലൂടെ അധീനപ്പെടുത്തിത്തന്നവരില്‍ നിന്റെ വലംകൈ ഉടമപ്പെടുത്തിയ അടിമസ്ത്രീകളെയും നിന്നോടൊപ്പം സ്വദേശം വെടിഞ്ഞ് പലായനം ചെയ്തെത്തിയ നിന്റെ പിതൃവ്യപുത്രിമാര്‍, പിതൃസഹോദരീപുത്രിമാര്‍, മാതൃസഹോദരപുത്രിമാര്‍, മാതൃസഹോദരീപുത്രിമാര്‍ എന്നിവരെയും വിവാഹം ചെയ്യാന്‍ അനുവാദമുണ്ട്. സത്യവിശ്വാസിയായ സ്ത്രീ സ്വന്തത്തെ പ്രവാചകന് ദാനം ചെയ്യുകയും അവളെ വിവാഹം കഴിക്കാനുദ്ദേശിക്കുകയുമാണെങ്കില്‍ അതിനും വിരോധമില്ല. സത്യവിശ്വാസികള്‍ക്ക് പൊതുവായി ബാധകമല്ലാത്ത നിനക്കു മാത്രമുള്ള നിയമമാണിത്. അവരുടെ ഭാര്യമാരുടെയും അടിമകളുടെയും കാര്യത്തില്‍ നാം നിയമമാക്കിയ കാര്യങ്ങള്‍ നമുക്കു നന്നായറിയാം. നിനക്ക് ഒന്നിലും ഒരു പ്രയാസവുമുണ്ടാവാതിരിക്കാനാണിത്. അല്ലാഹു ഏറെ പൊറുക്കുന്നവനും പരമദയാലുവുമാണ്.
\end{malayalam}}
\flushright{\begin{Arabic}
\quranayah[33][51]
\end{Arabic}}
\flushleft{\begin{malayalam}
ഭാര്യമാരില്‍ നിന്ന് നിനക്കിഷ്ടമുള്ളവരെ നിനക്കകറ്റി നിര്‍ത്താം. നീ ഉദ്ദേശിക്കുന്നവരെ അടുപ്പിച്ചുനിര്‍ത്താം. ഇഷ്ടമുള്ളവരെ അകറ്റിനിര്‍ത്തിയശേഷം അടുപ്പിച്ചു നിര്‍ത്തുന്നതിലും നിനക്കു കുറ്റമില്ല. അവരുടെ കണ്ണുകള്‍ കുളിര്‍ക്കാനും അവര്‍ ദുഃഖിക്കാതിരിക്കാനും നീ അവര്‍ക്കു നല്‍കിയതില്‍ അവര്‍ തൃപ്തരാകാനും ഏറ്റവും പറ്റിയതിതാണ്. നിങ്ങളുടെ മനസ്സിനകത്തുളളത് അല്ലാഹു അറിയുന്നു. അല്ലാഹു സര്‍വജ്ഞനാണ്. ഏറെ സഹനമുള്ളവനും അവന്‍ തന്നെ.
\end{malayalam}}
\flushright{\begin{Arabic}
\quranayah[33][52]
\end{Arabic}}
\flushleft{\begin{malayalam}
ഇനിമേല്‍ നിനക്കു ഒരു സ്ത്രീയെയും വിവാഹം ചെയ്യാന്‍ അനുവാദമില്ല. ഇവര്‍ക്കു പകരമായി മറ്റു ഭാര്യമാരെ സ്വീകരിക്കാനും പാടില്ല. അവരുടെ സൌന്ദര്യം നിന്നില്‍ കൌതുകമുണര്‍ത്തിയാലും ശരി. എന്നാല്‍ അടിമസ്ത്രീകളിതില്‍ നിന്നൊഴിവാണ്. അല്ലാഹു എല്ലാ കാര്യങ്ങളും നന്നായി നിരീക്ഷിക്കുന്നവന്‍ തന്നെ.
\end{malayalam}}
\flushright{\begin{Arabic}
\quranayah[33][53]
\end{Arabic}}
\flushleft{\begin{malayalam}
വിശ്വസിച്ചവരേ, പ്രവാചകന്റെ വീടുകളില്‍ നിങ്ങള്‍ അനുവാദമില്ലാതെ പ്രവേശിക്കരുത്. അവിടെ ആഹാരം പാകമാകുന്നത് പ്രതീക്ഷിച്ചിരിക്കരുത്. എന്നാല്‍ നിങ്ങളെ ഭക്ഷണത്തിനു ക്ഷണിച്ചാല്‍ നിങ്ങളവിടേക്കു ചെല്ലുക. ആഹാരം കഴിച്ചുകഴിഞ്ഞാല്‍ പിരിഞ്ഞുപോവുക. അവിടെ വര്‍ത്തമാനം പറഞ്ഞ് രസിച്ചിരിക്കരുത്. നിങ്ങളുടെ അത്തരം പ്രവൃത്തികള്‍ പ്രവാചകന്ന് പ്രയാസകരമാകുന്നുണ്ട്. എങ്കിലും നിങ്ങളോടതു തുറന്നുപറയാന്‍ പ്രവാചകന്‍ ലജ്ജിക്കുന്നു. എന്നാല്‍ അല്ലാഹു സത്യംപറയുന്നതിലൊട്ടും ലജ്ജിക്കുന്നില്ല. പ്രവാചക പത്നിമാരോട് നിങ്ങള്‍ വല്ലതും ചോദിക്കുന്നുവെങ്കില്‍ മറക്കുപിന്നില്‍ നിന്നാണ് നിങ്ങളവരോട് ചോദിക്കേണ്ടത്. അതാണ് നിങ്ങളുടെയും അവരുടെയും ഹൃദയശുദ്ധിക്ക് ഏറ്റം നല്ലത്. അല്ലാഹുവിന്റെ ദൂതനെ ശല്യപ്പെടുത്താന്‍ നിങ്ങള്‍ക്കനുവാദമില്ല. അദ്ദേഹത്തിന്റെ വിയോഗശേഷം അദ്ദേഹത്തിന്റെ ഭാര്യമാരെ വിവാഹം കഴിക്കാനും പാടില്ല. ഇതൊക്കെയും അല്ലാഹുവിങ്കല്‍ ഗൌരവമുള്ള കാര്യം തന്നെ.
\end{malayalam}}
\flushright{\begin{Arabic}
\quranayah[33][54]
\end{Arabic}}
\flushleft{\begin{malayalam}
നിങ്ങള്‍ എന്തെങ്കിലും വെളിപ്പെടുത്തിയാലും മറച്ചുവെച്ചാലും നിശ്ചയമായും അല്ലാഹു എല്ലാം നന്നായറിയുന്നവനാണ്.
\end{malayalam}}
\flushright{\begin{Arabic}
\quranayah[33][55]
\end{Arabic}}
\flushleft{\begin{malayalam}
പിതാക്കന്മാര്‍, പുത്രന്മാര്‍, സഹോദരന്മാര്‍, സഹോദരപുത്രന്മാര്‍, സഹോദരീപുത്രന്മാര്‍, തങ്ങളുടെ കൂട്ടത്തില്‍പ്പെട്ട സ്ത്രീകള്‍, തങ്ങളുടെ അടിമകള്‍ എന്നിവരുമായി ഇടപഴകുന്നതില്‍ പ്രവാചക പത്നിമാര്‍ക്കു കുറ്റമില്ല. നിങ്ങള്‍ അല്ലാഹുവെ സൂക്ഷിക്കുക. തീര്‍ച്ചയായും അല്ലാഹു എല്ലാ കാര്യങ്ങള്‍ക്കും സാക്ഷിയാണ്.
\end{malayalam}}
\flushright{\begin{Arabic}
\quranayah[33][56]
\end{Arabic}}
\flushleft{\begin{malayalam}
അല്ലാഹു പ്രവാചകനെ അനുഗ്രഹിക്കുന്നു. അവന്റെ മലക്കുകള്‍ അനുഗ്രഹത്തിനായി പ്രാര്‍ഥിക്കുന്നു. സത്യവിശ്വാസികളേ, നിങ്ങളും അദ്ദേഹത്തിന് കാരുണ്യവും ശാന്തിയുമുണ്ടാകാന്‍ പ്രാര്‍ഥിക്കുക.
\end{malayalam}}
\flushright{\begin{Arabic}
\quranayah[33][57]
\end{Arabic}}
\flushleft{\begin{malayalam}
അല്ലാഹുവെയും അവന്റെ ദൂതനെയും ദ്രോഹിക്കുന്നവരെ ഇഹത്തിലും പരത്തിലും അല്ലാഹു ശപിച്ചിരിക്കുന്നു. നന്നെ നിന്ദ്യമായ ശിക്ഷ അവര്‍ക്കായി തയ്യാറാക്കിവെച്ചിട്ടുണ്ട്.
\end{malayalam}}
\flushright{\begin{Arabic}
\quranayah[33][58]
\end{Arabic}}
\flushleft{\begin{malayalam}
സത്യവിശ്വാസികളെയും വിശ്വാസിനികളെയും, അവര്‍ തെറ്റൊന്നും ചെയ്യാതിരിക്കെ ദ്രോഹിക്കുന്നവര്‍ കള്ളവാര്‍ത്ത ചമച്ചവരത്രെ. പ്രകടമായ കുറ്റം ചെയ്തവരും.
\end{malayalam}}
\flushright{\begin{Arabic}
\quranayah[33][59]
\end{Arabic}}
\flushleft{\begin{malayalam}
നബിയേ, നിന്റെ പത്നിമാര്‍, പുത്രിമാര്‍, വിശ്വാസികളുടെ സ്ത്രീകള്‍ ഇവരോടെല്ലാം തങ്ങളുടെ മൂടുപടങ്ങള്‍ ശരീരത്തില്‍ താഴ്ത്തിയിടാന്‍ നിര്‍ദേശിക്കുക. അവരെ തിരിച്ചറിയാന്‍ ഏറ്റം പറ്റിയ മാര്‍ഗമതാണ്; ശല്യം ചെയ്യപ്പെടാതിരിക്കാനും. അല്ലാഹു ഏറെ പൊറുക്കുന്നവനും പരമദയാലുവുമാണ്.
\end{malayalam}}
\flushright{\begin{Arabic}
\quranayah[33][60]
\end{Arabic}}
\flushleft{\begin{malayalam}
കപടവിശ്വാസികളും, ദീനംപിടിച്ച മനസ്സുള്ളവരും, മദീനയില്‍ ഭീതിയുണര്‍ത്തുന്ന കള്ളവാര്‍ത്തകള്‍ പരത്തുന്നവരും തങ്ങളുടെ ചെയ്തികള്‍ക്ക് അറുതി വരുത്തുന്നില്ലെങ്കില്‍ അവര്‍ക്കെതിരെ നിന്നെ നാം തിരിച്ചുവിടുക തന്നെ ചെയ്യും. പിന്നെ അവര്‍ക്ക് ഈ പട്ടണത്തില്‍ ഇത്തിരി കാലമേ നിന്നോടൊപ്പം കഴിയാനൊക്കുകയുള്ളൂ.
\end{malayalam}}
\flushright{\begin{Arabic}
\quranayah[33][61]
\end{Arabic}}
\flushleft{\begin{malayalam}
അവര്‍ ശപിക്കപ്പെട്ടവരായിരിക്കും. എവിടെ കണ്ടെത്തിയാലും അവരെ പിടികൂടി വകവരുത്തും.
\end{malayalam}}
\flushright{\begin{Arabic}
\quranayah[33][62]
\end{Arabic}}
\flushleft{\begin{malayalam}
നേരത്തെ കഴിഞ്ഞുപോയവരുടെ കാര്യത്തില്‍ അല്ലാഹു സ്വീകരിച്ച നടപടിക്രമം തന്നെയാണിത്. അല്ലാഹുവിന്റെ നടപടിക്രമത്തിലൊരു മാറ്റവും നിനക്കു കണ്ടെത്താനാവില്ല.
\end{malayalam}}
\flushright{\begin{Arabic}
\quranayah[33][63]
\end{Arabic}}
\flushleft{\begin{malayalam}
ജനം അന്ത്യദിനത്തെപ്പറ്റി നിന്നോടു ചോദിക്കുന്നു. പറയുക: "അതേക്കുറിച്ച അറിവ് അല്ലാഹുവിങ്കല്‍ മാത്രമേയുള്ളൂ." അതേപ്പറ്റി നിനക്കെന്തറിയാം? ഒരുവേള അത് വളരെ അടുത്തുതന്നെയായേക്കാം.
\end{malayalam}}
\flushright{\begin{Arabic}
\quranayah[33][64]
\end{Arabic}}
\flushleft{\begin{malayalam}
സംശയമില്ല; അല്ലാഹു സത്യനിഷേധികളെ ശപിച്ചിരിക്കുന്നു. അവര്‍ക്ക് കത്തിക്കാളുന്ന നരകത്തീ തയ്യാറാക്കിവെച്ചിട്ടുമുണ്ട്.
\end{malayalam}}
\flushright{\begin{Arabic}
\quranayah[33][65]
\end{Arabic}}
\flushleft{\begin{malayalam}
അവരവിടെ, എന്നെന്നും സ്ഥിരവാസികളായിരിക്കും. അവര്‍ക്കവിടെ ഒരു രക്ഷകനെയും സഹായിയെയും കണ്ടെത്താനാവില്ല.
\end{malayalam}}
\flushright{\begin{Arabic}
\quranayah[33][66]
\end{Arabic}}
\flushleft{\begin{malayalam}
അവരുടെ മുഖങ്ങള്‍ നരകത്തീയില്‍ തിരിച്ചുമറിക്കപ്പെടും. അന്ന് അവര്‍ പറയും: "ഞങ്ങള്‍ അല്ലാഹുവെയും അവന്റെ ദൂതനെയും അനുസരിച്ചിരുന്നെങ്കില്‍ എത്ര നന്നായേനെ."
\end{malayalam}}
\flushright{\begin{Arabic}
\quranayah[33][67]
\end{Arabic}}
\flushleft{\begin{malayalam}
അവര്‍ വിലപിക്കും: "ഞങ്ങളുടെ നാഥാ, ഞങ്ങള്‍ ഞങ്ങളുടെ നേതാക്കളെയും പ്രമാണിമാരെയും അനുസരിച്ചു. അവര്‍ ഞങ്ങളെ വഴിപിഴപ്പിച്ചു.
\end{malayalam}}
\flushright{\begin{Arabic}
\quranayah[33][68]
\end{Arabic}}
\flushleft{\begin{malayalam}
"ഞങ്ങളുടെ നാഥാ, അവര്‍ക്കു നീ രണ്ടിരട്ടി ശിക്ഷ നല്‍കേണമേ; അവരെ നീ കൊടുംശാപത്തിനിരയാക്കേണമേ."
\end{malayalam}}
\flushright{\begin{Arabic}
\quranayah[33][69]
\end{Arabic}}
\flushleft{\begin{malayalam}
വിശ്വസിച്ചവരേ, നിങ്ങള്‍ മൂസാക്കു മനോവിഷമമുണ്ടാക്കിയവരെപ്പോലെയാകരുത്. പിന്നെ അല്ലാഹു അദ്ദേഹത്തെ അവരുടെ ദുരാരോപണങ്ങളില്‍നിന്ന് മോചിപ്പിച്ചു. അദ്ദേഹം അല്ലാഹുവിന്റെയടുത്ത് അന്തസ്സുള്ളവനാണ്.
\end{malayalam}}
\flushright{\begin{Arabic}
\quranayah[33][70]
\end{Arabic}}
\flushleft{\begin{malayalam}
വിശ്വസിച്ചവരേ, നിങ്ങള്‍ ദൈവഭക്തരാവുക. നല്ലതുമാത്രം പറയുക.
\end{malayalam}}
\flushright{\begin{Arabic}
\quranayah[33][71]
\end{Arabic}}
\flushleft{\begin{malayalam}
എങ്കില്‍ അല്ലാഹു നിങ്ങള്‍ക്ക് നിങ്ങളുടെ കര്‍മങ്ങള്‍ നന്നാക്കിത്തരും. നിങ്ങളുടെ പാപങ്ങള്‍ പൊറുത്തുതരും. അല്ലാഹുവെയും അവന്റെ ദൂതനെയും അനുസരിക്കുന്നവന്‍ മഹത്തായ വിജയം കൈവരിച്ചിരിക്കുന്നു.
\end{malayalam}}
\flushright{\begin{Arabic}
\quranayah[33][72]
\end{Arabic}}
\flushleft{\begin{malayalam}
തീര്‍ച്ചയായും ആകാശഭൂമികളുടെയും പര്‍വതങ്ങളുടെയും മുമ്പില്‍ നാം ഈ അമാനത്ത് സമര്‍പ്പിച്ചു. അപ്പോള്‍ അതേറ്റെടുക്കാന്‍ അവ വിസമ്മതിച്ചു. അവ അതിനെ ഭയപ്പെട്ടു. എന്നാല്‍ മനുഷ്യന്‍ അതേറ്റെടുത്തു. അവന്‍ കൊടിയ അക്രമിയും തികഞ്ഞ അവിവേകിയും തന്നെ.
\end{malayalam}}
\flushright{\begin{Arabic}
\quranayah[33][73]
\end{Arabic}}
\flushleft{\begin{malayalam}
കപടവിശ്വാസികളും ബഹുദൈവവിശ്വാസികളുമായ സ്ത്രീപുരുഷന്മാരെ അല്ലാഹു ശിക്ഷിക്കുന്നതിനു വേണ്ടിയാണിത്. സത്യവിശ്വാസികളായ സ്ത്രീപുരുഷന്മാരുടെ പശ്ചാത്താപം അവന്‍ സ്വീകരിക്കാനും. അല്ലാഹു ഏറെ പൊറുക്കുന്നവനാണ്. പരമദയാലുവും.
\end{malayalam}}
\chapter{\textmalayalam{സബഅ്}}
\begin{Arabic}
\Huge{\centerline{\basmalah}}\end{Arabic}
\flushright{\begin{Arabic}
\quranayah[34][1]
\end{Arabic}}
\flushleft{\begin{malayalam}
ആകാശഭൂമികളിലുള്ള എല്ലാറ്റിന്റെയും ഉടമയായ അല്ലാഹുവിനാണ് സര്‍വസ്തുതിയും. പരലോകത്തും സ്തുതി അവനുതന്നെ. അവന്‍ യുക്തിമാനാണ്. സൂക്ഷ്മമായി അറിയുന്നവനും.
\end{malayalam}}
\flushright{\begin{Arabic}
\quranayah[34][2]
\end{Arabic}}
\flushleft{\begin{malayalam}
ഭൂമിയില്‍ പ്രവേശിക്കുന്നത്, അതില്‍നിന്ന് പുറത്തുവരുന്നത്, ആകാശത്തുനിന്നിറങ്ങുന്നത്, അവിടേക്കു കയറിപ്പോകുന്നത്; എല്ലാം അവനറിയുന്നു. അവന്‍ പരമ കാരുണികനാണ്. ഏറെ പൊറുക്കുന്നവനും.
\end{malayalam}}
\flushright{\begin{Arabic}
\quranayah[34][3]
\end{Arabic}}
\flushleft{\begin{malayalam}
സത്യനിഷേധികള്‍ പറയുന്നു: "എന്താണ് ആ അന്ത്യസമയം ഞങ്ങള്‍ക്കിങ്ങു വന്നെത്താത്തത്?" പറയുക: "എന്റെ നാഥനാണ് സത്യം. അതു നിങ്ങള്‍ക്കു വന്നെത്തുക തന്നെ ചെയ്യും. അഭൌതിക കാര്യങ്ങളറിയുന്ന എന്റെ നാഥനില്‍നിന്ന് ഒളിഞ്ഞുകിടക്കുന്ന ഒരണുപോലുമില്ല. ആകാശങ്ങളിലില്ല; ഭൂമിയിലുമില്ല. അണുവെക്കാള്‍ ചെറുതുമില്ല; വലുതുമില്ല. എല്ലാം സുവ്യക്തമായ ഒരു പ്രമാണത്തിലുണ്ട്. അതിലില്ലാത്ത ഒന്നുമില്ല.
\end{malayalam}}
\flushright{\begin{Arabic}
\quranayah[34][4]
\end{Arabic}}
\flushleft{\begin{malayalam}
സത്യവിശ്വാസം സ്വീകരിക്കുകയും സല്‍ക്കര്‍മങ്ങള്‍ പ്രവര്‍ത്തിക്കുകയും ചെയ്തവര്‍ക്ക് പ്രതിഫലം നല്‍കാനാണിത്. അവര്‍ക്ക് പാപമോചനമുണ്ട്. മാന്യമായ ജീവിതവിഭവങ്ങളും.
\end{malayalam}}
\flushright{\begin{Arabic}
\quranayah[34][5]
\end{Arabic}}
\flushleft{\begin{malayalam}
നമ്മെ പരാജയപ്പെടുത്താനുദ്ദേശിച്ച് നമ്മുടെ വചനങ്ങളെ എതിര്‍ക്കാന്‍ ശ്രമിച്ചവര്‍ക്കാണ് നോവേറിയ കഠിനശിക്ഷയുള്ളത്.
\end{malayalam}}
\flushright{\begin{Arabic}
\quranayah[34][6]
\end{Arabic}}
\flushleft{\begin{malayalam}
അറിവുള്ളവര്‍ കണ്ടു മനസ്സിലാക്കുന്നു, നിന്റെ നാഥനില്‍ നിന്ന് നിനക്കിറക്കിക്കിട്ടിയതുതന്നെയാണ് സത്യമെന്ന്. അത് പ്രതാപിയും സ്തുത്യര്‍ഹനുമായ അല്ലാഹുവിന്റെ മാര്‍ഗത്തിലേക്ക് നയിക്കുന്നതാണെന്നും.
\end{malayalam}}
\flushright{\begin{Arabic}
\quranayah[34][7]
\end{Arabic}}
\flushleft{\begin{malayalam}
സത്യനിഷേധികള്‍ പരിഹാസത്തോടെ പറയുന്നു: "ഒരുത്തനെപ്പറ്റി ഞങ്ങള്‍ നിങ്ങള്‍ക്കറിയിച്ചു തരട്ടെയോ? നിങ്ങള്‍ മരിച്ച് തീര്‍ത്തും ഛിന്നഭിന്നമായി മാറിയാലും വീണ്ടും പുതുതായി സൃഷ്ടിക്കപ്പെടുമെന്ന് നിങ്ങളോടു പറയുന്നവനാണവന്‍.
\end{malayalam}}
\flushright{\begin{Arabic}
\quranayah[34][8]
\end{Arabic}}
\flushleft{\begin{malayalam}
"അവന്‍ അല്ലാഹുവിന്റെ പേരില്‍ കള്ളം കെട്ടിപ്പറയുകയാണോ? അതല്ല; അവന് ഭ്രാന്തു ബാധിച്ചതാണോ?" അറിയുക: പരലോകത്തില്‍ വിശ്വസിക്കാത്തവര്‍ ശിക്ഷാര്‍ഹരാണ്. അളവറ്റ വഴികേടിലും.
\end{malayalam}}
\flushright{\begin{Arabic}
\quranayah[34][9]
\end{Arabic}}
\flushleft{\begin{malayalam}
അവരുടെ മുന്നിലും പിന്നിലുമുള്ള ആകാശഭൂമികളവര്‍ നോക്കികണ്ടിട്ടില്ലേ? നാം ഇച്ഛിക്കുകയാണെങ്കില്‍ നാമവരെ ഭൂമിയില്‍ ആഴ്ത്തിക്കളയും. അല്ലെങ്കില്‍ അവര്‍ക്കുമേല്‍ ആകാശത്തിന്റെ അടലുകള്‍ വീഴ്ത്തും. പശ്ചാത്തപിച്ച് മടങ്ങുന്ന ഏതൊരു ദാസനും തീര്‍ച്ചയായും ഇതില്‍ ദൃഷ്ടാന്തമുണ്ട്.
\end{malayalam}}
\flushright{\begin{Arabic}
\quranayah[34][10]
\end{Arabic}}
\flushleft{\begin{malayalam}
സംശയമില്ല; ദാവൂദിന് നാം നമ്മില്‍ നിന്നുള്ള അനുഗ്രഹമേകി. നാം നിര്‍ദേശിച്ചു: "മലകളേ; നിങ്ങള്‍ അദ്ദേഹത്തോടൊപ്പം സങ്കീര്‍ത്തനമാലപിക്കുക. പക്ഷികളേ; നിങ്ങളും." അദ്ദേഹത്തിന് നാം ഇരുമ്പ് മയപ്പെടുത്തിക്കൊടുത്തു.
\end{malayalam}}
\flushright{\begin{Arabic}
\quranayah[34][11]
\end{Arabic}}
\flushleft{\begin{malayalam}
മികവുറ്റ പടയങ്കികളുണ്ടാക്കുക. അതിന്റെ കണ്ണികള്‍ക്ക് കൃത്യത വരുത്തുക. സല്‍ക്കര്‍മങ്ങള്‍ പ്രവര്‍ത്തിക്കുക. നിങ്ങള്‍ ചെയ്യുന്നതെല്ലാം നന്നായി കണ്ടുകൊണ്ടിരിക്കുന്നവനാണ് നാം; തീര്‍ച്ച.
\end{malayalam}}
\flushright{\begin{Arabic}
\quranayah[34][12]
\end{Arabic}}
\flushleft{\begin{malayalam}
സുലൈമാന്ന് നാം കാറ്റിനെ അധീനപ്പെടുത്തിക്കൊടുത്തു. അതിന്റെ പ്രഭാതസഞ്ചാരം ഒരു മാസത്തെ വഴിദൂരമാണ്. സായാഹ്ന സഞ്ചാരവും ഒരുമാസത്തെ വഴിദൂരം തന്നെ. അദ്ദേഹത്തിന് നാം ചെമ്പിന്റെ ഉരുകിയ ഉറവ ഒഴുക്കിക്കൊടുത്തു. അദ്ദേഹത്തിന്റെ അടുത്ത് കുറേ ജിന്നുകളും ജോലിക്കാരായുണ്ടായിരുന്നു. അദ്ദേഹത്തിന്റെ നാഥന്റെ നിര്‍ദേശാനുസരണമായിരുന്നു അത്. അവരിലാരെങ്കിലും നമ്മുടെ കല്‍പന ലംഘിക്കുകയാണെങ്കില്‍ നാമവനെ ആളിക്കത്തുന്ന നരകത്തീയിന്റെ രുചി ആസ്വദിപ്പിക്കും.
\end{malayalam}}
\flushright{\begin{Arabic}
\quranayah[34][13]
\end{Arabic}}
\flushleft{\begin{malayalam}
അവര്‍ അദ്ദേഹമാഗ്രഹിക്കുന്നതൊക്കെ ഉണ്ടാക്കിക്കൊടുത്തു. കൂറ്റന്‍ കെട്ടിടങ്ങള്‍, കൌതുകകരമായ പ്രതിമകള്‍, തടാകങ്ങള്‍ പോലുള്ള തളികകള്‍, വെച്ചിടത്തുനിന്നിളകാത്ത കനത്ത പാചകപ്പാത്രങ്ങള്‍; എല്ലാം. ദാവൂദ് കുടുംബമേ! നിങ്ങള്‍ നന്ദിപൂര്‍വം പ്രവര്‍ത്തിക്കുക. എന്റെ ദാസന്മാരില്‍ നന്ദിയുള്ളവര്‍ വളരെ വിരളമാണ്.
\end{malayalam}}
\flushright{\begin{Arabic}
\quranayah[34][14]
\end{Arabic}}
\flushleft{\begin{malayalam}
പിന്നീട് സുലൈമാന്ന് നാം മരണം വിധിച്ചു. അപ്പോള്‍ ആരും ആ മരണം ജിന്നുകളെ അറിയിച്ചില്ല. അദ്ദേഹത്തിന്റെ ഊന്നുവടി തിന്നുകൊണ്ടിരുന്ന ചിതലുകളൊഴികെ. അങ്ങനെ സുലൈമാന്‍ നിലം പതിച്ചപ്പോള്‍ ജിന്നുകള്‍ക്ക് ബോധ്യമായി; തങ്ങള്‍ക്ക് അഭൌതിക കാര്യങ്ങള്‍ അറിയുമായിരുന്നെങ്കില്‍ അപമാനകരമായ ഈ ദുരവസ്ഥയില്‍ കഴിഞ്ഞുകൂടേണ്ടിവരില്ലായിരുന്നുവെന്ന്.
\end{malayalam}}
\flushright{\begin{Arabic}
\quranayah[34][15]
\end{Arabic}}
\flushleft{\begin{malayalam}
സബഅ് നിവാസികള്‍ക്ക് അവരുടെ താമസസ്ഥലത്തുതന്നെ ദൃഷ്ടാന്തമുണ്ടായിരുന്നു. വലത്തും ഇടത്തുമുള്ള രണ്ടു തോട്ടങ്ങളാണത്. "നിങ്ങളുടെ നാഥന്‍ തന്ന അന്നം തിന്നുകൊള്ളുക. അവനോട് നന്ദി കാണിക്കുക. എന്തു നല്ല നാട്! എത്ര നന്നായി പൊറുക്കുന്ന നാഥന്‍."
\end{malayalam}}
\flushright{\begin{Arabic}
\quranayah[34][16]
\end{Arabic}}
\flushleft{\begin{malayalam}
എന്നാല്‍ അവര്‍ പിന്തിരിഞ്ഞുകളഞ്ഞു. അവസാനം അവരുടെ നേരെ നാം അണക്കെട്ടില്‍ നിന്നുള്ള അണമുറിയാത്ത ജലപ്രവാഹമയച്ചു. അവരുടെ ആ രണ്ടു തോട്ടങ്ങളും നശിച്ചു. പകരം നാം വേറെ രണ്ടു തോട്ടങ്ങള്‍ നല്‍കി. അവ കയ്പ്പുറ്റ കനികളും കാറ്റാടി മരങ്ങളും ഏതാനും ഇലന്ത മരങ്ങളും മാത്രമുള്ളതായിരുന്നു.
\end{malayalam}}
\flushright{\begin{Arabic}
\quranayah[34][17]
\end{Arabic}}
\flushleft{\begin{malayalam}
അവര്‍ നന്ദികേട് കാണിച്ചതിന് നാം അവര്‍ക്കു നല്‍കിയ പ്രതിഫലം. നന്ദികെട്ടവര്‍ക്കല്ലാതെ നാം ഇത്തരം പ്രതിഫലം നല്‍കുമോ?
\end{malayalam}}
\flushright{\begin{Arabic}
\quranayah[34][18]
\end{Arabic}}
\flushleft{\begin{malayalam}
അവര്‍ക്കും, നാം അനുഗ്രഹിച്ച ഗ്രാമങ്ങള്‍ക്കുമിടയില്‍ തെളിഞ്ഞുകാണാവുന്ന പല പ്രദേശങ്ങളും നാമുണ്ടാക്കി. അവയില്‍ നാം സഞ്ചാര ദൈര്‍ഘ്യം നിര്‍ണയിക്കുകയും ചെയ്തു. നിങ്ങളവയിലൂടെ രാപ്പകലുകളില്‍ നിര്‍ഭയം സഞ്ചരിച്ചുകൊള്ളുക.
\end{malayalam}}
\flushright{\begin{Arabic}
\quranayah[34][19]
\end{Arabic}}
\flushleft{\begin{malayalam}
അപ്പോള്‍ അവര്‍ പറഞ്ഞു: "ഞങ്ങളുടെ നാഥാ, ഞങ്ങളുടെ യാത്രാ താവളങ്ങള്‍ക്കിടയില്‍ നീ ദൂരം വര്‍ധിപ്പിച്ചുതരേണമേ." അങ്ങനെ അവര്‍ തങ്ങള്‍ക്കുതന്നെ ദ്രോഹം വരുത്തുകയായിരുന്നു. അവസാനം നാമവരെ കേവലം കഥകളാക്കി. അപ്പാടെ ഛിന്നഭിന്നമാക്കി. നിശ്ചയമായും നല്ല ക്ഷമാശീലര്‍ക്കും നന്ദിയുള്ളവര്‍ക്കും ഇതില്‍ ധാരാളം ദൃഷ്ടാന്തങ്ങളുണ്ട്.
\end{malayalam}}
\flushright{\begin{Arabic}
\quranayah[34][20]
\end{Arabic}}
\flushleft{\begin{malayalam}
അങ്ങനെ അവരെ സംബന്ധിച്ച തന്റെ ധാരണ ശരിയാണെന്ന് ഇബ്ലീസ് തെളിയിച്ചു. അവര്‍ അവനെ പിന്തുടര്‍ന്നു. സത്യവിശ്വാസികളുടെ സംഘമൊഴികെ.
\end{malayalam}}
\flushright{\begin{Arabic}
\quranayah[34][21]
\end{Arabic}}
\flushleft{\begin{malayalam}
ഇബ്ലീസിന് അവരുടെമേല്‍ ഒരധികാരവുമുണ്ടായിരുന്നില്ല. പരലോകത്തില്‍ വിശ്വസിക്കുന്നവരെയും സംശയിക്കുന്നവരെയും വേര്‍തിരിച്ചറിയാന്‍ മാത്രമാണിത്. നിന്റെ നാഥന്‍ സകല സംഗതികളും ശ്രദ്ധാപൂര്‍വം നിരീക്ഷിച്ചുകൊണ്ടിരിക്കുന്നവനാണ്.
\end{malayalam}}
\flushright{\begin{Arabic}
\quranayah[34][22]
\end{Arabic}}
\flushleft{\begin{malayalam}
പറയുക: അല്ലാഹുവെക്കൂടാതെ നിങ്ങള്‍ ദൈവമായി സങ്കല്‍പിച്ചുണ്ടാക്കിയവരോടൊക്കെ വിളിച്ചു പ്രാര്‍ഥിച്ചുനോക്കുക. ആകാശത്ത് ഒരണുത്തൂക്കത്തിന്റെ ഉടമാവകാശംപോലും അവര്‍ക്കില്ല. ഭൂമിയിലുമില്ല. അവ രണ്ടിലും അവര്‍ക്കൊരു പങ്കുമില്ല. അവരിലൊന്നും അല്ലാഹുവിന് ഒരു സഹായിയുമില്ല.
\end{malayalam}}
\flushright{\begin{Arabic}
\quranayah[34][23]
\end{Arabic}}
\flushleft{\begin{malayalam}
അല്ലാഹുവിന്റെ അടുത്ത് ശിപാര്‍ശയൊട്ടും ഉപകരിക്കുകയില്ല; അവന്‍ അനുമതി നല്‍കിയവര്‍ക്കല്ലാതെ. അങ്ങനെ അവരുടെ ഹൃദയങ്ങളില്‍നിന്ന് പരിഭ്രമം നീങ്ങിയില്ലാതാകുമ്പോള്‍ അവര്‍ ശിപാര്‍ശകരോടു ചോദിക്കുന്നു: "നിങ്ങളുടെ നാഥന്‍ എന്താണ് പറഞ്ഞത്?" അവര്‍ മറുപടി പറയും: "സത്യം തന്നെ. അവന്‍ അത്യുന്നതനാണ്. എല്ലാ നിലക്കും വലിയവനും."
\end{malayalam}}
\flushright{\begin{Arabic}
\quranayah[34][24]
\end{Arabic}}
\flushleft{\begin{malayalam}
ചോദിക്കുക: "ആകാശഭൂമികളില്‍നിന്ന് നിങ്ങള്‍ക്ക് അന്നം തരുന്നത് ആരാണ്?" പറയുക: അല്ലാഹു. അപ്പോള്‍ ഞങ്ങളോ നിങ്ങളോ രണ്ടിലൊരു വിഭാഗം നേര്‍വഴിയിലാണ്. അല്ലെങ്കില്‍ പ്രകടമായ വഴികേടിലും.
\end{malayalam}}
\flushright{\begin{Arabic}
\quranayah[34][25]
\end{Arabic}}
\flushleft{\begin{malayalam}
പറയുക: "ഞങ്ങള്‍ തെറ്റ് ചെയ്യുന്നതിനെപ്പറ്റി നിങ്ങളോടാരും ചോദിക്കുകയില്ല. നിങ്ങള്‍ പ്രവര്‍ത്തിക്കുന്നതിനെപ്പറ്റി ഞങ്ങളോടും ചോദിക്കുകയില്ല."
\end{malayalam}}
\flushright{\begin{Arabic}
\quranayah[34][26]
\end{Arabic}}
\flushleft{\begin{malayalam}
പറയുക: "നമ്മുടെ നാഥന്‍ നമ്മെയെല്ലാം ഒരുമിച്ചുകൂട്ടും. പിന്നീട് അവന്‍ നമുക്കിടയില്‍ ന്യായമായ തീരുമാനമെടുക്കും. അവന്‍ എല്ലാം അറിയുന്ന വിധികര്‍ത്താവാണ്."
\end{malayalam}}
\flushright{\begin{Arabic}
\quranayah[34][27]
\end{Arabic}}
\flushleft{\begin{malayalam}
പറയുക: "നിങ്ങള്‍ അവന് സമന്മാരാക്കി സങ്കല്‍പിച്ച ആ പങ്കാളികളെ എനിക്കൊന്നു കാണിച്ചുതരൂ!" ഒരിക്കലുമില്ല. അവനു സമന്മാരില്ല. അവന്‍ അല്ലാഹു മാത്രം. പ്രതാപിയും യുക്തിമാനും എല്ലാം അവന്‍ തന്നെ.
\end{malayalam}}
\flushright{\begin{Arabic}
\quranayah[34][28]
\end{Arabic}}
\flushleft{\begin{malayalam}
മനുഷ്യര്‍ക്കാകമാനം ശുഭവാര്‍ത്ത അറിയിക്കുന്നവനും മുന്നറിയിപ്പ് നല്‍കുന്നവനുമായല്ലാതെ നിന്നെ നാം അയച്ചിട്ടില്ല. പക്ഷേ, ഏറെപ്പേരും അതറിയുന്നില്ല.
\end{malayalam}}
\flushright{\begin{Arabic}
\quranayah[34][29]
\end{Arabic}}
\flushleft{\begin{malayalam}
അവര്‍ ചോദിക്കുന്നു: "ഈ വാഗ്ദാനം എപ്പോഴാണ് പുലരുക? നിങ്ങള്‍ സത്യവാദികളെങ്കില്‍."
\end{malayalam}}
\flushright{\begin{Arabic}
\quranayah[34][30]
\end{Arabic}}
\flushleft{\begin{malayalam}
പറയുക: "നിങ്ങള്‍ക്ക് ഒരു നിശ്ചിത അവധി ദിനമുണ്ട്. നിങ്ങളതില്‍ നിന്നൊട്ടും പിറകോട്ടു പോവില്ല. മുന്നോട്ടുവരികയുമില്ല."
\end{malayalam}}
\flushright{\begin{Arabic}
\quranayah[34][31]
\end{Arabic}}
\flushleft{\begin{malayalam}
സത്യനിഷേധികള്‍ പറയുന്നു: "ഞങ്ങള്‍ ഈ ഖുര്‍ആനിലൊരിക്കലും വിശ്വസിക്കില്ല. അതിനു മുമ്പുള്ള വേദങ്ങളിലും വിശ്വസിക്കില്ല." ഈ അതിക്രമികളെ അവരുടെ നാഥന്റെ അടുത്തു നിര്‍ത്തുന്നത് നീ കണ്ടിരുന്നെങ്കില്‍! അന്നേരമവര്‍ പരസ്പരം കുറ്റാരോപണം നടത്തിക്കൊണ്ടിരിക്കും. ഇഹലോകത്ത് മര്‍ദിച്ചൊതുക്കപ്പെട്ടിരുന്നവര്‍ അഹന്ത നടിച്ചിരുന്നവരോടു പറയും: "നിങ്ങളില്ലായിരുന്നെങ്കില്‍ ഞങ്ങള്‍ വിശ്വാസികളായിരുന്നേനെ."
\end{malayalam}}
\flushright{\begin{Arabic}
\quranayah[34][32]
\end{Arabic}}
\flushleft{\begin{malayalam}
അഹങ്കരിച്ചിരുന്നവര്‍ അടിച്ചമര്‍ത്തപ്പെട്ടിരുന്നവരോട് പറയും: "നിങ്ങള്‍ക്ക് നേര്‍വഴി വന്നെത്തിയശേഷം നിങ്ങളെ അതില്‍നിന്ന് തടഞ്ഞുനിര്‍ത്തിയത് ഞങ്ങളാണോ? അല്ല; നിങ്ങള്‍ കുറ്റവാളികള്‍ തന്നെയായിരുന്നു."
\end{malayalam}}
\flushright{\begin{Arabic}
\quranayah[34][33]
\end{Arabic}}
\flushleft{\begin{malayalam}
അടിച്ചമര്‍ത്തപ്പെട്ടിരുന്നവര്‍ അഹന്ത നടിച്ചിരുന്നവരോടു പറയും: "അല്ല, രാപ്പകലുകളിലെ നിങ്ങളുടെ കുതന്ത്രത്തിന്റെ ഫലമാണിത്. ഞങ്ങള്‍ അല്ലാഹുവെ നിഷേധിക്കാനും അവനു സമന്മാരെ സങ്കല്‍പിക്കാനും നിങ്ങള്‍ കല്‍പിച്ചുകൊണ്ടിരുന്ന കാര്യം ഓര്‍ക്കുക." അവസാനം ശിക്ഷ കാണുമ്പോള്‍ അവര്‍ ദുഃഖം ഉള്ളിലൊളിപ്പിക്കും. സത്യനിഷേധികളുടെ കഴുത്തില്‍ നാം കൂച്ചുവിലങ്ങിടും. അവര്‍ പ്രവര്‍ത്തിച്ചുകൊണ്ടിരുന്നതിന്റെ പ്രതിഫലമല്ലേ അവര്‍ക്കുണ്ടാവൂ.
\end{malayalam}}
\flushright{\begin{Arabic}
\quranayah[34][34]
\end{Arabic}}
\flushleft{\begin{malayalam}
ഏതൊരു നാട്ടിലേക്ക് നാം മുന്നറിയിപ്പുകാരെ അയച്ചുവോ, അപ്പോഴൊക്കെ അവിടങ്ങളിലെ ധൂര്‍ത്തന്മാര്‍ പറഞ്ഞു: "നിങ്ങള്‍ കൊണ്ടുവന്ന സന്ദേശത്തെ ഞങ്ങളിതാ തള്ളിക്കളയുന്നു."
\end{malayalam}}
\flushright{\begin{Arabic}
\quranayah[34][35]
\end{Arabic}}
\flushleft{\begin{malayalam}
അവര്‍ പറഞ്ഞുകൊണ്ടിരുന്നു: "ഞങ്ങള്‍ കൂടുതല്‍ സമ്പത്തും സന്താനങ്ങളുമുള്ളവരാണ്. ഞങ്ങളെന്തായാലും ശിക്ഷിക്കപ്പെടുകയില്ല."
\end{malayalam}}
\flushright{\begin{Arabic}
\quranayah[34][36]
\end{Arabic}}
\flushleft{\begin{malayalam}
പറയുക: "എന്റെ നാഥന്‍ അവനിച്ഛിക്കുന്നവര്‍ക്ക് ഉപജീവനത്തില്‍ ഉദാരത വരുത്തുന്നു. അവനിച്ഛിക്കുന്നവര്‍ക്ക് അതിലിടുക്കമുണ്ടാക്കുകയും ചെയ്യുന്നു." പക്ഷേ, അധികമാളുകളും അതറിയുന്നില്ല.
\end{malayalam}}
\flushright{\begin{Arabic}
\quranayah[34][37]
\end{Arabic}}
\flushleft{\begin{malayalam}
നിങ്ങളുടെ സമ്പത്തും സന്താനങ്ങളും നിങ്ങളെ നമ്മോട് ഒട്ടും അടുപ്പിക്കുകയില്ല. സത്യവിശ്വാസം സ്വീകരിക്കുകയും സല്‍ക്കര്‍മങ്ങള്‍ പ്രവര്‍ത്തിക്കുകയും ചെയ്തവരെയൊഴികെ. അവര്‍ക്ക് തങ്ങളുടെ കര്‍മങ്ങളുടെ ഇരട്ടി പ്രതിഫലം കിട്ടും. അവര്‍ അത്യുന്നത സൌധങ്ങളില്‍ നിര്‍ഭയരായി കഴിയുന്നവരായിരിക്കും.
\end{malayalam}}
\flushright{\begin{Arabic}
\quranayah[34][38]
\end{Arabic}}
\flushleft{\begin{malayalam}
നമ്മെ പരാജയപ്പെടുത്താനായി നമ്മുടെ വചനങ്ങളെ തള്ളിപ്പറയാന്‍ ശ്രമിക്കുന്നവരെ കൊടിയശിക്ഷക്കിരയാക്കും.
\end{malayalam}}
\flushright{\begin{Arabic}
\quranayah[34][39]
\end{Arabic}}
\flushleft{\begin{malayalam}
പറയുക: "എന്റെ നാഥന്‍ തന്റെ ദാസന്മാരില്‍ അവനിച്ഛിക്കുന്നവര്‍ക്ക് വിഭവങ്ങളില്‍ വിശാലത വരുത്തുന്നു. അവനിച്ഛിക്കുന്നവര്‍ക്ക് ഇടുക്കമുണ്ടാക്കുന്നു. നിങ്ങള്‍ സത്യമാര്‍ഗത്തില്‍ ചെലവഴിക്കുന്ന എന്തിനും അവന്‍ പകരം നല്‍കും. അന്നം നല്‍കുന്നവരില്‍ അത്യുത്തമനാണവന്‍."
\end{malayalam}}
\flushright{\begin{Arabic}
\quranayah[34][40]
\end{Arabic}}
\flushleft{\begin{malayalam}
അവന്‍ അവരെയെല്ലാം ഒരുമിച്ചുകൂട്ടുന്ന നാളിനെക്കുറിച്ചോര്‍ക്കുക. പിന്നെയവന്‍ മലക്കുകളോട് ചോദിക്കും: "നിങ്ങളെയാണോ ഇവര്‍ പൂജിച്ചിരുന്നത്?"
\end{malayalam}}
\flushright{\begin{Arabic}
\quranayah[34][41]
\end{Arabic}}
\flushleft{\begin{malayalam}
അവര്‍ പറയും: "നീയെത്ര പരിശുദ്ധന്‍! നീയാണ് ഞങ്ങളുടെ രക്ഷകന്‍. ഇവരാരുമല്ല. വാസ്തവത്തില്‍ ജിന്നുകളെയാണ് അവര്‍ പൂജിച്ചിരുന്നത്. അവരിലേറെ പേരും ജിന്നുകളില്‍ വിശ്വസിക്കുന്നവരുമായിരുന്നു."
\end{malayalam}}
\flushright{\begin{Arabic}
\quranayah[34][42]
\end{Arabic}}
\flushleft{\begin{malayalam}
അന്ന് നിങ്ങളിലാര്‍ക്കും പരസ്പരം ഗുണമോ ദോഷമോ ചെയ്യാനാവില്ല. അക്രമം കാണിച്ചവരോട് നാമിങ്ങനെ പറയും: "നിങ്ങള്‍ തള്ളിപ്പറഞ്ഞുകൊണ്ടിരുന്ന ആ നരകത്തീയിന്റെ ശിക്ഷ അനുഭവിച്ചുകൊള്ളുക."
\end{malayalam}}
\flushright{\begin{Arabic}
\quranayah[34][43]
\end{Arabic}}
\flushleft{\begin{malayalam}
നമ്മുടെ വചനങ്ങള്‍ വളരെ വ്യക്തമായി വായിച്ചുകേള്‍പ്പിച്ചാല്‍ അവര്‍ പറയും: "ഇവനൊരു മനുഷ്യന്‍ മാത്രമാണ്. നിങ്ങളുടെ പിതാക്കന്മാര്‍ പൂജിച്ചുകൊണ്ടിരുന്നതില്‍നിന്ന് നിങ്ങളെ തെറ്റിക്കാനാണ് ഇവനാഗ്രഹിക്കുന്നത്." അവര്‍ ഇത്രകൂടി പറയുന്നു: "ഈ ഖുര്‍ആന്‍ കെട്ടിച്ചമച്ചുണ്ടാക്കിയ കള്ളം മാത്രമാണ്." തങ്ങള്‍ക്കു സത്യം വന്നെത്തിയപ്പോള്‍ സത്യനിഷേധികള്‍ പറഞ്ഞു: "ഇതു വ്യക്തമായ മായാജാലം മാത്രമാണ്."
\end{malayalam}}
\flushright{\begin{Arabic}
\quranayah[34][44]
\end{Arabic}}
\flushleft{\begin{malayalam}
എന്നാല്‍ നാം അവര്‍ക്കു വായിച്ചുപഠിക്കാന്‍ വേദപുസ്തകങ്ങളൊന്നും നല്‍കിയിരുന്നില്ല. നിനക്കുമുമ്പ് നാം അവരിലേക്കൊരു മുന്നറിയിപ്പുകാരനെയും അയച്ചിരുന്നുമില്ല.
\end{malayalam}}
\flushright{\begin{Arabic}
\quranayah[34][45]
\end{Arabic}}
\flushleft{\begin{malayalam}
ഇവര്‍ക്കു മുമ്പുള്ളവരും സത്യത്തെ തള്ളിപ്പറഞ്ഞിട്ടുണ്ട്. നാമവര്‍ക്കു നല്‍കിയിരുന്ന സൌകര്യത്തിന്റെ പത്തിലൊരംശംപോലും ഇവര്‍ക്ക് കിട്ടിയിട്ടില്ല. എന്നിട്ടും അവര്‍ നമ്മുടെ ദൂതന്മാരെ തള്ളിപ്പറഞ്ഞു. അപ്പോള്‍ നമ്മുടെ ശിക്ഷ എവ്വിധമായിരുന്നു!
\end{malayalam}}
\flushright{\begin{Arabic}
\quranayah[34][46]
\end{Arabic}}
\flushleft{\begin{malayalam}
പറയുക: "ഞാന്‍ നിങ്ങളോട് ഒന്നേ ഉപദേശിക്കുന്നുള്ളൂ. അല്ലാഹുവിന്റെ മുമ്പില്‍ നിങ്ങള്‍ ഓരോരുത്തരായോ ഈരണ്ടുപേര്‍ വീതമോ എഴുന്നേറ്റുനില്‍ക്കുക. എന്നിട്ട് ചിന്തിക്കുക. അപ്പോള്‍ ബോധ്യമാകും. നിങ്ങളുടെ കൂട്ടുകാരന് ഭ്രാന്തില്ലെന്ന്. കഠിനമായ ശിക്ഷ നിങ്ങളെ ബാധിക്കുംമുമ്പെ നിങ്ങള്‍ക്ക് മുന്നറിയിപ്പ് നല്‍കുന്നവന്‍ മാത്രമാണ് അദ്ദേഹമെന്നും."
\end{malayalam}}
\flushright{\begin{Arabic}
\quranayah[34][47]
\end{Arabic}}
\flushleft{\begin{malayalam}
പറയുക: "ഞാന്‍ നിങ്ങളോട് വല്ല പ്രതിഫലവും ചോദിച്ചിട്ടുണ്ടെങ്കില്‍ അതു നിങ്ങള്‍ക്കുവേണ്ടിത്തന്നെയാണ്. എനിക്കുള്ള പ്രതിഫലം അല്ലാഹുവിങ്കല്‍ മാത്രമാണ്. അവന്‍ സകല സംഗതികള്‍ക്കും സാക്ഷിയാണല്ലോ."
\end{malayalam}}
\flushright{\begin{Arabic}
\quranayah[34][48]
\end{Arabic}}
\flushleft{\begin{malayalam}
പറയുക: "എന്റെ നാഥന്‍ എനിക്ക് സത്യമെത്തിച്ചുതരുന്നു. അവന്‍ അഭൌതിക കാര്യങ്ങളെല്ലാം അറിയുന്നവനാണ്."
\end{malayalam}}
\flushright{\begin{Arabic}
\quranayah[34][49]
\end{Arabic}}
\flushleft{\begin{malayalam}
പറയുക: "സത്യം വന്നെത്തിയിരിക്കുന്നു. ഇനി അസത്യം ഒന്നിനും തുടക്കം കുറിക്കുകയില്ല. അത് ഒന്നിനെയും പുനഃസ്ഥാപിക്കുകയുമില്ല."
\end{malayalam}}
\flushright{\begin{Arabic}
\quranayah[34][50]
\end{Arabic}}
\flushleft{\begin{malayalam}
പറയുക: "ഞാന്‍ വഴികേടിലാണെങ്കില്‍ എന്റെ വഴികേടിന്റെ വിപത്ത് എനിക്കുതന്നെയാണ്. ഞാന്‍ നേര്‍വഴിയിലാണെങ്കിലോ, അത് എന്റെ നാഥന്‍ എനിക്ക് ബോധനം നല്‍കിയതിനാലാണ്. തീര്‍ച്ചയായും അവന്‍ എല്ലാം കേള്‍ക്കുന്നവനാണ്. വളരെ സമീപസ്ഥനും."
\end{malayalam}}
\flushright{\begin{Arabic}
\quranayah[34][51]
\end{Arabic}}
\flushleft{\begin{malayalam}
അവര്‍ പരിഭ്രാന്തരായിത്തീരുന്ന സന്ദര്‍ഭം നീ കണ്ടിരുന്നെങ്കില്‍! അന്ന് അവര്‍ക്ക് ഒരു നിലക്കും രക്ഷപ്പെടാനാവില്ല. ഏറ്റവുമടുത്ത സ്ഥലത്തുവെച്ചുതന്നെ അവരെ പിടികൂടും.
\end{malayalam}}
\flushright{\begin{Arabic}
\quranayah[34][52]
\end{Arabic}}
\flushleft{\begin{malayalam}
അപ്പോഴവര്‍ പറയും: "ഞങ്ങള്‍ അദ്ദേഹത്തില്‍ വിശ്വസിച്ചിരിക്കുന്നു." എന്നാല്‍ കാര്യം ഏറെ ദൂരെയായിപ്പോയി. കൈവിട്ടകന്നത് എങ്ങനെ കൈവരിക്കാനാണ്?
\end{malayalam}}
\flushright{\begin{Arabic}
\quranayah[34][53]
\end{Arabic}}
\flushleft{\begin{malayalam}
നേരത്തെ അവരദ്ദേഹത്തെ തള്ളിപ്പറഞ്ഞതാണല്ലോ. കാര്യം നേരിട്ടറിയാതെ ഏറെ ദൂരെനിന്ന് അവര്‍ ദുരാരോപണം നടത്തുകയായിരുന്നു.
\end{malayalam}}
\flushright{\begin{Arabic}
\quranayah[34][54]
\end{Arabic}}
\flushleft{\begin{malayalam}
ഇപ്പോള്‍ ഇവര്‍ക്കും ഇവര്‍ ആഗ്രഹിക്കുന്നതിനുമിടയില്‍ തടസ്സം സൃഷ്ടിക്കുന്ന തിരശ്ശീല വീണുകഴിഞ്ഞിരിക്കുന്നു. ഇവരുടെ മുന്‍ഗാമികളായ കക്ഷികള്‍ക്കും സംഭവിച്ചത് ഇതുതന്നെ. തീര്‍ച്ചയായും അവര്‍ അവിശ്വാസമുളവാക്കുന്ന സംശയത്തിലായിരുന്നു.
\end{malayalam}}
\chapter{\textmalayalam{ഫാത്വിര്‍ ( സ്രഷ്ടാവ് )}}
\begin{Arabic}
\Huge{\centerline{\basmalah}}\end{Arabic}
\flushright{\begin{Arabic}
\quranayah[35][1]
\end{Arabic}}
\flushleft{\begin{malayalam}
സര്‍വ സ്തുതിയും അല്ലാഹുവിന്. ആകാശഭൂമികളെ സൃഷ്ടിച്ചവനാണവന്‍. രണ്ടും മൂന്നും നാലും ചിറകുകളുള്ള മലക്കുകളെ ദൂതന്മാരായി നിയോഗിച്ചവനും. സൃഷ്ടിയില്‍ താനിച്ഛിക്കുന്നത് അവന്‍ വര്‍ധിപ്പിക്കുന്നു. അല്ലാഹു എല്ലാ കാര്യങ്ങള്‍ക്കും കഴിവുറ്റവനാണ്.
\end{malayalam}}
\flushright{\begin{Arabic}
\quranayah[35][2]
\end{Arabic}}
\flushleft{\begin{malayalam}
അല്ലാഹു മനുഷ്യര്‍ക്ക് അനുഗ്രഹത്തിന്റെ വല്ല കവാടവും തുറന്നു കൊടുക്കുകയാണെങ്കില്‍ അത് തടയാന്‍ ആര്‍ക്കും സാധ്യമല്ല. അവന്‍ എന്തെങ്കിലും തടഞ്ഞുവെക്കുകയാണെങ്കില്‍ അതു വിട്ടുകൊടുക്കാനും ആര്‍ക്കുമാവില്ല. അവന്‍ പ്രതാപിയും യുക്തിമാനുമാണ്.
\end{malayalam}}
\flushright{\begin{Arabic}
\quranayah[35][3]
\end{Arabic}}
\flushleft{\begin{malayalam}
മനുഷ്യരേ, അല്ലാഹു നിങ്ങള്‍ക്കേകിയ അനുഗ്രഹങ്ങള്‍ ഓര്‍ക്കുക. ആകാശഭൂമികളില്‍ നിന്ന് നിങ്ങള്‍ക്ക് അന്നം നല്‍കുന്ന അല്ലാഹുവല്ലാത്ത വല്ല സ്രഷ്ടാവുമുണ്ടോ? അവനല്ലാതെ ദൈവമില്ല. പിന്നെയെങ്ങനെയാണ് നിങ്ങള്‍ വഴിതെറ്റിപ്പോകുന്നത്?
\end{malayalam}}
\flushright{\begin{Arabic}
\quranayah[35][4]
\end{Arabic}}
\flushleft{\begin{malayalam}
അവര്‍ നിന്നെ തള്ളിപ്പറയുന്നുവെങ്കില്‍ അറിയുക: നിനക്കു മുമ്പും ധാരാളം ദൈവദൂതന്മാരെ കള്ളമാക്കി തള്ളിയിട്ടുണ്ട്. കാര്യങ്ങളൊക്കെയും മടക്കപ്പെടുക അല്ലാഹുവിങ്കലേക്കാണ്.
\end{malayalam}}
\flushright{\begin{Arabic}
\quranayah[35][5]
\end{Arabic}}
\flushleft{\begin{malayalam}
മനുഷ്യരേ, അല്ലാഹുവിന്റെ വാഗ്ദാനം സത്യമാണ്. അതിനാല്‍ ഇഹലോക ജീവിതം നിങ്ങളെ ചതിക്കാതിരിക്കട്ടെ. പെരുംവഞ്ചകനായ ചെകുത്താനും അല്ലാഹുവിന്റെ കാര്യത്തില്‍ നിങ്ങളെ ചതിക്കാതിരിക്കട്ടെ.
\end{malayalam}}
\flushright{\begin{Arabic}
\quranayah[35][6]
\end{Arabic}}
\flushleft{\begin{malayalam}
തീര്‍ച്ചയായും ചെകുത്താന്‍ നിങ്ങളുടെ ശത്രുവാണ്. അതിനാല്‍ നിങ്ങളവനെ ശത്രുവായിത്തന്നെ കാണുക. അവന്‍ തന്റെ സംഘത്തെ ക്ഷണിക്കുന്നത് അവരെ നരകാവകാശികളാക്കിത്തീര്‍ക്കാനാണ്.
\end{malayalam}}
\flushright{\begin{Arabic}
\quranayah[35][7]
\end{Arabic}}
\flushleft{\begin{malayalam}
സത്യത്തെ തള്ളിപ്പറഞ്ഞവര്‍ക്ക് കഠിനമായ ശിക്ഷയുണ്ട്. സത്യവിശ്വാസം സ്വീകരിക്കുകയും സല്‍ക്കര്‍മങ്ങള്‍ പ്രവര്‍ത്തിക്കുകയും ചെയ്തവരോ അവര്‍ക്ക് പാപമോചനവും മഹത്തായ പ്രതിഫലവുമുണ്ട്.
\end{malayalam}}
\flushright{\begin{Arabic}
\quranayah[35][8]
\end{Arabic}}
\flushleft{\begin{malayalam}
എന്നാല്‍ തന്റെ ചീത്തപ്രവൃത്തി ചേതോഹരമായി തോന്നുകയും അങ്ങനെ അതു നല്ലതായി കാണുകയും ചെയ്തവന്റെ സ്ഥിതിയോ? സംശയമില്ല; അല്ലാഹു അവനിച്ഛിക്കുന്നവരെ വഴികേടിലാക്കുന്നു. അവനിച്ഛിക്കുന്നവരെ നേര്‍വഴിയിലുമാക്കുന്നു. അതിനാല്‍ അവരെക്കുറിച്ചോര്‍ത്ത് കൊടും ദുഃഖത്താല്‍ നീ നിന്റെ ജീവന്‍ കളയേണ്ടതില്ല. അവര്‍ പ്രവര്‍ത്തിക്കുന്നതിനെപ്പറ്റി നന്നായറിയുന്നവനാണ് അല്ലാഹു.
\end{malayalam}}
\flushright{\begin{Arabic}
\quranayah[35][9]
\end{Arabic}}
\flushleft{\begin{malayalam}
കാറ്റുകളെ അയച്ചുകൊണ്ടിരിക്കുന്നവന്‍ അല്ലാഹുവാണ്. അങ്ങനെ അത് മേഘത്തെ തള്ളിനീക്കുന്നു. പിന്നീട് നാമതിനെ മൃതമായ നാട്ടിലേക്ക് നയിക്കുന്നു. അതുവഴി നാം ഭൂമിയെ അതിന്റെ നിര്‍ജീവാവസ്ഥക്കുശേഷം ജീവനുള്ളതാക്കുന്നു. അവ്വിധം തന്നെയാണ് ഉയിര്‍ത്തെഴുന്നേല്‍പും.
\end{malayalam}}
\flushright{\begin{Arabic}
\quranayah[35][10]
\end{Arabic}}
\flushleft{\begin{malayalam}
ആരെങ്കിലും അന്തസ്സ് ആഗ്രഹിക്കുന്നുവെങ്കില്‍ അറിയുക: അന്തസ്സൊക്കെയും അല്ലാഹുവിന്റെ അധീനതയിലാണ്. നല്ല വാക്കുകള്‍ കയറിപ്പോകുന്നത് അവങ്കലേക്കാണ്. സല്‍പ്രവൃത്തികളെ അവന്‍ സമുന്നതമാക്കുന്നു. എന്നാല്‍ കുടിലമായ കുതന്ത്രങ്ങള്‍ കാണിക്കുന്നവര്‍ക്ക് കഠിനശിക്ഷയുണ്ട്. അവരുടെ കുതന്ത്രം തകരുകതന്നെ ചെയ്യും.
\end{malayalam}}
\flushright{\begin{Arabic}
\quranayah[35][11]
\end{Arabic}}
\flushleft{\begin{malayalam}
അല്ലാഹു നിങ്ങളെ മണ്ണില്‍നിന്ന് സൃഷ്ടിച്ചു. പിന്നെ ബീജകണത്തില്‍നിന്നും. അതിനുശേഷം അവന്‍ നിങ്ങളെ ഇണകളാക്കി. അവന്റെ അറിവോടെയല്ലാതെ ഒരു സ്ത്രീയും ഗര്‍ഭം ചുമക്കുന്നില്ല. പ്രസവിക്കുന്നുമില്ല. ഒരു ഗ്രന്ഥത്തില്‍ രേഖപ്പെടുത്തിവെച്ചിട്ടല്ലാതെ ഒരു വൃദ്ധനും ആയുസ്സ് നീട്ടിക്കൊടുക്കുന്നില്ല; ആരുടെയും ആയുസ്സില്‍ കുറവു വരുത്തുന്നുമില്ല. അല്ലാഹുവിന് ഇതൊക്കെയും വളരെ എളുപ്പമാണ്.
\end{malayalam}}
\flushright{\begin{Arabic}
\quranayah[35][12]
\end{Arabic}}
\flushleft{\begin{malayalam}
രണ്ടു ജലാശയങ്ങള്‍; അവയൊരിക്കലും ഒരേപോലെയല്ല. ഒന്ന് ശുദ്ധവും ദാഹമകറ്റുന്നതും കുടിക്കാന്‍ രുചികരവുമാണ്. മറ്റൊന്ന് ചവര്‍പ്പുള്ള ഉപ്പുവെള്ളവും. എന്നാല്‍ രണ്ടില്‍ നിന്നും നിങ്ങള്‍ക്കു തിന്നാന്‍ പുതുമാംസം ലഭിക്കുന്നു. നിങ്ങള്‍ക്ക് അണിയാനുള്ള ആഭരണങ്ങളും നിങ്ങളവയില്‍നിന്ന് പുറത്തെടുക്കുന്നു. അവ പിളര്‍ന്ന് കപ്പലുകള്‍ സഞ്ചരിക്കുന്നത് നിങ്ങള്‍ക്കു കാണാം. അതിലൂടെ നിങ്ങള്‍ അല്ലാഹുവിന്റെ അനുഗ്രഹം തേടാനാണത്. നിങ്ങള്‍ നന്ദിയുള്ളവരാകാനും.
\end{malayalam}}
\flushright{\begin{Arabic}
\quranayah[35][13]
\end{Arabic}}
\flushleft{\begin{malayalam}
അവന്‍ രാവിനെ പകലില്‍ കടത്തിവിടുന്നു. പകലിനെ രാവില്‍ പ്രവേശിപ്പിക്കുന്നു. സൂര്യചന്ദ്രന്മാര്‍ അവന്റെ അധീനതയിലാണ്. അവയെല്ലാം ഒരു നിശ്ചിത അവധിവരെ ചരിക്കുന്നു. അങ്ങനെയെല്ലാമുള്ള അല്ലാഹുവാണ് നിങ്ങളുടെ നാഥന്‍. ആധിപത്യം അവന്റേതു മാത്രമാണ്. അവനെക്കൂടാതെ നിങ്ങള്‍ വിളിച്ചുപ്രാര്‍ഥിക്കുന്നവരോ, ഈത്തപ്പഴക്കുരുവിന്റെ പാടപോലും അവരുടെ ഉടമസ്ഥതയിലില്ല.
\end{malayalam}}
\flushright{\begin{Arabic}
\quranayah[35][14]
\end{Arabic}}
\flushleft{\begin{malayalam}
നിങ്ങളവരെ വിളിച്ചാല്‍ നിങ്ങളുടെ വിളി അവര്‍ കേള്‍ക്കുക പോലുമില്ല. അഥവാ കേട്ടാലും നിങ്ങള്‍ക്ക് ഉത്തരമേകാന്‍ അവര്‍ക്കാവില്ല. ഉയിര്‍ത്തെഴുന്നേല്‍പുനാളില്‍ നിങ്ങളവരെ പങ്കാളികളാക്കിയതിനെ അവര്‍ നിഷേധിക്കും. അല്ലാഹുവല്ലാതെ ഇങ്ങനെ സൂക്ഷ്മജ്ഞാനിയെപ്പോലെ നിങ്ങള്‍ക്ക് ഇത്തരം വിവരംതരുന്ന ആരുമില്ല.
\end{malayalam}}
\flushright{\begin{Arabic}
\quranayah[35][15]
\end{Arabic}}
\flushleft{\begin{malayalam}
മനുഷ്യരേ, നിങ്ങള്‍ അല്ലാഹുവിന്റെ ആശ്രിതരാണ്. അല്ലാഹുവോ സ്വയംപര്യാപ്തനും സ്തുത്യര്‍ഹനും.
\end{malayalam}}
\flushright{\begin{Arabic}
\quranayah[35][16]
\end{Arabic}}
\flushleft{\begin{malayalam}
അവനിച്ഛിക്കുന്നുവെങ്കില്‍ നിങ്ങളെ തൂത്തുമാറ്റി പകരം പുതിയൊരു സൃഷ്ടിയെ അവന്‍ കൊണ്ടുവരും.
\end{malayalam}}
\flushright{\begin{Arabic}
\quranayah[35][17]
\end{Arabic}}
\flushleft{\begin{malayalam}
അത് അല്ലാഹുവിന് ഒട്ടും പ്രയാസകരമല്ല.
\end{malayalam}}
\flushright{\begin{Arabic}
\quranayah[35][18]
\end{Arabic}}
\flushleft{\begin{malayalam}
പാപഭാരം പേറുന്ന ആരും അപരന്റെ ഭാരം വഹിക്കുകയില്ല. ഭാരത്താല്‍ ഞെരുങ്ങുന്നവന്‍ തന്റെ ചുമട് വഹിക്കാനാരെയെങ്കിലും വിളിച്ചാല്‍ അതില്‍നിന്ന് ഒന്നുംതന്നെ ആരും ഏറ്റെടുക്കുകയില്ല. അതാവശ്യപ്പെടുന്നത് അടുത്ത ബന്ധുവായാല്‍പ്പോലും. നിന്റെ മുന്നറിയിപ്പ് ഉപകരിക്കുക തങ്ങളുടെ നാഥനെ നേരില്‍ കാണാതെതന്നെ ഭയപ്പെടുകയും നമസ്കാരം നിഷ്ഠയോടെ നിര്‍വഹിക്കുകയും ചെയ്യുന്നവര്‍ക്കു മാത്രമാണ്. വല്ലവനും വിശുദ്ധി വരിക്കുന്നുവെങ്കില്‍ അത് തന്റെ സ്വന്തം നന്മക്കുവേണ്ടി തന്നെയാണ്. എല്ലാവരുടെയും മടക്കം അല്ലാഹുവിങ്കലേക്കാണ്.
\end{malayalam}}
\flushright{\begin{Arabic}
\quranayah[35][19]
\end{Arabic}}
\flushleft{\begin{malayalam}
കാഴ്ചയുള്ളവനും ഇല്ലാത്തവനും തുല്യരല്ല.
\end{malayalam}}
\flushright{\begin{Arabic}
\quranayah[35][20]
\end{Arabic}}
\flushleft{\begin{malayalam}
ഇരുളും വെളിച്ചവും സമമല്ല.
\end{malayalam}}
\flushright{\begin{Arabic}
\quranayah[35][21]
\end{Arabic}}
\flushleft{\begin{malayalam}
തണലും വെയിലും ഒരുപോലെയല്ല.
\end{malayalam}}
\flushright{\begin{Arabic}
\quranayah[35][22]
\end{Arabic}}
\flushleft{\begin{malayalam}
ജീവിച്ചിരിക്കുന്നവരും മരിച്ചവരും സമമാവുകയില്ല. തീര്‍ച്ചയായും അല്ലാഹു അവനിച്ഛിക്കുന്നവരെ കേള്‍പ്പിക്കുന്നു. കുഴിമാടങ്ങളില്‍ കിടക്കുന്നവരെ കേള്‍പ്പിക്കാന്‍ നിനക്കാവില്ല.
\end{malayalam}}
\flushright{\begin{Arabic}
\quranayah[35][23]
\end{Arabic}}
\flushleft{\begin{malayalam}
നീയൊരു മുന്നറിയിപ്പുകാരന്‍ മാത്രം.
\end{malayalam}}
\flushright{\begin{Arabic}
\quranayah[35][24]
\end{Arabic}}
\flushleft{\begin{malayalam}
നാം നിന്നെ അയച്ചത് സത്യസന്ദേശവുമായാണ്. ശുഭവാര്‍ത്ത അറിയിക്കുന്നവനും മുന്നറിയിപ്പുകാരനുമായാണ്. മുന്നറിയിപ്പുകാരന്‍ വന്നുപോകാത്ത ഒരു സമുദായവും ഇല്ല.
\end{malayalam}}
\flushright{\begin{Arabic}
\quranayah[35][25]
\end{Arabic}}
\flushleft{\begin{malayalam}
ഈ ജനം നിന്നെ തള്ളിപ്പറയുന്നുവെങ്കില്‍ അവര്‍ക്കു മുമ്പുള്ളവരും അവ്വിധം തള്ളിപ്പറഞ്ഞിട്ടുണ്ട്. വ്യക്തമായ തെളിവുകളും പ്രമാണങ്ങളും വെളിച്ചം നല്‍കുന്ന വേദപുസ്തകവുമായി അവര്‍ക്കുള്ള ദൂതന്മാര്‍ അവരുടെയടുത്ത് ചെന്നിട്ടുണ്ടായിരുന്നു.
\end{malayalam}}
\flushright{\begin{Arabic}
\quranayah[35][26]
\end{Arabic}}
\flushleft{\begin{malayalam}
പിന്നീട് സത്യത്തെ തള്ളിപ്പറഞ്ഞവരെ നാം പിടികൂടി. അപ്പോഴെന്റെ ശിക്ഷ എവ്വിധമായിരുന്നു!
\end{malayalam}}
\flushright{\begin{Arabic}
\quranayah[35][27]
\end{Arabic}}
\flushleft{\begin{malayalam}
അല്ലാഹു മാനത്തുനിന്ന് മഴ വീഴ്ത്തുന്നത് നീ കാണുന്നില്ലേ? അതുവഴി നാനാ നിറമുള്ള പലയിനം പഴങ്ങള്‍ നാം ഉല്‍പ്പാദിപ്പിക്കുന്നു. പര്‍വതങ്ങളിലുമുണ്ട് വെളുത്തതും ചുവന്നതുമായ വ്യത്യസ്ത വര്‍ണമുള്ള വഴികള്‍. കറുത്തിരുണ്ടതുമുണ്ട്.
\end{malayalam}}
\flushright{\begin{Arabic}
\quranayah[35][28]
\end{Arabic}}
\flushleft{\begin{malayalam}
മനുഷ്യരിലും മൃഗങ്ങളിലും കന്നുകാലികളിലും വ്യത്യസ്ത വര്‍ണമുള്ളവയുണ്ട്. തീര്‍ച്ചയായും ദൈവദാസന്മാരില്‍ അവനെ ഭയപ്പെടുന്നത് അറിവുള്ളവര്‍ മാത്രമാണ്. സംശയമില്ല; അല്ലാഹു പ്രതാപിയാണ്. ഏറെ പൊറുക്കുന്നവനും.
\end{malayalam}}
\flushright{\begin{Arabic}
\quranayah[35][29]
\end{Arabic}}
\flushleft{\begin{malayalam}
ദൈവികഗ്രന്ഥം പാരായണം ചെയ്യുകയും നമസ്കാരം നിഷ്ഠയോടെ നിര്‍വഹിക്കുകയും നാം നല്‍കിയതില്‍നിന്ന് രഹസ്യമായും പരസ്യമായും ചെലവഴിക്കുകയും ചെയ്യുന്നവര്‍ തീര്‍ച്ചയായും നഷ്ടം പറ്റാത്ത കച്ചവടം കൊതിക്കുന്നവരാണ്.
\end{malayalam}}
\flushright{\begin{Arabic}
\quranayah[35][30]
\end{Arabic}}
\flushleft{\begin{malayalam}
അല്ലാഹു അവര്‍ക്ക് അവരര്‍ഹിക്കുന്ന പ്രതിഫലം പൂര്‍ണമായും നല്‍കാനാണിത്. തന്റെ അനുഗ്രഹത്തില്‍നിന്ന് കൂടുതലായി കൊടുക്കാനും. തീര്‍ച്ചയായും അവന്‍ ഏറെ പൊറുക്കുന്നവനാണ്. വളരെ നന്ദിയുള്ളവനും.
\end{malayalam}}
\flushright{\begin{Arabic}
\quranayah[35][31]
\end{Arabic}}
\flushleft{\begin{malayalam}
നാം നിനക്കു ബോധനമായി നല്‍കിയ വേദപുസ്തകം സത്യമാണ്. അതിനു മുമ്പുള്ള വേദങ്ങളെ ശരിവെക്കുന്നതും. നിശ്ചയം അല്ലാഹു തന്റെ ദാസന്മാരെ സംബന്ധിച്ച് സൂക്ഷ്മമായി അറിയുന്നവനും എല്ലാം കാണുന്നവനുമാണ്.
\end{malayalam}}
\flushright{\begin{Arabic}
\quranayah[35][32]
\end{Arabic}}
\flushleft{\begin{malayalam}
പിന്നീട് നമ്മുടെ ദാസന്മാരില്‍ നിന്ന് പ്രത്യേകം തെരഞ്ഞെടുത്തവരെ നാം ഈ വേദപുസ്തകത്തിന്റെ അവകാശികളാക്കി. അവരില്‍ തങ്ങളോടുതന്നെ അതിക്രമം കാട്ടുന്നവരുണ്ട്. മധ്യനിലപാട് പുലര്‍ത്തുന്നവരുണ്ട്. ദൈവഹിതത്തിനൊത്ത് നന്മകളില്‍ മുന്നേറുന്നവരും അവരിലുണ്ട്. ഇതു തന്നെയാണ് അതിമഹത്തായ അനുഗ്രഹം.
\end{malayalam}}
\flushright{\begin{Arabic}
\quranayah[35][33]
\end{Arabic}}
\flushleft{\begin{malayalam}
അവര്‍ നിത്യജീവിതത്തിനുള്ള സ്വര്‍ഗീയാരാമങ്ങളില്‍ പ്രവേശിക്കും. അവരവിടെ സ്വര്‍ണവളകളും മുത്തും അണിയിക്കപ്പെടും. അവിടെ അവര്‍ ധരിക്കുക പട്ടുവസ്ത്രമായിരിക്കും.
\end{malayalam}}
\flushright{\begin{Arabic}
\quranayah[35][34]
\end{Arabic}}
\flushleft{\begin{malayalam}
അവര്‍ പറയും: "ഞങ്ങളില്‍ നിന്ന് ദുഃഖമകറ്റിയ അല്ലാഹുവിനു സ്തുതി. ഞങ്ങളുടെ നാഥന്‍ ഏറെ പൊറുക്കുന്നവനാണ്; വളരെ നന്ദിയുള്ളവനും.
\end{malayalam}}
\flushright{\begin{Arabic}
\quranayah[35][35]
\end{Arabic}}
\flushleft{\begin{malayalam}
"തന്റെ അനുഗ്രഹത്താല്‍ നമ്മെ നിത്യവാസത്തിനുള്ള വസതിയില്‍ കുടിയിരുത്തിയവനാണവന്‍. ഇവിടെ ഇനി നമ്മെ ഒരുവിധ പ്രയാസവും ബാധിക്കുകയില്ല. നേരിയ ക്ഷീണംപോലും നമ്മെ സ്പര്‍ശിക്കില്ല."
\end{malayalam}}
\flushright{\begin{Arabic}
\quranayah[35][36]
\end{Arabic}}
\flushleft{\begin{malayalam}
സത്യത്തെ തള്ളിപ്പറഞ്ഞവര്‍ക്കുള്ളതാണ് നരകത്തീ. അവര്‍ക്ക് അവിടെ മരണമില്ല. അതുണ്ടായിരുന്നെങ്കില്‍ മരിച്ചു രക്ഷപ്പെടാമായിരുന്നു. നരകശിക്ഷയില്‍നിന്ന് അവര്‍ക്കൊട്ടും ഇളവു കിട്ടുകയില്ല. അവ്വിധമാണ് നാം എല്ലാ നന്ദികെട്ടവര്‍ക്കും പ്രതിഫലം നല്‍കുന്നത്.
\end{malayalam}}
\flushright{\begin{Arabic}
\quranayah[35][37]
\end{Arabic}}
\flushleft{\begin{malayalam}
അവരവിടെ വച്ച് ഇങ്ങനെ അലമുറയിടും: "ഞങ്ങളുടെ നാഥാ, ഞങ്ങളെയൊന്ന് പുറത്തയക്കേണമേ. ഞങ്ങള്‍ മുമ്പ് ചെയ്തിരുന്നതില്‍ നിന്ന് വ്യത്യസ്തമായി നല്ല കാര്യങ്ങള്‍ ചെയ്തുകൊള്ളാം." അല്ലാഹു പറയും: "പാഠമുള്‍ക്കൊള്ളുന്നവര്‍ക്ക് അതുള്‍ക്കൊള്ളാന്‍ മാത്രം നാം ആയുസ്സ് നല്‍കിയിരുന്നില്ലേ? നിങ്ങളുടെയടുത്ത് മുന്നറിയിപ്പുകാരന്‍ വന്നിട്ടുമുണ്ടായിരുന്നില്ലേ? അതിനാലിനി അനുഭവിച്ചുകൊള്ളുക. അക്രമികള്‍ക്കിവിടെ സഹായിയായി ആരുമില്ല."
\end{malayalam}}
\flushright{\begin{Arabic}
\quranayah[35][38]
\end{Arabic}}
\flushleft{\begin{malayalam}
തീര്‍ച്ചയായും അല്ലാഹു ആകാശഭൂമികളില്‍ ഒളിഞ്ഞു കിടക്കുന്നവയൊക്കെയും അറിയുന്നവനാണ്. സംശയമില്ല, മനസ്സുകള്‍ ഒളിപ്പിച്ചുവെക്കുന്നതെല്ലാം നന്നായറിയുന്നവനാണവന്‍.
\end{malayalam}}
\flushright{\begin{Arabic}
\quranayah[35][39]
\end{Arabic}}
\flushleft{\begin{malayalam}
നിങ്ങളെ ഭൂമിയില്‍ പ്രതിനിധികളാക്കിയത് അവനാണ്. ആരെങ്കിലും അവിശ്വസിക്കുന്നുവെങ്കില്‍ ആ അവിശ്വാസത്തിന്റെ ദോഷം അവനു തന്നെയാണ്. സത്യനിഷേധികള്‍ക്ക് അവരുടെ സത്യനിഷേധം തങ്ങളുടെ നാഥന്റെയടുത്ത് അവന്റെ കോപമല്ലാതൊന്നും വര്‍ധിപ്പിക്കുകയില്ല. സത്യനിഷേധികള്‍ക്ക് അവരുടെ സത്യനിഷേധം നഷ്ടമല്ലാതൊന്നും പെരുപ്പിക്കുകയില്ല.
\end{malayalam}}
\flushright{\begin{Arabic}
\quranayah[35][40]
\end{Arabic}}
\flushleft{\begin{malayalam}
പറയുക: "അല്ലാഹുവെക്കൂടാതെ നിങ്ങള്‍ വിളിച്ചുപ്രാര്‍ഥിക്കുന്ന നിങ്ങളുടെ പങ്കാളികളെപ്പറ്റി നിങ്ങളെപ്പോഴെങ്കിലും ചിന്തിച്ചു നോക്കിയിട്ടുണ്ടോ? ഭൂമിയില്‍ എന്താണ് അവര്‍ സൃഷ്ടിച്ചതെന്ന് എനിക്കൊന്നു കാണിച്ചുതരൂ. അല്ലെങ്കില്‍ ആകാശങ്ങളിലവര്‍ക്ക് വല്ല പങ്കുമുണ്ടോ? അതല്ലെങ്കില്‍ നാം അവര്‍ക്കെന്തെങ്കിലും പ്രമാണം നല്‍കിയിട്ടുണ്ടോ? അതില്‍നിന്നുള്ള തെളിവനുസരിച്ചാണോ അവര്‍ നിലകൊള്ളുന്നത്?" എന്നാല്‍ അതൊന്നുമല്ല; അക്രമികള്‍ അന്യോന്യം വാഗ്ദാനം ചെയ്തുകൊണ്ടിരിക്കുന്നത് വെറും വഞ്ചന മാത്രമാണ്.
\end{malayalam}}
\flushright{\begin{Arabic}
\quranayah[35][41]
\end{Arabic}}
\flushleft{\begin{malayalam}
അല്ലാഹു ആകാശഭൂമികളെ നീങ്ങിപ്പോകാതെ പിടിച്ചുനിര്‍ത്തുന്നു. അഥവാ, അവ നീങ്ങിപ്പോവുകയാണെങ്കില്‍ അവനെക്കൂടാതെ അവയെ പിടിച്ചുനിര്‍ത്തുന്ന ആരുമില്ല. അവന്‍ സഹനശാലിയും ഏറെ പൊറുക്കുന്നവനുമാണ്.
\end{malayalam}}
\flushright{\begin{Arabic}
\quranayah[35][42]
\end{Arabic}}
\flushleft{\begin{malayalam}
ബഹുദൈവവിശ്വാസികള്‍ തങ്ങള്‍ക്കാവും വിധം അല്ലാഹുവിന്റെ പേരില്‍ ആണയിട്ടു പറഞ്ഞു, തങ്ങള്‍ക്ക് ഒരു മുന്നറിയിപ്പുകാരന്‍ വന്നെത്തിയാല്‍ തങ്ങള്‍ മറ്റേതു സമുദായത്തെക്കാളും സന്മാര്‍ഗം സ്വീകരിക്കുന്നവരാകുമെന്ന്. എന്നാല്‍ മുന്നറിയിപ്പുകാരന്‍ അവരുടെ അടുത്തു ചെന്നപ്പോള്‍ അത് അവരില്‍ വെറുപ്പ് മാത്രമേ വര്‍ധിപ്പിച്ചുള്ളൂ.
\end{malayalam}}
\flushright{\begin{Arabic}
\quranayah[35][43]
\end{Arabic}}
\flushleft{\begin{malayalam}
അവര്‍ ഭൂമിയില്‍ അഹങ്കരിച്ചു നടന്നതിനാലാണിത്. ഹീനതന്ത്രങ്ങളിലേര്‍പ്പെട്ടതിനാലും. കുടിലതന്ത്രം അതു പയറ്റുന്നവരെത്തന്നെയാണ് ബാധിക്കുക. അതിനാല്‍ മുന്‍ഗാമികളുടെ കാര്യത്തിലുണ്ടായ ദുരന്താനുഭവങ്ങളല്ലാതെ മറ്റെന്താണ് അവര്‍ക്ക് കാത്തിരിക്കാനുള്ളത്? അല്ലാഹുവിന്റെ നടപടിക്രമത്തിലൊരു മാറ്റവും നിനക്കു കാണാനാവില്ല. അല്ലാഹുവിന്റെ നടപടിക്രമത്തില്‍ വ്യത്യാസം വരുത്തുന്ന ഒന്നും നിനക്കു കണ്ടെത്താനാവില്ല.
\end{malayalam}}
\flushright{\begin{Arabic}
\quranayah[35][44]
\end{Arabic}}
\flushleft{\begin{malayalam}
ഇക്കൂട്ടര്‍ ഭൂമിയിലൂടെ സഞ്ചരിച്ച് തങ്ങളുടെ പൂര്‍വികരുടെ പര്യവസാനം എവ്വിധമായിരുന്നുവെന്ന് നോക്കിക്കാണുന്നില്ലേ? അവര്‍ ഇവരെക്കാള്‍ എത്രയോ കരുത്തരായിരുന്നു. അറിയുക: അല്ലാഹുവെ തോല്‍പിക്കുന്ന ഒന്നുമില്ല. ആകാശത്തുമില്ല. ഭൂമിയിലുമില്ല. തീര്‍ച്ചയായും അവന്‍ സകലം അറിയുന്നവനാണ്. എല്ലാറ്റിനും കഴിവുറ്റവനും.
\end{malayalam}}
\flushright{\begin{Arabic}
\quranayah[35][45]
\end{Arabic}}
\flushleft{\begin{malayalam}
അല്ലാഹു മനുഷ്യരെ, അവര്‍ ചെയ്തുകൂട്ടിയതിന്റെ പേരില്‍ പിടികൂടി ശിക്ഷിക്കുകയാണെങ്കില്‍ ഭൂമുഖത്ത് ഒരു ജന്തുവെയും അവന്‍ ബാക്കിവെക്കുമായിരുന്നില്ല. എന്നാല്‍ ഒരു നിശ്ചിത അവധിവരെ അവനവര്‍ക്ക് അവസരം നീട്ടിക്കൊടുക്കുന്നു. അങ്ങനെ അവരുടെ കാലാവധി വന്നെത്തിയാല്‍ തീര്‍ച്ചയായും അല്ലാഹു തന്റെ ദാസന്മാരെ കണ്ടറിയുന്നതാണ്.
\end{malayalam}}
\chapter{\textmalayalam{യാസീന്‍}}
\begin{Arabic}
\Huge{\centerline{\basmalah}}\end{Arabic}
\flushright{\begin{Arabic}
\quranayah[36][1]
\end{Arabic}}
\flushleft{\begin{malayalam}
യാസീന്‍.
\end{malayalam}}
\flushright{\begin{Arabic}
\quranayah[36][2]
\end{Arabic}}
\flushleft{\begin{malayalam}
തത്ത്വങ്ങള്‍ നിറഞ്ഞുനില്‍ക്കുന്ന ഖുര്‍ആന്‍ തന്നെ സത്യം.
\end{malayalam}}
\flushright{\begin{Arabic}
\quranayah[36][3]
\end{Arabic}}
\flushleft{\begin{malayalam}
തീര്‍ച്ചയായും നീ ദൈവദൂതന്മാരില്‍ ഒരുവനാകുന്നു.
\end{malayalam}}
\flushright{\begin{Arabic}
\quranayah[36][4]
\end{Arabic}}
\flushleft{\begin{malayalam}
ഉറപ്പായും നീ നേര്‍വഴിയിലാണ്.
\end{malayalam}}
\flushright{\begin{Arabic}
\quranayah[36][5]
\end{Arabic}}
\flushleft{\begin{malayalam}
പ്രതാപിയും പരമകാരുണികനുമായവന്‍ ഇറക്കിയതാണ് ഈ ഖുര്‍ആന്‍.
\end{malayalam}}
\flushright{\begin{Arabic}
\quranayah[36][6]
\end{Arabic}}
\flushleft{\begin{malayalam}
ഒരു ജനതക്കു മുന്നറിയിപ്പു നല്‍കാനാണിത്. അവരുടെ പിതാക്കള്‍ക്ക് ഇതുപോലൊരു മുന്നറിയിപ്പുണ്ടായിട്ടില്ല. അതിനാലവര്‍ ബോധമില്ലാത്തവരാണ്.
\end{malayalam}}
\flushright{\begin{Arabic}
\quranayah[36][7]
\end{Arabic}}
\flushleft{\begin{malayalam}
അവരിലേറെ പേരും ശിക്ഷാവിധിക്കര്‍ഹരായിരിക്കുന്നു. അതിനാല്‍ അവരിതു വിശ്വസിക്കുകയില്ല.
\end{malayalam}}
\flushright{\begin{Arabic}
\quranayah[36][8]
\end{Arabic}}
\flushleft{\begin{malayalam}
അവരുടെ കണ്ഠങ്ങളില്‍ നാം കൂച്ചുവിലങ്ങണിയിച്ചിരിക്കുന്നു. അതവരുടെ താടിയെല്ലുകള്‍ വരെയുണ്ട്. അതിനാലവര്‍ക്ക് തല പൊക്കിപ്പിടിച്ചേ നില്‍ക്കാനാവൂ.
\end{malayalam}}
\flushright{\begin{Arabic}
\quranayah[36][9]
\end{Arabic}}
\flushleft{\begin{malayalam}
നാം അവരുടെ മുന്നിലൊരു മതില്‍ക്കെട്ടുയര്‍ത്തിയിട്ടുണ്ട്. അവരുടെ പിന്നിലും മതില്‍ക്കെട്ടുണ്ട്. അങ്ങനെ നാമവരെ മൂടിക്കളഞ്ഞു. അതിനാലവര്‍ക്കൊന്നും കാണാനാവില്ല.
\end{malayalam}}
\flushright{\begin{Arabic}
\quranayah[36][10]
\end{Arabic}}
\flushleft{\begin{malayalam}
നീ അവര്‍ക്കു താക്കീതു നല്‍കുന്നതും നല്‍കാതിരിക്കുന്നതും ഒരുപോലെയാണ്. എന്തായാലും അവര്‍ വിശ്വസിക്കുകയില്ല.
\end{malayalam}}
\flushright{\begin{Arabic}
\quranayah[36][11]
\end{Arabic}}
\flushleft{\begin{malayalam}
നിന്റെ താക്കീതുപകരിക്കുക ഉദ്ബോധനം പിന്‍പറ്റുകയും ദയാപരനായ അല്ലാഹുവെ കാണാതെ തന്നെ ഭയപ്പെടുകയും ചെയ്യുന്നവര്‍ക്കു മാത്രമാണ്. അതിനാലവരെ പാപമോചനത്തെയും ഉദാരമായ പ്രതിഫലത്തെയും സംബന്ധിച്ച ശുഭവാര്‍ത്ത അറിയിക്കുക.
\end{malayalam}}
\flushright{\begin{Arabic}
\quranayah[36][12]
\end{Arabic}}
\flushleft{\begin{malayalam}
നിശ്ചയമായും നാം മരിച്ചവരെ ജീവിപ്പിക്കുന്നു. അവര്‍ ചെയ്തുകൂട്ടിയതും അവയുടെ അനന്തര ഫലങ്ങളും നാം രേഖപ്പെടുത്തുന്നു. എല്ലാ കാര്യങ്ങളും നാം വ്യക്തമായ ഒരു രേഖയില്‍ കൃത്യമായി ചേര്‍ത്തിരിക്കുന്നു.
\end{malayalam}}
\flushright{\begin{Arabic}
\quranayah[36][13]
\end{Arabic}}
\flushleft{\begin{malayalam}
ഒരു ഉദാഹരണമെന്ന നിലയില്‍ ആ നാട്ടുകാരുടെ കഥ ഇവര്‍ക്ക് പറഞ്ഞുകൊടുക്കുക: ദൈവദൂതന്മാര്‍ അവിടെ ചെന്ന സന്ദര്‍ഭം!
\end{malayalam}}
\flushright{\begin{Arabic}
\quranayah[36][14]
\end{Arabic}}
\flushleft{\begin{malayalam}
നാം അവരുടെ അടുത്തേക്ക് രണ്ടു ദൈവദൂതന്മാരെ അയച്ചു. അപ്പോള്‍ അവരിരുവരെയും ആ ജനം തള്ളിപ്പറഞ്ഞു. പിന്നെ നാം മൂന്നാമതൊരാളെ അയച്ച് അവര്‍ക്ക് പിന്‍ബലമേകി. അങ്ങനെ അവരെല്ലാം ആവര്‍ത്തിച്ചു പറഞ്ഞു: "ഞങ്ങള്‍ നിങ്ങളുടെ അടുത്തേക്ക് അയക്കപ്പെട്ട ദൈവദൂതന്മാരാണ്."
\end{malayalam}}
\flushright{\begin{Arabic}
\quranayah[36][15]
\end{Arabic}}
\flushleft{\begin{malayalam}
ആ ജനം പറഞ്ഞു: "നിങ്ങള്‍ ഞങ്ങളെപ്പോലുള്ള മനുഷ്യര്‍ മാത്രമാണ്. പരമദയാലുവായ ദൈവം ഒന്നും തന്നെ അവതരിപ്പിച്ചിട്ടില്ല. നിങ്ങള്‍ കള്ളം പറയുകയാണ്."
\end{malayalam}}
\flushright{\begin{Arabic}
\quranayah[36][16]
\end{Arabic}}
\flushleft{\begin{malayalam}
അവര്‍ പറഞ്ഞു: "ഞങ്ങളുടെ നാഥന്നറിയാം; ഉറപ്പായും ഞങ്ങള്‍ നിങ്ങളുടെ അടുത്തേക്കയക്കപ്പെട്ട ദൈവദൂതന്മാരാണെന്ന്.
\end{malayalam}}
\flushright{\begin{Arabic}
\quranayah[36][17]
\end{Arabic}}
\flushleft{\begin{malayalam}
"സന്ദേശം വ്യക്തമായി എത്തിച്ചുതരുന്നതില്‍ കവിഞ്ഞ ഉത്തരവാദിത്തമൊന്നും ഞങ്ങള്‍ക്കില്ല."
\end{malayalam}}
\flushright{\begin{Arabic}
\quranayah[36][18]
\end{Arabic}}
\flushleft{\begin{malayalam}
ആ ജനം പറഞ്ഞു: "തീര്‍ച്ചയായും ഞങ്ങള്‍ നിങ്ങളെ ദുശ്ശകുനമായാണ് കാണുന്നത്. നിങ്ങളിത് നിറുത്തുന്നില്ലെങ്കില്‍ ഉറപ്പായും ഞങ്ങള്‍ നിങ്ങളെ എറിഞ്ഞാട്ടും. ഞങ്ങളില്‍നിന്ന് നിങ്ങള്‍ നോവുറ്റ ശിക്ഷ അനുഭവിക്കുക തന്നെ ചെയ്യും."
\end{malayalam}}
\flushright{\begin{Arabic}
\quranayah[36][19]
\end{Arabic}}
\flushleft{\begin{malayalam}
ദൂതന്മാര്‍ പറഞ്ഞു: "നിങ്ങളുടെ ദുശ്ശകുനം നിങ്ങളോടൊപ്പമുള്ളതു തന്നെയാണ്. നിങ്ങള്‍ക്ക് ഉദ്ബോധനം നല്‍കിയതിനാലാണോ ഇതൊക്കെ? എങ്കില്‍ നിങ്ങള്‍ വല്ലാതെ പരിധിവിട്ട ജനം തന്നെ."
\end{malayalam}}
\flushright{\begin{Arabic}
\quranayah[36][20]
\end{Arabic}}
\flushleft{\begin{malayalam}
ആ പട്ടണത്തിന്റെ അങ്ങേയറ്റത്തുനിന്ന് ഒരാള്‍ ഓടിവന്നു പറഞ്ഞു: "എന്റെ ജനമേ, നിങ്ങള്‍ ഈ ദൈവദൂതന്മാരെ പിന്‍പറ്റുക.
\end{malayalam}}
\flushright{\begin{Arabic}
\quranayah[36][21]
\end{Arabic}}
\flushleft{\begin{malayalam}
"നിങ്ങളോട് പ്രതിഫലമൊന്നും ആവശ്യപ്പെടാത്തവരും നേരെ ചൊവ്വെ ജീവിക്കുന്നവരുമായ ഇവരെ പിന്‍തുടരുക.
\end{malayalam}}
\flushright{\begin{Arabic}
\quranayah[36][22]
\end{Arabic}}
\flushleft{\begin{malayalam}
"ആരാണോ എന്നെ സൃഷ്ടിച്ചത്; ആരിലേക്കാണോ നിങ്ങള്‍ തിരിച്ചുചെല്ലേണ്ടത്; ആ അല്ലാഹുവെ വഴിപ്പെടാതിരിക്കാന്‍ എനിക്കെന്തു ന്യായം?
\end{malayalam}}
\flushright{\begin{Arabic}
\quranayah[36][23]
\end{Arabic}}
\flushleft{\begin{malayalam}
"അവനെയല്ലാതെ മറ്റുള്ളവയെ ഞാന്‍ ദൈവങ്ങളായി സ്വീകരിക്കുകയോ? ആ പരമകാരുണികന്‍ എനിക്കു വല്ല വിപത്തും വരുത്താനുദ്ദേശിച്ചാല്‍ അവരുടെ ശിപാര്‍ശയൊന്നും എനിക്കൊട്ടും ഉപകരിക്കുകയില്ല. അവരെന്നെ രക്ഷിക്കുകയുമില്ല.
\end{malayalam}}
\flushright{\begin{Arabic}
\quranayah[36][24]
\end{Arabic}}
\flushleft{\begin{malayalam}
"അങ്ങനെ ചെയ്താല്‍ സംശയമില്ല. ഞാന്‍ വ്യക്തമായ വഴികേടിലായിരിക്കും.
\end{malayalam}}
\flushright{\begin{Arabic}
\quranayah[36][25]
\end{Arabic}}
\flushleft{\begin{malayalam}
"തീര്‍ച്ചയായും ഞാന്‍ നിങ്ങളുടെ നാഥനില്‍ വിശ്വസിച്ചിരിക്കുന്നു. അതിനാല്‍ നിങ്ങളെന്റെ വാക്ക് കേള്‍ക്കുക."
\end{malayalam}}
\flushright{\begin{Arabic}
\quranayah[36][26]
\end{Arabic}}
\flushleft{\begin{malayalam}
“നീ സ്വര്‍ഗത്തില്‍ പ്രവേശിച്ചുകൊള്ളുക” എന്ന് അദ്ദേഹത്തോട് പറഞ്ഞു. അദ്ദേഹം പറഞ്ഞു: "ഹാ, എന്റെ ജനത ഇതറിഞ്ഞിരുന്നെങ്കില്‍!
\end{malayalam}}
\flushright{\begin{Arabic}
\quranayah[36][27]
\end{Arabic}}
\flushleft{\begin{malayalam}
"അഥവാ, എന്റെ നാഥന്‍ എനിക്കു മാപ്പേകിയതും എന്നെ ആദരണീയരിലുള്‍പ്പെടുത്തിയതും."
\end{malayalam}}
\flushright{\begin{Arabic}
\quranayah[36][28]
\end{Arabic}}
\flushleft{\begin{malayalam}
അതിനുശേഷം നാം അദ്ദേഹത്തിന്റെ ജനതയുടെ നേരെ ഉപരിലോകത്തുനിന്ന് ഒരു സൈന്യത്തെയും ഇറക്കിയിട്ടില്ല. അങ്ങനെ ഇറക്കേണ്ട ആവശ്യവും നമുക്കുണ്ടായിട്ടില്ല.
\end{malayalam}}
\flushright{\begin{Arabic}
\quranayah[36][29]
\end{Arabic}}
\flushleft{\begin{malayalam}
അതൊരു ഘോരഗര്‍ജനം മാത്രമായിരുന്നു. അപ്പോഴേക്കും അവരൊക്കെയും നാമാവശേഷമായി.
\end{malayalam}}
\flushright{\begin{Arabic}
\quranayah[36][30]
\end{Arabic}}
\flushleft{\begin{malayalam}
ആ അടിമകളുടെ കാര്യമെത്ര ദയനീയം! അവരിലേക്ക് ചെന്ന ഒരൊറ്റ ദൈവദൂതനെപ്പോലും അവര്‍ പുച്ഛിക്കാതിരുന്നിട്ടില്ല.
\end{malayalam}}
\flushright{\begin{Arabic}
\quranayah[36][31]
\end{Arabic}}
\flushleft{\begin{malayalam}
ഇവര്‍ക്കു മുമ്പ് എത്രയെത്ര തലമുറകളെയാണ് നാം നശിപ്പിച്ചത്? പിന്നെ അവരാരും ഇവരുടെ അടുത്തേക്ക് തിരിച്ചുവന്നിട്ടില്ല. ഇതൊന്നും ഇക്കൂട്ടര്‍ കാണുന്നില്ലേ?
\end{malayalam}}
\flushright{\begin{Arabic}
\quranayah[36][32]
\end{Arabic}}
\flushleft{\begin{malayalam}
സംശയമില്ല; അവരെല്ലാം നമ്മുടെ മുമ്പില്‍ ഹാജരാക്കപ്പെടും.
\end{malayalam}}
\flushright{\begin{Arabic}
\quranayah[36][33]
\end{Arabic}}
\flushleft{\begin{malayalam}
ഈ ജനത്തിന് വ്യക്തമായ ഒരു ദൃഷ്ടാന്തമിതാ: ചത്തുകിടക്കുന്ന ഭൂമി, നാം അതിനെ ജീവനുള്ളതാക്കി. അതില്‍ ധാരാളം ധാന്യം വിളയിച്ചു. എന്നിട്ട് അതില്‍ നിന്നിവര്‍ തിന്നുന്നു.
\end{malayalam}}
\flushright{\begin{Arabic}
\quranayah[36][34]
\end{Arabic}}
\flushleft{\begin{malayalam}
നാമതില്‍ ഈന്തപ്പനയുടെയും മുന്തിരിയുടെയും തോട്ടങ്ങളുണ്ടാക്കി. അതിലെത്രയോ ഉറവകള്‍ ഒഴുക്കി!
\end{malayalam}}
\flushright{\begin{Arabic}
\quranayah[36][35]
\end{Arabic}}
\flushleft{\begin{malayalam}
അതിന്റെ പഴങ്ങളിവര്‍ തിന്നാനാണിതെല്ലാമുണ്ടാക്കിയത്. ഇവരുടെ കൈകള്‍ അധ്വാനിച്ചുണ്ടാക്കിയവയല്ല ഇതൊന്നും. എന്നിട്ടും ഇക്കൂട്ടര്‍ നന്ദി കാണിക്കുന്നില്ലേ?
\end{malayalam}}
\flushright{\begin{Arabic}
\quranayah[36][36]
\end{Arabic}}
\flushleft{\begin{malayalam}
ഭൂമിയില്‍ മുളച്ചുണ്ടാവുന്ന സസ്യങ്ങള്‍, മനുഷ്യവര്‍ഗം, മനുഷ്യര്‍ക്കറിയാത്ത മറ്റനേകം വസ്തുക്കള്‍ എല്ലാറ്റിനെയും ഇണകളായി സൃഷ്ടിച്ച അല്ലാഹു എത്ര പരിശുദ്ധന്‍.
\end{malayalam}}
\flushright{\begin{Arabic}
\quranayah[36][37]
\end{Arabic}}
\flushleft{\begin{malayalam}
രാവും ഇവര്‍ക്കൊരു ദൃഷ്ടാന്തമാണ്. അതില്‍നിന്ന് പകലിനെ നാം ഊരിയെടുക്കുന്നു. അതോടെ ഇവര്‍ ഇരുളിലകപ്പെടുന്നു.
\end{malayalam}}
\flushright{\begin{Arabic}
\quranayah[36][38]
\end{Arabic}}
\flushleft{\begin{malayalam}
സൂര്യന്‍ അതിന്റെ സങ്കേതത്തിലേക്ക് സഞ്ചരിച്ചുകൊണ്ടിരിക്കുന്നു. പ്രതാപിയും എല്ലാം അറിയുന്നവനുമായ അല്ലാഹുവിന്റെ സൂക്ഷ്മമായ പദ്ധതിയനുസരിച്ചാണത്.
\end{malayalam}}
\flushright{\begin{Arabic}
\quranayah[36][39]
\end{Arabic}}
\flushleft{\begin{malayalam}
ചന്ദ്രന്നും നാം ചില മണ്ഡലങ്ങള്‍ നിശ്ചയിച്ചിരിക്കുന്നു. അതിലൂടെ കടന്നുപോയി അത് ഉണങ്ങി വളഞ്ഞ ഈന്തപ്പനക്കുലയുടെ തണ്ടുപോലെയായിത്തീരുന്നു.
\end{malayalam}}
\flushright{\begin{Arabic}
\quranayah[36][40]
\end{Arabic}}
\flushleft{\begin{malayalam}
ചന്ദ്രനെ എത്തിപ്പിടിക്കാന്‍ സൂര്യനു സാധ്യമല്ല. പകലിനെ മറികടക്കാന്‍ രാവിനുമാവില്ല. എല്ലാ ഓരോന്നും നിശ്ചിത ഭ്രമണപഥത്തില്‍ നീന്തിത്തുടിക്കുകയാണ്.
\end{malayalam}}
\flushright{\begin{Arabic}
\quranayah[36][41]
\end{Arabic}}
\flushleft{\begin{malayalam}
ഇവരുടെ സന്താനങ്ങളെ നാം ഭാരം നിറച്ച കപ്പലില്‍ കയറ്റിക്കൊണ്ടുപോയതും ഇവര്‍ക്കൊരു ദൃഷ്ടാന്തമാണ്.
\end{malayalam}}
\flushright{\begin{Arabic}
\quranayah[36][42]
\end{Arabic}}
\flushleft{\begin{malayalam}
ഇവര്‍ക്കായി ഇതുപോലുള്ള വേറെയും വാഹനങ്ങള്‍ നാമുണ്ടാക്കിക്കൊടുത്തിട്ടുണ്ട്.
\end{malayalam}}
\flushright{\begin{Arabic}
\quranayah[36][43]
\end{Arabic}}
\flushleft{\begin{malayalam}
നാമിച്ഛിക്കുന്നുവെങ്കില്‍ നാമവരെ മുക്കിക്കൊല്ലും. അപ്പോഴിവരുടെ നിലവിളി കേള്‍ക്കാനാരുമുണ്ടാവില്ല. ഇവര്‍ രക്ഷപ്പെടുകയുമില്ല.
\end{malayalam}}
\flushright{\begin{Arabic}
\quranayah[36][44]
\end{Arabic}}
\flushleft{\begin{malayalam}
അങ്ങനെയൊന്ന് സംഭവിക്കാത്തത് നമ്മുടെ കാരുണ്യംകൊണ്ട് മാത്രമാണ്. ഇവര്‍ നിശ്ചിത പരിധിവരെ ജീവിതസുഖം അനുഭവിക്കാനും.
\end{malayalam}}
\flushright{\begin{Arabic}
\quranayah[36][45]
\end{Arabic}}
\flushleft{\begin{malayalam}
"നിങ്ങള്‍ക്കു മുന്നില്‍ സംഭവിക്കുന്നതും നേരത്തെ സംഭവിച്ചുകഴിഞ്ഞതുമായ വിപത്തുകളെ സൂക്ഷിക്കുക. നിങ്ങള്‍ക്കു കാരുണ്യം കിട്ടിയേക്കാം" എന്ന് ഇവരോടാവശ്യപ്പെട്ടാല്‍ ഇവരത് തീരേ ശ്രദ്ധിക്കുകയില്ല.
\end{malayalam}}
\flushright{\begin{Arabic}
\quranayah[36][46]
\end{Arabic}}
\flushleft{\begin{malayalam}
ഇവര്‍ക്ക് തങ്ങളുടെ നാഥന്റെ ദൃഷ്ടാന്തങ്ങളില്‍നിന്ന് ഏതൊന്ന് വന്നെത്തിയാലും ഇവരത് പാടേ അവഗണിച്ചുതള്ളുന്നു.
\end{malayalam}}
\flushright{\begin{Arabic}
\quranayah[36][47]
\end{Arabic}}
\flushleft{\begin{malayalam}
"നിങ്ങള്‍ക്ക് അല്ലാഹു നല്‍കിയതില്‍നിന്ന് ചെലവഴിക്കുക" എന്നാവശ്യപ്പെട്ടാല്‍ സത്യനിഷേധികള്‍ വിശ്വാസികളോടു പറയും: "അല്ലാഹു വിചാരിച്ചിരുന്നെങ്കില്‍ അവന്‍ തന്നെ ഇവര്‍ക്ക് അന്നം നല്‍കുമായിരുന്നല്ലോ. പിന്നെ ഞങ്ങളിവര്‍ക്ക് എന്തിന് അന്നം നല്‍കണം? നിങ്ങള്‍ വ്യക്തമായ വഴികേടില്‍ തന്നെ."
\end{malayalam}}
\flushright{\begin{Arabic}
\quranayah[36][48]
\end{Arabic}}
\flushleft{\begin{malayalam}
ഇക്കൂട്ടര്‍ ചോദിക്കുന്നു: "ഈ വാഗ്ദാനം എപ്പോഴാണ് പുലരുക- നിങ്ങള്‍ സത്യവാന്മാരെങ്കില്‍?"
\end{malayalam}}
\flushright{\begin{Arabic}
\quranayah[36][49]
\end{Arabic}}
\flushleft{\begin{malayalam}
യഥാര്‍ഥത്തിലിവര്‍ കാത്തിരിക്കുന്നത് ഒരൊറ്റ ഘോരശബ്ദം മാത്രമാണ്. അവരന്യോന്യം തര്‍ക്കിച്ചുകൊണ്ടിരിക്കെ അതവരെ പിടികൂടും.
\end{malayalam}}
\flushright{\begin{Arabic}
\quranayah[36][50]
\end{Arabic}}
\flushleft{\begin{malayalam}
അപ്പോഴിവര്‍ക്ക് ഒരു വസിയ്യത്ത് ചെയ്യാന്‍പോലും സാധിക്കുകയില്ല. തങ്ങളുടെ കുടുംബത്തിലേക്ക് മടങ്ങാനും കഴിയില്ല.
\end{malayalam}}
\flushright{\begin{Arabic}
\quranayah[36][51]
\end{Arabic}}
\flushleft{\begin{malayalam}
കാഹളത്തില്‍ ഊതപ്പെടും. അപ്പോഴിവര്‍ കുഴിമാടങ്ങളില്‍നിന്ന് തങ്ങളുടെ നാഥങ്കലേക്ക് കുതിച്ചോടും.
\end{malayalam}}
\flushright{\begin{Arabic}
\quranayah[36][52]
\end{Arabic}}
\flushleft{\begin{malayalam}
അവര്‍ പറയും: "നമ്മുടെ നാശമേ, നമ്മുടെ ഉറക്കത്തില്‍ നിന്ന് നമ്മെ ഉണര്‍ത്തി എഴുന്നേല്‍പിച്ചത് ആരാണ്? ഇത് ആ പരമ കാരുണികന്‍ വാഗ്ദാനം ചെയ്തതാണല്ലോ. ദൈവദൂതന്മാര്‍ പറഞ്ഞത് സത്യംതന്നെ."
\end{malayalam}}
\flushright{\begin{Arabic}
\quranayah[36][53]
\end{Arabic}}
\flushleft{\begin{malayalam}
അതൊരു ഘോരശബ്ദം മാത്രമായിരിക്കും. അപ്പോഴേക്കും അവരതാ ഒന്നടങ്കം നമ്മുടെ സന്നിധിയില്‍ ഹാജരാക്കപ്പെടുന്നു.
\end{malayalam}}
\flushright{\begin{Arabic}
\quranayah[36][54]
\end{Arabic}}
\flushleft{\begin{malayalam}
അന്നാളില്‍ ആരോടും അല്‍പവും അനീതി ഉണ്ടാവില്ല. നിങ്ങള്‍ പ്രവര്‍ത്തിച്ചുകൊണ്ടിരുന്നതിനുള്ള പ്രതിഫലമാണ് നിങ്ങള്‍ക്കുണ്ടാവുക.
\end{malayalam}}
\flushright{\begin{Arabic}
\quranayah[36][55]
\end{Arabic}}
\flushleft{\begin{malayalam}
സംശയംവേണ്ട; അന്ന് സ്വര്‍ഗാവകാശികള്‍ ഓരോ പ്രവൃത്തികളിലായി പരമാനന്ദത്തിലായിരിക്കും.
\end{malayalam}}
\flushright{\begin{Arabic}
\quranayah[36][56]
\end{Arabic}}
\flushleft{\begin{malayalam}
അവരും അവരുടെ ഇണകളും സ്വര്‍ഗത്തണലുകളില്‍ കട്ടിലുകളില്‍ ചാരിയിരിക്കുന്നവരായിരിക്കും.
\end{malayalam}}
\flushright{\begin{Arabic}
\quranayah[36][57]
\end{Arabic}}
\flushleft{\begin{malayalam}
അവര്‍ക്കവിടെ രുചികരമായ പഴങ്ങളുണ്ട്. അവരാവശ്യപ്പെടുന്നതെന്തും അവിടെ കിട്ടും.
\end{malayalam}}
\flushright{\begin{Arabic}
\quranayah[36][58]
\end{Arabic}}
\flushleft{\begin{malayalam}
സലാം - സമാധാനം - ഇതായിരിക്കും ദയാപരനായ നാഥനില്‍നിന്ന് അവര്‍ക്കുള്ള അഭിവാദ്യം.
\end{malayalam}}
\flushright{\begin{Arabic}
\quranayah[36][59]
\end{Arabic}}
\flushleft{\begin{malayalam}
“കുറ്റവാളികളേ, നിങ്ങളിന്ന് വേറെ മാറിനില്‍ക്കുക.”
\end{malayalam}}
\flushright{\begin{Arabic}
\quranayah[36][60]
\end{Arabic}}
\flushleft{\begin{malayalam}
ആദം സന്തതികളേ, ഞാന്‍ നിങ്ങളെ ഉപദേശിച്ചിരുന്നില്ലേ, ചെകുത്താന് വഴിപ്പെടരുതെന്ന്; അവന്‍ നിങ്ങളുടെ പ്രത്യക്ഷ ശത്രുവാണെന്ന്.
\end{malayalam}}
\flushright{\begin{Arabic}
\quranayah[36][61]
\end{Arabic}}
\flushleft{\begin{malayalam}
നിങ്ങള്‍ എനിക്കു വഴിപ്പെടുക, ഇതാണ് നേര്‍വഴിയെന്നും.
\end{malayalam}}
\flushright{\begin{Arabic}
\quranayah[36][62]
\end{Arabic}}
\flushleft{\begin{malayalam}
സംശയമില്ല; നിങ്ങളിലെ നിരവധി സംഘങ്ങളെ പിശാച് പിഴപ്പിച്ചിട്ടുണ്ട്. എന്നിട്ടും നിങ്ങള്‍ ചിന്തിച്ചു മനസ്സിലാക്കുന്നില്ലേ?
\end{malayalam}}
\flushright{\begin{Arabic}
\quranayah[36][63]
\end{Arabic}}
\flushleft{\begin{malayalam}
ഇതാ, നിങ്ങള്‍ക്കു വാഗ്ദാനം ചെയ്തിരുന്ന നരകം!
\end{malayalam}}
\flushright{\begin{Arabic}
\quranayah[36][64]
\end{Arabic}}
\flushleft{\begin{malayalam}
നിങ്ങള്‍ സത്യത്തെ തള്ളിപ്പറഞ്ഞു. അതിന്റെ ഫലമായി നിങ്ങളിന്ന് നരകത്തില്‍ കിടന്നെരിയുക.
\end{malayalam}}
\flushright{\begin{Arabic}
\quranayah[36][65]
\end{Arabic}}
\flushleft{\begin{malayalam}
അന്ന് നാമവരുടെ വായ അടച്ചു മുദ്രവെക്കും. അവരുടെ കൈകള്‍ നമ്മോടു സംസാരിക്കും. കാലുകള്‍ സാക്ഷ്യംവഹിക്കും- അവര്‍ ചെയ്തുകൊണ്ടിരുന്നതെന്താണെന്ന്.
\end{malayalam}}
\flushright{\begin{Arabic}
\quranayah[36][66]
\end{Arabic}}
\flushleft{\begin{malayalam}
നാം ഇച്ഛിച്ചിരുന്നെങ്കില്‍ അവരുടെ കണ്ണുകളെത്തന്നെ നാം മായ്ച്ചുകളയുമായിരുന്നു. അപ്പോഴവര്‍ വഴിയിലൂടെ മുന്നോട്ട് കുതിക്കാന്‍ നോക്കും. എന്നാല്‍ അവരെങ്ങനെ വഴി കാണാനാണ്?
\end{malayalam}}
\flushright{\begin{Arabic}
\quranayah[36][67]
\end{Arabic}}
\flushleft{\begin{malayalam}
നാം ഉദ്ദേശിച്ചിരുന്നെങ്കില്‍ നാമവരെ അവര്‍ നില്‍ക്കുന്നേടത്തുവെച്ചുതന്നെ രൂപമാറ്റം വരുത്തുമായിരുന്നു. അപ്പോഴവര്‍ക്കു മുന്നോട്ടു പോവാനാവില്ല. പിന്നോട്ടു മടങ്ങാനും കഴിയില്ല.
\end{malayalam}}
\flushright{\begin{Arabic}
\quranayah[36][68]
\end{Arabic}}
\flushleft{\begin{malayalam}
നാം ആര്‍ക്കെങ്കിലും ദീര്‍ഘായുസ്സ് നല്‍കുകയാണെങ്കില്‍ അയാളുടെ പ്രകൃതി തന്നെ പാടെ മാറ്റിമറിക്കുന്നു. എന്നിട്ടും ഇതൊന്നും അവരൊട്ടും ആലോചിച്ചറിയുന്നില്ലേ?
\end{malayalam}}
\flushright{\begin{Arabic}
\quranayah[36][69]
\end{Arabic}}
\flushleft{\begin{malayalam}
നാം അദ്ദേഹത്തെ കവിത പരിശീലിപ്പിച്ചിട്ടില്ല. അത് അദ്ദേഹത്തിന് യോജിച്ചതല്ല. ഇതാകട്ടെ ഗൌരവപൂര്‍ണമായ ഒരുദ്ബോധനമാണ്. സ്പഷ്ടമായി വായിക്കാവുന്ന വേദപുസ്തകം.
\end{malayalam}}
\flushright{\begin{Arabic}
\quranayah[36][70]
\end{Arabic}}
\flushleft{\begin{malayalam}
ജീവനുള്ളവര്‍ക്ക് മുന്നറിയിപ്പ് നല്‍കാനാണിത്. സത്യത്തെ തള്ളിപ്പറയുന്നവര്‍ക്കെതിരെ ന്യായം സ്ഥാപിച്ചെടുക്കാനും.
\end{malayalam}}
\flushright{\begin{Arabic}
\quranayah[36][71]
\end{Arabic}}
\flushleft{\begin{malayalam}
നമ്മുടെ കരങ്ങളുണ്ടാക്കിയവയില്‍പെട്ടവയാണ് കന്നുകാലികളെന്ന് അവര്‍ കാണുന്നില്ലേ; അവര്‍ക്കു വേണ്ടിയാണ് നാമത് സൃഷ്ടിച്ചതെന്നും. ഇപ്പോഴവ അവരുടെ അധീനതയിലാണല്ലോ.
\end{malayalam}}
\flushright{\begin{Arabic}
\quranayah[36][72]
\end{Arabic}}
\flushleft{\begin{malayalam}
അവയെ നാമവര്‍ക്ക് മെരുക്കിയൊതുക്കിക്കൊടുത്തിരിക്കുന്നു. അവയില്‍ ചിലത് അവരുടെ വാഹനമാണ്. ചിലതിനെ അവര്‍ ആഹരിക്കുകയും ചെയ്യുന്നു.
\end{malayalam}}
\flushright{\begin{Arabic}
\quranayah[36][73]
\end{Arabic}}
\flushleft{\begin{malayalam}
അവര്‍ക്ക് അവയില്‍ പല പ്രയോജനങ്ങളുമുണ്ട്. പാനീയങ്ങളുമുണ്ട്. എന്നിട്ടും അവര്‍ നന്ദി കാണിക്കുന്നില്ലേ?
\end{malayalam}}
\flushright{\begin{Arabic}
\quranayah[36][74]
\end{Arabic}}
\flushleft{\begin{malayalam}
തങ്ങള്‍ക്കു സഹായം കിട്ടാനായി അല്ലാഹുവെക്കൂടാതെ പല ദൈവങ്ങളെയും അവര്‍ പങ്കാളികളായി സങ്കല്‍പിച്ചുവെച്ചിരിക്കുന്നു.
\end{malayalam}}
\flushright{\begin{Arabic}
\quranayah[36][75]
\end{Arabic}}
\flushleft{\begin{malayalam}
എന്നാല്‍ അവരെ സഹായിക്കാന്‍ അവയ്ക്കു സാധ്യമല്ല. യഥാര്‍ഥത്തിലവര്‍ ആ ദൈവങ്ങള്‍ക്കായി തയ്യാറായി നില്‍ക്കുന്ന സൈന്യമാണ്.
\end{malayalam}}
\flushright{\begin{Arabic}
\quranayah[36][76]
\end{Arabic}}
\flushleft{\begin{malayalam}
അതിനാല്‍ അവരുടെ വാക്കുകള്‍ നിന്നെ വേദനിപ്പിക്കാതിരിക്കട്ടെ. തീര്‍ച്ചയായും അവര്‍ പരസ്യമാക്കുന്നതും രഹസ്യമാക്കുന്നതുമൊക്കെ നാം നന്നായറിയുന്നുണ്ട്.
\end{malayalam}}
\flushright{\begin{Arabic}
\quranayah[36][77]
\end{Arabic}}
\flushleft{\begin{malayalam}
മനുഷ്യനെ നാമൊരു ബീജകണത്തില്‍ നിന്നാണ് സൃഷ്ടിച്ചതെന്ന് അവന്‍ മനസ്സിലാക്കിയിട്ടില്ലേ. എന്നിട്ടിപ്പോള്‍ അവനിതാ ഒരു പ്രത്യക്ഷശത്രുവായി മാറിയിരിക്കുന്നു.
\end{malayalam}}
\flushright{\begin{Arabic}
\quranayah[36][78]
\end{Arabic}}
\flushleft{\begin{malayalam}
അവന്‍ നമുക്ക് ഉപമചമച്ചിരിക്കുന്നു. തന്നെ സൃഷ്ടിച്ച കാര്യമവന്‍ തീരെ മറന്നുകളഞ്ഞു. അവന്‍ ചോദിക്കുന്നു: എല്ലുകള്‍ പറ്റെ ദ്രവിച്ചുകഴിഞ്ഞ ശേഷം അവയെ ആര് ജീവിപ്പിക്കാനാണ്?
\end{malayalam}}
\flushright{\begin{Arabic}
\quranayah[36][79]
\end{Arabic}}
\flushleft{\begin{malayalam}
പറയുക: ഒന്നാം തവണ അവയെ സൃഷ്ടിച്ചവന്‍ തന്നെ വീണ്ടും അവയെ ജീവിപ്പിക്കും. അവന്‍ എല്ലാവിധ സൃഷ്ടിപ്പിനെപ്പറ്റിയും നന്നായറിയുന്നവനാണ്.
\end{malayalam}}
\flushright{\begin{Arabic}
\quranayah[36][80]
\end{Arabic}}
\flushleft{\begin{malayalam}
പച്ചമരത്തില്‍നിന്ന് നിങ്ങള്‍ക്ക് തീയുണ്ടാക്കിത്തന്നവനാണവന്‍. അങ്ങനെ നിങ്ങളിപ്പോഴിതാ അതുപയോഗിച്ച് തീ കത്തിക്കുന്നു.
\end{malayalam}}
\flushright{\begin{Arabic}
\quranayah[36][81]
\end{Arabic}}
\flushleft{\begin{malayalam}
ആകാശഭൂമികളെ പടച്ചവന്‍ അവരെപ്പോലുള്ളവരെ സൃഷ്ടിക്കാന്‍ കഴിവുള്ളവനല്ലെന്നോ? അങ്ങനെയല്ല. അവന്‍ കഴിവുറ്റ സ്രഷ്ടാവാണ്. എല്ലാം അറിയുന്നവനും.
\end{malayalam}}
\flushright{\begin{Arabic}
\quranayah[36][82]
\end{Arabic}}
\flushleft{\begin{malayalam}
അവന്‍ ഒരു കാര്യം ഉദ്ദേശിച്ചാല്‍ അതിനോട് “ഉണ്ടാകൂ” എന്ന് പറയുകയേ വേണ്ടൂ. അപ്പോഴേക്കും അതുണ്ടാകുന്നു. ഇതാണവന്റെ അവസ്ഥ.
\end{malayalam}}
\flushright{\begin{Arabic}
\quranayah[36][83]
\end{Arabic}}
\flushleft{\begin{malayalam}
സകല സംഗതികളുടെയും സമഗ്രാധിപത്യം ആരുടെ കയ്യിലാണോ, നിങ്ങള്‍ മടങ്ങിച്ചെല്ലുന്നത് ആരുടെ അടുത്തേക്കാണോ, അവനാണ് പരിശുദ്ധന്‍!
\end{malayalam}}
\chapter{\textmalayalam{സ്വാഫ്ഫാത്ത് ( അണിനിരന്നവ‍ )}}
\begin{Arabic}
\Huge{\centerline{\basmalah}}\end{Arabic}
\flushright{\begin{Arabic}
\quranayah[37][1]
\end{Arabic}}
\flushleft{\begin{malayalam}
അണിയണിയായി നിരന്നുനില്‍ക്കുന്നവര്‍ സത്യം.
\end{malayalam}}
\flushright{\begin{Arabic}
\quranayah[37][2]
\end{Arabic}}
\flushleft{\begin{malayalam}
പിന്നെ ശക്തമായി ചെറുത്തുനില്‍ക്കുന്നവര്‍തന്നെ സത്യം.
\end{malayalam}}
\flushright{\begin{Arabic}
\quranayah[37][3]
\end{Arabic}}
\flushleft{\begin{malayalam}
എന്നിട്ടു കീര്‍ത്തനം ചൊല്ലുന്നവര്‍ സത്യം.
\end{malayalam}}
\flushright{\begin{Arabic}
\quranayah[37][4]
\end{Arabic}}
\flushleft{\begin{malayalam}
തീര്‍ച്ചയായും നിങ്ങളുടെയെല്ലാം ദൈവം ഏകനാണ്.
\end{malayalam}}
\flushright{\begin{Arabic}
\quranayah[37][5]
\end{Arabic}}
\flushleft{\begin{malayalam}
ആകാശഭൂമികളുടെയും അവയ്ക്കിടയിലുള്ളവയുടെയും നാഥനാണവന്‍. ഉദയ സ്ഥാനങ്ങളുടെ പരിരക്ഷകന്‍.
\end{malayalam}}
\flushright{\begin{Arabic}
\quranayah[37][6]
\end{Arabic}}
\flushleft{\begin{malayalam}
അടുത്തുള്ള ആകാശത്തെ നാം നക്ഷത്രാലങ്കാരങ്ങളാല്‍ മനോഹരമാക്കിയിരിക്കുന്നു.
\end{malayalam}}
\flushright{\begin{Arabic}
\quranayah[37][7]
\end{Arabic}}
\flushleft{\begin{malayalam}
ധിക്കാരിയായ ഏതു ചെകുത്താനില്‍നിന്നും അതിനെ സുരക്ഷിതമാക്കിയിരിക്കുന്നു.
\end{malayalam}}
\flushright{\begin{Arabic}
\quranayah[37][8]
\end{Arabic}}
\flushleft{\begin{malayalam}
അത്യുന്നത സഭയിലെ സംസാരം ചെവികൊടുത്തുകേള്‍ക്കാന്‍ ഈ ചെകുത്താന്മാര്‍ക്കാവില്ല. നാനാഭാഗത്തുനിന്നും അവര്‍ എറിഞ്ഞോടിക്കപ്പെടും.
\end{malayalam}}
\flushright{\begin{Arabic}
\quranayah[37][9]
\end{Arabic}}
\flushleft{\begin{malayalam}
ബഹിഷ്കൃതരായിക്കൊണ്ട്‌; അവര്‍ക്ക് അറുതിയില്ലാത്ത ശിക്ഷയുണ്ട്.
\end{malayalam}}
\flushright{\begin{Arabic}
\quranayah[37][10]
\end{Arabic}}
\flushleft{\begin{malayalam}
എന്നാല്‍, അവരിലാരെങ്കിലും അതില്‍നിന്ന് വല്ലതും തട്ടിയെടുക്കുകയാണെങ്കില്‍ തീക്ഷ്ണമായ തീജ്ജ്വാല അവനെ പിന്തുടരും.
\end{malayalam}}
\flushright{\begin{Arabic}
\quranayah[37][11]
\end{Arabic}}
\flushleft{\begin{malayalam}
അതിനാല്‍ നീ അവരോട് ചോദിക്കുക: ഇവരെ സൃഷ്ടിക്കുന്നതാണോ കൂടുതല്‍ പ്രയാസകരം, അതോ നാം ഉണ്ടാക്കിയ മറ്റുള്ളവയെ സൃഷ്ടിക്കുന്നതോ? തീര്‍ച്ചയായും നാമിവരെ സൃഷ്ടിച്ചത് പറ്റിപ്പിടിക്കുന്ന കളിമണ്ണില്‍ നിന്നാണ്.
\end{malayalam}}
\flushright{\begin{Arabic}
\quranayah[37][12]
\end{Arabic}}
\flushleft{\begin{malayalam}
എന്നാല്‍, നിനക്ക് വിസ്മയം തോന്നുന്നു. അവരോ അതിനെ പരിഹസിക്കുകയും ചെയ്യുന്നു.
\end{malayalam}}
\flushright{\begin{Arabic}
\quranayah[37][13]
\end{Arabic}}
\flushleft{\begin{malayalam}
അവരെ ഉപദേശിച്ചാലും അവരതേക്കുറിച്ചാലോചിക്കുന്നില്ല.
\end{malayalam}}
\flushright{\begin{Arabic}
\quranayah[37][14]
\end{Arabic}}
\flushleft{\begin{malayalam}
ഏതൊരു ദൃഷ്ടാന്തം കണ്ടാലും അവരതിനെ പുച്ഛിച്ചുതള്ളുന്നു.
\end{malayalam}}
\flushright{\begin{Arabic}
\quranayah[37][15]
\end{Arabic}}
\flushleft{\begin{malayalam}
അവര്‍ പറയുന്നു: "ഇതു പ്രകടമായ ജാലവിദ്യ തന്നെ.
\end{malayalam}}
\flushright{\begin{Arabic}
\quranayah[37][16]
\end{Arabic}}
\flushleft{\begin{malayalam}
"നാം മരിച്ച് മണ്ണും എല്ലുമായിക്കഴിഞ്ഞാല്‍ വീണ്ടും ഉയിര്‍ത്തെഴുന്നേല്‍പിക്കപ്പെടുമെന്നോ?
\end{malayalam}}
\flushright{\begin{Arabic}
\quranayah[37][17]
\end{Arabic}}
\flushleft{\begin{malayalam}
"നമ്മുടെ പൂര്‍വ പിതാക്കളും ഉയിര്‍ത്തെഴുന്നേല്‍ക്കുമെന്നോ?"
\end{malayalam}}
\flushright{\begin{Arabic}
\quranayah[37][18]
\end{Arabic}}
\flushleft{\begin{malayalam}
പറയുക: അതെ. അങ്ങനെ സംഭവിക്കും. നിങ്ങളന്ന് പറ്റെ പതിതരായിത്തീരും.
\end{malayalam}}
\flushright{\begin{Arabic}
\quranayah[37][19]
\end{Arabic}}
\flushleft{\begin{malayalam}
അതൊരു ഘോരഗര്‍ജനം മാത്രമായിരിക്കും. അപ്പോഴേക്കും അവര്‍ കണ്ണുതുറന്ന് നോക്കുന്നവരായിത്തീരും.
\end{malayalam}}
\flushright{\begin{Arabic}
\quranayah[37][20]
\end{Arabic}}
\flushleft{\begin{malayalam}
അവര്‍ പറയും: "അയ്യോ, നമുക്ക് നാശം! ഇത് പ്രതിഫലത്തിന്റെ ദിനം തന്നെ."
\end{malayalam}}
\flushright{\begin{Arabic}
\quranayah[37][21]
\end{Arabic}}
\flushleft{\begin{malayalam}
അതെ, നിങ്ങള്‍ തള്ളിപ്പറഞ്ഞ ആ വിധിത്തീര്‍പ്പിന്റെ ദിനം തന്നെയാണിത്.
\end{malayalam}}
\flushright{\begin{Arabic}
\quranayah[37][22]
\end{Arabic}}
\flushleft{\begin{malayalam}
"അക്രമം പ്രവര്‍ത്തിച്ചവരെയും അവരുടെ ഇണകളെയും അവര്‍ ആരാധിച്ചിരുന്നവയെയും നിങ്ങള്‍ ഒരുമിച്ചുകൂട്ടുക; --
\end{malayalam}}
\flushright{\begin{Arabic}
\quranayah[37][23]
\end{Arabic}}
\flushleft{\begin{malayalam}
"അല്ലാഹുവെക്കൂടാതെ; എന്നിട്ടവരെയെല്ലാം നിങ്ങള്‍ നരകത്തിലേക്കുള്ള വഴിയില്‍ നയിക്കുക" എന്ന കല്‍പനയുണ്ടാകും.
\end{malayalam}}
\flushright{\begin{Arabic}
\quranayah[37][24]
\end{Arabic}}
\flushleft{\begin{malayalam}
അവരെയൊന്ന് നിര്‍ത്തൂ അവരെ ചോദ്യം ചെയ്യേണ്ടതുണ്ട്.
\end{malayalam}}
\flushright{\begin{Arabic}
\quranayah[37][25]
\end{Arabic}}
\flushleft{\begin{malayalam}
"അല്ല; നിങ്ങള്‍ക്കെന്തുപറ്റി? നിങ്ങള്‍ പരസ്പരം സഹായിക്കുന്നില്ലല്ലോ."
\end{malayalam}}
\flushright{\begin{Arabic}
\quranayah[37][26]
\end{Arabic}}
\flushleft{\begin{malayalam}
എന്നാല്‍ അവരിന്ന് കീഴൊതുങ്ങിയവരായിരിക്കും.
\end{malayalam}}
\flushright{\begin{Arabic}
\quranayah[37][27]
\end{Arabic}}
\flushleft{\begin{malayalam}
അവര്‍ ചേരിതിരിഞ്ഞ് പരസ്പരം ചോദ്യം ചെയ്യും.
\end{malayalam}}
\flushright{\begin{Arabic}
\quranayah[37][28]
\end{Arabic}}
\flushleft{\begin{malayalam}
അനുയായികള്‍ പറയും: "നിങ്ങള്‍ നന്മ ചമഞ്ഞ് ഞങ്ങളുടെ അടുത്ത് വന്നിരുന്നുവല്ലോ."
\end{malayalam}}
\flushright{\begin{Arabic}
\quranayah[37][29]
\end{Arabic}}
\flushleft{\begin{malayalam}
നേതാക്കള്‍ മറുപടി പറയും: "നിങ്ങള്‍ സ്വയംതന്നെ സത്യവിശ്വാസികളായിരുന്നില്ല.
\end{malayalam}}
\flushright{\begin{Arabic}
\quranayah[37][30]
\end{Arabic}}
\flushleft{\begin{malayalam}
"ഞങ്ങള്‍ക്ക് നിങ്ങളുടെമേല്‍ ഒരധികാരവുമുണ്ടായിരുന്നില്ലല്ലോ. എന്നല്ല; നിങ്ങള്‍ സ്വയംതന്നെ അതിക്രമികളായ ജനമായിരുന്നു.
\end{malayalam}}
\flushright{\begin{Arabic}
\quranayah[37][31]
\end{Arabic}}
\flushleft{\begin{malayalam}
"അങ്ങനെ നമ്മുടെ നാഥന്റെ ശിക്ഷയുടെ വചനം സത്യമായി പുലര്‍ന്നിരിക്കുന്നു. തീര്‍ച്ചയായും നാമതനുഭവിക്കാന്‍ പോവുകയാണ്.
\end{malayalam}}
\flushright{\begin{Arabic}
\quranayah[37][32]
\end{Arabic}}
\flushleft{\begin{malayalam}
"അങ്ങനെ ഞങ്ങള്‍ നിങ്ങളെ വഴിതെറ്റിച്ചു. തീര്‍ച്ചയായും ഞങ്ങള്‍ സ്വയം വഴിപിഴച്ചവരായിരുന്നു."
\end{malayalam}}
\flushright{\begin{Arabic}
\quranayah[37][33]
\end{Arabic}}
\flushleft{\begin{malayalam}
നിശ്ചയമായും അന്ന് അവരെല്ലാം ശിക്ഷയില്‍ പങ്കാളികളായിരിക്കും.
\end{malayalam}}
\flushright{\begin{Arabic}
\quranayah[37][34]
\end{Arabic}}
\flushleft{\begin{malayalam}
ഉറപ്പായും കുറ്റവാളികളോട് നാം അങ്ങനെതന്നെയാണ് ചെയ്യുക.
\end{malayalam}}
\flushright{\begin{Arabic}
\quranayah[37][35]
\end{Arabic}}
\flushleft{\begin{malayalam}
“അല്ലാഹുവല്ലാതെ ദൈവമില്ലെ”ന്ന് അവരോട് പറഞ്ഞാല്‍ അവര്‍ അഹങ്കാരത്തോടെ മുഖം തിരിക്കുമായിരുന്നു.
\end{malayalam}}
\flushright{\begin{Arabic}
\quranayah[37][36]
\end{Arabic}}
\flushleft{\begin{malayalam}
അവരിങ്ങനെ ചോദിക്കുമായിരുന്നു: "ഭ്രാന്തനായ ഒരു കവിക്കു വേണ്ടി ഞങ്ങള്‍ ഞങ്ങളുടെ ദൈവങ്ങളെ ഉപേക്ഷിക്കണമെന്നോ?"
\end{malayalam}}
\flushright{\begin{Arabic}
\quranayah[37][37]
\end{Arabic}}
\flushleft{\begin{malayalam}
എന്നാല്‍ സത്യവുമായാണ് അദ്ദേഹം വന്നെത്തിയത്. ദൈവദൂതന്മാരെയെല്ലാം അദ്ദേഹം ശരിവെച്ചിട്ടുമുണ്ട്.
\end{malayalam}}
\flushright{\begin{Arabic}
\quranayah[37][38]
\end{Arabic}}
\flushleft{\begin{malayalam}
തീര്‍ച്ചയായും നിങ്ങള്‍ നോവേറിയ ശിക്ഷ അനുഭവിക്കേണ്ടവര്‍ തന്നെ.
\end{malayalam}}
\flushright{\begin{Arabic}
\quranayah[37][39]
\end{Arabic}}
\flushleft{\begin{malayalam}
നിങ്ങള്‍ പ്രവര്‍ത്തിച്ചുകൊണ്ടിരുന്നതിന്റെ പ്രതിഫലം മാത്രമേ നിങ്ങള്‍ക്കു നല്‍കുകയുള്ളൂ.
\end{malayalam}}
\flushright{\begin{Arabic}
\quranayah[37][40]
\end{Arabic}}
\flushleft{\begin{malayalam}
അല്ലാഹുവിന്റെ ആത്മാര്‍ഥതയുള്ള അടിമകള്‍ക്കൊഴികെ.
\end{malayalam}}
\flushright{\begin{Arabic}
\quranayah[37][41]
\end{Arabic}}
\flushleft{\begin{malayalam}
അവര്‍ക്കാണ് അറിയപ്പെട്ട വിഭവങ്ങളുള്ളത്.
\end{malayalam}}
\flushright{\begin{Arabic}
\quranayah[37][42]
\end{Arabic}}
\flushleft{\begin{malayalam}
പലതരം പഴങ്ങള്‍. അവരവിടെ ആദരണീയരുമായിരിക്കും.
\end{malayalam}}
\flushright{\begin{Arabic}
\quranayah[37][43]
\end{Arabic}}
\flushleft{\begin{malayalam}
അനുഗൃഹീതമായ സ്വര്‍ഗീയാരാമങ്ങളില്‍.
\end{malayalam}}
\flushright{\begin{Arabic}
\quranayah[37][44]
\end{Arabic}}
\flushleft{\begin{malayalam}
മഞ്ചങ്ങളില്‍ അഭിമുഖമായി ഇരിക്കുന്നവരായിരിക്കും അവര്‍.
\end{malayalam}}
\flushright{\begin{Arabic}
\quranayah[37][45]
\end{Arabic}}
\flushleft{\begin{malayalam}
സവിശേഷമായ ഉറവുവെള്ളം നിറച്ച കോപ്പകള്‍ അവര്‍ക്കിടയില്‍ കറങ്ങിക്കൊണ്ടിരിക്കും.
\end{malayalam}}
\flushright{\begin{Arabic}
\quranayah[37][46]
\end{Arabic}}
\flushleft{\begin{malayalam}
വെളുത്തതും കുടിക്കുന്നവര്‍ക്ക് അത്യധികം ആസ്വാദ്യകരവുമായ പാനീയം.
\end{malayalam}}
\flushright{\begin{Arabic}
\quranayah[37][47]
\end{Arabic}}
\flushleft{\begin{malayalam}
അത് ദേഹത്തിനൊട്ടും ദോഷംവരുത്തില്ല. അതുവഴി അവര്‍ക്ക് ലഹരി ബാധിക്കുകയുമില്ല.
\end{malayalam}}
\flushright{\begin{Arabic}
\quranayah[37][48]
\end{Arabic}}
\flushleft{\begin{malayalam}
അവരുടെ അടുത്ത് നോട്ടം നിയന്ത്രിക്കുന്നവരും വിശാലാക്ഷികളുമായ കുലീനകളുണ്ടായിരിക്കും.
\end{malayalam}}
\flushright{\begin{Arabic}
\quranayah[37][49]
\end{Arabic}}
\flushleft{\begin{malayalam}
സൂക്ഷിക്കപ്പെട്ട മുട്ടകള്‍ പോലിരിക്കും അവര്‍.
\end{malayalam}}
\flushright{\begin{Arabic}
\quranayah[37][50]
\end{Arabic}}
\flushleft{\begin{malayalam}
അവര്‍ പരസ്പരം അഭിമുഖീകരിച്ച് അന്യോന്യം അന്വേഷിച്ചുകൊണ്ടിരിക്കും.
\end{malayalam}}
\flushright{\begin{Arabic}
\quranayah[37][51]
\end{Arabic}}
\flushleft{\begin{malayalam}
അവരിലൊരാള്‍ പറയും: "തീര്‍ച്ചയായും എനിക്കൊരു കൂട്ടുകാരനുണ്ടായിരുന്നു.
\end{malayalam}}
\flushright{\begin{Arabic}
\quranayah[37][52]
\end{Arabic}}
\flushleft{\begin{malayalam}
"അവന്‍ ചോദിക്കാറുണ്ടായിരുന്നു: “നീ പരലോകത്തെ ശരിവെക്കുന്നവനാണോ?
\end{malayalam}}
\flushright{\begin{Arabic}
\quranayah[37][53]
\end{Arabic}}
\flushleft{\begin{malayalam}
“നാം മരിച്ച് മണ്ണും എല്ലുമായി മാറിയാലും നമുക്ക് നമ്മുടെ കര്‍മഫലം കിട്ടുമെന്ന വാദത്തെ അംഗീകരിക്കുന്നവനും?"
\end{malayalam}}
\flushright{\begin{Arabic}
\quranayah[37][54]
\end{Arabic}}
\flushleft{\begin{malayalam}
തുടര്‍ന്ന് അയാള്‍ പറയും: "നിങ്ങള്‍ ആ സത്യനിഷേധിയെ എത്തിനോക്കുന്നുണ്ടോ?"
\end{malayalam}}
\flushright{\begin{Arabic}
\quranayah[37][55]
\end{Arabic}}
\flushleft{\begin{malayalam}
അങ്ങനെ അദ്ദേഹമവനെ എത്തിനോക്കും. അപ്പോള്‍ നരകത്തിന്റെ നടുവിലവനെ കാണും.
\end{malayalam}}
\flushright{\begin{Arabic}
\quranayah[37][56]
\end{Arabic}}
\flushleft{\begin{malayalam}
അദ്ദേഹമവനോട് പറയും: "അല്ലാഹു തന്നെ സത്യം. നീയെന്നെയും നശിപ്പിക്കുമായിരുന്നേനെ.
\end{malayalam}}
\flushright{\begin{Arabic}
\quranayah[37][57]
\end{Arabic}}
\flushleft{\begin{malayalam}
"എന്റെ നാഥന്റെ അനുഗ്രഹമില്ലായിരുന്നെങ്കില്‍ ഞാനും നരകത്തില്‍ ഹാജരാക്കപ്പെടുന്നവരില്‍ പെടുമായിരുന്നു.
\end{malayalam}}
\flushright{\begin{Arabic}
\quranayah[37][58]
\end{Arabic}}
\flushleft{\begin{malayalam}
"ഇനി നമുക്ക് മരണമില്ലല്ലോ.
\end{malayalam}}
\flushright{\begin{Arabic}
\quranayah[37][59]
\end{Arabic}}
\flushleft{\begin{malayalam}
"നമ്മുടെ ആദ്യത്തെ ആ മരണമല്ലാതെ. ഇനി നാം ശിക്ഷിക്കപ്പെടുകയുമില്ല."
\end{malayalam}}
\flushright{\begin{Arabic}
\quranayah[37][60]
\end{Arabic}}
\flushleft{\begin{malayalam}
തീര്‍ച്ചയായും ഇതുതന്നെയാണ് മഹത്തായ വിജയം.
\end{malayalam}}
\flushright{\begin{Arabic}
\quranayah[37][61]
\end{Arabic}}
\flushleft{\begin{malayalam}
ഇതുപോലുള്ള നേട്ടങ്ങള്‍ക്കുവേണ്ടിയാണ് പണിയെടുക്കുന്നവരൊക്കെയും ശ്രമിക്കേണ്ടത്.
\end{malayalam}}
\flushright{\begin{Arabic}
\quranayah[37][62]
\end{Arabic}}
\flushleft{\begin{malayalam}
ഇതോ അതോ സഖൂം മരമോ ഏതാണ് ഉത്തമമായ സല്‍ക്കാരം?
\end{malayalam}}
\flushright{\begin{Arabic}
\quranayah[37][63]
\end{Arabic}}
\flushleft{\begin{malayalam}
തീര്‍ച്ചയായും നാമതിനെ അക്രമികള്‍ക്കൊരു പരീക്ഷണമാക്കിയിരിക്കുന്നു.
\end{malayalam}}
\flushright{\begin{Arabic}
\quranayah[37][64]
\end{Arabic}}
\flushleft{\begin{malayalam}
നരകത്തിന്റെ അടിത്തട്ടില്‍നിന്ന് മുളച്ചുപൊങ്ങുന്ന മരമാണത്.
\end{malayalam}}
\flushright{\begin{Arabic}
\quranayah[37][65]
\end{Arabic}}
\flushleft{\begin{malayalam}
അതിന്റെ കുലകള്‍ ചെകുത്താന്മാരുടെ തലകള്‍ പോലിരിക്കും.
\end{malayalam}}
\flushright{\begin{Arabic}
\quranayah[37][66]
\end{Arabic}}
\flushleft{\begin{malayalam}
നരകവാസികള്‍ അത് തിന്നും. അങ്ങനെ അതുകൊണ്ട് അവര്‍ വയറ് നിറക്കും.
\end{malayalam}}
\flushright{\begin{Arabic}
\quranayah[37][67]
\end{Arabic}}
\flushleft{\begin{malayalam}
തുടര്‍ന്ന് അവര്‍ക്ക് അതിനുമീതെ കുടിക്കാന്‍ ചുട്ടുപൊള്ളുന്ന വെള്ളമാണ് കിട്ടുക.
\end{malayalam}}
\flushright{\begin{Arabic}
\quranayah[37][68]
\end{Arabic}}
\flushleft{\begin{malayalam}
പിന്നെ തീര്‍ച്ചയായും അവരുടെ മടക്കം നരകത്തീയിലേക്കുതന്നെ.
\end{malayalam}}
\flushright{\begin{Arabic}
\quranayah[37][69]
\end{Arabic}}
\flushleft{\begin{malayalam}
സംശയമില്ല; അവര്‍ തങ്ങളുടെ പൂര്‍വികരെ കണ്ടെത്തിയത് തീര്‍ത്തും വഴിപിഴച്ചവരായാണ്.
\end{malayalam}}
\flushright{\begin{Arabic}
\quranayah[37][70]
\end{Arabic}}
\flushleft{\begin{malayalam}
എന്നിട്ടും അവര്‍ ആ പൂര്‍വികരുടെ കാല്‍പ്പാടുകള്‍ തന്നെ താല്‍പര്യത്തോടെ പിന്തുടര്‍ന്നു.
\end{malayalam}}
\flushright{\begin{Arabic}
\quranayah[37][71]
\end{Arabic}}
\flushleft{\begin{malayalam}
അവര്‍ക്കുമുമ്പെ അവരുടെ പൂര്‍വികരിലേറെ പേരും വഴിപിഴച്ചിരുന്നു.
\end{malayalam}}
\flushright{\begin{Arabic}
\quranayah[37][72]
\end{Arabic}}
\flushleft{\begin{malayalam}
അവരില്‍ നാം മുന്നറിയിപ്പുകാരെ അയച്ചിട്ടുണ്ടായിരുന്നു.
\end{malayalam}}
\flushright{\begin{Arabic}
\quranayah[37][73]
\end{Arabic}}
\flushleft{\begin{malayalam}
നോക്കൂ; ആ മുന്നറിയിപ്പ് നല്‍കപ്പെട്ടവരുടെ അന്ത്യം എവ്വിധമായിരുന്നുവെന്ന്.
\end{malayalam}}
\flushright{\begin{Arabic}
\quranayah[37][74]
\end{Arabic}}
\flushleft{\begin{malayalam}
അല്ലാഹുവിന്റെ ആത്മാര്‍ഥതയുള്ള അടിമകളുടേതൊഴികെ.
\end{malayalam}}
\flushright{\begin{Arabic}
\quranayah[37][75]
\end{Arabic}}
\flushleft{\begin{malayalam}
നൂഹ് നമ്മോട് പ്രാര്‍ഥിച്ചു. അപ്പോള്‍ ഉത്തരം നല്‍കിയവന്‍ എത്ര അനുഗൃഹീതന്‍.
\end{malayalam}}
\flushright{\begin{Arabic}
\quranayah[37][76]
\end{Arabic}}
\flushleft{\begin{malayalam}
അദ്ദേഹത്തെയും അദ്ദേഹത്തിന്റെ കൂടെയുള്ളവരെയും നാം വന്‍ ദുരന്തത്തില്‍നിന്ന് രക്ഷിച്ചു.
\end{malayalam}}
\flushright{\begin{Arabic}
\quranayah[37][77]
\end{Arabic}}
\flushleft{\begin{malayalam}
അദ്ദേഹത്തിന്റെ സന്തതികളെ നാം ഭൂമിയില്‍ ബാക്കിയാക്കി.
\end{malayalam}}
\flushright{\begin{Arabic}
\quranayah[37][78]
\end{Arabic}}
\flushleft{\begin{malayalam}
പിന്നാലെ വന്നവരില്‍ അദ്ദേഹത്തിന്റെ സല്‍ക്കീര്‍ത്തി നിലനിര്‍ത്തി.
\end{malayalam}}
\flushright{\begin{Arabic}
\quranayah[37][79]
\end{Arabic}}
\flushleft{\begin{malayalam}
മുഴുവന്‍ ലോകവാസികളിലും നൂഹിന് സമാധാനം.
\end{malayalam}}
\flushright{\begin{Arabic}
\quranayah[37][80]
\end{Arabic}}
\flushleft{\begin{malayalam}
തീര്‍ച്ചയായും അവ്വിധമാണ് നാം സദ്വൃത്തര്‍ക്ക് പ്രതിഫലം നല്‍കുക.
\end{malayalam}}
\flushright{\begin{Arabic}
\quranayah[37][81]
\end{Arabic}}
\flushleft{\begin{malayalam}
സംശയമില്ല; അദ്ദേഹം നമ്മുടെ സത്യവിശ്വാസികളായ ദാസന്മാരില്‍ പെട്ടവനാണ്.
\end{malayalam}}
\flushright{\begin{Arabic}
\quranayah[37][82]
\end{Arabic}}
\flushleft{\begin{malayalam}
പിന്നീട് മറ്റുള്ളവരെ നാം മുക്കിക്കൊന്നു.
\end{malayalam}}
\flushright{\begin{Arabic}
\quranayah[37][83]
\end{Arabic}}
\flushleft{\begin{malayalam}
ഉറപ്പായും അദ്ദേഹത്തിന്റെ കക്ഷിയില്‍പെട്ടവന്‍ തന്നെയാണ് ഇബ്റാഹീം.
\end{malayalam}}
\flushright{\begin{Arabic}
\quranayah[37][84]
\end{Arabic}}
\flushleft{\begin{malayalam}
ശുദ്ധഹൃദയനായി അദ്ദേഹം തന്റെ നാഥന്റെ സന്നിധിയില്‍ ചെന്ന സന്ദര്‍ഭം:
\end{malayalam}}
\flushright{\begin{Arabic}
\quranayah[37][85]
\end{Arabic}}
\flushleft{\begin{malayalam}
അദ്ദേഹം തന്റെ പിതാവിനോടും ജനതയോടും ചോദിച്ചു: "നിങ്ങള്‍ എന്തിനെയാണ് പൂജിച്ചുകൊണ്ടിരിക്കുന്നത്?
\end{malayalam}}
\flushright{\begin{Arabic}
\quranayah[37][86]
\end{Arabic}}
\flushleft{\begin{malayalam}
"അല്ലാഹുവെക്കൂടാതെ വ്യാജദൈവങ്ങളെ പൂജിക്കാനാണോ നിങ്ങളാഗ്രഹിക്കുന്നത്?
\end{malayalam}}
\flushright{\begin{Arabic}
\quranayah[37][87]
\end{Arabic}}
\flushleft{\begin{malayalam}
"അപ്പോള്‍ പ്രപഞ്ചനാഥനെപ്പറ്റി നിങ്ങളുടെ വിചാരമെന്താണ്?"
\end{malayalam}}
\flushright{\begin{Arabic}
\quranayah[37][88]
\end{Arabic}}
\flushleft{\begin{malayalam}
പിന്നെ അദ്ദേഹം നക്ഷത്രങ്ങളുടെ നേരെ നോക്കി.
\end{malayalam}}
\flushright{\begin{Arabic}
\quranayah[37][89]
\end{Arabic}}
\flushleft{\begin{malayalam}
എന്നിട്ടിങ്ങനെ പറഞ്ഞു: "എനിക്കു സുഖമില്ല."
\end{malayalam}}
\flushright{\begin{Arabic}
\quranayah[37][90]
\end{Arabic}}
\flushleft{\begin{malayalam}
അപ്പോള്‍ അവര്‍ അദ്ദേഹത്തെ വിട്ട് പിരിഞ്ഞുപോയി.
\end{malayalam}}
\flushright{\begin{Arabic}
\quranayah[37][91]
\end{Arabic}}
\flushleft{\begin{malayalam}
അങ്ങനെ അദ്ദേഹം അവരുടെ ദൈവങ്ങളുടെനേരെ തിരിഞ്ഞു. അദ്ദേഹം ചോദിച്ചു: "നിങ്ങള്‍ തിന്നുന്നില്ലേ?
\end{malayalam}}
\flushright{\begin{Arabic}
\quranayah[37][92]
\end{Arabic}}
\flushleft{\begin{malayalam}
"നിങ്ങള്‍ക്കെന്തുപറ്റി? നിങ്ങളൊന്നും മിണ്ടുന്നില്ലല്ലോ!?"
\end{malayalam}}
\flushright{\begin{Arabic}
\quranayah[37][93]
\end{Arabic}}
\flushleft{\begin{malayalam}
പിന്നീട് അദ്ദേഹം അവയുടെ നേരെ നീങ്ങി. അങ്ങനെ തന്റെ വലംകൈകൊണ്ട് അവയെ വെട്ടിവീഴ്ത്തി.
\end{malayalam}}
\flushright{\begin{Arabic}
\quranayah[37][94]
\end{Arabic}}
\flushleft{\begin{malayalam}
ആളുകള്‍ അദ്ദേഹത്തിന്റെ നേരെ പാഞ്ഞടുത്തു.
\end{malayalam}}
\flushright{\begin{Arabic}
\quranayah[37][95]
\end{Arabic}}
\flushleft{\begin{malayalam}
അദ്ദേഹം ചോദിച്ചു: "നിങ്ങള്‍ തന്നെ ചെത്തിയുണ്ടാക്കുന്നവയെയാണോ നിങ്ങള്‍ പൂജിക്കുന്നത്?
\end{malayalam}}
\flushright{\begin{Arabic}
\quranayah[37][96]
\end{Arabic}}
\flushleft{\begin{malayalam}
"അല്ലാഹുവാണല്ലോ നിങ്ങളെയും നിങ്ങള്‍ നിര്‍മിക്കുന്നവയെയും സൃഷ്ടിച്ചത്."
\end{malayalam}}
\flushright{\begin{Arabic}
\quranayah[37][97]
\end{Arabic}}
\flushleft{\begin{malayalam}
അവര്‍ പരസ്പരം പറഞ്ഞു: "ഇവനുവേണ്ടി ഒരു തീക്കുണ്ഡമുണ്ടാക്കുക. എന്നിട്ടിവനെ കത്തിക്കാളുന്ന തിയ്യിലെറിയുക."
\end{malayalam}}
\flushright{\begin{Arabic}
\quranayah[37][98]
\end{Arabic}}
\flushleft{\begin{malayalam}
അങ്ങനെ അവരദ്ദേഹത്തിനെതിരെ തന്ത്രം മെനഞ്ഞു. പക്ഷേ, നാമവരെ പറ്റെ പതിതരാക്കി.
\end{malayalam}}
\flushright{\begin{Arabic}
\quranayah[37][99]
\end{Arabic}}
\flushleft{\begin{malayalam}
ഇബ്റാഹീം പറഞ്ഞു: "ഞാനെന്റെ നാഥന്റെ അടുത്തേക്കു പോവുകയാണ്. അവനെന്നെ നേര്‍വഴിയില്‍ നയിക്കും.
\end{malayalam}}
\flushright{\begin{Arabic}
\quranayah[37][100]
\end{Arabic}}
\flushleft{\begin{malayalam}
"എന്റെ നാഥാ, എനിക്കു നീ സച്ചരിതനായ ഒരു മകനെ നല്‍കേണമേ."
\end{malayalam}}
\flushright{\begin{Arabic}
\quranayah[37][101]
\end{Arabic}}
\flushleft{\begin{malayalam}
അപ്പോള്‍ നാം അദ്ദേഹത്തെ സഹനശാലിയായ ഒരു പുത്രനെ സംബന്ധിച്ച ശുഭവാര്‍ത്ത അറിയിച്ചു.
\end{malayalam}}
\flushright{\begin{Arabic}
\quranayah[37][102]
\end{Arabic}}
\flushleft{\begin{malayalam}
ആ കുട്ടി അദ്ദേഹത്തോടൊപ്പം എന്തെങ്കിലും ചെയ്യാവുന്ന പ്രായമെത്തിയപ്പോള്‍ അദ്ദേഹം പറഞ്ഞു: "എന്റെ പ്രിയ മോനേ, ഞാന്‍ നിന്നെ അറുക്കുന്നതായി സ്വപ്നം കണ്ടിരിക്കുന്നു. അതിനാല്‍ നോക്കൂ; നിന്റെ അഭിപ്രായമെന്താണ്." അവന്‍ പറഞ്ഞു: "എന്റുപ്പാ, അങ്ങ് കല്‍പന നടപ്പാക്കിയാലും. അല്ലാഹു ഇച്ഛിച്ചെങ്കില്‍ ക്ഷമാശീലരുടെ കൂട്ടത്തില്‍ അങ്ങയ്ക്കെന്നെ കാണാം."
\end{malayalam}}
\flushright{\begin{Arabic}
\quranayah[37][103]
\end{Arabic}}
\flushleft{\begin{malayalam}
അങ്ങനെ അവരിരുവരും കല്‍പനക്കു വഴങ്ങി. അദ്ദേഹമവനെ ചെരിച്ചു കിടത്തി.
\end{malayalam}}
\flushright{\begin{Arabic}
\quranayah[37][104]
\end{Arabic}}
\flushleft{\begin{malayalam}
അപ്പോള്‍ നാം അദ്ദേഹത്തെ വിളിച്ചു: "ഇബ്റാഹീമേ,
\end{malayalam}}
\flushright{\begin{Arabic}
\quranayah[37][105]
\end{Arabic}}
\flushleft{\begin{malayalam}
"സംശയമില്ല; നീ സ്വപ്നം സാക്ഷാല്‍ക്കരിച്ചിരിക്കുന്നു." അവ്വിധമാണ് നാം സച്ചരിതര്‍ക്ക് പ്രതിഫലം നല്‍കുന്നത്.
\end{malayalam}}
\flushright{\begin{Arabic}
\quranayah[37][106]
\end{Arabic}}
\flushleft{\begin{malayalam}
ഉറപ്പായും ഇതൊരു വ്യക്തമായ പരീക്ഷണം തന്നെയായിരുന്നു.
\end{malayalam}}
\flushright{\begin{Arabic}
\quranayah[37][107]
\end{Arabic}}
\flushleft{\begin{malayalam}
നാം അവനുപകരം ബലിയര്‍പ്പിക്കാനായി മഹത്തായ ഒരു മൃഗത്തെ നല്‍കി.
\end{malayalam}}
\flushright{\begin{Arabic}
\quranayah[37][108]
\end{Arabic}}
\flushleft{\begin{malayalam}
പിന്മുറക്കാരില്‍ അദ്ദേഹത്തിന്റെ സല്‍ക്കീര്‍ത്തി നിലനിര്‍ത്തുകയും ചെയ്തു.
\end{malayalam}}
\flushright{\begin{Arabic}
\quranayah[37][109]
\end{Arabic}}
\flushleft{\begin{malayalam}
ഇബ്റാഹീമിനു സമാധാനം.
\end{malayalam}}
\flushright{\begin{Arabic}
\quranayah[37][110]
\end{Arabic}}
\flushleft{\begin{malayalam}
ഇവ്വിധമാണ് നാം സച്ചരിതര്‍ക്ക് പ്രതിഫലം നല്‍കുന്നത്.
\end{malayalam}}
\flushright{\begin{Arabic}
\quranayah[37][111]
\end{Arabic}}
\flushleft{\begin{malayalam}
സംശയമില്ല; അദ്ദേഹം നമ്മുടെ സത്യവിശ്വാസികളായ ദാസന്മാരില്‍ പെട്ടവനായിരുന്നു.
\end{malayalam}}
\flushright{\begin{Arabic}
\quranayah[37][112]
\end{Arabic}}
\flushleft{\begin{malayalam}
അദ്ദേഹത്തെ നാം, സച്ചരിതരില്‍പെട്ട പ്രവാചകനാകാന്‍ പോകുന്ന ഇസ്ഹാഖിന്റെ ജനനത്തെ സംബന്ധിച്ചും ശുഭവാര്‍ത്ത അറിയിച്ചു.
\end{malayalam}}
\flushright{\begin{Arabic}
\quranayah[37][113]
\end{Arabic}}
\flushleft{\begin{malayalam}
അദ്ദേഹത്തെയും ഇസ്ഹാഖിനെയും നാം അനുഗ്രഹിച്ചു. അവരിരുവരുടെയും സന്താനങ്ങളില്‍ നല്ലവരുണ്ട്. തന്നോടുതന്നെ വ്യക്തമായ അതിക്രമം ചെയ്യുന്നവരുമുണ്ട്.
\end{malayalam}}
\flushright{\begin{Arabic}
\quranayah[37][114]
\end{Arabic}}
\flushleft{\begin{malayalam}
നിശ്ചയമായും മൂസായോടും ഹാറൂനോടും നാം അളവറ്റ ഔദാര്യം കാണിച്ചു.
\end{malayalam}}
\flushright{\begin{Arabic}
\quranayah[37][115]
\end{Arabic}}
\flushleft{\begin{malayalam}
അവരിരുവരെയും അവരുടെ ജനതയെയും കൊടുംവിപത്തില്‍ നിന്ന് രക്ഷപ്പെടുത്തി.
\end{malayalam}}
\flushright{\begin{Arabic}
\quranayah[37][116]
\end{Arabic}}
\flushleft{\begin{malayalam}
അവരെ നാം സഹായിച്ചു. അങ്ങനെ അവര്‍ വിജയികളായിത്തീര്‍ന്നു.
\end{malayalam}}
\flushright{\begin{Arabic}
\quranayah[37][117]
\end{Arabic}}
\flushleft{\begin{malayalam}
അവരിരുവര്‍ക്കും സത്യം വേര്‍തിരിച്ചു കാണിക്കുന്ന വേദപുസ്തകം നല്‍കി.
\end{malayalam}}
\flushright{\begin{Arabic}
\quranayah[37][118]
\end{Arabic}}
\flushleft{\begin{malayalam}
ഇരുവരെയും നാം നേര്‍വഴിയില്‍ നയിക്കുകയും ചെയ്തു.
\end{malayalam}}
\flushright{\begin{Arabic}
\quranayah[37][119]
\end{Arabic}}
\flushleft{\begin{malayalam}
പിന്മുറക്കാരില്‍ നാം അവരുടെ സല്‍ക്കീര്‍ത്തി നിലനിര്‍ത്തി.
\end{malayalam}}
\flushright{\begin{Arabic}
\quranayah[37][120]
\end{Arabic}}
\flushleft{\begin{malayalam}
മൂസായ്ക്കും ഹാറൂന്നും സമാധാനം!
\end{malayalam}}
\flushright{\begin{Arabic}
\quranayah[37][121]
\end{Arabic}}
\flushleft{\begin{malayalam}
അവ്വിധമാണ് നാം സച്ചരിതര്‍ക്ക് പ്രതിഫലം നല്‍കുന്നത്.
\end{malayalam}}
\flushright{\begin{Arabic}
\quranayah[37][122]
\end{Arabic}}
\flushleft{\begin{malayalam}
അവരിരുവരും സത്യവിശ്വാസികളായ നമ്മുടെ ദാസന്മാരില്‍ പെട്ടവരായിരുന്നു.
\end{malayalam}}
\flushright{\begin{Arabic}
\quranayah[37][123]
\end{Arabic}}
\flushleft{\begin{malayalam}
സംശയമില്ല; ഇല്‍യാസും ദൈവദൂതന്മാരിലൊരാളാണ്.
\end{malayalam}}
\flushright{\begin{Arabic}
\quranayah[37][124]
\end{Arabic}}
\flushleft{\begin{malayalam}
അദ്ദേഹം തന്റെ ജനതയോടു ചോദിച്ച സന്ദര്‍ഭം: "നിങ്ങള്‍ ഭക്തിപുലര്‍ത്തുന്നില്ലേ?
\end{malayalam}}
\flushright{\begin{Arabic}
\quranayah[37][125]
\end{Arabic}}
\flushleft{\begin{malayalam}
"നിങ്ങള്‍ ബഅ്ലിനെ വിളിച്ച് പ്രാര്‍ഥിക്കുകയാണോ? ഏറ്റവും ശ്രേഷ്ഠനായ സൃഷ്ടികര്‍ത്താവിനെ ഉപേക്ഷിക്കുകയും?
\end{malayalam}}
\flushright{\begin{Arabic}
\quranayah[37][126]
\end{Arabic}}
\flushleft{\begin{malayalam}
"നിങ്ങളുടെയും നിങ്ങളുടെ പൂര്‍വ പിതാക്കളുടെയും നാഥനായ അല്ലാഹുവെ?"
\end{malayalam}}
\flushright{\begin{Arabic}
\quranayah[37][127]
\end{Arabic}}
\flushleft{\begin{malayalam}
അപ്പോള്‍ അവരദ്ദേഹത്തെ തള്ളിപ്പറഞ്ഞു. അതിനാലവര്‍ ശിക്ഷയ്ക്ക് കൊണ്ടുവരപ്പെടും; തീര്‍ച്ച.
\end{malayalam}}
\flushright{\begin{Arabic}
\quranayah[37][128]
\end{Arabic}}
\flushleft{\begin{malayalam}
അല്ലാഹുവിന്റെ ആത്മാര്‍ഥതയുള്ള അടിമകളെയൊഴികെ.
\end{malayalam}}
\flushright{\begin{Arabic}
\quranayah[37][129]
\end{Arabic}}
\flushleft{\begin{malayalam}
പിന്മുറക്കാരില്‍ അദ്ദേഹത്തിന്റെ സല്‍ക്കീര്‍ത്തി നാം നിലനിര്‍ത്തി.
\end{malayalam}}
\flushright{\begin{Arabic}
\quranayah[37][130]
\end{Arabic}}
\flushleft{\begin{malayalam}
ഇല്‍യാസിന് സമാധാനം.
\end{malayalam}}
\flushright{\begin{Arabic}
\quranayah[37][131]
\end{Arabic}}
\flushleft{\begin{malayalam}
അവ്വിധമാണ് നാം സച്ചരിതര്‍ക്ക് പ്രതിഫലം നല്‍കുന്നത്.
\end{malayalam}}
\flushright{\begin{Arabic}
\quranayah[37][132]
\end{Arabic}}
\flushleft{\begin{malayalam}
സംശയമില്ല; അദ്ദേഹം നമ്മുടെ സത്യവിശ്വാസികളായ ദാസന്മാരില്‍ പെട്ടവനായിരുന്നു.
\end{malayalam}}
\flushright{\begin{Arabic}
\quranayah[37][133]
\end{Arabic}}
\flushleft{\begin{malayalam}
ലൂത്വും ദൈവദൂതരിലൊരുവന്‍തന്നെ!
\end{malayalam}}
\flushright{\begin{Arabic}
\quranayah[37][134]
\end{Arabic}}
\flushleft{\begin{malayalam}
അദ്ദേഹത്തെയും അദ്ദേഹത്തിന്റെ മുഴുവന്‍ ആള്‍ക്കാരെയും നാം രക്ഷപ്പെടുത്തി.
\end{malayalam}}
\flushright{\begin{Arabic}
\quranayah[37][135]
\end{Arabic}}
\flushleft{\begin{malayalam}
പിറകില്‍ മാറിനിന്ന ഒരു കിഴവിയെ ഒഴികെ.
\end{malayalam}}
\flushright{\begin{Arabic}
\quranayah[37][136]
\end{Arabic}}
\flushleft{\begin{malayalam}
പിന്നെ മറ്റുള്ളവരെയെല്ലാം നാം നശിപ്പിച്ചു.
\end{malayalam}}
\flushright{\begin{Arabic}
\quranayah[37][137]
\end{Arabic}}
\flushleft{\begin{malayalam}
തീര്‍ച്ചയായും നിങ്ങള്‍ പ്രഭാതവേളയില്‍ അവരുടെ അരികിലൂടെ കടന്നുപോകുന്നു;
\end{malayalam}}
\flushright{\begin{Arabic}
\quranayah[37][138]
\end{Arabic}}
\flushleft{\begin{malayalam}
വൈകുന്നേരവും. എന്നിട്ടും നിങ്ങളൊന്നും ചിന്തിച്ചറിയുന്നില്ലേ?
\end{malayalam}}
\flushright{\begin{Arabic}
\quranayah[37][139]
\end{Arabic}}
\flushleft{\begin{malayalam}
സംശയമില്ല; യൂനുസും ദൈവദൂതന്മാരിലൊരുവന്‍ തന്നെ.
\end{malayalam}}
\flushright{\begin{Arabic}
\quranayah[37][140]
\end{Arabic}}
\flushleft{\begin{malayalam}
ഭാരംനിറച്ച കപ്പലിലേക്ക് അദ്ദേഹം ഒളിച്ചുകയറിയതോര്‍ക്കുക.
\end{malayalam}}
\flushright{\begin{Arabic}
\quranayah[37][141]
\end{Arabic}}
\flushleft{\begin{malayalam}
അങ്ങനെ അദ്ദേഹം നറുക്കെടുപ്പില്‍ പങ്കാളിയായി. അതോടെ പുറന്തള്ളപ്പെട്ടവരിലൊരുവനായി.
\end{malayalam}}
\flushright{\begin{Arabic}
\quranayah[37][142]
\end{Arabic}}
\flushleft{\begin{malayalam}
അപ്പോള്‍ അദ്ദേഹത്തെ മത്സ്യം വിഴുങ്ങി. അദ്ദേഹം ആക്ഷേപാര്‍ഹനായിരുന്നു.
\end{malayalam}}
\flushright{\begin{Arabic}
\quranayah[37][143]
\end{Arabic}}
\flushleft{\begin{malayalam}
അദ്ദേഹം അല്ലാഹുവിന്റെ വിശുദ്ധി വാഴ്ത്തുന്നവരില്‍ പെട്ടവനായിരുന്നില്ലെങ്കില്‍;
\end{malayalam}}
\flushright{\begin{Arabic}
\quranayah[37][144]
\end{Arabic}}
\flushleft{\begin{malayalam}
നിശ്ചയമായും ഉയിര്‍ത്തെഴുന്നേല്‍പുനാള്‍ വരെയും അതിന്റെ വയറ്റില്‍ കഴിയേണ്ടിവരുമായിരുന്നു.
\end{malayalam}}
\flushright{\begin{Arabic}
\quranayah[37][145]
\end{Arabic}}
\flushleft{\begin{malayalam}
പിന്നീട് അദ്ദേഹത്തെ നാം കടലോരത്തെ ഒരു വെളിപ്രദേശത്തേക്കു തള്ളി. അദ്ദേഹമപ്പോള്‍ രോഗിയായിരുന്നു.
\end{malayalam}}
\flushright{\begin{Arabic}
\quranayah[37][146]
\end{Arabic}}
\flushleft{\begin{malayalam}
അദ്ദേഹത്തിനു നാം ഒരു വള്ളിച്ചെടി മുളപ്പിച്ചുകൊടുത്തു.
\end{malayalam}}
\flushright{\begin{Arabic}
\quranayah[37][147]
\end{Arabic}}
\flushleft{\begin{malayalam}
അദ്ദേഹത്തെ നാം ഒരു ലക്ഷമോ അതിലേറെയോ വരുന്ന വമ്പിച്ച ഒരാള്‍ക്കൂട്ടത്തിലേക്കയച്ചു.
\end{malayalam}}
\flushright{\begin{Arabic}
\quranayah[37][148]
\end{Arabic}}
\flushleft{\begin{malayalam}
അതോടെ അവരെല്ലാം വിശ്വസിച്ചു. അതിനാല്‍ ഒരു നിശ്ചിത കാലംവരെ നാമവര്‍ക്ക് സുഖജീവിതം നല്‍കി.
\end{malayalam}}
\flushright{\begin{Arabic}
\quranayah[37][149]
\end{Arabic}}
\flushleft{\begin{malayalam}
നബിയേ, ഈ ജനത്തോടൊന്ന് ചോദിച്ചു നോക്കൂ: "നിന്റെ നാഥന്ന് പെണ്‍മക്കളും അവര്‍ക്ക് ആണ്‍മക്കളുമാണോ എന്ന്."
\end{malayalam}}
\flushright{\begin{Arabic}
\quranayah[37][150]
\end{Arabic}}
\flushleft{\begin{malayalam}
"അതല്ല; നാം മലക്കുകളെ സ്ത്രീകളായി സൃഷ്ടിച്ചുവെന്നോ? അവരതിന് സാക്ഷികളായിരുന്നോ?"
\end{malayalam}}
\flushright{\begin{Arabic}
\quranayah[37][151]
\end{Arabic}}
\flushleft{\begin{malayalam}
അറിയുക: അവരിപ്പറയുന്നത് അവര്‍ കെട്ടിച്ചമച്ചുണ്ടാക്കിയതില്‍ പെട്ടതാണ്;
\end{malayalam}}
\flushright{\begin{Arabic}
\quranayah[37][152]
\end{Arabic}}
\flushleft{\begin{malayalam}
“അല്ലാഹു മക്കളെ ജനിപ്പിച്ചു”വെന്നത്. സംശയമില്ല; അവര്‍ കള്ളം പറയുന്നവര്‍ തന്നെയാണ്.
\end{malayalam}}
\flushright{\begin{Arabic}
\quranayah[37][153]
\end{Arabic}}
\flushleft{\begin{malayalam}
അല്ലാഹു തനിക്കായി ആണ്‍മക്കളെക്കാള്‍ പെണ്‍മക്കളെ തെരഞ്ഞെടുത്തെന്നോ?
\end{malayalam}}
\flushright{\begin{Arabic}
\quranayah[37][154]
\end{Arabic}}
\flushleft{\begin{malayalam}
നിങ്ങള്‍ക്കെന്തുപറ്റി? എങ്ങനെയൊക്കെയാണ് നിങ്ങള്‍ തീര്‍പ്പുകല്‍പിക്കുന്നത്?
\end{malayalam}}
\flushright{\begin{Arabic}
\quranayah[37][155]
\end{Arabic}}
\flushleft{\begin{malayalam}
നിങ്ങള്‍ ചിന്തിച്ചറിയുന്നില്ലേ?
\end{malayalam}}
\flushright{\begin{Arabic}
\quranayah[37][156]
\end{Arabic}}
\flushleft{\begin{malayalam}
അതല്ലെങ്കില്‍ നിങ്ങളുടെ വശം വ്യക്തമായ വല്ല പ്രമാണവുമുണ്ടോ?
\end{malayalam}}
\flushright{\begin{Arabic}
\quranayah[37][157]
\end{Arabic}}
\flushleft{\begin{malayalam}
എങ്കില്‍ നിങ്ങള്‍ നിങ്ങളുടെ ആ രേഖയിങ്ങു കൊണ്ടുവരിക. നിങ്ങള്‍ സത്യവാന്മാരെങ്കില്‍!
\end{malayalam}}
\flushright{\begin{Arabic}
\quranayah[37][158]
\end{Arabic}}
\flushleft{\begin{malayalam}
ഇക്കൂട്ടര്‍ അല്ലാഹുവിനും ജിന്നുകള്‍ക്കുമിടയില്‍ കുടുംബബന്ധമാരോപിച്ചിരിക്കുന്നു. എന്നാല്‍ ജിന്നുകള്‍ക്കറിയാം; തങ്ങള്‍ ശിക്ഷക്ക് ഹാജരാക്കപ്പെടുമെന്ന്.
\end{malayalam}}
\flushright{\begin{Arabic}
\quranayah[37][159]
\end{Arabic}}
\flushleft{\begin{malayalam}
അവരാരോപിക്കുന്നതില്‍ നിന്നെല്ലാം അല്ലാഹു എത്രയോ പരിശുദ്ധനാണ്.
\end{malayalam}}
\flushright{\begin{Arabic}
\quranayah[37][160]
\end{Arabic}}
\flushleft{\begin{malayalam}
അല്ലാഹുവിന്റെ ആത്മാര്‍ഥതയുള്ള അടിമകള്‍ ഇവരില്‍പെട്ടവരല്ല.
\end{malayalam}}
\flushright{\begin{Arabic}
\quranayah[37][161]
\end{Arabic}}
\flushleft{\begin{malayalam}
എന്നാല്‍ തീര്‍ച്ചയായും നിങ്ങള്‍ക്കും നിങ്ങളുടെ ആരാധ്യര്‍ക്കും,
\end{malayalam}}
\flushright{\begin{Arabic}
\quranayah[37][162]
\end{Arabic}}
\flushleft{\begin{malayalam}
അല്ലാഹുവിനെതിരില്‍ ആരെയും കുഴപ്പത്തിലാക്കാനാവില്ല;
\end{malayalam}}
\flushright{\begin{Arabic}
\quranayah[37][163]
\end{Arabic}}
\flushleft{\begin{malayalam}
കത്തിക്കാളുന്ന നരകത്തീയില്‍ വെന്തെരിയേണ്ടവരെയല്ലാതെ.
\end{malayalam}}
\flushright{\begin{Arabic}
\quranayah[37][164]
\end{Arabic}}
\flushleft{\begin{malayalam}
"നിര്‍ണിതമായ സ്ഥാനമില്ലാത്ത ആരുംതന്നെ ഞങ്ങളിലില്ല.
\end{malayalam}}
\flushright{\begin{Arabic}
\quranayah[37][165]
\end{Arabic}}
\flushleft{\begin{malayalam}
"തീര്‍ച്ചയായും ഞങ്ങള്‍ സേവനത്തിനായി അണിനിന്നവരാണ്.
\end{malayalam}}
\flushright{\begin{Arabic}
\quranayah[37][166]
\end{Arabic}}
\flushleft{\begin{malayalam}
"നിശ്ചയമായും ഞങ്ങള്‍ അല്ലാഹുവിന്റെ വിശുദ്ധി വാഴ്ത്തുന്നവരുമാണ്."
\end{malayalam}}
\flushright{\begin{Arabic}
\quranayah[37][167]
\end{Arabic}}
\flushleft{\begin{malayalam}
ഇക്കൂട്ടര്‍ പറയാറുണ്ടായിരുന്നു:
\end{malayalam}}
\flushright{\begin{Arabic}
\quranayah[37][168]
\end{Arabic}}
\flushleft{\begin{malayalam}
"മുന്‍ഗാമികള്‍ക്കു കിട്ടിയ വല്ല ഉദ്ബോധനവും ഞങ്ങളുടെ വശം ഉണ്ടായിരുന്നെങ്കില്‍;
\end{malayalam}}
\flushright{\begin{Arabic}
\quranayah[37][169]
\end{Arabic}}
\flushleft{\begin{malayalam}
"ഞങ്ങള്‍ അല്ലാഹുവിന്റെ ആത്മാര്‍ഥതയുള്ള അടിമകളാകുമായിരുന്നു."
\end{malayalam}}
\flushright{\begin{Arabic}
\quranayah[37][170]
\end{Arabic}}
\flushleft{\begin{malayalam}
എന്നിട്ടും അവരിതിനെ തള്ളിപ്പറയുകയാണുണ്ടായത്. അതിനാല്‍ അടുത്തുതന്നെ അവരെല്ലാം അറിയും!
\end{malayalam}}
\flushright{\begin{Arabic}
\quranayah[37][171]
\end{Arabic}}
\flushleft{\begin{malayalam}
ദൂതന്മാരായി അയച്ച നമ്മുടെ ദാസന്മാരുടെ കാര്യത്തില്‍ നമ്മുടെ കല്‍പന നേരത്തെ വന്നുകഴിഞ്ഞിട്ടുണ്ട്:
\end{malayalam}}
\flushright{\begin{Arabic}
\quranayah[37][172]
\end{Arabic}}
\flushleft{\begin{malayalam}
"ഉറപ്പായും അവര്‍ക്ക് സഹായം ലഭിക്കു"മെന്ന്.
\end{malayalam}}
\flushright{\begin{Arabic}
\quranayah[37][173]
\end{Arabic}}
\flushleft{\begin{malayalam}
തീര്‍ച്ചയായും നമ്മുടെ സൈന്യം തന്നെയാണ് വിജയിക്കുന്നവര്‍.
\end{malayalam}}
\flushright{\begin{Arabic}
\quranayah[37][174]
\end{Arabic}}
\flushleft{\begin{malayalam}
അതിനാല്‍ ഒരവധിവരെ നീ അവരില്‍ നിന്ന് മാറിനില്‍ക്കുക.
\end{malayalam}}
\flushright{\begin{Arabic}
\quranayah[37][175]
\end{Arabic}}
\flushleft{\begin{malayalam}
അവരെ നീ ശ്രദ്ധിച്ചുകൊണ്ടിരിക്കുക. അടുത്തുതന്നെ അവരെല്ലാം കണ്ടറിഞ്ഞുകൊള്ളും.
\end{malayalam}}
\flushright{\begin{Arabic}
\quranayah[37][176]
\end{Arabic}}
\flushleft{\begin{malayalam}
നമ്മുടെ ശിക്ഷക്കുവേണ്ടിയാണോ ഇവരിങ്ങനെ തിടുക്കം കൂട്ടുന്നത്?
\end{malayalam}}
\flushright{\begin{Arabic}
\quranayah[37][177]
\end{Arabic}}
\flushleft{\begin{malayalam}
എന്നാല്‍ ആ ശിക്ഷ അവരുടെ മുറ്റത്ത് വന്നിറങ്ങിയാല്‍ ആ താക്കീതു നല്‍കപ്പെട്ടവരുടെ പ്രഭാതം എത്ര ചീത്തയായിരിക്കും.
\end{malayalam}}
\flushright{\begin{Arabic}
\quranayah[37][178]
\end{Arabic}}
\flushleft{\begin{malayalam}
അതിനാല്‍ ഒരവധിവരെ അവരില്‍നിന്ന് മാറിനില്‍ക്കുക.
\end{malayalam}}
\flushright{\begin{Arabic}
\quranayah[37][179]
\end{Arabic}}
\flushleft{\begin{malayalam}
നീ അവരെ ശ്രദ്ധിച്ചുകൊണ്ടിരിക്കുക. അടുത്തുതന്നെ അവരെല്ലാം കണ്ടറിഞ്ഞുകൊള്ളും.
\end{malayalam}}
\flushright{\begin{Arabic}
\quranayah[37][180]
\end{Arabic}}
\flushleft{\begin{malayalam}
പ്രതാപിയായ നിന്റെ നാഥന്‍, അവരാരോപിക്കുന്നതില്‍ നിന്നെല്ലാം എത്രയോ പരിശുദ്ധനാണ്.
\end{malayalam}}
\flushright{\begin{Arabic}
\quranayah[37][181]
\end{Arabic}}
\flushleft{\begin{malayalam}
ദൈവദൂതന്മാര്‍ക്ക് സമാധാനം!
\end{malayalam}}
\flushright{\begin{Arabic}
\quranayah[37][182]
\end{Arabic}}
\flushleft{\begin{malayalam}
പ്രപഞ്ചനാഥനായ അല്ലാഹുവിന് സ്തുതി.
\end{malayalam}}
\chapter{\textmalayalam{സ്വാദ്}}
\begin{Arabic}
\Huge{\centerline{\basmalah}}\end{Arabic}
\flushright{\begin{Arabic}
\quranayah[38][1]
\end{Arabic}}
\flushleft{\begin{malayalam}
സ്വാദ്. ഉദ്ബോധനമുള്‍ക്കൊള്ളുന്ന ഖുര്‍ആന്‍ തന്നെ സത്യം.
\end{malayalam}}
\flushright{\begin{Arabic}
\quranayah[38][2]
\end{Arabic}}
\flushleft{\begin{malayalam}
എന്നാല്‍ സത്യനിഷേധികള്‍ ഔദ്ധത്യത്തിലും കിടമത്സരത്തിലുമാണ്.
\end{malayalam}}
\flushright{\begin{Arabic}
\quranayah[38][3]
\end{Arabic}}
\flushleft{\begin{malayalam}
ഇവര്‍ക്കുമുമ്പ് എത്രയെത്ര തലമുറകളെ നാം നശിപ്പിച്ചിട്ടുണ്ട്. അപ്പോഴെല്ലാമവര്‍ അലമുറയിട്ടു. എന്നാല്‍ അത് രക്ഷപ്പെടാനുള്ള അവസരമായിരുന്നില്ല.
\end{malayalam}}
\flushright{\begin{Arabic}
\quranayah[38][4]
\end{Arabic}}
\flushleft{\begin{malayalam}
തങ്ങളില്‍ നിന്നു തന്നെയുള്ള ഒരു മുന്നറിയിപ്പുകാരന്‍ തങ്ങളിലേക്കു വന്നത് ഇക്കൂട്ടരെ വല്ലാതെ അദ്ഭുതപ്പെടുത്തിയിരിക്കുന്നു. സത്യനിഷേധികള്‍ പറഞ്ഞു: "ഇവന്‍ കള്ളവാദിയായ ഒരു ജാലവിദ്യക്കാരന്‍ തന്നെ.
\end{malayalam}}
\flushright{\begin{Arabic}
\quranayah[38][5]
\end{Arabic}}
\flushleft{\begin{malayalam}
"ഇവന്‍ സകല ദൈവങ്ങളെയും ഒരൊറ്റ ദൈവമാക്കി മാറ്റിയിരിക്കയാണോ? എങ്കിലിത് വല്ലാത്തൊരു വിസ്മയകരമായ കാര്യം തന്നെ!"
\end{malayalam}}
\flushright{\begin{Arabic}
\quranayah[38][6]
\end{Arabic}}
\flushleft{\begin{malayalam}
പ്രമാണിമാര്‍ ഇങ്ങനെ പറഞ്ഞു സ്ഥലംവിട്ടു: "നിങ്ങള്‍ പോകൂ; നിങ്ങള്‍ നിങ്ങളുടെ ദൈവങ്ങളില്‍ തന്നെ ഉറച്ചുനില്‍ക്കൂ. ഇത് ഉദ്ദേശ്യപൂര്‍വം ചെയ്യുന്ന കാര്യം തന്നെ.
\end{malayalam}}
\flushright{\begin{Arabic}
\quranayah[38][7]
\end{Arabic}}
\flushleft{\begin{malayalam}
"അവസാനം വന്നെത്തിയ സമുദായത്തില്‍ ഇതേപ്പറ്റി ഞങ്ങളൊന്നും കേട്ടിട്ടില്ല. ഇതൊരു കൃത്രിമ സൃഷ്ടി മാത്രമാണ്.
\end{malayalam}}
\flushright{\begin{Arabic}
\quranayah[38][8]
\end{Arabic}}
\flushleft{\begin{malayalam}
"നമുക്കിടയില്‍ നിന്ന് ഇവന്നാണോ ഉദ്ബോധനം ഇറക്കിക്കിട്ടിയത്?" എന്നാല്‍ അങ്ങനെയല്ല. ഇവര്‍ എന്റെ ഉദ്ബോധനത്തെ സംബന്ധിച്ച് തികഞ്ഞ സംശയത്തിലാണ്. ഇവര്‍ ഇതേവരെ നമ്മുടെ ശിക്ഷ ആസ്വദിക്കാത്തതിനാലാണിത്.
\end{malayalam}}
\flushright{\begin{Arabic}
\quranayah[38][9]
\end{Arabic}}
\flushleft{\begin{malayalam}
അതല്ല; പ്രതാപിയും അത്യുദാരനുമായ നിന്റെ നാഥന്റെ അനുഗ്രഹത്തിന്റെ ഭണ്ഡാരങ്ങള്‍ ഇവരുടെ വശമാണോ?
\end{malayalam}}
\flushright{\begin{Arabic}
\quranayah[38][10]
\end{Arabic}}
\flushleft{\begin{malayalam}
അതുമല്ലെങ്കില്‍ ആകാശഭൂമികളുടെയും അവയ്ക്കിടയിലുള്ളവയുടെയും ആധിപത്യം ഇവര്‍ക്കാണോ? എങ്കില്‍ ആ മാര്‍ഗങ്ങളിലൂടെ ഇവരൊന്ന് കയറിനോക്കട്ടെ.
\end{malayalam}}
\flushright{\begin{Arabic}
\quranayah[38][11]
\end{Arabic}}
\flushleft{\begin{malayalam}
ഇവിടെയുള്ളത് ഒരു സൈനിക സംഘമാണ്. വിവിധ കക്ഷികളില്‍ നിന്നുള്ളവരാണ്. തോല്‍ക്കാന്‍പോകുന്ന ദുര്‍ബല സംഘം.
\end{malayalam}}
\flushright{\begin{Arabic}
\quranayah[38][12]
\end{Arabic}}
\flushleft{\begin{malayalam}
ഇവര്‍ക്കുമുമ്പ് നൂഹിന്റെ ജനതയും ആദും സത്യത്തെ തള്ളിപ്പറഞ്ഞിട്ടുണ്ട്. ആണിയടിച്ചുറപ്പിച്ചിരുന്ന ഫറവോനും.
\end{malayalam}}
\flushright{\begin{Arabic}
\quranayah[38][13]
\end{Arabic}}
\flushleft{\begin{malayalam}
സമൂദ് സമുദായവും ലൂത്വിന്റെ ജനതയും ഐക്ക നിവാസികളും ഇതുതന്നെ ചെയ്തിട്ടുണ്ട്. അവരാണ് ആ സംഘങ്ങള്‍.
\end{malayalam}}
\flushright{\begin{Arabic}
\quranayah[38][14]
\end{Arabic}}
\flushleft{\begin{malayalam}
ദൈവദൂതന്മാരെ തള്ളിപ്പറയാത്ത ആരും അവരിലില്ല. അതിനാല്‍ എന്റെ ശിക്ഷ അനിവാര്യമായിത്തീര്‍ന്നു.
\end{malayalam}}
\flushright{\begin{Arabic}
\quranayah[38][15]
\end{Arabic}}
\flushleft{\begin{malayalam}
ഒരൊറ്റ ഘോരഗര്‍ജനം മാത്രമാണ് ഇക്കൂട്ടര്‍ കാത്തിരിക്കുന്നത്. അതിനുശേഷം കാലതാമസമുണ്ടാവില്ല.
\end{malayalam}}
\flushright{\begin{Arabic}
\quranayah[38][16]
\end{Arabic}}
\flushleft{\begin{malayalam}
ഇവര്‍ പറയുന്നു: "ഞങ്ങളുടെ നാഥാ, വിചാരണ നാളിനു മുമ്പുതന്നെ ഞങ്ങള്‍ക്കുള്ള ശിക്ഷയുടെ വിഹിതം ഞങ്ങള്‍ക്കു നീ വേഗം നല്‍കേണമേ."
\end{malayalam}}
\flushright{\begin{Arabic}
\quranayah[38][17]
\end{Arabic}}
\flushleft{\begin{malayalam}
ഇവര്‍ പറയുന്നതൊക്കെ ക്ഷമിക്കുക. നമ്മുടെ കരുത്തനായ ദാസന്‍ ദാവൂദിന്റെ കഥ ഇവര്‍ക്കു പറഞ്ഞുകൊടുക്കുക: തീര്‍ച്ചയായും അദ്ദേഹം പശ്ചാത്തപിച്ചു മടങ്ങിയവനാണ്.
\end{malayalam}}
\flushright{\begin{Arabic}
\quranayah[38][18]
\end{Arabic}}
\flushleft{\begin{malayalam}
മലകളെ നാം അദ്ദേഹത്തിന് അധീനപ്പെടുത്തി. അങ്ങനെ വൈകുന്നേരവും രാവിലെയും അവ അദ്ദേഹത്തോടൊപ്പം സങ്കീര്‍ത്തനം ചെയ്യാറുണ്ടായിരുന്നു.
\end{malayalam}}
\flushright{\begin{Arabic}
\quranayah[38][19]
\end{Arabic}}
\flushleft{\begin{malayalam}
ഒരുമിച്ചു പറക്കുന്ന പറവകളെയും നാം അദ്ദേഹത്തിനു വിധേയമാക്കി. എല്ലാം അവന്റെ സങ്കീര്‍ത്തനങ്ങളില്‍ മുഴുകിയിരുന്നു.
\end{malayalam}}
\flushright{\begin{Arabic}
\quranayah[38][20]
\end{Arabic}}
\flushleft{\begin{malayalam}
അദ്ദേഹത്തിന്റെ ആധിപത്യം നാം ഭദ്രമാക്കി. അദ്ദേഹത്തിനു നാം തത്ത്വജ്ഞാനം നല്‍കി. തീര്‍പ്പുകല്‍പിക്കാന്‍ പോന്ന സംസാരശേഷിയും.
\end{malayalam}}
\flushright{\begin{Arabic}
\quranayah[38][21]
\end{Arabic}}
\flushleft{\begin{malayalam}
മതില്‍ കയറി മറിഞ്ഞ് ചാടിവന്ന ആ വഴക്കിടുന്ന കക്ഷികളുടെ വാര്‍ത്ത നിനക്കു വന്നെത്തിയിട്ടുണ്ടോ?
\end{malayalam}}
\flushright{\begin{Arabic}
\quranayah[38][22]
\end{Arabic}}
\flushleft{\begin{malayalam}
അവര്‍ ദാവൂദിന്റെ അടുത്തുകടന്നു ചെന്ന സന്ദര്‍ഭം! അദ്ദേഹം അവരെക്കണ്ട് പരിഭ്രാന്തനായി. അവര്‍ പറഞ്ഞു: "പേടിക്കേണ്ട; തര്‍ക്കത്തിലുള്ള രണ്ടു കക്ഷികളാണ് ഞങ്ങള്‍. ഞങ്ങളിലൊരുകൂട്ടര്‍ മറുകക്ഷിയോട് അതിക്രമം കാണിച്ചിരിക്കുന്നു. അതിനാല്‍ അങ്ങ് ഞങ്ങള്‍ക്കിടയില്‍ ന്യായമായ നിലയില്‍ തീര്‍പ്പുണ്ടാക്കണം. നീതികേട് കാട്ടരുത്. ഞങ്ങളെ നേര്‍വഴിയില്‍ നയിക്കുകയും വേണം.
\end{malayalam}}
\flushright{\begin{Arabic}
\quranayah[38][23]
\end{Arabic}}
\flushleft{\begin{malayalam}
"ഇതാ, ഇവനെന്റെ സഹോദരനാണ്. ഇവന്ന് തൊണ്ണൂറ്റൊമ്പത് പെണ്ണാടുണ്ട്. എനിക്കൊരു പെണ്ണാടും. എന്നിട്ടും ഇവന്‍ പറയുന്നു, അതുംകൂടി തനിക്ക് ഏല്‍പിച്ചുതരണമെന്ന്. വര്‍ത്തമാനത്തില്‍ ഇവനെന്നെ തോല്‍പിക്കുകയാണ്."
\end{malayalam}}
\flushright{\begin{Arabic}
\quranayah[38][24]
\end{Arabic}}
\flushleft{\begin{malayalam}
ദാവൂദ് പറഞ്ഞു: "തന്റെ ആടുകളുടെ കൂട്ടത്തിലേക്ക് നിന്റെ ആടിനെക്കൂടി ആവശ്യപ്പെടുന്നതിലൂടെ അവന്‍ നിന്നോട് അനീതി ചെയ്യുകയാണ്. കൂട്ടാളികളായി കഴിയുന്നവരിലേറെ പേരും പരസ്പരം അതിക്രമം പ്രവര്‍ത്തിക്കുന്നവരാണ്. സത്യവിശ്വാസം സ്വീകരിക്കുകയും സല്‍ക്കര്‍മങ്ങള്‍ പ്രവര്‍ത്തിക്കുകയും ചെയ്തവരൊഴികെ. എന്നാല്‍ അത്തരക്കാരുടെ എണ്ണം വളരെ കുറവാണ്." ദാവൂദിന് മനസ്സിലായി; നാം അദ്ദേഹത്തെ പരീക്ഷിച്ചതായിരുന്നുവെന്ന്. അതിനാല്‍ അദ്ദേഹം തന്റെ നാഥനോട് പാപമോചനം തേടി. കുമ്പിട്ടു വീണു. പശ്ചാത്തപിച്ചു മടങ്ങി.
\end{malayalam}}
\flushright{\begin{Arabic}
\quranayah[38][25]
\end{Arabic}}
\flushleft{\begin{malayalam}
അപ്പോള്‍ നാം അദ്ദേഹത്തിന് പൊറുത്തുകൊടുത്തു. തീര്‍ച്ചയായും അദ്ദേഹത്തിന് നമ്മുടെ സന്നിധിയില്‍ അടുത്ത സ്ഥാനമുണ്ട്. ഉത്തമമായ പര്യവസാനവും.
\end{malayalam}}
\flushright{\begin{Arabic}
\quranayah[38][26]
\end{Arabic}}
\flushleft{\begin{malayalam}
അല്ലാഹു പറഞ്ഞു: "അല്ലയോ ദാവൂദ്, നിശ്ചയമായും നിന്നെ നാം ഭൂമിയില്‍ നമ്മുടെ പ്രതിനിധിയാക്കിയിരിക്കുന്നു. അതിനാല്‍ ജനങ്ങള്‍ക്കിടയില്‍ നീതിപൂര്‍വം ഭരണം നടത്തുക. തന്നിഷ്ടത്തെ പിന്‍പറ്റരുത്. അത് നിന്നെ അല്ലാഹുവിന്റെ മാര്‍ഗത്തില്‍ നിന്ന് തെറ്റിക്കും. അല്ലാഹുവിന്റെ മാര്‍ഗത്തില്‍നിന്ന് തെറ്റിപ്പോകുന്നവര്‍ക്ക് കഠിനമായ ശിക്ഷയുണ്ട്. അവര്‍ വിചാരണ നാളിനെ മറന്നു കളഞ്ഞതിനാലാണിത്."
\end{malayalam}}
\flushright{\begin{Arabic}
\quranayah[38][27]
\end{Arabic}}
\flushleft{\begin{malayalam}
ആകാശഭൂമികളെയും അവയ്ക്കിടയിലുള്ളവയെയും നാം വെറുതെ സൃഷ്ടിച്ചതല്ല. അത് സത്യനിഷേധികളുടെ ധാരണയാണ്. സത്യത്തെ തള്ളിപ്പറഞ്ഞവര്‍ക്കുള്ളതാണ് നരകശിക്ഷയുടെ കൊടുംനാശം.
\end{malayalam}}
\flushright{\begin{Arabic}
\quranayah[38][28]
\end{Arabic}}
\flushleft{\begin{malayalam}
അല്ല, സത്യവിശ്വാസം സ്വീകരിക്കുകയും സല്‍ക്കര്‍മങ്ങള്‍ പ്രവര്‍ത്തിക്കുകയും ചെയ്തവരെ നാം ഭൂമിയില്‍ കുഴപ്പമുണ്ടാക്കുന്നവരെപ്പോലെയാക്കുമോ? അതല്ല; ഭക്തന്മാരെ നാം തെമ്മാടികളെപ്പോലെയാക്കുമോ?
\end{malayalam}}
\flushright{\begin{Arabic}
\quranayah[38][29]
\end{Arabic}}
\flushleft{\begin{malayalam}
നിനക്കു നാം ഇറക്കിത്തന്ന അനുഗൃഹീതമായ വേദപുസ്തകമാണിത്. ഇതിലെ വചനങ്ങളെപ്പറ്റി ഇവര്‍ ചിന്തിച്ചറിയാന്‍. വിചാരശാലികള്‍ പാഠമുള്‍ക്കൊള്ളാനും.
\end{malayalam}}
\flushright{\begin{Arabic}
\quranayah[38][30]
\end{Arabic}}
\flushleft{\begin{malayalam}
ദാവൂദിനു നാം സുലൈമാനെ സമ്മാനിച്ചു. എത്ര നല്ല ദാസന്‍! തീര്‍ച്ചയായും അദ്ദേഹം പശ്ചാത്തപിച്ചു മടങ്ങുന്നവനാണ്.
\end{malayalam}}
\flushright{\begin{Arabic}
\quranayah[38][31]
\end{Arabic}}
\flushleft{\begin{malayalam}
കുതിച്ചുപായാന്‍ തയ്യാറായി നില്‍ക്കുന്ന മേത്തരം കുതിരകള്‍ വൈകുന്നേരം അദ്ദേഹത്തിന്റെ മുമ്പില്‍ കൊണ്ടുവന്ന സന്ദര്‍ഭം.
\end{malayalam}}
\flushright{\begin{Arabic}
\quranayah[38][32]
\end{Arabic}}
\flushleft{\begin{malayalam}
അപ്പോള്‍ അദ്ദേഹം പറഞ്ഞു: "ഞാന്‍ ഈ സമ്പത്തിനെ സ്നേഹിക്കുന്നത് എന്റെ നാഥനെ സ്മരിക്കുന്നതുകൊണ്ടാണ്." അങ്ങനെ ആ കുതിരകള്‍ മുന്നില്‍നിന്ന് പോയി മറഞ്ഞു.
\end{malayalam}}
\flushright{\begin{Arabic}
\quranayah[38][33]
\end{Arabic}}
\flushleft{\begin{malayalam}
അദ്ദേഹം കല്‍പിച്ചു: "നിങ്ങളവയെ എന്റെ അടുത്തേക്ക് തിരിച്ചുകൊണ്ടുവരിക." എന്നിട്ട് അദ്ദേഹം അവയുടെ കാലുകളിലും കഴുത്തുകളിലും തടവാന്‍ തുടങ്ങി.
\end{malayalam}}
\flushright{\begin{Arabic}
\quranayah[38][34]
\end{Arabic}}
\flushleft{\begin{malayalam}
സുലൈമാനെയും നാം പരീക്ഷിച്ചു. അദ്ദേഹത്തിന്റെ സിംഹാസനത്തില്‍ ഒരു ജഡം കൊണ്ടിട്ടു. പിന്നെ അദ്ദേഹം ഖേദിച്ചു മടങ്ങി.
\end{malayalam}}
\flushright{\begin{Arabic}
\quranayah[38][35]
\end{Arabic}}
\flushleft{\begin{malayalam}
അദ്ദേഹം പറഞ്ഞു: "നാഥാ, എനിക്കു നീ പൊറുത്തുതരേണമേ! എനിക്കുശേഷം മറ്റാര്‍ക്കും തരപ്പെടാത്ത രാജാധിപത്യം നീ എനിക്കു നല്‍കേണമേ. നീ തന്നെയാണ് എല്ലാം തരുന്നവന്‍; തീര്‍ച്ച."
\end{malayalam}}
\flushright{\begin{Arabic}
\quranayah[38][36]
\end{Arabic}}
\flushleft{\begin{malayalam}
അപ്പോള്‍ നാം അദ്ദേഹത്തിന് കാറ്റിനെ കീഴ്പ്പെടുത്തിക്കൊടുത്തു. താനിച്ഛിക്കുന്നേടത്തേക്ക് തന്റെ കല്‍പന പ്രകാരം അത് സൌമ്യമായി വീശിയിരുന്നു.
\end{malayalam}}
\flushright{\begin{Arabic}
\quranayah[38][37]
\end{Arabic}}
\flushleft{\begin{malayalam}
ചെകുത്താന്മാരെയും കീഴ്പെടുത്തിക്കൊടുത്തു. അവരിലെ എല്ലാ കെട്ടിട നിര്‍മാതാക്കളെയും മുങ്ങല്‍ വിദഗ്ധരെയും.
\end{malayalam}}
\flushright{\begin{Arabic}
\quranayah[38][38]
\end{Arabic}}
\flushleft{\begin{malayalam}
ചങ്ങലകളിട്ടു പൂട്ടിയ മറ്റു ചിലരെയും അധീനപ്പെടുത്തിക്കൊടുത്തു.
\end{malayalam}}
\flushright{\begin{Arabic}
\quranayah[38][39]
\end{Arabic}}
\flushleft{\begin{malayalam}
നമ്മുടെ സമ്മാനമാണിത്. അതിനാല്‍ നിനക്കവരോട് ഔദാര്യം കാണിക്കാം. അല്ലെങ്കില്‍ അവരെ കൈവശം വെക്കാം. ആരും അതേക്കുറിച്ച് ചോദിക്കുകയില്ല.
\end{malayalam}}
\flushright{\begin{Arabic}
\quranayah[38][40]
\end{Arabic}}
\flushleft{\begin{malayalam}
സംശയമില്ല; അദ്ദേഹത്തിന് നമ്മുടെയടുത്ത് ഉറ്റ സാമീപ്യമുണ്ട്. മെച്ചപ്പെട്ട പര്യവസാനവും.
\end{malayalam}}
\flushright{\begin{Arabic}
\quranayah[38][41]
\end{Arabic}}
\flushleft{\begin{malayalam}
നമ്മുടെ ദാസന്‍ അയ്യൂബിനെ ഓര്‍ക്കുക: അദ്ദേഹം തന്റെ നാഥനെ വിളിച്ചിങ്ങനെ പറഞ്ഞു: "ചെകുത്താന്‍ എന്നെ ദുരിതവും പീഡനവും ഏല്‍പിച്ചല്ലോ."
\end{malayalam}}
\flushright{\begin{Arabic}
\quranayah[38][42]
\end{Arabic}}
\flushleft{\begin{malayalam}
നാം നിര്‍ദേശിച്ചു: "നിന്റെ കാലുകൊണ്ട് നിലത്തു ചവിട്ടുക. ഇതാ തണുത്ത വെള്ളം! കുളിക്കാനും കുടിക്കാനും."
\end{malayalam}}
\flushright{\begin{Arabic}
\quranayah[38][43]
\end{Arabic}}
\flushleft{\begin{malayalam}
അദ്ദേഹത്തിന് തന്റെ ആളുകളെയും അവരോടൊപ്പം അത്രതന്നെ വേറെ ആളുകളെയും നാം സമ്മാനിച്ചു. നമ്മുടെ ഭാഗത്തുനിന്നുള്ള അനുഗ്രഹമായാണത്. വിചാരശാലികള്‍ക്ക് ഉദ്ബോധനമായും.
\end{malayalam}}
\flushright{\begin{Arabic}
\quranayah[38][44]
\end{Arabic}}
\flushleft{\begin{malayalam}
നാം പറഞ്ഞു: "നീ ഒരുപിടി പുല്ല് കയ്യിലെടുക്കുക. എന്നിട്ട് അതുകൊണ്ട് അടിക്കുക. അങ്ങനെ ശപഥം പാലിക്കുക." സംശയമില്ല; നാം അദ്ദേഹത്തെ അങ്ങേയറ്റം ക്ഷമാശീലനായി കണ്ടു. വളരെ നല്ല അടിമ! തീര്‍ച്ചയായും അദ്ദേഹം പശ്ചാത്തപിച്ചു മടങ്ങുന്നവനാകുന്നു.
\end{malayalam}}
\flushright{\begin{Arabic}
\quranayah[38][45]
\end{Arabic}}
\flushleft{\begin{malayalam}
നമ്മുടെ ദാസന്മാരായ ഇബ്റാഹീം, ഇസ്ഹാഖ്, യഅ്്ഖൂബ് എന്നിവരെയും ഓര്‍ക്കുക: കൈക്കരുത്തും ദീര്‍ഘദൃഷ്ടിയുമുള്ളവരായിരുന്നു അവര്‍.
\end{malayalam}}
\flushright{\begin{Arabic}
\quranayah[38][46]
\end{Arabic}}
\flushleft{\begin{malayalam}
പരലോകസ്മരണ എന്ന വിശിഷ്ട ഗുണം കാരണം നാമവരെ പ്രത്യേകം തെരഞ്ഞെടുത്തു.
\end{malayalam}}
\flushright{\begin{Arabic}
\quranayah[38][47]
\end{Arabic}}
\flushleft{\begin{malayalam}
സംശയമില്ല; അവര്‍ നമ്മുടെ അടുത്ത് പ്രത്യേകം തെരഞ്ഞെടുക്കപ്പെട്ട സച്ചരിതരില്‍പെട്ടവരാണ്.
\end{malayalam}}
\flushright{\begin{Arabic}
\quranayah[38][48]
\end{Arabic}}
\flushleft{\begin{malayalam}
ഇസ്മാഈലിനെയും അല്‍യസഇനെയും ദുല്‍കിഫ്ലിനെയും ഓര്‍ക്കുക: ഇവരൊക്കെയും നല്ലവരായിരുന്നു.
\end{malayalam}}
\flushright{\begin{Arabic}
\quranayah[38][49]
\end{Arabic}}
\flushleft{\begin{malayalam}
ഇതൊരുദ്ബോധനമാണ്. തീര്‍ച്ചയായും ഭക്തജനത്തിന് മെച്ചപ്പെട്ട വാസസ്ഥലമുണ്ട്.
\end{malayalam}}
\flushright{\begin{Arabic}
\quranayah[38][50]
\end{Arabic}}
\flushleft{\begin{malayalam}
നിത്യവാസത്തിനുള്ള സ്വര്‍ഗീയാരാമങ്ങളാണത്. അതിന്റെ വാതിലുകള്‍ അവര്‍ക്കായി തുറന്നുവെച്ചവയാണ്.
\end{malayalam}}
\flushright{\begin{Arabic}
\quranayah[38][51]
\end{Arabic}}
\flushleft{\begin{malayalam}
അവരവിടെ ചാരിയിരിക്കും. ധാരാളം പഴങ്ങളും പാനീയങ്ങളും യഥേഷ്ടം ആവശ്യപ്പെട്ടുകൊണ്ടിരിക്കും.
\end{malayalam}}
\flushright{\begin{Arabic}
\quranayah[38][52]
\end{Arabic}}
\flushleft{\begin{malayalam}
അവരുടെ അടുത്ത് നോട്ടം നിയന്ത്രിക്കുന്ന സമപ്രായക്കാരായ തരുണികളുണ്ടായിരിക്കും.
\end{malayalam}}
\flushright{\begin{Arabic}
\quranayah[38][53]
\end{Arabic}}
\flushleft{\begin{malayalam}
ഇതത്രെ വിചാരണനാളില്‍ നിങ്ങള്‍ക്കു നല്‍കാമെന്ന് നാം വാഗ്ദാനം ചെയ്യുന്നത്.
\end{malayalam}}
\flushright{\begin{Arabic}
\quranayah[38][54]
\end{Arabic}}
\flushleft{\begin{malayalam}
സംശയമില്ല; നാം നല്‍കുന്ന ജീവിതവിഭവങ്ങളാണിവ. അതൊരിക്കലും തീര്‍ന്നുപോവുകയില്ല.
\end{malayalam}}
\flushright{\begin{Arabic}
\quranayah[38][55]
\end{Arabic}}
\flushleft{\begin{malayalam}
ഇതൊരവസ്ഥ. എന്നാല്‍ അതിക്രമികള്‍ക്ക് വളരെ ചീത്തയായ വാസസ്ഥലമാണുണ്ടാവുക.
\end{malayalam}}
\flushright{\begin{Arabic}
\quranayah[38][56]
\end{Arabic}}
\flushleft{\begin{malayalam}
നരകത്തീയാണത്. അവരതില്‍ കത്തിയെരിയും. അതെത്ര ചീത്ത സങ്കേതം.
\end{malayalam}}
\flushright{\begin{Arabic}
\quranayah[38][57]
\end{Arabic}}
\flushleft{\begin{malayalam}
ഇതാണവര്‍ക്കുള്ളത്. അതിനാലവരിത് അനുഭവിച്ചുകൊള്ളട്ടെ. ചുട്ടുപൊള്ളുന്ന വെള്ളവും ചീഞ്ഞളിഞ്ഞ ചലവും.
\end{malayalam}}
\flushright{\begin{Arabic}
\quranayah[38][58]
\end{Arabic}}
\flushleft{\begin{malayalam}
ഇതുപോലുള്ള മറ്റു പലതരം ശിക്ഷകളും അവിടെയുണ്ട്.
\end{malayalam}}
\flushright{\begin{Arabic}
\quranayah[38][59]
\end{Arabic}}
\flushleft{\begin{malayalam}
അവരോട് അല്ലാഹു പറയും: "ഇത് നിങ്ങളോടൊപ്പം നരകത്തില്‍ തിങ്ങിക്കൂടാനുള്ള ആള്‍ക്കൂട്ടമാണ്." അപ്പോഴവര്‍ പറയും: "ഇവര്‍ക്ക് സ്വാഗതോപചാരമൊന്നുമില്ല. തീര്‍ച്ചയായും ഇവര്‍ നരകത്തില്‍ കത്തിയെരിയേണ്ടവര്‍ തന്നെ."
\end{malayalam}}
\flushright{\begin{Arabic}
\quranayah[38][60]
\end{Arabic}}
\flushleft{\begin{malayalam}
ആ കടന്നുവരുന്നവര്‍ പറയും: "അല്ല; നിങ്ങള്‍ക്കു തന്നെയാണ് സ്വാഗതോപചാരമില്ലാത്തത്. നിങ്ങളാണ് ഞങ്ങള്‍ക്ക് ഈ ദുരവസ്ഥ വരുത്തിവെച്ചത്. വളരെ ചീത്ത സങ്കേതം തന്നെയാണിത്."
\end{malayalam}}
\flushright{\begin{Arabic}
\quranayah[38][61]
\end{Arabic}}
\flushleft{\begin{malayalam}
അവര്‍ പറയും: "ഞങ്ങളുടെ നാഥാ, ഞങ്ങള്‍ക്ക് ഈ ശിക്ഷ വരുത്തിവെച്ചവര്‍ക്ക് നീ നരകത്തീയില്‍ ഇരട്ടി ശിക്ഷ നല്‍കേണമേ."
\end{malayalam}}
\flushright{\begin{Arabic}
\quranayah[38][62]
\end{Arabic}}
\flushleft{\begin{malayalam}
അവര്‍ പറയും: "നമുക്കെന്തു പറ്റി? ചീത്ത മനുഷ്യരെന്ന് നാം കരുതിയിരുന്ന പലരെയും ഇവിടെ കാണുന്നില്ലല്ലോ.
\end{malayalam}}
\flushright{\begin{Arabic}
\quranayah[38][63]
\end{Arabic}}
\flushleft{\begin{malayalam}
"നാം അവരെ പരിഹാസപാത്രമാക്കിയിരുന്നുവല്ലോ. അതല്ല അവര്‍ നമ്മുടെ കണ്ണില്‍പെടാത്തതാണോ?"
\end{malayalam}}
\flushright{\begin{Arabic}
\quranayah[38][64]
\end{Arabic}}
\flushleft{\begin{malayalam}
നരകവാസികള്‍ തമ്മിലുള്ള തര്‍ക്കം തീര്‍ച്ചയായും സംഭവിക്കാന്‍ പോവുന്നതു തന്നെയാണ്.
\end{malayalam}}
\flushright{\begin{Arabic}
\quranayah[38][65]
\end{Arabic}}
\flushleft{\begin{malayalam}
നബിയേ പറയുക: "ഞാനൊരു മുന്നറിയിപ്പുകാരന്‍ മാത്രമാണ്. അല്ലാഹുവല്ലാതെ ദൈവമില്ല. അവന്‍ ഏകനാണ്. സര്‍വാധിപതിയും.
\end{malayalam}}
\flushright{\begin{Arabic}
\quranayah[38][66]
\end{Arabic}}
\flushleft{\begin{malayalam}
"ആകാശഭൂമികളുടെയും അവയ്ക്കിടയിലുള്ളവയുടെയും സംരക്ഷകനാണ്. പ്രതാപിയാണ്. ഏറെ പൊറുക്കുന്നവനും."
\end{malayalam}}
\flushright{\begin{Arabic}
\quranayah[38][67]
\end{Arabic}}
\flushleft{\begin{malayalam}
പറയുക: "ഇതൊരു മഹത്തായ സന്ദേശം തന്നെ.
\end{malayalam}}
\flushright{\begin{Arabic}
\quranayah[38][68]
\end{Arabic}}
\flushleft{\begin{malayalam}
"എന്നാല്‍ നിങ്ങളതിനെ അവഗണിക്കുന്നവരാണ്.
\end{malayalam}}
\flushright{\begin{Arabic}
\quranayah[38][69]
\end{Arabic}}
\flushleft{\begin{malayalam}
"അത്യുന്നതങ്ങളില്‍ വിശിഷ്ട സമൂഹം സംവാദം നടത്തിയ സന്ദര്‍ഭത്തെ സംബന്ധിച്ച് എനിക്കൊന്നും അറിയുമായിരുന്നില്ല.
\end{malayalam}}
\flushright{\begin{Arabic}
\quranayah[38][70]
\end{Arabic}}
\flushleft{\begin{malayalam}
"അതേക്കുറിച്ച് എനിക്കു ബോധനം ലഭിച്ചത് ഞാന്‍ വ്യക്തമായൊരു മുന്നറിയിപ്പുകാരന്‍ എന്ന നിലക്കു മാത്രമാണ്."
\end{malayalam}}
\flushright{\begin{Arabic}
\quranayah[38][71]
\end{Arabic}}
\flushleft{\begin{malayalam}
നിന്റെ നാഥന്‍ മലക്കുകളോടു പറഞ്ഞു: "ഉറപ്പായും ഞാന്‍ കളിമണ്ണില്‍നിന്ന് മനുഷ്യനെ സൃഷ്ടിക്കാന്‍ പോവുകയാണ്.
\end{malayalam}}
\flushright{\begin{Arabic}
\quranayah[38][72]
\end{Arabic}}
\flushleft{\begin{malayalam}
"അങ്ങനെ ഞാനവനെ ശരിപ്പെടുത്തുകയും എന്റെ ആത്മാവില്‍ നിന്ന് അതിലൂതുകയും ചെയ്താല്‍ നിങ്ങളവന്റെ മുന്നില്‍ സാഷ്ടാംഗം പ്രണമിക്കണം."
\end{malayalam}}
\flushright{\begin{Arabic}
\quranayah[38][73]
\end{Arabic}}
\flushleft{\begin{malayalam}
അപ്പോള്‍ മലക്കുകളൊക്കെയും സാഷ്ടാംഗം പ്രണമിച്ചു.
\end{malayalam}}
\flushright{\begin{Arabic}
\quranayah[38][74]
\end{Arabic}}
\flushleft{\begin{malayalam}
ഇബ്ലീസൊഴികെ. അവന്‍ അഹങ്കരിച്ചു. അങ്ങനെ അവന്‍ സത്യനിഷേധിയായി.
\end{malayalam}}
\flushright{\begin{Arabic}
\quranayah[38][75]
\end{Arabic}}
\flushleft{\begin{malayalam}
അല്ലാഹു ചോദിച്ചു: "ഇബ്ലീസേ, ഞാനെന്റെ കൈകൊണ്ട് പടച്ചുണ്ടാക്കിയവന്ന് പ്രണമിക്കുന്നതില്‍നിന്ന് നിന്നെ തടഞ്ഞതെന്താണ്? നീ അഹങ്കരിച്ചോ? അതല്ല; നീ പൊങ്ങച്ചക്കാരില്‍പെട്ടുപോയോ?"
\end{malayalam}}
\flushright{\begin{Arabic}
\quranayah[38][76]
\end{Arabic}}
\flushleft{\begin{malayalam}
ഇബ്ലീസ് പറഞ്ഞു: "മനുഷ്യനെക്കാള്‍ ശ്രേഷ്ഠന്‍ ഞാനാണ്. നീയെന്നെ പടച്ചത് തീയില്‍ നിന്നാണ്. അവനെ സൃഷ്ടിച്ചതോ കളിമണ്ണില്‍ നിന്നും."
\end{malayalam}}
\flushright{\begin{Arabic}
\quranayah[38][77]
\end{Arabic}}
\flushleft{\begin{malayalam}
അല്ലാഹു കല്‍പിച്ചു: "എങ്കില്‍ ഇവിടെ നിന്നിറങ്ങിപ്പോകണം. സംശയമില്ല; ഇനിമുതല്‍ ആട്ടിയോടിക്കപ്പെട്ടവനാണ് നീ.
\end{malayalam}}
\flushright{\begin{Arabic}
\quranayah[38][78]
\end{Arabic}}
\flushleft{\begin{malayalam}
"വിധിദിനം വരെ നിന്റെമേല്‍ എന്റെ ശാപമുണ്ട്; തീര്‍ച്ച."
\end{malayalam}}
\flushright{\begin{Arabic}
\quranayah[38][79]
\end{Arabic}}
\flushleft{\begin{malayalam}
ഇബ്ലീസ് പറഞ്ഞു: "എന്റെ നാഥാ, എങ്കില്‍ അവര്‍ വീണ്ടും ഉയിര്‍ത്തെഴുന്നേല്‍പിക്കപ്പെടുന്നനാള്‍ വരെ നീ എനിക്കു അവസരം തരേണമേ."
\end{malayalam}}
\flushright{\begin{Arabic}
\quranayah[38][80]
\end{Arabic}}
\flushleft{\begin{malayalam}
അല്ലാഹു അറിയിച്ചു: "നീ അവസരം നല്‍കപ്പെട്ടവരുടെ കൂട്ടത്തിലാണ്.
\end{malayalam}}
\flushright{\begin{Arabic}
\quranayah[38][81]
\end{Arabic}}
\flushleft{\begin{malayalam}
"നിശ്ചിതമായ ആ സമയം വന്നെത്തുന്ന ദിവസം വരെ."
\end{malayalam}}
\flushright{\begin{Arabic}
\quranayah[38][82]
\end{Arabic}}
\flushleft{\begin{malayalam}
ഇബ്ലീസ് പറഞ്ഞു: "നിന്റെ പ്രതാപമാണ് സത്യം. തീര്‍ച്ചയായും ഇവരെയൊക്കെ ഞാന്‍ വഴിപിഴപ്പിക്കും.
\end{malayalam}}
\flushright{\begin{Arabic}
\quranayah[38][83]
\end{Arabic}}
\flushleft{\begin{malayalam}
"ഇവരിലെ നിന്റെ ആത്മാര്‍ഥതയുള്ള അടിമകളെയൊഴികെ."
\end{malayalam}}
\flushright{\begin{Arabic}
\quranayah[38][84]
\end{Arabic}}
\flushleft{\begin{malayalam}
അല്ലാഹു പറഞ്ഞു: "എങ്കില്‍ സത്യം ഇതാണ്. സത്യം മാത്രമേ ഞാന്‍ പറയുകയുള്ളൂ.
\end{malayalam}}
\flushright{\begin{Arabic}
\quranayah[38][85]
\end{Arabic}}
\flushleft{\begin{malayalam}
"നിന്നെയും നിന്നെ പിന്‍പറ്റിയ മറ്റെല്ലാവരെയുംകൊണ്ട് നാം നരകം നിറക്കുക തന്നെ ചെയ്യും."
\end{malayalam}}
\flushright{\begin{Arabic}
\quranayah[38][86]
\end{Arabic}}
\flushleft{\begin{malayalam}
പറയുക: "ഇതിന്റെ പേരില്‍ ഞാന്‍ നിങ്ങളോടൊരു പ്രതിഫലവും ആവശ്യപ്പെടുന്നില്ല. പിന്നെ ഞാന്‍ കള്ളം കെട്ടിച്ചമച്ചുണ്ടാക്കുന്നവനുമല്ല."
\end{malayalam}}
\flushright{\begin{Arabic}
\quranayah[38][87]
\end{Arabic}}
\flushleft{\begin{malayalam}
ഇത് ലോകര്‍ക്കാകമാനമുള്ള ഉദ്ബോധനമാണ്.
\end{malayalam}}
\flushright{\begin{Arabic}
\quranayah[38][88]
\end{Arabic}}
\flushleft{\begin{malayalam}
നിശ്ചിത കാലത്തിനുശേഷം ഈ വൃത്താന്തത്തിന്റെ നിജസ്ഥിതി നിങ്ങളറിയുക തന്നെ ചെയ്യും.
\end{malayalam}}
\chapter{\textmalayalam{സുമര്‍ ( കൂട്ടങ്ങള്‍ )}}
\begin{Arabic}
\Huge{\centerline{\basmalah}}\end{Arabic}
\flushright{\begin{Arabic}
\quranayah[39][1]
\end{Arabic}}
\flushleft{\begin{malayalam}
ഈ വേദപുസ്തകത്തിന്റെ അവതരണം പ്രതാപിയും യുക്തിമാനുമായ അല്ലാഹുവില്‍നിന്നാണ്.
\end{malayalam}}
\flushright{\begin{Arabic}
\quranayah[39][2]
\end{Arabic}}
\flushleft{\begin{malayalam}
തീര്‍ച്ചയായും നിനക്കു നാം ഈ വേദപുസ്തകം ഇറക്കിത്തന്നത് സത്യസന്ദേശവുമായാണ്. അതിനാല്‍ കീഴ്വണക്കം അല്ലാഹുവിന് മാത്രമാക്കി അവന് വഴിപ്പെടുക.
\end{malayalam}}
\flushright{\begin{Arabic}
\quranayah[39][3]
\end{Arabic}}
\flushleft{\begin{malayalam}
അറിയുക: കളങ്കമറ്റ കീഴ്വണക്കം അല്ലാഹുവിനു മാത്രം അവകാശപ്പെട്ടതാണ്. അവനെക്കൂടാതെ മറ്റുള്ളവരെ രക്ഷാധികാരികളായി സ്വീകരിക്കുന്നവര്‍ അവകാശപ്പെടുന്നു: "ഞങ്ങളെ അല്ലാഹുവുമായി കൂടുതല്‍ അടുപ്പിക്കാന്‍ വേണ്ടി മാത്രമാണ് ഞങ്ങള്‍ അവരെ വണങ്ങുന്നത്." എന്നാല്‍ ഭിന്നാഭിപ്രായമുള്ള കാര്യത്തില്‍ അല്ലാഹു അവര്‍ക്കിടയില്‍ തീര്‍പ്പ് കല്‍പിക്കുന്നതാണ്. നിശ്ചയമായും നുണയനെയും നന്ദികെട്ടവനെയും അല്ലാഹു നേര്‍വഴിയിലാക്കുകയില്ല.
\end{malayalam}}
\flushright{\begin{Arabic}
\quranayah[39][4]
\end{Arabic}}
\flushleft{\begin{malayalam}
പുത്രനെ വരിക്കണമെന്ന് അല്ലാഹു ഇച്ഛിച്ചിരുന്നെങ്കില്‍ അവന്‍ തന്റെ സൃഷ്ടികളില്‍നിന്ന് താനിഷ്ടപ്പെടുന്നവരെ തെരഞ്ഞെടുക്കുമായിരുന്നു. എന്നാല്‍ അവനെത്ര പരിശുദ്ധന്‍. അവനാണ് അല്ലാഹു. ഏകന്‍; സകലാധിനാഥന്‍!
\end{malayalam}}
\flushright{\begin{Arabic}
\quranayah[39][5]
\end{Arabic}}
\flushleft{\begin{malayalam}
ആകാശഭൂമികളെ അവന്‍ യാഥാര്‍ഥ്യത്തോടെയാണ് സൃഷ്ടിച്ചത്. അവന്‍ പകലിനെ രാവുകൊണ്ട് ചുറ്റിപ്പൊതിയുന്നു. രാവിനെ പകലുകൊണ്ടും ചുറ്റിപ്പൊതിയുന്നു. സൂര്യചന്ദ്രന്മാരെ അവന്‍ തന്റെ വരുതിയിലൊതുക്കിയിരിക്കുന്നു. അവയെല്ലാം നിശ്ചിത കാലപരിധിക്കകത്തു സഞ്ചരിക്കുന്നു. അറിയുക: അവന്‍ പ്രതാപിയാണ്. ഏറെ പൊറുക്കുന്നവനും.
\end{malayalam}}
\flushright{\begin{Arabic}
\quranayah[39][6]
\end{Arabic}}
\flushleft{\begin{malayalam}
ഒരൊറ്റ സത്തയില്‍നിന്ന് അവന്‍ നിങ്ങളെയെല്ലാം സൃഷ്ടിച്ചു. പിന്നെ അതില്‍നിന്ന് അതിന്റെ ഇണയെ ഉണ്ടാക്കി. നിങ്ങള്‍ക്കായി കന്നുകാലികളില്‍ നിന്ന് എട്ട് ജോടികളെയും അവനൊരുക്കിത്തന്നു. നിങ്ങളുടെ മാതാക്കളുടെ ഉദരത്തില്‍ അവന്‍ നിങ്ങളെ സൃഷ്ടിക്കുന്നു. മൂന്ന് ഇരുളുകള്‍ക്കുള്ളില്‍ ഒന്നിനു പിറകെ ഒന്നായി; ഘട്ടംഘട്ടമായി നിങ്ങളെ അവന്‍ രൂപപ്പെടുത്തിയെടുക്കുന്നു. ഇതൊക്കെയും ചെയ്യുന്ന അല്ലാഹുവാണ് നിങ്ങളുടെ നാഥന്‍. ആധിപത്യം അവനു മാത്രമാണ്. അവനല്ലാതെ ദൈവമില്ല. എന്നിട്ടും നിങ്ങളെങ്ങോട്ടാണ് വഴിതെറ്റിപ്പോകുന്നത്.
\end{malayalam}}
\flushright{\begin{Arabic}
\quranayah[39][7]
\end{Arabic}}
\flushleft{\begin{malayalam}
നിങ്ങള്‍ നന്ദികേട് കാട്ടുകയാണെങ്കില്‍, സംശയമില്ല; അല്ലാഹു നിങ്ങളുടെയൊന്നും ആശ്രയമാവശ്യമില്ലാത്തവനാണ്. എന്നാല്‍ തന്റെ ദാസന്മാരുടെ നന്ദികേട് അവനൊട്ടും ഇഷ്ടപ്പെടുന്നില്ല. നിങ്ങള്‍ നന്ദി കാണിക്കുന്നുവെങ്കില്‍ അതുകാരണം നിങ്ങളോടവന്‍ സംതൃപ്തനായിത്തീരും. സ്വന്തം പാപഭാരമല്ലാതെ ആരും അപരന്റെ ഭാരം ചുമക്കുകയില്ല. പിന്നീട് നിങ്ങളുടെ നാഥന്റെ അടുത്തേക്കാണ് നിങ്ങളുടെയൊക്കെ മടക്കം. നിങ്ങള്‍ പ്രവര്‍ത്തിച്ചുകൊണ്ടിരിക്കുന്നതിനെപ്പറ്റി അപ്പോഴവന്‍ നിങ്ങളെ വിവരമറിയിക്കും. നെഞ്ചകങ്ങളിലുള്ളതൊക്കെയും നന്നായറിയുന്നവനാണവന്‍.
\end{malayalam}}
\flushright{\begin{Arabic}
\quranayah[39][8]
\end{Arabic}}
\flushleft{\begin{malayalam}
മനുഷ്യന് വല്ല വിപത്തും ബാധിച്ചാല്‍ അവന്‍ തന്റെ നാഥങ്കലേക്ക് താഴ്മയോടെ മടങ്ങി അവനോട് പ്രാര്‍ഥിക്കുന്നു. പിന്നീട് അല്ലാഹു തന്നില്‍നിന്നുള്ള അനുഗ്രഹം പ്രദാനം ചെയ്താല്‍ നേരത്തെ അല്ലാഹുവോട് പ്രാര്‍ഥിച്ചിരുന്ന കാര്യംതന്നെ അവന്‍ മറന്നുകളയുന്നു. ദൈവമാര്‍ഗത്തില്‍നിന്ന് വഴിതെറ്റിക്കാനായി അവന്‍ അല്ലാഹുവിന് സമന്മാരെ സങ്കല്‍പിക്കുകയും ചെയ്യുന്നു. പറയുക: "അല്‍പകാലം നീ നിന്റെ സത്യനിഷേധവുമായി സുഖിച്ചുകൊള്ളുക. സംശയമില്ല; നീ നരകാവകാശികളില്‍ പെട്ടവന്‍ തന്നെ."
\end{malayalam}}
\flushright{\begin{Arabic}
\quranayah[39][9]
\end{Arabic}}
\flushleft{\begin{malayalam}
അവനെപ്പോലെയാണോ സാഷ്ടാംഗം പ്രണമിച്ചും നിന്ന് പ്രാര്‍ഥിച്ചും രാത്രി കീഴ്വണക്കത്തോടെ കഴിച്ചുകൂട്ടുന്നവന്‍. പരലോകത്തെ പേടിക്കുന്നവനാണിവന്‍. തന്റെ നാഥന്റെ കാരുണ്യം കൊതിക്കുന്നവനും. അറിവുള്ളവരും ഇല്ലാത്തവരും ഒരുപോലെയാണോ? വിചാരശീലര്‍ മാത്രമേ ആലോചിച്ചറിയുന്നുള്ളൂ.
\end{malayalam}}
\flushright{\begin{Arabic}
\quranayah[39][10]
\end{Arabic}}
\flushleft{\begin{malayalam}
പറയുക: "എന്റെ വിശ്വാസികളായ ദാസന്മാരേ, നിങ്ങള്‍ നിങ്ങളുടെ നാഥനോട് ഭക്തി പുലര്‍ത്തുക. ഈ ലോകത്ത് നന്മ ചെയ്തവര്‍ക്ക് മേന്മയുണ്ട്. അല്ലാഹുവിന്റെ ഭൂമി വളരെ വിശാലമാണ്. ക്ഷമ പാലിക്കുന്നവര്‍ക്കാണ് അവരുടെ പ്രതിഫലം കണക്കില്ലാതെ കിട്ടുക.
\end{malayalam}}
\flushright{\begin{Arabic}
\quranayah[39][11]
\end{Arabic}}
\flushleft{\begin{malayalam}
പറയുക: "കീഴ്വണക്കം അല്ലാഹുവിനു മാത്രമാക്കി അവനു വഴിപ്പെടണമെന്ന് എന്നോടവന്‍ കല്‍പിച്ചിരിക്കുന്നു.
\end{malayalam}}
\flushright{\begin{Arabic}
\quranayah[39][12]
\end{Arabic}}
\flushleft{\begin{malayalam}
"മുസ്ലിംകളില്‍ ഒന്നാമനാകണമെന്നും എന്നോട് അവനാജ്ഞാപിച്ചിരിക്കുന്നു."
\end{malayalam}}
\flushright{\begin{Arabic}
\quranayah[39][13]
\end{Arabic}}
\flushleft{\begin{malayalam}
പറയുക: "ഞാനെന്റെ നാഥനെ ധിക്കരിക്കുകയാണെങ്കില്‍ ഭയങ്കരമായൊരു നാളിന്റെ ശിക്ഷയെ ഞാന്‍ ഭയപ്പെടുന്നു.
\end{malayalam}}
\flushright{\begin{Arabic}
\quranayah[39][14]
\end{Arabic}}
\flushleft{\begin{malayalam}
പറയുക: "ഞാനെന്റെ കീഴ്വണക്കം അല്ലാഹുവിനു മാത്രമാക്കി. അവനെ മാത്രം വഴിപ്പെടുന്നു.
\end{malayalam}}
\flushright{\begin{Arabic}
\quranayah[39][15]
\end{Arabic}}
\flushleft{\begin{malayalam}
"എന്നാല്‍ നിങ്ങള്‍ അവനെക്കൂടാതെ തോന്നിയവയെയൊക്കെ പൂജിച്ചുകൊള്ളുക." പറയുക: "ഉയിര്‍ത്തെഴുന്നേല്‍പുനാളില്‍ സ്വന്തത്തിനും സ്വന്തക്കാര്‍ക്കും നഷ്ടം വരുത്തിവെച്ചവര്‍ തന്നെയാണ് തീര്‍ച്ചയായും തുലഞ്ഞവര്‍; അറിയുക: അതുതന്നെയാണ് പ്രകടമായ നഷ്ടം!"
\end{malayalam}}
\flushright{\begin{Arabic}
\quranayah[39][16]
\end{Arabic}}
\flushleft{\begin{malayalam}
അവര്‍ക്കു മീതെ നരകത്തീയിന്റെ ജ്വാലയാണ് തണലായുണ്ടാവുക. താഴെയുമുണ്ട് തീത്തട്ടുകള്‍. അതിനെപ്പറ്റിയാണ് അല്ലാഹു തന്റെ ദാസന്മാരെ ഭയപ്പെടുത്തുന്നത്. അതിനാല്‍ എന്റെ ദാസന്മാരേ, എന്നോട് ഭക്തിയുള്ളവരാവുക.
\end{malayalam}}
\flushright{\begin{Arabic}
\quranayah[39][17]
\end{Arabic}}
\flushleft{\begin{malayalam}
പൈശാചിക ശക്തികള്‍ക്ക് വഴിപ്പെടുന്നത് വര്‍ജിക്കുകയും അല്ലാഹുവിങ്കലേക്ക് താഴ്മയോടെ തിരിച്ചുചെല്ലുകയും ചെയ്യുന്നവര്‍ക്കുള്ളതാണ് ശുഭവാര്‍ത്ത. അതിനാല്‍ എന്റെ ദാസന്മാരെ ശുഭവാര്‍ത്ത അറിയിക്കുക.
\end{malayalam}}
\flushright{\begin{Arabic}
\quranayah[39][18]
\end{Arabic}}
\flushleft{\begin{malayalam}
വചനങ്ങള്‍ ശ്രദ്ധയോടെ കേള്‍ക്കുകയും എന്നിട്ടവയിലേറ്റവും നല്ലത് പിന്‍പറ്റുകയും ചെയ്യുന്നവരാണവര്‍. അവരെത്തന്നെയാണ് അല്ലാഹു നേര്‍വഴിയിലാക്കിയത്. ബുദ്ധിശാലികളും അവര്‍ തന്നെ.
\end{malayalam}}
\flushright{\begin{Arabic}
\quranayah[39][19]
\end{Arabic}}
\flushleft{\begin{malayalam}
അപ്പോള്‍ ശിക്ഷാവിധി സ്ഥിരപ്പെട്ടുകഴിഞ്ഞവന്റെ സ്ഥിതിയോ; നരകത്തീയിലുള്ളവനെ രക്ഷിക്കാന്‍ നിനക്കാവുമോ?
\end{malayalam}}
\flushright{\begin{Arabic}
\quranayah[39][20]
\end{Arabic}}
\flushleft{\begin{malayalam}
എന്നാല്‍ തങ്ങളുടെ നാഥനോട് ഭക്തിപുലര്‍ത്തിയവര്‍ക്ക് തട്ടിനുമേല്‍ തട്ടുകളായി നിര്‍മിച്ച മണിമേടകളുണ്ട്. അവയുടെ താഴ്ഭാഗത്തൂടെ ആറുകളൊഴുകിക്കൊണ്ടിരിക്കും. അല്ലാഹുവിന്റെ വാഗ്ദാനമാണിത്. അല്ലാഹു വാഗ്ദാനം ലംഘിക്കുകയില്ല.
\end{malayalam}}
\flushright{\begin{Arabic}
\quranayah[39][21]
\end{Arabic}}
\flushleft{\begin{malayalam}
നീ കാണുന്നില്ലേ; അല്ലാഹു മാനത്തുനിന്ന് വെള്ളം വീഴ്ത്തുന്നത്. അങ്ങനെ അതിനെ ഭൂമിയില്‍ ഉറവകളായി ഒഴുക്കുന്നതും. പിന്നീട് അതുവഴി അല്ലാഹു വര്‍ണ വൈവിധ്യമുള്ള വിളകളുല്‍പാദിപ്പിക്കുന്നു. അതിനുശേഷം അവ ഉണങ്ങുന്നു. അപ്പോഴവ മഞ്ഞച്ചതായി നിനക്കു കാണാം. പിന്നെ അവനവയെ കച്ചിത്തുരുമ്പാക്കുന്നു. വിചാരമതികള്‍ക്കിതില്‍ ഗുണപാഠമുണ്ട്.
\end{malayalam}}
\flushright{\begin{Arabic}
\quranayah[39][22]
\end{Arabic}}
\flushleft{\begin{malayalam}
അല്ലാഹു ഒരാള്‍ക്ക് ഇസ്ലാം സ്വീകരിക്കാന്‍ ഹൃദയവിശാലത നല്‍കി. അങ്ങനെ അവന്‍ തന്റെ നാഥനില്‍ നിന്നുള്ള വെളിച്ചത്തിലൂടെ ചരിക്കാന്‍ തുടങ്ങി. അയാളും അങ്ങനെയല്ലാത്തവനും ഒരുപോലെയാകുമോ? അതിനാല്‍, ദൈവസ്മരണയില്‍ നിന്നകന്ന് ഹൃദയം കടുത്തുപോയവര്‍ക്കാണ് കൊടിയ നാശം! അവര്‍ വ്യക്തമായ വഴികേടിലാണ്.
\end{malayalam}}
\flushright{\begin{Arabic}
\quranayah[39][23]
\end{Arabic}}
\flushleft{\begin{malayalam}
ഏറ്റവും വിശിഷ്ടമായ വര്‍ത്തമാനമാണ് അല്ലാഹു ഇറക്കിത്തന്നത്. വചനങ്ങളില്‍ പരസ്പര ചേര്‍ച്ചയും ആവര്‍ത്തനവുമുള്ള ഗ്രന്ഥമാണിത്. അതു കേള്‍ക്കുമ്പോള്‍ തങ്ങളുടെ നാഥനെ ഭയപ്പെടുന്നവരുടെ ചര്‍മങ്ങള്‍ രോമാഞ്ചമണിയുന്നു. പിന്നീട് അവരുടെ ചര്‍മങ്ങളും ഹൃദയങ്ങളും അല്ലാഹുവെ ഓര്‍ക്കാന്‍ പാകത്തില്‍ വിനീതമാകുന്നു. ഇതാണ് അല്ലാഹുവിന്റെ മാര്‍ഗദര്‍ശനം. അതുവഴി അവനിച്ഛിക്കുന്നവരെ അവന്‍ നേര്‍വഴിയിലാക്കുന്നു. അല്ലാഹു വഴികേടിലാക്കുന്നവരെ നേര്‍വഴിയിലാക്കാന്‍ ആര്‍ക്കുമാവില്ല.
\end{malayalam}}
\flushright{\begin{Arabic}
\quranayah[39][24]
\end{Arabic}}
\flushleft{\begin{malayalam}
എന്നാല്‍ ഉയിര്‍ത്തെഴുന്നേല്‍പുനാളില്‍ തനിക്കു നേരെ വരുന്ന കഠിനശിക്ഷയെ തന്റെ മുഖം കൊണ്ടു തടുക്കേണ്ടിവരുന്നവന്റെ സ്ഥിതിയോ? അതിക്രമികളോട് അന്ന് പറയും: "നിങ്ങള്‍ സമ്പാദിച്ചുകൊണ്ടിരുന്നതത്രയും നിങ്ങള്‍തന്നെ ആസ്വദിച്ചുകൊള്ളുക."
\end{malayalam}}
\flushright{\begin{Arabic}
\quranayah[39][25]
\end{Arabic}}
\flushleft{\begin{malayalam}
ഇവര്‍ക്കു മുമ്പുള്ളവരും സത്യത്തെ തള്ളിപ്പറഞ്ഞു. അവസാനം അവരോര്‍ക്കാത്ത ഭാഗത്തുനിന്ന് വിപത്തുകള്‍ അവരില്‍ വന്നെത്തി.
\end{malayalam}}
\flushright{\begin{Arabic}
\quranayah[39][26]
\end{Arabic}}
\flushleft{\begin{malayalam}
അങ്ങനെ അല്ലാഹു അവരെ ഐഹികജീവിതത്തില്‍ തന്നെ അപമാനം ആസ്വദിപ്പിച്ചു. പരലോകശിക്ഷയോ അതിലും എത്രയോ കൂടുതല്‍ കഠിനമത്രേ. ഇക്കൂട്ടരിതറിഞ്ഞിരുന്നെങ്കില്‍!
\end{malayalam}}
\flushright{\begin{Arabic}
\quranayah[39][27]
\end{Arabic}}
\flushleft{\begin{malayalam}
നാം ഈ ഖുര്‍ആനിലൂടെ മനുഷ്യര്‍ക്കായി വിവിധയിനം ഉദാഹരണങ്ങള്‍ വ്യക്തമായി വിശദീകരിച്ചിട്ടുണ്ട്. അവര്‍ ആലോചിച്ചറിയാന്‍.
\end{malayalam}}
\flushright{\begin{Arabic}
\quranayah[39][28]
\end{Arabic}}
\flushleft{\begin{malayalam}
അറബി ഭാഷയിലുള്ള ഖുര്‍ആനാണിത്. ഇതിലൊട്ടും വളച്ചുകെട്ടില്ല. അവര്‍ ഭക്തിയുള്ളവരാകാന്‍ വേണ്ടിയാണിത്.
\end{malayalam}}
\flushright{\begin{Arabic}
\quranayah[39][29]
\end{Arabic}}
\flushleft{\begin{malayalam}
അല്ലാഹു ഇതാ ഒരുദാഹരണം സമര്‍പ്പിക്കുന്നു: ഒരു മനുഷ്യന്‍. അനേകമാളുകള്‍ അവന്റെ ഉടമസ്ഥതയില്‍ പങ്കാളികളാണ്. അവര്‍ പരസ്പരം കലഹിക്കുന്നവരുമാണ്. മറ്റൊരു മനുഷ്യന്‍; ഒരു യജമാനനു മാത്രം കീഴ്പെട്ട് കഴിയുന്നവനാണയാള്‍. ഈ രണ്ടുപേരും ഒരുപോലെയാകുമോ? അല്ലാഹുവിന് സ്തുതി. എന്നാല്‍ അവരിലേറെ പേരും കാര്യം മനസ്സിലാക്കുന്നില്ല.
\end{malayalam}}
\flushright{\begin{Arabic}
\quranayah[39][30]
\end{Arabic}}
\flushleft{\begin{malayalam}
സംശയമില്ല; ഒരുനാള്‍ നീ മരിക്കും. അവരും മരിക്കും.
\end{malayalam}}
\flushright{\begin{Arabic}
\quranayah[39][31]
\end{Arabic}}
\flushleft{\begin{malayalam}
പിന്നെ, ഉയിര്‍ത്തെഴുന്നേല്‍പു നാളില്‍ നിങ്ങളുടെ നാഥന്റെ സന്നിധിയില്‍ വെച്ച് നിങ്ങള്‍ കലഹിക്കും.
\end{malayalam}}
\flushright{\begin{Arabic}
\quranayah[39][32]
\end{Arabic}}
\flushleft{\begin{malayalam}
അപ്പോള്‍ അല്ലാഹുവിന്റെ പേരില്‍ കള്ളം പറയുകയും തനിക്കു സത്യം വന്നെത്തിയപ്പോള്‍ അതിനെ തള്ളിപ്പറയുകയും ചെയ്തവനെക്കാള്‍ കടുത്ത അക്രമി ആരുണ്ട്? നരകത്തീയല്ലയോ സത്യനിഷേധികള്‍ക്കുള്ള വാസസ്ഥലം.
\end{malayalam}}
\flushright{\begin{Arabic}
\quranayah[39][33]
\end{Arabic}}
\flushleft{\begin{malayalam}
സത്യസന്ദേശവുമായി വന്നവനും അതിനെ സത്യപ്പെടുത്തിയവനും തന്നെയാണ് ഭക്തി പുലര്‍ത്തുന്നവര്‍.
\end{malayalam}}
\flushright{\begin{Arabic}
\quranayah[39][34]
\end{Arabic}}
\flushleft{\begin{malayalam}
അവര്‍ക്ക് തങ്ങളുടെ നാഥന്റെ അടുത്ത് അവരാഗ്രഹിക്കുന്നതൊക്കെ കിട്ടും. അതാണ് സച്ചരിതര്‍ക്കുള്ള പ്രതിഫലം.
\end{malayalam}}
\flushright{\begin{Arabic}
\quranayah[39][35]
\end{Arabic}}
\flushleft{\begin{malayalam}
അവര്‍ ചെയ്തുപോയതില്‍ ഏറ്റവും ചീത്ത പ്രവൃത്തിപോലും അല്ലാഹു അവരില്‍നിന്ന് മായ്ച്ചുകളയാനാണിത്. അവര്‍ ചെയ്തുകൊണ്ടിരുന്ന ഏറ്റം നല്ല പ്രവര്‍ത്തനങ്ങളുടെ അടിസ്ഥാനത്തിലവര്‍ക്കു പ്രതിഫലം നല്‍കാനും.
\end{malayalam}}
\flushright{\begin{Arabic}
\quranayah[39][36]
\end{Arabic}}
\flushleft{\begin{malayalam}
അല്ലാഹു പോരേ അവന്റെ അടിമയ്ക്ക്? അവന് പുറമെയുള്ളവരുടെ പേരില്‍ അവര്‍ നിന്നെ പേടിപ്പിക്കുന്നു. അല്ലാഹു ആരെയെങ്കിലും വഴികേടിലാക്കുകയാണെങ്കില്‍ അവനെ നേര്‍വഴിയിലാക്കാന്‍ മറ്റാര്‍ക്കുമാവില്ല.
\end{malayalam}}
\flushright{\begin{Arabic}
\quranayah[39][37]
\end{Arabic}}
\flushleft{\begin{malayalam}
വല്ലവനെയും അല്ലാഹു നേര്‍വഴിയിലാക്കുകയാണെങ്കില്‍ അവനെ വഴികേടിലാക്കാനും ആര്‍ക്കും സാധ്യമല്ല. അല്ലാഹു പ്രതാപിയും ശിക്ഷാനടപടി സ്വീകരിക്കുന്നവനും അല്ലെന്നോ?
\end{malayalam}}
\flushright{\begin{Arabic}
\quranayah[39][38]
\end{Arabic}}
\flushleft{\begin{malayalam}
ആകാശഭൂമികളെ സൃഷ്ടിച്ചത് ആരെന്ന് നീ അവരോട് ചോദിച്ചാല്‍ തീര്‍ച്ചയായും അവര്‍ പറയും, “അല്ലാഹു”വെന്ന്. എങ്കില്‍ ചോദിക്കുക: "അല്ലാഹുവെക്കൂടാതെ നിങ്ങള്‍ വിളിച്ചു പ്രാര്‍ഥിക്കുന്നവയെപ്പറ്റി നിങ്ങള്‍ ചിന്തിച്ചുനോക്കിയിട്ടുണ്ടോ? എനിക്കു വല്ല വിപത്തും വരുത്താന്‍ അല്ലാഹു ഉദ്ദേശിച്ചുവെങ്കില്‍ അവയ്ക്ക് ആ വിപത്ത് തട്ടിമാറ്റാനാകുമോ?" അല്ലെങ്കില്‍ അവനെനിക്ക് എന്തെങ്കിലും അനുഗ്രഹമേകാനുദ്ദേശിച്ചാല്‍ അവക്ക് അവന്റെ അനുഗ്രഹം തടഞ്ഞുവെക്കാന്‍ കഴിയുമോ?" പറയുക: എനിക്ക് അല്ലാഹു മതി. ഭരമേല്‍പിക്കുന്നവരൊക്കെയും അവനില്‍ ഭരമേല്‍പിക്കട്ടെ.
\end{malayalam}}
\flushright{\begin{Arabic}
\quranayah[39][39]
\end{Arabic}}
\flushleft{\begin{malayalam}
പറയുക: "എന്റെ ജനമേ, നിങ്ങള്‍ നിങ്ങള്‍ക്കാവുംപോലെ പ്രവര്‍ത്തിച്ചുകൊള്ളുക. ഞാനും പ്രവര്‍ത്തിച്ചുകൊണ്ടിരിക്കാം. അടുത്തുതന്നെ നിങ്ങള്‍ക്കു മനസ്സിലായിക്കൊള്ളും;
\end{malayalam}}
\flushright{\begin{Arabic}
\quranayah[39][40]
\end{Arabic}}
\flushleft{\begin{malayalam}
"ആര്‍ക്കാണ് അപമാനകരമായ ശിക്ഷ വന്നെത്തുകയെന്ന്. സ്ഥിരമായ ശിക്ഷ വന്നിറങ്ങുക ആരുടെ മേലാണെന്നും."
\end{malayalam}}
\flushright{\begin{Arabic}
\quranayah[39][41]
\end{Arabic}}
\flushleft{\begin{malayalam}
സംശയമില്ല; മനുഷ്യര്‍ക്കാകമാനമുള്ള സത്യസന്ദേശവുമായി നാം നിനക്ക് ഈ വേദപുസ്തകം ഇറക്കിത്തന്നിരിക്കുന്നു. അതിനാല്‍ ആരെങ്കിലും നേര്‍വഴി സ്വീകരിച്ചാല്‍ അതിന്റെ നന്മ അവനു തന്നെയാണ്. വല്ലവനും വഴികേടിലായാല്‍ അതിന്റെ ദോഷവും അവനുതന്നെ. നീ അവരുടെ കൈകാര്യകര്‍ത്താവൊന്നുമല്ല.
\end{malayalam}}
\flushright{\begin{Arabic}
\quranayah[39][42]
\end{Arabic}}
\flushleft{\begin{malayalam}
മരണവേളയില്‍ ആത്മാക്കളെ പിടിച്ചെടുക്കുന്നത് അല്ലാഹുവാണ്. ഇനിയും മരിച്ചിട്ടില്ലാത്തവരുടെ ആത്മാവിനെ അവരുടെ ഉറക്കത്തില്‍ പിടിച്ചുവെക്കുന്നതും അവന്‍ തന്നെ. അങ്ങനെ താന്‍ മരണംവിധിച്ച ആത്മാക്കളെ അവന്‍ പിടിച്ചുവെക്കുന്നു. മറ്റുള്ളവയെ ഒരു നിശ്ചിത കാലാവധി വരെ അവന്‍ തിരിച്ചയക്കുന്നു. ചിന്തിക്കുന്ന ജനത്തിന് തീര്‍ച്ചയായും ഇതില്‍ ധാരാളം ദൃഷ്ടാന്തങ്ങളുണ്ട്.
\end{malayalam}}
\flushright{\begin{Arabic}
\quranayah[39][43]
\end{Arabic}}
\flushleft{\begin{malayalam}
അതല്ല; അല്ലാഹുവെക്കൂടാതെ അവര്‍ ശിപാര്‍ശകരെ ഉണ്ടാക്കിവെച്ചിരിക്കുകയാണോ? ചോദിക്കുക: ഒന്നിന്റെയും ഉടമാവകാശമില്ലാത്തവരും ഒന്നും ആലോചിക്കാത്തവരുമാണെങ്കിലും അവര്‍ ശിപാര്‍ശ ചെയ്യുമെന്നോ?
\end{malayalam}}
\flushright{\begin{Arabic}
\quranayah[39][44]
\end{Arabic}}
\flushleft{\begin{malayalam}
പറയുക: "ശിപാര്‍ശക്കുള്ള അവകാശമൊക്കെയും അല്ലാഹുവിന് മാത്രമുള്ളതാണ്. അവന്നാണ് ആകാശഭൂമികളുടെ ആധിപത്യം. പിന്നീട് നിങ്ങള്‍ മടങ്ങിച്ചെല്ലുന്നതും അവങ്കലേക്കുതന്നെ."
\end{malayalam}}
\flushright{\begin{Arabic}
\quranayah[39][45]
\end{Arabic}}
\flushleft{\begin{malayalam}
ഏകനായ അല്ലാഹുവെപ്പറ്റി പറയുമ്പോള്‍ പരലോകവിശ്വാസമില്ലാത്തവരുടെ മനസ്സുകള്‍ക്ക് സഹികേടു തോന്നുന്നു. അവനു പുറമെയുള്ളവരെപ്പററി പറഞ്ഞാലോ അവര്‍ അത്യധികം സന്തോഷിക്കുകയും ചെയ്യുന്നു.
\end{malayalam}}
\flushright{\begin{Arabic}
\quranayah[39][46]
\end{Arabic}}
\flushleft{\begin{malayalam}
പറയുക: "അല്ലാഹുവേ, ആകാശഭൂമികളുടെ സ്രഷ്ടാവേ, ദൃശ്യവും അദൃശ്യവും അറിയുന്നവനേ, നിന്റെ അടിമകള്‍ക്കിടയില്‍ അഭിപ്രായ വ്യത്യാസമുള്ള വിഷയങ്ങളില്‍ അവസാനം വിധി തീര്‍പ്പുണ്ടാക്കുന്നത് നീയാണല്ലോ."
\end{malayalam}}
\flushright{\begin{Arabic}
\quranayah[39][47]
\end{Arabic}}
\flushleft{\begin{malayalam}
ഭൂമിയിലുള്ളതൊക്കെയും അതോടൊപ്പം അത്രയും, അതിക്രമം കാണിച്ചവരുടെ വശമുണ്ടെങ്കില്‍ ഉയിര്‍ത്തെഴുന്നേല്‍പുനാളിലെ കടുത്ത ശിക്ഷയില്‍നിന്നു രക്ഷനേടാന്‍ അതൊക്കെയും അവര്‍ പിഴയായി നല്‍കാന്‍ തയ്യാറാകും. നേരത്തെ ഒരിക്കലും അവര്‍ ഊഹിക്കുകപോലും ചെയ്തിട്ടില്ലാത്ത പലതും അവിടെ അവര്‍ക്ക് അല്ലാഹുവിങ്കല്‍നിന്ന് വെളിപ്പെടുന്നു.
\end{malayalam}}
\flushright{\begin{Arabic}
\quranayah[39][48]
\end{Arabic}}
\flushleft{\begin{malayalam}
അവര്‍ ശേഖരിച്ചുവെച്ചതിന്റെ ദുഷ്ഫലങ്ങളവര്‍ക്ക് വെളിപ്പെടും. അന്നോളം അവര്‍ പുച്ഛിച്ചു തള്ളിയിരുന്ന അതേശിക്ഷ തന്നെ അവരെ ബാധിക്കുകയും ചെയ്യും.
\end{malayalam}}
\flushright{\begin{Arabic}
\quranayah[39][49]
\end{Arabic}}
\flushleft{\begin{malayalam}
വല്ല വിപത്തും ബാധിച്ചാല്‍ മനുഷ്യന്‍ നമ്മെ വിളിച്ചുപ്രാര്‍ഥിക്കും. പിന്നീട് നാം വല്ല അനുഗ്രഹവും നല്‍കിയാലോ അവന്‍ പറയും: "ഇതെനിക്ക് എന്റെ അറിവിന്റെ അടിസ്ഥാനത്തില്‍ കിട്ടിയതാണ്." എന്നാല്‍ യഥാര്‍ഥത്തിലതൊരു പരീക്ഷണമാണ്. പക്ഷേ, അവരിലേറെ പേരും അതറിയുന്നില്ല.
\end{malayalam}}
\flushright{\begin{Arabic}
\quranayah[39][50]
\end{Arabic}}
\flushleft{\begin{malayalam}
ഇവര്‍ക്കു മുമ്പുള്ളവരും ഇവ്വിധം പറഞ്ഞിട്ടുണ്ടായിരുന്നു. എന്നാല്‍ അവര്‍ സമ്പാദിച്ചതൊന്നും അവര്‍ക്കൊട്ടും ഉപകരിച്ചില്ല.
\end{malayalam}}
\flushright{\begin{Arabic}
\quranayah[39][51]
\end{Arabic}}
\flushleft{\begin{malayalam}
അങ്ങനെ അവര്‍ സമ്പാദിച്ചതിന്റെ ദുരന്തഫലങ്ങള്‍ അവരെ ബാധിച്ചു. അതേപോലെ ഇക്കൂട്ടരില്‍ അതിക്രമികള്‍ക്കും അവര്‍ നേടിയതിന്റെ ദുഷ്ഫലങ്ങള്‍ ബാധിക്കാന്‍ പോവുകയാണ്. ഇവര്‍ക്കും നമ്മെ തോല്‍പിക്കാനാവില്ല.
\end{malayalam}}
\flushright{\begin{Arabic}
\quranayah[39][52]
\end{Arabic}}
\flushleft{\begin{malayalam}
ഇവര്‍ മനസ്സിലാക്കുന്നില്ലേ; അല്ലാഹു അവനിച്ഛിക്കുന്നവര്‍ക്ക് വിഭവങ്ങള്‍ വിപുലമാക്കിക്കൊടുക്കുന്നു. അവനിച്ഛിക്കുന്നവര്‍ക്ക് അതില്‍ കുറവു വരുത്തുന്നു. സത്യവിശ്വാസികളായ ജനത്തിന് തീര്‍ച്ചയായും ഇതില്‍ ധാരാളം ദൃഷ്ടാന്തങ്ങളുണ്ട്.
\end{malayalam}}
\flushright{\begin{Arabic}
\quranayah[39][53]
\end{Arabic}}
\flushleft{\begin{malayalam}
പറയുക: തങ്ങളോടുതന്നെ അതിക്രമം കാണിച്ച എന്റെ ദാസന്മാരേ, അല്ലാഹുവിന്റെ കാരുണ്യത്തെപ്പറ്റി നിങ്ങള്‍ നിരാശരാവരുത്. സംശയംവേണ്ട. അല്ലാഹു എല്ലാ പാപങ്ങളും പൊറുത്തുതരുന്നവനാണ്. ഉറപ്പായും അവന്‍ ഏറെ പൊറുക്കുന്നവനാണ്. പരമദയാലുവും.
\end{malayalam}}
\flushright{\begin{Arabic}
\quranayah[39][54]
\end{Arabic}}
\flushleft{\begin{malayalam}
നിങ്ങള്‍ക്കു ശിക്ഷ വന്നെത്തും മുമ്പെ നിങ്ങള്‍ നിങ്ങളുടെ നാഥങ്കലേക്ക് പശ്ചാത്തപിച്ചു മടങ്ങുക. അവന് കീഴ്പെടുക. ശിക്ഷ വന്നെത്തിയാല്‍ പിന്നെ നിങ്ങള്‍ക്ക് എങ്ങുനിന്നും സഹായം കിട്ടുകയില്ല.
\end{malayalam}}
\flushright{\begin{Arabic}
\quranayah[39][55]
\end{Arabic}}
\flushleft{\begin{malayalam}
നിങ്ങളറിയാതെ, പെട്ടെന്നാണ് നിങ്ങള്‍ക്കു ശിക്ഷ വന്നെത്തുക. അതിനു മുമ്പേ നിങ്ങളുടെ നാഥനില്‍നിന്ന് നിങ്ങള്‍ക്ക് ഇറക്കിക്കിട്ടിയ വേദത്തിലെ വചനങ്ങളെ പിന്‍പറ്റുക.
\end{malayalam}}
\flushright{\begin{Arabic}
\quranayah[39][56]
\end{Arabic}}
\flushleft{\begin{malayalam}
ആരും ഇങ്ങനെ പറയാന്‍ ഇടവരാതിരിക്കട്ടെ: "എന്റെ നാശം, അല്ലാഹുവോടുള്ള ബാധ്യതാ നിര്‍വഹണത്തില്‍ ഞാന്‍ വല്ലാതെ വീഴ്ചവരുത്തിയല്ലോ. തീര്‍ച്ചയായും ഞാന്‍ അതിനെ പരിഹസിക്കുന്നവരുടെ കൂട്ടത്തിലായിപ്പോയി.
\end{malayalam}}
\flushright{\begin{Arabic}
\quranayah[39][57]
\end{Arabic}}
\flushleft{\begin{malayalam}
അല്ലെങ്കില്‍ ഇങ്ങനെയും പറയേണ്ടിവരാതിരിക്കട്ടെ: "അല്ലാഹു എന്നെ നേര്‍വഴിയിലാക്കിയിരുന്നുവെങ്കില്‍ ഞാന്‍ ഭക്തന്മാരിലുള്‍പ്പെടുമായിരുന്നേനെ."
\end{malayalam}}
\flushright{\begin{Arabic}
\quranayah[39][58]
\end{Arabic}}
\flushleft{\begin{malayalam}
അതുമല്ലെങ്കില്‍ ശിക്ഷ നേരില്‍ കാണുമ്പോള്‍ ഇവ്വിധം പറയാനിടവരരുത്: "എനിക്കൊന്ന് മടങ്ങിപ്പോകാന്‍ കഴിഞ്ഞിരുന്നെങ്കില്‍ ഉറപ്പായും ഞാന്‍ സച്ചരിതരില്‍പെടുമായിരുന്നു."
\end{malayalam}}
\flushright{\begin{Arabic}
\quranayah[39][59]
\end{Arabic}}
\flushleft{\begin{malayalam}
എന്നാല്‍ സംശയമില്ല; എന്റെ വചനങ്ങള്‍ നിനക്ക് വന്നെത്തിയിരുന്നു. അപ്പോള്‍ നീ അവയെ തള്ളിപ്പറഞ്ഞു. അഹങ്കരിക്കുകയും ചെയ്തു. അങ്ങനെ നീ സത്യനിഷേധികളിലുള്‍പ്പെട്ടു.
\end{malayalam}}
\flushright{\begin{Arabic}
\quranayah[39][60]
\end{Arabic}}
\flushleft{\begin{malayalam}
അല്ലാഹുവിന്റെ പേരില്‍ കള്ളം പറഞ്ഞവരുടെ മുഖങ്ങള്‍ ഉയിര്‍ത്തെഴുന്നേല്‍പുനാളില്‍ കറുത്തിരുണ്ടവയായി നിനക്കു കാണാം. നരകത്തീയല്ലയോ അഹങ്കാരികളുടെ വാസസ്ഥലം.
\end{malayalam}}
\flushright{\begin{Arabic}
\quranayah[39][61]
\end{Arabic}}
\flushleft{\begin{malayalam}
ഭക്തിപുലര്‍ത്തിയവരെ അവരവലംബിച്ച വിജയകരമായ ജീവിതം കാരണം അല്ലാഹു രക്ഷപ്പെടുത്തും. ശിക്ഷ അവരെ ബാധിക്കുകയില്ല. അവര്‍ ദുഃഖിക്കേണ്ടിവരില്ല.
\end{malayalam}}
\flushright{\begin{Arabic}
\quranayah[39][62]
\end{Arabic}}
\flushleft{\begin{malayalam}
അല്ലാഹു സകല വസ്തുക്കളുടെയും സ്രഷ്ടാവാണ്. എല്ലാ കാര്യങ്ങള്‍ക്കും മേല്‍നോട്ടം വഹിക്കുന്നവനും.
\end{malayalam}}
\flushright{\begin{Arabic}
\quranayah[39][63]
\end{Arabic}}
\flushleft{\begin{malayalam}
ആകാശഭൂമികളുടെ താക്കോലുകള്‍ അവന്റെ വശമാണുള്ളത്. അല്ലാഹുവിന്റെ വചനങ്ങളെ തള്ളിപ്പറയുന്നവര്‍ തന്നെയാണ് തുലഞ്ഞവര്‍.
\end{malayalam}}
\flushright{\begin{Arabic}
\quranayah[39][64]
\end{Arabic}}
\flushleft{\begin{malayalam}
ചോദിക്കുക: "വിവേകംകെട്ടവരേ, ഞാന്‍ അല്ലാഹു അല്ലാത്തവരെ പൂജിക്കണമെന്നാണോ നിങ്ങളെന്നോടാവശ്യപ്പെടുന്നത്?"
\end{malayalam}}
\flushright{\begin{Arabic}
\quranayah[39][65]
\end{Arabic}}
\flushleft{\begin{malayalam}
സംശയമില്ല; നിനക്കും നിനക്കു മുമ്പുള്ളവര്‍ക്കും ബോധനമായി നല്‍കിയതിതാണ്: “നീ അല്ലാഹുവില്‍ പങ്കുചേര്‍ത്താല്‍ ഉറപ്പായും നിന്റെ പ്രവര്‍ത്തനങ്ങളൊക്കെ പാഴാകും. നീ എല്ലാം നഷ്ടപ്പെട്ടവരില്‍പെടുകയും ചെയ്യും.”
\end{malayalam}}
\flushright{\begin{Arabic}
\quranayah[39][66]
\end{Arabic}}
\flushleft{\begin{malayalam}
അതിനാല്‍ നീ അല്ലാഹുവിനു മാത്രം വഴിപ്പെടുക. നന്ദി കാണിക്കുന്നവരിലുള്‍പ്പെടുക.
\end{malayalam}}
\flushright{\begin{Arabic}
\quranayah[39][67]
\end{Arabic}}
\flushleft{\begin{malayalam}
അല്ലാഹുവെ പരിഗണിക്കേണ്ട വിധം ഇക്കൂട്ടര്‍ പരിഗണിച്ചിട്ടില്ല. ഉയിര്‍ത്തെഴുന്നേല്‍പുനാളില്‍ ഭൂമി മുഴുവന്‍ അവന്റെ കൈപ്പിടിയിലൊതുങ്ങും. ആകാശങ്ങള്‍ അവന്റെ വലംകയ്യില്‍ ചുരുട്ടിക്കൂട്ടിയതായിത്തീരും. അവനെത്ര പരിശുദ്ധന്‍! ഇവരാരോപിക്കുന്ന പങ്കാളികള്‍ക്കെല്ലാം അതീതനും അത്യുന്നതനുമാണവന്‍.
\end{malayalam}}
\flushright{\begin{Arabic}
\quranayah[39][68]
\end{Arabic}}
\flushleft{\begin{malayalam}
അന്ന് കാഹളത്തില്‍ ഊതപ്പെടും. അപ്പോള്‍ ആകാശഭൂമികളിലുള്ളവരൊക്കെ ചലനമറ്റവരായിത്തീരും. അല്ലാഹു ഉദ്ദേശിച്ചവരൊഴികെ. പിന്നീട് വീണ്ടുമൊരിക്കല്‍ കാഹളത്തിലൂതപ്പെടും. അപ്പോഴതാ എല്ലാവരും എഴുന്നേറ്റ് നോക്കാന്‍ തുടങ്ങുന്നു.
\end{malayalam}}
\flushright{\begin{Arabic}
\quranayah[39][69]
\end{Arabic}}
\flushleft{\begin{malayalam}
അന്ന് ഭൂമി അതിന്റെ നാഥന്റെ പ്രഭയാല്‍ പ്രകാശിതമാകും. കര്‍മപുസ്തകം സമര്‍പ്പിക്കപ്പെടും. പ്രവാചകന്മാരും സാക്ഷികളും ഹാജരാക്കപ്പെടും. അങ്ങനെ ജനങ്ങള്‍ക്കിടയില്‍ നീതിപൂര്‍വം വിധിത്തീര്‍പ്പുണ്ടാകും. ആരും അനീതിക്കിരയാവില്ല.
\end{malayalam}}
\flushright{\begin{Arabic}
\quranayah[39][70]
\end{Arabic}}
\flushleft{\begin{malayalam}
ഓരോ വ്യക്തിക്കും താന്‍ പ്രവര്‍ത്തിച്ചതിന് അര്‍ഹമായ പ്രതിഫലം പൂര്‍ണമായും ലഭിക്കും. അവര്‍ ചെയ്യുന്നതൊക്കെയും നന്നായറിയുന്നവനാണ് അല്ലാഹു.
\end{malayalam}}
\flushright{\begin{Arabic}
\quranayah[39][71]
\end{Arabic}}
\flushleft{\begin{malayalam}
സത്യനിഷേധികള്‍ കൂട്ടംകൂട്ടമായി നരകത്തീയിലേക്ക് നയിക്കപ്പെടും. അങ്ങനെ അവര്‍ അതിനടുത്തെത്തിയാല്‍ അതിന്റെ കവാടങ്ങള്‍ തുറക്കപ്പെടും. അതിന്റെ കാവല്‍ക്കാര്‍ അവരോടിങ്ങനെ ചോദിക്കും: "നിങ്ങളുടെ നാഥന്റെ വചനങ്ങള്‍ ഓതിക്കേള്‍പ്പിച്ചു തരികയും ഈ ദിനത്തെ കണ്ടുമുട്ടേണ്ടിവരുമെന്ന് മുന്നറിയിപ്പു നല്‍കുകയും ചെയ്ത, നിങ്ങളില്‍നിന്നുതന്നെയുള്ള ദൈവദൂതന്മാര്‍ നിങ്ങളിലേക്ക് വന്നെത്തിയിരുന്നില്ലേ?" അവര്‍ പറയും: “അതെ. പക്ഷേ, സത്യനിഷേധികള്‍ക്ക് ശിക്ഷാവിധി സ്ഥിരപ്പെട്ടുപോയി.”
\end{malayalam}}
\flushright{\begin{Arabic}
\quranayah[39][72]
\end{Arabic}}
\flushleft{\begin{malayalam}
അവരോടു പറയും: "നിങ്ങള്‍ നരക വാതിലുകളിലൂടെ കടന്നുകൊള്ളുക. നിങ്ങളിവിടെ സ്ഥിരവാസികളായിരിക്കും. അഹങ്കാരികളുടെ താവളം എത്ര ചീത്ത!"
\end{malayalam}}
\flushright{\begin{Arabic}
\quranayah[39][73]
\end{Arabic}}
\flushleft{\begin{malayalam}
തങ്ങളുടെ നാഥനോട് ഭക്തി പുലര്‍ത്തിയവര്‍ സ്വര്‍ഗത്തിലേക്ക് കൂട്ടംകൂട്ടമായി നയിക്കപ്പെടും. അങ്ങനെ അവരവിടെ എത്തുമ്പോള്‍ അതിന്റെ വാതിലുകള്‍ അവര്‍ക്കായി തുറന്നുവെച്ചവയായിരിക്കും. അതിന്റെ കാവല്‍ക്കാര്‍ അരോടു പറയും: "നിങ്ങള്‍ക്കു സമാധാനം. നിങ്ങള്‍ക്കു നല്ലതു വരട്ടെ. സ്ഥിരവാസികളായി നിങ്ങളിതില്‍ പ്രവേശിച്ചുകൊള്ളുക."
\end{malayalam}}
\flushright{\begin{Arabic}
\quranayah[39][74]
\end{Arabic}}
\flushleft{\begin{malayalam}
അവര്‍ പറയും: ഞങ്ങളോടുള്ള വാഗ്ദാനം പൂര്‍ത്തീകരിച്ചു തരികയും ഞങ്ങളെ ഭൂമിയുടെ അവകാശികളാക്കുകയും ചെയ്ത അല്ലാഹുവിന് സ്തുതി. ഈ സ്വര്‍ഗത്തില്‍ നാമുദ്ദേശിക്കുന്നേടത്ത് നമുക്കു താമസിക്കാമല്ലോ. അപ്പോള്‍ കര്‍മം ചെയ്യുന്നവരുടെ പ്രതിഫലം എത്ര മഹത്തരം!
\end{malayalam}}
\flushright{\begin{Arabic}
\quranayah[39][75]
\end{Arabic}}
\flushleft{\begin{malayalam}
മലക്കുകള്‍ തങ്ങളുടെ നാഥനെ വാഴ്ത്തിയും കീര്‍ത്തനം ചെയ്തും സിംഹാസനത്തിനു ചുറ്റും അണിനിരന്നതായി നിനക്കു കാണാം. അപ്പോള്‍ ജനത്തിനിടയില്‍ നീതിപൂര്‍വമായ വിധിത്തീര്‍പ്പുണ്ടാകും. “പ്രപഞ്ച നാഥനായ അല്ലാഹുവിന് സ്തുതി”യെന്ന് പറയപ്പെടുകയും ചെയ്യും.
\end{malayalam}}
\chapter{\textmalayalam{മുഅ്മിന്‍‍ ( വിശ്വാസി )}}
\begin{Arabic}
\Huge{\centerline{\basmalah}}\end{Arabic}
\flushright{\begin{Arabic}
\quranayah[40][1]
\end{Arabic}}
\flushleft{\begin{malayalam}
ഹാ - മീം.
\end{malayalam}}
\flushright{\begin{Arabic}
\quranayah[40][2]
\end{Arabic}}
\flushleft{\begin{malayalam}
ഈ വേദപുസ്തകത്തിന്റെ അവതരണം പ്രതാപിയും എല്ലാം അറിയുന്നവനുമായ അല്ലാഹുവില്‍ നിന്നാണ്.
\end{malayalam}}
\flushright{\begin{Arabic}
\quranayah[40][3]
\end{Arabic}}
\flushleft{\begin{malayalam}
അവന്‍ പാപം പൊറുക്കുന്നവനാണ്. പശ്ചാത്താപം സ്വീകരിക്കുന്നവനും കഠിനമായി ശിക്ഷിക്കുന്നവനുമാണ്. അതിരുകളില്ലാത്ത കഴിവുകളുള്ളവനും. അവനല്ലാതെ ദൈവമില്ല. അവങ്കലേക്കാണ് എല്ലാറ്റിന്റെയും മടക്കം.
\end{malayalam}}
\flushright{\begin{Arabic}
\quranayah[40][4]
\end{Arabic}}
\flushleft{\begin{malayalam}
സത്യത്തെ തള്ളിപ്പറഞ്ഞവരല്ലാതെ അല്ലാഹുവിന്റെ വചനങ്ങളെപ്പറ്റി തര്‍ക്കിക്കുകയില്ല. അതിനാല്‍ നാട്ടിലെങ്ങുമുള്ള അവരുടെ സ്വൈരവിഹാരം നിന്നെ വഞ്ചിതനാക്കാതിരിക്കട്ടെ.
\end{malayalam}}
\flushright{\begin{Arabic}
\quranayah[40][5]
\end{Arabic}}
\flushleft{\begin{malayalam}
ഇവര്‍ക്കു മുമ്പ് നൂഹിന്റെ ജനതയും സത്യത്തെ തള്ളിപ്പറഞ്ഞിട്ടുണ്ട്. അവര്‍ക്കു പിറകെ വന്ന പല ജനപദങ്ങളും അതുതന്നെ ചെയ്തു. ഓരോ ജനപദവും തങ്ങളുടെ ദൈവദൂതനെ പിടികൂടാന്‍ ഒരുമ്പെട്ടു. അസത്യമുപയോഗിച്ച് സത്യത്തെ തകര്‍ക്കാന്‍ അവര്‍ തര്‍ക്കിച്ചുകൊണ്ടിരുന്നു. അതിനാല്‍ ഞാനവരെ പിടികൂടി. അപ്പോള്‍ എന്റെ ശിക്ഷ എത്രമാത്രം കഠിനമായിരുന്നു!
\end{malayalam}}
\flushright{\begin{Arabic}
\quranayah[40][6]
\end{Arabic}}
\flushleft{\begin{malayalam}
അങ്ങനെ സത്യനിഷേധികള്‍ നരകാവകാശികളാണെന്ന നിന്റെ നാഥന്റെ വചനം സ്ഥാപിതമായി.
\end{malayalam}}
\flushright{\begin{Arabic}
\quranayah[40][7]
\end{Arabic}}
\flushleft{\begin{malayalam}
സിംഹാസനം വഹിക്കുന്നവരും അതിനു ചുറ്റുമുള്ളവരും തങ്ങളുടെ നാഥനെ കീര്‍ത്തിക്കുന്നതോടൊപ്പം അവന്റെ വിശുദ്ധി വാഴ്ത്തുന്നു. അവനില്‍ അടിയുറച്ചു വിശ്വസിക്കുന്നു. വിശ്വാസികളുടെ പാപമോചനത്തിനായി ഇങ്ങനെ പ്രാര്‍ഥിക്കുകയും ചെയ്യുന്നു: "ഞങ്ങളുടെ നാഥാ, നിന്റെ അനുഗ്രഹവും അറിവും സകല വസ്തുക്കളെയും വലയം ചെയ്തു നില്‍ക്കുന്നവയാണല്ലോ. അതിനാല്‍ പശ്ചാത്തപിക്കുകയും നിന്റെ പാത പിന്തുടരുകയും ചെയ്തവര്‍ക്ക് നീ പൊറുത്തുകൊടുക്കേണമേ. അവരെ നരകശിക്ഷയില്‍നിന്ന് രക്ഷിക്കേണമേ.
\end{malayalam}}
\flushright{\begin{Arabic}
\quranayah[40][8]
\end{Arabic}}
\flushleft{\begin{malayalam}
"ഞങ്ങളുടെ നാഥാ, അവര്‍ക്കു നീ വാഗ്ദാനം ചെയ്ത നിത്യവാസത്തിനുള്ള സ്വര്‍ഗത്തില്‍ അവരെ പ്രവേശിപ്പിക്കേണമേ. അവരുടെ മാതാപിതാക്കള്‍, ഇണകള്‍, മക്കള്‍ എന്നിവരിലെ സച്ചരിതരെയും. നിശ്ചയം നീയാണ് പ്രതാപിയും യുക്തിമാനും.
\end{malayalam}}
\flushright{\begin{Arabic}
\quranayah[40][9]
\end{Arabic}}
\flushleft{\begin{malayalam}
"അവരെ നീ തിന്മകളില്‍നിന്ന് അകറ്റിനിര്‍ത്തേണമേ. ഉയിര്‍ത്തെഴുന്നേല്‍പുനാളില്‍ നീ ആരെ തിന്മയില്‍ നിന്ന് കാക്കുന്നുവോ, അവനോട് നീ തീര്‍ച്ചയായും കരുണ കാണിച്ചിരിക്കുന്നു. അതിമഹത്തായ വിജയവും അതുതന്നെ."
\end{malayalam}}
\flushright{\begin{Arabic}
\quranayah[40][10]
\end{Arabic}}
\flushleft{\begin{malayalam}
സത്യത്തെ തള്ളിപ്പറഞ്ഞവരോട് അന്ന് വിളിച്ചുപറയും: "ഇന്ന് നിങ്ങള്‍ക്ക് നിങ്ങളോടുതന്നെ കഠിനമായ വെറുപ്പുണ്ട്. എന്നാല്‍ നിങ്ങളെ സത്യവിശ്വാസത്തിലേക്ക് വിളിക്കുകയും നിങ്ങളതിനെ തള്ളിക്കളയുകയും ചെയ്തപ്പോഴുള്ള അല്ലാഹുവിന്റെ വെറുപ്പ് ഇതിനെക്കാള്‍ എത്രയോ രൂക്ഷമായിരുന്നു."
\end{malayalam}}
\flushright{\begin{Arabic}
\quranayah[40][11]
\end{Arabic}}
\flushleft{\begin{malayalam}
അവര്‍ പറയും: "ഞങ്ങളുടെ നാഥാ, നീ ഞങ്ങളെ രണ്ടുതവണ മരിപ്പിച്ചു. രണ്ടു തവണ ജീവിപ്പിക്കുകയും ചെയ്തു. ഇപ്പോള്‍ ഞങ്ങളിതാ ഞങ്ങളുടെ കുറ്റങ്ങള്‍ ഏറ്റുപറയുന്നു. അതിനാല്‍ ഞങ്ങള്‍ക്ക് ഇവിടെനിന്ന് പുറത്തുകടക്കാന്‍ വല്ല വഴിയുമുണ്ടോ?"
\end{malayalam}}
\flushright{\begin{Arabic}
\quranayah[40][12]
\end{Arabic}}
\flushleft{\begin{malayalam}
ഈ അവസ്ഥക്കു കാരണമിതാണ്. ഏകനായ അല്ലാഹുവിനോട് പ്രാര്‍ഥിച്ചപ്പോള്‍ നിങ്ങളത് നിരാകരിച്ചു. അവനില്‍ മറ്റുള്ളവരെ പങ്കുചേര്‍ത്തപ്പോള്‍ നിങ്ങളത് വിശ്വസിക്കുകയും ചെയ്തു. എന്നാലിന്ന് വിധിത്തീര്‍പ്പ് മഹാനും അത്യുന്നതനുമായ ദൈവത്തിന്റേതാണ്.
\end{malayalam}}
\flushright{\begin{Arabic}
\quranayah[40][13]
\end{Arabic}}
\flushleft{\begin{malayalam}
അവനാണ് നിങ്ങള്‍ക്ക് തന്റെ ദൃഷ്ടാന്തങ്ങള്‍ കാണിച്ചുതന്നത്. ആകാശത്തു നിന്ന് അവന്‍ നിങ്ങള്‍ക്ക് അന്നം ഇറക്കിത്തരുന്നു. പശ്ചാത്തപിച്ചു മടങ്ങുന്നവന്‍ മാത്രമാണ് ചിന്തിച്ചു മനസ്സിലാക്കുന്നത്.
\end{malayalam}}
\flushright{\begin{Arabic}
\quranayah[40][14]
\end{Arabic}}
\flushleft{\begin{malayalam}
അതിനാല്‍ കീഴ്വണക്കം അല്ലാഹുവിനു മാത്രമാക്കി അവനോട് പ്രാര്‍ഥിക്കുക. സത്യനിഷേധികള്‍ക്ക് അതെത്ര അനിഷ്ടകരമാണെങ്കിലും!
\end{malayalam}}
\flushright{\begin{Arabic}
\quranayah[40][15]
\end{Arabic}}
\flushleft{\begin{malayalam}
അവന്‍ ഉന്നത പദവികളുടെ ഉടമയാണ്. സിംഹാസനത്തിനധിപനും. തന്റെ ദാസന്മാരില്‍ താനിച്ഛിക്കുന്നവര്‍ക്ക് തന്റെ സന്ദേശത്തിന്റെ ചൈതന്യം അവന്‍ നല്‍കുന്നു. കൂടിക്കാഴ്ചയുടെ നാളിനെപ്പറ്റി മുന്നറിയിപ്പ് നല്‍കാനാണിത്.
\end{malayalam}}
\flushright{\begin{Arabic}
\quranayah[40][16]
\end{Arabic}}
\flushleft{\begin{malayalam}
എല്ലാവരും പുറത്തുവരുന്ന ദിനമാണ് അതുണ്ടാവുക. അന്ന് അവരുടെ ഒരു കാര്യവും അല്ലാഹുവില്‍ നിന്നൊളിഞ്ഞിരിക്കുകയില്ല. ആര്‍ക്കാണ് അന്ന് ആധിപത്യം? ഏകനും എല്ലാറ്റിനെയും അടക്കിഭരിക്കുന്നവനുമായ അല്ലാഹുവിനു മാത്രം.
\end{malayalam}}
\flushright{\begin{Arabic}
\quranayah[40][17]
\end{Arabic}}
\flushleft{\begin{malayalam}
അന്ന് ഓരോ വ്യക്തിക്കും അവന്‍ സമ്പാദിച്ചതിന്റെ പ്രതിഫലം നല്‍കും. അന്ന് ഒരനീതിയുമുണ്ടാവില്ല. അല്ലാഹു വളരെവേഗം വിചാരണ ചെയ്യുന്നവനാണ്.
\end{malayalam}}
\flushright{\begin{Arabic}
\quranayah[40][18]
\end{Arabic}}
\flushleft{\begin{malayalam}
അടുത്തെത്തിക്കഴിഞ്ഞ ആ നാളിനെപ്പറ്റി നീ അവര്‍ക്ക് മുന്നറിയിപ്പ് നല്‍കുക. ഹൃദയങ്ങള്‍ തൊണ്ടക്കുഴികളിലേക്കുയര്‍ന്നുവരികയും ജനം കൊടിയ ദുഃഖിതരാവുകയും ചെയ്യുന്ന സന്ദര്‍ഭമാണത്. അക്രമികള്‍ക്ക് അന്ന് ആത്മ മിത്രമോ സ്വീകാര്യനായ ശിപാര്‍ശകനോ ഉണ്ടാവുകയില്ല.
\end{malayalam}}
\flushright{\begin{Arabic}
\quranayah[40][19]
\end{Arabic}}
\flushleft{\begin{malayalam}
കണ്ണുകളുടെ കട്ടുനോട്ടവും മനസ്സുകള്‍ മറച്ചുവെക്കുന്നതുമെല്ലാം അല്ലാഹു അറിയുന്നു.
\end{malayalam}}
\flushright{\begin{Arabic}
\quranayah[40][20]
\end{Arabic}}
\flushleft{\begin{malayalam}
അല്ലാഹു സത്യനിഷ്ഠമായ വിധി ത്തീര്‍പ്പുണ്ടാക്കുന്നു. അവനെയല്ലാതെ അവര്‍ വിളിച്ചു പ്രാര്‍ഥിക്കുന്നവരാരുംതന്നെ ഒന്നിലും ഒരു തീര്‍പ്പും കല്‍പിക്കുന്നില്ല. അല്ലാഹു എല്ലാം കേള്‍ക്കുന്നവനും കാണുന്നവനുമാണ്.
\end{malayalam}}
\flushright{\begin{Arabic}
\quranayah[40][21]
\end{Arabic}}
\flushleft{\begin{malayalam}
ഇക്കൂട്ടര്‍ ഭൂമിയില്‍ സഞ്ചരിച്ച് തങ്ങള്‍ക്കു മുമ്പുണ്ടായിരുന്നവരുടെ അന്ത്യം എവ്വിധമായിരുന്നുവെന്ന് കണ്ട് മനസ്സിലാക്കിയിട്ടില്ലേ? അവര്‍ കരുത്ത് കൊണ്ടും ഭൂമിയില്‍ ബാക്കിവെച്ച പ്രൌഢമായ പാരമ്പര്യംകൊണ്ടും ഇവരെക്കാളേറെ പ്രബലന്മാരായിരുന്നു. അങ്ങനെ അവരുടെ തെറ്റുകുറ്റങ്ങള്‍ കാരണം അല്ലാഹു അവരെ പിടികൂടി. അല്ലാഹുവിന്റെ ശിക്ഷയില്‍നിന്ന് അവരെ രക്ഷിക്കാന്‍ ആരുമുണ്ടായിരുന്നില്ല.
\end{malayalam}}
\flushright{\begin{Arabic}
\quranayah[40][22]
\end{Arabic}}
\flushleft{\begin{malayalam}
അതിനു കാരണമിതാണ്. അവരിലേക്കുള്ള ദൈവദൂതന്മാര്‍ വ്യക്തമായ തെളിവുകളുമായി അവരുടെ അടുത്തെത്താറുണ്ടായിരുന്നു. അപ്പോഴെല്ലാം അവര്‍ ആ ദൂതന്മാരെ തള്ളിപ്പറഞ്ഞു. അതിനാല്‍ അല്ലാഹു അവരെ പിടികൂടി. നിശ്ചയം അല്ലാഹു അതിശക്തനാണ്. കഠിനമായി ശിക്ഷിക്കുന്നവനും.
\end{malayalam}}
\flushright{\begin{Arabic}
\quranayah[40][23]
\end{Arabic}}
\flushleft{\begin{malayalam}
മൂസായെ നമ്മുടെ ദൃഷ്ടാന്തങ്ങളും വ്യക്തമായ പ്രമാണവുമായി നാം അയക്കുകയുണ്ടായി.
\end{malayalam}}
\flushright{\begin{Arabic}
\quranayah[40][24]
\end{Arabic}}
\flushleft{\begin{malayalam}
ഫറവോന്റെയും ഹാമാന്റെയും ഖാറൂന്റെയും അടുത്തേക്ക്. അപ്പോള്‍ അവര്‍ പറഞ്ഞു: "ഇവന്‍ കള്ളവാദിയായ ജാലവിദ്യക്കാരനാണ്."
\end{malayalam}}
\flushright{\begin{Arabic}
\quranayah[40][25]
\end{Arabic}}
\flushleft{\begin{malayalam}
അങ്ങനെ നമ്മുടെ ഭാഗത്തുനിന്നുള്ള സത്യവുമായി അദ്ദേഹം അവരുടെ അടുത്തു ചെന്നപ്പോള്‍ അവര്‍ പറഞ്ഞു: "ഇവനോടൊപ്പം വിശ്വസിച്ചവരുടെ ആണ്‍കുട്ടികളെ നിങ്ങള്‍ കൊന്നുകളയുക. പെണ്‍കുട്ടികളെ ജീവിക്കാന്‍ വിടുക." എന്നാല്‍ സത്യനിഷേധികളുടെ തന്ത്രം പിഴച്ചുപോയി.
\end{malayalam}}
\flushright{\begin{Arabic}
\quranayah[40][26]
\end{Arabic}}
\flushleft{\begin{malayalam}
ഫറവോന്‍ പറഞ്ഞു: "എന്നെ വിടൂ. മൂസായെ ഞാന്‍ കൊല്ലുകയാണ്. അവന്‍ അവന്റെ നാഥനോട് പ്രാര്‍ഥിച്ചുകൊള്ളട്ടെ. അവന്‍ നിങ്ങളുടെ ജീവിതക്രമം മാറ്റിമറിക്കുകയോ നാട്ടില്‍ കുഴപ്പം കുത്തിപ്പൊക്കുകയോ ചെയ്തേക്കുമെന്ന് ഞാന്‍ ഭയപ്പെടുന്നു."
\end{malayalam}}
\flushright{\begin{Arabic}
\quranayah[40][27]
\end{Arabic}}
\flushleft{\begin{malayalam}
മൂസ പറഞ്ഞു: "വിചാരണ നാളില്‍ വിശ്വസിക്കാത്ത എല്ലാ അഹങ്കാരികളില്‍നിന്നും എന്റെയും നിങ്ങളുടെയും നാഥനില്‍ ഞാന്‍ ശരണം തേടുന്നു."
\end{malayalam}}
\flushright{\begin{Arabic}
\quranayah[40][28]
\end{Arabic}}
\flushleft{\begin{malayalam}
സത്യവിശ്വാസിയായ ഒരാള്‍ പറഞ്ഞു -അയാള്‍ ഫറവോന്റെ വംശത്തില്‍പെട്ട വിശ്വാസം ഒളിപ്പിച്ചുവെച്ച ഒരാളായിരുന്നു: "എന്റെ നാഥന്‍ അല്ലാഹുവാണ് എന്നു പറഞ്ഞതിന്റെ പേരില്‍ നിങ്ങള്‍ ഒരു മനുഷ്യനെ കൊല്ലുകയോ? അദ്ദേഹം നിങ്ങളുടെ നാഥനില്‍ നിന്നുള്ള വ്യക്തമായ തെളിവുകള്‍ കൊണ്ടുവന്നിട്ടും! അദ്ദേഹം കള്ളം പറയുന്നവനാണെങ്കില്‍ ആ കളവിന്റെ ദോഷഫലം അദ്ദേഹത്തിനു തന്നെയാണ്. മറിച്ച് സത്യവാനാണെങ്കിലോ, അദ്ദേഹം നിങ്ങളെ താക്കീതു ചെയ്യുന്ന ശിക്ഷകളില്‍ ചിലതെങ്കിലും നിങ്ങളെ ബാധിക്കും. തീര്‍ച്ചയായും പരിധി വിടുന്നവരെയും കള്ളം പറയുന്നവരെയും അല്ലാഹു നേര്‍വഴിയിലാക്കുകയില്ല.
\end{malayalam}}
\flushright{\begin{Arabic}
\quranayah[40][29]
\end{Arabic}}
\flushleft{\begin{malayalam}
"എന്റെ ജനമേ, ഇന്ന് നിങ്ങള്‍ക്കിവിടെ ആധിപത്യമുണ്ട്. നാട്ടില്‍ ജയിച്ചുനില്‍ക്കുന്നവരും നിങ്ങള്‍ തന്നെ. എന്നാല്‍ ദൈവശിക്ഷ വന്നെത്തിയാല്‍ നമ്മെ സഹായിക്കാന്‍ ആരാണുണ്ടാവുക?" ഫറവോന്‍ പറഞ്ഞു: "എനിക്കു ശരിയായി തോന്നുന്ന കാര്യമാണ് ഞാന്‍ നിങ്ങള്‍ക്കു കാണിച്ചുതരുന്നത്. നേര്‍വഴിയില്‍ തന്നെയാണ് ഞാന്‍ നിങ്ങളെ നയിക്കുന്നത്."
\end{malayalam}}
\flushright{\begin{Arabic}
\quranayah[40][30]
\end{Arabic}}
\flushleft{\begin{malayalam}
ആ സത്യവിശ്വാസി പറഞ്ഞു: "എന്റെ ജനമേ, ആ കക്ഷികള്‍ക്കുണ്ടായ ദുര്‍ദിനം പോലൊന്ന് നിങ്ങള്‍ക്കുമുണ്ടാകുമോയെന്ന് ഞാന്‍ ഭയപ്പെടുന്നു.
\end{malayalam}}
\flushright{\begin{Arabic}
\quranayah[40][31]
\end{Arabic}}
\flushleft{\begin{malayalam}
"നൂഹിന്റെ ജനതക്കും ആദിനും സമൂദിനും അവര്‍ക്കു ശേഷമുള്ളവര്‍ക്കും ഉണ്ടായതുപോലുള്ള അനുഭവം. അല്ലാഹു തന്റെ ദാസന്മാരോട് അതിക്രമം കാണിക്കാനുദ്ദേശിക്കുന്നില്ല.
\end{malayalam}}
\flushright{\begin{Arabic}
\quranayah[40][32]
\end{Arabic}}
\flushleft{\begin{malayalam}
"എന്റെ ജനമേ, അന്യോന്യം വിളിച്ച് അലമുറയിടേണ്ടി വരുന്ന ഒരു ദിനം നിങ്ങള്‍ക്കുണ്ടാകുമോയെന്ന് ഞാന്‍ ഭയപ്പെടുന്നു.
\end{malayalam}}
\flushright{\begin{Arabic}
\quranayah[40][33]
\end{Arabic}}
\flushleft{\begin{malayalam}
"നിങ്ങള്‍ രക്ഷക്കായി പിന്തിരിഞ്ഞോടുന്ന ദിനം. അന്ന് അല്ലാഹുവിന്റെ ശിക്ഷയില്‍നിന്ന് നിങ്ങളെ രക്ഷിക്കാന്‍ ആരുമുണ്ടാവില്ല. അല്ലാഹു വഴികേടിലാക്കുന്നവരെ നേര്‍വഴിയിലാക്കുന്ന ആരുമില്ല.
\end{malayalam}}
\flushright{\begin{Arabic}
\quranayah[40][34]
\end{Arabic}}
\flushleft{\begin{malayalam}
"വ്യക്തമായ തെളിവുകളുമായി മുമ്പ് യൂസുഫ് നിങ്ങളുടെ അടുത്ത് വന്നു. അപ്പോള്‍ അദ്ദേഹം കൊണ്ടുവന്ന സന്ദേശങ്ങളില്‍ നിങ്ങള്‍ സംശയിച്ചുകൊണ്ടേയിരുന്നു. അദ്ദേഹം മരണമടഞ്ഞപ്പോള്‍ നിങ്ങള്‍ പറഞ്ഞു: “ഇദ്ദേഹത്തിനുശേഷം അല്ലാഹു ഇനിയൊരു ദൂതനെയും അയക്കുകയേ ഇല്ലെ”ന്ന്. ഇവ്വിധം അതിരുവിടുന്നവരെയും സംശയാലുക്കളെയും അല്ലാഹു വഴികേടിലാക്കുന്നു."
\end{malayalam}}
\flushright{\begin{Arabic}
\quranayah[40][35]
\end{Arabic}}
\flushleft{\begin{malayalam}
അല്ലാഹുവില്‍നിന്ന് വന്നുകിട്ടിയ ഒരുവിധ തെളിവുമില്ലാതെ അവന്റെ വചനങ്ങളെപ്പറ്റി തര്‍ക്കിക്കുന്നവരാണവര്‍. ഇക്കാര്യം അല്ലാഹുവിന്റെയും സത്യവിശ്വാസികളുടെയും അടുത്ത് വളരെ വെറുക്കപ്പെട്ടതാണ്. അത്തരം അഹങ്കാരികളും ഗര്‍വിഷ്ഠരുമായ എല്ലാവരുടെയും ഹൃദയങ്ങള്‍ക്ക് അല്ലാഹു ഇവ്വിധം മുദ്രവെക്കുന്നു.
\end{malayalam}}
\flushright{\begin{Arabic}
\quranayah[40][36]
\end{Arabic}}
\flushleft{\begin{malayalam}
ഫറവോന്‍ പറഞ്ഞു: "ഹാമാന്‍, എനിക്ക് ഒരു ഗോപുരം ഉണ്ടാക്കിത്തരിക. ഞാന്‍ ആ വഴികളിലൊന്ന് എത്തട്ടെ.
\end{malayalam}}
\flushright{\begin{Arabic}
\quranayah[40][37]
\end{Arabic}}
\flushleft{\begin{malayalam}
"ആകാശത്തിന്റെ വഴികളില്‍. അങ്ങനെ മൂസായുടെ ദൈവത്തെ ഞാനൊന്ന് എത്തിനോക്കട്ടെ. നിശ്ചയമായും മൂസ നുണപറയുകയാണെന്നാണ് ഞാന്‍ കരുതുന്നത്." അവ്വിധം ഫറവോന്ന് അവന്റെ ചെയ്തികള്‍ ചേതോഹരമായി തോന്നി. അവന്‍ നേര്‍വഴിയില്‍നിന്ന് തടയപ്പെടുകയും ചെയ്തു. ഫറവോന്റെ തന്ത്രങ്ങളൊക്കെയും പരാജയപ്പെടുകയായിരുന്നു.
\end{malayalam}}
\flushright{\begin{Arabic}
\quranayah[40][38]
\end{Arabic}}
\flushleft{\begin{malayalam}
ആ വിശ്വാസി പറഞ്ഞു: "എന്റെ ജനമേ, നിങ്ങളെന്നെ പിന്‍പറ്റുക. ഞാന്‍ നിങ്ങളെ വിവേകത്തിന്റെ വഴിയിലൂടെ നയിക്കാം.
\end{malayalam}}
\flushright{\begin{Arabic}
\quranayah[40][39]
\end{Arabic}}
\flushleft{\begin{malayalam}
"എന്റെ ജനമേ, ഈ ഐഹിക ജീവിതസുഖം താല്‍ക്കാലിക വിഭവം മാത്രമാണ്. തീര്‍ച്ചയായും പരലോകം തന്നെയാണ് സ്ഥിരവാസത്തിനുള്ള ഭവനം."
\end{malayalam}}
\flushright{\begin{Arabic}
\quranayah[40][40]
\end{Arabic}}
\flushleft{\begin{malayalam}
ആ കാവല്‍ക്കാര്‍ തിന്മ ചെയ്താല്‍ അതിനു തുല്യമായ പ്രതിഫലമേ ഉണ്ടാവുകയുള്ളൂ. എന്നാല്‍ സ്ത്രീയാവട്ടെ, പുരുഷനാവട്ടെ, സത്യവിശ്വാസിയായി സല്‍ക്കര്‍മം പ്രവര്‍ത്തിക്കുന്നവര്‍ സ്വര്‍ഗത്തില്‍ പ്രവേശിക്കും. അവര്‍ക്കവിടെ കണക്കറ്റ ജീവിതവിഭവം ലഭിച്ചുകൊണ്ടിരിക്കും.
\end{malayalam}}
\flushright{\begin{Arabic}
\quranayah[40][41]
\end{Arabic}}
\flushleft{\begin{malayalam}
"എന്റെ ജനമേ, എന്തൊരവസ്ഥയാണെന്റേത്? ഞാന്‍ നിങ്ങളെ രക്ഷയിലേക്കു ക്ഷണിക്കുന്നു. നിങ്ങളോ എന്നെ നരകത്തിലേക്ക് വിളിക്കുന്നു.
\end{malayalam}}
\flushright{\begin{Arabic}
\quranayah[40][42]
\end{Arabic}}
\flushleft{\begin{malayalam}
"ഞാന്‍ അല്ലാഹുവെ ധിക്കരിക്കണമെന്നും എനിക്കൊട്ടും അറിഞ്ഞുകൂടാത്തവയെ ഞാനവനില്‍ പങ്കുചേര്‍ക്കണമെന്നുമാണല്ലോ നിങ്ങളെന്നോടാവശ്യപ്പെടുന്നത്. ഞാന്‍ നിങ്ങളെ വിളിക്കുന്നതോ പ്രതാപിയും ഏറെ പൊറുക്കുന്നവനുമായ ദൈവത്തിലേക്കും.
\end{malayalam}}
\flushright{\begin{Arabic}
\quranayah[40][43]
\end{Arabic}}
\flushleft{\begin{malayalam}
"സംശയമില്ല; ഏതൊന്നിലേക്കാണോ നിങ്ങളെന്നെ ക്ഷണിച്ചുകൊണ്ടിരിക്കുന്നത് അതിന് ഇഹലോകത്ത് ഒരു സന്ദേശവും നല്‍കാനില്ല. പരലോകത്തുമില്ല. നമ്മുടെയൊക്കെ മടക്കം അല്ലാഹുവിങ്കലേക്കാണ്. തീര്‍ച്ചയായും അതിക്രമികള്‍ തന്നെയാണ് നരകാവകാശികള്‍.
\end{malayalam}}
\flushright{\begin{Arabic}
\quranayah[40][44]
\end{Arabic}}
\flushleft{\begin{malayalam}
"ഇപ്പോള്‍ ഞാന്‍ പറയുന്നത് പിന്നെയൊരിക്കല്‍ നിങ്ങളോര്‍ക്കുക തന്നെ ചെയ്യും. എന്റെ സര്‍വവും ഞാനിതാ അല്ലാഹുവില്‍ സമര്‍പ്പിക്കുന്നു. തീര്‍ച്ചയായും അല്ലാഹു അവന്റെ ദാസന്മാരെ സദാ കണ്ടുകൊണ്ടിരിക്കുന്നവനാണ്."
\end{malayalam}}
\flushright{\begin{Arabic}
\quranayah[40][45]
\end{Arabic}}
\flushleft{\begin{malayalam}
അപ്പോള്‍ അവരുണ്ടാക്കിയ കുതന്ത്രങ്ങളുടെ ദുരന്തങ്ങളില്‍ നിന്നെല്ലാം അല്ലാഹു അദ്ദേഹത്തെ രക്ഷിച്ചു. ഫറവോന്റെ ആള്‍ക്കാര്‍ കടുത്ത ശിക്ഷാവലയത്തിലകപ്പെടുകയും ചെയ്തു.
\end{malayalam}}
\flushright{\begin{Arabic}
\quranayah[40][46]
\end{Arabic}}
\flushleft{\begin{malayalam}
കത്തിയാളുന്ന നരകത്തീ! രാവിലെയും വൈകുന്നേരവും അവരെ അതിനുമുമ്പില്‍ ഹാജരാക്കും. അന്ത്യസമയം വന്നെത്തുന്ന നാളില്‍ ഇങ്ങനെ ഒരു ഉത്തരവുണ്ടാകും: “ഫറവോന്റെ ആളുകളെ കൊടിയ ശിക്ഷയിലേക്ക് തള്ളിവിടുക.”
\end{malayalam}}
\flushright{\begin{Arabic}
\quranayah[40][47]
\end{Arabic}}
\flushleft{\begin{malayalam}
നരകത്തില്‍ അവര്‍ അന്യോന്യം കശപിശ കൂടുന്നതിനെക്കുറിച്ച് ഓര്‍ത്തുനോക്കൂ. അപ്പോള്‍ ഭൂമിയില്‍ ദുര്‍ബലരായിരുന്നവര്‍ കേമന്മാരായി നടിച്ചിരുന്നവരോടു പറയും: "തീര്‍ച്ചയായും ഞങ്ങള്‍ നിങ്ങളെ പിന്‍പറ്റിക്കഴിയുകയായിരുന്നു. അതിനാല്‍ ഞങ്ങളെ ഈ നരകശിക്ഷയില്‍ നിന്ന് അല്‍പമെങ്കിലും രക്ഷിക്കാന്‍ നിങ്ങള്‍ക്കാകുമോ?"
\end{malayalam}}
\flushright{\begin{Arabic}
\quranayah[40][48]
\end{Arabic}}
\flushleft{\begin{malayalam}
കേമത്തം നടിച്ചവര്‍ പറയും: "തീര്‍ച്ചയായും നാമൊക്കെ ഇവിടെ ഈ അവസ്ഥയിലാണ്. അല്ലാഹു തന്റെ ദാസന്മാര്‍ക്കിടയില്‍ വിധി നടപ്പാക്കിക്കഴിഞ്ഞു."
\end{malayalam}}
\flushright{\begin{Arabic}
\quranayah[40][49]
\end{Arabic}}
\flushleft{\begin{malayalam}
നരകാവകാശികള്‍ അതിന്റെ കാവല്‍ക്കാരോടു പറയും: "നിങ്ങള്‍ നിങ്ങളുടെ നാഥനോടൊന്നു പ്രാര്‍ഥിച്ചാലും. അവന്‍ ഒരു ദിവസത്തെ ശിക്ഷയെങ്കിലും ഞങ്ങള്‍ക്ക് ലഘൂകരിച്ചുതന്നാല്‍ നന്നായേനെ."
\end{malayalam}}
\flushright{\begin{Arabic}
\quranayah[40][50]
\end{Arabic}}
\flushleft{\begin{malayalam}
ആ കാവല്‍ക്കാര്‍ ചോദിക്കും: "നിങ്ങള്‍ക്കുള്ള ദൈവദൂതന്മാര്‍ വ്യക്തമായ തെളിവുകളുമായി നിങ്ങളുടെ അടുത്ത് വന്നിട്ടുണ്ടായിരുന്നില്ലേ?" അവര്‍ പറയും: "അതെ." അപ്പോള്‍ ആ കാവല്‍ക്കാര്‍ പറയും: "എങ്കില്‍ നിങ്ങള്‍തന്നെ പ്രാര്‍ഥിച്ചുകൊള്ളുക." സത്യനിഷേധികളുടെ പ്രാര്‍ഥന തീര്‍ത്തും നിഷ്ഫലമത്രെ.
\end{malayalam}}
\flushright{\begin{Arabic}
\quranayah[40][51]
\end{Arabic}}
\flushleft{\begin{malayalam}
തീര്‍ച്ചയായും നമ്മുടെ ദൂതന്മാരെയും സത്യവിശ്വാസികളെയും നാം സഹായിക്കും. ഈ ഐഹിക ജീവിതത്തിലും സാക്ഷികള്‍ രംഗത്തുവരുന്ന അന്ത്യനാളിലും.
\end{malayalam}}
\flushright{\begin{Arabic}
\quranayah[40][52]
\end{Arabic}}
\flushleft{\begin{malayalam}
അന്ന് അക്രമികള്‍ക്ക് അവരുടെ ഒഴികഴിവുകള്‍ ഒട്ടും ഉപകരിക്കുകയില്ല. അവര്‍ക്കാണ് കൊടും ശാപം. വളരെ ചീത്തയായ പാര്‍പ്പിടമാണ് അവര്‍ക്കുണ്ടാവുക.
\end{malayalam}}
\flushright{\begin{Arabic}
\quranayah[40][53]
\end{Arabic}}
\flushleft{\begin{malayalam}
മൂസാക്കു നാം നേര്‍വഴി നല്‍കി. ഇസ്രയേല്‍ മക്കളെ നാം വേദപുസ്തകത്തിന്റെ അവകാശികളാക്കി.
\end{malayalam}}
\flushright{\begin{Arabic}
\quranayah[40][54]
\end{Arabic}}
\flushleft{\begin{malayalam}
അത് വിചാരമതികള്‍ക്ക് വഴികാട്ടിയും ഉത്തമമായ ഉദ്ബോധനവുമായിരുന്നു.
\end{malayalam}}
\flushright{\begin{Arabic}
\quranayah[40][55]
\end{Arabic}}
\flushleft{\begin{malayalam}
അതിനാല്‍ നീ ക്ഷമിക്കുക. സംശയമില്ല; അല്ലാഹുവിന്റെ വാഗ്ദാനം സത്യമാണ്. നിന്റെ പാപങ്ങള്‍ക്ക് മാപ്പിരക്കുക. രാവിലെയും വൈകുന്നേരവും നിന്റെ നാഥനെ വാഴ്ത്തുക. അവനെ കീര്‍ത്തിക്കുക.
\end{malayalam}}
\flushright{\begin{Arabic}
\quranayah[40][56]
\end{Arabic}}
\flushleft{\begin{malayalam}
ഒരു തെളിവുമില്ലാതെ അല്ലാഹുവിന്റെ വചനങ്ങളെപ്പറ്റി തര്‍ക്കിക്കുന്നതാരോ, ഉറപ്പായും അവരുടെ ഹൃദയങ്ങളില്‍ അഹങ്കാരം മാത്രമേയുള്ളൂ. എന്നാല്‍ അവര്‍ക്കാര്‍ക്കും ഉയരങ്ങളിലെത്താനാവില്ല. അതിനാല്‍ നീ അല്ലാഹുവോട് രക്ഷതേടുക. അവന്‍ എല്ലാം കേള്‍ക്കുന്നവനും കാണുന്നവനുമാണ്.
\end{malayalam}}
\flushright{\begin{Arabic}
\quranayah[40][57]
\end{Arabic}}
\flushleft{\begin{malayalam}
ആകാശഭൂമികളുടെ സൃഷ്ടി മനുഷ്യസൃഷ്ടിയെക്കാള്‍ എത്രയോ വലിയ കാര്യമാണ്. പക്ഷേ, അധികമാളുകളും അതറിയുന്നില്ല.
\end{malayalam}}
\flushright{\begin{Arabic}
\quranayah[40][58]
\end{Arabic}}
\flushleft{\begin{malayalam}
കുരുടനും കാഴ്ചയുള്ളവനും ഒരുപോലെയല്ല. സത്യവിശ്വാസം സ്വീകരിച്ച് സല്‍ക്കര്‍മം പ്രവര്‍ത്തിച്ചവരും ചീത്ത ചെയ്തവരും സമമാവുകയില്ല. നിങ്ങള്‍ വളരെ കുറച്ചേ ചിന്തിച്ചു മനസ്സിലാക്കുന്നുള്ളൂ.
\end{malayalam}}
\flushright{\begin{Arabic}
\quranayah[40][59]
\end{Arabic}}
\flushleft{\begin{malayalam}
ആ അന്ത്യസമയം വന്നെത്തുകതന്നെ ചെയ്യും. അതിലൊട്ടും സംശയം വേണ്ട. എന്നാല്‍ മനുഷ്യരിലേറെ പേരും വിശ്വസിക്കുന്നില്ല.
\end{malayalam}}
\flushright{\begin{Arabic}
\quranayah[40][60]
\end{Arabic}}
\flushleft{\begin{malayalam}
നിങ്ങളുടെ നാഥന്‍ പറഞ്ഞിരിക്കുന്നു: നിങ്ങളെന്നോടു പ്രാര്‍ഥിക്കുക. ഞാന്‍ നിങ്ങള്‍ക്കുത്തരം തരാം. എന്നെ വഴിപ്പെടാതെ അഹന്ത നടിക്കുന്നവര്‍ ഏറെ നിന്ദ്യരായി നരകത്തില്‍ പ്രവേശിക്കും.
\end{malayalam}}
\flushright{\begin{Arabic}
\quranayah[40][61]
\end{Arabic}}
\flushleft{\begin{malayalam}
അല്ലാഹുവാണ് നിങ്ങള്‍ക്ക് രാവൊരുക്കിത്തന്നത്, നിങ്ങള്‍ ശാന്തി നേടാന്‍. പകലിനെ പ്രകാശപൂരിതമാക്കിയതും അവനാണ്. തീര്‍ച്ചയായും അല്ലാഹു മനുഷ്യരോട് ഏറെ ഔദാര്യമുള്ളവനാണ്. എന്നാല്‍ മനുഷ്യരിലേറെ പേരും നന്ദി കാണിക്കുന്നില്ല.
\end{malayalam}}
\flushright{\begin{Arabic}
\quranayah[40][62]
\end{Arabic}}
\flushleft{\begin{malayalam}
അവനാണ് നിങ്ങളുടെ നാഥനായ അല്ലാഹു. സകല വസ്തുക്കളുടെയും സ്രഷ്ടാവ്. അവനല്ലാതെ ദൈവമില്ല. എന്നിട്ടും നിങ്ങളെങ്ങനെ വഴിതെറ്റിപ്പോകുന്നു?
\end{malayalam}}
\flushright{\begin{Arabic}
\quranayah[40][63]
\end{Arabic}}
\flushleft{\begin{malayalam}
അല്ലാഹുവിന്റെ വചനങ്ങളെ തള്ളിപ്പറയുന്നവര്‍ ഇങ്ങനെത്തന്നെയാണ് വഴിതെറ്റിപ്പോകുന്നത്.
\end{malayalam}}
\flushright{\begin{Arabic}
\quranayah[40][64]
\end{Arabic}}
\flushleft{\begin{malayalam}
അല്ലാഹു തന്നെയാണ് നിങ്ങള്‍ക്കു ഭൂമിയെ പാര്‍ക്കാന്‍ പറ്റിയതാക്കിയത്. മാനത്തെ മേല്‍പ്പുരയാക്കിയതും അവന്‍ തന്നെ. അവന്‍ നിങ്ങള്‍ക്കു രൂപമേകി. ആ രൂപത്തെ ഏറെ മികവുറ്റതാക്കി. വിശിഷ്ട വസ്തുക്കളില്‍നിന്ന് നിങ്ങള്‍ക്ക് അന്നം തന്നു. ആ അല്ലാഹു തന്നെയാണ് നിങ്ങളുടെ നാഥന്‍. പ്രപഞ്ചനാഥനായ അല്ലാഹു അനുഗ്രഹപൂര്‍ണന്‍ തന്നെ.
\end{malayalam}}
\flushright{\begin{Arabic}
\quranayah[40][65]
\end{Arabic}}
\flushleft{\begin{malayalam}
അവന്‍ എന്നെന്നും ജീവിച്ചിരിക്കുന്നവനാണ്. അവനല്ലാതെ ദൈവമില്ല. അതിനാല്‍ ആത്മാര്‍ഥതയോടെ അവനു മാത്രം കീഴ്പ്പെടുക. അവനോടു മാത്രം പ്രാര്‍ഥിക്കുക. പ്രപഞ്ചനാഥനായ അല്ലാഹുവിനാണ് സര്‍വസ്തുതിയും.
\end{malayalam}}
\flushright{\begin{Arabic}
\quranayah[40][66]
\end{Arabic}}
\flushleft{\begin{malayalam}
പറയുക: അല്ലാഹുവെക്കൂടാതെ നിങ്ങള്‍ പ്രാര്‍ഥിച്ചുകൊണ്ടിരിക്കുന്നവയെ പൂജിക്കാന്‍ എനിക്കനുവാദമില്ല. എനിക്കെന്റെ നാഥനില്‍ നിന്നു വ്യക്തമായ തെളിവുകള്‍ വന്നെത്തിയിരിക്കുന്നു. പ്രപഞ്ചനാഥന്ന് സമസ്തവും സമര്‍പ്പിക്കാനാണ് അവനെന്നോടു കല്‍പിച്ചിരിക്കുന്നത്.
\end{malayalam}}
\flushright{\begin{Arabic}
\quranayah[40][67]
\end{Arabic}}
\flushleft{\begin{malayalam}
അവനാണ് നിങ്ങളെ മണ്ണില്‍ നിന്ന് സൃഷ്ടിച്ചത്. പിന്നെ ബീജകണത്തില്‍ നിന്ന്. പിന്നീട് ഭ്രൂണത്തില്‍നിന്നും. തുടര്‍ന്ന് ശിശുവായി അവന്‍ നിങ്ങളെ പുറത്തുകൊണ്ടുവരുന്നു. അതിനുശേഷം നിങ്ങള്‍ കരുത്തുനേടാനാണിത്. അവസാനം നിങ്ങള്‍ വൃദ്ധരായിത്തീരാനും. നിങ്ങളില്‍ ചിലര്‍ നേരത്തെ തന്നെ മരണമടയുന്നു. നിങ്ങള്‍ക്കു നിശ്ചയിക്കപ്പെട്ട അവധിയിലെത്താനുമാണിത്. ഒരുവേള നിങ്ങള്‍ ചിന്തിച്ച് മനസ്സിലാക്കിയെങ്കിലോ.
\end{malayalam}}
\flushright{\begin{Arabic}
\quranayah[40][68]
\end{Arabic}}
\flushleft{\begin{malayalam}
അവനാണ് ജീവിപ്പിക്കുന്നതും മരിപ്പിക്കുന്നതും. അവനൊരു കാര്യം തീരുമാനിച്ചുകഴിഞ്ഞാല്‍ “ഉണ്ടാവട്ടെ” എന്ന് പറയുകയേ വേണ്ടൂ, അതുണ്ടാവുന്നു.
\end{malayalam}}
\flushright{\begin{Arabic}
\quranayah[40][69]
\end{Arabic}}
\flushleft{\begin{malayalam}
അല്ലാഹുവിന്റെ വചനങ്ങളെപ്പറ്റി തര്‍ക്കിക്കുന്നവരെ നീ കണ്ടിട്ടില്ലേ. അവരെങ്ങനെയാണ് വഴിതെറ്റിപ്പോകുന്നതെന്ന്.
\end{malayalam}}
\flushright{\begin{Arabic}
\quranayah[40][70]
\end{Arabic}}
\flushleft{\begin{malayalam}
വേദപുസ്തകത്തെയും നമ്മുടെ ദൂതന്മാരോടൊപ്പം നാമയച്ച സന്ദേശത്തെയും തള്ളിപ്പറഞ്ഞവരാണവര്‍. ഏറെ വൈകാതെ എല്ലാം അവരറിയും.
\end{malayalam}}
\flushright{\begin{Arabic}
\quranayah[40][71]
\end{Arabic}}
\flushleft{\begin{malayalam}
അവരുടെ കഴുത്തുകളില്‍ കുരുക്കുകളും ചങ്ങലകളുമായി അവര്‍ വലിച്ചിഴക്കപ്പെടുമ്പോഴായിരിക്കുമത്.
\end{malayalam}}
\flushright{\begin{Arabic}
\quranayah[40][72]
\end{Arabic}}
\flushleft{\begin{malayalam}
ചുട്ടുപൊള്ളുന്ന വെള്ളത്തിലൂടെ; പിന്നെയവര്‍ നരകത്തീയില്‍ എരിയും.
\end{malayalam}}
\flushright{\begin{Arabic}
\quranayah[40][73]
\end{Arabic}}
\flushleft{\begin{malayalam}
പിന്നീട് അവരോടിങ്ങനെ ചോദിക്കും: "നിങ്ങള്‍ പങ്കുചേര്‍ത്തിരുന്നവരെവിടെ?"
\end{malayalam}}
\flushright{\begin{Arabic}
\quranayah[40][74]
\end{Arabic}}
\flushleft{\begin{malayalam}
"അല്ലാഹുവെക്കൂടാതെ. ..?" അവര്‍ പറയും: "ആ പങ്കാളികള്‍ ഞങ്ങളെ വിട്ടുപോയിരിക്കുന്നു. അല്ല; ഞങ്ങള്‍ മുമ്പ് ഒന്നിനെയും വിളിച്ചുപ്രാര്‍ഥിച്ചിരുന്നില്ല." ഇങ്ങനെയാണ് അല്ലാഹു സത്യനിഷേധികളെ വഴികേടിലാക്കുന്നത്.
\end{malayalam}}
\flushright{\begin{Arabic}
\quranayah[40][75]
\end{Arabic}}
\flushleft{\begin{malayalam}
നിങ്ങള്‍ ഭൂമിയില്‍ അനര്‍ഹമായി പൊങ്ങച്ചം കാണിച്ചതിനാലും അഹങ്കരിച്ചതിനാലുമാണിത്.
\end{malayalam}}
\flushright{\begin{Arabic}
\quranayah[40][76]
\end{Arabic}}
\flushleft{\begin{malayalam}
ഇനി നിങ്ങള്‍ നരക കവാടങ്ങള്‍ കടന്നുകൊള്ളുക. നിങ്ങളവിടെ സ്ഥിരവാസികളായിരിക്കും. അഹങ്കാരികളുടെ താവളം വളരെ ചീത്ത തന്നെ.
\end{malayalam}}
\flushright{\begin{Arabic}
\quranayah[40][77]
\end{Arabic}}
\flushleft{\begin{malayalam}
അതിനാല്‍ നീ ക്ഷമിക്കുക. അല്ലാഹുവിന്റെ വാഗ്ദാനം സത്യമാണ്. നാം അവര്‍ക്കു വാഗ്ദാനം ചെയ്യുന്ന ശിക്ഷകളില്‍ ചിലത് നിന്നെ നാം കാണിച്ചുതന്നേക്കാം. അല്ലെങ്കില്‍ അതിനു മുമ്പെ നിന്നെ നാം മരിപ്പിച്ചേക്കാം. ഏതായാലും അവര്‍ തിരിച്ചുവരിക നമ്മുടെ അടുത്തേക്കാണ്.
\end{malayalam}}
\flushright{\begin{Arabic}
\quranayah[40][78]
\end{Arabic}}
\flushleft{\begin{malayalam}
നിനക്കു മുമ്പ് നിരവധി ദൂതന്മാരെ നാം നിയോഗിച്ചിട്ടുണ്ട്. അവരില്‍ ചിലരുടെ ചരിത്രം നിനക്കു നാം വിവരിച്ചുതന്നിരിക്കുന്നു. വിവരിച്ചുതരാത്ത ചിലരുമുണ്ട്. ഒരു ദൈവദൂതന്നും അല്ലാഹുവിന്റെ അനുമതിയോടെയല്ലാതെ ഒരു ദൃഷ്ടാന്തവും കൊണ്ടുവരാനാവില്ല. അല്ലാഹുവിന്റെ കല്‍പന വന്നാല്‍ ന്യായമായ വിധിത്തീര്‍പ്പുണ്ടാവും. അതോടെ അസത്യവാദികള്‍ കൊടും നഷ്ടത്തിലകപ്പെടും.
\end{malayalam}}
\flushright{\begin{Arabic}
\quranayah[40][79]
\end{Arabic}}
\flushleft{\begin{malayalam}
നിങ്ങള്‍ക്കു കന്നുകാലികളെ സൃഷ്ടിച്ചുതന്നത് അല്ലാഹുവാണ്. അവയില്‍ ചിലത് നിങ്ങള്‍ക്കു സവാരി ചെയ്യാനാണ്. ചിലത് ആഹരിക്കാനും.
\end{malayalam}}
\flushright{\begin{Arabic}
\quranayah[40][80]
\end{Arabic}}
\flushleft{\begin{malayalam}
അവകൊണ്ട് നിങ്ങള്‍ക്ക് വളരെയേറെ പ്രയോജനമുണ്ട്. അവയിലൂടെ നിങ്ങളുടെ മനസ്സിലെ പല ആഗ്രഹങ്ങളും നിങ്ങള്‍ എത്തിപ്പിടിക്കുന്നു. അവയുടെ പുറത്തിരുന്നും കപ്പലുകളിലുമാണല്ലോ നിങ്ങള്‍ യാത്ര ചെയ്യുന്നത്.
\end{malayalam}}
\flushright{\begin{Arabic}
\quranayah[40][81]
\end{Arabic}}
\flushleft{\begin{malayalam}
അല്ലാഹു തന്റെ ദൃഷ്ടാന്തങ്ങള്‍ നിങ്ങള്‍ക്കിതാ കാണിച്ചുതരുന്നു. എന്നിട്ടും അല്ലാഹുവിന്റെ ദൃഷ്ടാന്തങ്ങളില്‍ ഏതിനെയാണ് നിങ്ങള്‍ നിഷേധിക്കുന്നത്?
\end{malayalam}}
\flushright{\begin{Arabic}
\quranayah[40][82]
\end{Arabic}}
\flushleft{\begin{malayalam}
അവര്‍ ഭൂമിയില്‍ സഞ്ചരിച്ച് അവരുടെ മുമ്പുണ്ടായിരുന്നവരുടെ അന്ത്യം എവ്വിധമായിരുന്നുവെന്ന് നോക്കി മനസ്സിലാക്കിയിട്ടില്ലേ? അവര്‍ ഇവരെക്കാള്‍ അംഗബലമുള്ളവരായിരുന്നു. കരുത്തുകൊണ്ടും ഭൂമിയില്‍ ശേഷിപ്പിച്ച പൈതൃകം കൊണ്ടും കൂടുതല്‍ പ്രബലരായിരുന്നു. എന്നിട്ടും അവര്‍ സമ്പാദിച്ചുകൊണ്ടിരുന്നതൊന്നും അവര്‍ക്ക് ഉപകരിച്ചില്ല.
\end{malayalam}}
\flushright{\begin{Arabic}
\quranayah[40][83]
\end{Arabic}}
\flushleft{\begin{malayalam}
അങ്ങനെ അവര്‍ക്കുള്ള ദൂതന്മാര്‍ വ്യക്തമായ തെളിവുകളുമായി അവരുടെ അടുത്തു ചെന്നപ്പോള്‍ തങ്ങളുടെ വശമുള്ള വിജ്ഞാനംകൊണ്ട് അവര്‍ പുളകംകൊള്ളുകയാണുണ്ടായത്. അതിനാല്‍ അവര്‍ പരിഹസിച്ചുകൊണ്ടിരുന്ന ശിക്ഷ അവരെ ആവരണം ചെയ്തു.
\end{malayalam}}
\flushright{\begin{Arabic}
\quranayah[40][84]
\end{Arabic}}
\flushleft{\begin{malayalam}
നമ്മുടെ ശിക്ഷ നേരില്‍ കണ്ടപ്പോള്‍ അവര്‍ പറഞ്ഞു: "ഞങ്ങളിതാ ഏകനായ അല്ലാഹുവില്‍ മാത്രം വിശ്വസിക്കുന്നു. അവനില്‍ പങ്കുചേര്‍ത്തിരുന്ന സകലതിനെയും ഞങ്ങളിതാ തള്ളിപ്പറയുന്നു."
\end{malayalam}}
\flushright{\begin{Arabic}
\quranayah[40][85]
\end{Arabic}}
\flushleft{\begin{malayalam}
എന്നാല്‍ നമ്മുടെ ശിക്ഷ കണ്ടുകഴിഞ്ഞ ശേഷമുള്ള വിശ്വാസം അവര്‍ക്കൊട്ടും ഉപകരിച്ചില്ല. അല്ലാഹു തന്റെ ദാസന്മാരുടെ കാര്യത്തില്‍ നേരത്തെ സ്വീകരിച്ചുപോന്ന നടപടിക്രമമാണിത്. അതോടെ സത്യനിഷേധികള്‍ കൊടിയ നഷ്ടത്തിലകപ്പെടുന്നു.
\end{malayalam}}
\chapter{\textmalayalam{ഫുസ്സിലത്ത്}}
\begin{Arabic}
\Huge{\centerline{\basmalah}}\end{Arabic}
\flushright{\begin{Arabic}
\quranayah[41][1]
\end{Arabic}}
\flushleft{\begin{malayalam}
ഹാ-മീം.
\end{malayalam}}
\flushright{\begin{Arabic}
\quranayah[41][2]
\end{Arabic}}
\flushleft{\begin{malayalam}
പരമകാരുണികനും ദയാപരനുമായ അല്ലാഹുവില്‍ നിന്ന് അവതീര്‍ണമായതാണിത്.
\end{malayalam}}
\flushright{\begin{Arabic}
\quranayah[41][3]
\end{Arabic}}
\flushleft{\begin{malayalam}
വചനങ്ങളെല്ലാം വിശദമായി വിവരിക്കപ്പെട്ട വേദപുസ്തകം. അറബി ഭാഷയിലുള്ള ഖുര്‍ആന്‍. മനസ്സിലാക്കുന്ന ജനത്തിനുവേണ്ടിയാണിത്.
\end{malayalam}}
\flushright{\begin{Arabic}
\quranayah[41][4]
\end{Arabic}}
\flushleft{\begin{malayalam}
ഇത് ശുഭവാര്‍ത്ത അറിയിക്കുന്നതാണ്. മുന്നറിയിപ്പു നല്‍കുന്നതും. എന്നിട്ടും ജനങ്ങളിലേറെ പേരും ഇതിനെ അവഗണിച്ചു. അവരിതു കേള്‍ക്കുന്നുപോലുമില്ല.
\end{malayalam}}
\flushright{\begin{Arabic}
\quranayah[41][5]
\end{Arabic}}
\flushleft{\begin{malayalam}
അവര്‍ പറയുന്നു: "നീ ഞങ്ങളെ ക്ഷണിക്കുന്ന സന്ദേശത്തിനു നേരെ ഞങ്ങളുടെ ഹൃദയങ്ങള്‍ കൊട്ടിയടക്കപ്പെട്ടിരിക്കുന്നു. ഞങ്ങളുടെ കാതുകളെ ബധിരത ബാധിച്ചിരിക്കുന്നു. നമുക്കിടയില്‍ ഒരു മറയുണ്ട്. അതിനാല്‍ നീ നിന്റെ പണി ചെയ്യുക. ഞങ്ങള്‍ ഞങ്ങളുടെ പണി നോക്കാം."
\end{malayalam}}
\flushright{\begin{Arabic}
\quranayah[41][6]
\end{Arabic}}
\flushleft{\begin{malayalam}
പറയുക: "ഞാന്‍ നിങ്ങളെപ്പോലുള്ള ഒരു മനുഷ്യന്‍ മാത്രമാണ്. എന്നാല്‍ എനിക്കിങ്ങനെ ദിവ്യബോധനം ലഭിക്കുന്നു: “നിങ്ങള്‍ക്ക് ഒരേയൊരു ദൈവമേയുള്ളൂ. അതിനാല്‍ നിങ്ങള്‍ അവങ്കലേക്കുള്ള നേര്‍വഴിയില്‍ നിലകൊള്ളുക. അവനോടു പാപമോചനം തേടുക. ബഹുദൈവ വിശ്വാസികള്‍ക്കാണ് കൊടും നാശം."
\end{malayalam}}
\flushright{\begin{Arabic}
\quranayah[41][7]
\end{Arabic}}
\flushleft{\begin{malayalam}
സകാത്ത് നല്‍കാത്തവരാണവര്‍. പരലോകത്തെ തീര്‍ത്തും തള്ളിപ്പറഞ്ഞവരും.
\end{malayalam}}
\flushright{\begin{Arabic}
\quranayah[41][8]
\end{Arabic}}
\flushleft{\begin{malayalam}
സംശയമില്ല; സത്യവിശ്വാസം സ്വീകരിക്കുകയും സല്‍ക്കര്‍മങ്ങള്‍ പ്രവര്‍ത്തിക്കുകയും ചെയ്തവര്‍ക്ക് അറുതിയില്ലാത്ത പ്രതിഫലമുണ്ട്.
\end{malayalam}}
\flushright{\begin{Arabic}
\quranayah[41][9]
\end{Arabic}}
\flushleft{\begin{malayalam}
പറയുക: "രണ്ടു നാളുകള്‍കൊണ്ട് ഭൂമിയെ സൃഷ്ടിച്ച ദൈവത്തെ നിഷേധിക്കുകയാണോ നിങ്ങള്‍? നിങ്ങളവന് സമന്മാരെ സങ്കല്‍പിക്കുകയുമാണോ? അറിയുക: അവനാണ് സര്‍വലോകങ്ങളുടെയും സംരക്ഷകന്‍."
\end{malayalam}}
\flushright{\begin{Arabic}
\quranayah[41][10]
\end{Arabic}}
\flushleft{\begin{malayalam}
അവന്‍ ഭൂമിയുടെ മുകള്‍പരപ്പില്‍ ഉറച്ചുനില്‍ക്കുന്ന മലകളുണ്ടാക്കി. അതില്‍ അളവറ്റ അനുഗ്രഹങ്ങളൊരുക്കി. അതിലെ ആഹാരങ്ങള്‍ ക്രമപ്പെടുത്തി. നാലു നാളുകളിലായാണ് ഇതൊക്കെ ചെയ്തത്. ആവശ്യക്കാര്‍ക്കെല്ലാം ശരിയായ അനുപാതത്തിലാണ് അതില്‍ ആഹാരമൊരുക്കിയത്.
\end{malayalam}}
\flushright{\begin{Arabic}
\quranayah[41][11]
\end{Arabic}}
\flushleft{\begin{malayalam}
പിന്നെ അവന്‍ ആകാശത്തിനു നേരെ തിരിഞ്ഞു. അത് പുകയായിരുന്നു. അതിനോടും ഭൂമിയോടും അവന്‍ പറഞ്ഞു: "ഉണ്ടായിവരിക; നിങ്ങളിഷ്ടപ്പെട്ടാലും ഇല്ലെങ്കിലും." അപ്പോള്‍ അവ രണ്ടും അറിയിച്ചു: "ഞങ്ങളിതാ അനുസരണമുള്ളവയായി വന്നിരിക്കുന്നു."
\end{malayalam}}
\flushright{\begin{Arabic}
\quranayah[41][12]
\end{Arabic}}
\flushleft{\begin{malayalam}
അങ്ങനെ അവന്‍ രണ്ടു നാളുകളിലായി ഏഴാകാശങ്ങളുണ്ടാക്കി. ഓരോ ആകാശത്തിനും അതിന്റെ നിയമം ബോധനംനല്‍കി. അടുത്തുള്ള ആകാശത്തെ വിളക്കുകളാല്‍ അലങ്കരിച്ചു. നല്ലപോലെ ഭദ്രവുമാക്കി. പ്രതാപിയും സകലതും അറിയുന്നവനുമായ അല്ലാഹുവിന്റെ ക്രമീകരണമാണിത്.
\end{malayalam}}
\flushright{\begin{Arabic}
\quranayah[41][13]
\end{Arabic}}
\flushleft{\begin{malayalam}
ഇനിയും അവരവഗണിക്കുന്നുവെങ്കില്‍ പറയുക: "ആദ്, സമൂദ് സമൂഹങ്ങള്‍ക്കു സംഭവിച്ചത് പോലുള്ള ശിക്ഷയെ സംബന്ധിച്ച് ഞാനിതാ നിങ്ങളെ താക്കീത് ചെയ്യുന്നു."
\end{malayalam}}
\flushright{\begin{Arabic}
\quranayah[41][14]
\end{Arabic}}
\flushleft{\begin{malayalam}
ദൈവദൂതന്മാര്‍ മുന്നിലൂടെയും പിന്നിലൂടെയും അവരെ സമീപിച്ച് ആവശ്യപ്പെട്ടു: "നിങ്ങള്‍ അല്ലാഹുവിനല്ലാതെ വഴിപ്പെടരുത്." അപ്പോള്‍ അവര്‍ പറഞ്ഞു: "ഞങ്ങളുടെ നാഥന്‍ ഇച്ഛിച്ചിരുന്നെങ്കില്‍ അവന്‍ മലക്കുകളെ ഇറക്കുമായിരുന്നു. അതിനാല്‍ ഏതൊരു സന്ദേശവുമായാണോ നിങ്ങളെ അയച്ചിരിക്കുന്നത് ആ സന്ദേശത്തെ ഞങ്ങള്‍ തള്ളിക്കളയുന്നു."
\end{malayalam}}
\flushright{\begin{Arabic}
\quranayah[41][15]
\end{Arabic}}
\flushleft{\begin{malayalam}
അങ്ങനെ ആദ് സമുദായം ഭൂമിയില്‍ അനര്‍ഹമായി അഹങ്കരിച്ചു. അവര്‍ പറഞ്ഞു: "ഞങ്ങളേക്കാള്‍ കരുത്തുള്ള ആരുണ്ട്?" അവരെ പടച്ച അല്ലാഹു അവരെക്കാളെത്രയോ കരുത്തനാണെന്ന് അവര്‍ കാണുന്നില്ലേ? അവര്‍ നമ്മുടെ വചനങ്ങളെ നിഷേധിക്കുന്നവരായിരുന്നു.
\end{malayalam}}
\flushright{\begin{Arabic}
\quranayah[41][16]
\end{Arabic}}
\flushleft{\begin{malayalam}
അവസാനം നാം ദുരിതം നിറഞ്ഞ നാളുകളില്‍ അവരുടെ നേരെ അത്യുഗ്രമായ കൊടുങ്കാറ്റയച്ചു. ഐഹികജീവിതത്തില്‍ തന്നെ അവരെ അപമാനകരമായ ശിക്ഷ ആസ്വദിപ്പിക്കാനായിരുന്നു അത്. പരലോകശിക്ഷ ഇതിനെക്കാള്‍ എത്രയോ കൂടുതല്‍ അപമാനകരമാണ്. അവര്‍ക്കെങ്ങുനിന്നും ഒരു സഹായവും കിട്ടുകയില്ല.
\end{malayalam}}
\flushright{\begin{Arabic}
\quranayah[41][17]
\end{Arabic}}
\flushleft{\begin{malayalam}
എന്നാല്‍ സമൂദിന്റെ സ്ഥിതിയോ, നാമവര്‍ക്ക് നേര്‍വഴി കാണിച്ചുകൊടുത്തു. എന്നാല്‍ നേര്‍വഴി കാണുന്നതിനേക്കാള്‍ അവരിഷ്ടപ്പെട്ടത് അന്ധതയാണ്. അതിനാല്‍ അപമാനകരമായ കൊടിയ ശിക്ഷ അവരെ പിടികൂടി. അവര്‍ പ്രവര്‍ത്തിച്ചുകൊണ്ടിരുന്നതിന്റെ ഫലമായിരുന്നു അത്.
\end{malayalam}}
\flushright{\begin{Arabic}
\quranayah[41][18]
\end{Arabic}}
\flushleft{\begin{malayalam}
സത്യവിശ്വാസം സ്വീകരിക്കുകയും ഭക്തി പുലര്‍ത്തുകയും ചെയ്തവരെ നാം രക്ഷപ്പെടുത്തി.
\end{malayalam}}
\flushright{\begin{Arabic}
\quranayah[41][19]
\end{Arabic}}
\flushleft{\begin{malayalam}
ദൈവത്തിന്റെ ശത്രുക്കളെ നരകത്തിലേക്ക് നയിക്കാനായി ഒരുമിച്ചുചേര്‍ക്കുന്ന നാളിനെക്കുറിച്ച് ഓര്‍ത്തുനോക്കുക.
\end{malayalam}}
\flushright{\begin{Arabic}
\quranayah[41][20]
\end{Arabic}}
\flushleft{\begin{malayalam}
അവരവിടെ എത്തിയാല്‍ അവര്‍ പ്രവര്‍ത്തിച്ചുകൊണ്ടിരുന്നതിന്റെ ഫലമായി അവരുടെ കാതുകളും കണ്ണുകളും ചര്‍മങ്ങളും അവര്‍ക്കെതിരെ സാക്ഷ്യംവഹിക്കും.
\end{malayalam}}
\flushright{\begin{Arabic}
\quranayah[41][21]
\end{Arabic}}
\flushleft{\begin{malayalam}
അപ്പോള്‍ അവര്‍ തൊലിയോടു ചോദിക്കും: "നിങ്ങളെന്തിനാണ് ഞങ്ങള്‍ക്കെതിരെ സാക്ഷ്യംവഹിച്ചത്?" അവ പറയും: "സകല വസ്തുക്കള്‍ക്കും സംസാരകഴിവു നല്‍കിയ അല്ലാഹു ഞങ്ങളെയും സംസാരിപ്പിച്ചു." അവനാണ് ആദ്യതവണ നിങ്ങളെ സൃഷ്ടിച്ചത്. നിങ്ങള്‍ തിരിച്ചുചെല്ലേണ്ടതും അവങ്കലേക്കുതന്നെ.
\end{malayalam}}
\flushright{\begin{Arabic}
\quranayah[41][22]
\end{Arabic}}
\flushleft{\begin{malayalam}
നിങ്ങളുടെ കാതുകളും കണ്ണുകളും ചര്‍മങ്ങളും നിങ്ങള്‍ക്കെതിരെ സാക്ഷ്യം വഹിക്കുമെന്ന് നിങ്ങള്‍ കരുതിയിരുന്നില്ല. അതിനാല്‍ അവയില്‍ നിന്ന് നിങ്ങള്‍ ഒന്നും ഒളിപ്പിച്ചുവെക്കാറുണ്ടായിരുന്നില്ല. നിങ്ങള്‍ പ്രവര്‍ത്തിച്ചുകൊണ്ടിരിക്കുന്നതിലേറെയും അല്ലാഹു അറിയില്ലെന്നാണ് നിങ്ങള്‍ ധരിച്ചിരുന്നത്.
\end{malayalam}}
\flushright{\begin{Arabic}
\quranayah[41][23]
\end{Arabic}}
\flushleft{\begin{malayalam}
അതായിരുന്നു നിങ്ങളുടെ നാഥനെപ്പറ്റി നിങ്ങളുടെ വിചാരം. അതു നിങ്ങളെ നശിപ്പിച്ചിരിക്കുന്നു. അങ്ങനെ നിങ്ങള്‍ എല്ലാം നഷ്ടപ്പെട്ടവരില്‍ പെട്ടുപോയി.
\end{malayalam}}
\flushright{\begin{Arabic}
\quranayah[41][24]
\end{Arabic}}
\flushleft{\begin{malayalam}
ഇനിയിപ്പോള്‍ അവരെത്ര ക്ഷമിച്ചാലും നരകം തന്നെയാണവരുടെ താവളം. അവരെത്ര വിട്ടുവീഴ്ച തേടിയാലും വിട്ടുവീഴ്ച കിട്ടുകയുമില്ല.
\end{malayalam}}
\flushright{\begin{Arabic}
\quranayah[41][25]
\end{Arabic}}
\flushleft{\begin{malayalam}
നാം അവര്‍ക്ക് ചില കൂട്ടുകാരെ ഉണ്ടാക്കിക്കൊടുത്തു. ആ കൂട്ടുകാര്‍ അവരുടെ മുന്നിലുള്ളതും പിന്നിലുള്ളതും അവര്‍ക്ക് അലംകൃതമായി തോന്നിപ്പിച്ചു. അതോടെ അവര്‍ക്ക് ശിക്ഷ സ്ഥിരപ്പെട്ടു. അവര്‍ക്ക് മുമ്പെ കഴിഞ്ഞുപോയ ജിന്നുകളിലും മനുഷ്യരിലുമുള്ളവര്‍ക്ക് ബാധകമായ അതേ ശിക്ഷ. ഉറപ്പായും അവര്‍ നഷ്ടം പറ്റിയവര്‍ തന്നെ.
\end{malayalam}}
\flushright{\begin{Arabic}
\quranayah[41][26]
\end{Arabic}}
\flushleft{\begin{malayalam}
സത്യനിഷേധികള്‍ പറഞ്ഞു: "നിങ്ങള്‍ ഈ ഖുര്‍ആന്‍ കേട്ടുപോകരുത്. അതു കേള്‍ക്കുമ്പോള്‍ നിങ്ങള്‍ ഒച്ചവെക്കുക. അങ്ങനെ നിങ്ങള്‍ക്കതിനെ അതിജയിക്കാം."
\end{malayalam}}
\flushright{\begin{Arabic}
\quranayah[41][27]
\end{Arabic}}
\flushleft{\begin{malayalam}
സത്യനിഷേധികളെ നാം കൊടിയ ശിക്ഷയുടെ രുചി ആസ്വദിപ്പിക്കും. അവര്‍ ചെയ്തുകൊണ്ടിരുന്ന ചീത്തപ്രവര്‍ത്തനങ്ങള്‍ക്കുള്ള പ്രതിഫലം നാം നല്‍കുകയും ചെയ്യും.
\end{malayalam}}
\flushright{\begin{Arabic}
\quranayah[41][28]
\end{Arabic}}
\flushleft{\begin{malayalam}
അതാണ് ദൈവവിരോധികള്‍ക്ക് ലഭിക്കാനിരിക്കുന്ന പ്രതിഫലം; നരകം. അവരുടെ സ്ഥിരവാസത്തിനുള്ള ഭവനവും അവിടെത്തന്നെ. നമ്മുടെ വചനങ്ങളെ തള്ളിപ്പറഞ്ഞുകൊണ്ടിരുന്നതിനുള്ള പ്രതിഫലമാണത്.
\end{malayalam}}
\flushright{\begin{Arabic}
\quranayah[41][29]
\end{Arabic}}
\flushleft{\begin{malayalam}
സത്യനിഷേധികള്‍ പറയും: "ഞങ്ങളുടെ നാഥാ, ജിന്നുകളില്‍ നിന്നും മനുഷ്യരില്‍ നിന്നും ഞങ്ങളെ വഴിപിഴപ്പിച്ചവരെ ഞങ്ങള്‍ക്കു കാണിച്ചുതരേണമേ! ഞങ്ങളവരെ കാല്‍ച്ചുവട്ടിലിട്ട് ചവിട്ടിത്തേക്കട്ടെ. അവര്‍ പറ്റെ നിന്ദ്യരും നീചരുമാകാന്‍."
\end{malayalam}}
\flushright{\begin{Arabic}
\quranayah[41][30]
\end{Arabic}}
\flushleft{\begin{malayalam}
“ഞങ്ങളുടെ നാഥന്‍ അല്ലാഹുവാണെ”ന്ന് പ്രഖ്യാപിക്കുകയും പിന്നെ അതിലടിയുറച്ചു നില്‍ക്കുകയും ചെയ്തവരുടെ അടുത്ത് തീര്‍ച്ചയായും മലക്കുകളിറങ്ങിവന്ന് ഇങ്ങനെ പറയും: "നിങ്ങള്‍ ഭയപ്പെടേണ്ട. ദുഃഖിക്കേണ്ട. നിങ്ങള്‍ക്ക് വാഗ്ദാനം ചെയ്ത സ്വര്‍ഗത്തെ സംബന്ധിച്ച ശുഭവാര്‍ത്തയില്‍ സന്തുഷ്ടരാവുക.
\end{malayalam}}
\flushright{\begin{Arabic}
\quranayah[41][31]
\end{Arabic}}
\flushleft{\begin{malayalam}
"ഈ ലോകത്തും പരലോകത്തും ഞങ്ങള്‍ നിങ്ങളുടെ ഉറ്റമിത്രങ്ങളാകുന്നു. നിങ്ങള്‍ക്ക് അവിടെ നിങ്ങളുടെ മനം മോഹിക്കുന്നതൊക്കെ കിട്ടും. നിങ്ങള്‍ക്ക്അവിടെ നിങ്ങളാവശ്യപ്പെടുന്നതെന്തും ലഭിക്കും.
\end{malayalam}}
\flushright{\begin{Arabic}
\quranayah[41][32]
\end{Arabic}}
\flushleft{\begin{malayalam}
"ഏറെ പൊറുക്കുന്നവനും പരമദയാലുവുമായ ദൈവത്തിങ്കല്‍നിന്നുള്ള സല്‍ക്കാരമാണത്."
\end{malayalam}}
\flushright{\begin{Arabic}
\quranayah[41][33]
\end{Arabic}}
\flushleft{\begin{malayalam}
അല്ലാഹുവിങ്കലേക്ക് ക്ഷണിക്കുകയും സല്‍ക്കര്‍മങ്ങള്‍ പ്രവര്‍ത്തിക്കുകയും “ഞാന്‍ മുസ്ലിംകളില്‍പെട്ടവനാണെ”ന്ന് പ്രഖ്യാപിക്കുകയും ചെയ്തവനേക്കാള്‍ നല്ല വചനം മൊഴിഞ്ഞ ആരുണ്ട്?
\end{malayalam}}
\flushright{\begin{Arabic}
\quranayah[41][34]
\end{Arabic}}
\flushleft{\begin{malayalam}
നന്മയും തിന്മയും തുല്യമാവുകയില്ല. തിന്മയെ ഏറ്റവും നല്ല നന്മകൊണ്ട് തടയുക. അപ്പോള്‍ നിന്നോട് ശത്രുതയില്‍ കഴിയുന്നവന്‍ ആത്മമിത്രത്തെപ്പോലെയായിത്തീരും.
\end{malayalam}}
\flushright{\begin{Arabic}
\quranayah[41][35]
\end{Arabic}}
\flushleft{\begin{malayalam}
ക്ഷമ പാലിക്കുന്നവര്‍ക്കല്ലാതെ ഈ നിലവാരത്തിലെത്താനാവില്ല. മഹാഭാഗ്യവാനല്ലാതെ ഈ പദവി ലഭ്യമല്ല.
\end{malayalam}}
\flushright{\begin{Arabic}
\quranayah[41][36]
\end{Arabic}}
\flushleft{\begin{malayalam}
പിശാചില്‍ നിന്നുള്ള വല്ല ദുഷ്പ്രേരണയും നിന്നെ ബാധിച്ചാല്‍ നീ അല്ലാഹുവില്‍ ശരണംതേടുക. അവന്‍ എല്ലാം കേള്‍ക്കുന്നവനും അറിയുന്നവനുമാണ്.
\end{malayalam}}
\flushright{\begin{Arabic}
\quranayah[41][37]
\end{Arabic}}
\flushleft{\begin{malayalam}
രാപ്പകലുകളും സൂര്യചന്ദ്രന്മാരും അവന്റെ അടയാളങ്ങളില്‍പെട്ടതാണ്. അതിനാല്‍ നിങ്ങള്‍ സൂര്യനെയോ ചന്ദ്രനെയോ പ്രണമിക്കരുത്. അവയെ പടച്ച അല്ലാഹുവിനെ മാത്രം പ്രണമിക്കുക. നിങ്ങള്‍ അവനു മാത്രം വഴിപ്പെടുന്നവരെങ്കില്‍!
\end{malayalam}}
\flushright{\begin{Arabic}
\quranayah[41][38]
\end{Arabic}}
\flushleft{\begin{malayalam}
അഥവാ, അവര്‍ അഹങ്കരിക്കുകയാണെങ്കില്‍ അറിയുക: നിന്റെ നാഥന്റെ സമീപത്തെ മലക്കുകള്‍ രാപ്പകലില്ലാതെ അവനെ കീര്‍ത്തിക്കുന്നു. അവര്‍ക്കതിലൊട്ടും മടുപ്പില്ല.
\end{malayalam}}
\flushright{\begin{Arabic}
\quranayah[41][39]
\end{Arabic}}
\flushleft{\begin{malayalam}
ഭൂമിയെ വരണ്ടതായി നീ കാണുന്നു. പിന്നെ നാം അതില്‍ വെള്ളം വീഴ്ത്തിയാല്‍ പെട്ടെന്നത് ചലനമുള്ളതായിത്തീരുന്നു. വികസിച്ചു വലുതാവുന്നു. ഇതും അവന്റെ ദൃഷ്ടാന്തങ്ങളില്‍ പെട്ടതത്രെ. മൃതമായ ഈ ഭൂമിയെ ജീവനുള്ളതാക്കുന്നവന്‍ തീര്‍ച്ചയായും മരിച്ചവരെ ജീവിപ്പിക്കും. അവന്‍ എല്ലാ കാര്യത്തിനുംകഴിവുറ്റവനാണ്.
\end{malayalam}}
\flushright{\begin{Arabic}
\quranayah[41][40]
\end{Arabic}}
\flushleft{\begin{malayalam}
നമ്മുടെ വചനങ്ങളെ വളച്ചൊടിച്ച് വികൃതമാക്കുന്നവര്‍ നമ്മുടെ കണ്‍വെട്ടത്തുനിന്ന് മറഞ്ഞുനില്‍ക്കുന്നവരല്ല. നരകത്തിലെറിയപ്പെടുന്നവനോ, അതല്ല, ഉയിര്‍ത്തെഴുന്നേല്‍പുനാളില്‍ നിര്‍ഭയനായി വന്നെത്തുന്നവനോ ആരാണ് നല്ലവന്‍? നിങ്ങള്‍ക്കു തോന്നുന്നതെന്തും ചെയ്തുകൊള്ളുക. നിങ്ങള്‍ ചെയ്യുന്നതൊക്കെയും കണ്ടറിയുന്നവനാണ് അല്ലാഹു.
\end{malayalam}}
\flushright{\begin{Arabic}
\quranayah[41][41]
\end{Arabic}}
\flushleft{\begin{malayalam}
ഈ ഉദ്ബോധനം തങ്ങള്‍ക്കു വന്നെത്തിയപ്പോള്‍ അതിനെ തള്ളിപ്പറഞ്ഞവര്‍ നശിച്ചതുതന്നെ. ഇത് അന്തസ്സുറ്റ വേദപുസ്തകമാണ്; തീര്‍ച്ച.
\end{malayalam}}
\flushright{\begin{Arabic}
\quranayah[41][42]
\end{Arabic}}
\flushleft{\begin{malayalam}
ഇതില്‍ അസത്യം വന്നുചേരുകയില്ല. മുന്നിലൂടെയുമില്ല; പിന്നിലൂടെയുമില്ല. യുക്തിമാനും സ്തുത്യര്‍ഹനുമായ അല്ലാഹുവില്‍ നിന്ന് ഇറക്കിക്കിട്ടിയതാണിത്.
\end{malayalam}}
\flushright{\begin{Arabic}
\quranayah[41][43]
\end{Arabic}}
\flushleft{\begin{malayalam}
നിനക്കു മുമ്പുണ്ടായിരുന്ന ദൈവദൂതന്മാരോടു പറയാത്തതൊന്നും നിന്നോടും പറയുന്നില്ല. നിശ്ചയമായും നിന്റെ നാഥന്‍ പാപം പൊറുക്കുന്നവനാണ്; നോവുറ്റ ശിക്ഷ നല്‍കുന്നവനും.
\end{malayalam}}
\flushright{\begin{Arabic}
\quranayah[41][44]
\end{Arabic}}
\flushleft{\begin{malayalam}
നാം ഇതിനെ അറബിയല്ലാത്ത മറ്റേതെങ്കിലും ഭാഷയിലെ ഖുര്‍ആന്‍ ആക്കിയിരുന്നുവെങ്കില്‍ അവര്‍ പറയുമായിരുന്നു: "എന്തുകൊണ്ട് ഇതിലെ വചനങ്ങള്‍ വ്യക്തമായി വിശദമാക്കപ്പെടുന്നില്ല? ഗ്രന്ഥം അനറബിയും പ്രവാചകന്‍ അറബിയുമാവുകയോ?" പറയുക: സത്യവിശ്വാസികള്‍ക്ക് ഇത് വ്യക്തമായ വഴികാട്ടിയാണ്. ഫലവത്തായ ശമനൌഷധവും. വിശ്വസിക്കാത്തവര്‍ക്കോ, അവരുടെ കാതുകളുടെ കേള്‍വി കെടുത്തിക്കളയുന്നതാണ്. കണ്ണുകളുടെ കാഴ്ച നശിപ്പിക്കുന്നതും. ഏതോ വിദൂരതയില്‍ നിന്നു വിളിക്കുന്നതുപോലെ അവ്യക്തമായ വിളിയായാണ് അവര്‍ക്കനുഭവപ്പെടുക.
\end{malayalam}}
\flushright{\begin{Arabic}
\quranayah[41][45]
\end{Arabic}}
\flushleft{\begin{malayalam}
മൂസാക്കും നാം വേദം നല്‍കിയിരുന്നു. അപ്പോള്‍ അതിന്റെ കാര്യത്തിലും ഭിന്നിപ്പുകളുണ്ടായിരുന്നു. നിന്റെ നാഥന്റെ കല്‍പന നേരത്തെ ഉണ്ടായിട്ടില്ലായിരുന്നുവെങ്കില്‍ അവര്‍ക്കിടയില്‍ ഇപ്പോള്‍ തന്നെ തീര്‍പ്പ് കല്‍പിക്കപ്പെടുമായിരുന്നു. സംശയമില്ല; അവരിതേപ്പറ്റി സങ്കീര്‍ണമായ സംശയത്തിലാണ്.
\end{malayalam}}
\flushright{\begin{Arabic}
\quranayah[41][46]
\end{Arabic}}
\flushleft{\begin{malayalam}
ആരെങ്കിലും നന്മ ചെയ്താല്‍ അതിന്റെ ഗുണം അവനുതന്നെയാണ്. വല്ലവനും തിന്മ ചെയ്താല്‍ അതിന്റെ ദോഷവും അവനുതന്നെ. നിന്റെ നാഥന്‍ തന്റെ ദാസന്മാരോടു തീരേ അനീതി ചെയ്യുന്നവനല്ല.
\end{malayalam}}
\flushright{\begin{Arabic}
\quranayah[41][47]
\end{Arabic}}
\flushleft{\begin{malayalam}
ആ അന്ത്യസമയം സംബന്ധിച്ച അറിവ് അല്ലാഹുവിനു മാത്രമേയുള്ളൂ. അവന്റെ അറിവോടെയല്ലാതെ പഴങ്ങള്‍ അവയുടെ പോളകളില്‍ നിന്നു പുറത്തുവരികയോ ഒരു സ്ത്രീയും ഗര്‍ഭം ചുമക്കുകയോ പ്രസവിക്കുകയോ ഇല്ല. അവന്‍ അവരോടിങ്ങനെ വിളിച്ചു ചോദിക്കുന്ന ദിവസം: "എന്റെ പങ്കാളികളെവിടെ?" അവര്‍ പറയും: "ഞങ്ങളിതാ നിന്നെ അറിയിക്കുന്നു. ഞങ്ങളിലാരും തന്നെ അതിനു സാക്ഷികളല്ല."
\end{malayalam}}
\flushright{\begin{Arabic}
\quranayah[41][48]
\end{Arabic}}
\flushleft{\begin{malayalam}
അവര്‍ നേരത്തെ വിളിച്ചുപ്രാര്‍ഥിച്ചിരുന്നവയെല്ലാം അവരില്‍നിന്ന് വിട്ടകന്നുപോകും. തങ്ങള്‍ക്കിനിയൊരു രക്ഷാമാര്‍ഗവുമില്ലെന്ന് അവര്‍ക്ക് ബോധ്യമാവുകയും ചെയ്യും.
\end{malayalam}}
\flushright{\begin{Arabic}
\quranayah[41][49]
\end{Arabic}}
\flushleft{\begin{malayalam}
നന്മ തേടുന്നതില്‍ മനുഷ്യനൊട്ടും മടുപ്പനുഭവപ്പെടുന്നില്ല. എന്നാല്‍ വല്ല വിപത്തും അവനെ ബാധിച്ചാലോ അവന്‍ മനംമടുത്തവനും കടുത്തനിരാശനുമായിത്തീരുന്നു.
\end{malayalam}}
\flushright{\begin{Arabic}
\quranayah[41][50]
\end{Arabic}}
\flushleft{\begin{malayalam}
അവനെ ബാധിച്ച വിപത്ത് വിട്ടൊഴിഞ്ഞശേഷം നമ്മുടെ അനുഗ്രഹം നാമവനെ ആസ്വദിപ്പിച്ചാല്‍ തീര്‍ച്ചയായും അവന്‍ പറയും: "ഇത് എനിക്ക് അവകാശപ്പെട്ടതുതന്നെയാണ്. അന്ത്യസമയം ആസന്നമാകുമെന്ന് ഞാന്‍ കരുതുന്നില്ല. അഥവാ, ഞാനെന്റെ നാഥന്റെ അടുത്തേക്ക് തിരിച്ചയക്കപ്പെട്ടാലും എനിക്ക് അവന്റെയടുത്ത് നല്ല അവസ്ഥയാണുണ്ടാവുക." എന്നാല്‍ ഇത്തരം സത്യനിഷേധികള്‍ക്ക് അവര്‍ പ്രവര്‍ത്തിച്ചുകൊണ്ടിരുന്നതിനെപ്പറ്റി നാം വിവരമറിയിക്കും. കഠിനമായ ശിക്ഷ അവരെ ആസ്വദിപ്പിക്കും.
\end{malayalam}}
\flushright{\begin{Arabic}
\quranayah[41][51]
\end{Arabic}}
\flushleft{\begin{malayalam}
മനുഷ്യന് നാം വല്ല ഔദാര്യവും ചെയ്യുമ്പോള്‍ അവനത് അവഗണിക്കുന്നു. അഹന്ത നടിക്കുന്നു. വല്ല വിപത്തും അവനെ ബാധിച്ചാലോ, അവനതാ ദീര്‍ഘമായ പ്രാര്‍ഥനയിലേര്‍പ്പെടുന്നു.
\end{malayalam}}
\flushright{\begin{Arabic}
\quranayah[41][52]
\end{Arabic}}
\flushleft{\begin{malayalam}
ചോദിക്കുക: ഈ ഖുര്‍ആന്‍ അല്ലാഹുവില്‍ നിന്നുള്ളതുതന്നെയായിരിക്കുകയും എന്നിട്ട് നിങ്ങളതിനെ തള്ളിപ്പറയുകയും അങ്ങനെ ഇതിനോടുള്ള എതിര്‍പ്പില്‍ ഏറെ ദൂരം പിന്നിട്ടവനായിത്തീരുകയുമാണെങ്കില്‍ അവനെക്കാള്‍ പിഴച്ചവനായി ആരാണുണ്ടാവുകയെന്ന് നിങ്ങള്‍ ചിന്തിച്ചുനോക്കിയിട്ടുണ്ടോ?
\end{malayalam}}
\flushright{\begin{Arabic}
\quranayah[41][53]
\end{Arabic}}
\flushleft{\begin{malayalam}
അടുത്തുതന്നെ വിവിധ ദിക്കുകളിലും അവരില്‍ തന്നെയും നമ്മുടെ ദൃഷ്ടാന്തങ്ങള്‍ നാമവര്‍ക്കു കാണിച്ചുകൊടുക്കും. ഈ ഖുര്‍ആന്‍ സത്യമാണെന്ന് അവര്‍ക്ക് വ്യക്തമാകുംവിധമായിരിക്കുമത്. നിന്റെ നാഥന്‍ സകല സംഗതികള്‍ക്കും സാക്ഷിയാണെന്ന കാര്യം തന്നെ പോരേ അവരതില്‍ വിശ്വാസമുള്ളവരാകാന്‍?
\end{malayalam}}
\flushright{\begin{Arabic}
\quranayah[41][54]
\end{Arabic}}
\flushleft{\begin{malayalam}
അറിയുക: തീര്‍ച്ചയായും ഈ ജനം തങ്ങളുടെ നാഥനെ കണ്ടുമുട്ടുമെന്ന കാര്യത്തില്‍ സംശയത്തിലാണ്. ഓര്‍ക്കുക: അവന്‍ സകല സംഗതികളെയും വലയം ചെയ്യുന്നവനാണ്.
\end{malayalam}}
\chapter{\textmalayalam{ശൂറാ ( കൂടിയാലോചന )}}
\begin{Arabic}
\Huge{\centerline{\basmalah}}\end{Arabic}
\flushright{\begin{Arabic}
\quranayah[42][1]
\end{Arabic}}
\flushleft{\begin{malayalam}
ഹാ - മീം.
\end{malayalam}}
\flushright{\begin{Arabic}
\quranayah[42][2]
\end{Arabic}}
\flushleft{\begin{malayalam}
ഐന്‍- സീന്‍- ഖാഫ്.
\end{malayalam}}
\flushright{\begin{Arabic}
\quranayah[42][3]
\end{Arabic}}
\flushleft{\begin{malayalam}
പ്രതാപിയും യുക്തിമാനുമായ അല്ലാഹു നിനക്കും നിനക്കുമുമ്പുള്ളവര്‍ക്കും ഇവ്വിധം ദിവ്യബോധനം നല്‍കുന്നു.
\end{malayalam}}
\flushright{\begin{Arabic}
\quranayah[42][4]
\end{Arabic}}
\flushleft{\begin{malayalam}
ആകാശഭൂമികളിലുള്ളതെല്ലാം അവന്റേതാണ്. അവന്‍ അത്യുന്നതനും മഹാനുമാണ്.
\end{malayalam}}
\flushright{\begin{Arabic}
\quranayah[42][5]
\end{Arabic}}
\flushleft{\begin{malayalam}
ആകാശങ്ങള്‍ അവയുടെ മുകള്‍ഭാഗത്തുനിന്ന് പൊട്ടിച്ചിതറാനടുത്തിരിക്കുന്നു. മലക്കുകള്‍ തങ്ങളുടെ നാഥനെ കീര്‍ത്തിക്കുന്നു. വാഴ്ത്തുന്നു. ഭൂമിയിലുള്ളവര്‍ക്കായി അവര്‍ പാപമോചനം തേടുന്നു. അറിയുക; തീര്‍ച്ചയായും അല്ലാഹു ഏറെ പൊറുക്കുന്നവനാണ്; പരമദയാലുവും.
\end{malayalam}}
\flushright{\begin{Arabic}
\quranayah[42][6]
\end{Arabic}}
\flushleft{\begin{malayalam}
അല്ലാഹുവെക്കൂടാതെ മറ്റു രക്ഷാധികാരികളെ സ്വീകരിച്ചവരുണ്ടല്ലോ. അല്ലാഹു അവരെ സൂക്ഷ്മമായി നിരീക്ഷിക്കുന്നവനാണ്. നിനക്ക് അവരുടെ മേല്‍നോട്ടബാധ്യതയില്ല.
\end{malayalam}}
\flushright{\begin{Arabic}
\quranayah[42][7]
\end{Arabic}}
\flushleft{\begin{malayalam}
ഇവ്വിധം നിനക്കു നാം അറബിഭാഷയിലുള്ള ഖുര്‍ആന്‍ ബോധനം നല്‍കിയിരിക്കുന്നു. നീ മാതൃനഗരത്തിലുള്ളവര്‍ക്കും അതിനു ചുറ്റുമുള്ളവര്‍ക്കും മുന്നറിയിപ്പു നല്‍കാനാണിത്. സംഭവിക്കുമെന്ന കാര്യത്തില്‍ സംശയസാധ്യതപോലുമില്ലാത്ത ആ മഹാസംഗമത്തെ സംബന്ധിച്ച് മുന്നറിയിപ്പു നല്‍കാനും. അന്നൊരു സംഘം സ്വര്‍ഗത്തിലായിരിക്കും. മറ്റൊരു സംഘം കത്തിക്കാളുന്ന നരകത്തീയിലും.
\end{malayalam}}
\flushright{\begin{Arabic}
\quranayah[42][8]
\end{Arabic}}
\flushleft{\begin{malayalam}
അല്ലാഹു ഉദ്ദേശിച്ചിരുന്നെങ്കില്‍ മനുഷ്യരെ മുഴുവന്‍ അവന്‍ ഒരൊറ്റ സമുദായമാക്കുമായിരുന്നു. എന്നാല്‍ അവനിച്ഛിക്കുന്നവരെ അവന്‍ തന്റെ അനുഗ്രഹത്തിന് അവകാശിയാക്കുന്നു. അക്രമികള്‍ക്ക് രക്ഷകനോ സഹായിയോ ഇല്ല.
\end{malayalam}}
\flushright{\begin{Arabic}
\quranayah[42][9]
\end{Arabic}}
\flushleft{\begin{malayalam}
ഇക്കൂട്ടര്‍ അവനെക്കൂടാതെ മറ്റു രക്ഷകരെ സ്വീകരിച്ചിരിക്കയാണോ? എന്നാല്‍ അറിയുക; യഥാര്‍ഥ രക്ഷകന്‍ അല്ലാഹുവാണ്. അവന്‍ മരിച്ചവരെ ജീവിപ്പിക്കുന്നു. അവന്‍ എല്ലാ കാര്യങ്ങള്‍ക്കും കഴിവുറ്റവനാണ്.
\end{malayalam}}
\flushright{\begin{Arabic}
\quranayah[42][10]
\end{Arabic}}
\flushleft{\begin{malayalam}
നിങ്ങള്‍ക്കിടയില്‍ ഭിന്നതയുള്ളത് ഏതു കാര്യത്തിലായാലും അതില്‍ വിധിത്തീര്‍പ്പുണ്ടാക്കേണ്ടത് അല്ലാഹുവാണ്. അവന്‍ മാത്രമാണ് എന്റെ നാഥനായ അല്ലാഹു. ഞാന്‍ അവനില്‍ ഭരമേല്‍പിച്ചിരിക്കുന്നു. ഞാന്‍ ഖേദിച്ചു മടങ്ങുന്നതും അവങ്കലേക്കുതന്നെ.
\end{malayalam}}
\flushright{\begin{Arabic}
\quranayah[42][11]
\end{Arabic}}
\flushleft{\begin{malayalam}
ആകാശഭൂമികളുടെ സ്രഷ്ടാവാണവന്‍. അവന്‍ നിങ്ങള്‍ക്ക് നിങ്ങളില്‍ നിന്നു തന്നെ ഇണകളെ സൃഷ്ടിച്ചു തന്നിരിക്കുന്നു. നാല്‍ക്കാലികളിലും ഇണകളെ ഉണ്ടാക്കിയിരിക്കുന്നു. അതിലൂടെ അവന്‍ നിങ്ങളെ സൃഷ്ടിച്ച് വംശം വികസിപ്പിക്കുന്നു. അല്ലാഹുവിനു തുല്യമായി ഒന്നുമില്ല. അവന്‍ എല്ലാം കേള്‍ക്കുന്നവനാണ്. കാണുന്നവനും.
\end{malayalam}}
\flushright{\begin{Arabic}
\quranayah[42][12]
\end{Arabic}}
\flushleft{\begin{malayalam}
ആകാശഭൂമികളുടെ താക്കോലുകള്‍ അവന്റെ അധീനതയിലാണ്. അവനുദ്ദേശിക്കുന്നവര്‍ക്ക് അളവറ്റ വിഭവങ്ങള്‍ നല്‍കുന്നു. അവനിച്ഛിക്കുന്നവര്‍ക്ക് അതില്‍ കുറവ് വരുത്തുന്നു. അവന്‍ സകല സംഗതികളും നന്നായറിയുന്നവനാണ്.
\end{malayalam}}
\flushright{\begin{Arabic}
\quranayah[42][13]
\end{Arabic}}
\flushleft{\begin{malayalam}
നൂഹിനോടു കല്‍പിച്ചതും നിനക്കു നാം ദിവ്യബോധനമായി നല്‍കിയതും ഇബ്റാഹീം, മൂസാ, ഈസാ എന്നിവരോടനുശാസിച്ചതുമായ കാര്യം തന്നെ അവന്‍ നിങ്ങള്‍ക്കു മതനിയമമായി നിശ്ചയിച്ചു തന്നിരിക്കുന്നു. “നിങ്ങള്‍ ഈ ജീവിതവ്യവസ്ഥ സ്ഥാപിക്കുക; അതില്‍ ഭിന്നിക്കാതിരിക്കുക”യെന്നതാണത്. നിങ്ങള്‍ പ്രബോധനം ചെയ്തുകൊണ്ടിരിക്കുന്ന ഈ സന്ദേശം ബഹുദൈവവിശ്വാസികള്‍ക്ക് വളരെ വലിയ ഭാരമായിത്തോന്നുന്നു. അല്ലാഹു താനിച്ഛിക്കുന്നവരെ തനിക്കുവേണ്ടി പ്രത്യേകം തെരഞ്ഞെടുക്കുന്നു. പശ്ചാത്തപിച്ചു തന്നിലേക്കു മടങ്ങുന്നവരെ, അല്ലാഹു നേര്‍വഴിയില്‍ നയിക്കുന്നു.
\end{malayalam}}
\flushright{\begin{Arabic}
\quranayah[42][14]
\end{Arabic}}
\flushleft{\begin{malayalam}
ശരിയായ അറിവു വന്നെത്തിയശേഷമല്ലാതെ ജനം ഭിന്നിച്ചിട്ടില്ല. ആ ഭിന്നതയോ അവര്‍ക്കിടയിലുണ്ടായിരുന്ന വിരോധം മൂലമാണ്. ഒരു നിശ്ചിത അവധിവരെ അന്ത്യവിധി സംഭവിക്കില്ലെന്ന നിന്റെ നാഥന്റെ തീരുമാനം ഉണ്ടായിരുന്നില്ലെങ്കില്‍ അവര്‍ക്കിടയില്‍ ഇപ്പോള്‍ തന്നെ വിധിത്തീര്‍പ്പ് കല്‍പിക്കുമായിരുന്നു. അവര്‍ക്കുശേഷം വേദപുസ്തകത്തിന് അവകാശികളായിത്തീര്‍ന്നവര്‍ തീര്‍ച്ചയായും അതേക്കുറിച്ച് സങ്കീര്‍ണമായ സംശയത്തിലാണ്.
\end{malayalam}}
\flushright{\begin{Arabic}
\quranayah[42][15]
\end{Arabic}}
\flushleft{\begin{malayalam}
അതിനാല്‍ നീ സത്യപ്രബോധനം നടത്തുക. കല്‍പിക്കപ്പെട്ടപോലെ നേരാംവിധം നിലകൊള്ളുക. അവരുടെ തന്നിഷ്ടങ്ങളെ പിന്‍പറ്റരുത്. പറയുക: "അല്ലാഹു ഇറക്കിത്തന്ന എല്ലാ വേദപുസ്തകത്തിലും ഞാന്‍ വിശ്വസിക്കുന്നു. നിങ്ങള്‍ക്കിടയില്‍ നീതി സ്ഥാപിക്കാന്‍ ഞാന്‍ കല്‍പിക്കപ്പെട്ടിരിക്കുന്നു. അല്ലാഹുവാണ് ഞങ്ങളുടെയും നിങ്ങളുടെയും നാഥന്‍. ഞങ്ങള്‍ക്ക് ഞങ്ങളുടെ കര്‍മങ്ങള്‍. നിങ്ങള്‍ക്ക് നിങ്ങളുടെ കര്‍മങ്ങളും. നമുക്കിടയില്‍ തര്‍ക്കമൊന്നുമില്ല. ഒരു നാള്‍ അല്ലാഹു നമ്മെയെല്ലാം ഒരുമിച്ചുകൂട്ടും. എല്ലാവര്‍ക്കും മടങ്ങിച്ചെല്ലാനുള്ളത് അവങ്കലേക്കുതന്നെയാണല്ലോ."
\end{malayalam}}
\flushright{\begin{Arabic}
\quranayah[42][16]
\end{Arabic}}
\flushleft{\begin{malayalam}
അല്ലാഹുവിന്റെ ക്ഷണം സ്വീകരിച്ചശേഷം അത് സ്വീകരിച്ചവരോട് അല്ലാഹുവെക്കുറിച്ച് തര്‍ക്കിക്കുന്നവരുടെ വാദം അവരുടെ നാഥന്റെയടുത്ത് തീര്‍ത്തും നിരര്‍ഥകമാണ്. അവര്‍ക്ക് ദൈവകോപമുണ്ട്. കഠിനമായ ശിക്ഷയും.
\end{malayalam}}
\flushright{\begin{Arabic}
\quranayah[42][17]
\end{Arabic}}
\flushleft{\begin{malayalam}
സത്യസന്ദേശവുമായി വേദപുസ്തകവും തുലാസുമിറക്കിത്തന്നത് അല്ലാഹുവാണ്. നിനക്കെന്തറിയാം. ആ അന്ത്യസമയം അടുത്തുതന്നെ വന്നെത്തിയേക്കാം.
\end{malayalam}}
\flushright{\begin{Arabic}
\quranayah[42][18]
\end{Arabic}}
\flushleft{\begin{malayalam}
ആ അന്ത്യദിനത്തില്‍ വിശ്വസിക്കാത്തവരാണ് അതിനായി ധൃതി കൂട്ടുന്നത്. വിശ്വസിക്കുന്നവര്‍ അതേക്കുറിച്ച് ഭയപ്പെടുന്നവരാണ്. അവര്‍ക്കറിയാം അത് സംഭവിക്കാന്‍പോകുന്ന സത്യമാണെന്ന്. അറിയുക: അന്ത്യസമയത്തെ സംബന്ധിച്ച് തര്‍ക്കിക്കുന്നവര്‍ തീര്‍ച്ചയായും വഴികേടില്‍ ഏറെ ദൂരം പിന്നിട്ടിരിക്കുന്നു.
\end{malayalam}}
\flushright{\begin{Arabic}
\quranayah[42][19]
\end{Arabic}}
\flushleft{\begin{malayalam}
അല്ലാഹു തന്റെ ദാസന്മാരോട് ദയാമയനാണ്. അവനിച്ഛിക്കുന്നവര്‍ക്ക് അവന്‍ അന്നം നല്‍കുന്നു. അവന്‍ കരുത്തനാണ്; പ്രതാപിയും.
\end{malayalam}}
\flushright{\begin{Arabic}
\quranayah[42][20]
\end{Arabic}}
\flushleft{\begin{malayalam}
വല്ലവനും പരലോകത്തെ വിളവാണ് ആഗ്രഹിക്കുന്നതെങ്കില്‍ നാമവനത് സമൃദ്ധമായി നല്‍കും. ആരെങ്കിലും ഇഹലോക വിളവാണ് ആഗ്രഹിക്കുന്നതെങ്കില്‍ അവന് നാമതും നല്‍കും. അപ്പോഴവന് പരലോക വിഭവങ്ങളൊന്നുമുണ്ടാവുകയില്ല.
\end{malayalam}}
\flushright{\begin{Arabic}
\quranayah[42][21]
\end{Arabic}}
\flushleft{\begin{malayalam}
ഈ ജനത്തിന്, അല്ലാഹു അനുവദിച്ചിട്ടില്ലാത്ത കാര്യം മതനിയമമായി നിശ്ചയിച്ചുകൊടുത്ത വല്ല പങ്കാളികളുമുണ്ടോ? വിധി ത്തീര്‍പ്പിനെ സംബന്ധിച്ച കല്‍പന നേരത്തെ വന്നിട്ടില്ലായിരുന്നുവെങ്കില്‍ അവര്‍ക്കിടയില്‍ പെട്ടെന്നു തന്നെ വിധിത്തീര്‍പ്പുണ്ടാകുമായിരുന്നു. സംശയമില്ല; അക്രമികള്‍ക്ക് നോവേറിയ ശിക്ഷയുണ്ട്.
\end{malayalam}}
\flushright{\begin{Arabic}
\quranayah[42][22]
\end{Arabic}}
\flushleft{\begin{malayalam}
ആ അക്രമികള്‍ തങ്ങള്‍ നേടിവെച്ചതിനെക്കുറിച്ചോര്‍ത്ത് പേടിച്ചു വിറക്കുന്നത് നിനക്കു കാണാം. അവരിലത് വന്നെത്തുക തന്നെ ചെയ്യും. എന്നാല്‍ സത്യവിശ്വാസം സ്വീകരിക്കുകയും സല്‍ക്കര്‍മങ്ങള്‍ പ്രവര്‍ത്തിക്കുകയും ചെയ്തവര്‍ ഉറപ്പായും സ്വര്‍ഗീയാരാമങ്ങളിലായിരിക്കും. അവര്‍ക്ക് തങ്ങളുടെ നാഥന്റെയടുത്ത് അവരാഗ്രഹിക്കുന്നതൊക്കെ കിട്ടും. അതു തന്നെയാണ് അതിമഹത്തായ അനുഗ്രഹം.
\end{malayalam}}
\flushright{\begin{Arabic}
\quranayah[42][23]
\end{Arabic}}
\flushleft{\begin{malayalam}
സത്യവിശ്വാസം സ്വീകരിക്കുകയും സല്‍ക്കര്‍മങ്ങള്‍ പ്രവര്‍ത്തിക്കുകയും ചെയ്ത തന്റെ ദാസന്മാര്‍ക്ക് അല്ലാഹു നല്‍കുന്ന ശുഭവാര്‍ത്തയാണിത്. പറയുക: "ഇതിന്റെ പേരില്‍ ഞാന്‍ നിങ്ങളോടൊരു പ്രതിഫലവും ആവശ്യപ്പെടുന്നില്ല. അടുത്തബന്ധത്തിന്റെ പേരിലുള്ള ആത്മാര്‍ഥമായ സ്നേഹമല്ലാതെ." ആരെങ്കിലും വല്ല നന്മയും നേടുകയാണെങ്കില്‍ നാമതില്‍ വര്‍ധനവ് വരുത്തും. തീര്‍ച്ചയായും അല്ലാഹു ഏറെ പൊറുക്കുന്നവനും ദയാപരനുമാണ്.
\end{malayalam}}
\flushright{\begin{Arabic}
\quranayah[42][24]
\end{Arabic}}
\flushleft{\begin{malayalam}
അല്ല; ഈ പ്രവാചകന്‍ അല്ലാഹുവിന്റെ പേരില്‍ കള്ളം കെട്ടിച്ചമച്ചുവെന്നാണോ ഇക്കൂട്ടര്‍ പറയുന്നത്? എന്നാല്‍ അറിയുക; അല്ലാഹു ഇച്ഛിച്ചിരുന്നെങ്കില്‍ നിന്റെ മനസ്സിനും അവന്‍ മുദ്രവെക്കുമായിരുന്നു. അല്ലാഹു അസത്യത്തെ തുടച്ചുനീക്കുന്നു. സത്യത്തെ തന്റെ വചനങ്ങളിലൂടെ സ്ഥാപിക്കുന്നു. സംശയമില്ല; അവന്‍ മനസ്സിനുള്ളിലുള്ളതെല്ലാം നന്നായറിയുന്നവനാണ്.
\end{malayalam}}
\flushright{\begin{Arabic}
\quranayah[42][25]
\end{Arabic}}
\flushleft{\begin{malayalam}
അവനാണ് തന്റെ ദാസന്മാരുടെ പശ്ചാത്താപം സ്വീകരിക്കുന്നവന്‍. പാപകൃത്യങ്ങള്‍ പൊറുത്തുകൊടുക്കുന്നവനും അവന്‍ തന്നെ. നിങ്ങള്‍ ചെയ്യുന്നതെല്ലാം നന്നായറിയുന്നവനാണവന്‍.
\end{malayalam}}
\flushright{\begin{Arabic}
\quranayah[42][26]
\end{Arabic}}
\flushleft{\begin{malayalam}
സത്യവിശ്വാസം സ്വീകരിക്കുകയും സല്‍ക്കര്‍മങ്ങള്‍ പ്രവര്‍ത്തിക്കുകയും ചെയ്തവരുടെ പ്രാര്‍ഥനകള്‍ക്ക് അവനുത്തരം നല്‍കുന്നു. അവര്‍ക്ക് തന്റെ അനുഗ്രഹങ്ങള്‍ വര്‍ധിപ്പിച്ചുകൊടുക്കുന്നു. സത്യനിഷേധികളോ, അവര്‍ക്ക് കൊടിയ ശിക്ഷയാണുണ്ടാവുക.
\end{malayalam}}
\flushright{\begin{Arabic}
\quranayah[42][27]
\end{Arabic}}
\flushleft{\begin{malayalam}
അല്ലാഹു തന്റെ ദാസന്മാര്‍ക്കെല്ലാം വിഭവം സുലഭമായി നല്‍കിയിരുന്നുവെങ്കില്‍ അവര്‍ ഭൂമിയില്‍ അതിക്രമം കാണിക്കുമായിരുന്നു. എന്നാല്‍ അവന്‍ താനിച്ഛിക്കുന്നവര്‍ക്ക് നിശ്ചിത തോതനുസരിച്ച് അതിറക്കിക്കൊടുക്കുന്നു. സംശയമില്ല; അവന്‍ തന്റെ ദാസന്മാരെ സംബന്ധിച്ച് സൂക്ഷ്മമായി അറിയുന്നവനാണ്. സ്പഷ്ടമായി കാണുന്നവനും.
\end{malayalam}}
\flushright{\begin{Arabic}
\quranayah[42][28]
\end{Arabic}}
\flushleft{\begin{malayalam}
ജനം നന്നെ നിരാശരായിക്കഴിഞ്ഞാല്‍ അവര്‍ക്കു മഴ വീഴ്ത്തിക്കൊടുക്കുന്നത് അവനാണ്. തന്റെ അനുഗ്രഹം വിപുലമാക്കുന്നവനുമാണവന്‍. രക്ഷകനും സ്തുത്യര്‍ഹനും അവന്‍ തന്നെ.
\end{malayalam}}
\flushright{\begin{Arabic}
\quranayah[42][29]
\end{Arabic}}
\flushleft{\begin{malayalam}
ആകാശഭൂമികളെ സൃഷ്ടിച്ചതും അവ രണ്ടിലും ജീവജാലങ്ങളെ വ്യാപിപ്പിച്ചതും അല്ലാഹുവിന്റെ ദൃഷ്ടാന്തങ്ങളില്‍പെട്ടവയാണ്. അവനിച്ഛിക്കുമ്പോള്‍ അവരെയൊക്കെ ഒരുമിച്ചുകൂട്ടാന്‍ കഴിവുറ്റവനാണവന്‍.
\end{malayalam}}
\flushright{\begin{Arabic}
\quranayah[42][30]
\end{Arabic}}
\flushleft{\begin{malayalam}
നിങ്ങള്‍ക്കു വന്നുപെട്ട വിപത്തുകളൊക്കെയും നിങ്ങളുടെ കൈകള്‍ ചെയ്തുകൂട്ടിയ പാപങ്ങളുടെ ഫലം തന്നെയാണ്. പല പാപങ്ങളുമവന്‍ പൊറുത്തുതരുന്നുമുണ്ട്.
\end{malayalam}}
\flushright{\begin{Arabic}
\quranayah[42][31]
\end{Arabic}}
\flushleft{\begin{malayalam}
ഈ ഭൂമിയില്‍ വെച്ച് നിങ്ങള്‍ക്ക് അല്ലാഹുവെ തോല്‍പിക്കാനാവില്ല. അല്ലാഹുവെക്കൂടാതെ നിങ്ങള്‍ക്കൊരു രക്ഷകനോ സഹായിയോ ഇല്ല.
\end{malayalam}}
\flushright{\begin{Arabic}
\quranayah[42][32]
\end{Arabic}}
\flushleft{\begin{malayalam}
കടലില്‍ മലകള്‍പോലെ കാണുന്ന കപ്പലുകള്‍ അവന്റെ ദൃഷ്ടാന്തങ്ങളില്‍പെട്ടവയാണ്.
\end{malayalam}}
\flushright{\begin{Arabic}
\quranayah[42][33]
\end{Arabic}}
\flushleft{\begin{malayalam}
അവനിച്ഛിക്കുമ്പോള്‍ അവന്‍ കാറ്റിനെ ഒതുക്കിനിര്‍ത്തുന്നു. അപ്പോള്‍ ആ കപ്പലുകള്‍ കടല്‍പ്പരപ്പില്‍ അനക്കമറ്റു നിന്നുപോകുന്നു. നന്നായി ക്ഷമിക്കുന്നവര്‍ക്കും നന്ദി കാണിക്കുന്നവര്‍ക്കും നിശ്ചയമായും അതില്‍ ധാരാളം ദൃഷ്ടാന്തങ്ങളുണ്ട്.
\end{malayalam}}
\flushright{\begin{Arabic}
\quranayah[42][34]
\end{Arabic}}
\flushleft{\begin{malayalam}
അല്ലെങ്കില്‍ അതിലെ യാത്രക്കാര്‍ പ്രവര്‍ത്തിച്ച പാപങ്ങളുടെ പേരില്‍ അവനവയെ നശിപ്പിച്ചേക്കാം. എന്നാല്‍ ഏറെയും അവന്‍ മാപ്പാക്കുന്നു.
\end{malayalam}}
\flushright{\begin{Arabic}
\quranayah[42][35]
\end{Arabic}}
\flushleft{\begin{malayalam}
നമ്മുടെ ദൃഷ്ടാന്തങ്ങളെപ്പറ്റി തര്‍ക്കിക്കുന്നവര്‍ക്ക് അപ്പോള്‍ ബോധ്യമാകും; തങ്ങള്‍ക്കൊരു രക്ഷാകേന്ദ്രവുമില്ലെന്ന്.
\end{malayalam}}
\flushright{\begin{Arabic}
\quranayah[42][36]
\end{Arabic}}
\flushleft{\begin{malayalam}
നിങ്ങള്‍ക്കു നല്‍കിയതെന്തും ഐഹികജീവിതത്തിലെ താല്‍ക്കാലികവിഭവം മാത്രമാണ്. അല്ലാഹുവിന്റെ അടുത്തുളളതാണ് കൂടുതലുത്തമം. എന്നെന്നും നിലനില്‍ക്കുന്നതും അതുതന്നെ. അത് സത്യവിശ്വാസം സ്വീകരിക്കുകയും തങ്ങളുടെ നാഥനില്‍ ഭരമേല്‍പിക്കുകയും ചെയ്യുന്നവര്‍ക്കുള്ളതാണ്.
\end{malayalam}}
\flushright{\begin{Arabic}
\quranayah[42][37]
\end{Arabic}}
\flushleft{\begin{malayalam}
വന്‍പാപങ്ങളില്‍ നിന്നും നീചകൃത്യങ്ങളില്‍ നിന്നും വിട്ടുനില്‍ക്കുന്നവരാണവര്‍. കോപം വരുമ്പോള്‍ മാപ്പേകുന്നവരും.
\end{malayalam}}
\flushright{\begin{Arabic}
\quranayah[42][38]
\end{Arabic}}
\flushleft{\begin{malayalam}
തങ്ങളുടെ നാഥന്റെ വിളിക്കുത്തരം നല്‍കുകയും നമസ്കാരം നിഷ്ഠയോടെ നിര്‍വഹിക്കുകയും തങ്ങളുടെ കാര്യങ്ങള്‍ പരസ്പരം കൂടിയാലോചിച്ച് തീരുമാനിക്കുകയും നാം നല്‍കിയതില്‍ നിന്ന് ചെലവഴിക്കുകയും ചെയ്യുന്നവരുമാണ്.
\end{malayalam}}
\flushright{\begin{Arabic}
\quranayah[42][39]
\end{Arabic}}
\flushleft{\begin{malayalam}
തങ്ങള്‍ അതിക്രമങ്ങള്‍ക്കിരയായാല്‍ രക്ഷാനടപടി സ്വീകരിക്കുന്നവരും.
\end{malayalam}}
\flushright{\begin{Arabic}
\quranayah[42][40]
\end{Arabic}}
\flushleft{\begin{malayalam}
തിന്മക്കുള്ള പ്രതിഫലം തത്തുല്യമായ തിന്മ തന്നെ. എന്നാല്‍ ആരെങ്കിലും മാപ്പേകുകയും യോജിപ്പുണ്ടാക്കുകയുമാണെങ്കില്‍ അവന് പ്രതിഫലം നല്‍കുക അല്ലാഹുവിന്റെ ബാധ്യതയത്രേ. അവന്‍ അക്രമികളെ ഒട്ടും ഇഷ്ടപ്പെടുന്നില്ല.
\end{malayalam}}
\flushright{\begin{Arabic}
\quranayah[42][41]
\end{Arabic}}
\flushleft{\begin{malayalam}
അക്രമത്തിനിരയായവര്‍ ആത്മരക്ഷാപ്രവര്‍ത്തനം നടത്തുന്നുവെങ്കില്‍ അങ്ങനെ ചെയ്യുന്നവര്‍ കുറ്റക്കാരല്ല.
\end{malayalam}}
\flushright{\begin{Arabic}
\quranayah[42][42]
\end{Arabic}}
\flushleft{\begin{malayalam}
ജനങ്ങളെ ദ്രോഹിക്കുകയും അന്യായമായി ഭൂമിയില്‍ അതിക്രമം പ്രവര്‍ത്തിക്കുകയും ചെയ്യുന്നവര്‍ മാത്രമാണ് കുറ്റക്കാര്‍. അത്തരക്കാര്‍ക്കു തന്നെയാണ് നോവേറിയ ശിക്ഷയുള്ളത്.
\end{malayalam}}
\flushright{\begin{Arabic}
\quranayah[42][43]
\end{Arabic}}
\flushleft{\begin{malayalam}
എന്നാല്‍ ആരെങ്കിലും ക്ഷമിക്കുകയും പൊറുക്കുകയുമാണെങ്കില്‍ തീര്‍ച്ചയായും അത് ഇച്ഛാശക്തി ആവശ്യമുള്ള കാര്യങ്ങളില്‍പെട്ടതുതന്നെ.
\end{malayalam}}
\flushright{\begin{Arabic}
\quranayah[42][44]
\end{Arabic}}
\flushleft{\begin{malayalam}
അല്ലാഹു ആരെയെങ്കിലും വഴികേടിലാക്കുകയാണെങ്കില്‍ പിന്നെ, അയാളെ രക്ഷിക്കാന്‍ ആര്‍ക്കുമാവില്ല. ശിക്ഷ നേരില്‍ കാണുംനേരം അക്രമികള്‍ “ഒരു തിരിച്ചുപോക്കിനു വല്ല വഴിയുമുണ്ടോ” എന്നു ചോദിക്കുന്നതായി നിനക്കു കാണാം.
\end{malayalam}}
\flushright{\begin{Arabic}
\quranayah[42][45]
\end{Arabic}}
\flushleft{\begin{malayalam}
നാണക്കേടിനാല്‍ തലകുനിച്ചവരായി നരകത്തിനു മുമ്പിലവരെ ഹാജരാക്കുന്നത് നിനക്കു കാണാം. ഒളികണ്ണിട്ട് അവര്‍ നരകത്തെ നോക്കും. അപ്പോള്‍ സത്യവിശ്വാസികള്‍ പറയും: "ഉയിര്‍ത്തെഴുന്നേല്‍പുനാളില്‍ തങ്ങളെയും തങ്ങളുടെ സ്വന്തക്കാരെയും നഷ്ടത്തില്‍പെടുത്തിയവര്‍തന്നെയാണ് തീര്‍ച്ചയായും തുലഞ്ഞവര്‍." അറിയുക: അക്രമികളെന്നെന്നും കഠിനശിക്ഷയിലായിരിക്കും.
\end{malayalam}}
\flushright{\begin{Arabic}
\quranayah[42][46]
\end{Arabic}}
\flushleft{\begin{malayalam}
അല്ലാഹുവെ കൂടാതെ തങ്ങളെ തുണക്കുന്ന രക്ഷാധികാരികളാരും അന്ന് അവര്‍ക്കുണ്ടാവുകയില്ല. അല്ലാഹു ആരെയെങ്കിലും വഴികേടിലാക്കിയാല്‍ പിന്നെ അവന്നു രക്ഷാമാര്‍ഗമൊന്നുമില്ല.
\end{malayalam}}
\flushright{\begin{Arabic}
\quranayah[42][47]
\end{Arabic}}
\flushleft{\begin{malayalam}
അല്ലാഹുവില്‍ നിന്ന് ആരാലും തട്ടിമാറ്റാനാവാത്ത ഒരു ദിനം വന്നെത്തും മുമ്പെ നിങ്ങള്‍ നിങ്ങളുടെ നാഥന്റെ വിളിക്കുത്തരം നല്‍കുക. അന്നാളില്‍ നിങ്ങള്‍ക്കൊരഭയകേന്ദ്രവുമുണ്ടാവുകയില്ല. നിങ്ങളുടെ ദുരവസ്ഥക്ക് അറുതിവരുത്താനും ആരുമുണ്ടാവില്ല.
\end{malayalam}}
\flushright{\begin{Arabic}
\quranayah[42][48]
\end{Arabic}}
\flushleft{\begin{malayalam}
അഥവാ, ഇനിയും അവര്‍ പിന്തിരിഞ്ഞുപോവുകയാണെങ്കില്‍, നിന്നെ നാം അവരുടെ സംരക്ഷകനായൊന്നും അയച്ചിട്ടില്ല. നിന്റെ ബാധ്യത സന്ദേശമെത്തിക്കല്‍ മാത്രമാണ്. മനുഷ്യനെ നാം നമ്മുടെ ഭാഗത്തുനിന്നുള്ള അനുഗ്രഹം ആസ്വദിപ്പിച്ചാല്‍ അതിലവന്‍ മതിമറന്നാഹ്ളാദിക്കുന്നു. എന്നാല്‍ തങ്ങളുടെ തന്നെ കൈക്കുറ്റങ്ങള്‍ കാരണമായി വല്ല വിപത്തും വന്നുപെട്ടാലോ, അപ്പോഴേക്കും മനുഷ്യന്‍ പറ്റെ നന്ദികെട്ടവനായിത്തീരുന്നു.
\end{malayalam}}
\flushright{\begin{Arabic}
\quranayah[42][49]
\end{Arabic}}
\flushleft{\begin{malayalam}
ആകാശഭൂമികളുടെ ആധിപത്യം അല്ലാഹുവിനാണ്. അവനിച്ഛിക്കുന്നത് അവന്‍ സൃഷ്ടിക്കുന്നു. അവനിച്ഛിക്കുന്നവര്‍ക്ക് അവന്‍ പെണ്‍മക്കളെ പ്രദാനം ചെയ്യുന്നു. അവനിച്ഛിക്കുന്നവര്‍ക്ക് ആണ്‍കുട്ടികളെയും സമ്മാനിക്കുന്നു.
\end{malayalam}}
\flushright{\begin{Arabic}
\quranayah[42][50]
\end{Arabic}}
\flushleft{\begin{malayalam}
അല്ലെങ്കില്‍ അവനവര്‍ക്ക് ആണ്‍കുട്ടികളെയും പെണ്‍കുട്ടികളെയും ഇടകലര്‍ത്തിക്കൊടുക്കുന്നു. അവനിച്ഛിക്കുന്നവരെ വന്ധ്യരാക്കുന്നു. തീര്‍ച്ചയായും അവന്‍ സകലതും അറിയുന്നവനാണ്. എല്ലാറ്റിനും കഴിവുറ്റവനും.
\end{malayalam}}
\flushright{\begin{Arabic}
\quranayah[42][51]
\end{Arabic}}
\flushleft{\begin{malayalam}
അല്ലാഹു ഒരു മനുഷ്യനോടും നേര്‍ക്കുനേരെ സംസാരിക്കാറില്ല. അതുണ്ടാവുന്നത് ഒന്നുകില്‍ ദിവ്യബോധനത്തിലൂടെയാണ്. അല്ലെങ്കില്‍ മറയ്ക്കുപിന്നില്‍ നിന്ന്, അതുമല്ലെങ്കില്‍ ഒരു ദൂതനെ അയച്ചുകൊണ്ട്. അങ്ങനെ അല്ലാഹുവിന്റെ അനുമതിയോടെ അവനിച്ഛിക്കുന്നത് ആ ദൂതനിലൂടെ ബോധനം നല്‍കുന്നു. സംശയമില്ല; അല്ലാഹു അത്യുന്നതനാണ്. യുക്തിമാനും.
\end{malayalam}}
\flushright{\begin{Arabic}
\quranayah[42][52]
\end{Arabic}}
\flushleft{\begin{malayalam}
ഇവ്വിധം നാം നിനക്ക് നമ്മുടെ കല്‍പനയാല്‍ ചൈതന്യവത്തായ ഒരു സന്ദേശം ബോധനം നല്‍കിയിരിക്കുന്നു. വേദപുസ്തകത്തെപ്പറ്റിയോ സത്യവിശ്വാസത്തെ സംബന്ധിച്ചോ നിനക്കൊന്നുമറിയുമായിരുന്നില്ല. അങ്ങനെ ആ സന്ദേശത്തെ നാമൊരു വെളിച്ചമാക്കിയിരിക്കുന്നു. അതുവഴി നമ്മുടെ ദാസന്മാരില്‍ നിന്ന് നാം ഇച്ഛിക്കുന്നവരെ നേര്‍വഴിയില്‍ നയിക്കുന്നു. തീര്‍ച്ചയായും നീ നേര്‍മാര്‍ഗത്തിലേക്കാണ് വഴി നടത്തുന്നത്;
\end{malayalam}}
\flushright{\begin{Arabic}
\quranayah[42][53]
\end{Arabic}}
\flushleft{\begin{malayalam}
ആകാശഭൂമികളിലുള്ളവയുടെയെല്ലാം ഉടമയായ അല്ലാഹുവിന്റെ മാര്‍ഗത്തിലേക്ക്. അറിയുക: കാര്യങ്ങളൊക്കെയും മടങ്ങിയെത്തുക അല്ലാഹുവിങ്കലാണ്.
\end{malayalam}}
\chapter{\textmalayalam{സുഖ്റുഫ് ( സുവര്‍ണ്ണാലങ്കാരം )}}
\begin{Arabic}
\Huge{\centerline{\basmalah}}\end{Arabic}
\flushright{\begin{Arabic}
\quranayah[43][1]
\end{Arabic}}
\flushleft{\begin{malayalam}
ഹാ - മീം.
\end{malayalam}}
\flushright{\begin{Arabic}
\quranayah[43][2]
\end{Arabic}}
\flushleft{\begin{malayalam}
സുവ്യക്തമായ ഈ വേദപുസ്തകം തന്നെ സത്യം.
\end{malayalam}}
\flushright{\begin{Arabic}
\quranayah[43][3]
\end{Arabic}}
\flushleft{\begin{malayalam}
തീര്‍ച്ചയായും നാം ഇതിനെ അറബി ഭാഷയിലുള്ള ഖുര്‍ആന്‍ ആക്കിയിരിക്കുന്നു. നിങ്ങള്‍ ചിന്തിച്ചറിയാന്‍.
\end{malayalam}}
\flushright{\begin{Arabic}
\quranayah[43][4]
\end{Arabic}}
\flushleft{\begin{malayalam}
സംശയമില്ല; ഇത് ഒരു മൂലപ്രമാണത്തിലുള്ളതാണ്. നമ്മുടെയടുത്ത് അത്യുന്നത സ്ഥാനമുള്ളതും തത്ത്വപൂര്‍ണവുമാണിത്.
\end{malayalam}}
\flushright{\begin{Arabic}
\quranayah[43][5]
\end{Arabic}}
\flushleft{\begin{malayalam}
നിങ്ങള്‍ അതിരുവിട്ട് കഴിയുന്ന ജനമായതിനാല്‍ നിങ്ങളെ മാറ്റിനിര്‍ത്തി, നിങ്ങള്‍ക്ക് ഈ ഉദ്ബോധനം നല്‍കുന്നത് നാം നിര്‍ത്തിവെക്കുകയോ?
\end{malayalam}}
\flushright{\begin{Arabic}
\quranayah[43][6]
\end{Arabic}}
\flushleft{\begin{malayalam}
പൂര്‍വസമൂഹങ്ങളില്‍ നാം നിരവധി പ്രവാചകന്മാരെ അയച്ചിട്ടുണ്ട്.
\end{malayalam}}
\flushright{\begin{Arabic}
\quranayah[43][7]
\end{Arabic}}
\flushleft{\begin{malayalam}
ജനങ്ങള്‍ തങ്ങള്‍ക്ക് വന്നെത്തിയ ഒരു പ്രവാചകനെയും പരിഹസിക്കാതിരുന്നിട്ടില്ല.
\end{malayalam}}
\flushright{\begin{Arabic}
\quranayah[43][8]
\end{Arabic}}
\flushleft{\begin{malayalam}
അങ്ങനെ ഇവരെക്കാള്‍ എത്രയോ കയ്യൂക്കും കരുത്തുമുണ്ടായിരുന്നവരെ നാം നശിപ്പിച്ചിട്ടുണ്ട്. പൂര്‍വികരുടെ ഉദാഹരണങ്ങള്‍ നേരത്തെ കഴിഞ്ഞുപോയിട്ടുമുണ്ട്.
\end{malayalam}}
\flushright{\begin{Arabic}
\quranayah[43][9]
\end{Arabic}}
\flushleft{\begin{malayalam}
ആകാശഭൂമികളെ സൃഷ്ടിച്ചതാരെന്ന് നീ അവരോട് ചോദിച്ചാല്‍ ഉറപ്പായും അവര്‍ പറയും: "പ്രതാപിയും എല്ലാം അറിയുന്നവനുമായവനാണ് അവയെ സൃഷ്ടിച്ചത്."
\end{malayalam}}
\flushright{\begin{Arabic}
\quranayah[43][10]
\end{Arabic}}
\flushleft{\begin{malayalam}
നിങ്ങള്‍ക്കായി ഭൂമിയെ തൊട്ടിലാക്കിത്തന്നവനാണവന്‍. അതില്‍ പാതകളൊരുക്കിത്തന്നവനും. നിങ്ങള്‍ വഴിയറിയുന്നവരാകാന്‍.
\end{malayalam}}
\flushright{\begin{Arabic}
\quranayah[43][11]
\end{Arabic}}
\flushleft{\begin{malayalam}
മാനത്തുനിന്ന് നിശ്ചിതതോതില്‍ വെള്ളം വീഴ്ത്തിത്തന്നതും അവനാണ്. അങ്ങനെ അതുവഴി നാം ചത്തുകിടക്കുന്ന ഭൂമിയെ ചൈതന്യവത്താക്കി. അവ്വിധം ഒരുനാള്‍ നിങ്ങളെയും ജീവനേകി പുറത്തെടുക്കും.
\end{malayalam}}
\flushright{\begin{Arabic}
\quranayah[43][12]
\end{Arabic}}
\flushleft{\begin{malayalam}
എല്ലാറ്റിലും ഇണകളെ സൃഷ്ടിച്ചവനും അവന്‍ തന്നെ. കപ്പലുകളിലും കന്നുകാലികളിലും നിങ്ങള്‍ക്ക് യാത്ര സൌകര്യപ്പെടുത്തിയതും മറ്റാരുമല്ല.
\end{malayalam}}
\flushright{\begin{Arabic}
\quranayah[43][13]
\end{Arabic}}
\flushleft{\begin{malayalam}
നിങ്ങളവയുടെ പുറത്തുകയറി ഇരിപ്പുറപ്പിക്കാനാണിത്. അങ്ങനെ, നിങ്ങള്‍ അവിടെ ഇരുപ്പുറപ്പിച്ചാല്‍ നിങ്ങളുടെ നാഥന്റെ അനുഗ്രഹങ്ങള്‍ ഓര്‍ക്കാനും നിങ്ങളിങ്ങനെ പറയാനുമാണ്: "ഞങ്ങള്‍ക്കിവയെ അധീനപ്പെടുത്തിത്തന്നവന്‍ എത്ര പരിശുദ്ധന്‍! നമുക്ക് സ്വയമവയെ കീഴ്പെടുത്താന്‍ കഴിയുമായിരുന്നില്ല.
\end{malayalam}}
\flushright{\begin{Arabic}
\quranayah[43][14]
\end{Arabic}}
\flushleft{\begin{malayalam}
"തീര്‍ച്ചയായും ഞങ്ങള്‍ ഞങ്ങളുടെ നാഥന്റെ അടുത്തേക്ക് തിരിച്ചുചെല്ലേണ്ടവരാണ്."
\end{malayalam}}
\flushright{\begin{Arabic}
\quranayah[43][15]
\end{Arabic}}
\flushleft{\begin{malayalam}
ഈ ജനം അല്ലാഹുവിന്റെ ദാസന്മാരില്‍ ഒരു വിഭാഗത്തെ അവന്റെ ഭാഗമാക്കി 1 വെച്ചിരിക്കുന്നു. മനുഷ്യന്‍ പ്രത്യക്ഷത്തില്‍ തന്നെ വളരെ നന്ദികെട്ടവനാണ്.
\end{malayalam}}
\flushright{\begin{Arabic}
\quranayah[43][16]
\end{Arabic}}
\flushleft{\begin{malayalam}
അതല്ല; അല്ലാഹു തന്റെ സൃഷ്ടികളില്‍ പെണ്‍മക്കളെ തനിക്കുമാത്രമാക്കി വെക്കുകയും ആണ്‍കുട്ടികളെ നിങ്ങള്‍ക്ക് പ്രത്യേകം തരികയും ചെയ്തുവെന്നോ?
\end{malayalam}}
\flushright{\begin{Arabic}
\quranayah[43][17]
\end{Arabic}}
\flushleft{\begin{malayalam}
പരമകാരുണികനായ അല്ലാഹുവോട് ചേര്‍ത്തിപ്പറയുന്ന പെണ്ണിന്റെ പിറവിയെപ്പറ്റി അവരിലൊരാള്‍ക്ക് ശുഭവാര്‍ത്ത അറിയിച്ചാല്‍ അവന്റെ മുഖം കറുത്തിരുണ്ടതായിത്തീരുന്നു. അവന്‍ അത്യധികം ദുഃഖിതനാവുന്നു.
\end{malayalam}}
\flushright{\begin{Arabic}
\quranayah[43][18]
\end{Arabic}}
\flushleft{\begin{malayalam}
ആഭരണങ്ങളണിയിച്ച് വളര്‍ത്തപ്പെടുന്ന, തര്‍ക്കങ്ങളില്‍ തന്റെ നിലപാട് തെളിയിക്കാന്‍ കഴിവില്ലാത്ത സന്തതിയെയാണോ അല്ലാഹുവിന്റെ പേരില്‍ ആരോപിക്കുന്നത്?
\end{malayalam}}
\flushright{\begin{Arabic}
\quranayah[43][19]
\end{Arabic}}
\flushleft{\begin{malayalam}
പരമകാരുണികനായ അല്ലാഹുവിന്റെ അടിമകളായ മലക്കുകളെ ഇവര്‍ സ്ത്രീകളായി സങ്കല്‍പിച്ചിരിക്കുന്നു. അവരുടെ സൃഷ്ടികര്‍മത്തിന് ഇവര്‍ സാക്ഷികളായിരുന്നോ? ഇവരുടെ സാക്ഷ്യം രേഖപ്പെടുത്തുക തന്നെ ചെയ്യും. അതിന്റെ പേരിലിവരെ ചോദ്യം ചെയ്യുന്നതുമാണ്.
\end{malayalam}}
\flushright{\begin{Arabic}
\quranayah[43][20]
\end{Arabic}}
\flushleft{\begin{malayalam}
ഇക്കൂട്ടര്‍ പറയുന്നു: "പരമകാരുണികനായ അല്ലാഹു ഇച്ഛിച്ചിരുന്നെങ്കില്‍ ഞങ്ങളൊരിക്കലും അവരെ പൂജിക്കുമായിരുന്നില്ല." സത്യത്തിലിവര്‍ക്ക് അതേപ്പറ്റി ഒന്നുമറിയില്ല. വെറും അനുമാനങ്ങള്‍ മെനഞ്ഞുണ്ടാക്കുകയാണിവര്‍.
\end{malayalam}}
\flushright{\begin{Arabic}
\quranayah[43][21]
\end{Arabic}}
\flushleft{\begin{malayalam}
അതല്ല; നാം ഇവര്‍ക്ക് നേരത്തെ വല്ല വേദപുസ്തകവും കൊടുത്തിട്ടുണ്ടോ? അങ്ങനെ ഇവരത് മുറുകെപ്പിടിക്കുകയാണോ?
\end{malayalam}}
\flushright{\begin{Arabic}
\quranayah[43][22]
\end{Arabic}}
\flushleft{\begin{malayalam}
എന്നാല്‍ ഇവര്‍ പറയുന്നതിതാണ്: "ഞങ്ങളുടെ പിതാക്കള്‍ ഒരു വഴിയില്‍ നിലകൊണ്ടതായി ഞങ്ങള്‍ കണ്ടിരിക്കുന്നു. തീര്‍ച്ചയായും ഞങ്ങള്‍ അവരുടെ പാത പിന്തുടര്‍ന്ന് നേര്‍വഴിയില്‍ നീങ്ങുകയാണ്."
\end{malayalam}}
\flushright{\begin{Arabic}
\quranayah[43][23]
\end{Arabic}}
\flushleft{\begin{malayalam}
ഇവ്വിധം നാം നിനക്കുമുമ്പ് പല നാടുകളിലേക്കും മുന്നറിയിപ്പുകാരെ അയച്ചു; അപ്പോഴെല്ലാം അവരിലെ സുഖലോലുപര്‍ പറഞ്ഞിരുന്നത് ഇതാണ്: "ഞങ്ങളുടെ പൂര്‍വ പിതാക്കള്‍ ഒരു മാര്‍ഗമവലംബിക്കുന്നവരായി ഞങ്ങള്‍ കണ്ടിട്ടുണ്ട്. തീര്‍ച്ചയായും ഞങ്ങള്‍ അവരുടെ പാരമ്പര്യം മുറുകെപ്പിടിക്കുകയാണ്."
\end{malayalam}}
\flushright{\begin{Arabic}
\quranayah[43][24]
\end{Arabic}}
\flushleft{\begin{malayalam}
ആ മുന്നറിയിപ്പുകാരന്‍ ചോദിച്ചു: "നിങ്ങളുടെ പിതാക്കള്‍ പിന്തുടരുന്നതായി നിങ്ങള്‍ കണ്ട മാര്‍ഗത്തെക്കാള്‍ ഏറ്റം ചൊവ്വായ വഴിയുമായി ഞാന്‍ നിങ്ങളുടെ അടുത്തുവന്നാലും നിങ്ങളതംഗീകരിക്കില്ലേ?" അവര്‍ അപ്പോഴൊക്കെ പറഞ്ഞിരുന്നതിതാണ്: "നിങ്ങള്‍ ഏതൊരു ജീവിതമാര്‍ഗവുമായാണോ അയക്കപ്പെട്ടിരിക്കുന്നത് അതിനെ ഞങ്ങളിതാ തള്ളിപ്പറയുന്നു."
\end{malayalam}}
\flushright{\begin{Arabic}
\quranayah[43][25]
\end{Arabic}}
\flushleft{\begin{malayalam}
അവസാനം നാം അവരോട് പ്രതികാരം ചെയ്തു. നോക്കൂ; സത്യത്തെ തള്ളിപ്പറഞ്ഞവരുടെ അന്ത്യം എവ്വിധമായിരുന്നുവെന്ന്.
\end{malayalam}}
\flushright{\begin{Arabic}
\quranayah[43][26]
\end{Arabic}}
\flushleft{\begin{malayalam}
ഇബ്റാഹീം തന്റെ പിതാവിനോടും ജനതയോടും പറഞ്ഞ സന്ദര്‍ഭം: "നിങ്ങള്‍ പൂജിച്ചുകൊണ്ടിരിക്കുന്നവയില്‍ നിന്നെല്ലാം തീര്‍ത്തും മുക്തനാണ് ഞാന്‍.
\end{malayalam}}
\flushright{\begin{Arabic}
\quranayah[43][27]
\end{Arabic}}
\flushleft{\begin{malayalam}
"എന്നെ സൃഷ്ടിച്ചവനില്‍നിന്നൊഴികെ. അവനെന്നെ നേര്‍വഴിയിലാക്കും."
\end{malayalam}}
\flushright{\begin{Arabic}
\quranayah[43][28]
\end{Arabic}}
\flushleft{\begin{malayalam}
ഈ വചനത്തെ ഇബ്റാഹീം തന്റെ പിന്‍ഗാമികളിലും ബാക്കിവെച്ചു. അവര്‍ സത്യത്തിലേക്ക് തിരിച്ചുവരാന്‍.
\end{malayalam}}
\flushright{\begin{Arabic}
\quranayah[43][29]
\end{Arabic}}
\flushleft{\begin{malayalam}
ഇക്കൂട്ടരെയും ഇവരുടെ മുന്‍ഗാമികളെയും ഞാന്‍ ജീവിതം ആസ്വദിപ്പിച്ചു. സത്യസന്ദേശവും അത് വ്യക്തമായി വിവരിച്ചുകൊടുക്കുന്ന ദൈവദൂതനും അവര്‍ക്ക് വന്നെത്തുംവരെ.
\end{malayalam}}
\flushright{\begin{Arabic}
\quranayah[43][30]
\end{Arabic}}
\flushleft{\begin{malayalam}
അങ്ങനെ അവര്‍ക്ക് സത്യം വന്നെത്തി. അപ്പോള്‍ അവര്‍ പറഞ്ഞു: "ഇത് വെറുമൊരു മായാജാലമാണ്. ഞങ്ങളിതിനെ ഇതാ തള്ളിപ്പറയുന്നു."
\end{malayalam}}
\flushright{\begin{Arabic}
\quranayah[43][31]
\end{Arabic}}
\flushleft{\begin{malayalam}
ഇവര്‍ ചോദിക്കുന്നു: "ഈ ഖുര്‍ആന്‍ ഈ രണ്ട് പട്ടണങ്ങളിലെ ഏതെങ്കിലും മഹാപുരുഷന്ന് ഇറക്കിക്കിട്ടാത്തതെന്ത്?"
\end{malayalam}}
\flushright{\begin{Arabic}
\quranayah[43][32]
\end{Arabic}}
\flushleft{\begin{malayalam}
ഇവരാണോ നിന്റെ നാഥന്റെ അനുഗ്രഹം വീതംവെച്ചുകൊടുക്കുന്നത്? ഐഹികജീവിതത്തില്‍ ഇവര്‍ക്കിടയില്‍ തങ്ങളുടെ ജീവിതവിഭവം വീതംവെച്ചുകൊടുത്തത് നാമാണ്. അങ്ങനെ ഇവരില്‍ ചിലര്‍ക്കു മറ്റുചിലരെക്കാള്‍ നാം പല പദവികളും നല്‍കി. ഇവരില്‍ ചിലര്‍ മറ്റു ചിലരെ തങ്ങളുടെ ചൊല്‍പ്പടിയില്‍ നിര്‍ത്താനാണിത്. ഇവര്‍ ശേഖരിച്ചുവെക്കുന്നതിനെക്കാളെല്ലാം ഉത്തമം നിന്റെ നാഥന്റെ അനുഗ്രഹംതന്നെ.
\end{malayalam}}
\flushright{\begin{Arabic}
\quranayah[43][33]
\end{Arabic}}
\flushleft{\begin{malayalam}
ജനം ഒരൊറ്റ സമുദായമായിപ്പോകുമായിരുന്നില്ലെങ്കില്‍ പരമകാരുണികനായ അല്ലാഹുവെ തള്ളിപ്പറയുന്നവര്‍ക്ക്, അവരുടെ വീടുകള്‍ക്ക് വെള്ളികൊണ്ടുള്ള മേല്‍പ്പുരകളും അവര്‍ക്ക് കയറിപ്പോകാന്‍ വെള്ളികൊണ്ടുള്ള കോണികളും നാം ഉണ്ടാക്കിക്കൊടുക്കുമായിരുന്നു.
\end{malayalam}}
\flushright{\begin{Arabic}
\quranayah[43][34]
\end{Arabic}}
\flushleft{\begin{malayalam}
അങ്ങനെ അവരുടെ വീടുകള്‍ക്ക് വാതിലുകളും അവര്‍ക്ക് ചാരിയിരിക്കാനുള്ള കട്ടിലുകളും നല്‍കുമായിരുന്നു.
\end{malayalam}}
\flushright{\begin{Arabic}
\quranayah[43][35]
\end{Arabic}}
\flushleft{\begin{malayalam}
സ്വര്‍ണത്താലുള്ള അലങ്കാരങ്ങളും. എന്നാല്‍ ഇതെല്ലാം ഐഹികജീവിതത്തിലെ സുഖഭോഗവിഭവം മാത്രമാണ്. പരലോകം നിന്റെ നാഥന്റെ അടുത്ത് ഭക്തന്മാര്‍ക്ക് മാത്രമുള്ളതാണ്.
\end{malayalam}}
\flushright{\begin{Arabic}
\quranayah[43][36]
\end{Arabic}}
\flushleft{\begin{malayalam}
പരമകാരുണികന്റെ ഉദ്ബോധനത്തോട് അന്ധത നടിക്കുന്നവന്ന് നാം ഒരു ചെകുത്താനെ ഏര്‍പ്പെടുത്തും. അങ്ങനെ ആ ചെകുത്താന്‍ അവന്റെ ചങ്ങാതിയായിത്തീരും.
\end{malayalam}}
\flushright{\begin{Arabic}
\quranayah[43][37]
\end{Arabic}}
\flushleft{\begin{malayalam}
തീര്‍ച്ചയായും ആ ചെകുത്താന്മാര്‍ അവരെ നേര്‍വഴിയില്‍ നിന്ന് തടയുന്നു. അതോടൊപ്പം തങ്ങള്‍ നേര്‍വഴിയില്‍ തന്നെയാണെന്ന് അവര്‍ വിചാരിക്കുന്നു.
\end{malayalam}}
\flushright{\begin{Arabic}
\quranayah[43][38]
\end{Arabic}}
\flushleft{\begin{malayalam}
അവസാനം നമ്മുടെയടുത്ത് വന്നെത്തുമ്പോള്‍ അയാള്‍ തന്നോടൊപ്പമുള്ള ചെകുത്താനോട് പറയും: "എനിക്കും നിനക്കുമിടയില്‍ ഉദയാസ്തമയ സ്ഥാനങ്ങള്‍ തമ്മിലുള്ള അകലം ഉണ്ടായിരുന്നെങ്കില്‍! നീയെത്ര ചീത്ത ചങ്ങാതി!"
\end{malayalam}}
\flushright{\begin{Arabic}
\quranayah[43][39]
\end{Arabic}}
\flushleft{\begin{malayalam}
നിങ്ങള്‍ അക്രമം പ്രവര്‍ത്തിച്ചിരിക്കെ, എല്ലാവരും ശിക്ഷയില്‍ പങ്കാളികളാണെന്നതുകൊണ്ട് ഇന്ന് നിങ്ങള്‍ക്ക് പ്രയോജനമൊന്നുമില്ല.
\end{malayalam}}
\flushright{\begin{Arabic}
\quranayah[43][40]
\end{Arabic}}
\flushleft{\begin{malayalam}
നിനക്ക് ബധിരന്മാരെ കേള്‍പ്പിക്കാനാകുമോ? കണ്ണില്ലാത്തവരെയും വ്യക്തമായ വഴികേടിലായവരെയും നേര്‍വഴിയിലാക്കാന്‍ നിനക്ക് കഴിയുമോ?
\end{malayalam}}
\flushright{\begin{Arabic}
\quranayah[43][41]
\end{Arabic}}
\flushleft{\begin{malayalam}
ഏതായാലുംശരി, നാമവരെ ശിക്ഷിക്കുക തന്നെ ചെയ്യും. ഒരുവേള നിന്നെ നാം ഇഹലോകത്തുനിന്ന് കൊണ്ടുപോയിക്കഴിഞ്ഞ ശേഷമാവാം;
\end{malayalam}}
\flushright{\begin{Arabic}
\quranayah[43][42]
\end{Arabic}}
\flushleft{\begin{malayalam}
അല്ലെങ്കില്‍ നാമവര്‍ക്ക് വാഗ്ദാനം ചെയ്ത ശിക്ഷ നിനക്കു നാം കാണിച്ചുതന്നേക്കാം. തീര്‍ച്ചയായും അവരെ ശിക്ഷിക്കാന്‍ നാം തികച്ചും കഴിവുറ്റവന്‍ തന്നെ.
\end{malayalam}}
\flushright{\begin{Arabic}
\quranayah[43][43]
\end{Arabic}}
\flushleft{\begin{malayalam}
അതിനാല്‍ നിനക്ക് നാം ബോധനം നല്‍കിയത് മുറുകെപ്പിടിക്കുക. ഉറപ്പായും നീ നേര്‍വഴിയിലാണ്.
\end{malayalam}}
\flushright{\begin{Arabic}
\quranayah[43][44]
\end{Arabic}}
\flushleft{\begin{malayalam}
തീര്‍ച്ചയായും ഈ വേദം നിനക്കും നിന്റെ ജനത്തിനും ഒരു ഉദ്ബോധനമാണ്. ഒരുനാള്‍ അതേക്കുറിച്ച് നിങ്ങളെ ചോദ്യം ചെയ്യും.
\end{malayalam}}
\flushright{\begin{Arabic}
\quranayah[43][45]
\end{Arabic}}
\flushleft{\begin{malayalam}
നിനക്കുമുമ്പ് നാം നിയോഗിച്ച നമ്മുടെ ദൂതന്മാരോട് ചോദിച്ചുനോക്കൂ, പരമകാരുണികനെ ക്കൂടാതെ പൂജിക്കപ്പെടേണ്ട വല്ല ദൈവങ്ങളെയും നാം നിശ്ചയിച്ചിട്ടുണ്ടോയെന്ന്.
\end{malayalam}}
\flushright{\begin{Arabic}
\quranayah[43][46]
\end{Arabic}}
\flushleft{\begin{malayalam}
മൂസായെ നാം നമ്മുടെ ദൃഷ്ടാന്തങ്ങളുമായി ഫറവോന്റെയും അവന്റെ പ്രധാനികളുടെയും അടുത്തേക്കയച്ചു. അപ്പോള്‍ അദ്ദേഹം പറഞ്ഞു: "സംശയം വേണ്ട; ഞാന്‍ പ്രപഞ്ചനാഥന്റെ ദൂതനാണ്."
\end{malayalam}}
\flushright{\begin{Arabic}
\quranayah[43][47]
\end{Arabic}}
\flushleft{\begin{malayalam}
അങ്ങനെ അദ്ദേഹം നമ്മുടെ തെളിവുകളുമായി അവരുടെ അടുത്ത് ചെന്നപ്പോഴോ, അവരതാ അവയെ പരിഹസിച്ചു ചിരിക്കുന്നു.
\end{malayalam}}
\flushright{\begin{Arabic}
\quranayah[43][48]
\end{Arabic}}
\flushleft{\begin{malayalam}
അവര്‍ക്കു നാം തെളിവുകള്‍ ഓരോന്നോരോന്നായി കാണിച്ചുകൊടുത്തു. അവയോരോന്നും അതിന്റെ മുമ്പത്തേതിനെക്കാള്‍ ഗംഭീരമായിരുന്നു. അവസാനം നാം അവരെ നമ്മുടെ ശിക്ഷയാല്‍ പിടികൂടി. എല്ലാം അവരതില്‍ നിന്ന് തിരിച്ചുവരാന്‍ വേണ്ടിയായിരുന്നു.
\end{malayalam}}
\flushright{\begin{Arabic}
\quranayah[43][49]
\end{Arabic}}
\flushleft{\begin{malayalam}
അവര്‍ പറഞ്ഞു: "അല്ലയോ ജാലവിദ്യക്കാരാ, നീയുമായി നിന്റെ നാഥനുണ്ടാക്കിയ കരാറനുസരിച്ച് നീ നിന്റെ നാഥനോട് ഞങ്ങള്‍ക്കുവേണ്ടി പ്രാര്‍ഥിക്കുക. ഉറപ്പായും ഞങ്ങള്‍ നേര്‍വഴിയില്‍ വന്നുകൊള്ളാം."
\end{malayalam}}
\flushright{\begin{Arabic}
\quranayah[43][50]
\end{Arabic}}
\flushleft{\begin{malayalam}
അങ്ങനെ നാം അവരില്‍നിന്ന് ആ ശിക്ഷ നീക്കിക്കളഞ്ഞപ്പോള്‍ അവരതാ തങ്ങളുടെ വാക്ക് ലംഘിക്കുന്നു.
\end{malayalam}}
\flushright{\begin{Arabic}
\quranayah[43][51]
\end{Arabic}}
\flushleft{\begin{malayalam}
ഫറവോന്‍ തന്റെ ജനത്തോട് വിളിച്ചുചോദിച്ചു: "എന്റെ ജനമേ, ഈജിപ്തിന്റെ ആധിപത്യം എനിക്കല്ലേ? ഈ നദികളൊഴുകുന്നത് എന്റെ താഴ്ഭാഗത്തൂടെയല്ലേ? എന്നിട്ടും നിങ്ങള്‍ കാര്യം കണ്ടറിയുന്നില്ലേ?
\end{malayalam}}
\flushright{\begin{Arabic}
\quranayah[43][52]
\end{Arabic}}
\flushleft{\begin{malayalam}
"അല്ല, നന്നെ നിസ്സാരനും വ്യക്തമായി സംസാരിക്കാന്‍ പോലും കഴിയാത്തവനുമായ ഇവനെക്കാളുത്തമന്‍ ഞാന്‍ തന്നെയല്ലേ?
\end{malayalam}}
\flushright{\begin{Arabic}
\quranayah[43][53]
\end{Arabic}}
\flushleft{\begin{malayalam}
"ഇവന്‍ പ്രവാചകനെങ്കില്‍ ഇവനെ സ്വര്‍ണവളകളണിയിക്കാത്തതെന്ത്? അല്ലെങ്കില്‍ ഇവനോടൊത്ത് അകമ്പടിക്കാരായി മലക്കുകള്‍ വരാത്തതെന്ത്?"
\end{malayalam}}
\flushright{\begin{Arabic}
\quranayah[43][54]
\end{Arabic}}
\flushleft{\begin{malayalam}
അങ്ങനെ ഫറവോന്‍ തന്റെ ജനത്തെ വിഡ്ഢികളാക്കി. അതോടെ അവര്‍ അവനെ അനുസരിച്ചു. അവര്‍ തീര്‍ത്തും അധാര്‍മികരായ ജനതയായിരുന്നു.
\end{malayalam}}
\flushright{\begin{Arabic}
\quranayah[43][55]
\end{Arabic}}
\flushleft{\begin{malayalam}
അവസാനം അവര്‍ നമ്മെ പ്രകോപിപ്പിച്ചപ്പോള്‍ നാം അവരോട് പ്രതികാരം ചെയ്തു. അവരെയൊക്കെ മുക്കിയൊടുക്കി.
\end{malayalam}}
\flushright{\begin{Arabic}
\quranayah[43][56]
\end{Arabic}}
\flushleft{\begin{malayalam}
അങ്ങനെ അവരെ നാം പിന്‍ഗാമികള്‍ക്ക് ഒരു മാതൃകയാക്കി. ഒപ്പം ഗുണപാഠമാകുന്ന ഒരുദാഹരണവും.
\end{malayalam}}
\flushright{\begin{Arabic}
\quranayah[43][57]
\end{Arabic}}
\flushleft{\begin{malayalam}
മര്‍യമിന്റെ മകനെ മാതൃകാ പുരുഷനായി എടുത്തുകാണിച്ചപ്പോഴും നിന്റെ ജനതയിതാ അതിന്റെ പേരില്‍ ഒച്ചവെക്കുന്നു.
\end{malayalam}}
\flushright{\begin{Arabic}
\quranayah[43][58]
\end{Arabic}}
\flushleft{\begin{malayalam}
അവര്‍ ചോദിക്കുന്നു: "ഞങ്ങളുടെ ദൈവങ്ങളാണോ ഉത്തമം; അതല്ല ഇവനോ?" അവര്‍ നിന്നോട് ഇതെടുത്തുപറയുന്നത് തര്‍ക്കത്തിനുവേണ്ടി മാത്രമാണ്. സത്യത്തിലവര്‍ തീര്‍ത്തും താര്‍ക്കികരായ ജനം തന്നെയാണ്.
\end{malayalam}}
\flushright{\begin{Arabic}
\quranayah[43][59]
\end{Arabic}}
\flushleft{\begin{malayalam}
അദ്ദേഹം നമ്മുടെ ഒരു ദാസന്‍ മാത്രമാണ്. നാം അദ്ദേഹത്തിന് അനുഗ്രഹമേകി. അദ്ദേഹത്തെ ഇസ്രയേല്‍ മക്കള്‍ക്ക് മാതൃകയാക്കുകയും ചെയ്തു.
\end{malayalam}}
\flushright{\begin{Arabic}
\quranayah[43][60]
\end{Arabic}}
\flushleft{\begin{malayalam}
നാം ഉദ്ദേശിച്ചിരുന്നെങ്കില്‍ നിങ്ങള്‍ക്ക് പകരം നിങ്ങളില്‍ നിന്നുതന്നെ മലക്കുകളെ ഭൂമിയില്‍ പ്രതിനിധികളാക്കുമായിരുന്നു.
\end{malayalam}}
\flushright{\begin{Arabic}
\quranayah[43][61]
\end{Arabic}}
\flushleft{\begin{malayalam}
സംശയമില്ല; ഈസാനബി അന്ത്യസമയത്തിനുള്ള ഒരറിയിപ്പാണ്. നിങ്ങളതിലൊട്ടും സംശയിക്കരുത്. നിങ്ങളെന്നെ പിന്‍പറ്റുക. ഇതുതന്നെയാണ് നേര്‍വഴി.
\end{malayalam}}
\flushright{\begin{Arabic}
\quranayah[43][62]
\end{Arabic}}
\flushleft{\begin{malayalam}
പിശാച് നിങ്ങളെ ഇതില്‍നിന്ന് തടയാതിരിക്കട്ടെ. സംശയം വേണ്ട; അവന്‍ നിങ്ങളുടെ പ്രത്യക്ഷ ശത്രുവാണ്.
\end{malayalam}}
\flushright{\begin{Arabic}
\quranayah[43][63]
\end{Arabic}}
\flushleft{\begin{malayalam}
ഈസാ വ്യക്തമായ തെളിവുകളുമായി വന്ന് ഇങ്ങനെ പറഞ്ഞു: "ഞാനിതാ തത്ത്വജ്ഞാനവുമായി നിങ്ങളുടെ അടുത്ത് വന്നിരിക്കുന്നു, നിങ്ങള്‍ക്കിടയില്‍ അഭിപ്രായവ്യത്യാസമുള്ള കാര്യങ്ങളില്‍ നിങ്ങള്‍ക്ക് വിശദീകരണം നല്‍കാന്‍. അതിനാല്‍ നിങ്ങള്‍ അല്ലാഹുവോട് ഭക്തിയുള്ളവരാവുക. എന്നെ അനുസരിക്കുക.
\end{malayalam}}
\flushright{\begin{Arabic}
\quranayah[43][64]
\end{Arabic}}
\flushleft{\begin{malayalam}
"എന്റെയും നിങ്ങളുടെയും നാഥന്‍ അല്ലാഹുവാണ്. അതിനാല്‍ അവനെ മാത്രം വഴിപ്പെടുക. ഇതാണ് ഏറ്റവും ചൊവ്വായ മാര്‍ഗം."
\end{malayalam}}
\flushright{\begin{Arabic}
\quranayah[43][65]
\end{Arabic}}
\flushleft{\begin{malayalam}
അപ്പോള്‍ അവര്‍ പല കക്ഷികളായി ഭിന്നിച്ചു. അതിനാല്‍ അതിക്രമം കാണിച്ചവര്‍ക്ക് നോവുറ്റ നാളിന്റെ കടുത്തശിക്ഷയുടെ കൊടുംനാശമാണുണ്ടാവുക.
\end{malayalam}}
\flushright{\begin{Arabic}
\quranayah[43][66]
\end{Arabic}}
\flushleft{\begin{malayalam}
അവരറിയാതെ പെട്ടെന്ന് വന്നെത്തുന്ന അന്ത്യദിനമല്ലാതെ മറ്റെന്താണ് അവര്‍ക്ക് പ്രതീക്ഷിക്കാനുള്ളത്?
\end{malayalam}}
\flushright{\begin{Arabic}
\quranayah[43][67]
\end{Arabic}}
\flushleft{\begin{malayalam}
കൂട്ടുകാരൊക്കെയും അന്നാളില്‍ പരസ്പരം ശത്രുക്കളായി മാറും; ഭക്തന്മാരൊഴികെ.
\end{malayalam}}
\flushright{\begin{Arabic}
\quranayah[43][68]
\end{Arabic}}
\flushleft{\begin{malayalam}
"എന്റെ ദാസന്മാരേ, ഇന്ന് നിങ്ങളൊട്ടും പേടിക്കേണ്ടതില്ല. തീരേ ദുഃഖിക്കേണ്ടതുമില്ല.
\end{malayalam}}
\flushright{\begin{Arabic}
\quranayah[43][69]
\end{Arabic}}
\flushleft{\begin{malayalam}
"നമ്മുടെ വചനങ്ങളില്‍ വിശ്വസിച്ചവരാണ് നിങ്ങള്‍. അല്ലാഹുവിന് കീഴൊതുങ്ങിക്കഴിഞ്ഞവരും.
\end{malayalam}}
\flushright{\begin{Arabic}
\quranayah[43][70]
\end{Arabic}}
\flushleft{\begin{malayalam}
"നിങ്ങളും നിങ്ങളുടെ ഇണകളും സന്തോഷപൂര്‍വം സ്വര്‍ഗത്തില്‍ പ്രവേശിച്ചുകൊള്ളുക."
\end{malayalam}}
\flushright{\begin{Arabic}
\quranayah[43][71]
\end{Arabic}}
\flushleft{\begin{malayalam}
സ്വര്‍ണത്താലങ്ങളും കോപ്പകളും അവര്‍ക്ക് ചുറ്റും കറങ്ങിക്കൊണ്ടിരിക്കും. മനസ്സ് മോഹിക്കുന്നതും കണ്ണുകള്‍ക്ക് ആനന്ദകരമായതുമൊക്കെ അവിടെ കിട്ടും. "നിങ്ങളവിടെ നിത്യവാസികളായിരിക്കും.
\end{malayalam}}
\flushright{\begin{Arabic}
\quranayah[43][72]
\end{Arabic}}
\flushleft{\begin{malayalam}
"നിങ്ങള്‍ പ്രവര്‍ത്തിച്ചിരുന്നതിന്റെ ഫലമായി നിങ്ങള്‍ ഈ സ്വര്‍ഗത്തിനവകാശികളായിത്തീര്‍ന്നിരിക്കുന്നു.
\end{malayalam}}
\flushright{\begin{Arabic}
\quranayah[43][73]
\end{Arabic}}
\flushleft{\begin{malayalam}
"നിങ്ങള്‍ക്കതില്‍ ധാരാളം പഴങ്ങളുണ്ട്. അതില്‍ നിന്ന് ഇഷ്ടംപോലെ ഭക്ഷിക്കാം."
\end{malayalam}}
\flushright{\begin{Arabic}
\quranayah[43][74]
\end{Arabic}}
\flushleft{\begin{malayalam}
സംശയമില്ല; കുറ്റവാളികള്‍ നരകശിക്ഷയില്‍ എന്നെന്നും കഴിയേണ്ടവരാണ്.
\end{malayalam}}
\flushright{\begin{Arabic}
\quranayah[43][75]
\end{Arabic}}
\flushleft{\begin{malayalam}
അവര്‍ക്കതിലൊരിളവും കിട്ടുകയില്ല. അവരതില്‍ നിരാശരായി കഴിയേണ്ടിവരും.
\end{malayalam}}
\flushright{\begin{Arabic}
\quranayah[43][76]
\end{Arabic}}
\flushleft{\begin{malayalam}
നാം അവരോട് ഒരതിക്രമവും കാട്ടിയിട്ടില്ല. എന്നാല്‍ അവര്‍ തങ്ങളോടുതന്നെ അതിക്രമം കാണിക്കുകയായിരുന്നു.
\end{malayalam}}
\flushright{\begin{Arabic}
\quranayah[43][77]
\end{Arabic}}
\flushleft{\begin{malayalam}
അവര്‍ വിളിച്ചുകേഴും: "മാലികേ, അങ്ങയുടെ നാഥന്‍ ഞങ്ങള്‍ക്ക് ഇപ്പോള്‍തന്നെ മരണം തന്നിരുന്നെങ്കില്‍ നന്നായേനെ." മാലിക് പറയും: "നിങ്ങളിവിടെ താമസിക്കേണ്ടവര്‍ തന്നെയാണ്.
\end{malayalam}}
\flushright{\begin{Arabic}
\quranayah[43][78]
\end{Arabic}}
\flushleft{\begin{malayalam}
"തീര്‍ച്ചയായും ഞങ്ങള്‍ നിങ്ങള്‍ക്ക് സത്യം എത്തിച്ചുതന്നിട്ടുണ്ടായിരുന്നു. എന്നാല്‍ നിങ്ങളിലേറെ പേരും സത്യത്തെ വെറുക്കുന്നവരായിരുന്നു."
\end{malayalam}}
\flushright{\begin{Arabic}
\quranayah[43][79]
\end{Arabic}}
\flushleft{\begin{malayalam}
അതല്ല; ഇക്കൂട്ടരിവിടെ വല്ല പദ്ധതിയും നടപ്പാക്കാന്‍ തീരുമാനിച്ചിരിക്കയാണോ? എങ്കില്‍ നാമും ഒരു തീരുമാനമെടുക്കാം.
\end{malayalam}}
\flushright{\begin{Arabic}
\quranayah[43][80]
\end{Arabic}}
\flushleft{\begin{malayalam}
അല്ല; അവരുടെ കുശുകുശുക്കലുകളും ഗൂഢാലോചനകളുമൊന്നും നാം കേള്‍ക്കുന്നില്ലെന്നാണോ അവര്‍ കരുതുന്നത്. തീര്‍ച്ചയായും നമ്മുടെ ദൂതന്മാര്‍ എല്ലാം എഴുതിയെടുക്കുന്നവരായി അവര്‍ക്കൊപ്പം തന്നെയുണ്ട്.
\end{malayalam}}
\flushright{\begin{Arabic}
\quranayah[43][81]
\end{Arabic}}
\flushleft{\begin{malayalam}
പറയുക: "പരമകാരുണികനായ അല്ലാഹുവിന് ഒരു പുത്രനുണ്ടായിരുന്നെങ്കില്‍ അവനെ പൂജിക്കുന്നവരില്‍ ഒന്നാമന്‍ ഞാനാകുമായിരുന്നു."
\end{malayalam}}
\flushright{\begin{Arabic}
\quranayah[43][82]
\end{Arabic}}
\flushleft{\begin{malayalam}
ആകാശഭൂമികളുടെ സംരക്ഷകനും സിംഹാസനത്തിനുടമയുമായ അല്ലാഹു അവര്‍ പറഞ്ഞുപരത്തുന്നതില്‍ നിന്നെല്ലാം എത്രയോ പരിശുദ്ധനത്രെ.
\end{malayalam}}
\flushright{\begin{Arabic}
\quranayah[43][83]
\end{Arabic}}
\flushleft{\begin{malayalam}
നീ അവരെ അവരുടെ പാട്ടിന് വിട്ടേക്കുക. അവര്‍ അസംബന്ധങ്ങളിലാണ്ട് കളിതമാശകളില്‍ മുഴുകിക്കഴിഞ്ഞുകൊള്ളട്ടെ; അവരോട് വാഗ്ദാനം ചെയ്ത അവരുടെ ആ ദിനവുമായി അവര്‍ കണ്ടുമുട്ടുംവരെ.
\end{malayalam}}
\flushright{\begin{Arabic}
\quranayah[43][84]
\end{Arabic}}
\flushleft{\begin{malayalam}
അവനാണ് ആകാശത്തിലെ ദൈവം. ഭൂമിയിലെ ദൈവവും അവന്‍ തന്നെ. അവന്‍ യുക്തിമാനാണ്. എല്ലാം അറിയുന്നവനും.
\end{malayalam}}
\flushright{\begin{Arabic}
\quranayah[43][85]
\end{Arabic}}
\flushleft{\begin{malayalam}
ആകാശഭൂമികളുടെയും അവയ്ക്കിടയിലുള്ളവയുടെയും ഉടമയായ അല്ലാഹു അനുഗ്രഹപൂര്‍ണനാണ്. അവന് മാത്രമേ അന്ത്യസമയത്തെ സംബന്ധിച്ച അറിവുള്ളൂ. നിങ്ങളെല്ലാം മടങ്ങിച്ചെല്ലേണ്ടത് അവങ്കലേക്കാണ്.
\end{malayalam}}
\flushright{\begin{Arabic}
\quranayah[43][86]
\end{Arabic}}
\flushleft{\begin{malayalam}
അവനെക്കൂടാതെ ഇക്കൂട്ടര്‍ വിളിച്ചുപ്രാര്‍ഥിക്കുന്നവര്‍ ശിപാര്‍ശക്കധികാരമുള്ളവരല്ല; ബോധപൂര്‍വം സത്യസാക്ഷ്യം നിര്‍വഹിച്ചവരൊഴികെ.
\end{malayalam}}
\flushright{\begin{Arabic}
\quranayah[43][87]
\end{Arabic}}
\flushleft{\begin{malayalam}
ആരാണ് അവരെ സൃഷ്ടിച്ചതെന്ന് നീ അവരോട് ചോദിച്ചാല്‍ ഉറപ്പായും അവര്‍ പറയും, അല്ലാഹുവെന്ന്. എന്നിട്ടും എങ്ങനെയാണവര്‍ വഴിതെറ്റിപ്പോകുന്നത്?
\end{malayalam}}
\flushright{\begin{Arabic}
\quranayah[43][88]
\end{Arabic}}
\flushleft{\begin{malayalam}
"എന്റെ നാഥാ, തീര്‍ച്ചയായും ഇക്കൂട്ടര്‍ വിശ്വസിക്കാത്ത ജനതയാണെ"ന്ന പ്രവാചകന്റെ വചനവും അവനറിയുന്നു.
\end{malayalam}}
\flushright{\begin{Arabic}
\quranayah[43][89]
\end{Arabic}}
\flushleft{\begin{malayalam}
അതിനാല്‍ നീ അവരോട് വിട്ടുവീഴ്ച കാണിക്കുക. “നിങ്ങള്‍ക്കു സലാം” എന്നു പറയുക. അടുത്തുതന്നെ അവരെല്ലാം അറിഞ്ഞുകൊള്ളും.
\end{malayalam}}
\chapter{\textmalayalam{ദുഖാന്‍ ( പുക )}}
\begin{Arabic}
\Huge{\centerline{\basmalah}}\end{Arabic}
\flushright{\begin{Arabic}
\quranayah[44][1]
\end{Arabic}}
\flushleft{\begin{malayalam}
ഹാ - മീം.
\end{malayalam}}
\flushright{\begin{Arabic}
\quranayah[44][2]
\end{Arabic}}
\flushleft{\begin{malayalam}
സുവ്യക്തമായ വേദപുസ്തകംതന്നെ സത്യം.
\end{malayalam}}
\flushright{\begin{Arabic}
\quranayah[44][3]
\end{Arabic}}
\flushleft{\begin{malayalam}
അനുഗൃഹീതമായ ഒരു രാവിലാണ് നാം ഇതിറക്കിയത്. തീര്‍ച്ചയായും നാം മുന്നറിയിപ്പ് നല്‍കുന്നവനാണ്.
\end{malayalam}}
\flushright{\begin{Arabic}
\quranayah[44][4]
\end{Arabic}}
\flushleft{\begin{malayalam}
ആ രാവില്‍ യുക്തിപൂര്‍ണമായ സകല സംഗതികളും വേര്‍തിരിച്ച് വിശദീകരിക്കുന്നതാണ്.
\end{malayalam}}
\flushright{\begin{Arabic}
\quranayah[44][5]
\end{Arabic}}
\flushleft{\begin{malayalam}
നമ്മുടെ ഭാഗത്തുനിന്നുള്ള തീരുമാനമാണിത്. നാം ആവശ്യാനുസൃതം ദൂതന്മാരെ നിയോഗിക്കുന്നവനാണ്.
\end{malayalam}}
\flushright{\begin{Arabic}
\quranayah[44][6]
\end{Arabic}}
\flushleft{\begin{malayalam}
നിന്റെ നാഥനില്‍ നിന്നുള്ള അനുഗ്രഹമാണിത്. തീര്‍ച്ചയായും അവന്‍ എല്ലാം കേള്‍ക്കുന്നവനും അറിയുന്നവനുമാണ്.
\end{malayalam}}
\flushright{\begin{Arabic}
\quranayah[44][7]
\end{Arabic}}
\flushleft{\begin{malayalam}
ആകാശഭൂമികളുടെയും അവയ്ക്കിടയിലുള്ളവയുടെയും നാഥനാണവന്‍. നിങ്ങള്‍ അടിയുറച്ചു വിശ്വസിക്കുന്നവരെങ്കില്‍ നിങ്ങള്‍ക്കിതു ബോധ്യമാകും.
\end{malayalam}}
\flushright{\begin{Arabic}
\quranayah[44][8]
\end{Arabic}}
\flushleft{\begin{malayalam}
അവനല്ലാതെ ദൈവമില്ല. അവന്‍ ജീവിപ്പിക്കുകയും മരിപ്പിക്കുകയും ചെയ്യുന്നു. അവന്‍ നിങ്ങളുടെയും നിങ്ങളുടെ പൂര്‍വപിതാക്കളുടെയും നാഥനാണ്.
\end{malayalam}}
\flushright{\begin{Arabic}
\quranayah[44][9]
\end{Arabic}}
\flushleft{\begin{malayalam}
എന്നിട്ടും അവര്‍ സംശയത്തിലകപ്പെട്ട് ആടിക്കളിക്കുകയാണ്.
\end{malayalam}}
\flushright{\begin{Arabic}
\quranayah[44][10]
\end{Arabic}}
\flushleft{\begin{malayalam}
അതിനാല്‍ ആകാശം, തെളിഞ്ഞ പുക വരുത്തുന്ന നാള്‍ വരെ കാത്തിരിക്കുക.
\end{malayalam}}
\flushright{\begin{Arabic}
\quranayah[44][11]
\end{Arabic}}
\flushleft{\begin{malayalam}
അത് മനുഷ്യരാശിയെയാകെ മൂടിപ്പൊതിയും. ഇത് നോവേറിയ ശിക്ഷ തന്നെ.
\end{malayalam}}
\flushright{\begin{Arabic}
\quranayah[44][12]
\end{Arabic}}
\flushleft{\begin{malayalam}
അപ്പോഴവര്‍ പറയും: "ഞങ്ങളുടെ നാഥാ, ഞങ്ങളെ ഈ ശിക്ഷയില്‍നിന്ന് ഒന്നൊഴിവാക്കിത്തരേണമേ, തീര്‍ച്ചയായും ഞങ്ങള്‍ വിശ്വസിച്ചുകൊള്ളാം."
\end{malayalam}}
\flushright{\begin{Arabic}
\quranayah[44][13]
\end{Arabic}}
\flushleft{\begin{malayalam}
ഉദ്ബോധനം എങ്ങനെയാണവര്‍ക്ക് ഉപകരിക്കുക? എല്ലാം വ്യക്തമാക്കിക്കൊടുക്കുന്ന ദൈവദൂതന്‍ അവരുടെ അടുത്തെത്തിയിരുന്നു.
\end{malayalam}}
\flushright{\begin{Arabic}
\quranayah[44][14]
\end{Arabic}}
\flushleft{\begin{malayalam}
അപ്പോള്‍ അവരദ്ദേഹത്തെ അവഗണിച്ച് പിന്തിരിയുകയാണുണ്ടായത്. അവരിങ്ങനെ പറയുകയും ചെയ്തു: "ഇവന്‍ പരിശീലനം ലഭിച്ച ഒരു ഭ്രാന്തന്‍ തന്നെ."
\end{malayalam}}
\flushright{\begin{Arabic}
\quranayah[44][15]
\end{Arabic}}
\flushleft{\begin{malayalam}
തീര്‍ച്ചയായും നാം ശിക്ഷ അല്‍പം ഒഴിവാക്കിത്തരാം. എന്നാലും നിങ്ങള്‍ പഴയപടി എല്ലാം ആവര്‍ത്തിച്ചുകൊണ്ടിരിക്കും.
\end{malayalam}}
\flushright{\begin{Arabic}
\quranayah[44][16]
\end{Arabic}}
\flushleft{\begin{malayalam}
ഒരുനാള്‍ കുതറിമാറാനാവാത്തവിധം കൊടുംപിടുത്തം നടക്കും. തീര്‍ച്ചയായും അന്നാണ് നാം പ്രതികാരം ചെയ്യുക.
\end{malayalam}}
\flushright{\begin{Arabic}
\quranayah[44][17]
\end{Arabic}}
\flushleft{\begin{malayalam}
ഇവര്‍ക്ക് മുമ്പ് ഫറവോന്റെ ജനതയെ നാം പരീക്ഷിച്ചിട്ടുണ്ട്. ആദരണീയനായ ദൈവദൂതന്‍ അവരുടെയടുത്ത് ചെന്നു.
\end{malayalam}}
\flushright{\begin{Arabic}
\quranayah[44][18]
\end{Arabic}}
\flushleft{\begin{malayalam}
അദ്ദേഹം പറഞ്ഞു: "അല്ലാഹുവിന്റെ അടിമകളെ നിങ്ങളെനിക്ക് വിട്ടുതരിക. ഞാന്‍ നിങ്ങളിലേക്കുള്ള വിശ്വസ്തനായ ദൈവദൂതനാണ്.
\end{malayalam}}
\flushright{\begin{Arabic}
\quranayah[44][19]
\end{Arabic}}
\flushleft{\begin{malayalam}
"നിങ്ങള്‍ അല്ലാഹുവിനെതിരെ ധിക്കാരം കാണിക്കരുത്. ഉറപ്പായും ഞാന്‍ വ്യക്തമായ തെളിവുകള്‍ നിങ്ങളുടെ മുന്നില്‍ സമര്‍പ്പിക്കാം.
\end{malayalam}}
\flushright{\begin{Arabic}
\quranayah[44][20]
\end{Arabic}}
\flushleft{\begin{malayalam}
"ഞാനിതാ എന്റെയും നിങ്ങളുടെയും നാഥനില്‍ ശരണം തേടുന്നു; നിങ്ങളുടെ കല്ലേറില്‍നിന്ന് രക്ഷകിട്ടാന്‍.
\end{malayalam}}
\flushright{\begin{Arabic}
\quranayah[44][21]
\end{Arabic}}
\flushleft{\begin{malayalam}
"നിങ്ങള്‍ക്കെന്നെ വിശ്വാസമില്ലെങ്കില്‍ എന്നില്‍നിന്നു വിട്ടകന്നുപോവുക."
\end{malayalam}}
\flushright{\begin{Arabic}
\quranayah[44][22]
\end{Arabic}}
\flushleft{\begin{malayalam}
ഒടുവില്‍ അദ്ദേഹം തന്റെ നാഥനെ വിളിച്ചു പറഞ്ഞു: “ഈ ജനം കുറ്റവാളികളാകുന്നു.”
\end{malayalam}}
\flushright{\begin{Arabic}
\quranayah[44][23]
\end{Arabic}}
\flushleft{\begin{malayalam}
അപ്പോള്‍ അല്ലാഹു പറഞ്ഞു: "എന്റെ ദാസന്മാരെയും കൊണ്ട് രാത്രി തന്നെ പുറപ്പെടുക. അവര്‍ നിങ്ങളെ പിന്തുടരുന്നുണ്ട്."
\end{malayalam}}
\flushright{\begin{Arabic}
\quranayah[44][24]
\end{Arabic}}
\flushleft{\begin{malayalam}
സമുദ്രത്തെ അത് പിളര്‍ന്ന അവസ്ഥയില്‍തന്നെ വിട്ടേക്കുക. സംശയം വേണ്ട; അവര്‍ മുങ്ങിയൊടുങ്ങാന്‍ പോകുന്ന സൈന്യമാണ്.
\end{malayalam}}
\flushright{\begin{Arabic}
\quranayah[44][25]
\end{Arabic}}
\flushleft{\begin{malayalam}
എത്രയെത്ര ആരാമങ്ങളും അരുവികളുമാണവര്‍ വിട്ടേച്ചുപോയത്!
\end{malayalam}}
\flushright{\begin{Arabic}
\quranayah[44][26]
\end{Arabic}}
\flushleft{\begin{malayalam}
കൃഷിയിടങ്ങളും മാന്യമായ മണിമേടകളും!
\end{malayalam}}
\flushright{\begin{Arabic}
\quranayah[44][27]
\end{Arabic}}
\flushleft{\begin{malayalam}
അവര്‍ ആനന്ദത്തോടെ അനുഭവിച്ചുപോന്ന എന്തെല്ലാം സൌഭാഗ്യങ്ങള്‍!
\end{malayalam}}
\flushright{\begin{Arabic}
\quranayah[44][28]
\end{Arabic}}
\flushleft{\begin{malayalam}
അങ്ങനെയായിരുന്നു അവയുടെ ഒടുക്കം. അതൊക്കെയും നാം മറ്റൊരു ജനതക്ക് അവകാശപ്പെടുത്തിക്കൊടുത്തു.
\end{malayalam}}
\flushright{\begin{Arabic}
\quranayah[44][29]
\end{Arabic}}
\flushleft{\begin{malayalam}
അപ്പോള്‍ അവര്‍ക്കുവേണ്ടി ആകാശമോ ഭൂമിയോ കണ്ണീര്‍ വാര്‍ത്തില്ല. അവര്‍ക്കൊട്ടും അവസരം നല്‍കിയതുമില്ല.
\end{malayalam}}
\flushright{\begin{Arabic}
\quranayah[44][30]
\end{Arabic}}
\flushleft{\begin{malayalam}
ഇസ്രയേല്‍ മക്കളെ നാം നിന്ദ്യമായ ശിക്ഷയില്‍നിന്ന് രക്ഷിച്ചു.
\end{malayalam}}
\flushright{\begin{Arabic}
\quranayah[44][31]
\end{Arabic}}
\flushleft{\begin{malayalam}
ഫറവോനില്‍ നിന്ന്. അവന്‍ കടുത്ത അഹങ്കാരിയായിരുന്നു; അങ്ങേയറ്റം അതിരുകടന്നവനും.
\end{malayalam}}
\flushright{\begin{Arabic}
\quranayah[44][32]
\end{Arabic}}
\flushleft{\begin{malayalam}
അവരുടെ നിജസ്ഥിതിയറിഞ്ഞു കൊണ്ടുതന്നെ നാമവരെ ലോകത്താരെക്കാളും പ്രമുഖരായി തെരഞ്ഞെടുത്തു.
\end{malayalam}}
\flushright{\begin{Arabic}
\quranayah[44][33]
\end{Arabic}}
\flushleft{\begin{malayalam}
പ്രകടമായ പരീക്ഷണമുള്‍ക്കൊള്ളുന്ന പല ദൃഷ്ടാന്തങ്ങളും അവര്‍ക്ക് നല്‍കി.
\end{malayalam}}
\flushright{\begin{Arabic}
\quranayah[44][34]
\end{Arabic}}
\flushleft{\begin{malayalam}
ഇക്കൂട്ടരിതാ പറയുന്നു:
\end{malayalam}}
\flushright{\begin{Arabic}
\quranayah[44][35]
\end{Arabic}}
\flushleft{\begin{malayalam}
"നമുക്ക് ഈ ഒന്നാമത്തെ മരണമല്ലാതൊന്നുമില്ല. നാമിനി ഉയിര്‍ത്തെഴുന്നേല്‍പിക്കപ്പെടുകയുമില്ല.
\end{malayalam}}
\flushright{\begin{Arabic}
\quranayah[44][36]
\end{Arabic}}
\flushleft{\begin{malayalam}
"അങ്ങനെ സംഭവിക്കുമെങ്കില്‍ ഞങ്ങളുടെ പൂര്‍വപിതാക്കളെയിങ്ങ് ഉയിര്‍ത്തെഴുന്നേല്‍പിച്ചുകൊണ്ടുവരിക. നിങ്ങള്‍ സത്യവാന്മാരെങ്കില്‍?"
\end{malayalam}}
\flushright{\begin{Arabic}
\quranayah[44][37]
\end{Arabic}}
\flushleft{\begin{malayalam}
ഇവരാണോ കൂടുതല്‍ വമ്പന്മാര്‍; അതോ തുബ്ബഇന്റെ ജനതയും അവര്‍ക്കു മുമ്പുള്ളവരുമോ? അവരെയൊക്കെ നാം നശിപ്പിച്ചു. കാരണം അവര്‍ കുറ്റവാളികളായിരുന്നു.
\end{malayalam}}
\flushright{\begin{Arabic}
\quranayah[44][38]
\end{Arabic}}
\flushleft{\begin{malayalam}
നാം ആകാശഭൂമികളെയും അവയ്ക്കിടയിലുള്ളവയെയും വെറും വിനോദത്തിനു വേണ്ടി സൃഷ്ടിച്ചതല്ല.
\end{malayalam}}
\flushright{\begin{Arabic}
\quranayah[44][39]
\end{Arabic}}
\flushleft{\begin{malayalam}
തികഞ്ഞ യാഥാര്‍ഥ്യത്തോടെയല്ലാതെ നാമവയെ ഉണ്ടാക്കിയിട്ടില്ല. എന്നാല്‍ ഇവരിലേറെ പേരും ഇതൊന്നുമറിയുന്നില്ല.
\end{malayalam}}
\flushright{\begin{Arabic}
\quranayah[44][40]
\end{Arabic}}
\flushleft{\begin{malayalam}
ആ വിധിത്തീര്‍പ്പിന്റെ നാളിലാണ് അവരുടെയൊക്കെ ഉയിര്‍ത്തെഴുന്നേല്‍പുണ്ടാവുന്ന നിശ്ചിതസമയം.
\end{malayalam}}
\flushright{\begin{Arabic}
\quranayah[44][41]
\end{Arabic}}
\flushleft{\begin{malayalam}
അന്നാളില്‍ ഒരു കൂട്ടുകാരന്നും തന്റെ ഉറ്റവനെ ഒട്ടും ഉപകരിക്കുകയില്ല. ആര്‍ക്കും ഒരുവിധ സഹായവും ആരില്‍നിന്നും കിട്ടുകയുമില്ല.
\end{malayalam}}
\flushright{\begin{Arabic}
\quranayah[44][42]
\end{Arabic}}
\flushleft{\begin{malayalam}
അല്ലാഹു അനുഗ്രഹിച്ചവര്‍ക്കൊഴികെ. തീര്‍ച്ചയായും അവന്‍ പ്രതാപിയാണ്; പരമദയാലുവും.
\end{malayalam}}
\flushright{\begin{Arabic}
\quranayah[44][43]
\end{Arabic}}
\flushleft{\begin{malayalam}
നിശ്ചയമായും “സഖൂം” വൃക്ഷമാണ്;
\end{malayalam}}
\flushright{\begin{Arabic}
\quranayah[44][44]
\end{Arabic}}
\flushleft{\begin{malayalam}
പാപികള്‍ക്കാഹാരം.
\end{malayalam}}
\flushright{\begin{Arabic}
\quranayah[44][45]
\end{Arabic}}
\flushleft{\begin{malayalam}
ഉരുകിയലോഹം പോലെയാണത്. വയറ്റില്‍ കിടന്ന് അത് തിളച്ചുമറിയും.
\end{malayalam}}
\flushright{\begin{Arabic}
\quranayah[44][46]
\end{Arabic}}
\flushleft{\begin{malayalam}
ചുടുവെള്ളം തിളയ്ക്കുംപോലെ.
\end{malayalam}}
\flushright{\begin{Arabic}
\quranayah[44][47]
\end{Arabic}}
\flushleft{\begin{malayalam}
“നിങ്ങളവനെ പിടിക്കൂ. എന്നിട്ട് നരകത്തിന്റെ മധ്യത്തിലേക്ക് വലിച്ചിഴച്ചുകൊണ്ടു പോകൂ” എന്ന് കല്‍പനയുണ്ടാകും.
\end{malayalam}}
\flushright{\begin{Arabic}
\quranayah[44][48]
\end{Arabic}}
\flushleft{\begin{malayalam}
പിന്നെയവന്റെ തലക്കു മുകളില്‍ തിളച്ചവെള്ളം കൊണ്ടുപോയി ഒഴിക്കാനാവശ്യപ്പെടും.
\end{malayalam}}
\flushright{\begin{Arabic}
\quranayah[44][49]
\end{Arabic}}
\flushleft{\begin{malayalam}
"ഇത് ആസ്വദിച്ചുകൊള്ളുക. തീര്‍ച്ചയായും നീ ഏറെ പ്രതാപിയും ബഹുമാന്യനുമാണല്ലോ!
\end{malayalam}}
\flushright{\begin{Arabic}
\quranayah[44][50]
\end{Arabic}}
\flushleft{\begin{malayalam}
"നീ സംശയിച്ചുകൊണ്ടിരുന്ന അക്കാര്യമില്ലേ; അതു തന്നെയാണിത്; തീര്‍ച്ച."
\end{malayalam}}
\flushright{\begin{Arabic}
\quranayah[44][51]
\end{Arabic}}
\flushleft{\begin{malayalam}
എന്നാല്‍ ഭക്തിപുലര്‍ത്തിയവര്‍ ഭീതിയേതുമില്ലാത്ത ഒരിടത്തായിരിക്കും.
\end{malayalam}}
\flushright{\begin{Arabic}
\quranayah[44][52]
\end{Arabic}}
\flushleft{\begin{malayalam}
ആരാമങ്ങളിലും അരുവികളിലും!
\end{malayalam}}
\flushright{\begin{Arabic}
\quranayah[44][53]
\end{Arabic}}
\flushleft{\begin{malayalam}
അവര്‍ അഴകാര്‍ന്ന പട്ടിന്‍ വസ്ത്രവും കസവിന്‍ തുണിയും അണിയും. അവര്‍ അഭിമുഖമായാണിരിക്കുക.
\end{malayalam}}
\flushright{\begin{Arabic}
\quranayah[44][54]
\end{Arabic}}
\flushleft{\begin{malayalam}
ഇതാണവരുടെ പ്രഭവാവസ്ഥ. വിശാലാക്ഷികളായ തരുണീമണികളെ നാമവര്‍ക്ക് ഇണകളായി കൊടുക്കും.
\end{malayalam}}
\flushright{\begin{Arabic}
\quranayah[44][55]
\end{Arabic}}
\flushleft{\begin{malayalam}
അവരവിടെ സ്വസ്ഥതയോടെ പലവിധ പഴങ്ങളും ആവശ്യപ്പെട്ടുകൊണ്ടിരിക്കും.
\end{malayalam}}
\flushright{\begin{Arabic}
\quranayah[44][56]
\end{Arabic}}
\flushleft{\begin{malayalam}
ആദ്യത്തെ മരണമല്ലാതെ മറ്റൊരു മരണം അവര്‍ക്കവിടെ അനുഭവിക്കേണ്ടിവരില്ല. അല്ലാഹു അവരെ നരകശിക്ഷയില്‍നിന്ന് രക്ഷിച്ചിരിക്കുന്നു.
\end{malayalam}}
\flushright{\begin{Arabic}
\quranayah[44][57]
\end{Arabic}}
\flushleft{\begin{malayalam}
നിന്റെ നാഥനില്‍ നിന്നുള്ള അനുഗ്രഹമാണത്. അതു തന്നെയാണ് അതിമഹത്തായ വിജയം!
\end{malayalam}}
\flushright{\begin{Arabic}
\quranayah[44][58]
\end{Arabic}}
\flushleft{\begin{malayalam}
നിനക്കു നിന്റെ ഭാഷയില്‍ ഈ വേദപുസ്തകത്തെ നാം വളരെ ലളിതമാക്കിത്തന്നിരിക്കുന്നു. ജനം ചിന്തിച്ചറിയാന്‍.
\end{malayalam}}
\flushright{\begin{Arabic}
\quranayah[44][59]
\end{Arabic}}
\flushleft{\begin{malayalam}
അതിനാല്‍ നീ കാത്തിരിക്കുക. അവരും കാത്തിരിക്കുന്നുണ്ട്.
\end{malayalam}}
\chapter{\textmalayalam{ജാഥിയ ( മുട്ടുകുത്തുന്നവര്‍ )}}
\begin{Arabic}
\Huge{\centerline{\basmalah}}\end{Arabic}
\flushright{\begin{Arabic}
\quranayah[45][1]
\end{Arabic}}
\flushleft{\begin{malayalam}
ഹാ-മീം.
\end{malayalam}}
\flushright{\begin{Arabic}
\quranayah[45][2]
\end{Arabic}}
\flushleft{\begin{malayalam}
ഈ വേദപുസ്തകത്തിന്റെ അവതരണം പ്രതാപിയും യുക്തിമാനുമായ അല്ലാഹുവില്‍ നിന്നാണ്.
\end{malayalam}}
\flushright{\begin{Arabic}
\quranayah[45][3]
\end{Arabic}}
\flushleft{\begin{malayalam}
തീര്‍ച്ചയായും ആകാശഭൂമികളില്‍ സത്യവിശ്വാസികള്‍ക്ക് എണ്ണമറ്റ തെളിവുകളുണ്ട്.
\end{malayalam}}
\flushright{\begin{Arabic}
\quranayah[45][4]
\end{Arabic}}
\flushleft{\begin{malayalam}
നിങ്ങളുടെ സൃഷ്ടിപ്പിലും അല്ലാഹു ജീവജാലങ്ങളെ ഭൂമിയില്‍ പരത്തിയതിലും, അടിയുറച്ച വിശ്വാസമുള്ള ജനത്തിന് അളവറ്റ അടയാളങ്ങളുണ്ട്.
\end{malayalam}}
\flushright{\begin{Arabic}
\quranayah[45][5]
\end{Arabic}}
\flushleft{\begin{malayalam}
രാപ്പകലുകള്‍ മാറിമാറി വരുന്നതില്‍; അല്ലാഹു മാനത്തുനിന്ന് ജീവിതവിഭവം ഇറക്കിത്തരുന്നതില്‍; അതു വഴി ചത്ത ഭൂമിയെ ചൈതന്യവത്താക്കുന്നതില്‍; കാറ്റുകളുടെ ഗതി നിയന്ത്രിക്കുന്നതില്‍; എല്ലാറ്റിലും ചിന്തിക്കുന്ന ജനത്തിന് ഒട്ടേറെ അടയാളങ്ങളുണ്ട്.
\end{malayalam}}
\flushright{\begin{Arabic}
\quranayah[45][6]
\end{Arabic}}
\flushleft{\begin{malayalam}
അല്ലാഹുവിന്റെ വചനങ്ങളാണിവ. നാമവയെ നിനക്കു ക്രമാനുസൃതം ഓതിത്തരുന്നു. അല്ലാഹുവിലും അവന്റെ വചനങ്ങളിലുമല്ലാതെ മറ്റേതു വൃത്താന്തത്തിലാണ് ഈ ജനം ഇനി വിശ്വസിക്കുക.
\end{malayalam}}
\flushright{\begin{Arabic}
\quranayah[45][7]
\end{Arabic}}
\flushleft{\begin{malayalam}
പെരുംനുണ കെട്ടിപ്പറയുന്ന പാപികള്‍ക്കൊക്കെയും കൊടിയനാശം!
\end{malayalam}}
\flushright{\begin{Arabic}
\quranayah[45][8]
\end{Arabic}}
\flushleft{\begin{malayalam}
അവന്റെ മുമ്പില്‍ അല്ലാഹുവിന്റെ വചനങ്ങള്‍ വായിക്കപ്പെടുന്നു. അവനത് കേള്‍ക്കുന്നു. എന്നിട്ടുമത് കേട്ടിട്ടില്ലെന്ന മട്ടില്‍ അഹന്ത നടിച്ച് പഴയപോലെത്തന്നെ സത്യനിഷേധത്തിലുറച്ചു നില്‍ക്കുന്നു. അതിനാല്‍ അവനെ നോവേറുന്ന ശിക്ഷയെ സംബന്ധിച്ച “സുവാര്‍ത്ത” അറിയിക്കുക.
\end{malayalam}}
\flushright{\begin{Arabic}
\quranayah[45][9]
\end{Arabic}}
\flushleft{\begin{malayalam}
നമ്മുടെ വചനങ്ങളില്‍ വല്ലതും അവന്‍ അറിഞ്ഞാല്‍ ഉടനെ അവനതിനെ പുച്ഛിക്കുന്നു. അത്തരക്കാര്‍ക്ക് ഉറപ്പായും ഏറ്റം നിന്ദ്യമായ ശിക്ഷയുണ്ട്.
\end{malayalam}}
\flushright{\begin{Arabic}
\quranayah[45][10]
\end{Arabic}}
\flushleft{\begin{malayalam}
അവരെ പിന്തുടരുന്നത് കത്തിപ്പടരുന്ന തിയ്യാണ്. അവര്‍ നേടിയതൊന്നും അവര്‍ക്ക് ഉപകരിക്കുകയില്ല. അല്ലാഹുവെക്കൂടാതെ അവര്‍ കൊണ്ടുനടക്കുന്ന രക്ഷാധികാരികളാരും അവര്‍ക്കൊരിക്കലും പ്രയോജനപ്പെടുകയില്ല. അവര്‍ക്ക് കടുത്ത ശിക്ഷയുണ്ട്.
\end{malayalam}}
\flushright{\begin{Arabic}
\quranayah[45][11]
\end{Arabic}}
\flushleft{\begin{malayalam}
ഈ ഖുര്‍ആന്‍ വഴികാട്ടിയാണ്. തങ്ങളുടെ നാഥന്റെ വചനങ്ങളെ തള്ളിപ്പറയുന്നവര്‍ക്ക് നോവുറ്റ ഹീനമായ ശിക്ഷയുണ്ട്.
\end{malayalam}}
\flushright{\begin{Arabic}
\quranayah[45][12]
\end{Arabic}}
\flushleft{\begin{malayalam}
അല്ലാഹുവാണ് നിങ്ങള്‍ക്ക് കടലിനെ കീഴ്പ്പെടുത്തിത്തന്നത്. അവന്റെ കല്‍പനപ്രകാരം അതില്‍ കപ്പലോട്ടാന്‍; നിങ്ങളവന്റെ മഹത്തായ അനുഗ്രഹങ്ങള്‍ പരതാനും. നിങ്ങള്‍ നന്ദിയുള്ളവരായെങ്കിലോ.
\end{malayalam}}
\flushright{\begin{Arabic}
\quranayah[45][13]
\end{Arabic}}
\flushleft{\begin{malayalam}
ആകാശഭൂമികളിലുള്ളതൊക്കെയും അവന്‍ നിങ്ങള്‍ക്ക് അധീനപ്പെടുത്തിത്തന്നിരിക്കുന്നു. എല്ലാം അവനില്‍ നിന്നുള്ളതാണ്. തീര്‍ച്ചയായും ചിന്തിക്കുന്ന ജനത്തിന് ഇതിലൊക്കെയും ധാരാളം തെളിവുകളുണ്ട്.
\end{malayalam}}
\flushright{\begin{Arabic}
\quranayah[45][14]
\end{Arabic}}
\flushleft{\begin{malayalam}
സത്യവിശ്വാസികളോടു പറയൂ: അല്ലാഹുവിന്റെ ശിക്ഷയുടെ നാളുകളെ പ്രതീക്ഷിക്കാത്ത സത്യനിഷേധികളോട് അവര്‍ വിട്ടുവീഴ്ച കാണിക്കട്ടെ. ഓരോ ജനതക്കും അവര്‍ നേടിയെടുത്തതിന്റെ ഫലം നല്‍കാന്‍ അല്ലാഹുവിന് അവസരമുണ്ടാകട്ടെ.
\end{malayalam}}
\flushright{\begin{Arabic}
\quranayah[45][15]
\end{Arabic}}
\flushleft{\begin{malayalam}
ആരെങ്കിലും നന്മ ചെയ്താല്‍ അതിന്റെ ഗുണം അവനുതന്നെയാണ്. വല്ലവനും തിന്മ ചെയ്താല്‍ അതിന്റെ ദോഷവും അവനുതന്നെ. പിന്നെ നിങ്ങളൊക്കെ മടക്കപ്പെടുക നിങ്ങളുടെ നാഥങ്കലേക്കാണ്.
\end{malayalam}}
\flushright{\begin{Arabic}
\quranayah[45][16]
\end{Arabic}}
\flushleft{\begin{malayalam}
തീര്‍ച്ചയായും നാം ഇസ്രയേല്‍ മക്കള്‍ക്ക് വേദപുസ്തകമേകി. ആധിപത്യവും പ്രവാചകത്വവും നല്‍കി. ഉത്തമ വസ്തുക്കളില്‍ നിന്ന് അന്നം നല്‍കി. ലോകത്ത് നാമവരെ മറ്റാരെക്കാളും ശ്രേഷ്ഠരാക്കുകയും ചെയ്തു.
\end{malayalam}}
\flushright{\begin{Arabic}
\quranayah[45][17]
\end{Arabic}}
\flushleft{\begin{malayalam}
അവര്‍ക്കു നാം എല്ലാ കാര്യങ്ങളിലും വ്യക്തമായ പ്രമാണങ്ങള്‍ നല്‍കി. വിജ്ഞാനം വന്നെത്തിയ ശേഷം മാത്രമാണവര്‍ ഭിന്നിച്ചത്. അവര്‍ക്കിടയിലെ കിടമത്സരം കാരണമായാണത്. അവര്‍ക്കിടയില്‍ അഭിപ്രായ ഭിന്നതയുള്ള കാര്യങ്ങളില്‍ നിന്റെ നാഥന്‍ ഉയിര്‍ത്തെഴുന്നേല്‍പുനാളില്‍ വിധിത്തീര്‍പ്പ് കല്‍പിക്കുന്നതാണ്.
\end{malayalam}}
\flushright{\begin{Arabic}
\quranayah[45][18]
\end{Arabic}}
\flushleft{\begin{malayalam}
പിന്നീട് നിന്നെ നാം ഇക്കാര്യത്തില്‍ വ്യക്തമായ നിയമവ്യവസ്ഥയിലാക്കിയിരിക്കുന്നു. അതിനാല്‍ നീ ആ മാര്‍ഗം പിന്തുടരുക. വിവരമില്ലാത്തവരുടെ തന്നിഷ്ടങ്ങളെ പിന്‍പറ്റരുത്.
\end{malayalam}}
\flushright{\begin{Arabic}
\quranayah[45][19]
\end{Arabic}}
\flushleft{\begin{malayalam}
അല്ലാഹുവില്‍ നിന്നുള്ള ഒരു കാര്യത്തിലും നിനക്കൊരുപകാരവും ചെയ്യാന്‍ അവര്‍ക്കാവില്ല. തീര്‍ച്ചയായും അക്രമികള്‍ പരസ്പരം സഹായികളാണ്. എന്നാല്‍ ഭക്തന്മാരുടെ രക്ഷാധികാരി അല്ലാഹുവാണ്.
\end{malayalam}}
\flushright{\begin{Arabic}
\quranayah[45][20]
\end{Arabic}}
\flushleft{\begin{malayalam}
ഇത് മുഴുവന്‍ മനുഷ്യര്‍ക്കും ഉള്‍ക്കാഴ്ച നല്‍കുന്നതാണ്. വിശ്വസിക്കുന്ന ജനത്തിന് വഴികാട്ടിയാണ്. മഹത്തായ അനുഗ്രഹവും.
\end{malayalam}}
\flushright{\begin{Arabic}
\quranayah[45][21]
\end{Arabic}}
\flushleft{\begin{malayalam}
ചീത്ത വൃത്തികള്‍ ചെയ്തുകൂട്ടിയവര്‍ കരുതുന്നോ, അവരെ നാം സത്യവിശ്വാസം സ്വീകരിക്കുകയും സല്‍ക്കര്‍മങ്ങള്‍ പ്രവര്‍ത്തിക്കുകയും ചെയ്തവരെപ്പോലെ ആക്കുമെന്ന്. അഥവാ, അവരുടെ ജീവിതവും മരണവും ഒരേപോലെയാകുമെന്ന്. അവരുടെ വിധിത്തീര്‍പ്പ് വളരെ ചീത്ത തന്നെ.
\end{malayalam}}
\flushright{\begin{Arabic}
\quranayah[45][22]
\end{Arabic}}
\flushleft{\begin{malayalam}
അല്ലാഹു ആകാശഭൂമികളെ യാഥാര്‍ഥ്യനിഷ്ഠമായാണ് സൃഷ്ടിച്ചത്. ഓരോരുത്തര്‍ക്കും താന്‍ പ്രവര്‍ത്തിച്ചതിന്റെ പ്രതിഫലം നല്‍കാനാണിത്. ആരോടും ഒട്ടും അനീതി ഉണ്ടാവുകയില്ല.
\end{malayalam}}
\flushright{\begin{Arabic}
\quranayah[45][23]
\end{Arabic}}
\flushleft{\begin{malayalam}
തന്റെ ദേഹേച്ഛയെ ദൈവമാക്കിയവനെ നീ കണ്ടോ? അല്ലാഹു അവനെ ബോധപൂര്‍വം വഴികേടിലാക്കിയിരിക്കുന്നു. അവന്റെ കാതിനും മനസ്സിനും മുദ്രവെച്ചിരിക്കുന്നു. കണ്ണുകള്‍ക്ക് മൂടിയിട്ടിരിക്കുന്നു. അപ്പോള്‍ അല്ലാഹുവെ കൂടാതെ അവനെ നേര്‍വഴിയിലാക്കാന്‍ ആരുണ്ട്? എന്നിട്ടും നിങ്ങളൊട്ടും ചിന്തിച്ചറിയുന്നില്ലേ?
\end{malayalam}}
\flushright{\begin{Arabic}
\quranayah[45][24]
\end{Arabic}}
\flushleft{\begin{malayalam}
അവര്‍ പറഞ്ഞു: "നമ്മുടെ ഈ ലോകജീവിതമല്ലാതെ ജീവിതമില്ല. നാം മരിക്കുന്നു. ജീവിക്കുന്നു. കാലം മാത്രമാണ് നമ്മെ നശിപ്പിക്കുന്നത്." യഥാര്‍ഥത്തില്‍ അവര്‍ക്ക് അതേപ്പറ്റി ഒരു വിവരവുമില്ല. അവര്‍ വെറുതെ ഊഹിച്ചുപറയുകയാണ്.
\end{malayalam}}
\flushright{\begin{Arabic}
\quranayah[45][25]
\end{Arabic}}
\flushleft{\begin{malayalam}
നമ്മുടെ വചനങ്ങള്‍ അവരെ വ്യക്തമായി വായിച്ചുകേള്‍പ്പിച്ചാല്‍ അവര്‍ക്കു ന്യായവാദമായി പറയാനുള്ളത് ഇതുമാത്രമായിരിക്കും: "നിങ്ങള്‍ ഞങ്ങളുടെ പിതാക്കളെ ജീവിപ്പിച്ചുകൊണ്ടുവരിക; നിങ്ങള്‍ സത്യവാദികളെങ്കില്‍!"
\end{malayalam}}
\flushright{\begin{Arabic}
\quranayah[45][26]
\end{Arabic}}
\flushleft{\begin{malayalam}
പറയുക: അല്ലാഹുവാണ് നിങ്ങളെ ജീവിപ്പിക്കുന്നത്. പിന്നെ നിങ്ങളെയവന്‍ മരിപ്പിക്കും. പിന്നീട് ഉയിര്‍ത്തെഴുന്നേല്‍പുനാളില്‍ നിങ്ങളെയവന്‍ ഒരുമിച്ചുകൂട്ടും. ഇതിലൊട്ടും സംശയമില്ല. എന്നാല്‍ മനുഷ്യരിലേറെ പേരും ഇതറിയുന്നില്ല.
\end{malayalam}}
\flushright{\begin{Arabic}
\quranayah[45][27]
\end{Arabic}}
\flushleft{\begin{malayalam}
ആകാശഭൂമികളുടെ ആധിപത്യം അല്ലാഹുവിനാണ്. ആ അന്ത്യസമയം വരുംനാളില്‍ അസത്യവാദികള്‍ കൊടിയ നഷ്ടത്തിലായിരിക്കും.
\end{malayalam}}
\flushright{\begin{Arabic}
\quranayah[45][28]
\end{Arabic}}
\flushleft{\begin{malayalam}
അന്ന് ഓരോ സമുദായവും മുട്ടുകുത്തി വീണുകിടക്കുന്നതായി നിനക്കു കാണാം. എല്ലാ ഓരോ സമുദായത്തെയും തങ്ങളുടെ കര്‍മരേഖ നോക്കാന്‍ വിളിക്കും. നിങ്ങള്‍ പ്രവര്‍ത്തിച്ചുകൊണ്ടിരുന്നതിന് ഇന്ന് നിങ്ങള്‍ക്ക് പ്രതിഫലം നല്‍കും.
\end{malayalam}}
\flushright{\begin{Arabic}
\quranayah[45][29]
\end{Arabic}}
\flushleft{\begin{malayalam}
നമ്മുടെ കര്‍മരേഖ ഇതാ! ഇത് നിങ്ങള്‍ക്കെതിരെ സത്യം തുറന്നുപറയും. നിങ്ങള്‍ ചെയ്തുകൊണ്ടിരുന്നതെല്ലാം നാം കൃത്യമായി എഴുതിയെടുപ്പിക്കുന്നുണ്ടായിരുന്നു.
\end{malayalam}}
\flushright{\begin{Arabic}
\quranayah[45][30]
\end{Arabic}}
\flushleft{\begin{malayalam}
സത്യവിശ്വാസം സ്വീകരിക്കുകയും സല്‍ക്കര്‍മങ്ങള്‍ പ്രവര്‍ത്തിക്കുകയും ചെയ്തവരെ അവരുടെ നാഥന്‍ തന്റെ കാരുണ്യവലയത്തില്‍ പ്രവേശിപ്പിക്കും. വ്യക്തമായ വിജയവും അതുതന്നെ.
\end{malayalam}}
\flushright{\begin{Arabic}
\quranayah[45][31]
\end{Arabic}}
\flushleft{\begin{malayalam}
മറിച്ച് സത്യത്തെ തള്ളിപ്പറഞ്ഞവരോ; അവരോടിങ്ങനെ പറയും: "എന്റെ വചനങ്ങള്‍ നിങ്ങള്‍ക്ക് വ്യക്തമായി ഓതിക്കേള്‍പ്പിച്ചുതന്നിരുന്നില്ലേ? എന്നിട്ടും നിങ്ങള്‍ അഹങ്കരിച്ചു. നിങ്ങള്‍ കുറ്റവാളികളായ ജനമായിത്തീര്‍ന്നു."
\end{malayalam}}
\flushright{\begin{Arabic}
\quranayah[45][32]
\end{Arabic}}
\flushleft{\begin{malayalam}
"തീര്‍ച്ചയായും അല്ലാഹുവിന്റെ വാഗ്ദാനം സത്യം തന്നെ. ആ അന്ത്യദിനത്തിന്റെ കാര്യത്തിലൊട്ടും സംശയമില്ല" എന്ന് പറഞ്ഞാല്‍ നിങ്ങള്‍ പറയും: "ഞങ്ങള്‍ക്കറിയില്ലല്ലോ; എന്താണ് ഈ അന്ത്യദിനമെന്ന്. ഞങ്ങള്‍ക്ക് ഊഹം മാത്രമേയുള്ളൂ. ഞങ്ങള്‍ക്ക് ഇക്കാര്യത്തിലൊരുറപ്പുമില്ല."
\end{malayalam}}
\flushright{\begin{Arabic}
\quranayah[45][33]
\end{Arabic}}
\flushleft{\begin{malayalam}
അവര്‍ ചെയ്തുകൊണ്ടിരുന്ന ദുര്‍വൃത്തികളുടെ ദുരന്തഫലം അവര്‍ക്ക് വെളിപ്പെടുകതന്നെ ചെയ്യും. അവര്‍ പരിഹസിച്ച് അവഗണിച്ച ശിക്ഷ അവരെ വലയം ചെയ്യും.
\end{malayalam}}
\flushright{\begin{Arabic}
\quranayah[45][34]
\end{Arabic}}
\flushleft{\begin{malayalam}
അപ്പോള്‍ അവരോടു പറയും: "ഈ ദിനത്തെ അഭിമുഖീകരിക്കേണ്ടിവരുമെന്ന കാര്യം നിങ്ങള്‍ മറന്നിരുന്നപോലെത്തന്നെ നിങ്ങളെ നാമുമിന്ന് മറന്നിരിക്കുന്നു. നിങ്ങളുടെ താവളം ആളിക്കത്തുന്ന നരകത്തീയാണ്. നിങ്ങളെ സഹായിക്കാന്‍ ആരുമുണ്ടാവുകയില്ല.
\end{malayalam}}
\flushright{\begin{Arabic}
\quranayah[45][35]
\end{Arabic}}
\flushleft{\begin{malayalam}
"അല്ലാഹുവിന്റെ വചനങ്ങളെ നിങ്ങള്‍ പുച്ഛിച്ചുതള്ളി. ഐഹിക ജീവിതം നിങ്ങളെ വഞ്ചിച്ചു. അതിനാലാണിങ്ങനെ സംഭവിച്ചത്." ഇന്ന് അവരെ നരകത്തീയില്‍ നിന്ന് പുറത്തുചാടാനനുവദിക്കുകയില്ല. അവരോട് പ്രായശ്ചിത്തത്തിന് ആവശ്യപ്പെടുകയുമില്ല.
\end{malayalam}}
\flushright{\begin{Arabic}
\quranayah[45][36]
\end{Arabic}}
\flushleft{\begin{malayalam}
അതിനാല്‍ അല്ലാഹുവിന് സ്തുതി. അവന്‍ ആകാശങ്ങളുടെ നാഥനാണ്. ഭൂമിയുടെയും നാഥനാണ്. സര്‍വലോക സംരക്ഷകനും.
\end{malayalam}}
\flushright{\begin{Arabic}
\quranayah[45][37]
\end{Arabic}}
\flushleft{\begin{malayalam}
ഉന്നതങ്ങളില്‍ അവനാണ് മഹത്വം. ഭൂമിയിലും ഔന്നത്യം അവന്നുതന്നെ. ഏറെ പ്രതാപിയാണ് അവന്‍. അതീവ യുക്തിമാനും.
\end{malayalam}}
\chapter{\textmalayalam{അഹ്ഖാഫ്}}
\begin{Arabic}
\Huge{\centerline{\basmalah}}\end{Arabic}
\flushright{\begin{Arabic}
\quranayah[46][1]
\end{Arabic}}
\flushleft{\begin{malayalam}
ഹാ-മീം
\end{malayalam}}
\flushright{\begin{Arabic}
\quranayah[46][2]
\end{Arabic}}
\flushleft{\begin{malayalam}
ഈ വേദ പുസ്തകത്തിന്റെ അവതരണം പ്രതാപിയും യുക്തിമാനുമായ അല്ലാഹുവില്‍നിന്നാകുന്നു.
\end{malayalam}}
\flushright{\begin{Arabic}
\quranayah[46][3]
\end{Arabic}}
\flushleft{\begin{malayalam}
ആകാശ ഭൂമികളെയും അവയ്ക്കിടയിലുള്ളവയെയും യാഥാര്‍ഥ്യ നിഷ്ഠമായും കാലാവധി നിര്‍ണയിച്ചുമല്ലാതെ നാം സൃഷ്ടിച്ചിട്ടില്ല. എന്നാല്‍ സത്യനിഷേധികള്‍ തങ്ങള്‍ക്കു നല്‍കപ്പെട്ട താക്കീതുകളെ അപ്പാടെ അവഗണിക്കുന്നവരാണ്.
\end{malayalam}}
\flushright{\begin{Arabic}
\quranayah[46][4]
\end{Arabic}}
\flushleft{\begin{malayalam}
ചോദിക്കുക: അല്ലാഹുവെ വിട്ട് നിങ്ങള്‍ വിളിച്ചു പ്രാര്‍ഥിച്ചുകൊണ്ടിരിക്കുന്നവരെപ്പറ്റി നിങ്ങള്‍ ചിന്തിച്ചിട്ടുണ്ടോ? ഭൂമിയില്‍ അവരെന്തു സൃഷ്ടിച്ചുവെന്ന് നിങ്ങളെനിക്കൊന്നു കാണിച്ചു തരിക. അതല്ല; ആകാശങ്ങളുടെ സൃഷ്ടിയില്‍ അവര്‍ക്ക് വല്ല പങ്കുമുണ്ടോ? തെളിവായി ഇതിനു മുമ്പുള്ള ഏതെങ്കിലും വേദമോ അറിവിന്റെ വല്ല ശേഷിപ്പോ ഉണ്ടെങ്കില്‍ അതിങ്ങു കൊണ്ടുവരിക. നിങ്ങള്‍ സത്യവാദികളെങ്കില്‍!
\end{malayalam}}
\flushright{\begin{Arabic}
\quranayah[46][5]
\end{Arabic}}
\flushleft{\begin{malayalam}
അല്ലാഹുവെ വിട്ട്, അന്ത്യനാള്‍ വരെ കാത്തിരുന്നാലും ഉത്തരമേകാത്തവയോട് പ്രാര്‍ഥിക്കുന്നവനെക്കാള്‍ വഴിതെറ്റിയവനാരുണ്ട്? അവരോ, ഇവരുടെ പ്രാര്‍ഥനയെപ്പറ്റി തീര്‍ത്തും അശ്രദ്ധരാണ്.
\end{malayalam}}
\flushright{\begin{Arabic}
\quranayah[46][6]
\end{Arabic}}
\flushleft{\begin{malayalam}
മനുഷ്യരെയൊക്കെയും ഒരുമിച്ചുകൂട്ടുമ്പോള്‍ ആ ആരാധ്യര്‍ ഈ ആരാധകരുടെ വിരോധികളായിരിക്കും; ഇവര്‍ തങ്ങളെ ആരാധിച്ചുകൊണ്ടിരുന്നവരാണെന്ന കാര്യം തള്ളിപ്പറയുന്നവരും.
\end{malayalam}}
\flushright{\begin{Arabic}
\quranayah[46][7]
\end{Arabic}}
\flushleft{\begin{malayalam}
നമ്മുടെ തെളിവുറ്റ വചനങ്ങള്‍ ഓതിക്കേള്‍പ്പിക്കുമ്പോള്‍, തങ്ങള്‍ക്കു വന്നെത്തിയ ആ സത്യത്തെ നിഷേധിച്ചവര്‍ പറയും: ഇത് പ്രകടമായ മായാജാലം തന്നെ.
\end{malayalam}}
\flushright{\begin{Arabic}
\quranayah[46][8]
\end{Arabic}}
\flushleft{\begin{malayalam}
അല്ല; ഇത് ദൈവദൂതന്‍ ചമച്ചുണ്ടാക്കിയതാണെന്നാണോ ആ സത്യനിഷേധികള്‍ വാദിക്കുന്നത്? പറയുക: ഞാനിത് സ്വയം ചമച്ചുണ്ടാക്കിയതാണെങ്കില്‍ അല്ലാഹുവില്‍ നിന്നെന്നെ കാക്കാന്‍ ആര്‍ക്കും കഴിയില്ല. നിങ്ങള്‍ പറഞ്ഞുപരത്തുന്നവയെപ്പറ്റി ഏറ്റവും നന്നായറിയുന്നവന്‍ അല്ലാഹുവാണ്. എനിക്കും നിങ്ങള്‍ക്കുമിടയില്‍ സാക്ഷിയായി അവന്‍ മതി. അവന്‍ ഏറെ പൊറുക്കുന്നവനും പരമ ദയാലുവുമാകുന്നു.
\end{malayalam}}
\flushright{\begin{Arabic}
\quranayah[46][9]
\end{Arabic}}
\flushleft{\begin{malayalam}
പറയുക: ദൈവദൂതന്മാരില്‍ ആദ്യത്തെവനൊന്നുമല്ല ഞാന്‍. എനിക്കും നിങ്ങള്‍ക്കും എന്തൊക്കെ സംഭവിക്കുമെന്ന് എനിക്കറിയില്ല. എനിക്കു ബോധനമായി നല്‍കപ്പെടുന്ന സന്ദേശം പിന്‍പറ്റുക മാത്രമാണ് ഞാന്‍. വ്യക്തമായ മുന്നറിയിപ്പുകാരനല്ലാതാരുമല്ല ഞാന്‍.
\end{malayalam}}
\flushright{\begin{Arabic}
\quranayah[46][10]
\end{Arabic}}
\flushleft{\begin{malayalam}
ചോദിക്കുക: നിങ്ങള്‍ ചിന്തിച്ചോ? ഇതു ദൈവത്തില്‍നിന്നുള്ളതു തന്നെ ആവുകയും എന്നിട്ട് നിങ്ങളതിനെ നിഷേധിക്കുകയുമാണെങ്കിലോ? ഇങ്ങനെ ഒന്നിന് ഇസ്രായേല്‍ മക്കളിലെ ഒരു സാക്ഷി തെളിവു നല്‍കിയിട്ടുണ്ട്. അങ്ങനെ അയാള്‍ വിശ്വസിച്ചു. നിങ്ങളോ ഗര്‍വ് നടിച്ചു. ഇത്തരം അക്രമികളായ ജനത്തെ അല്ലാഹു നേര്‍വഴിയിലാക്കുകയില്ല; തീര്‍ച്ച.
\end{malayalam}}
\flushright{\begin{Arabic}
\quranayah[46][11]
\end{Arabic}}
\flushleft{\begin{malayalam}
സത്യവിശ്വാസികളോട് സത്യനിഷേധികള്‍ പറഞ്ഞു: "ഈ ഖുര്‍ആന്‍ നല്ലതായിരുന്നെങ്കില്‍ ഇതിലിവര്‍ ഞങ്ങളെ മുന്‍കടക്കുമായിരുന്നില്ല." ഇതുവഴി അവര്‍ നേര്‍വഴിയിലാകാത്തതിനാല്‍ അവര്‍ പറയും: "ഇതൊരു പഴഞ്ചന്‍ കെട്ടുകഥതന്നെ!"
\end{malayalam}}
\flushright{\begin{Arabic}
\quranayah[46][12]
\end{Arabic}}
\flushleft{\begin{malayalam}
ഒരു മാതൃകയും അനുഗ്രഹവുമെന്ന നിലയില്‍ മൂസായുടെ വേദം ഇതിനു മുമ്പേയുള്ളതാണല്ലോ. അതിനെ സത്യപ്പെടുത്തുന്ന അറബി ഭാഷയിലുള്ള വേദപുസ്തകമാണിത്. അക്രമികളെ താക്കീത് ചെയ്യാന്‍. സദ്വൃത്തരെ സുവാര്‍ത്ത അറിയിക്കാനും.
\end{malayalam}}
\flushright{\begin{Arabic}
\quranayah[46][13]
\end{Arabic}}
\flushleft{\begin{malayalam}
ഞങ്ങളുടെ നാഥന്‍ അല്ലാഹുവാണെന്ന് പ്രഖ്യാപിക്കുകയും പിന്നെ നേര്‍വഴിയില്‍ നിലയുറപ്പിക്കുകയും ചെയ്തവര്‍ ഒന്നും പേടിക്കേണ്ടതില്ല. അവര്‍ ദുഃഖിക്കേണ്ടി വരില്ല.
\end{malayalam}}
\flushright{\begin{Arabic}
\quranayah[46][14]
\end{Arabic}}
\flushleft{\begin{malayalam}
അവരാണ് സ്വര്‍ഗാവകാശികള്‍. അവരതില്‍ സ്ഥിരവാസികളായിരിക്കും. അവരിവിടെ പ്രവര്‍ത്തിച്ചിരുന്നതിന്റെ പ്രതിഫലമാണത്.
\end{malayalam}}
\flushright{\begin{Arabic}
\quranayah[46][15]
\end{Arabic}}
\flushleft{\begin{malayalam}
മാതാപിതാക്കളോട് നന്നായി വര്‍ത്തിക്കണമെന്ന് നാം മനുഷ്യനെ ഉപദേശിച്ചിരിക്കുന്നു. ക്ളേശത്തോടെയാണ് മാതാവ് അവനെ ഗര്‍ഭം ചുമന്നത്. അവനെ പ്രസവിച്ചതും പ്രയാസം സഹിച്ചുതന്നെ. ഗര്‍ഭകാലവും മുലകുടിയും കൂടി മുപ്പതു മാസം. അവനങ്ങനെ കരുത്തനാവുകയും നാല്‍പത് വയസ്സാവുകയും ചെയ്താല്‍ ഇങ്ങനെ പ്രാര്‍ഥിക്കും: "എനിക്കും എന്റെ മാതാപിതാക്കള്‍ക്കും നീയേകിയ അനുഗ്രഹങ്ങള്‍ക്ക് നന്ദി കാണിക്കാന്‍ നീയെന്നെ തുണയ്ക്കേണമേ; നിനക്കു ഹിതകരമായ സുകൃതം പ്രവര്‍ത്തിക്കാനും. എന്റെ മക്കളുടെ കാര്യത്തിലും നീ എനിക്കു നന്മ വരുത്തേണമേ. ഞാനിതാ നിന്നിലേക്കു പശ്ചാത്തപിച്ചു മടങ്ങിയിരിക്കുന്നു. ഉറപ്പായും ഞാന്‍ അനുസരണമുള്ളവരില്‍ പെട്ടവനാണ്."
\end{malayalam}}
\flushright{\begin{Arabic}
\quranayah[46][16]
\end{Arabic}}
\flushleft{\begin{malayalam}
അത്തരക്കാരില്‍ നിന്ന് അവരുടെ സുകൃതങ്ങള്‍ നാം സ്വീകരിക്കും. ദുര്‍വൃത്തികളോട് വിട്ടുവീഴ്ച കാണിക്കും. അവര്‍ സ്വര്‍ഗവാസികളുടെ കൂട്ടത്തിലായിരിക്കും. അവര്‍ക്കു നല്‍കിയിരുന്ന സത്യവാഗ്ദാനമനുസരിച്ച്.
\end{malayalam}}
\flushright{\begin{Arabic}
\quranayah[46][17]
\end{Arabic}}
\flushleft{\begin{malayalam}
എന്നാല്‍ തന്റെ മാതാപിതാക്കളോട് ഇങ്ങനെ പറയുന്നവനോ; "നിങ്ങള്‍ക്കു നാശം! ഞാന്‍ മരണശേഷം ഉയിര്‍ത്തെഴുന്നേല്‍പിക്കപ്പെടുമെന്നാണോ നിങ്ങളെന്നോട് വാഗ്ദാനം ചെയ്യുന്നത്? എന്നാല്‍ എനിക്കുമുമ്പേ എത്രയോ തലമുറകള്‍ കഴിഞ്ഞുപോയിട്ടുണ്ട്." അപ്പോള്‍ അവന്റെ മാതാപിതാക്കള്‍ ദൈവസഹായം തേടിക്കൊണ്ടു പറയുന്നു: "നിനക്കു നാശം! നീ വിശ്വസിക്കുക! അല്ലാഹുവിന്റെ വാഗ്ദാനം സത്യം തന്നെ. തീര്‍ച്ച." അപ്പോള്‍ അവന്‍ പിറുപിറുക്കും: "ഇതൊക്കെയും പൂര്‍വികരുടെ പഴങ്കഥകള്‍ മാത്രം."
\end{malayalam}}
\flushright{\begin{Arabic}
\quranayah[46][18]
\end{Arabic}}
\flushleft{\begin{malayalam}
ഇവരത്രെ ശിക്ഷാവിധി ബാധകമായിക്കഴിഞ്ഞവര്‍. ഇതേവിധം ഇവര്‍ക്കു മുമ്പേ കഴിഞ്ഞുപോയ ജിന്നുകളുടെയും മനുഷ്യരുടെയും കൂട്ടത്തില്‍ തന്നെയാണിവരും. കൊടും നഷ്ടത്തിലകപ്പെട്ടവരാണിവര്‍.
\end{malayalam}}
\flushright{\begin{Arabic}
\quranayah[46][19]
\end{Arabic}}
\flushleft{\begin{malayalam}
ഓരോരുത്തര്‍ക്കും തങ്ങള്‍ പ്രവര്‍ത്തിച്ചതിനൊത്ത പദവികളാണുണ്ടാവുക. ഏവര്‍ക്കും തങ്ങളുടെ കര്‍മഫലം തികവോടെ നല്‍കാനാണിത്. ആരും തീരെ അനീതിക്കിരയാവില്ല.
\end{malayalam}}
\flushright{\begin{Arabic}
\quranayah[46][20]
\end{Arabic}}
\flushleft{\begin{malayalam}
സത്യനിഷേധികളെ നരകത്തിനു മുന്നില്‍ കൊണ്ടുവരുന്ന ദിവസം അവരോട് പറയും: ഐഹിക ജീവിതത്തില്‍ തന്നെ നിങ്ങളുടെ വിശിഷ്ട വിഭവങ്ങളൊക്കെയും നിങ്ങള്‍ തുലച്ചുകളഞ്ഞിരിക്കുന്നു. അതിന്റെ ആനന്ദം ആസ്വദിക്കുകയും ചെയ്തു. ഇന്നു നിങ്ങള്‍ക്ക് പ്രതിഫലമായുള്ളത് വളരെ നിന്ദ്യമായ ശിക്ഷയാണ്. നിങ്ങള്‍ അനര്‍ഹമായി ഭൂമിയില്‍ നിഗളിച്ചു നടന്നതിനാലാണിത്. അധര്‍മം പ്രവര്‍ ത്തിച്ചതിനാലും.
\end{malayalam}}
\flushright{\begin{Arabic}
\quranayah[46][21]
\end{Arabic}}
\flushleft{\begin{malayalam}
ആദിന്റെ സഹോദരന്റെ വിവരം അറിയിച്ചുകൊടുക്കുക. അഹ്ഖാഫിലെ തന്റെ ജനതക്ക് അദ്ദേഹം മുന്നറിയിപ്പ് നല്‍കിയ കാര്യം. മുന്നറിയിപ്പുകാര്‍ അദ്ദേഹത്തിനു മുമ്പും പിമ്പും കഴിഞ്ഞുപോയിട്ടുണ്ട്. ആ മുന്നറിയിപ്പിതാ: അല്ലാഹുവിനല്ലാതെ മറ്റാര്‍ക്കും നിങ്ങള്‍ വഴിപ്പെട്ട് ജീവിക്കരുത്. നിങ്ങളുടെ മേല്‍ ഭീകരനാളിലെ ശിക്ഷ വന്നെത്തുമെന്ന് ഞാന്‍ ഭയപ്പെടുന്നു.
\end{malayalam}}
\flushright{\begin{Arabic}
\quranayah[46][22]
\end{Arabic}}
\flushleft{\begin{malayalam}
അവര്‍ ചോദിച്ചു: ഞങ്ങളുടെ ദൈവങ്ങളില്‍നിന്ന് ഞങ്ങളെ തെറ്റിക്കാനാണോ നീ ഞങ്ങളുടെ അടുത്ത് വന്നിരിക്കുന്നത്? എന്നാല്‍ നീ ഭീഷണിപ്പെടുത്തിക്കൊണ്ടിരിക്കുന്ന ആ ശിക്ഷയിങ്ങ് കൊണ്ടുവരിക! നീ സത്യവാനെങ്കില്‍!
\end{malayalam}}
\flushright{\begin{Arabic}
\quranayah[46][23]
\end{Arabic}}
\flushleft{\begin{malayalam}
അദ്ദേഹം പറഞ്ഞു: അതേക്കുറിച്ച അറിവ് അല്ലാഹുവിങ്കല്‍ മാത്രം! എന്നെ ഏല്പിച്ചയച്ച സന്ദേശം ഞാനിതാ നിങ്ങള്‍ക്കെത്തിച്ചു തരുന്നു. എന്നാല്‍ തീര്‍ത്തും അവിവേകികളായ ജനമായാണല്ലോ നിങ്ങളെ ഞാന്‍ കാണുന്നത്.
\end{malayalam}}
\flushright{\begin{Arabic}
\quranayah[46][24]
\end{Arabic}}
\flushleft{\begin{malayalam}
അങ്ങനെ ആ ശിക്ഷ ഒരിരുണ്ട മേഘമായി തങ്ങളുടെ താഴ്വരയുടെ നേരെ വരുന്നത് കണ്ടപ്പോള്‍ അവര്‍ പറഞ്ഞു: "നമുക്കു മഴ തരാന്‍ വരുന്ന മേഘം!" എന്നാല്‍ നിങ്ങള്‍ ധൃതി കൂട്ടിക്കൊണ്ടിരുന്ന കാര്യമാണിത്. നോവേറിയ ശിക്ഷയുടെ കൊടുങ്കാറ്റ്!
\end{malayalam}}
\flushright{\begin{Arabic}
\quranayah[46][25]
\end{Arabic}}
\flushleft{\begin{malayalam}
അത് തന്റെ നാഥന്റെ കല്‍പനയനുസരിച്ച് സകലതിനെയും തകര്‍ത്ത് തരിപ്പണമാക്കുന്നു. അങ്ങനെ അവരുടെ പാര്‍പ്പിടങ്ങളല്ലാതെ അവരെയാരെയും അവിടെ കാണാതായി. ഇവ്വിധമാണ് കുറ്റവാളികള്‍ക്ക് നാം പ്രതിഫലമേകുന്നത്.
\end{malayalam}}
\flushright{\begin{Arabic}
\quranayah[46][26]
\end{Arabic}}
\flushleft{\begin{malayalam}
നിങ്ങള്‍ക്കു തന്നിട്ടില്ലാത്ത ചില സൌകര്യങ്ങള്‍ നാം അവര്‍ക്ക് നല്‍കിയിരുന്നു. അവര്‍ക്കു നാം കേള്‍വിയും കാഴ്ചയും ബുദ്ധിയുമേകി. എന്നാല്‍ ആ കേള്‍വിയോ കാഴ്ചയോ ബുദ്ധിയോ അവര്‍ക്ക് ഒട്ടും ഉപകരിച്ചില്ല. കാരണം, അവര്‍ അല്ലാഹുവിന്റെ വചനങ്ങളെ നിഷേധിച്ചു തള്ളുകയായിരുന്നു. അങ്ങനെ അവര്‍ ഏതിനെയാണോ പരിഹസിച്ചുകൊണ്ടിരുന്നത് അതവരെ വലയം ചെയ്തു.
\end{malayalam}}
\flushright{\begin{Arabic}
\quranayah[46][27]
\end{Arabic}}
\flushleft{\begin{malayalam}
നിങ്ങളുടെ ചുറ്റുമുള്ള ചില നാടുകളെയും നാം നശിപ്പിക്കുകയുണ്ടായി. അവര്‍ സത്യത്തിലേക്കു തിരിച്ചുവരാനായി നമ്മുടെ വചനങ്ങള്‍ നാം അവര്‍ക്ക് വിശദമായി വിവരിച്ചുകൊടുത്തിരുന്നു.
\end{malayalam}}
\flushright{\begin{Arabic}
\quranayah[46][28]
\end{Arabic}}
\flushleft{\begin{malayalam}
അല്ലാഹുവിന്റെ സാമീപ്യം സിദ്ധിക്കാനായി അവനെക്കൂടാതെ അവര്‍ സ്വീകരിച്ച ദൈവങ്ങള്‍ ശിക്ഷാവേളയില്‍ എന്തുകൊണ്ട് അവരെ സഹായിച്ചില്ല? ആ ദൈവങ്ങള്‍ അവരില്‍നിന്ന് അപ്രത്യക്ഷരായിരിക്കുന്നു. ഇതാണ് അവരുടെ പൊള്ളത്തരത്തിന്റെയും അവര്‍ കെട്ടിച്ചമച്ചതിന്റെയും അവസ്ഥ.
\end{malayalam}}
\flushright{\begin{Arabic}
\quranayah[46][29]
\end{Arabic}}
\flushleft{\begin{malayalam}
ജിന്നുകളില്‍ ഒരു സംഘത്തെ ഖുര്‍ആന്‍ കേട്ടു മനസ്സിലാക്കാനായി നിന്നിലേക്ക് തിരിച്ചുവിട്ടത് ഓര്‍ക്കുക. അങ്ങനെ അതിന് ഹാജറായപ്പോള്‍ അവര്‍ പരസ്പരം പറഞ്ഞു: "നിശ്ശബ്ദത പാലിക്കുക." പിന്നെ അതില്‍നിന്ന് വിരമിച്ചപ്പോള്‍ അവര്‍ സ്വന്തം ജനത്തിലേക്ക് മുന്നറിയിപ്പുകാരായി തിരിച്ചുപോയി.
\end{malayalam}}
\flushright{\begin{Arabic}
\quranayah[46][30]
\end{Arabic}}
\flushleft{\begin{malayalam}
അവര്‍ അറിയിച്ചു: "ഞങ്ങളുടെ സമുദായമേ, ഞങ്ങള്‍ ഒരു വേദഗ്രന്ഥം കേട്ടു. അത് മൂസാക്കുശേഷം അവതീര്‍ണമായതാണ്. മുമ്പുണ്ടായിരുന്ന വേദങ്ങളെ ശരിവെക്കുന്നതും. അത് സത്യത്തിലേക്ക് വഴിനയിക്കുന്നു. നേര്‍വഴിയിലേക്കും.
\end{malayalam}}
\flushright{\begin{Arabic}
\quranayah[46][31]
\end{Arabic}}
\flushleft{\begin{malayalam}
"ഞങ്ങളുടെ സമുദായമേ, അല്ലാഹുവിലേക്ക് വിളിക്കുന്നവന് ഉത്തരമേകുക. അദ്ദേഹത്തില്‍ വിശ്വസിക്കുക. എങ്കില്‍ നിങ്ങളുടെ പാപങ്ങള്‍ അല്ലാഹു പൊറുത്തുതരും. നോവേറും ശിക്ഷയില്‍ നിന്ന് നിങ്ങളെ രക്ഷിക്കും."
\end{malayalam}}
\flushright{\begin{Arabic}
\quranayah[46][32]
\end{Arabic}}
\flushleft{\begin{malayalam}
അല്ലാഹുവിലേക്ക് ക്ഷണിക്കുന്നവന് ആരെങ്കിലും ഉത്തരം നല്‍കുന്നില്ലെങ്കിലോ, അവന് ഈ ഭൂമിയില്‍ അല്ലാഹുവിനെ തോല്‍പിക്കാനൊന്നുമാവില്ല. അല്ലാഹുവല്ലാതെ അവന് രക്ഷകരായി ആരുമില്ല. അവര്‍ വ്യക്തമായ വഴികേടില്‍ തന്നെ.
\end{malayalam}}
\flushright{\begin{Arabic}
\quranayah[46][33]
\end{Arabic}}
\flushleft{\begin{malayalam}
അവര്‍ കണ്ടറിയുന്നില്ലേ; ആകാശഭൂമികളെ സൃഷ്ടിച്ചവനും അവയുടെ സൃഷ്ടിയാലൊട്ടും തളരാത്തവനുമായ അല്ലാഹു മരിച്ചവരെ ജീവിപ്പിക്കാന്‍ കഴിവുറ്റവനാണെന്ന്? അറിയുക: ഉറപ്പായും അവന്‍ എല്ലാ കാര്യങ്ങള്‍ക്കും കഴിവുറ്റവന്‍ തന്നെ.
\end{malayalam}}
\flushright{\begin{Arabic}
\quranayah[46][34]
\end{Arabic}}
\flushleft{\begin{malayalam}
സത്യനിഷേധികളെ നരകത്തിന്നടുത്ത് കൊണ്ടുവരുംനാള്‍ അവരോട് ചോദിക്കും: "ഇതു യാഥാര്‍ഥ്യം തന്നെയല്ലേ?" അവര്‍ പറയും: "അതെ! ഞങ്ങളുടെ നാഥന്‍ തന്നെ സത്യം!" അല്ലാഹു പറയും: "നിങ്ങള്‍ നിഷേധിച്ചിരുന്നതിന്റെ ശിക്ഷ അനുഭവിച്ചുകൊള്ളുക."
\end{malayalam}}
\flushright{\begin{Arabic}
\quranayah[46][35]
\end{Arabic}}
\flushleft{\begin{malayalam}
അതിനാല്‍ നീ ക്ഷമിക്കുക. ഇഛാശക്തിയുള്ള ദൈവദൂതന്മാര്‍ ക്ഷമിച്ചപോലെ. ഈ സത്യനിഷേധികളുടെ കാര്യത്തില്‍ നീ തിരക്കു കൂട്ടാതിരിക്കുക. അവര്‍ക്ക് വാഗ്ദാനം നല്‍കപ്പെടുന്ന ശിക്ഷ നേരില്‍ കാണുന്ന ദിവസം അവര്‍ക്കു തോന്നും: തങ്ങള്‍ പകലില്‍നിന്നൊരു വിനാഴിക നേരമല്ലാതെ ഭൂലോകത്ത് വസിച്ചിട്ടില്ലെന്ന്. ഇത് ഒരറിയിപ്പാണ്. ഇനിയും അധര്‍മികളല്ലാതെ ആരെങ്കിലും നാശത്തിന്നര്‍ഹരാകുമോ?
\end{malayalam}}
\chapter{\textmalayalam{മുഹമ്മദ്}}
\begin{Arabic}
\Huge{\centerline{\basmalah}}\end{Arabic}
\flushright{\begin{Arabic}
\quranayah[47][1]
\end{Arabic}}
\flushleft{\begin{malayalam}
സത്യത്തെ തള്ളിക്കളയുകയും ദൈവമാര്‍ഗത്തില്‍നിന്ന് ജനത്തെ തടയുകയും ചെയ്തവരുടെ പ്രവര്‍ത്തനങ്ങളെ അല്ലാഹു പാഴാക്കിയിരിക്കുന്നു.
\end{malayalam}}
\flushright{\begin{Arabic}
\quranayah[47][2]
\end{Arabic}}
\flushleft{\begin{malayalam}
എന്നാല്‍ സത്യവിശ്വാസം സ്വീകരിക്കുകയും സല്‍ക്കര്‍മങ്ങളാചരിക്കുകയും മുഹമ്മദിന് അവതീര്‍ണമായതില്‍- തങ്ങളുടെ നാഥനില്‍നിന്നുള്ള പരമസത്യമാണത്- വിശ്വസിക്കുകയും ചെയ്തവരുടെ തിന്മകളെ അല്ലാഹു തേച്ചുമായിച്ചു കളഞ്ഞിരിക്കുന്നു. അവരുടെ സ്ഥിതി മെച്ചപ്പെടുത്തിയിരിക്കുന്നു.
\end{malayalam}}
\flushright{\begin{Arabic}
\quranayah[47][3]
\end{Arabic}}
\flushleft{\begin{malayalam}
അതെന്തുകൊണ്ടെന്നാല്‍ സത്യത്തെ തള്ളിക്കളഞ്ഞവര്‍ അസത്യത്തെയാണ് പിന്‍പറ്റുന്നത്. വിശ്വാസികളോ, തങ്ങളുടെ നാഥനില്‍നിന്നുള്ള സത്യത്തെ പിന്തുടരുന്നു. അല്ലാഹു ഇവ്വിധമാണ് ജനങ്ങള്‍ക്ക് അവരുടെ അവസ്ഥകള്‍ വിശദീകരിച്ചു കൊടുക്കുന്നത്.
\end{malayalam}}
\flushright{\begin{Arabic}
\quranayah[47][4]
\end{Arabic}}
\flushleft{\begin{malayalam}
അതിനാല്‍ യുദ്ധത്തില്‍ സത്യനിഷേധികളുമായി ഏറ്റുമുട്ടിയാല്‍ അവരുടെ കഴുത്ത് വെട്ടുക. അങ്ങനെ നിങ്ങളവരെ കീഴ്പ്പെടുത്തിയാല്‍ അവരെ പിടിച്ചുകെട്ടുക. പിന്നെ അവരോട് ഉദാരനയം സ്വീകരിക്കുകയോ മോചനമൂല്യം വാങ്ങി വിട്ടയക്കുകയോ ചെയ്യുക. യുദ്ധം അവസാനിക്കുന്നതുവരെയാണിത്. അതാണ് യുദ്ധനയം. അല്ലാഹു ഇഛിച്ചിരുന്നുവെങ്കില്‍ അവന്‍ തന്നെ ശത്രുക്കളെ കീഴ്പ്പെടുത്തുമായിരുന്നു. എന്നാല്‍ ഈ നടപടി നിങ്ങളില്‍ ചിലരെ മറ്റു ചിലരാല്‍ പരീക്ഷിക്കാനാണ്. ദൈവമാര്‍ഗത്തില്‍ വധിക്കപ്പെട്ടവരുടെ പ്രവര്‍ത്തനങ്ങളെ അവനൊട്ടും പാഴാക്കുകയില്ല.
\end{malayalam}}
\flushright{\begin{Arabic}
\quranayah[47][5]
\end{Arabic}}
\flushleft{\begin{malayalam}
അല്ലാഹു അവരെ നേര്‍വഴിയിലാക്കും. അവരുടെ സ്ഥിതി മെച്ചപ്പെടുത്തും.
\end{malayalam}}
\flushright{\begin{Arabic}
\quranayah[47][6]
\end{Arabic}}
\flushleft{\begin{malayalam}
അവര്‍ക്കു പരിചയപ്പെടുത്തിയ സ്വര്‍ഗത്തിലവരെ പ്രവേശിപ്പിക്കുകയും ചെയ്യും.
\end{malayalam}}
\flushright{\begin{Arabic}
\quranayah[47][7]
\end{Arabic}}
\flushleft{\begin{malayalam}
വിശ്വസിച്ചവരേ, നിങ്ങള്‍ അല്ലാഹുവെ തുണക്കുന്നുവെങ്കില്‍ അവന്‍ നിങ്ങളെയും തുണക്കും. നിങ്ങളുടെ പാദങ്ങളെ ഉറപ്പിച്ചുനിര്‍ത്തും.
\end{malayalam}}
\flushright{\begin{Arabic}
\quranayah[47][8]
\end{Arabic}}
\flushleft{\begin{malayalam}
സത്യത്തെ തള്ളിപ്പറഞ്ഞവര്‍ തുലഞ്ഞതുതന്നെ. അല്ലാഹു അവരുടെ പ്രവര്‍ത്തനങ്ങളെ പാഴാക്കിയിരിക്കുന്നു.
\end{malayalam}}
\flushright{\begin{Arabic}
\quranayah[47][9]
\end{Arabic}}
\flushleft{\begin{malayalam}
അതിനുകാരണം അല്ലാഹു അവതരിപ്പിച്ചതിനെ അവര്‍ വെറുത്തതുതന്നെ. അതിനാലവന്‍ അവരുടെ പ്രവര്‍ത്തനങ്ങളെ പാഴാക്കി.
\end{malayalam}}
\flushright{\begin{Arabic}
\quranayah[47][10]
\end{Arabic}}
\flushleft{\begin{malayalam}
അവരീ ഭൂമിയില്‍ സഞ്ചരിച്ച് തങ്ങളുടെ പൂര്‍വികരുടെ പര്യവസാനം എവ്വിധമായിരുന്നുവെന്ന് നോക്കിക്കാണുന്നില്ലേ? അല്ലാഹു അവരെ അപ്പാടെ നശിപ്പിച്ചു. ഈ സത്യനിഷേധികള്‍ക്കും സംഭവിക്കുക അതു തന്നെ.
\end{malayalam}}
\flushright{\begin{Arabic}
\quranayah[47][11]
\end{Arabic}}
\flushleft{\begin{malayalam}
കാരണം, സത്യവിശ്വാസികളുടെ രക്ഷകന്‍ അല്ലാഹുവാണ്. എന്നാല്‍ സത്യനിഷേധികള്‍ക്ക് രക്ഷകനേയില്ല.
\end{malayalam}}
\flushright{\begin{Arabic}
\quranayah[47][12]
\end{Arabic}}
\flushleft{\begin{malayalam}
സംശയം വേണ്ട; സത്യവിശ്വാസം സ്വീകരിക്കുകയും സല്‍ക്കര്‍മങ്ങളനുഷ്ഠിക്കുകയും ചെയ്തവരെ അല്ലാഹു താഴ്ഭാഗത്തൂടെ ആറുകളൊഴുകുന്ന സ്വര്‍ഗീയാരാമങ്ങളില്‍ പ്രവേശിപ്പിക്കും. എന്നാല്‍ സത്യനിഷേധികളോ, അവര്‍ സുഖിക്കുകയാണ്. നാല്‍ക്കാലികള്‍ തിന്നുംപോലെ തിന്നുകയാണ്. നരകം തന്നെയാണ് അവരുടെ വാസസ്ഥലം.
\end{malayalam}}
\flushright{\begin{Arabic}
\quranayah[47][13]
\end{Arabic}}
\flushleft{\begin{malayalam}
നിന്നെ പുറത്താക്കിയ നിന്റെ പട്ടണത്തെക്കാള്‍ പ്രബലമായ എത്രയെത്ര പട്ടണങ്ങള്‍! അവരെ നാം നിശ്ശേഷം നശിപ്പിച്ചു. അപ്പോഴവരെ സഹായിക്കാനാരുമുണ്ടായിരുന്നില്ല.
\end{malayalam}}
\flushright{\begin{Arabic}
\quranayah[47][14]
\end{Arabic}}
\flushleft{\begin{malayalam}
തന്റെ നാഥനില്‍ നിന്നുള്ള വ്യക്തമായ തെളിവനുസരിച്ച് നിലകൊള്ളുന്നവന്‍, തന്റെ ചീത്ത വൃത്തികളെ ചേതോഹരമായി കരുതുകയും തന്നിഷ്ടങ്ങളെ പിന്‍പറ്റുകയും ചെയ്യുന്നവനെപ്പോലെയാണോ?
\end{malayalam}}
\flushright{\begin{Arabic}
\quranayah[47][15]
\end{Arabic}}
\flushleft{\begin{malayalam}
സൂക്ഷ്മത പുലര്‍ത്തുന്നവര്‍ക്ക് വാഗ്ദാനം ചെയ്യപ്പെട്ട സ്വര്‍ഗത്തിന്റെ ഉപമ; അതില്‍ കലര്‍പ്പില്ലാത്ത തെളിനീരരുവികളുണ്ട്. രുചിഭേദമൊട്ടുമില്ലാത്ത പാലൊഴുകും പുഴകളുണ്ട്. കുടിക്കുന്നവര്‍ക്ക് ആസ്വാദ്യകരമായ മദ്യനദികളുണ്ട്. ശുദ്ധമായ തേനരുവികളും. അവര്‍ക്കതില്‍ സകലയിനം പഴങ്ങളുമുണ്ട്. തങ്ങളുടെ നാഥനില്‍ നിന്നുള്ള പാപമോചനവും. ഇതിന്നര്‍ഹരാകുന്നവര്‍ നരകത്തില്‍ നിത്യവാസിയായവനെപ്പോലെയാണോ? അവരവിടെ കുടിപ്പിക്കപ്പെടുക കൊടും ചൂടുള്ള വെള്ളമായിരിക്കും. അതവരുടെ കുടലുകളെ കീറിപ്പൊളിക്കും.
\end{malayalam}}
\flushright{\begin{Arabic}
\quranayah[47][16]
\end{Arabic}}
\flushleft{\begin{malayalam}
നീ പറയുന്നതൊക്കെ ശ്രദ്ധാപൂര്‍വം ശ്രവിക്കുന്നതായി ഭാവിക്കുന്ന ചിലരുണ്ട്. എന്നാല്‍ നിന്റെ അടുത്തുനിന്ന് പുറത്തുപോയാല്‍ വേദവിജ്ഞാനം നല്‍കപ്പെട്ടവരോട് അവര്‍ ചോദിക്കുന്നു: "ഇദ്ദേഹമിപ്പോള്‍ ഇപ്പറഞ്ഞതെന്താണ്?" അത്തരക്കാരുടെ ഹൃദയങ്ങള്‍ക്കാണ് അല്ലാഹു മുദ്രവെച്ചിരിക്കുന്നത്. തന്നിഷ്ടങ്ങളെയാണവന്‍ പിന്‍പറ്റുന്നത്.
\end{malayalam}}
\flushright{\begin{Arabic}
\quranayah[47][17]
\end{Arabic}}
\flushleft{\begin{malayalam}
സന്മാര്‍ഗം സ്വീകരിച്ചവരോ, അല്ലാഹു അവര്‍ക്ക് കൂടുതല്‍ മാര്‍ഗദര്‍ശനമേകുന്നു. അവര്‍ക്കാവശ്യമായ സൂക്ഷ്മത നല്‍കുന്നു.
\end{malayalam}}
\flushright{\begin{Arabic}
\quranayah[47][18]
\end{Arabic}}
\flushleft{\begin{malayalam}
അന്ത്യദിനം ആകസ്മികമായി ആസന്നമാകുന്നതല്ലാതെ വല്ലതും അവര്‍ക്ക് കാത്തിരിക്കാനുണ്ടോ? അതിന്റെ അടയാളങ്ങള്‍ ആഗതമായിരിക്കുന്നു. അതവരില്‍ വന്നെത്തിയാല്‍ പിന്നെ തങ്ങള്‍ക്കുള്ള ഉദ്ബോധനം ഉള്‍ക്കൊള്ളാന്‍ അവര്‍ക്കെങ്ങനെ കഴിയും!
\end{malayalam}}
\flushright{\begin{Arabic}
\quranayah[47][19]
\end{Arabic}}
\flushleft{\begin{malayalam}
അതിനാല്‍ അറിയുക: അല്ലാഹുവല്ലാതെ ദൈവമേയില്ല. നിന്റെയും മുഴുവന്‍ സത്യവിശ്വാസികളുടെയും വിശ്വാസിനികളുടെയും പാപങ്ങള്‍ക്ക് നീ മാപ്പിരക്കുക. നിങ്ങളുടെ പോക്കുവരവും നില്‍പുമെല്ലാം അല്ലാഹു അറിയുന്നുണ്ട്.
\end{malayalam}}
\flushright{\begin{Arabic}
\quranayah[47][20]
\end{Arabic}}
\flushleft{\begin{malayalam}
വിശ്വാസികള്‍ പറയാറുണ്ടല്ലോ: "യുദ്ധാനുമതിനല്‍കുന്ന ഒരധ്യായം അവതീര്‍ണമാകാത്തതെന്ത്?" എന്നാല്‍ ഖണ്ഡിതമായ ഒരധ്യായം അവതീര്‍ണമാവുകയും അതില്‍ യുദ്ധം പരാമര്‍ശിക്കപ്പെടുകയും ചെയ്താല്‍ മനസ്സില്‍ രോഗമുള്ളവര്‍, മരണവെപ്രാളത്തില്‍ പെട്ടവന്‍ നോക്കുംപോലെ നിന്നെ നോക്കുന്നതു കാണാം. അതിനാലവര്‍ക്കു നാശം.
\end{malayalam}}
\flushright{\begin{Arabic}
\quranayah[47][21]
\end{Arabic}}
\flushleft{\begin{malayalam}
അനുസരണവും മാന്യമായ സംസാരവുമാണാവശ്യം. യുദ്ധകാര്യം തീരുമാനമായപ്പോള്‍ അവര്‍ അല്ലാഹുവോട് സത്യസന്ധത പുലര്‍ത്തിയിരുന്നെങ്കില്‍. അതാകുമായിരുന്നു അവര്‍ക്കുത്തമം.
\end{malayalam}}
\flushright{\begin{Arabic}
\quranayah[47][22]
\end{Arabic}}
\flushleft{\begin{malayalam}
നിങ്ങള്‍ പിന്തിരിഞ്ഞുപോവുകയാണെങ്കില്‍ പിന്നെ നിങ്ങള്‍ ഭൂമിയില്‍ കുഴപ്പമുണ്ടാക്കുകയല്ലാതെന്താണ് ചെയ്യുന്നത്? നിങ്ങളുടെ കുടുംബ ബന്ധങ്ങളെ മുറിച്ചുകളയുകയും?
\end{malayalam}}
\flushright{\begin{Arabic}
\quranayah[47][23]
\end{Arabic}}
\flushleft{\begin{malayalam}
അത്തരക്കാരെയാണ് അല്ലാഹു ശപിച്ചത്. അങ്ങനെ അവനവരെ ചെവികേള്‍ക്കാത്തവരും കണ്ണുകാണാത്തവരുമാക്കി.
\end{malayalam}}
\flushright{\begin{Arabic}
\quranayah[47][24]
\end{Arabic}}
\flushleft{\begin{malayalam}
അവര്‍ ഖുര്‍ആന്‍ ആഴത്തില്‍ ചിന്തിച്ചു മനസ്സിലാക്കുന്നില്ലേ? അതല്ല; അവരുടെ ഹൃദയങ്ങളെ താഴിട്ട് പൂട്ടിയിട്ടുണ്ടോ?
\end{malayalam}}
\flushright{\begin{Arabic}
\quranayah[47][25]
\end{Arabic}}
\flushleft{\begin{malayalam}
നേര്‍വഴി വ്യക്തമായിട്ടും അത് വിട്ട് പിന്തിരിഞ്ഞു പോയവര്‍ക്ക് ചെകുത്താന്‍ അവരുടെ ചെയ്തികള്‍ ചേതോഹരമാക്കിത്തോന്നിക്കുന്നു. അവനവരെ വ്യാമോഹത്തിലകപ്പെടുത്തുകയാണ്.
\end{malayalam}}
\flushright{\begin{Arabic}
\quranayah[47][26]
\end{Arabic}}
\flushleft{\begin{malayalam}
അല്ലാഹു അവതരിപ്പിച്ചതിനെ വെറുക്കുന്നവരോട് “ചില കാര്യങ്ങളില്‍ ഞങ്ങള്‍ നിങ്ങളെ അനുസരിച്ചുകൊള്ളാ”മെന്ന് കപടവിശ്വാസികള്‍ വാക്കുകൊടുത്തതിനാലാണത്. അവര്‍ രഹസ്യമാക്കിവെക്കുന്നതൊക്കെയും അല്ലാഹു അറിയുന്നു.
\end{malayalam}}
\flushright{\begin{Arabic}
\quranayah[47][27]
\end{Arabic}}
\flushleft{\begin{malayalam}
മലക്കുകള്‍ അവരെ മുഖത്തും മുതുകിലും അടിച്ച് മരിപ്പിക്കുമ്പോള്‍ എന്തായിരിക്കും അവരുടെ അവസ്ഥ?
\end{malayalam}}
\flushright{\begin{Arabic}
\quranayah[47][28]
\end{Arabic}}
\flushleft{\begin{malayalam}
അല്ലാഹുവിന് അനിഷ്ടമുണ്ടാക്കുന്നവയെ അനുധാവനം ചെയ്യുകയും അവന്റെ തൃപ്തിയെ വെറുക്കുകയും ചെയ്തതിനാലാണിത്. അതുകൊണ്ടുതന്നെ അല്ലാഹു അവരുടെ പ്രവര്‍ത്തനങ്ങളെ പാഴാക്കിയിരിക്കുന്നു.
\end{malayalam}}
\flushright{\begin{Arabic}
\quranayah[47][29]
\end{Arabic}}
\flushleft{\begin{malayalam}
ദീനം പിടിച്ച മനസ്സുള്ളവര്‍ കരുതുന്നുവോ; അവരുടെ ഉള്ളിലെ പക അല്ലാഹു വെളിക്ക് കൊണ്ടുവരില്ലെന്ന്.
\end{malayalam}}
\flushright{\begin{Arabic}
\quranayah[47][30]
\end{Arabic}}
\flushleft{\begin{malayalam}
നാം ഇഛിച്ചിരുന്നെങ്കില്‍ നിനക്കു നാമവരെ കാണിച്ചുതരുമായിരുന്നു. അപ്പോള്‍ അവരുടെ അടയാളം വഴി നിനക്കവരെ വേര്‍തിരിച്ചറിയാം. അവരുടെ സംസാരശൈലിയില്‍ നിന്ന് നിനക്കവരെ വ്യക്തമായി മനസ്സിലാകും; തീര്‍ച്ച. അല്ലാഹു നിങ്ങളുടെ കര്‍മങ്ങളൊക്കെയും അറിയുന്നു.
\end{malayalam}}
\flushright{\begin{Arabic}
\quranayah[47][31]
\end{Arabic}}
\flushleft{\begin{malayalam}
നിശ്ചയമായും നാം നിങ്ങളെ പരീക്ഷിക്കും; നിങ്ങളിലെ പോരാളികളും ക്ഷമ പാലിക്കുന്നവരും ആരെന്ന് വേര്‍തിരിച്ചറിയുകയും നിങ്ങളുടെ വൃത്താന്തങ്ങള്‍ പരിശോധിച്ചുനോക്കുകയും ചെയ്യുംവരെ.
\end{malayalam}}
\flushright{\begin{Arabic}
\quranayah[47][32]
\end{Arabic}}
\flushleft{\begin{malayalam}
നേര്‍വഴി വ്യക്തമായ ശേഷം സത്യത്തെ തള്ളിപ്പറയുകയും ദൈവമാര്‍ഗത്തില്‍നിന്ന് ജനത്തെ തടഞ്ഞുനിര്‍ത്തുകയും ദൈവദൂതനോട് പോര് കാണിക്കുകയും ചെയ്തവരോ, അവര്‍ അല്ലാഹുവിന് ഒരു ദ്രോഹവും വരുത്തുന്നില്ല. എന്നാല്‍ അല്ലാഹു അവരുടെ പ്രവര്‍ത്തനങ്ങളെ പാഴാക്കുന്നതാണ്.
\end{malayalam}}
\flushright{\begin{Arabic}
\quranayah[47][33]
\end{Arabic}}
\flushleft{\begin{malayalam}
വിശ്വസിച്ചവരേ, നിങ്ങള്‍ അല്ലാഹുവിനെ അനുസരിക്കുക. ദൈവദൂതനെയും അനുസരിക്കുക. നിങ്ങളുടെ പ്രവര്‍ത്തനങ്ങളെ നിങ്ങള്‍ പാഴാക്കരുത്.
\end{malayalam}}
\flushright{\begin{Arabic}
\quranayah[47][34]
\end{Arabic}}
\flushleft{\begin{malayalam}
സത്യത്തെ നിഷേധിച്ചു തള്ളുകയും ദൈവമാര്‍ഗത്തില്‍നിന്ന് ജനത്തെ തടഞ്ഞുനിര്‍ത്തുകയും അങ്ങനെ സത്യനിഷേധികളായിത്തന്നെ മരിക്കുകയും ചെയ്തവര്‍ക്ക് അല്ലാഹു മാപ്പേകുകയില്ല; ഉറപ്പ്.
\end{malayalam}}
\flushright{\begin{Arabic}
\quranayah[47][35]
\end{Arabic}}
\flushleft{\begin{malayalam}
അതിനാല്‍ നിങ്ങള്‍ ദുര്‍ബലരാകരുത്. നിങ്ങള്‍ അങ്ങോട്ട് സന്ധിക്ക് അപേക്ഷിക്കുകയുമരുത്. നിങ്ങള്‍ തന്നെയാണ് അതിജയിക്കുന്നവര്‍. അല്ലാഹു നിങ്ങളോടൊപ്പമുണ്ട്. നിങ്ങളുടെ പ്രവര്‍ത്തനങ്ങളില്‍ അവന്‍ നിങ്ങള്‍ക്കൊരു നഷ്ടവും വരുത്തുകയില്ല.
\end{malayalam}}
\flushright{\begin{Arabic}
\quranayah[47][36]
\end{Arabic}}
\flushleft{\begin{malayalam}
ഈ ഐഹിക ജീവിതം കളിയും തമാശയും മാത്രം. നിങ്ങള്‍ സത്യവിശ്വാസം സ്വീകരിക്കുകയും സൂക്ഷ്മതയുള്ളവരാവുകയുമാണെങ്കില്‍ നിങ്ങളര്‍ഹിക്കുന്ന പ്രതിഫലം അല്ലാഹു നിങ്ങള്‍ക്ക് നല്‍കും. നിങ്ങളോട് അവന്‍ നിങ്ങളുടെ സ്വത്തൊന്നും ചോദിക്കുന്നില്ലല്ലോ.
\end{malayalam}}
\flushright{\begin{Arabic}
\quranayah[47][37]
\end{Arabic}}
\flushleft{\begin{malayalam}
അഥവാ, നിങ്ങളോട് അവനതാവശ്യപ്പെട്ട് പ്രയാസപ്പെടുത്തിയിരുന്നുവെങ്കില്‍ നിങ്ങള്‍ പിശുക്കു കാണിക്കുമായിരുന്നു. അങ്ങനെ നിങ്ങളുടെ അകപ്പക അവന്‍ പുറത്തുകൊണ്ടുവരുമായിരുന്നു.
\end{malayalam}}
\flushright{\begin{Arabic}
\quranayah[47][38]
\end{Arabic}}
\flushleft{\begin{malayalam}
അല്ലയോ കൂട്ടരേ, നിങ്ങളോടിതാ ദൈവമാര്‍ഗത്തില്‍ ധനവ്യയമാവശ്യപ്പെടുന്നു. അപ്പോള്‍ നിങ്ങളില്‍ പിശുക്കു കാണിക്കുന്ന ചിലരുണ്ട്. ആര്‍ പിശുക്കു കാണിക്കുന്നുവോ അവന്‍ തനിക്കെതിരെ തന്നെയാണ് പിശുക്കു കാട്ടുന്നത്. അല്ലാഹു അന്യാശ്രയമാവശ്യമില്ലാത്തവനാണ്. നിങ്ങളോ അവന്റെ ആശ്രിതരും. നിങ്ങള്‍ നേര്‍വഴിയില്‍നിന്ന് പിന്തിരിയുകയാണെങ്കില്‍ അല്ലാഹു നിങ്ങള്‍ക്കു പകരം മറ്റൊരു ജനതയെ കൊണ്ടുവരും. പിന്നെ അവര്‍ നിങ്ങളെപ്പോലെയാവുകയില്ല.
\end{malayalam}}
\chapter{\textmalayalam{ഫതഹ് ( വിജയം )}}
\begin{Arabic}
\Huge{\centerline{\basmalah}}\end{Arabic}
\flushright{\begin{Arabic}
\quranayah[48][1]
\end{Arabic}}
\flushleft{\begin{malayalam}
നിശ്ചയമായും നിനക്കു നാം വ്യക്തമായ വിജയം നല്‍കിയിരിക്കുന്നു.
\end{malayalam}}
\flushright{\begin{Arabic}
\quranayah[48][2]
\end{Arabic}}
\flushleft{\begin{malayalam}
നിന്റെ വന്നതും വരാനുള്ളതുമായ പിഴവുകളൊക്കെയും പൊറുത്തു തരാനാണിത്; അല്ലാഹുവിന്റെ അനുഗ്രഹം നിനക്കു തികവോടെ നിറവേറ്റിത്തരാനും; നേരായ വഴിയിലൂടെ നിന്നെ നയിക്കാനും.
\end{malayalam}}
\flushright{\begin{Arabic}
\quranayah[48][3]
\end{Arabic}}
\flushleft{\begin{malayalam}
അന്തസ്സുറ്റ സഹായം നിനക്കേകാനും.
\end{malayalam}}
\flushright{\begin{Arabic}
\quranayah[48][4]
\end{Arabic}}
\flushleft{\begin{malayalam}
അല്ലാഹുവാണ് സത്യവിശ്വാസികളുടെ ഹൃദയങ്ങളില്‍ ശാന്തി വര്‍ഷിച്ചത്. അതുവഴി അവരുടെ വിശ്വാസം ഒന്നുകൂടി വര്‍ധിക്കാനാണിത്. ആകാശഭൂമികളിലെ സൈന്യം അല്ലാഹുവിന്റേതാണ്. അല്ലാഹു സര്‍വജ്ഞനും യുക്തിമാനുമല്ലോ.
\end{malayalam}}
\flushright{\begin{Arabic}
\quranayah[48][5]
\end{Arabic}}
\flushleft{\begin{malayalam}
സത്യവിശ്വാസികളെയും വിശ്വാസിനികളെയും താഴ്ഭാഗത്തൂടെ ആറുകളൊഴുകുന്ന സ്വര്‍ഗീയാരാമങ്ങളില്‍ നിത്യവാസികളായി പ്രവേശിപ്പിക്കാനാണ് ഇങ്ങനെ ചെയ്യുന്നത്. അവരില്‍നിന്ന് അവരുടെ പാപങ്ങള്‍ മായ്ച്ചു കളയാനും. അല്ലാഹുവിങ്കല്‍ ഇത് അതിമഹത്തായ വിജയം തന്നെ.
\end{malayalam}}
\flushright{\begin{Arabic}
\quranayah[48][6]
\end{Arabic}}
\flushleft{\begin{malayalam}
കപടവിശ്വാസികളും ബഹുദൈവ വിശ്വാസികളുമായ സ്ത്രീപുരുഷന്മാരെ ശിക്ഷിക്കാനുമാണിത്. അവര്‍ അല്ലാഹുവെപ്പറ്റി ചീത്ത ധാരണകള്‍ വെച്ചുപുലര്‍ത്തുന്നവരാണ്. അവര്‍ക്കു ചുറ്റും തിന്മയുടെ വലയമുണ്ട്. അല്ലാഹു അവരോട് കോപിച്ചിരിക്കുന്നു. അവരെ ശപിക്കുകയും ചെയ്തിരിക്കുന്നു. അവര്‍ക്കായി നരകം ഒരുക്കിവെച്ചിരിക്കുന്നു. അതെത്ര ചീത്ത സങ്കേതം!
\end{malayalam}}
\flushright{\begin{Arabic}
\quranayah[48][7]
\end{Arabic}}
\flushleft{\begin{malayalam}
ആകാശഭൂമികളിലെ സൈന്യങ്ങള്‍അല്ലാഹുവിന്റേതാണ്. അല്ലാഹു പ്രതാപിയും യുക്തിജ്ഞനുമാണ്.
\end{malayalam}}
\flushright{\begin{Arabic}
\quranayah[48][8]
\end{Arabic}}
\flushleft{\begin{malayalam}
നിശ്ചയം; നിന്നെ നാം സാക്ഷിയും സുവാര്‍ത്ത അറിയിക്കുന്നവനും മുന്നറിയിപ്പു നല്‍കുന്നവനുമായി നിയോഗിച്ചിരിക്കുന്നു.
\end{malayalam}}
\flushright{\begin{Arabic}
\quranayah[48][9]
\end{Arabic}}
\flushleft{\begin{malayalam}
നിങ്ങള്‍ അല്ലാഹുവിലും അവന്റെ ദൂതനിലും വിശ്വസിക്കാനാണിത്. നിങ്ങളവനെ പിന്തുണക്കാനാണ്. അവനോട് ആദരവ് പ്രകടിപ്പിക്കാനും രാവിലെയും വൈകുന്നേരവും അവന്റെ മഹത്വം കീര്‍ത്തിക്കാനും.
\end{malayalam}}
\flushright{\begin{Arabic}
\quranayah[48][10]
\end{Arabic}}
\flushleft{\begin{malayalam}
നിശ്ചയമായും നിന്നോട് പ്രതിജ്ഞ ചെയ്യുന്നവര്‍ അല്ലാഹുവോട് തന്നെയാണ് പ്രതിജ്ഞ ചെയ്യുന്നത്. അവരുടെ കൈകള്‍ക്കു മീതെ അല്ലാഹുവിന്റെ കൈയാണുള്ളത്. അതിനാല്‍ ആരെങ്കിലും അത് ലംഘിക്കുന്നുവെങ്കില്‍ അതിന്റെ ദുഷ്ഫലം അവനുതന്നെയാണ്. അല്ലാഹുവുമായി ചെയ്ത പ്രതിജ്ഞ പൂര്‍ത്തീകരിക്കുന്നവന് അവന്‍ അതിമഹത്തായ പ്രതിഫലം നല്‍കും.
\end{malayalam}}
\flushright{\begin{Arabic}
\quranayah[48][11]
\end{Arabic}}
\flushleft{\begin{malayalam}
മാറിനിന്ന ഗ്രാമീണ അറബികള്‍ നിന്നോട് പറയും: "ഞങ്ങളുടെ സ്വത്തും സ്വന്തക്കാരും ഞങ്ങളെ ജോലിത്തിരക്കുകളിലകപ്പെടുത്തി. അതിനാല്‍ താങ്കള്‍ ഞങ്ങളുടെ പാപം പൊറുക്കാന്‍ പ്രാര്‍ഥിക്കുക." അവരുടെ മനസ്സുകളിലില്ലാത്തതാണ് നാവുകൊണ്ട് അവര്‍ പറയുന്നത്. ചോദിക്കുക: "അല്ലാഹു നിങ്ങള്‍ക്ക് എന്തെങ്കിലും ഉപദ്രവമോ ഉപകാരമോ വരുത്താനുദ്ദേശിച്ചാല്‍ നിങ്ങള്‍ക്കുവേണ്ടി അവയെ തടയാന്‍ കഴിവുറ്റ ആരുണ്ട്? നിങ്ങള്‍ പ്രവര്‍ത്തിച്ചുകൊണ്ടിരിക്കുന്നവയെപ്പറ്റി നന്നായറിയുന്നവനാണ് അല്ലാഹു."
\end{malayalam}}
\flushright{\begin{Arabic}
\quranayah[48][12]
\end{Arabic}}
\flushleft{\begin{malayalam}
എന്നാല്‍ സംഗതി അതല്ല; ദൈവദൂതനും സത്യവിശ്വാസികളും തങ്ങളുടെ കുടുംബങ്ങളില്‍ ഒരിക്കലും തിരിച്ചെത്തില്ലെന്നാണ് നിങ്ങള്‍ കരുതിയത്. ആ തോന്നല്‍ നിങ്ങളുടെ ഹൃദയങ്ങള്‍ക്ക് ഹരമായിത്തീരുകയും ചെയ്തു. നന്നെ നീചമായ വിചാരമാണ് നിങ്ങള്‍ വെച്ചു പുലര്‍ത്തിയത്. നിങ്ങള്‍ തീര്‍ത്തും തുലഞ്ഞ ജനം തന്നെ.
\end{malayalam}}
\flushright{\begin{Arabic}
\quranayah[48][13]
\end{Arabic}}
\flushleft{\begin{malayalam}
അല്ലാഹുവിലും അവന്റെ ദൂതനിലും വിശ്വസിക്കാത്ത സത്യനിഷേധികള്‍ക്കു നാം കത്തിക്കാളും നരകത്തീ ഒരുക്കിയിരിക്കുന്നു;
\end{malayalam}}
\flushright{\begin{Arabic}
\quranayah[48][14]
\end{Arabic}}
\flushleft{\begin{malayalam}
ആകാശ ഭൂമികളുടെ ആധിപത്യം അല്ലാഹുവിനാണ്. അവനിഛിക്കുന്നവര്‍ക്ക് അവന്‍ പൊറുത്തുകൊടുക്കും. അവനുദ്ദേശിക്കുന്നവരെ ശിക്ഷിക്കുകയും ചെയ്യും. അല്ലാഹു ഏറെ പൊറുക്കുന്നവനും ദയാപരനും തന്നെ.
\end{malayalam}}
\flushright{\begin{Arabic}
\quranayah[48][15]
\end{Arabic}}
\flushleft{\begin{malayalam}
നിങ്ങള്‍ സമരാര്‍ജിത സ്വത്ത് ശേഖരിക്കാന്‍ പുറപ്പെടുമ്പോള്‍ യുദ്ധം ചെയ്യാതെ മാറിനിന്നവര്‍ പറയും: "ഞങ്ങളെ വിട്ടേക്കൂ. ഞങ്ങളും നിങ്ങളുടെ കൂടെ വരട്ടെ." ദൈവവചനങ്ങളെ മാറ്റിമറിക്കാനാണ് അവരാഗ്രഹിക്കുന്നത്. പറയുക: "നിങ്ങള്‍ക്കൊരിക്കലും ഞങ്ങളോടൊത്ത് വരാനാവില്ല. അല്ലാഹു നേരത്തെ തന്നെ അത് പറഞ്ഞറിയിച്ചിട്ടുണ്ട്." അപ്പോഴവര്‍ പറയും: "അല്ല; നിങ്ങള്‍ ഞങ്ങളോട് അസൂയ കാട്ടുകയാണ്." എന്നാല്‍; അവരൊന്നും മനസ്സിലാക്കുന്നില്ലെന്നതാണ് വസ്തുത; നന്നെക്കുറച്ചല്ലാതെ.
\end{malayalam}}
\flushright{\begin{Arabic}
\quranayah[48][16]
\end{Arabic}}
\flushleft{\begin{malayalam}
യുദ്ധത്തില്‍ നിന്നു വിട്ടുനിന്ന ഗ്രാമീണ അറബികളോട് പറയുക: "കഠിനമായ ആക്രമണ ശേഷിയുള്ള ജനത്തെ നേരിടാന്‍ നിങ്ങള്‍ക്ക് ആഹ്വാനം ലഭിക്കും. അവര്‍ കീഴടങ്ങും വരെ നിങ്ങളവരോട് പൊരുതേണ്ടിവരും. ആ ആഹ്വാനം നിങ്ങള്‍ അനുസരിച്ചാല്‍ അല്ലാഹു നിങ്ങള്‍ക്ക് അതിമഹത്തായ പ്രതിഫലം നല്‍കും. അഥവാ നേരത്തെ നിങ്ങള്‍ പിന്തിരിഞ്ഞപോലെ പിന്മാറുന്നപക്ഷം നിങ്ങളെ അവന്‍ ശിക്ഷിക്കും. നോവേറും ശിക്ഷ."
\end{malayalam}}
\flushright{\begin{Arabic}
\quranayah[48][17]
\end{Arabic}}
\flushleft{\begin{malayalam}
കുരുടന് കുറ്റമില്ല. മുടന്തന്നും കുറ്റമില്ല. രോഗിക്കും കുറ്റമില്ല. അല്ലാഹുവിനെയും അവന്റെ ദൂതനെയും അനുസരിക്കുന്നവനെ അവന്‍ താഴ്ഭാഗത്തൂടെ ആറുകളൊഴുകുന്ന സ്വര്‍ഗീയാരാമങ്ങളില്‍ പ്രവേശിപ്പിക്കും. പുറംതിരിഞ്ഞു മാറിനില്‍ക്കുന്നവനെ നോവേറും ശിക്ഷക്കിരയാക്കുകയും ചെയ്യും.
\end{malayalam}}
\flushright{\begin{Arabic}
\quranayah[48][18]
\end{Arabic}}
\flushleft{\begin{malayalam}
മരച്ചുവട്ടില്‍വെച്ച് സത്യവിശ്വാസികള്‍ നിന്നോട് പ്രതിജ്ഞ ചെയ്ത വേളയില്‍ ഉറപ്പായും അല്ലാഹു അവരെ തൃപ്തിപ്പെട്ടിരിക്കുന്നു. അപ്പോള്‍ അവരുടെ ഹൃദയങ്ങളിലുള്ളത് അല്ലാഹു തിരിച്ചറിഞ്ഞിരിക്കുന്നു. അങ്ങനെ അവന്‍ അവര്‍ക്ക് മനസ്സമാധാനമേകി. ആസന്നമായ വിജയം വഴി പ്രതിഫലം നല്‍കുകയും ചെയ്തു.
\end{malayalam}}
\flushright{\begin{Arabic}
\quranayah[48][19]
\end{Arabic}}
\flushleft{\begin{malayalam}
അവര്‍ക്കെടുക്കാന്‍ ഒട്ടേറെ സമരാര്‍ജിത സ്വത്തും അവനേകി. അല്ലാഹു പ്രതാപിയും യുക്തിജ്ഞനും തന്നെ.
\end{malayalam}}
\flushright{\begin{Arabic}
\quranayah[48][20]
\end{Arabic}}
\flushleft{\begin{malayalam}
നിങ്ങള്‍ക്കെടുക്കാന്‍ ധാരാളം സമരാര്‍ജിത സമ്പത്ത് അല്ലാഹു വാഗ്ദാനം ചെയ്തിരിക്കുന്നു. എന്നാലിത് അല്ലാഹു നിങ്ങള്‍ക്ക് മുന്‍കൂട്ടി തന്നെ തന്നിരിക്കുന്നു. നിങ്ങളില്‍നിന്ന് ജനത്തിന്റെ കൈകളെ അവന്‍ തടഞ്ഞുനിര്‍ത്തുകയും ചെയ്തിരിക്കുന്നു. സത്യവിശ്വാസികള്‍ക്കൊരടയാളമാകാനാണിത്. നിങ്ങളെ നേര്‍വഴിയില്‍ നയിക്കാനും.
\end{malayalam}}
\flushright{\begin{Arabic}
\quranayah[48][21]
\end{Arabic}}
\flushleft{\begin{malayalam}
നിങ്ങള്‍ക്കു കൈവരിക്കാനായിട്ടില്ലാത്ത മറ്റു നേട്ടങ്ങളും അവന്‍ നിങ്ങള്‍ക്ക് വാഗ്ദാനം നല്‍കിയിരിക്കുന്നു. അവയെയൊക്കെ അല്ലാഹു വലയം ചെയ്ത് വെച്ചിരിക്കുകയാണ്. അല്ലാഹു എല്ലാ കാര്യത്തിനും കഴിവുറ്റവന്‍ തന്നെ.
\end{malayalam}}
\flushright{\begin{Arabic}
\quranayah[48][22]
\end{Arabic}}
\flushleft{\begin{malayalam}
സത്യനിഷേധികള്‍ നിങ്ങളോട് യുദ്ധം ചെയ്തിരുന്നുവെങ്കില്‍ അവര്‍ പിന്തിരിഞ്ഞോടുമായിരുന്നു. പിന്നെ അവര്‍ക്കൊരു രക്ഷകനെയോ സഹായിയെയോ കണ്ടെത്താനാവില്ല.
\end{malayalam}}
\flushright{\begin{Arabic}
\quranayah[48][23]
\end{Arabic}}
\flushleft{\begin{malayalam}
മുമ്പു മുതലേ നടന്നുവരുന്ന അല്ലാഹുവിന്റെ നടപടി ക്രമമാണിത്. അല്ലാഹുവിന്റെ നടപടിക്രമങ്ങളിലൊരു മാറ്റവും നിനക്കു കാണാനാവില്ല.
\end{malayalam}}
\flushright{\begin{Arabic}
\quranayah[48][24]
\end{Arabic}}
\flushleft{\begin{malayalam}
മക്കയുടെ മാറിടത്തില്‍ വെച്ച് അവരുടെ കൈകളെ നിങ്ങളില്‍നിന്നും നിങ്ങളുടെ കൈകളെ അവരില്‍നിന്നും തടഞ്ഞുനിര്‍ത്തിയത് അല്ലാഹുവാണ് - അവന്‍ അവര്‍ക്കെതിരെ നിങ്ങള്‍ക്ക് വിജയമരുളിക്കഴിഞ്ഞിരിക്കെ. നിങ്ങള്‍ ചെയ്യുന്നതെല്ലാം കണ്ടറിയുന്നവനാണ് അല്ലാഹു.
\end{malayalam}}
\flushright{\begin{Arabic}
\quranayah[48][25]
\end{Arabic}}
\flushleft{\begin{malayalam}
മക്കയിലുണ്ടായിരുന്നവര്‍ സത്യത്തെ തള്ളിപ്പറഞ്ഞവരായിരുന്നു; നിങ്ങളെ മസ്ജിദുല്‍ ഹറാമില്‍നിന്ന് വിലക്കിയവരും ബലിമൃഗങ്ങളെ നിശ്ചിത സ്ഥലത്തെത്താനനുവദിക്കാതെ തടഞ്ഞു നിര്‍ത്തിയവരും. സത്യവിശ്വാസികളെന്ന് നിങ്ങള്‍ തിരിച്ചറിഞ്ഞിട്ടില്ലാത്ത ചില സ്ത്രീ പുരുഷന്മാരെ നിങ്ങള്‍ ചവിട്ടിമെതിക്കാനും അങ്ങനെ കാര്യമറിയാതെ അവര്‍ കാരണമായി നിങ്ങള്‍ തെറ്റിലകപ്പെടാനും സാധ്യതയില്ലായിരുന്നുവെങ്കില്‍ അല്ലാഹു അങ്ങനെ ചെയ്യുമായിരുന്നില്ല. അല്ലാഹു താനിഛിക്കുന്നവരെ തന്റെ അനുഗ്രഹത്തിലുള്‍പ്പെടുത്താനാണിത്. അവര്‍ വെവ്വേറെയാണ് വസിച്ചിരുന്നതെങ്കില്‍ അവരിലെ സത്യനിഷേധികള്‍ക്കു നാം നോവേറിയ ശിക്ഷ നല്‍കുമായിരുന്നു.
\end{malayalam}}
\flushright{\begin{Arabic}
\quranayah[48][26]
\end{Arabic}}
\flushleft{\begin{malayalam}
സത്യനിഷേധികള്‍ തങ്ങളുടെ മനസ്സുകളില്‍ ദുരഭിമാനം -അനിസ്ലാമികകാലത്തെ പക്ഷപാതിത്വ ദുരഭിമാനം-പുലര്‍ത്തിയപ്പോള്‍ അല്ലാഹു തന്റെ ദൂതന്നും വിശ്വാസികള്‍ക്കും മനശ്ശാന്തിയേകി. സൂക്ഷ്മത പാലിക്കാനുള്ള കല്‍പന പുല്‍കാനവരെ നിര്‍ബന്ധിക്കുകയും ചെയ്തു. അതംഗീകരിക്കാന്‍ ഏറ്റം അര്‍ഹരും അതിന്റെ അവകാശികളും അവര്‍തന്നെ. അല്ലാഹു എല്ലാ കാര്യങ്ങളും നന്നായറിയുന്നവനാണ്.
\end{malayalam}}
\flushright{\begin{Arabic}
\quranayah[48][27]
\end{Arabic}}
\flushleft{\begin{malayalam}
അല്ലാഹു തന്റെ ദൂതന്ന് സത്യനിഷ്ഠമായ സ്വപ്നം കാണിക്കുകയും അത് യാഥാര്‍ഥ്യമാക്കുകയും ചെയ്തിരിക്കുന്നു. അല്ലാഹു ഉദ്ദേശിച്ചെങ്കില്‍ നിങ്ങള്‍ നിര്‍ഭയരായി തല മുണ്ഡനം ചെയ്തും മുടി വെട്ടിയും ഒന്നും പേടിക്കാതെ മസ്ജിദുല്‍ ഹറാമില്‍ പ്രവേശിക്കുക തന്നെ ചെയ്യും, തീര്‍ച്ച. നിങ്ങളറിയാത്തത് അവനറിഞ്ഞു. അതിനാല്‍ ഇതുകൂടാതെ തൊട്ടുടനെത്തന്നെ അവന്‍ നിങ്ങള്‍ക്കു മഹത്തായ വിജയം നല്‍കി.
\end{malayalam}}
\flushright{\begin{Arabic}
\quranayah[48][28]
\end{Arabic}}
\flushleft{\begin{malayalam}
സന്മാര്‍ഗവും സത്യവ്യവസ്ഥയുമായി തന്റെ ദൂതനെ നിയോഗിച്ചത് അവനാണ്. മറ്റെല്ലാ വ്യവസ്ഥകളെക്കാളും അതിനെ വിജയിപ്പിക്കാനാണിത്. ഇതിനൊക്കെ സാക്ഷിയായി അല്ലാഹു മതി.
\end{malayalam}}
\flushright{\begin{Arabic}
\quranayah[48][29]
\end{Arabic}}
\flushleft{\begin{malayalam}
മുഹമ്മദ് ദൈവദൂതനാണ്. അവനോടൊപ്പമുള്ളവര്‍ സത്യനിഷേധികളോട് കാര്‍ക്കശ്യം കാണിക്കുന്നവരാണ്; പരസ്പരം കാരുണ്യത്തോടെ പെരുമാറുന്നവരും. അല്ലാഹുവിന്റെ അനുഗ്രഹവും പ്രീതിയും പ്രതീക്ഷിച്ച് അവര്‍ നമിക്കുന്നതും സാഷ്ടാംഗം പ്രണമിക്കുന്നതും നിനക്കു കാണാം. പ്രണാമത്തിന്റെ പാടുകള്‍ അവരുടെ മുഖങ്ങളിലുണ്ട്. ഇതാണ് തൌറാതില്‍ അവരുടെ വര്‍ണന. ഇന്‍ജീലിലെ അവരുടെ ഉപമയോ, അത് ഇവ്വിധമത്രെ: ഒരു വിള. അത് അതിന്റെ കൂമ്പ് വെളിവാക്കി. പിന്നെ അതിനെ പുഷ്ടിപ്പെടുത്തി. അങ്ങനെ അത് കരുത്തുനേടി. അത് കര്‍ഷകരില്‍ കൌതുകമുണര്‍ത്തി അതിന്റെ കാണ്ഡത്തില്‍ നിവര്‍ന്നുനില്‍ക്കുന്നു. ഇതേപോലെ വിശ്വാസികളുടെ വളര്‍ച്ച സത്യനിഷേധികളെ രോഷം കൊള്ളിക്കുന്നു. അവരിലെ വിശ്വസിക്കുകയും സല്‍ക്കര്‍മങ്ങള്‍ പ്രവര്‍ത്തിക്കുകയും ചെയ്തവര്‍ക്ക് അല്ലാഹു പാപമോചനവും മഹത്തായ പ്രതിഫലവും വാഗ്ദാനം നല്‍കിയിരിക്കുന്നു.
\end{malayalam}}
\chapter{\textmalayalam{ഹുജുറാത് ( അറകള്‍ )}}
\begin{Arabic}
\Huge{\centerline{\basmalah}}\end{Arabic}
\flushright{\begin{Arabic}
\quranayah[49][1]
\end{Arabic}}
\flushleft{\begin{malayalam}
വിശ്വസിച്ചവരേ, നിങ്ങള്‍ അല്ലാഹുവിനെയും അവന്റെ ദൂതനെയും മുന്‍കടന്നൊന്നും ചെയ്യരുത്. നിങ്ങള്‍ അല്ലാഹുവെ സൂക്ഷിക്കുക. അല്ലാഹു എല്ലാം കേള്‍ക്കുന്നവനും അറിയുന്നവനുമാണ്.
\end{malayalam}}
\flushright{\begin{Arabic}
\quranayah[49][2]
\end{Arabic}}
\flushleft{\begin{malayalam}
വിശ്വസിച്ചവരേ, നിങ്ങള്‍ പ്രവാചകന്റെ ശബ്ദത്തെക്കാള്‍ ശബ്ദമുയര്‍ത്തരുത്. നിങ്ങളന്യോന്യം ഒച്ചവെക്കുന്നപോലെ അദ്ദേഹത്തോട് ഒച്ചവെക്കരുത്. നിങ്ങളറിയാതെ നിങ്ങളുടെ പ്രവര്‍ത്തനങ്ങള്‍ പാഴാവാതിരിക്കാനാണിത്.
\end{malayalam}}
\flushright{\begin{Arabic}
\quranayah[49][3]
\end{Arabic}}
\flushleft{\begin{malayalam}
ദൈവദൂതന്റെ അടുത്ത് തങ്ങളുടെ സ്വരം താഴ്ത്തുന്നവരുണ്ടല്ലോ; ഉറപ്പായും അവരുടെ മനസ്സുകളെയാണ് അല്ലാഹു ഭയഭക്തിക്കായി പരീക്ഷിച്ചൊരുക്കിയത്. അവര്‍ക്ക് പാപമോചനമുണ്ട്. അതിമഹത്തായ പ്രതിഫലവും.
\end{malayalam}}
\flushright{\begin{Arabic}
\quranayah[49][4]
\end{Arabic}}
\flushleft{\begin{malayalam}
മുറികള്‍ക്കു വെളിയില്‍ നിന്ന് നിന്നെ വിളിക്കുന്നവരിലേറെ പേരും ഒന്നും ചിന്തിച്ചു മനസ്സിലാക്കാത്തവരാണ്.
\end{malayalam}}
\flushright{\begin{Arabic}
\quranayah[49][5]
\end{Arabic}}
\flushleft{\begin{malayalam}
നീ അവരുടെ അടുത്തേക്ക് വരുംവരെ അവര്‍ ക്ഷമയോടെ കാത്തിരുന്നുവെങ്കില്‍ അതായിരുന്നു അവര്‍ക്കുത്തമം. അല്ലാഹു ഏറെ പൊറുക്കുന്നവനും ദയാപരനുമല്ലോ.
\end{malayalam}}
\flushright{\begin{Arabic}
\quranayah[49][6]
\end{Arabic}}
\flushleft{\begin{malayalam}
വിശ്വസിച്ചവരേ, വല്ല കുബുദ്ധിയും എന്തെങ്കിലും വാര്‍ത്തയുമായി നിങ്ങളുടെ അടുത്ത് വന്നാല്‍ നിജസ്ഥിതി വ്യക്തമായി അന്വേഷിച്ചറിയുക. കാര്യമറിയാതെ ഏതെങ്കിലും ജനതക്ക് നിങ്ങള്‍ വിപത്ത് വരുത്താതിരിക്കാനാണിത്. അങ്ങനെ ആ ചെയ്തതിന്റെ പേരില്‍ നിങ്ങള്‍ ഖേദിക്കാതിരിക്കാനും.
\end{malayalam}}
\flushright{\begin{Arabic}
\quranayah[49][7]
\end{Arabic}}
\flushleft{\begin{malayalam}
അറിയുക: നിങ്ങള്‍ക്കിടയില്‍ ദൈവദൂതനുണ്ട്. പല കാര്യങ്ങളിലും അദ്ദേഹം നിങ്ങളെ അനുസരിക്കുന്നുവെങ്കില്‍ നിങ്ങളതിന്റെ പേരില്‍ ക്ളേശിക്കേണ്ടിവരും. എന്നാല്‍ അല്ലാഹു സത്യവിശ്വാസത്തെ നിങ്ങള്‍ക്ക് ഏറെ പ്രിയംകരമാക്കിയിരിക്കുന്നു. അതിനെ നിങ്ങളുടെ മനസ്സുകള്‍ക്ക് അലംകൃതവുമാക്കിയിരിക്കുന്നു. സത്യനിഷേധവും തെമ്മാടിത്തവും ധിക്കാരവും നിങ്ങള്‍ക്കവന്‍ ഏറെ വെറുപ്പുള്ളതാക്കുകയും ചെയ്തിരിക്കുന്നു. അത്തരക്കാരാകുന്നു നേര്‍വഴി പ്രാപിച്ചവര്‍.
\end{malayalam}}
\flushright{\begin{Arabic}
\quranayah[49][8]
\end{Arabic}}
\flushleft{\begin{malayalam}
അത് അല്ലാഹുവില്‍നിന്നുള്ള ഔദാര്യവും അനുഗ്രഹവുമാണ്. അല്ലാഹു എല്ലാം അറിയുന്നവനും യുക്തിജ്ഞനുമാണ്.
\end{malayalam}}
\flushright{\begin{Arabic}
\quranayah[49][9]
\end{Arabic}}
\flushleft{\begin{malayalam}
സത്യവിശ്വാസികളിലെ രണ്ടു വിഭാഗം പരസ്പരം പോരടിച്ചാല്‍ നിങ്ങള്‍ അവര്‍ക്കിടയില്‍ സന്ധിയുണ്ടാക്കുക. പിന്നെ അവരിലൊരു വിഭാഗം മറു വിഭാഗത്തിനെതിരെ അതിക്രമം കാട്ടിയാല്‍ അതിക്രമം കാണിച്ചവര്‍ക്കെതിരെ നിങ്ങള്‍ യുദ്ധം ചെയ്യുക; അവര്‍ അല്ലാഹുവിന്റെ കല്‍പനയിലേക്ക് മടങ്ങിവരും വരെ. അവര്‍ മടങ്ങി വരികയാണെങ്കില്‍ നിങ്ങള്‍ അവര്‍ക്കിടയില്‍ നീതിപൂര്‍വം സന്ധിയുണ്ടാക്കുക. നീതി പാലിക്കുക. നീതി പാലിക്കുന്നവരെ അല്ലാഹു ഇഷ്ടപ്പെടുന്നു.
\end{malayalam}}
\flushright{\begin{Arabic}
\quranayah[49][10]
\end{Arabic}}
\flushleft{\begin{malayalam}
സത്യവിശ്വാസികള്‍ പരസ്പരം സഹോദരങ്ങളാണ്. അതിനാല്‍ നിങ്ങള്‍ നിങ്ങളുടെ സഹോദരങ്ങള്‍ക്കിടയില്‍ ഐക്യമുണ്ടാക്കുക. നിങ്ങള്‍ അല്ലാഹുവെ സൂക്ഷിക്കുക. നിങ്ങള്‍ക്ക് കാരുണ്യം കിട്ടിയേക്കും.
\end{malayalam}}
\flushright{\begin{Arabic}
\quranayah[49][11]
\end{Arabic}}
\flushleft{\begin{malayalam}
സത്യവിശ്വാസികളേ, ഒരു ജനത മറ്റൊരു ജനതയെ പരിഹസിക്കരുത്. പരിഹസിക്കപ്പെടുന്നവര്‍ പരിഹസിക്കുന്നവരെക്കാള്‍ നല്ലവരായേക്കാം. സ്ത്രീകള്‍ സ്ത്രീകളെയും പരിഹസിക്കരുത്. പരിഹസിക്കപ്പെടുന്നവര്‍ പരിഹസിക്കുന്നവരെക്കാള്‍ ഉത്തമകളായേക്കാം. നിങ്ങളന്യോന്യം കുത്തുവാക്കു പറയരുത്. പരിഹാസപ്പേരുകളുപയോഗിച്ച് പരസ്പരം അപമാനിക്കരുത്. സത്യവിശ്വാസം സ്വീകരിച്ചശേഷം അധര്‍മത്തിന്റെ പേരുപയോഗിക്കുന്നത് വളരെ നീചം തന്നെ. ആര്‍ പശ്ചാത്തപിക്കുന്നില്ലയോ അവര്‍ തന്നെയാണ് അക്രമികള്‍.
\end{malayalam}}
\flushright{\begin{Arabic}
\quranayah[49][12]
\end{Arabic}}
\flushleft{\begin{malayalam}
വിശ്വസിച്ചവരേ, ഊഹങ്ങളേറെയും വര്‍ജിക്കുക. ഉറപ്പായും ഊഹങ്ങളില്‍ ചിലത് കുറ്റമാണ്. നിങ്ങള്‍ രഹസ്യം ചുഴിഞ്ഞന്വേഷിക്കരുത്. നിങ്ങളിലാരും മറ്റുള്ളവരെപ്പറ്റി അവരുടെ അസാന്നിധ്യത്തില്‍ മോശമായി സംസാരിക്കരുത്. മരിച്ചുകിടക്കുന്ന സഹോദരന്റെ മാംസം തിന്നാന്‍ നിങ്ങളാരെങ്കിലും ഇഷ്ടപ്പെടുമോ? തീര്‍ച്ചയായും നിങ്ങളത് വെറുക്കുന്നു. നിങ്ങള്‍ അല്ലാഹുവെ സൂക്ഷിക്കുക. അല്ലാഹു പശ്ചാത്താപം സ്വീകരിക്കുന്നവനും ദയാപരനുമല്ലോ.
\end{malayalam}}
\flushright{\begin{Arabic}
\quranayah[49][13]
\end{Arabic}}
\flushleft{\begin{malayalam}
മനുഷ്യരേ, നിങ്ങളെ നാം ഒരാണില്‍നിന്നും പെണ്ണില്‍നിന്നുമാണ് സൃഷ്ടിച്ചത്. നിങ്ങളെ വിവിധ വിഭാഗങ്ങളും ഗോത്രങ്ങളുമാക്കിയത് നിങ്ങളന്യോന്യം തിരിച്ചറിയാനാണ്. അല്ലാഹുവിന്റെ അടുത്ത് നിങ്ങളിലേറ്റം ആദരണീയന്‍ നിങ്ങളില്‍ കൂടുതല്‍ സൂക്ഷ്മതയുള്ളവനാണ്; തീര്‍ച്ച. അല്ലാഹു സര്‍വജ്ഞനും സൂക്ഷ്മജ്ഞനുമാകുന്നു.
\end{malayalam}}
\flushright{\begin{Arabic}
\quranayah[49][14]
\end{Arabic}}
\flushleft{\begin{malayalam}
ഗ്രാമീണരായ അറബികള്‍ അവകാശപ്പെടുന്നു: "ഞങ്ങളും വിശ്വസിച്ചിരിക്കുന്നു." പറയുക: നിങ്ങള്‍ വിശ്വസിച്ചിട്ടില്ല. എന്നാല്‍ “ഞങ്ങള്‍ കീഴൊതുങ്ങിയിരിക്കുന്നു”വെന്ന് നിങ്ങള്‍ പറഞ്ഞുകൊള്ളുക. വിശ്വാസം നിങ്ങളുടെ മനസ്സുകളില്‍ പ്രവേശിച്ചിട്ടില്ല. നിങ്ങള്‍ അല്ലാഹുവെയും അവന്റെ ദൂതനെയും അനുസരിക്കുന്നുവെങ്കില്‍ നിങ്ങളുടെ കര്‍മഫലങ്ങളില്‍ അവനൊരു കുറവും വരുത്തുകയില്ല. അല്ലാഹു ഏറെ പൊറുക്കുന്നവനും ദയാപരനുമാണ്.
\end{malayalam}}
\flushright{\begin{Arabic}
\quranayah[49][15]
\end{Arabic}}
\flushleft{\begin{malayalam}
തീര്‍ച്ചയായും അല്ലാഹുവിലും അവന്റെ ദൂതനിലും വിശ്വസിക്കുകയും പിന്നെ അതില്‍ അശേഷം സംശയിക്കാതിരിക്കുകയും തങ്ങളുടെ സമ്പത്തും ശരീരവുമുപയോഗിച്ച് ദൈവമാര്‍ഗത്തില്‍ സമരം നടത്തുകയും ചെയ്തവര്‍ മാത്രമാണ് സത്യവിശ്വാസികള്‍. സത്യസന്ധരും അവര്‍തന്നെ.
\end{malayalam}}
\flushright{\begin{Arabic}
\quranayah[49][16]
\end{Arabic}}
\flushleft{\begin{malayalam}
ചോദിക്കുക: നിങ്ങള്‍ നിങ്ങളുടെ മതത്തെ അല്ലാഹുവിന് പഠിപ്പിച്ചു കൊടുക്കുകയാണോ? അല്ലാഹുവോ, ആകാശഭൂമികളിലുള്ളവയൊക്കെയുമറിയുന്നു. അല്ലാഹു എല്ലാ കാര്യത്തെക്കുറിച്ചും നന്നായറിയുന്നവനാണ്.
\end{malayalam}}
\flushright{\begin{Arabic}
\quranayah[49][17]
\end{Arabic}}
\flushleft{\begin{malayalam}
തങ്ങള്‍ ഇസ്ലാം സ്വീകരിച്ചുവെന്നത് നിന്നോടുള്ള ഔദാര്യമായി അവര്‍ എടുത്തു കാണിക്കുന്നു. പറയുക: നിങ്ങള്‍ ഇസ്ലാം സ്വീകരിച്ചത് എന്നോടുള്ള ഔദാര്യമായി എടുത്ത് കാണിക്കരുത്. യഥാര്‍ഥത്തില്‍ നിങ്ങളെ വിശ്വാസത്തിലേക്ക് വഴികാണിക്കുക വഴി അല്ലാഹു നിങ്ങളോട് ഔദാര്യം കാണിച്ചിരിക്കുകയാണ്. നിങ്ങള്‍ സത്യവാന്മാരെങ്കില്‍ ഇതംഗീകരിക്കുക.
\end{malayalam}}
\flushright{\begin{Arabic}
\quranayah[49][18]
\end{Arabic}}
\flushleft{\begin{malayalam}
ആകാശഭൂമികളില്‍ മറഞ്ഞിരിക്കുന്നതെല്ലാം അല്ലാഹു അറിയുന്നു; നിങ്ങള്‍ ചെയ്യുന്നതൊക്കെ സൂക്ഷ്മമായി വീക്ഷിക്കുന്നവനാണ് അല്ലാഹു.
\end{malayalam}}
\chapter{\textmalayalam{ഖാഫ്}}
\begin{Arabic}
\Huge{\centerline{\basmalah}}\end{Arabic}
\flushright{\begin{Arabic}
\quranayah[50][1]
\end{Arabic}}
\flushleft{\begin{malayalam}
ഖാഫ്. ഉല്‍കൃഷ്ടമായ ഖുര്‍ആന്‍ സാക്ഷി.
\end{malayalam}}
\flushright{\begin{Arabic}
\quranayah[50][2]
\end{Arabic}}
\flushleft{\begin{malayalam}
തങ്ങളില്‍നിന്നു തന്നെയുള്ള ഒരു മുന്നറിയിപ്പുകാരന്‍ അവരിലേക്കു വന്നതുകാരണം അവര്‍ അദ്ഭുതം കൂറുകയാണ്. അങ്ങനെ സത്യനിഷേധികള്‍ പറഞ്ഞു: "ഇതു വളരെ വിസ്മയകരമായ കാര്യം തന്നെ.
\end{malayalam}}
\flushright{\begin{Arabic}
\quranayah[50][3]
\end{Arabic}}
\flushleft{\begin{malayalam}
"നാം മരിച്ചു മണ്ണായ ശേഷം മടങ്ങിവരികയോ? ആ മടക്കം അസാധ്യം തന്നെ."
\end{malayalam}}
\flushright{\begin{Arabic}
\quranayah[50][4]
\end{Arabic}}
\flushleft{\begin{malayalam}
അവരില്‍നിന്നു ഭൂമി കുറവു വരുത്തിക്കൊണ്ടിരിക്കുന്നത് നാം അറിഞ്ഞിട്ടുണ്ട്. നമ്മുടെ വശം എല്ലാം സൂക്ഷ്മമായുള്ള ഗ്രന്ഥവുമുണ്ട്.
\end{malayalam}}
\flushright{\begin{Arabic}
\quranayah[50][5]
\end{Arabic}}
\flushleft{\begin{malayalam}
എന്നാല്‍ സത്യം വന്നെത്തിയപ്പോള്‍ അവരതിനെ തള്ളിപ്പറഞ്ഞു. അങ്ങനെ അവര്‍ ആശയക്കുഴപ്പത്തിലായി.
\end{malayalam}}
\flushright{\begin{Arabic}
\quranayah[50][6]
\end{Arabic}}
\flushleft{\begin{malayalam}
തങ്ങളുടെ മീതെയുള്ള മാനത്തെ അവര്‍ നോക്കിക്കാണുന്നില്ലേ? എങ്ങനെയാണ് നാമത് നിര്‍മിക്കുകയും അലങ്കരിക്കുകയും ചെയ്തതെന്ന്? അതിലൊരു വിടവുമില്ല.
\end{malayalam}}
\flushright{\begin{Arabic}
\quranayah[50][7]
\end{Arabic}}
\flushleft{\begin{malayalam}
ഭൂമിയോ; അതിനെ നാം വിശാലമാക്കി വിരിച്ചിരിക്കുന്നു. നാമതില്‍ മലകളെ ഉറപ്പിച്ചു. കൌതുകകരങ്ങളായ സകലയിനം സസ്യങ്ങള്‍ മുളപ്പിക്കുകയും ചെയ്തു.
\end{malayalam}}
\flushright{\begin{Arabic}
\quranayah[50][8]
\end{Arabic}}
\flushleft{\begin{malayalam}
പശ്ചാത്തപിച്ചു മടങ്ങുന്ന ദാസന്മാര്‍ക്ക് ഉള്‍ക്കാഴ്ചയും ഉദ്ബോധനവും നല്‍കാനാണ് ഇതൊക്കെയും.
\end{malayalam}}
\flushright{\begin{Arabic}
\quranayah[50][9]
\end{Arabic}}
\flushleft{\begin{malayalam}
മാനത്തുനിന്നു നാം അനുഗൃഹീതമായ മഴ പെയ്യിച്ചു. അങ്ങനെ അതുവഴി വിവിധയിനം തോട്ടങ്ങളും കൊയ്തെടുക്കാന്‍ പറ്റുന്ന ധാന്യങ്ങളും ഉല്‍പാദിപ്പിച്ചു.
\end{malayalam}}
\flushright{\begin{Arabic}
\quranayah[50][10]
\end{Arabic}}
\flushleft{\begin{malayalam}
അട്ടിയട്ടിയായി പഴക്കുലകളുള്ള ഉയര്‍ന്നുനില്‍ക്കുന്ന ഈത്തപ്പനകളും;
\end{malayalam}}
\flushright{\begin{Arabic}
\quranayah[50][11]
\end{Arabic}}
\flushleft{\begin{malayalam}
നമ്മുടെ അടിമകള്‍ക്ക് ആഹാരമായി. ആ മഴമൂലം മൃതമായ നാടിനെ ജീവസ്സുറ്റതാക്കി. അങ്ങനെത്തന്നെയാണ് ഉയിര്‍ത്തെഴുന്നേല്‍പ്.
\end{malayalam}}
\flushright{\begin{Arabic}
\quranayah[50][12]
\end{Arabic}}
\flushleft{\begin{malayalam}
അവര്‍ക്കു മുമ്പെ നൂഹിന്റെ ജനതയും റസ്സുകാരും ഥമൂദ് ഗോത്രവും സത്യത്തെ നിഷേധിച്ചു;
\end{malayalam}}
\flushright{\begin{Arabic}
\quranayah[50][13]
\end{Arabic}}
\flushleft{\begin{malayalam}
ആദ് സമുദായവും ഫിര്‍ഔനും ലൂത്തിന്റെ സഹോദരങ്ങളും.
\end{malayalam}}
\flushright{\begin{Arabic}
\quranayah[50][14]
\end{Arabic}}
\flushleft{\begin{malayalam}
ഐക്ക നിവാസികളും തുബ്ബഇന്റെ ജനതയും അതു തന്നെ ചെയ്തു. അവരൊക്കെ ദൈവദൂതന്മാരെ തള്ളിപ്പറഞ്ഞു. അങ്ങനെ എന്റെ മുന്നറിയിപ്പ് അവരില്‍ യാഥാര്‍ഥ്യമായി പുലര്‍ന്നു.
\end{malayalam}}
\flushright{\begin{Arabic}
\quranayah[50][15]
\end{Arabic}}
\flushleft{\begin{malayalam}
ആദ്യ സൃഷ്ടികാരണം നാം തളര്‍ന്നെന്നോ? അല്ല; അവര്‍ പുതിയ സൃഷ്ടിപ്പിനെ സംബന്ധിച്ച് സംശയത്തിലാണ്.
\end{malayalam}}
\flushright{\begin{Arabic}
\quranayah[50][16]
\end{Arabic}}
\flushleft{\begin{malayalam}
നിശ്ചയമായും നാമാണ് മനുഷ്യനെ സൃഷ്ടിച്ചത്. അവന്റെ മനസ്സ് മന്ത്രിച്ചുകൊണ്ടിരിക്കുന്നതൊക്കെയും നാം നന്നായറിയുന്നു. അവന്റെ കണ്ഠനാഡിയെക്കാള്‍ അവനോട് അടുത്തവനാണ് നാം.
\end{malayalam}}
\flushright{\begin{Arabic}
\quranayah[50][17]
\end{Arabic}}
\flushleft{\begin{malayalam}
വലതു ഭാഗത്തും ഇടതു ഭാഗത്തും ഇരുന്ന് ഏറ്റുവാങ്ങുന്ന രണ്ടുപേര്‍ എല്ലാം ഏറ്റുവാങ്ങുന്ന കാര്യം ഓര്‍ക്കുക.
\end{malayalam}}
\flushright{\begin{Arabic}
\quranayah[50][18]
\end{Arabic}}
\flushleft{\begin{malayalam}
അവനോടൊപ്പം ഒരുങ്ങി നില്‍ക്കുന്ന നിരീക്ഷകരില്ലാതെ അവനൊരു വാക്കും ഉച്ചരിക്കുന്നില്ല.
\end{malayalam}}
\flushright{\begin{Arabic}
\quranayah[50][19]
\end{Arabic}}
\flushleft{\begin{malayalam}
മരണവെപ്രാളം യാഥാര്‍ഥ്യമായി ഭവിക്കുന്നു. നീ തെന്നിമാറാന്‍ ശ്രമിക്കുന്നതെന്തോ അതാണിത്.
\end{malayalam}}
\flushright{\begin{Arabic}
\quranayah[50][20]
\end{Arabic}}
\flushleft{\begin{malayalam}
കാഹളം ഊതപ്പെടും. അതാണ് താക്കീതിന്റെ ദിനം.
\end{malayalam}}
\flushright{\begin{Arabic}
\quranayah[50][21]
\end{Arabic}}
\flushleft{\begin{malayalam}
അന്ന് എല്ലാവരും വന്നെത്തും. നയിച്ച് കൊണ്ട് വരുന്നവനും സാക്ഷിയും കൂടെയുണ്ടാവും.
\end{malayalam}}
\flushright{\begin{Arabic}
\quranayah[50][22]
\end{Arabic}}
\flushleft{\begin{malayalam}
അന്ന് അവരോട് പറയും: തീര്‍ച്ചയായും നീ ഇതേക്കുറിച്ച് അശ്രദ്ധനായിരുന്നു; എന്നാല്‍ നാമിപ്പോള്‍ നിന്നില്‍നിന്ന് ആ മറ എടുത്തുമാറ്റിയിരിക്കുന്നു. അതിനാല്‍ നിന്റെ കാഴ്ച ഇന്ന് മൂര്‍ച്ചയേറിയതത്രെ.
\end{malayalam}}
\flushright{\begin{Arabic}
\quranayah[50][23]
\end{Arabic}}
\flushleft{\begin{malayalam}
അവന്റെ കൂടെയുള്ള മലക്ക് പറയും: ഇതാ ഈ കര്‍മപുസ്തകമാണ് എന്റെ വശം തയ്യാറുള്ളത്.
\end{malayalam}}
\flushright{\begin{Arabic}
\quranayah[50][24]
\end{Arabic}}
\flushleft{\begin{malayalam}
അല്ലാഹു കല്പിക്കും: "സത്യനിഷേധിയും ധിക്കാരിയുമായ ഏവരെയും നിങ്ങളിരുവരും ചേര്‍ന്ന് നരകത്തിലിടുക.
\end{malayalam}}
\flushright{\begin{Arabic}
\quranayah[50][25]
\end{Arabic}}
\flushleft{\begin{malayalam}
"നന്മയെ തടഞ്ഞവനും അതിക്രമിയും സന്ദേഹിയുമായ ഏവരെയും.
\end{malayalam}}
\flushright{\begin{Arabic}
\quranayah[50][26]
\end{Arabic}}
\flushleft{\begin{malayalam}
"അല്ലാഹുവോടൊപ്പം വേറെ ദൈവങ്ങളെ കല്‍പിച്ചവനെയും. നിങ്ങളവനെ കഠിനശിക്ഷയിലിടുക."
\end{malayalam}}
\flushright{\begin{Arabic}
\quranayah[50][27]
\end{Arabic}}
\flushleft{\begin{malayalam}
അവന്റെ കൂട്ടാളിയായ പിശാച് പറയും: ഞങ്ങളുടെ നാഥാ! ഞാനിവനെ വഴിപിഴപ്പിച്ചിട്ടില്ല. എന്നാലിവന്‍ സ്വയം തന്നെ വളരെയേറെ വഴികേടിലായിരുന്നു.
\end{malayalam}}
\flushright{\begin{Arabic}
\quranayah[50][28]
\end{Arabic}}
\flushleft{\begin{malayalam}
അല്ലാഹു പറയും: നിങ്ങള്‍ എന്റെ മുന്നില്‍ വെച്ച് തര്‍ക്കിക്കേണ്ട. ഞാന്‍ നേരത്തെത്തന്നെ നിങ്ങള്‍ക്ക് താക്കീത് തന്നിട്ടുണ്ട്.
\end{malayalam}}
\flushright{\begin{Arabic}
\quranayah[50][29]
\end{Arabic}}
\flushleft{\begin{malayalam}
എന്റെ അടുത്ത് വാക്ക് മാറ്റമില്ല. ഞാന്‍ എന്റെ ദാസന്മാരോട് ഒട്ടും അനീതി കാട്ടുന്നതുമല്ല.
\end{malayalam}}
\flushright{\begin{Arabic}
\quranayah[50][30]
\end{Arabic}}
\flushleft{\begin{malayalam}
നാം നരകത്തോട് ചോദിക്കുന്ന ദിനം: "നീ നിറഞ്ഞു കഴിഞ്ഞോ?" നരകം തിരിച്ചു ചോദിക്കും: "ഇനിയുമുണ്ടോ?"
\end{malayalam}}
\flushright{\begin{Arabic}
\quranayah[50][31]
\end{Arabic}}
\flushleft{\begin{malayalam}
ഭക്തന്മാര്‍ക്കായി സ്വര്‍ഗം അടുത്തുകൊണ്ടുവരും. ഒട്ടും ദൂരെയല്ലാത്ത വിധം.
\end{malayalam}}
\flushright{\begin{Arabic}
\quranayah[50][32]
\end{Arabic}}
\flushleft{\begin{malayalam}
സ്രഷ്ടാവിലേക്ക് മടങ്ങുകയും സൂക്ഷ്മത പുലര്‍ത്തുകയും ചെയ്യുന്ന ഏവര്‍ക്കും വാഗ്ദാനം ചെയ്യപ്പെട്ടതാണിത്.
\end{malayalam}}
\flushright{\begin{Arabic}
\quranayah[50][33]
\end{Arabic}}
\flushleft{\begin{malayalam}
അഥവാ, പരമകാരുണികനെ നേരില്‍ കാണാതെതന്നെ ഭയപ്പെടുകയും പശ്ചാത്താപ പൂര്‍ണമായ മനസ്സോടെ വന്നെത്തുകയും ചെയ്തവന്.
\end{malayalam}}
\flushright{\begin{Arabic}
\quranayah[50][34]
\end{Arabic}}
\flushleft{\begin{malayalam}
സമാധാനത്തോടെ നിങ്ങളതില്‍ പ്രവേശിച്ചുകൊള്ളുക. നിത്യവാസത്തിനുള്ള ദിനമാണത്.
\end{malayalam}}
\flushright{\begin{Arabic}
\quranayah[50][35]
\end{Arabic}}
\flushleft{\begin{malayalam}
അവര്‍ക്കവിടെ അവരാഗ്രഹിക്കുന്നതൊക്കെ ലഭിക്കും. നമ്മുടെ വശം വേറെയും ധാരാളമുണ്ട്.
\end{malayalam}}
\flushright{\begin{Arabic}
\quranayah[50][36]
\end{Arabic}}
\flushleft{\begin{malayalam}
അവര്‍ക്കുമുമ്പ് എത്ര തലമുറകളെയാണ് നാം നശിപ്പിച്ചത്. അവര്‍ ഇവരെക്കാള്‍ വളരെയേറെ ശക്തരായിരുന്നു. അങ്ങനെ അവര്‍ നാടായ നാടുകളിലൊക്കെ അന്വേഷിച്ചുനോക്കി. രക്ഷപ്പെടാന്‍ വല്ല ഇടവും ലഭിക്കുമോയെന്ന്.
\end{malayalam}}
\flushright{\begin{Arabic}
\quranayah[50][37]
\end{Arabic}}
\flushleft{\begin{malayalam}
ഹൃദയമുള്ളവന്നും മനസ്സറിഞ്ഞ് കേള്‍ക്കുന്നവന്നും ഇതില്‍ ഓര്‍ക്കാനേറെയുണ്ട്.
\end{malayalam}}
\flushright{\begin{Arabic}
\quranayah[50][38]
\end{Arabic}}
\flushleft{\begin{malayalam}
ആകാശഭൂമികളെയും അവയ്ക്കിടയിലുള്ളവയെയും നാം ആറു നാളുകളിലായി സൃഷ്ടിച്ചു. അതുകൊണ്ടൊന്നും നമുക്കൊട്ടും ക്ഷീണം ബാധിച്ചിട്ടില്ല.
\end{malayalam}}
\flushright{\begin{Arabic}
\quranayah[50][39]
\end{Arabic}}
\flushleft{\begin{malayalam}
അതിനാല്‍ അവര്‍ പറയുന്നതൊക്കെ ക്ഷമിക്കുക. സൂര്യോദയത്തിനും അസ്തമയത്തിനും മുമ്പെ നിന്റെ നാഥനെ വാഴ്ത്തുക. ഒപ്പം കീര്‍ത്തിക്കുകയും ചെയ്യുക.
\end{malayalam}}
\flushright{\begin{Arabic}
\quranayah[50][40]
\end{Arabic}}
\flushleft{\begin{malayalam}
രാവിലും സ്വല്‍പസമയം അവനെ കീര്‍ത്തിക്കുക. സാഷ്ടാംഗാനന്തരവും.
\end{malayalam}}
\flushright{\begin{Arabic}
\quranayah[50][41]
\end{Arabic}}
\flushleft{\begin{malayalam}
അടുത്തൊരിടത്തുനിന്ന് വിളിച്ചു പറയുന്നവന്‍ വിളംബരം ചെയ്യുന്ന ദിനത്തിന്നായി കാതോര്‍ക്കുക.
\end{malayalam}}
\flushright{\begin{Arabic}
\quranayah[50][42]
\end{Arabic}}
\flushleft{\begin{malayalam}
ആ ഘോരനാദം ഒരു യാഥാര്‍ഥ്യമായി അവര്‍ കേട്ടനുഭവിക്കും ദിനം. അത് പുറപ്പാടിന്റെ ദിനമത്രെ.
\end{malayalam}}
\flushright{\begin{Arabic}
\quranayah[50][43]
\end{Arabic}}
\flushleft{\begin{malayalam}
ജീവിപ്പിക്കുന്നതും മരിപ്പിക്കുന്നതും നാമാണ്. തിരിച്ചുവരവും നമ്മിലേക്കു തന്നെ.
\end{malayalam}}
\flushright{\begin{Arabic}
\quranayah[50][44]
\end{Arabic}}
\flushleft{\begin{malayalam}
ഭൂമി പിളര്‍ന്ന് മനുഷ്യര്‍ പുറത്ത് കടന്ന് അതിവേഗം ഓടിവരുന്ന ദിനം. അവ്വിധം അവരെ ഒരുമിച്ചു കൂട്ടല്‍ നമുക്ക് വളരെ എളുപ്പമാണ്.
\end{malayalam}}
\flushright{\begin{Arabic}
\quranayah[50][45]
\end{Arabic}}
\flushleft{\begin{malayalam}
അവര്‍ പറഞ്ഞുകൊണ്ടിരിക്കുന്നതിനെപ്പറ്റി നാം നന്നായറിയുന്നു. അവരുടെ മേല്‍ നിര്‍ബന്ധം ചെലുത്തേണ്ട ആവശ്യം നിനക്കില്ല. അതിനാല്‍ എന്റെ താക്കീത് ഭയപ്പെടുന്നവരെ നീ ഈ ഖുര്‍ആന്‍ വഴി ഉദ്ബോധിപ്പിക്കുക.
\end{malayalam}}
\chapter{\textmalayalam{ദാരിയാത് ( വിതറുന്നവ )}}
\begin{Arabic}
\Huge{\centerline{\basmalah}}\end{Arabic}
\flushright{\begin{Arabic}
\quranayah[51][1]
\end{Arabic}}
\flushleft{\begin{malayalam}
പൊടി പറത്തുന്നവ സാക്ഷി.
\end{malayalam}}
\flushright{\begin{Arabic}
\quranayah[51][2]
\end{Arabic}}
\flushleft{\begin{malayalam}
കനത്ത മേഘങ്ങളെ വഹിക്കുന്നവ സാക്ഷി.
\end{malayalam}}
\flushright{\begin{Arabic}
\quranayah[51][3]
\end{Arabic}}
\flushleft{\begin{malayalam}
തെന്നി നീങ്ങുന്നവ സാക്ഷി.
\end{malayalam}}
\flushright{\begin{Arabic}
\quranayah[51][4]
\end{Arabic}}
\flushleft{\begin{malayalam}
കാര്യങ്ങള്‍ വീതിച്ചു കൊടുക്കുന്നവ സാക്ഷി.
\end{malayalam}}
\flushright{\begin{Arabic}
\quranayah[51][5]
\end{Arabic}}
\flushleft{\begin{malayalam}
നിങ്ങള്‍ക്കു വാഗ്ദത്തം ചെയ്യപ്പെടുന്ന കാര്യം സത്യം തന്നെ; തീര്‍ച്ച.
\end{malayalam}}
\flushright{\begin{Arabic}
\quranayah[51][6]
\end{Arabic}}
\flushleft{\begin{malayalam}
ന്യായവിധി നടക്കുക തന്നെ ചെയ്യും.
\end{malayalam}}
\flushright{\begin{Arabic}
\quranayah[51][7]
\end{Arabic}}
\flushleft{\begin{malayalam}
വിവിധ സഞ്ചാരപഥങ്ങളുള്ള ആകാശം സാക്ഷി.
\end{malayalam}}
\flushright{\begin{Arabic}
\quranayah[51][8]
\end{Arabic}}
\flushleft{\begin{malayalam}
തീര്‍ച്ചയായും നിങ്ങള്‍ വ്യത്യസ്താ ഭിപ്രായക്കാരാണ്.
\end{malayalam}}
\flushright{\begin{Arabic}
\quranayah[51][9]
\end{Arabic}}
\flushleft{\begin{malayalam}
നേര്‍വഴിയില്‍ നിന്ന് അകന്നവന്‍ ഈ സത്യത്തില്‍ നിന്ന് വ്യതിചലിക്കുന്നു.
\end{malayalam}}
\flushright{\begin{Arabic}
\quranayah[51][10]
\end{Arabic}}
\flushleft{\begin{malayalam}
ഊഹങ്ങളെ അവലംബിക്കുന്നവര്‍ നശിച്ചതുതന്നെ.
\end{malayalam}}
\flushright{\begin{Arabic}
\quranayah[51][11]
\end{Arabic}}
\flushleft{\begin{malayalam}
അവരോ വിവരക്കേടില്‍ മതിമറന്നവര്‍.
\end{malayalam}}
\flushright{\begin{Arabic}
\quranayah[51][12]
\end{Arabic}}
\flushleft{\begin{malayalam}
അവര്‍ ചോദിക്കുന്നു, ന്യായവിധിയുടെ ദിനം എപ്പോഴെന്ന്!
\end{malayalam}}
\flushright{\begin{Arabic}
\quranayah[51][13]
\end{Arabic}}
\flushleft{\begin{malayalam}
അതോ, അവര്‍ നരകാഗ്നിയില്‍ എരിയുന്ന ദിനം തന്നെ.
\end{malayalam}}
\flushright{\begin{Arabic}
\quranayah[51][14]
\end{Arabic}}
\flushleft{\begin{malayalam}
അന്ന് അവരോട് പറയും: ഇതാ, നിങ്ങള്‍ക്കുള്ള ശിക്ഷ. ഇത് അനുഭവിച്ചുകൊള്ളുക. നിങ്ങള്‍ തിടുക്കം കാട്ടി ആവശ്യപ്പെട്ടുകൊണ്ടിരുന്നത് ഇതാണല്ലോ.
\end{malayalam}}
\flushright{\begin{Arabic}
\quranayah[51][15]
\end{Arabic}}
\flushleft{\begin{malayalam}
എന്നാല്‍ സൂക്ഷ്മത പാലിക്കുന്നവര്‍ സ്വര്‍ഗീയാരാമങ്ങളിലും അരുവികളിലുമായിരിക്കും.
\end{malayalam}}
\flushright{\begin{Arabic}
\quranayah[51][16]
\end{Arabic}}
\flushleft{\begin{malayalam}
തങ്ങളുടെ നാഥന്റെ വരദാനങ്ങള്‍ അനുഭവിക്കുന്നവരായി. അവര്‍ നേരത്തെ സദ്വൃത്തരായിരുന്നുവല്ലോ.
\end{malayalam}}
\flushright{\begin{Arabic}
\quranayah[51][17]
\end{Arabic}}
\flushleft{\begin{malayalam}
രാത്രിയില്‍ അല്‍പനേരമേ അവര്‍ ഉറങ്ങാറുണ്ടായിരുന്നുള്ളൂ.
\end{malayalam}}
\flushright{\begin{Arabic}
\quranayah[51][18]
\end{Arabic}}
\flushleft{\begin{malayalam}
അവര്‍ രാവിന്റെ ഒടുവുവേളകളില്‍ പാപമോചനം തേടുന്നവരുമായിരുന്നു.
\end{malayalam}}
\flushright{\begin{Arabic}
\quranayah[51][19]
\end{Arabic}}
\flushleft{\begin{malayalam}
അവരുടെ സമ്പാദ്യങ്ങളില്‍ ചോദിക്കുന്നവന്നും നിരാലംബനും അവകാശമുണ്ടായിരുന്നു.
\end{malayalam}}
\flushright{\begin{Arabic}
\quranayah[51][20]
\end{Arabic}}
\flushleft{\begin{malayalam}
ദൃഢവിശ്വാസികള്‍ക്ക് ഭൂമിയില്‍ നിരവധി തെളിവുകളുണ്ട്.
\end{malayalam}}
\flushright{\begin{Arabic}
\quranayah[51][21]
\end{Arabic}}
\flushleft{\begin{malayalam}
നിങ്ങളില്‍ തന്നെയുമുണ്ട്. എന്നിട്ടും നിങ്ങള്‍ അതൊന്നും കണ്ട് മനസ്സിലാക്കുന്നില്ലെന്നോ?
\end{malayalam}}
\flushright{\begin{Arabic}
\quranayah[51][22]
\end{Arabic}}
\flushleft{\begin{malayalam}
ആകാശത്തില്‍ നിങ്ങള്‍ക്ക് ഉപജീവനമുണ്ട്. നിങ്ങളെ താക്കീത് ചെയ്തുകൊണ്ടിരിക്കുന്ന ശിക്ഷയും.
\end{malayalam}}
\flushright{\begin{Arabic}
\quranayah[51][23]
\end{Arabic}}
\flushleft{\begin{malayalam}
ആകാശഭൂമികളുടെ നാഥന്‍ സാക്ഷി. നിങ്ങള്‍ സംസാരിച്ചുകൊണ്ടിരിക്കുന്നു എന്നപോലെ ഇത് സത്യമാകുന്നു.
\end{malayalam}}
\flushright{\begin{Arabic}
\quranayah[51][24]
\end{Arabic}}
\flushleft{\begin{malayalam}
ഇബ്റാഹീമിന്റെ ആദരണീയരായ അതിഥികളുടെ വിവരം നിനക്ക് വന്നെത്തിയോ?
\end{malayalam}}
\flushright{\begin{Arabic}
\quranayah[51][25]
\end{Arabic}}
\flushleft{\begin{malayalam}
അവരദ്ദേഹത്തിന്റെ അടുത്തുവന്ന സന്ദര്‍ഭം? അവരദ്ദേഹത്തിന് സലാം പറഞ്ഞു. അദ്ദേഹം പറഞ്ഞു: നിങ്ങള്‍ക്കും സലാം; അപരിചിതരാണല്ലോ.
\end{malayalam}}
\flushright{\begin{Arabic}
\quranayah[51][26]
\end{Arabic}}
\flushleft{\begin{malayalam}
അനന്തരം അദ്ദേഹം അതിവേഗം തന്റെ വീട്ടുകാരെ സമീപിച്ചു. അങ്ങനെ കൊഴുത്ത പശുക്കിടാവിനെ പാകം ചെയ്തുകൊണ്ടുവന്നു.
\end{malayalam}}
\flushright{\begin{Arabic}
\quranayah[51][27]
\end{Arabic}}
\flushleft{\begin{malayalam}
അതവരുടെ സമീപത്തുവെച്ചു. അദ്ദേഹം ചോദിച്ചു: നിങ്ങള്‍ തിന്നുന്നില്ലേ?
\end{malayalam}}
\flushright{\begin{Arabic}
\quranayah[51][28]
\end{Arabic}}
\flushleft{\begin{malayalam}
അപ്പോള്‍ അദ്ദേഹത്തിന് അവരെപ്പറ്റി ആശങ്ക തോന്നി. അവര്‍ പറഞ്ഞു: “പേടിക്കേണ്ട”. ജ്ഞാനിയായ ഒരു പുത്രന്റെ ജനനത്തെക്കുറിച്ച ശുഭവാര്‍ത്ത അവരദ്ദേഹത്തെ അറിയിച്ചു.
\end{malayalam}}
\flushright{\begin{Arabic}
\quranayah[51][29]
\end{Arabic}}
\flushleft{\begin{malayalam}
അപ്പോള്‍ അദ്ദേഹത്തിന്റെ ഭാര്യ ഒച്ചവെച്ച് ഓടിവന്നു. സ്വന്തം മുഖത്തടിച്ചുകൊണ്ട് അവര്‍ ചോദിച്ചു: "വന്ധ്യയായ ഈ കിഴവിക്കോ?”
\end{malayalam}}
\flushright{\begin{Arabic}
\quranayah[51][30]
\end{Arabic}}
\flushleft{\begin{malayalam}
അവര്‍ അറിയിച്ചു: "അതെ, അങ്ങനെ സംഭവിക്കുമെന്ന് നിന്റെ നാഥന്‍ അറിയിച്ചിരിക്കുന്നു. അവന്‍ യുക്തിമാനും അഭിജ്ഞനും തന്നെ; തീര്‍ച്ച.”
\end{malayalam}}
\flushright{\begin{Arabic}
\quranayah[51][31]
\end{Arabic}}
\flushleft{\begin{malayalam}
അദ്ദേഹം അന്വേഷിച്ചു: അല്ലയോ ദൂതന്മാരേ, നിങ്ങളുടെ യാത്രോദ്ദേശ്യം എന്താണ്?
\end{malayalam}}
\flushright{\begin{Arabic}
\quranayah[51][32]
\end{Arabic}}
\flushleft{\begin{malayalam}
അവര്‍ അറിയിച്ചു: "കുറ്റവാളികളായ ജനത്തിലേക്കാണ് ഞങ്ങളെ നിയോഗിച്ചിരിക്കുന്നത്.
\end{malayalam}}
\flushright{\begin{Arabic}
\quranayah[51][33]
\end{Arabic}}
\flushleft{\begin{malayalam}
"അവര്‍ക്കുമേല്‍ ചുട്ടെടുത്ത കളിമണ്‍കട്ട വാരിച്ചൊരിയാന്‍.
\end{malayalam}}
\flushright{\begin{Arabic}
\quranayah[51][34]
\end{Arabic}}
\flushleft{\begin{malayalam}
"അവ അതിക്രമികള്‍ക്കായി നിന്റെ നാഥന്റെ വശം പ്രത്യേകം അടയാളപ്പെടുത്തിവെച്ചവയാണ്.”
\end{malayalam}}
\flushright{\begin{Arabic}
\quranayah[51][35]
\end{Arabic}}
\flushleft{\begin{malayalam}
പിന്നെ അവിടെയുണ്ടായിരുന്ന സത്യവിശ്വാസികളെയെല്ലാം നാം രക്ഷപ്പെടുത്തി.
\end{malayalam}}
\flushright{\begin{Arabic}
\quranayah[51][36]
\end{Arabic}}
\flushleft{\begin{malayalam}
എന്നാല്‍ നാമവിടെ മുസ്ലിംകളുടേതായി ഒരു വീടല്ലാതൊന്നും കണ്ടില്ല.
\end{malayalam}}
\flushright{\begin{Arabic}
\quranayah[51][37]
\end{Arabic}}
\flushleft{\begin{malayalam}
നോവേറിയ ശിക്ഷയെ പേടിക്കുന്നവര്‍ക്ക് നാമവിടെ ഒരടയാളം ബാക്കിവെച്ചു.
\end{malayalam}}
\flushright{\begin{Arabic}
\quranayah[51][38]
\end{Arabic}}
\flushleft{\begin{malayalam}
മൂസായിലും നിങ്ങള്‍ക്ക് ദൃഷ്ടാന്തമുണ്ട്. വ്യക്തമായ തെളിവുമായി നാം അദ്ദേഹത്തെ ഫറവോന്റെ അടുത്തേക്കയച്ച സന്ദര്‍ഭം.
\end{malayalam}}
\flushright{\begin{Arabic}
\quranayah[51][39]
\end{Arabic}}
\flushleft{\begin{malayalam}
അവന്‍ തന്റെ കഴിവില്‍ ഗര്‍വ് നടിച്ച് പിന്തിരിഞ്ഞു. എന്നിട്ട് പറഞ്ഞു: ഇവനൊരു മായാജാലക്കാരന്‍; അല്ലെങ്കില്‍ ഭ്രാന്തന്‍.
\end{malayalam}}
\flushright{\begin{Arabic}
\quranayah[51][40]
\end{Arabic}}
\flushleft{\begin{malayalam}
അതിനാല്‍ അവനെയും അവന്റെ പട്ടാളത്തെയും നാം പിടികൂടി. പിന്നെ അവരെയൊക്കെ കടലിലെറിഞ്ഞു. അവന്‍ ആക്ഷേപാര്‍ഹന്‍ തന്നെ.
\end{malayalam}}
\flushright{\begin{Arabic}
\quranayah[51][41]
\end{Arabic}}
\flushleft{\begin{malayalam}
ആദ് ജനതയുടെ കാര്യത്തിലും നിങ്ങള്‍ക്ക് ദൃഷ്ടാന്തമുണ്ട്. വന്ധ്യമായ കാറ്റിനെ നാമവര്‍ക്കുനേരെ അയച്ച സന്ദര്‍ഭം.
\end{malayalam}}
\flushright{\begin{Arabic}
\quranayah[51][42]
\end{Arabic}}
\flushleft{\begin{malayalam}
തൊട്ടുഴിഞ്ഞ ഒന്നിനെയും അത് തുരുമ്പുപോലെ നുരുമ്പിച്ചതാക്കാതിരുന്നില്ല.
\end{malayalam}}
\flushright{\begin{Arabic}
\quranayah[51][43]
\end{Arabic}}
\flushleft{\begin{malayalam}
ഥമൂദിലും നിങ്ങള്‍ക്ക് ദൃഷ്ടാന്തമുണ്ട്. “ഒരു നിര്‍ണിത അവധി വരെ നിങ്ങള്‍ സുഖിച്ചു കൊള്ളുക” എന്ന് അവരോട ്പറഞ്ഞ സന്ദര്‍ഭം.
\end{malayalam}}
\flushright{\begin{Arabic}
\quranayah[51][44]
\end{Arabic}}
\flushleft{\begin{malayalam}
എന്നിട്ടും അവര്‍ തങ്ങളുടെ നാഥന്റെ കല്‍പനയെ ധിക്കരിച്ചു. അങ്ങനെ അവര്‍ നോക്കിനില്‍ക്കെ ഘോരമായൊരിടിനാദം അവരെ പിടികൂടി.
\end{malayalam}}
\flushright{\begin{Arabic}
\quranayah[51][45]
\end{Arabic}}
\flushleft{\begin{malayalam}
അപ്പോഴവര്‍ക്ക് എഴുന്നേല്‍ക്കാനോ രക്ഷാമാര്‍ഗം തേടാനോ കഴിഞ്ഞില്ല.
\end{malayalam}}
\flushright{\begin{Arabic}
\quranayah[51][46]
\end{Arabic}}
\flushleft{\begin{malayalam}
അവര്‍ക്കു മുമ്പെ നൂഹിന്റെ ജനതയെയും നാം നശിപ്പിച്ചിട്ടുണ്ട്. ഉറപ്പായും അവരും അധാര്‍മികരായിരുന്നു.
\end{malayalam}}
\flushright{\begin{Arabic}
\quranayah[51][47]
\end{Arabic}}
\flushleft{\begin{malayalam}
ആകാശത്തെ നാം കൈകളാല്‍ നിര്‍മിച്ചു. നാമതിനെ വികസിപ്പിച്ചുകൊണ്ടിരിക്കുകയാണ്.
\end{malayalam}}
\flushright{\begin{Arabic}
\quranayah[51][48]
\end{Arabic}}
\flushleft{\begin{malayalam}
ഭൂമിയെ നാം വിടര്‍ത്തി വിരിച്ചിരിക്കുന്നു. എത്ര വിശിഷ്ടമായി വിതാനിക്കുന്നവന്‍.
\end{malayalam}}
\flushright{\begin{Arabic}
\quranayah[51][49]
\end{Arabic}}
\flushleft{\begin{malayalam}
നാം എല്ലാ വസ്തുക്കളില്‍നിന്നും ഈരണ്ട് ഇണകളെ സൃഷ്ടിച്ചു. നിങ്ങള്‍ ചിന്തിച്ചറിയാന്‍.
\end{malayalam}}
\flushright{\begin{Arabic}
\quranayah[51][50]
\end{Arabic}}
\flushleft{\begin{malayalam}
അതിനാല്‍ നിങ്ങള്‍ അല്ലാഹുവിലേക്ക് ഓടിയെത്തുക. ഉറപ്പായും അവനില്‍നിന്ന് നിങ്ങളിലേക്കുള്ള തെളിഞ്ഞ താക്കീതുകാരനാണ് ഞാന്‍.
\end{malayalam}}
\flushright{\begin{Arabic}
\quranayah[51][51]
\end{Arabic}}
\flushleft{\begin{malayalam}
അല്ലാഹുവിനൊപ്പം മറ്റൊരു ദൈവത്തെയും സ്ഥാപിക്കാതിരിക്കുക. തീര്‍ച്ചയായും അവനില്‍നിന്ന് നിങ്ങള്‍ക്കുള്ള വ്യക്തമായ മുന്നറിയിപ്പു നല്‍കുന്നവനാണ് ഞാന്‍.
\end{malayalam}}
\flushright{\begin{Arabic}
\quranayah[51][52]
\end{Arabic}}
\flushleft{\begin{malayalam}
ഇവ്വിധം ഭ്രാന്തനെന്നോ മായാജാലക്കാരനെന്നോ ആക്ഷേപിക്കപ്പെടാത്ത ഒരൊറ്റ ദൈവദൂതനും ഇവര്‍ക്ക് മുമ്പുള്ളവരിലും വന്നിട്ടില്ല.
\end{malayalam}}
\flushright{\begin{Arabic}
\quranayah[51][53]
\end{Arabic}}
\flushleft{\begin{malayalam}
അവരൊക്കെയും അങ്ങനെ ചെയ്യാന്‍ അന്യോന്യം പറഞ്ഞുറപ്പിച്ചിരിക്കയാണോ? അല്ല; അവരൊക്കെയും അതിക്രമികളായ ജനം തന്നെ.
\end{malayalam}}
\flushright{\begin{Arabic}
\quranayah[51][54]
\end{Arabic}}
\flushleft{\begin{malayalam}
അതിനാല്‍ നീ അവരില്‍നിന്ന് പിന്മാറുക. എങ്കില്‍ നീ ആക്ഷേപാര്‍ഹനല്ല.
\end{malayalam}}
\flushright{\begin{Arabic}
\quranayah[51][55]
\end{Arabic}}
\flushleft{\begin{malayalam}
നീ ഉദ്ബോധനം തുടരുക. ഉറപ്പായും സത്യവിശ്വാസികള്‍ക്ക് ഉദ്ബോധനം ഉപകരിക്കും.
\end{malayalam}}
\flushright{\begin{Arabic}
\quranayah[51][56]
\end{Arabic}}
\flushleft{\begin{malayalam}
ജിന്നുകളെയും മനുഷ്യരെയും എനിക്കു വഴിപ്പെട്ടു ജീവിക്കാനല്ലാതെ ഞാന്‍ സൃഷ്ടിച്ചിട്ടില്ല.
\end{malayalam}}
\flushright{\begin{Arabic}
\quranayah[51][57]
\end{Arabic}}
\flushleft{\begin{malayalam}
ഞാന്‍ അവരില്‍നിന്ന് ഉപജീവനമൊന്നും കൊതിക്കുന്നില്ല. അവരെനിക്ക് തിന്നാന്‍ തരണമെന്നും ഞാനാഗ്രഹിക്കുന്നില്ല.
\end{malayalam}}
\flushright{\begin{Arabic}
\quranayah[51][58]
\end{Arabic}}
\flushleft{\begin{malayalam}
അല്ലാഹുവാണ് അന്നദാതാവ്, തീര്‍ച്ച. അവന്‍ അതിശക്തനും കരുത്തനും തന്നെ.
\end{malayalam}}
\flushright{\begin{Arabic}
\quranayah[51][59]
\end{Arabic}}
\flushleft{\begin{malayalam}
ഉറപ്പായും അക്രമം പ്രവര്‍ത്തിക്കുന്നവര്‍ക്ക് ശിക്ഷയുണ്ട്. അവരുടെ മുന്‍ഗാമികളായ കൂട്ടുകാര്‍ക്ക് കിട്ടിയ പോലുള്ള ശിക്ഷ. അതിനാല്‍ അവരെന്നോടതിനു തിടുക്കം കൂട്ടേണ്ടതില്ല.
\end{malayalam}}
\flushright{\begin{Arabic}
\quranayah[51][60]
\end{Arabic}}
\flushleft{\begin{malayalam}
സത്യനിഷേധികളോട് താക്കീത് നല്‍കിക്കൊണ്ടിരിക്കുന്ന ദിനമില്ലേ; അതവര്‍ക്ക് സര്‍വനാശത്തിന്റേതുതന്നെ.
\end{malayalam}}
\chapter{\textmalayalam{ത്വൂര്‍ ( ത്വൂര്‍ പര്‍വ്വതം)}}
\begin{Arabic}
\Huge{\centerline{\basmalah}}\end{Arabic}
\flushright{\begin{Arabic}
\quranayah[52][1]
\end{Arabic}}
\flushleft{\begin{malayalam}
ത്വൂര്‍ തന്നെ സാക്ഷി.
\end{malayalam}}
\flushright{\begin{Arabic}
\quranayah[52][2]
\end{Arabic}}
\flushleft{\begin{malayalam}
എഴുതിയ വേദപുസ്തകം സാക്ഷി-
\end{malayalam}}
\flushright{\begin{Arabic}
\quranayah[52][3]
\end{Arabic}}
\flushleft{\begin{malayalam}
വിടര്‍ത്തിവെച്ച തുകലില്‍ .
\end{malayalam}}
\flushright{\begin{Arabic}
\quranayah[52][4]
\end{Arabic}}
\flushleft{\begin{malayalam}
ജനനിബിഡമായ കഅ്ബാ മന്ദിരം സാക്ഷി.
\end{malayalam}}
\flushright{\begin{Arabic}
\quranayah[52][5]
\end{Arabic}}
\flushleft{\begin{malayalam}
ഉയരത്തിലുള്ള ആകാശം സാക്ഷി.
\end{malayalam}}
\flushright{\begin{Arabic}
\quranayah[52][6]
\end{Arabic}}
\flushleft{\begin{malayalam}
തിരതല്ലുന്ന സമുദ്രം സാക്ഷി.
\end{malayalam}}
\flushright{\begin{Arabic}
\quranayah[52][7]
\end{Arabic}}
\flushleft{\begin{malayalam}
നിശ്ചയം, നിന്റെ നാഥന്റെ ശിക്ഷ സംഭവിക്കുക തന്നെ ചെയ്യും.
\end{malayalam}}
\flushright{\begin{Arabic}
\quranayah[52][8]
\end{Arabic}}
\flushleft{\begin{malayalam}
അതിനെ തടുക്കുന്ന ആരുമില്ല.
\end{malayalam}}
\flushright{\begin{Arabic}
\quranayah[52][9]
\end{Arabic}}
\flushleft{\begin{malayalam}
ആകാശം അതിഭീകരമാംവിധം പ്രകമ്പനം കൊള്ളുന്ന ദിനമാണതുണ്ടാവുക.
\end{malayalam}}
\flushright{\begin{Arabic}
\quranayah[52][10]
\end{Arabic}}
\flushleft{\begin{malayalam}
-അന്ന് മലകള്‍ ഇളകി നീങ്ങും.
\end{malayalam}}
\flushright{\begin{Arabic}
\quranayah[52][11]
\end{Arabic}}
\flushleft{\begin{malayalam}
സത്യനിഷേധികള്‍ക്ക് അന്ന് കൊടും നാശം!
\end{malayalam}}
\flushright{\begin{Arabic}
\quranayah[52][12]
\end{Arabic}}
\flushleft{\begin{malayalam}
അനാവശ്യകാര്യങ്ങളില്‍ കളിച്ചുരസിക്കുന്നവരാണവര്‍.
\end{malayalam}}
\flushright{\begin{Arabic}
\quranayah[52][13]
\end{Arabic}}
\flushleft{\begin{malayalam}
അവരെ നരകത്തിലേക്ക് പിടിച്ചു തള്ളുന്ന ദിനം.
\end{malayalam}}
\flushright{\begin{Arabic}
\quranayah[52][14]
\end{Arabic}}
\flushleft{\begin{malayalam}
അന്ന് അവരോട് പറയും: "നിങ്ങള്‍ തള്ളിപ്പറഞ്ഞുകൊണ്ടിരുന്ന നരകമാണിത്.
\end{malayalam}}
\flushright{\begin{Arabic}
\quranayah[52][15]
\end{Arabic}}
\flushleft{\begin{malayalam}
"അല്ല; ഇത് മായാജാലമാണോ? അതല്ല, നിങ്ങള്‍ കാണുന്നില്ലെന്നുണ്ടോ?
\end{malayalam}}
\flushright{\begin{Arabic}
\quranayah[52][16]
\end{Arabic}}
\flushleft{\begin{malayalam}
"ഇനി നിങ്ങളതില്‍ കിടന്നു വെന്തെരിയുക. നിങ്ങളിത് സഹിക്കുകയോ സഹിക്കാതിരിക്കുകയോ ചെയ്യുക. രണ്ടും നിങ്ങള്‍ക്കു സമം തന്നെ. നിങ്ങള്‍ പ്രവര്‍ത്തിച്ചുകൊണ്ടിരുന്നതിന് അനുയോജ്യമായ പ്രതിഫലം തന്നെയാണ് നിങ്ങള്‍ക്കു നല്‍കുന്നത്.”
\end{malayalam}}
\flushright{\begin{Arabic}
\quranayah[52][17]
\end{Arabic}}
\flushleft{\begin{malayalam}
എന്നാല്‍ ദൈവഭക്തര്‍ സ്വര്‍ഗീയാരാമങ്ങളിലും സുഖസൌഭാഗ്യങ്ങളിലുമായിരിക്കും;
\end{malayalam}}
\flushright{\begin{Arabic}
\quranayah[52][18]
\end{Arabic}}
\flushleft{\begin{malayalam}
തങ്ങളുടെ നാഥന്‍ അവര്‍ക്കേകിയതില്‍ ആനന്ദം അനുഭവിക്കുന്നവരായി. കത്തിക്കാളുന്ന നരകത്തീയില്‍നിന്ന് അവരുടെ നാഥന്‍ അവരെ കാത്തുരക്ഷിക്കും.
\end{malayalam}}
\flushright{\begin{Arabic}
\quranayah[52][19]
\end{Arabic}}
\flushleft{\begin{malayalam}
അന്ന് അവരോട് പറയും: നിങ്ങള്‍ പ്രവര്‍ത്തിച്ചുകൊണ്ടിരുന്നതിന്റെ പ്രതിഫലമായി നിങ്ങള്‍ ആനന്ദത്തോടെ തിന്നുകയും കുടിക്കുകയും ചെയ്യുക.
\end{malayalam}}
\flushright{\begin{Arabic}
\quranayah[52][20]
\end{Arabic}}
\flushleft{\begin{malayalam}
വരിവരിയായി നിരത്തിയിട്ട കട്ടിലുകളില്‍ ചാരിയിരിക്കുന്നവരായിരിക്കും അവര്‍. വിശാലാക്ഷികളായ തരുണികളെ നാം അവര്‍ക്ക് ഇണകളായിക്കൊടുക്കും.
\end{malayalam}}
\flushright{\begin{Arabic}
\quranayah[52][21]
\end{Arabic}}
\flushleft{\begin{malayalam}
സത്യവിശ്വാസം സ്വീകരിച്ചവരെയും സത്യവിശ്വാസ സ്വീകരണത്തില്‍ അവരെ അനുഗമിച്ച അവരുടെ സന്താനങ്ങളെയും നാം ഒരുമിച്ചു ചേര്‍ക്കും. അവരുടെ കര്‍മഫലങ്ങളില്‍ നാമൊരു കുറവും വരുത്തുകയില്ല. ഓരോ മനുഷ്യനും താന്‍ സമ്പാദിച്ചതിന് അര്‍ഹനായിരിക്കും.
\end{malayalam}}
\flushright{\begin{Arabic}
\quranayah[52][22]
\end{Arabic}}
\flushleft{\begin{malayalam}
അവരാഗ്രഹിക്കുന്ന ഏതിനം പഴവും മാംസവും നാമവര്‍ക്ക് നിര്‍ലോഭം നല്കും.
\end{malayalam}}
\flushright{\begin{Arabic}
\quranayah[52][23]
\end{Arabic}}
\flushleft{\begin{malayalam}
അവര്‍ പാനപാത്രം പരസ്പരം കൈമാറിക്കൊണ്ടിരിക്കും. അസഭ്യവാക്കോ ദുര്‍വൃത്തിയോ അവിടെ ഉണ്ടാവുകയില്ല.
\end{malayalam}}
\flushright{\begin{Arabic}
\quranayah[52][24]
\end{Arabic}}
\flushleft{\begin{malayalam}
അവരുടെ പരിചരണത്തിനായി അവരുടെ അടുത്ത് ബാലന്മാര്‍ ചുറ്റിക്കറങ്ങിക്കൊണ്ടിരിക്കും. കാത്തുസൂക്ഷിക്കും മുത്തുകള്‍പോലിരിക്കും അവര്‍.
\end{malayalam}}
\flushright{\begin{Arabic}
\quranayah[52][25]
\end{Arabic}}
\flushleft{\begin{malayalam}
പരസ്പരം പലതും ചോദിച്ചുകൊണ്ട് അവരന്യോന്യം അഭിമുഖീകരിക്കും.
\end{malayalam}}
\flushright{\begin{Arabic}
\quranayah[52][26]
\end{Arabic}}
\flushleft{\begin{malayalam}
അവര്‍ പറയും: "നിശ്ചയമായും നാം ഇതിന് മുമ്പ് നമ്മുടെ കുടുംബത്തിലായിരുന്നപ്പോള്‍ ആശങ്കാകുലരായിരുന്നു.
\end{malayalam}}
\flushright{\begin{Arabic}
\quranayah[52][27]
\end{Arabic}}
\flushleft{\begin{malayalam}
"അതിനാല്‍ അല്ലാഹു നമ്മെ അനുഗ്രഹിച്ചു. ചുട്ടുപൊള്ളുന്ന നരക ശിക്ഷയില്‍നിന്ന് അവന്‍ നമ്മെ രക്ഷിച്ചു.
\end{malayalam}}
\flushright{\begin{Arabic}
\quranayah[52][28]
\end{Arabic}}
\flushleft{\begin{malayalam}
"നിശ്ചയമായും നാം മുമ്പേ അവനോട് മാത്രമാണ് പ്രാര്‍ഥിക്കാറുണ്ടായിരുന്നത്. അവന്‍ തന്നെയാണ് അത്യുദാരനും ദയാപരനും; തീര്‍ച്ച.”
\end{malayalam}}
\flushright{\begin{Arabic}
\quranayah[52][29]
\end{Arabic}}
\flushleft{\begin{malayalam}
അതിനാല്‍ നീ ഉദ്ബോധനം തുടര്‍ന്നുകൊണ്ടിരിക്കുക. നിന്റെ നാഥന്റെ അനുഗ്രഹത്താല്‍ നീ ജ്യോത്സ്യനോ ഭ്രാന്തനോ അല്ല.
\end{malayalam}}
\flushright{\begin{Arabic}
\quranayah[52][30]
\end{Arabic}}
\flushleft{\begin{malayalam}
“ഇയാള്‍ ഒരു കവിയാണ്. ഇയാളുടെ കാര്യത്തില്‍ കാലവിപത്ത് വരുന്നത് നമുക്കു കാത്തിരുന്നു കാണാം” എന്നാണോ അവര്‍ പറയുന്നത്?
\end{malayalam}}
\flushright{\begin{Arabic}
\quranayah[52][31]
\end{Arabic}}
\flushleft{\begin{malayalam}
എങ്കില്‍ നീ പറയുക: ശരി, നിങ്ങള്‍ കാത്തിരിക്കുക; നിങ്ങളോടൊപ്പം കാത്തിരിക്കുന്നവരില്‍ ഞാനുമുണ്ട്.
\end{malayalam}}
\flushright{\begin{Arabic}
\quranayah[52][32]
\end{Arabic}}
\flushleft{\begin{malayalam}
ഇവരുടെ ബുദ്ധി ഇവരോട് ഇവ്വിധം പറയാന്‍ ആജ്ഞാപിക്കുകയാണോ? അതോ; ഇവര്‍ അതിക്രമികളായ ജനത തന്നെയോ?
\end{malayalam}}
\flushright{\begin{Arabic}
\quranayah[52][33]
\end{Arabic}}
\flushleft{\begin{malayalam}
അല്ല; ഈ ഖുര്‍ആന്‍ അദ്ദേഹം സ്വയം കെട്ടിച്ചമച്ചുണ്ടാക്കിയതാണെന്നാണോ ഇവരാരോപിക്കുന്നത്? എന്നാല്‍ ഇവര്‍ വിശ്വസിക്കുന്നില്ലെന്നതാണ് സത്യം.
\end{malayalam}}
\flushright{\begin{Arabic}
\quranayah[52][34]
\end{Arabic}}
\flushleft{\begin{malayalam}
ഇവര്‍ സത്യവാന്മാരെങ്കില്‍ ഇവ്വിധമൊരു വചനം കൊണ്ടുവരട്ടെ.
\end{malayalam}}
\flushright{\begin{Arabic}
\quranayah[52][35]
\end{Arabic}}
\flushleft{\begin{malayalam}
അതല്ല; സ്രഷ്ടാവില്ലാതെ സ്വയം ഉണ്ടായവരാണോ ഇവര്‍? അതോ ഇവര്‍ തന്നെയാണോ ഇവരുടെ സ്രഷ്ടാക്കള്‍!
\end{malayalam}}
\flushright{\begin{Arabic}
\quranayah[52][36]
\end{Arabic}}
\flushleft{\begin{malayalam}
അല്ലെങ്കില്‍ ഇവരാണോ ആകാശ ഭൂമികളെ സൃഷ്ടിച്ചത്? എന്നാല്‍ ഇവര്‍ ദൃഢമായി വിശ്വസിക്കുന്നില്ലെന്നതാണ് സത്യം.
\end{malayalam}}
\flushright{\begin{Arabic}
\quranayah[52][37]
\end{Arabic}}
\flushleft{\begin{malayalam}
അതല്ല; നിന്റെ നാഥന്റെ ഖജനാവുകള്‍ ഇവരുടെ വശമാണോ? അല്ലെങ്കില്‍ ഇവരാണോ അതൊക്കെയും നിയന്ത്രിച്ചു നടത്തുന്നത്?
\end{malayalam}}
\flushright{\begin{Arabic}
\quranayah[52][38]
\end{Arabic}}
\flushleft{\begin{malayalam}
അതല്ല; വിവരങ്ങള്‍ കേട്ടറിയാനായി ഉപരിലോകത്തേക്ക് കയറാനിവര്‍ക്ക് വല്ല കോണിയുമുണ്ടോ? എങ്കില്‍ അവ്വിധം കേട്ടു മനസ്സിലാക്കുന്നവര്‍ അതിന് വ്യക്തമായ വല്ല തെളിവും കൊണ്ടുവരട്ടെ.
\end{malayalam}}
\flushright{\begin{Arabic}
\quranayah[52][39]
\end{Arabic}}
\flushleft{\begin{malayalam}
അല്ല; അല്ലാഹുവിന് പുത്രിമാരും നിങ്ങള്‍ക്ക് പുത്രന്മാരുമാണെന്നോ?
\end{malayalam}}
\flushright{\begin{Arabic}
\quranayah[52][40]
\end{Arabic}}
\flushleft{\begin{malayalam}
അതല്ല; നീ ഇവരോട് എന്തെങ്കിലും പ്രതിഫലം ആവശ്യപ്പെടുന്നുണ്ടോ? അങ്ങനെ അതിന്റെ കടഭാരത്താല്‍ പ്രയാസപ്പെടുകയാണോ ഇവര്‍?
\end{malayalam}}
\flushright{\begin{Arabic}
\quranayah[52][41]
\end{Arabic}}
\flushleft{\begin{malayalam}
അതല്ല; ഇവര്‍ക്ക് അഭൌതികജ്ഞാനം ലഭിക്കുകയും അങ്ങനെ ഇവരതെഴുതി വെക്കുകയും ചെയ്തിട്ടുണ്ടോ?
\end{malayalam}}
\flushright{\begin{Arabic}
\quranayah[52][42]
\end{Arabic}}
\flushleft{\begin{malayalam}
അതല്ല; ഇവര്‍ വല്ല കുതന്ത്രവും കാണിക്കാന്‍ ഉദ്ദേശിക്കുന്നുവോ? എങ്കില്‍ സത്യനിഷേധികളാരോ, അവര്‍ തന്നെയായിരിക്കും കുതന്ത്രത്തിന്നിരയാകുന്നവര്‍.
\end{malayalam}}
\flushright{\begin{Arabic}
\quranayah[52][43]
\end{Arabic}}
\flushleft{\begin{malayalam}
അതല്ല; ഇവര്‍ക്ക് അല്ലാഹുവല്ലാതെ മറ്റു വല്ല ദൈവവുമുണ്ടോ? ഇവര്‍ പങ്കുചേര്‍ക്കുന്നതില്‍ നിന്നെല്ലാം അല്ലാഹു എത്രയോ പരിശുദ്ധനാണ്.
\end{malayalam}}
\flushright{\begin{Arabic}
\quranayah[52][44]
\end{Arabic}}
\flushleft{\begin{malayalam}
ആകാശത്തിന്റെ ഒരടല് തന്നെ അടര്‍ന്ന് വീഴുന്നത് കണ്ടാലും അത് മേഘമലയാണെന്നായിരിക്കും ഇവര്‍ പറയുക.
\end{malayalam}}
\flushright{\begin{Arabic}
\quranayah[52][45]
\end{Arabic}}
\flushleft{\begin{malayalam}
അതിനാല്‍ ഇവരെ വിട്ടേക്കുക. ബോധരഹിതരായി വീഴുന്ന ദുര്‍ദിനത്തെയിവര്‍ കണ്ടുമുട്ടും വരെ.
\end{malayalam}}
\flushright{\begin{Arabic}
\quranayah[52][46]
\end{Arabic}}
\flushleft{\begin{malayalam}
ഇവരുടെ കുതന്ത്രങ്ങളൊന്നും ഇവര്‍ക്കൊട്ടും ഉപകരിക്കാത്ത ദിനം. ഇവര്‍ക്ക് അന്ന് ഒരു സഹായവും ലഭിക്കുകയില്ല.
\end{malayalam}}
\flushright{\begin{Arabic}
\quranayah[52][47]
\end{Arabic}}
\flushleft{\begin{malayalam}
തീര്‍ച്ചയായും അക്രമം പ്രവര്‍ത്തിച്ചവര്‍ക്ക് അതല്ലാത്ത ശിക്ഷയുമുണ്ട്; ഉറപ്പ്. എങ്കിലും ഇവരിലേറെ പേരും അതറിയുന്നില്ല.
\end{malayalam}}
\flushright{\begin{Arabic}
\quranayah[52][48]
\end{Arabic}}
\flushleft{\begin{malayalam}
അതിനാല്‍ നിന്റെ നാഥന്റെ തീരുമാനത്തെ ക്ഷമയോടെ കാത്തിരിക്കുക. നീ നമ്മുടെ കണ്‍പാടില്‍ തന്നെയാണ്. നീ ഉണര്‍ന്നെഴുന്നേല്‍ക്കുമ്പോള്‍ നിന്റെ നാഥനെ കീര്‍ത്തിക്കുന്നതോടൊപ്പം അവന്റെ വിശുദ്ധിയെ വാഴ്ത്തുകയും ചെയ്യുക.
\end{malayalam}}
\flushright{\begin{Arabic}
\quranayah[52][49]
\end{Arabic}}
\flushleft{\begin{malayalam}
ഇരവിലും അവന്റെ വിശുദ്ധിയെ വാഴ്ത്തുക; താരകങ്ങള്‍ പിന്‍വാങ്ങുമ്പോഴും.
\end{malayalam}}
\chapter{\textmalayalam{നജ്മ് ( നക്ഷത്രം )}}
\begin{Arabic}
\Huge{\centerline{\basmalah}}\end{Arabic}
\flushright{\begin{Arabic}
\quranayah[53][1]
\end{Arabic}}
\flushleft{\begin{malayalam}
നക്ഷത്രം സാക്ഷി. അത് അസ്തമിക്കുമ്പോള്‍.
\end{malayalam}}
\flushright{\begin{Arabic}
\quranayah[53][2]
\end{Arabic}}
\flushleft{\begin{malayalam}
നിങ്ങളുടെ കൂട്ടുകാരനായ പ്രവാചകന് വഴിതെറ്റിയിട്ടില്ല. ദുര്‍മാര്‍ഗിയായിട്ടുമില്ല.
\end{malayalam}}
\flushright{\begin{Arabic}
\quranayah[53][3]
\end{Arabic}}
\flushleft{\begin{malayalam}
അദ്ദേഹം തോന്നിയപോലെ സംസാരിക്കുന്നുമില്ല.
\end{malayalam}}
\flushright{\begin{Arabic}
\quranayah[53][4]
\end{Arabic}}
\flushleft{\begin{malayalam}
ഈ സന്ദേശം അദ്ദേഹത്തിനു നല്‍കപ്പെട്ട ദിവ്യ ബോധനം മാത്രമാണ്.
\end{malayalam}}
\flushright{\begin{Arabic}
\quranayah[53][5]
\end{Arabic}}
\flushleft{\begin{malayalam}
അദ്ദേഹത്തെ അത് അഭ്യസിപ്പിച്ചത് ഏറെ കരുത്തനാണ്.
\end{malayalam}}
\flushright{\begin{Arabic}
\quranayah[53][6]
\end{Arabic}}
\flushleft{\begin{malayalam}
പ്രബലനായ ഒരു വ്യക്തി. അങ്ങനെ അവന്‍ നിവര്‍ന്നുനിന്നു.
\end{malayalam}}
\flushright{\begin{Arabic}
\quranayah[53][7]
\end{Arabic}}
\flushleft{\begin{malayalam}
അത്യുന്നതമായ ചക്രവാളത്തിലായിക്കൊണ്ട്.
\end{malayalam}}
\flushright{\begin{Arabic}
\quranayah[53][8]
\end{Arabic}}
\flushleft{\begin{malayalam}
പിന്നെ അവന്‍ അടുത്തുവന്നു. വീണ്ടും അടുത്തു.
\end{malayalam}}
\flushright{\begin{Arabic}
\quranayah[53][9]
\end{Arabic}}
\flushleft{\begin{malayalam}
അങ്ങനെ രണ്ടു വില്ലോളമോ അതില്‍ കൂടുതലോ അടുത്ത് നിലകൊണ്ടു.
\end{malayalam}}
\flushright{\begin{Arabic}
\quranayah[53][10]
\end{Arabic}}
\flushleft{\begin{malayalam}
അപ്പോള്‍, അല്ലാഹു തന്റെ ദാസന് നല്‍കേണ്ട സന്ദേശം അവന്‍ ബോധനമായി നല്‍കി.
\end{malayalam}}
\flushright{\begin{Arabic}
\quranayah[53][11]
\end{Arabic}}
\flushleft{\begin{malayalam}
അദ്ദേഹം കണ്ണുകൊണ്ടു കണ്ടതിനെ മനസ്സ് കളവാക്കിയില്ല.
\end{malayalam}}
\flushright{\begin{Arabic}
\quranayah[53][12]
\end{Arabic}}
\flushleft{\begin{malayalam}
എന്നിട്ടും ആ പ്രവാചകന്‍ നേരില്‍ കണ്ടതിനെക്കുറിച്ച് നിങ്ങള്‍ അദ്ദേഹത്തോട് തര്‍ക്കിക്കുകയാണോ?
\end{malayalam}}
\flushright{\begin{Arabic}
\quranayah[53][13]
\end{Arabic}}
\flushleft{\begin{malayalam}
മറ്റൊരു ഇറങ്ങിവരവു വേളയിലും അദ്ദേഹം ജിബ്രീലിനെ കണ്ടിട്ടുണ്ട്.
\end{malayalam}}
\flushright{\begin{Arabic}
\quranayah[53][14]
\end{Arabic}}
\flushleft{\begin{malayalam}
സിദ്റതുല്‍ മുന്‍തഹായുടെ അടുത്ത് വെച്ച്.
\end{malayalam}}
\flushright{\begin{Arabic}
\quranayah[53][15]
\end{Arabic}}
\flushleft{\begin{malayalam}
അതിനടുത്താണ് അഭയസ്ഥാനമായ സ്വര്‍ഗം.
\end{malayalam}}
\flushright{\begin{Arabic}
\quranayah[53][16]
\end{Arabic}}
\flushleft{\begin{malayalam}
അന്നേരം സിദ്റയെ ആവരണം ചെയ്യുന്ന അതിഗംഭീരമായ പ്രഭാവം അതിനെ ആവരണം ചെയ്യുന്നുണ്ടായിരുന്നു.
\end{malayalam}}
\flushright{\begin{Arabic}
\quranayah[53][17]
\end{Arabic}}
\flushleft{\begin{malayalam}
അപ്പോള്‍ പ്രവാചകന്റെ ദൃഷ്ടി തെറ്റിപ്പോയില്ല. പരിധി ലംഘിച്ചുമില്ല.
\end{malayalam}}
\flushright{\begin{Arabic}
\quranayah[53][18]
\end{Arabic}}
\flushleft{\begin{malayalam}
ഉറപ്പായും അദ്ദേഹം തന്റെ നാഥന്റെ മഹത്തായ ചില ദൃഷ്ടാന്തങ്ങള്‍ കണ്ടിട്ടുണ്ട്.
\end{malayalam}}
\flushright{\begin{Arabic}
\quranayah[53][19]
\end{Arabic}}
\flushleft{\begin{malayalam}
“ലാതി”നെയും “ഉസ്സ”യെയും സംബന്ധിച്ച് നിങ്ങള്‍ ചിന്തിച്ചു നോക്കിയിട്ടുണ്ടോ?
\end{malayalam}}
\flushright{\begin{Arabic}
\quranayah[53][20]
\end{Arabic}}
\flushleft{\begin{malayalam}
കൂടാതെ മൂന്നാമതായുള്ള “മനാതി” നെക്കുറിച്ചും.
\end{malayalam}}
\flushright{\begin{Arabic}
\quranayah[53][21]
\end{Arabic}}
\flushleft{\begin{malayalam}
നിങ്ങള്‍ക്ക് ആണും അല്ലാഹുവിന് പെണ്ണും, അല്ലേ?
\end{malayalam}}
\flushright{\begin{Arabic}
\quranayah[53][22]
\end{Arabic}}
\flushleft{\begin{malayalam}
എങ്കില്‍ ഇത് തീര്‍ത്തും നീതി രഹിതമായ വിഭജനം തന്നെ.
\end{malayalam}}
\flushright{\begin{Arabic}
\quranayah[53][23]
\end{Arabic}}
\flushleft{\begin{malayalam}
യഥാര്‍ഥത്തില്‍ അവ, നിങ്ങളും നിങ്ങളുടെ പൂര്‍വ പിതാക്കളും വിളിച്ച ചില പേരുകളല്ലാതൊന്നുമല്ല. അല്ലാഹു ഇവയ്ക്കൊന്നും ഒരു തെളിവും നല്‍കിയിട്ടില്ല. ഊഹത്തെയും ദേഹേഛയെയും മാത്രമാണ് അവര്‍ പിന്‍പറ്റുന്നത്. നിശ്ചയം, അവര്‍ക്ക് തങ്ങളുടെ നാഥനില്‍ നിന്നുള്ള നേര്‍വഴി വന്നെത്തിയിട്ടുണ്ട്.
\end{malayalam}}
\flushright{\begin{Arabic}
\quranayah[53][24]
\end{Arabic}}
\flushleft{\begin{malayalam}
അതല്ല; മനുഷ്യന്‍ കൊതിച്ചതൊക്കെത്തന്നെയാണോ അവന്ന് കിട്ടുക?
\end{malayalam}}
\flushright{\begin{Arabic}
\quranayah[53][25]
\end{Arabic}}
\flushleft{\begin{malayalam}
എന്നാല്‍ അറിയുക: ഈ ലോകവും പരലോകവും അല്ലാഹുവിന്റേതാണ്.
\end{malayalam}}
\flushright{\begin{Arabic}
\quranayah[53][26]
\end{Arabic}}
\flushleft{\begin{malayalam}
മാനത്ത് എത്ര മലക്കുകളുണ്ട്! അവരുടെ ശുപാര്‍ശകളൊന്നും ഒട്ടും ഉപകരിക്കുകയില്ല. അല്ലാഹു ഇഛിക്കുകയും ഇഷ്ടപ്പെടുകയും ചെയ്യുന്നവര്‍ക്ക് അവന്‍ അനുമതി നല്‍കിയ ശേഷമല്ലാതെ.
\end{malayalam}}
\flushright{\begin{Arabic}
\quranayah[53][27]
\end{Arabic}}
\flushleft{\begin{malayalam}
പരലോക വിശ്വാസമില്ലാത്തവര്‍ മലക്കുകളെ സ്ത്രീനാമങ്ങളിലാണ് വിളിക്കുന്നത്.
\end{malayalam}}
\flushright{\begin{Arabic}
\quranayah[53][28]
\end{Arabic}}
\flushleft{\begin{malayalam}
അവര്‍ക്ക് അതേക്കുറിച്ച് ഒരറിവുമില്ല. അവര്‍ ഊഹത്തെ മാത്രം പിന്‍പറ്റുകയാണ്. ഊഹമോ, സത്യത്തിന് ഒരു പ്രയോജനവും ചെയ്യുകയില്ല.
\end{malayalam}}
\flushright{\begin{Arabic}
\quranayah[53][29]
\end{Arabic}}
\flushleft{\begin{malayalam}
അതിനാല്‍ നമ്മെ ഓര്‍ക്കുന്നതില്‍ നിന്ന് പിന്തിരിയുകയും ഐഹിക ജീവിതസുഖത്തിനപ്പുറമൊന്നും ലക്ഷ്യമാക്കാതിരിക്കുകയും ചെയ്യുന്നവരെ അവരുടെ പാട്ടിന് വിടുക.
\end{malayalam}}
\flushright{\begin{Arabic}
\quranayah[53][30]
\end{Arabic}}
\flushleft{\begin{malayalam}
അവര്‍ക്കു നേടാനായ അറിവ് അതുമാത്രമാണ്. തന്റെ മാര്‍ഗത്തില്‍നിന്ന് തെറ്റിയവര്‍ ആരെന്ന് ഏറ്റം നന്നായറിയുന്നവന്‍ നിന്റെ നാഥനാണ്. നേര്‍വഴി പ്രാപിച്ചവരെപ്പറ്റി നന്നായറിയുന്നവനും അവന്‍ തന്നെ.
\end{malayalam}}
\flushright{\begin{Arabic}
\quranayah[53][31]
\end{Arabic}}
\flushleft{\begin{malayalam}
ആകാശ ഭൂമികളിലുള്ളതൊക്കെയും അല്ലാഹുവിന്റേതാണ്. ദുര്‍വൃത്തര്‍ക്ക് അവരുടെ പ്രവര്‍ത്തനങ്ങള്‍ക്കൊത്ത പ്രതിഫലം നല്‍കാനാണത്. സദ്വൃത്തര്‍ക്ക് സദ്ഫലം സമ്മാനിക്കാനും.
\end{malayalam}}
\flushright{\begin{Arabic}
\quranayah[53][32]
\end{Arabic}}
\flushleft{\begin{malayalam}
അവരോ, വന്‍ പാപങ്ങളും നീചവൃത്തികളും വര്‍ജിക്കുന്നവരാണ്. കൊച്ചു വീഴ്ചകളൊഴികെ. നിശ്ചയമായും നിന്റെ നാഥന്‍ ഉദാരമായി പൊറുക്കുന്നവനാണ്. നിങ്ങളെ ഭൂമിയില്‍ നിന്ന് സൃഷ്ടിച്ചുണ്ടാക്കിയപ്പോഴും നിങ്ങള്‍ നിങ്ങളുടെ മാതാക്കളുടെ ഗര്‍ഭാശയത്തില്‍ ഭ്രൂണമായിരുന്നപ്പോഴും നിങ്ങളെപ്പറ്റി നന്നായറിയുന്നവന്‍ അവന്‍ തന്നെ. അതിനാല്‍ നിങ്ങള്‍ സ്വയം വിശുദ്ധി ചമയാതിരിക്കുക. യഥാര്‍ഥ ഭക്തനാരെന്ന് നന്നായറിയുന്നവന്‍ അവന്‍ മാത്രമാണ്.
\end{malayalam}}
\flushright{\begin{Arabic}
\quranayah[53][33]
\end{Arabic}}
\flushleft{\begin{malayalam}
എന്നാല്‍ സത്യത്തില്‍ നിന്ന് പിന്തിരിഞ്ഞവനെ നീ കണ്ടോ?
\end{malayalam}}
\flushright{\begin{Arabic}
\quranayah[53][34]
\end{Arabic}}
\flushleft{\begin{malayalam}
കുറച്ചു കൊടുത്തു നിര്‍ത്തിയവനെ?
\end{malayalam}}
\flushright{\begin{Arabic}
\quranayah[53][35]
\end{Arabic}}
\flushleft{\begin{malayalam}
അവന്റെ വശം വല്ല അഭൌതിക ജ്ഞാനവുമുണ്ടോ? അങ്ങനെ അവനത് കണ്ടുകൊണ്ടിരിക്കുകയാണോ?
\end{malayalam}}
\flushright{\begin{Arabic}
\quranayah[53][36]
\end{Arabic}}
\flushleft{\begin{malayalam}
അതല്ല; മൂസായുടെ ഏടുകളിലുള്ളവയെപ്പറ്റി അവന് അറിവ് ലഭിച്ചിട്ടില്ലേ?
\end{malayalam}}
\flushright{\begin{Arabic}
\quranayah[53][37]
\end{Arabic}}
\flushleft{\begin{malayalam}
ഉത്തരവാദിത്വങ്ങള്‍ പൂര്‍ത്തീകരിച്ച ഇബ്റാഹീമിന്റെയും?
\end{malayalam}}
\flushright{\begin{Arabic}
\quranayah[53][38]
\end{Arabic}}
\flushleft{\begin{malayalam}
അതെന്തെന്നാല്‍ പാപഭാരം ചുമക്കുന്ന ആരും അപരന്റെ പാപച്ചുമട് പേറുകയില്ല.
\end{malayalam}}
\flushright{\begin{Arabic}
\quranayah[53][39]
\end{Arabic}}
\flushleft{\begin{malayalam}
മനുഷ്യന് അവന്‍ പ്രവര്‍ത്തിച്ചതല്ലാതൊന്നുമില്ല.
\end{malayalam}}
\flushright{\begin{Arabic}
\quranayah[53][40]
\end{Arabic}}
\flushleft{\begin{malayalam}
തന്റെ കര്‍മഫലം താമസിയാതെ അവനെ കാണിക്കും.
\end{malayalam}}
\flushright{\begin{Arabic}
\quranayah[53][41]
\end{Arabic}}
\flushleft{\begin{malayalam}
പിന്നെ അവന്നതിന് തികവോടെ പ്രതിഫലം ലഭിക്കും.
\end{malayalam}}
\flushright{\begin{Arabic}
\quranayah[53][42]
\end{Arabic}}
\flushleft{\begin{malayalam}
ഒടുവില്‍ ഒക്കെയും നിന്റെ നാഥങ്കലാണ് ചെന്നെത്തുക.
\end{malayalam}}
\flushright{\begin{Arabic}
\quranayah[53][43]
\end{Arabic}}
\flushleft{\begin{malayalam}
ചിരിപ്പിക്കുന്നതും കരയിപ്പിക്കുന്നതും അവനാണ്.
\end{malayalam}}
\flushright{\begin{Arabic}
\quranayah[53][44]
\end{Arabic}}
\flushleft{\begin{malayalam}
മരിപ്പിക്കുന്നതും ജീവിപ്പിക്കുന്നതും അവന്‍ തന്നെ.
\end{malayalam}}
\flushright{\begin{Arabic}
\quranayah[53][45]
\end{Arabic}}
\flushleft{\begin{malayalam}
ഇണകളെ-ആണിനെയും പെണ്ണിനെയും-സൃഷ്ടിച്ചതും അവനാണ്.
\end{malayalam}}
\flushright{\begin{Arabic}
\quranayah[53][46]
\end{Arabic}}
\flushleft{\begin{malayalam}
ബീജത്തില്‍നിന്ന്; അത് സ്രവിക്കപ്പെട്ടാല്‍.
\end{malayalam}}
\flushright{\begin{Arabic}
\quranayah[53][47]
\end{Arabic}}
\flushleft{\begin{malayalam}
വീണ്ടും ജീവിപ്പിക്കുകയെന്നത് അവന്റെ ബാധ്യതയത്രെ.
\end{malayalam}}
\flushright{\begin{Arabic}
\quranayah[53][48]
\end{Arabic}}
\flushleft{\begin{malayalam}
ഐശ്വര്യമേകിയതും തൃപ്തനാക്കിയതും അവന്‍ തന്നെ.
\end{malayalam}}
\flushright{\begin{Arabic}
\quranayah[53][49]
\end{Arabic}}
\flushleft{\begin{malayalam}
പുണര്‍തം നക്ഷത്രത്തിന്റെ നാഥനും അവനാണ്.
\end{malayalam}}
\flushright{\begin{Arabic}
\quranayah[53][50]
\end{Arabic}}
\flushleft{\begin{malayalam}
പൌരാണിക ആദ് വര്‍ഗത്തെ നശിപ്പിച്ചതും അവന്‍ തന്നെ.
\end{malayalam}}
\flushright{\begin{Arabic}
\quranayah[53][51]
\end{Arabic}}
\flushleft{\begin{malayalam}
ഥമൂദിനെയും. അവരിലാരെയും ബാക്കിവെച്ചില്ല.
\end{malayalam}}
\flushright{\begin{Arabic}
\quranayah[53][52]
\end{Arabic}}
\flushleft{\begin{malayalam}
അതിനു മുമ്പെ നൂഹിന്റെ ജനതയെയും അവന്‍ നശിപ്പിച്ചു. കാരണം, അവര്‍ കടുത്ത അക്രമികളും ധിക്കാരികളുമായിരുന്നു.
\end{malayalam}}
\flushright{\begin{Arabic}
\quranayah[53][53]
\end{Arabic}}
\flushleft{\begin{malayalam}
കീഴ്മേല്‍ മറിഞ്ഞ നാടിനെയും അവന്‍ തകര്‍ത്തു തരിപ്പണമാക്കി.
\end{malayalam}}
\flushright{\begin{Arabic}
\quranayah[53][54]
\end{Arabic}}
\flushleft{\begin{malayalam}
അങ്ങനെ അവനതിനെ വന്‍ വിപത്തിനാല്‍ മൂടി.
\end{malayalam}}
\flushright{\begin{Arabic}
\quranayah[53][55]
\end{Arabic}}
\flushleft{\begin{malayalam}
എന്നിട്ടും നിന്റെ നാഥന്റെ അനുഗ്രഹങ്ങളില്‍ ഏതിനെയാണ് നീ സംശയിക്കുന്നത്?
\end{malayalam}}
\flushright{\begin{Arabic}
\quranayah[53][56]
\end{Arabic}}
\flushleft{\begin{malayalam}
ഈ പ്രവാചകന്‍ മുമ്പുള്ള താക്കീതുകാരുടെ കൂട്ടത്തില്‍പെട്ട മുന്നറിയിപ്പുകാരന്‍ തന്നെ.
\end{malayalam}}
\flushright{\begin{Arabic}
\quranayah[53][57]
\end{Arabic}}
\flushleft{\begin{malayalam}
വരാനിരിക്കുന്ന ആ സംഭവം അഥവാ ലോകാവസാനം ഇതാ അടുത്തെത്തിയിരിക്കുന്നു.
\end{malayalam}}
\flushright{\begin{Arabic}
\quranayah[53][58]
\end{Arabic}}
\flushleft{\begin{malayalam}
അതിനെ തട്ടിമാറ്റാന്‍ അല്ലാഹു അല്ലാതെ ആരുമില്ല.
\end{malayalam}}
\flushright{\begin{Arabic}
\quranayah[53][59]
\end{Arabic}}
\flushleft{\begin{malayalam}
എന്നിട്ടും ഈ വചനത്തെ സംബന്ധിച്ച് നിങ്ങള്‍ വിസ്മയം കൂറുകയാണോ?
\end{malayalam}}
\flushright{\begin{Arabic}
\quranayah[53][60]
\end{Arabic}}
\flushleft{\begin{malayalam}
നിങ്ങള്‍ ചിരിക്കുകയോ? കരയാതിരിക്കുകയും?
\end{malayalam}}
\flushright{\begin{Arabic}
\quranayah[53][61]
\end{Arabic}}
\flushleft{\begin{malayalam}
നിങ്ങള്‍ തികഞ്ഞ അശ്രദ്ധയില്‍ തന്നെ കഴിയുകയാണോ?
\end{malayalam}}
\flushright{\begin{Arabic}
\quranayah[53][62]
\end{Arabic}}
\flushleft{\begin{malayalam}
അതിനാല്‍ അല്ലാഹുവിന് സാഷ്ടാംഗം പ്രണമിക്കുക. അവന് മാത്രം വഴിപ്പെടുകയും ചെയ്യുക.
\end{malayalam}}
\chapter{\textmalayalam{ഖമര്‍ ( ചന്ദ്രന്‍ )}}
\begin{Arabic}
\Huge{\centerline{\basmalah}}\end{Arabic}
\flushright{\begin{Arabic}
\quranayah[54][1]
\end{Arabic}}
\flushleft{\begin{malayalam}
അന്ത്യനാള്‍ ആസന്നമായി. ചന്ദ്രന്‍ പിളര്‍ന്നു.
\end{malayalam}}
\flushright{\begin{Arabic}
\quranayah[54][2]
\end{Arabic}}
\flushleft{\begin{malayalam}
എന്നാല്‍ ഏതു ദൃഷ്ടാന്തം കണ്ടാലും അവരതിനെ അവഗണിക്കുന്നു. തുടര്‍ന്നു പോരുന്ന മായാജാലമെന്ന് പറയുകയും ചെയ്യുന്നു.
\end{malayalam}}
\flushright{\begin{Arabic}
\quranayah[54][3]
\end{Arabic}}
\flushleft{\begin{malayalam}
അവരതിനെ തള്ളിപ്പറഞ്ഞു. സ്വേഛകളെ പിന്‍പറ്റി. എന്നാല്‍ എല്ലാ കാര്യങ്ങളും ഒരു പര്യവസാനത്തിലെത്തുക തന്നെ ചെയ്യും.
\end{malayalam}}
\flushright{\begin{Arabic}
\quranayah[54][4]
\end{Arabic}}
\flushleft{\begin{malayalam}
തീര്‍ച്ചയായും അവര്‍ക്കു നേരത്തെ ചില വിവരങ്ങള്‍ വന്നെത്തിയിട്ടുണ്ട്. ദുര്‍മാര്‍ഗത്തില്‍ നിന്ന് തടഞ്ഞുനിര്‍ത്തുന്ന താക്കീതുകള്‍ അതിലുണ്ട്.
\end{malayalam}}
\flushright{\begin{Arabic}
\quranayah[54][5]
\end{Arabic}}
\flushleft{\begin{malayalam}
തികവാര്‍ന്ന തത്വങ്ങളും. എന്നിട്ടും താക്കീതുകള്‍ അവര്‍ക്കുപകരിക്കുന്നില്ല.
\end{malayalam}}
\flushright{\begin{Arabic}
\quranayah[54][6]
\end{Arabic}}
\flushleft{\begin{malayalam}
അതിനാല്‍ അവരെ വിട്ടകലുക. അതിഭീകരമായ ഒരു കാര്യത്തിലേക്ക് അവരെ വിളിക്കുന്ന ദിനം.
\end{malayalam}}
\flushright{\begin{Arabic}
\quranayah[54][7]
\end{Arabic}}
\flushleft{\begin{malayalam}
പേടിച്ചരണ്ട കണ്ണുകളോടെ അവര്‍ തങ്ങളുടെ ഖബറുകളില്‍നിന്ന് പുറത്തുവരും. പരന്നു പറക്കുന്ന വെട്ടുകിളികളെപ്പോലെ.
\end{malayalam}}
\flushright{\begin{Arabic}
\quranayah[54][8]
\end{Arabic}}
\flushleft{\begin{malayalam}
വിളിയാളന്റെ അടുത്തേക്ക് അവര്‍ പാഞ്ഞെത്തും. അന്ന് സത്യനിഷേധികള്‍ വിലപിക്കും: “ഇതൊരു ദുര്‍ദിനം തന്നെ.”
\end{malayalam}}
\flushright{\begin{Arabic}
\quranayah[54][9]
\end{Arabic}}
\flushleft{\begin{malayalam}
ഇവര്‍ക്കുമുമ്പ് നൂഹിന്റെ ജനതയും ഇവ്വിധം സത്യത്തെ നിഷേധിച്ചിട്ടുണ്ട്. അങ്ങനെ അവര്‍ നമ്മുടെ ദാസനെ തള്ളിപ്പറഞ്ഞു. ഭ്രാന്തനെന്ന് വിളിച്ചു. വിരട്ടിയോടിക്കുകയും ചെയ്തു.
\end{malayalam}}
\flushright{\begin{Arabic}
\quranayah[54][10]
\end{Arabic}}
\flushleft{\begin{malayalam}
അപ്പോഴദ്ദേഹം തന്റെ നാഥനെ വിളിച്ചു പ്രാര്‍ഥിച്ചു: "ഞാന്‍ തോറ്റിരിക്കുന്നു. അതിനാല്‍ നീയെന്നെ സഹായിക്കേണമേ.”
\end{malayalam}}
\flushright{\begin{Arabic}
\quranayah[54][11]
\end{Arabic}}
\flushleft{\begin{malayalam}
അങ്ങനെ കോരിച്ചൊരിയുന്ന പേമാരിയാല്‍ നാം വാനകവാടങ്ങള്‍ തുറന്നു.
\end{malayalam}}
\flushright{\begin{Arabic}
\quranayah[54][12]
\end{Arabic}}
\flushleft{\begin{malayalam}
ഭൂമിയെ പിളര്‍ത്തി അരുവികള്‍ പൊട്ടിയൊഴുക്കി. അങ്ങനെ, നിശ്ചയിക്കപ്പെട്ട കാര്യം നടക്കാനായി ഈ വെള്ളമൊക്കെയും സംഗമിച്ചു.
\end{malayalam}}
\flushright{\begin{Arabic}
\quranayah[54][13]
\end{Arabic}}
\flushleft{\begin{malayalam}
നൂഹിനെ നാം പലകകളും കീലങ്ങളുമുള്ള കപ്പലില്‍ കയറ്റി.
\end{malayalam}}
\flushright{\begin{Arabic}
\quranayah[54][14]
\end{Arabic}}
\flushleft{\begin{malayalam}
അത് നമ്മുടെ മേല്‍നോട്ടത്തിലാണ് നീങ്ങിയിരുന്നത്. ജനം നിഷേധിച്ചു തള്ളിയവന്നുള്ള പ്രതിഫലമാണത്.
\end{malayalam}}
\flushright{\begin{Arabic}
\quranayah[54][15]
\end{Arabic}}
\flushleft{\begin{malayalam}
ഉറപ്പായും നാമതിനെ ഒരു തെളിവായി ബാക്കി വെച്ചിട്ടുണ്ട്. അതിനാല്‍ ചിന്തിച്ച് പാഠമുള്‍ക്കൊള്ളുന്ന ആരെങ്കിലുമുണ്ടോ?
\end{malayalam}}
\flushright{\begin{Arabic}
\quranayah[54][16]
\end{Arabic}}
\flushleft{\begin{malayalam}
അപ്പോള്‍ എന്റെ ശിക്ഷയും താക്കീതും എവ്വിധമായിരുന്നുവെന്ന് അറിയുക.
\end{malayalam}}
\flushright{\begin{Arabic}
\quranayah[54][17]
\end{Arabic}}
\flushleft{\begin{malayalam}
ഈ ഖുര്‍ആനിനെ നാം ചിന്തിച്ചറിയാനായി ലളിതമാക്കിയിരിക്കുന്നു. അതിനാല്‍ ആലോചിച്ചറിയുന്ന ആരെങ്കിലുമുണ്ടോ?
\end{malayalam}}
\flushright{\begin{Arabic}
\quranayah[54][18]
\end{Arabic}}
\flushleft{\begin{malayalam}
ആദ് സമുദായം സത്യത്തെ നിഷേധിച്ചു. അപ്പോള്‍ എന്റെ ശിക്ഷയും താക്കീതും എവ്വിധമായിരുന്നുവെന്നോ?
\end{malayalam}}
\flushright{\begin{Arabic}
\quranayah[54][19]
\end{Arabic}}
\flushleft{\begin{malayalam}
അവരുടെ നേരെ നാം ചീറ്റിയടിക്കുന്ന കാറ്റിനെ അയച്ചു; വിട്ടൊഴിയാത്ത ദുശ്ശകുനത്തിന്റെ നാളില്‍.
\end{malayalam}}
\flushright{\begin{Arabic}
\quranayah[54][20]
\end{Arabic}}
\flushleft{\begin{malayalam}
അത് ആ ജനത്തെ പിഴുതുമാറ്റിക്കൊണ്ടിരുന്നു. കടപുഴകിവീണ ഈത്തപ്പനത്തടിപോലെ.
\end{malayalam}}
\flushright{\begin{Arabic}
\quranayah[54][21]
\end{Arabic}}
\flushleft{\begin{malayalam}
അപ്പോള്‍: എന്റെ ശിക്ഷയും താക്കീതും എമ്മട്ടിലായിരുന്നുവെന്നറിയുക.
\end{malayalam}}
\flushright{\begin{Arabic}
\quranayah[54][22]
\end{Arabic}}
\flushleft{\begin{malayalam}
ചിന്തിച്ചു മനസ്സിലാക്കാനായി ഈ ഖുര്‍ആനിനെ നാം ലളിതമാക്കിയിരിക്കുന്നു. ആലോചിച്ചു മനസ്സിലാക്കുന്ന ആരെങ്കിലുമുണ്ടോ?
\end{malayalam}}
\flushright{\begin{Arabic}
\quranayah[54][23]
\end{Arabic}}
\flushleft{\begin{malayalam}
സമൂദ് സമുദായം മുന്നറിയിപ്പുകളെ കള്ളമാക്കി തള്ളി.
\end{malayalam}}
\flushright{\begin{Arabic}
\quranayah[54][24]
\end{Arabic}}
\flushleft{\begin{malayalam}
അങ്ങനെ അവര്‍ ചോദിച്ചു: "നമ്മുടെ കൂട്ടത്തിലെ ഒരു മനുഷ്യനെ നാം പിന്തുടരുകയോ? എങ്കില്‍ നാം വഴികേടിലും ബുദ്ധിശൂന്യതയിലും അകപ്പെട്ടതുതന്നെ.
\end{malayalam}}
\flushright{\begin{Arabic}
\quranayah[54][25]
\end{Arabic}}
\flushleft{\begin{malayalam}
"നമുക്കിടയില്‍നിന്ന് ഇവന് മാത്രം ഉദ്ബോധനം നല്‍കപ്പെട്ടുവെന്നോ? ഇല്ല; ഇവന്‍ അഹങ്കാരിയായ പെരുങ്കള്ളനാണ്.”
\end{malayalam}}
\flushright{\begin{Arabic}
\quranayah[54][26]
\end{Arabic}}
\flushleft{\begin{malayalam}
എന്നാല്‍ നാളെ അവരറിയുകതന്നെ ചെയ്യും. ആരാണ് അഹങ്കാരിയായ പെരുങ്കള്ളനെന്ന്.
\end{malayalam}}
\flushright{\begin{Arabic}
\quranayah[54][27]
\end{Arabic}}
\flushleft{\begin{malayalam}
അവര്‍ക്ക് ഒരു പരീക്ഷണമെന്ന നിലയില്‍ നാം ഒരൊട്ടകത്തെ അയക്കുകയാണ്. അതിനാല്‍ നീ അവരെ നിരീക്ഷിച്ചുകൊണ്ടിരിക്കുക. ക്ഷമയവലംബിക്കുക.
\end{malayalam}}
\flushright{\begin{Arabic}
\quranayah[54][28]
\end{Arabic}}
\flushleft{\begin{malayalam}
അവരെ അറിയിക്കുക: കുടിവെള്ളം അവര്‍ക്കും ഒട്ടകത്തിനുമിടയില്‍ പങ്കുവെക്കപ്പെട്ടിരിക്കുന്നു. ഓരോരുത്തരും തങ്ങളുടെ ഊഴമനുസരിച്ചേ വെള്ളത്തിന് വരാവൂ.
\end{malayalam}}
\flushright{\begin{Arabic}
\quranayah[54][29]
\end{Arabic}}
\flushleft{\begin{malayalam}
അവസാനം അവര്‍ തങ്ങളുടെ കൂട്ടുകാരനെ വിളിച്ചു- അവന്‍ അക്കാര്യം ഏറ്റെടുത്തു. അങ്ങനെ അവന്‍ ഒട്ടകത്തെ കശാപ്പു ചെയ്തു.
\end{malayalam}}
\flushright{\begin{Arabic}
\quranayah[54][30]
\end{Arabic}}
\flushleft{\begin{malayalam}
അപ്പോള്‍ നമ്മുടെ ശിക്ഷയും താക്കീതും എവ്വിധമായിരുന്നുവെന്നോ?
\end{malayalam}}
\flushright{\begin{Arabic}
\quranayah[54][31]
\end{Arabic}}
\flushleft{\begin{malayalam}
നാം അവരുടെമേല്‍ ഒരു ഘോരഗര്‍ജനമയച്ചു. അപ്പോഴവര്‍ കാലിത്തൊഴുത്തിലെ കച്ചിത്തുരുമ്പുകള്‍ പോലെയായി.
\end{malayalam}}
\flushright{\begin{Arabic}
\quranayah[54][32]
\end{Arabic}}
\flushleft{\begin{malayalam}
ചിന്തിച്ചറിയാനായി നാം ഈ ഖുര്‍ആനിനെ ലളിതമാക്കിയിരിക്കുന്നു. എന്നാല്‍ ചിന്തിച്ചു മനസ്സിലാക്കുന്നവരായി ആരെങ്കിലുമുണ്ടോ?
\end{malayalam}}
\flushright{\begin{Arabic}
\quranayah[54][33]
\end{Arabic}}
\flushleft{\begin{malayalam}
ലൂത്വിന്റെ ജനത താക്കീതുകള്‍ തള്ളിക്കളഞ്ഞു.
\end{malayalam}}
\flushright{\begin{Arabic}
\quranayah[54][34]
\end{Arabic}}
\flushleft{\begin{malayalam}
നാം അവരുടെ നേരെ ചരല്‍ക്കാറ്റയച്ചു. ലൂത്വിന്റെ കുടുംബമേ അതില്‍ നിന്നൊഴിവായുള്ളൂ. രാവിന്റെ ഒടുവുവേളയില്‍ നാമവരെ രക്ഷപ്പെടുത്തി.
\end{malayalam}}
\flushright{\begin{Arabic}
\quranayah[54][35]
\end{Arabic}}
\flushleft{\begin{malayalam}
നമ്മില്‍ നിന്നുള്ള അനുഗ്രഹമായിരുന്നു അത്. അവ്വിധമാണ് നന്ദി കാണിക്കുന്നവര്‍ക്ക് നാം പ്രതിഫലമേകുന്നത്.
\end{malayalam}}
\flushright{\begin{Arabic}
\quranayah[54][36]
\end{Arabic}}
\flushleft{\begin{malayalam}
നമ്മുടെ ശിക്ഷയെ സംബന്ധിച്ച് ലൂത്വ് അവര്‍ക്ക് മുന്നറിയിപ്പ് നല്‍കിയിട്ടുണ്ടായിരുന്നു. എന്നാല്‍ അവര്‍ താക്കീതുകളെ സംശയിച്ച് തള്ളുകയായിരുന്നു.
\end{malayalam}}
\flushright{\begin{Arabic}
\quranayah[54][37]
\end{Arabic}}
\flushleft{\begin{malayalam}
അവര്‍ അദ്ദേഹത്തോട് തന്റെ അതിഥികളെ അവരുടെ ഇഛാപൂരണത്തിന് വിട്ടുകൊടുക്കാനാവശ്യപ്പെട്ടു. അപ്പോള്‍ നാം അവരുടെ കണ്ണുകളെ തുടച്ചുമായിച്ചു. എന്റെ ശിക്ഷയും താക്കീതും ആസ്വദിച്ചുകൊള്ളുക.
\end{malayalam}}
\flushright{\begin{Arabic}
\quranayah[54][38]
\end{Arabic}}
\flushleft{\begin{malayalam}
അതിരാവിലെത്തന്നെ സ്ഥായിയായ ശിക്ഷ അവരെ പിടികൂടിക്കഴിഞ്ഞിരുന്നു.
\end{malayalam}}
\flushright{\begin{Arabic}
\quranayah[54][39]
\end{Arabic}}
\flushleft{\begin{malayalam}
എന്റെ ശിക്ഷയും താക്കീതുകളും നിങ്ങളനുഭവിച്ചാസ്വദിച്ചുകൊള്ളുക.
\end{malayalam}}
\flushright{\begin{Arabic}
\quranayah[54][40]
\end{Arabic}}
\flushleft{\begin{malayalam}
ചിന്തിച്ചു മനസ്സിലാക്കാനായി നാം ഈ ഖുര്‍ആനിനെ ലളിതമാക്കിയിരിക്കുന്നു. ആലോചിച്ചറിയുന്നവരായി ആരെങ്കിലുമുണ്ടോ?
\end{malayalam}}
\flushright{\begin{Arabic}
\quranayah[54][41]
\end{Arabic}}
\flushleft{\begin{malayalam}
ഫറവോന്റെ ആള്‍ക്കാര്‍ക്കും താക്കീതുകള്‍ വന്നെത്തിയിരുന്നു.
\end{malayalam}}
\flushright{\begin{Arabic}
\quranayah[54][42]
\end{Arabic}}
\flushleft{\begin{malayalam}
അവര്‍ നമ്മുടെ ദൃഷ്ടാന്തങ്ങളെയൊക്കെ കള്ളമാക്കി തള്ളി. അപ്പോള്‍ നാം അവരെ പിടികൂടി. പ്രതാപിയും പ്രബലനുമായ ഒരുത്തന്റെ പിടികൂടല്‍പോലെ.
\end{malayalam}}
\flushright{\begin{Arabic}
\quranayah[54][43]
\end{Arabic}}
\flushleft{\begin{malayalam}
നിങ്ങളുടെ ഈ നിഷേധികള്‍ അവരെക്കാള്‍ മെച്ചമാണോ? അതല്ലെങ്കില്‍ വേദത്താളുകളില്‍ നിങ്ങളുടെ പാപമുക്തിക്കു വല്ല ഉപായങ്ങളുമുണ്ടോ?
\end{malayalam}}
\flushright{\begin{Arabic}
\quranayah[54][44]
\end{Arabic}}
\flushleft{\begin{malayalam}
അതല്ല; തങ്ങള്‍ സംഘടിതരാണെന്നും സ്വയം രക്ഷപ്രാപിച്ചുകൊള്ളാമെന്നും അവരവകാശപ്പെടുന്നുവോ?
\end{malayalam}}
\flushright{\begin{Arabic}
\quranayah[54][45]
\end{Arabic}}
\flushleft{\begin{malayalam}
എങ്കില്‍ അടുത്തുതന്നെ ഈ സംഘം പരാജിതരാവും, പിന്തിരിഞ്ഞോടുകയും ചെയ്യും.
\end{malayalam}}
\flushright{\begin{Arabic}
\quranayah[54][46]
\end{Arabic}}
\flushleft{\begin{malayalam}
എന്നാല്‍ ആ അന്ത്യനാളാണ് അവരുടെ കണക്ക് തീര്‍പ്പിനുള്ള നിശ്ചിതസമയം. ആ അന്ത്യസമയം അത്യന്തം ഭീകരവും തിക്തവും തന്നെ.
\end{malayalam}}
\flushright{\begin{Arabic}
\quranayah[54][47]
\end{Arabic}}
\flushleft{\begin{malayalam}
തീര്‍ച്ചയായും ഈ കുറ്റവാളികള്‍ വ്യക്തമായ വഴികേടിലാകുന്നു. തികഞ്ഞ ബുദ്ധിശൂന്യതയിലും.
\end{malayalam}}
\flushright{\begin{Arabic}
\quranayah[54][48]
\end{Arabic}}
\flushleft{\begin{malayalam}
ഇവരെ മുഖം നിലത്തുകുത്തിയവരായി നരകത്തിലേക്ക് വലിച്ചിഴക്കുന്ന ദിനം; അന്ന് അവരോട് പറയും: നിങ്ങള്‍ നരകസ്പര്‍ശം ആസ്വദിച്ചുകൊള്ളുക.
\end{malayalam}}
\flushright{\begin{Arabic}
\quranayah[54][49]
\end{Arabic}}
\flushleft{\begin{malayalam}
എല്ലാ വസ്തുക്കളെയും നാം സൃഷ്ടിച്ചത് കൃത്യതയോടെയാണ്.
\end{malayalam}}
\flushright{\begin{Arabic}
\quranayah[54][50]
\end{Arabic}}
\flushleft{\begin{malayalam}
നമ്മുടെ കല്പന ഒരൊറ്റ ഉത്തരവത്രെ. ഇമവെട്ടുമ്പോഴേക്കും അതു നടപ്പാവുന്നു.
\end{malayalam}}
\flushright{\begin{Arabic}
\quranayah[54][51]
\end{Arabic}}
\flushleft{\begin{malayalam}
നിശ്ചയമായും നിങ്ങളെ പോലുള്ള പല കക്ഷികളെയും നാം നശിപ്പിച്ചിട്ടുണ്ട്. അതിനാല്‍ ആലോചിച്ചറിയുന്ന ആരെങ്കിലുമുണ്ടോ?
\end{malayalam}}
\flushright{\begin{Arabic}
\quranayah[54][52]
\end{Arabic}}
\flushleft{\begin{malayalam}
അവര്‍ ചെയ്തതൊക്കെയും രേഖകളിലുണ്ട്.
\end{malayalam}}
\flushright{\begin{Arabic}
\quranayah[54][53]
\end{Arabic}}
\flushleft{\begin{malayalam}
നിസ്സാരവും ഗുരുതരവുമായ ഏതു കാര്യവും രേഖപ്പെടുത്തിയിട്ടുണ്ട്.
\end{malayalam}}
\flushright{\begin{Arabic}
\quranayah[54][54]
\end{Arabic}}
\flushleft{\begin{malayalam}
സൂക്ഷ്മത പുലര്‍ത്തുന്നവര്‍ ഉറപ്പായും സ്വര്‍ഗീയാരാമങ്ങളിലും അരുവികളിലുമായിരിക്കും.
\end{malayalam}}
\flushright{\begin{Arabic}
\quranayah[54][55]
\end{Arabic}}
\flushleft{\begin{malayalam}
സത്യത്തിന്റെ ആസ്ഥാനത്ത്. ശക്തനായ രാജാധിരാജന്റെ സന്നിധിയില്‍.
\end{malayalam}}
\chapter{\textmalayalam{റഹ് മാന്‍‍ ( പരമകാരുണികന്‍ )}}
\begin{Arabic}
\Huge{\centerline{\basmalah}}\end{Arabic}
\flushright{\begin{Arabic}
\quranayah[55][1]
\end{Arabic}}
\flushleft{\begin{malayalam}
പരമകാരുണികന്‍.
\end{malayalam}}
\flushright{\begin{Arabic}
\quranayah[55][2]
\end{Arabic}}
\flushleft{\begin{malayalam}
അവന്‍ ഈ ഖുര്‍ആന്‍ പഠിപ്പിച്ചു.
\end{malayalam}}
\flushright{\begin{Arabic}
\quranayah[55][3]
\end{Arabic}}
\flushleft{\begin{malayalam}
അവന്‍ മനുഷ്യനെ സൃഷ്ടിച്ചു.
\end{malayalam}}
\flushright{\begin{Arabic}
\quranayah[55][4]
\end{Arabic}}
\flushleft{\begin{malayalam}
അവനെ സംസാരം അഭ്യസിപ്പിച്ചു.
\end{malayalam}}
\flushright{\begin{Arabic}
\quranayah[55][5]
\end{Arabic}}
\flushleft{\begin{malayalam}
സൂര്യനും ചന്ദ്രനും നിശ്ചിത ക്രമമനുസരിച്ചാണ് സഞ്ചരിക്കുന്നത്.
\end{malayalam}}
\flushright{\begin{Arabic}
\quranayah[55][6]
\end{Arabic}}
\flushleft{\begin{malayalam}
താരവും മരവും അവന് പ്രണാമമര്‍പ്പിക്കുന്നു.
\end{malayalam}}
\flushright{\begin{Arabic}
\quranayah[55][7]
\end{Arabic}}
\flushleft{\begin{malayalam}
അവന്‍ മാനത്തെ ഉയര്‍ത്തി നിര്‍ത്തി. തുലാസ് സ്ഥാപിച്ചു.
\end{malayalam}}
\flushright{\begin{Arabic}
\quranayah[55][8]
\end{Arabic}}
\flushleft{\begin{malayalam}
നിങ്ങള്‍ തുലാസില്‍ ക്രമക്കേട് വരുത്താതിരിക്കാന്‍.
\end{malayalam}}
\flushright{\begin{Arabic}
\quranayah[55][9]
\end{Arabic}}
\flushleft{\begin{malayalam}
അതിനാല്‍ നീതിപൂര്‍വം കൃത്യതയോടെ തുലാസ് ഉപയോഗിക്കുക. തൂക്കത്തില്‍ കുറവു വരുത്തരുത്.
\end{malayalam}}
\flushright{\begin{Arabic}
\quranayah[55][10]
\end{Arabic}}
\flushleft{\begin{malayalam}
ഭൂമിയെ അവന്‍ സൃഷ്ടികള്‍ക്കായി സംവിധാനിച്ചു.
\end{malayalam}}
\flushright{\begin{Arabic}
\quranayah[55][11]
\end{Arabic}}
\flushleft{\begin{malayalam}
അതില്‍ ധാരാളം പഴമുണ്ട്. കൊതുമ്പുള്ള ഈത്തപ്പനകളും.
\end{malayalam}}
\flushright{\begin{Arabic}
\quranayah[55][12]
\end{Arabic}}
\flushleft{\begin{malayalam}
വൈക്കോലോടുകൂടിയ ധാന്യങ്ങളും സുഗന്ധച്ചെടികളുമുണ്ട്.
\end{malayalam}}
\flushright{\begin{Arabic}
\quranayah[55][13]
\end{Arabic}}
\flushleft{\begin{malayalam}
അപ്പോള്‍ നിങ്ങളിരുകൂട്ടരുടെയും നാഥന്റെ ഏതനുഗ്രഹത്തെയാണ് നിങ്ങള്‍ തള്ളിപ്പറയുക.
\end{malayalam}}
\flushright{\begin{Arabic}
\quranayah[55][14]
\end{Arabic}}
\flushleft{\begin{malayalam}
മണ്‍കുടം പോലെ മുട്ടിയാല്‍ മുഴങ്ങുന്ന കളിമണ്ണില്‍നിന്ന് അവന്‍ മനുഷ്യനെ സൃഷ്ടിച്ചു.
\end{malayalam}}
\flushright{\begin{Arabic}
\quranayah[55][15]
\end{Arabic}}
\flushleft{\begin{malayalam}
പുകയില്ലാത്ത അഗ്നിജ്ജ്വാലയില്‍നിന്ന് ജിന്നിനെയും സൃഷ്ടിച്ചു.
\end{malayalam}}
\flushright{\begin{Arabic}
\quranayah[55][16]
\end{Arabic}}
\flushleft{\begin{malayalam}
അപ്പോള്‍ നിങ്ങളിരുകൂട്ടരുടെയും നാഥന്റെ ഏതനുഗ്രഹത്തെയാണ് നിങ്ങള്‍ തള്ളിപ്പറയുക?
\end{malayalam}}
\flushright{\begin{Arabic}
\quranayah[55][17]
\end{Arabic}}
\flushleft{\begin{malayalam}
രണ്ട് ഉദയസ്ഥാനങ്ങളുടെയും രണ്ട് അസ്തമയസ്ഥാനങ്ങളുടെയും നാഥന്‍ അവനത്രെ.
\end{malayalam}}
\flushright{\begin{Arabic}
\quranayah[55][18]
\end{Arabic}}
\flushleft{\begin{malayalam}
അപ്പോള്‍ നിങ്ങളിരുകൂട്ടരുടെയും നാഥന്റെ ഏത് അനുഗ്രഹത്തെയാണ് നിങ്ങള്‍ തള്ളിപ്പറയുക?
\end{malayalam}}
\flushright{\begin{Arabic}
\quranayah[55][19]
\end{Arabic}}
\flushleft{\begin{malayalam}
അവന്‍ രണ്ട് സമുദ്രങ്ങളെ പരസ്പരം സംഗമിക്കാന്‍ സാധിക്കുമാറ് അയച്ചുവിട്ടിരിക്കുന്നു.
\end{malayalam}}
\flushright{\begin{Arabic}
\quranayah[55][20]
\end{Arabic}}
\flushleft{\begin{malayalam}
അവ രണ്ടിനുമിടയില്‍ ഒരു നിരോധപടലമുണ്ട്. അവ പരസ്പരം അതിക്രമിച്ചുകടക്കുകയില്ല.
\end{malayalam}}
\flushright{\begin{Arabic}
\quranayah[55][21]
\end{Arabic}}
\flushleft{\begin{malayalam}
അപ്പോള്‍ നിങ്ങളിരുകൂട്ടരുടെയും നാഥന്റെ ഏതനുഗ്രഹത്തെയാണ് നിങ്ങള്‍ തള്ളിപ്പറയുക.
\end{malayalam}}
\flushright{\begin{Arabic}
\quranayah[55][22]
\end{Arabic}}
\flushleft{\begin{malayalam}
അവ രണ്ടില്‍നിന്നും മുത്തും പവിഴവും കിട്ടുന്നു.
\end{malayalam}}
\flushright{\begin{Arabic}
\quranayah[55][23]
\end{Arabic}}
\flushleft{\begin{malayalam}
അപ്പോള്‍ നിങ്ങളിരുകൂട്ടരുടെയും നാഥന്റെ ഏതനുഗ്രഹത്തെയാണ് നിങ്ങള്‍ തള്ളിപ്പറയുക?
\end{malayalam}}
\flushright{\begin{Arabic}
\quranayah[55][24]
\end{Arabic}}
\flushleft{\begin{malayalam}
സമുദ്രത്തില്‍ സഞ്ചരിക്കുന്ന, പര്‍വതങ്ങള്‍പോലെ ഉയരമുള്ള കപ്പലുകള്‍ അവന്റേതാണ്.
\end{malayalam}}
\flushright{\begin{Arabic}
\quranayah[55][25]
\end{Arabic}}
\flushleft{\begin{malayalam}
അപ്പോള്‍ നിങ്ങളിരുകൂട്ടരുടെയും നാഥന്റെ ഏതനുഗ്രഹത്തെയാണ് നിങ്ങള്‍ തള്ളിപ്പറയുക?
\end{malayalam}}
\flushright{\begin{Arabic}
\quranayah[55][26]
\end{Arabic}}
\flushleft{\begin{malayalam}
ഭൂതലത്തിലുള്ളതൊക്കെയും നശിക്കുന്നവയാണ്.
\end{malayalam}}
\flushright{\begin{Arabic}
\quranayah[55][27]
\end{Arabic}}
\flushleft{\begin{malayalam}
മഹാനും ഗംഭീരനുമായ നിന്റെ നാഥന്റെ അസ്തിത്വം മാത്രമാണ് അവശേഷിക്കുക.
\end{malayalam}}
\flushright{\begin{Arabic}
\quranayah[55][28]
\end{Arabic}}
\flushleft{\begin{malayalam}
അപ്പോള്‍ നിങ്ങളിരുകൂട്ടരുടെയും നാഥന്റെ ഏതനുഗ്രഹത്തെയാണ് നിങ്ങള്‍ തള്ളിപ്പറയുക?
\end{malayalam}}
\flushright{\begin{Arabic}
\quranayah[55][29]
\end{Arabic}}
\flushleft{\begin{malayalam}
ആകാശഭൂമികളിലുള്ളവയൊക്കെയും തങ്ങളുടെ ആവശ്യങ്ങള്‍ അവനോട് ചോദിച്ചുകൊണ്ടിരിക്കുന്നു. അതിനാല്‍ അവനെന്നും കാര്യനിര്‍വഹണത്തിലാണ്.
\end{malayalam}}
\flushright{\begin{Arabic}
\quranayah[55][30]
\end{Arabic}}
\flushleft{\begin{malayalam}
അപ്പോള്‍ നിങ്ങളിരുകൂട്ടരുടെയും നാഥന്റെ ഏതനുഗ്രഹത്തെയാണ് നിങ്ങള്‍ തള്ളിപ്പറയുക?
\end{malayalam}}
\flushright{\begin{Arabic}
\quranayah[55][31]
\end{Arabic}}
\flushleft{\begin{malayalam}
ഭൂമിക്ക് ഭാരമായ ജിന്നുകളേ, മനുഷ്യരേ, നിങ്ങളുടെ വിചാരണക്കായി നാം ഒഴിഞ്ഞു വരുന്നുണ്ട്.
\end{malayalam}}
\flushright{\begin{Arabic}
\quranayah[55][32]
\end{Arabic}}
\flushleft{\begin{malayalam}
അപ്പോള്‍ നിങ്ങളിരുകൂട്ടരുടെയും നാഥന്റെ ഏതനുഗ്രഹത്തെയാണ് നിങ്ങള്‍ തള്ളിപ്പറയുക?
\end{malayalam}}
\flushright{\begin{Arabic}
\quranayah[55][33]
\end{Arabic}}
\flushleft{\begin{malayalam}
ജിന്നുകളുടെയും മനുഷ്യരുടെയും സമൂഹമേ, ആകാശഭൂമികളുടെ അതിരുകള്‍ ഭേദിച്ച് പുറത്തു പോകാനാകുമെങ്കില്‍ നിങ്ങള്‍ പുറത്തുപോവുക. നിങ്ങള്‍ക്ക് പുറത്തുകടക്കാനാവില്ല. ഒരു മഹാശക്തിയുടെ പിന്‍ബലമില്ലാതെ.
\end{malayalam}}
\flushright{\begin{Arabic}
\quranayah[55][34]
\end{Arabic}}
\flushleft{\begin{malayalam}
അപ്പോള്‍ നിങ്ങളിരുകൂട്ടരുടെയും നാഥന്റെ ഏതനുഗ്രഹത്തെയാണ് നിങ്ങള്‍ തള്ളിപ്പറയുക?
\end{malayalam}}
\flushright{\begin{Arabic}
\quranayah[55][35]
\end{Arabic}}
\flushleft{\begin{malayalam}
നിങ്ങളിരുകൂട്ടരുടെയും നേരെ തീക്ഷ്ണമായ തീജ്ജ്വാലകളും പുകപടലങ്ങളും അയക്കും. നിങ്ങള്‍ക്കവയെ അതിജയിക്കാനാവില്ല.
\end{malayalam}}
\flushright{\begin{Arabic}
\quranayah[55][36]
\end{Arabic}}
\flushleft{\begin{malayalam}
അപ്പോള്‍ നിങ്ങളിരുകൂട്ടരുടെയും നാഥന്റെ ഏതനുഗ്രഹത്തെയാണ് നിങ്ങള്‍ തള്ളിപ്പറയുക?
\end{malayalam}}
\flushright{\begin{Arabic}
\quranayah[55][37]
\end{Arabic}}
\flushleft{\begin{malayalam}
ആകാശം പൊട്ടിപ്പിളര്‍ന്ന് റോസാപ്പൂ നിറമുള്ളതും കുഴമ്പുപോലുള്ളതും ആയിത്തീരുമ്പോഴുള്ള അവസ്ഥ എന്തായിരിക്കും?
\end{malayalam}}
\flushright{\begin{Arabic}
\quranayah[55][38]
\end{Arabic}}
\flushleft{\begin{malayalam}
അപ്പോള്‍ നിങ്ങളിരുവിഭാഗത്തിന്റെയും നാഥന്റെ ഏതനുഗ്രഹത്തെയാണ് നിങ്ങള്‍ തള്ളിപ്പറയുക?
\end{malayalam}}
\flushright{\begin{Arabic}
\quranayah[55][39]
\end{Arabic}}
\flushleft{\begin{malayalam}
അന്നേ ദിനം മനുഷ്യനോടോ ജിന്നിനോടോ അവരുടെ പാപമെന്തെന്ന് ചോദിച്ചറിയേണ്ടതില്ലാത്തവിധമത് വ്യക്തമായിരിക്കും.
\end{malayalam}}
\flushright{\begin{Arabic}
\quranayah[55][40]
\end{Arabic}}
\flushleft{\begin{malayalam}
അപ്പോള്‍ നിങ്ങളിരുകൂട്ടരുടെയും നാഥന്റെ ഏതനുഗ്രഹത്തെയാണ് നിങ്ങള്‍ തള്ളിപ്പറയുക?
\end{malayalam}}
\flushright{\begin{Arabic}
\quranayah[55][41]
\end{Arabic}}
\flushleft{\begin{malayalam}
കുറ്റവാളികളെ അവരുടെ ലക്ഷണങ്ങള്‍ കൊണ്ടുതന്നെ തിരിച്ചറിയുന്നതാണ്. അവരെ കുടുമയിലും പാദങ്ങളിലും പിടിച്ച് വലിച്ചിഴക്കും.
\end{malayalam}}
\flushright{\begin{Arabic}
\quranayah[55][42]
\end{Arabic}}
\flushleft{\begin{malayalam}
അപ്പോള്‍ നിങ്ങളിരുകൂട്ടരുടെയും നാഥന്റെ ഏതനുഗ്രഹത്തെയാണ് നിങ്ങള്‍ തള്ളിപ്പറയുക?
\end{malayalam}}
\flushright{\begin{Arabic}
\quranayah[55][43]
\end{Arabic}}
\flushleft{\begin{malayalam}
ഇതാകുന്നു കുറ്റവാളികള്‍ തള്ളിപ്പറയുന്ന നരകം.
\end{malayalam}}
\flushright{\begin{Arabic}
\quranayah[55][44]
\end{Arabic}}
\flushleft{\begin{malayalam}
അതിനും തിളച്ചുമറിയുന്ന ചൂടുവെള്ളത്തിനുമിടയില്‍ അവര്‍ കറങ്ങിക്കൊണ്ടിരിക്കും.
\end{malayalam}}
\flushright{\begin{Arabic}
\quranayah[55][45]
\end{Arabic}}
\flushleft{\begin{malayalam}
അപ്പോള്‍ നിങ്ങളിരുകൂട്ടരുടെയും നാഥന്റെ ഏതനുഗ്രഹത്തെയാണ് നിങ്ങള്‍ തള്ളിപ്പറയുക?
\end{malayalam}}
\flushright{\begin{Arabic}
\quranayah[55][46]
\end{Arabic}}
\flushleft{\begin{malayalam}
തന്റെ നാഥന്റെ സന്നിധിയില്‍ തന്നെ കൊണ്ടുവരുമെന്ന് ഭയന്നവന് രണ്ട് സ്വര്‍ഗീയാരാമങ്ങളുണ്ട്.
\end{malayalam}}
\flushright{\begin{Arabic}
\quranayah[55][47]
\end{Arabic}}
\flushleft{\begin{malayalam}
അപ്പോള്‍ നിങ്ങളിരുകൂട്ടരുടെയും നാഥന്റെ ഏത് അനുഗ്രഹത്തെയാണ് നിങ്ങള്‍ തള്ളിപ്പറയുക?
\end{malayalam}}
\flushright{\begin{Arabic}
\quranayah[55][48]
\end{Arabic}}
\flushleft{\begin{malayalam}
അതു രണ്ടും നിരവധി സുഖൈശ്വര്യങ്ങളുള്ളവയാണ്.
\end{malayalam}}
\flushright{\begin{Arabic}
\quranayah[55][49]
\end{Arabic}}
\flushleft{\begin{malayalam}
അപ്പോള്‍ നിങ്ങളിരുകൂട്ടരുടെയും നാഥന്റെ ഏതനുഗ്രഹത്തെയാണ് നിങ്ങള്‍ തള്ളിപ്പറയുക?
\end{malayalam}}
\flushright{\begin{Arabic}
\quranayah[55][50]
\end{Arabic}}
\flushleft{\begin{malayalam}
അവ രണ്ടിലും ഒഴുകിക്കൊണ്ടിരിക്കുന്ന രണ്ട് അരുവികളുണ്ട്.
\end{malayalam}}
\flushright{\begin{Arabic}
\quranayah[55][51]
\end{Arabic}}
\flushleft{\begin{malayalam}
അപ്പോള്‍ നിങ്ങളിരുകൂട്ടരുടെയും രക്ഷിതാവിന്റെ ഏതനുഗ്രഹത്തെയാണ് നിങ്ങള്‍ തള്ളിപ്പറയുക?
\end{malayalam}}
\flushright{\begin{Arabic}
\quranayah[55][52]
\end{Arabic}}
\flushleft{\begin{malayalam}
അവ രണ്ടിലും ഓരോ പഴത്തില്‍നിന്നുമുള്ള ഈരണ്ടു ഇനങ്ങളുണ്ട്.
\end{malayalam}}
\flushright{\begin{Arabic}
\quranayah[55][53]
\end{Arabic}}
\flushleft{\begin{malayalam}
അപ്പോള്‍ നിങ്ങളിരുകൂട്ടരുടെയും നാഥന്റെ ഏതനുഗ്രഹത്തെയാണ് നിങ്ങള്‍ തള്ളിപ്പറയുക?
\end{malayalam}}
\flushright{\begin{Arabic}
\quranayah[55][54]
\end{Arabic}}
\flushleft{\begin{malayalam}
അവര്‍ ചില മെത്തകളില്‍ ചാരിക്കിടക്കുന്നവരായിരിക്കും. അവയുടെ ഉള്‍ഭാഗം കട്ടികൂടിയ പട്ടുകൊണ്ടുള്ളതായിരിക്കും. ആ രണ്ടു തോട്ടങ്ങളിലെയും പഴങ്ങള്‍ താഴ്ന്നു കിടക്കുന്നവയുമായിരിക്കും.
\end{malayalam}}
\flushright{\begin{Arabic}
\quranayah[55][55]
\end{Arabic}}
\flushleft{\begin{malayalam}
അപ്പോള്‍ നിങ്ങള്‍ ഇരുകൂട്ടരുടെയും നാഥന്റെ ഏതനുഗ്രഹത്തെയാണ് നിങ്ങള്‍ തള്ളിപ്പറയുക?
\end{malayalam}}
\flushright{\begin{Arabic}
\quranayah[55][56]
\end{Arabic}}
\flushleft{\begin{malayalam}
അവയില്‍ നോട്ടം നിയന്ത്രിക്കുന്ന തരുണികളുണ്ടായിരിക്കും. ഇവര്‍ക്കു മുമ്പേ മനുഷ്യനോ ജിന്നോ അവരെ തൊട്ടിട്ടില്ല.
\end{malayalam}}
\flushright{\begin{Arabic}
\quranayah[55][57]
\end{Arabic}}
\flushleft{\begin{malayalam}
അപ്പോള്‍ നിങ്ങളിരുകൂട്ടരുടെയും നാഥന്റെ ഏതനുഗ്രഹത്തെയാണ് നിങ്ങള്‍ തള്ളിപ്പറയുക?
\end{malayalam}}
\flushright{\begin{Arabic}
\quranayah[55][58]
\end{Arabic}}
\flushleft{\begin{malayalam}
അവര്‍ മാണിക്യവും പവിഴവും പോലിരിക്കും.
\end{malayalam}}
\flushright{\begin{Arabic}
\quranayah[55][59]
\end{Arabic}}
\flushleft{\begin{malayalam}
അപ്പോള്‍ നിങ്ങളിരുകൂട്ടരുടെയും നാഥന്റെ ഏതനുഗ്രഹത്തെയാണ് നിങ്ങള്‍ തള്ളിപ്പറയുക?
\end{malayalam}}
\flushright{\begin{Arabic}
\quranayah[55][60]
\end{Arabic}}
\flushleft{\begin{malayalam}
നന്മയുടെ പ്രതിഫലം നന്മയല്ലാതെന്ത്?
\end{malayalam}}
\flushright{\begin{Arabic}
\quranayah[55][61]
\end{Arabic}}
\flushleft{\begin{malayalam}
അപ്പോള്‍ നിങ്ങളിരു കൂട്ടരുടെയും നാഥന്റെ ഏതനുഗ്രഹത്തെയാണ് നിങ്ങള്‍ തള്ളിപ്പറയുക?
\end{malayalam}}
\flushright{\begin{Arabic}
\quranayah[55][62]
\end{Arabic}}
\flushleft{\begin{malayalam}
അവ രണ്ടും കൂടാതെ വേറെയും രണ്ട് സ്വര്‍ഗത്തോപ്പുകളുണ്ട്.
\end{malayalam}}
\flushright{\begin{Arabic}
\quranayah[55][63]
\end{Arabic}}
\flushleft{\begin{malayalam}
അപ്പോള്‍ നിങ്ങളിരു കൂട്ടരുടെയും നാഥന്റെ ഏതനുഗ്രഹത്തെയാണ് നിങ്ങള്‍ തള്ളിപ്പറയുക?
\end{malayalam}}
\flushright{\begin{Arabic}
\quranayah[55][64]
\end{Arabic}}
\flushleft{\begin{malayalam}
പച്ചപ്പുനിറഞ്ഞ രണ്ടു സ്വര്‍ഗീയാരാമങ്ങള്‍.
\end{malayalam}}
\flushright{\begin{Arabic}
\quranayah[55][65]
\end{Arabic}}
\flushleft{\begin{malayalam}
അപ്പോള്‍ നിങ്ങളിരു കൂട്ടരുടെയും നാഥന്റെ ഏതനുഗ്രഹത്തെയാണ് നിങ്ങള്‍ തള്ളിപ്പറയുക?
\end{malayalam}}
\flushright{\begin{Arabic}
\quranayah[55][66]
\end{Arabic}}
\flushleft{\begin{malayalam}
അവ രണ്ടിലും കുതിച്ചൊഴുകുന്ന രണ്ട് അരുവികളുണ്ട്.
\end{malayalam}}
\flushright{\begin{Arabic}
\quranayah[55][67]
\end{Arabic}}
\flushleft{\begin{malayalam}
അപ്പോള്‍ നിങ്ങളിരു കൂട്ടരുടെയും നാഥന്റെ ഏതനുഗ്രഹത്തെയാണ് നിങ്ങള്‍ തള്ളിപ്പറയുക?
\end{malayalam}}
\flushright{\begin{Arabic}
\quranayah[55][68]
\end{Arabic}}
\flushleft{\begin{malayalam}
അവ രണ്ടിലും പലയിനം പഴങ്ങളുണ്ട്. ഈത്തപ്പനകളും ഉറുമാന്‍ പഴങ്ങളുമുണ്ട്.
\end{malayalam}}
\flushright{\begin{Arabic}
\quranayah[55][69]
\end{Arabic}}
\flushleft{\begin{malayalam}
അപ്പോള്‍ നിങ്ങളിരു കൂട്ടരുടെയും നാഥന്റെ ഏതനുഗ്രഹത്തെയാണ് നിങ്ങള്‍ തള്ളിപ്പറയുക?
\end{malayalam}}
\flushright{\begin{Arabic}
\quranayah[55][70]
\end{Arabic}}
\flushleft{\begin{malayalam}
അവയില്‍ സുശീലകളും സുന്ദരികളുമായ തരുണികളുണ്ട്.
\end{malayalam}}
\flushright{\begin{Arabic}
\quranayah[55][71]
\end{Arabic}}
\flushleft{\begin{malayalam}
അപ്പോള്‍ നിങ്ങളിരു കൂട്ടരുടെയും നാഥന്റെ ഏതനുഗ്രഹത്തെയാണ് നിങ്ങള്‍ തള്ളിപ്പറയുക?
\end{malayalam}}
\flushright{\begin{Arabic}
\quranayah[55][72]
\end{Arabic}}
\flushleft{\begin{malayalam}
അവര്‍ കൂടാരങ്ങളില്‍ ഒതുങ്ങിക്കഴിയുന്ന ഹൂറികളാണ്.
\end{malayalam}}
\flushright{\begin{Arabic}
\quranayah[55][73]
\end{Arabic}}
\flushleft{\begin{malayalam}
അപ്പോള്‍ നിങ്ങളിരു കൂട്ടരുടെയും നാഥന്റെ ഏതനുഗ്രഹത്തെയാണ് നിങ്ങള്‍ തള്ളിപ്പറയുക?
\end{malayalam}}
\flushright{\begin{Arabic}
\quranayah[55][74]
\end{Arabic}}
\flushleft{\begin{malayalam}
ഇവര്‍ക്കു മുമ്പേ മനുഷ്യനോ ജിന്നോ അവരെ തൊട്ടിട്ടില്ല.
\end{malayalam}}
\flushright{\begin{Arabic}
\quranayah[55][75]
\end{Arabic}}
\flushleft{\begin{malayalam}
അപ്പോള്‍ നിങ്ങളിരു കൂട്ടരുടെയും നാഥന്റെ ഏതനുഗ്രഹത്തെയാണ് നിങ്ങള്‍ തള്ളിപ്പറയുക?
\end{malayalam}}
\flushright{\begin{Arabic}
\quranayah[55][76]
\end{Arabic}}
\flushleft{\begin{malayalam}
അവര്‍ ചാരുതയാര്‍ന്ന പരവതാനികളിലും പച്ചപ്പട്ടിന്റെ തലയണകളിലും ചാരിക്കിടക്കുന്നവരായിരിക്കും.
\end{malayalam}}
\flushright{\begin{Arabic}
\quranayah[55][77]
\end{Arabic}}
\flushleft{\begin{malayalam}
എന്നിട്ടും നിങ്ങളിരു കൂട്ടരുടെയും നാഥന്റെ ഏതനുഗ്രഹത്തെയാണ് നിങ്ങള്‍ തള്ളിപ്പറയുക?
\end{malayalam}}
\flushright{\begin{Arabic}
\quranayah[55][78]
\end{Arabic}}
\flushleft{\begin{malayalam}
മഹോന്നതനും അത്യുദാരനുമായ നിന്റെ നാഥന്റെ നാമം അത്യുല്‍കൃഷ്ടം തന്നെ.
\end{malayalam}}
\chapter{\textmalayalam{അല്‍ വാഖിഅ ( സംഭവം )}}
\begin{Arabic}
\Huge{\centerline{\basmalah}}\end{Arabic}
\flushright{\begin{Arabic}
\quranayah[56][1]
\end{Arabic}}
\flushleft{\begin{malayalam}
ആ സംഭവം നടന്നുകഴിഞ്ഞാല്‍.
\end{malayalam}}
\flushright{\begin{Arabic}
\quranayah[56][2]
\end{Arabic}}
\flushleft{\begin{malayalam}
പിന്നെ അങ്ങനെ സംഭവിക്കുമെന്നത് നിഷേധിക്കുന്നവരുണ്ടാവില്ല.
\end{malayalam}}
\flushright{\begin{Arabic}
\quranayah[56][3]
\end{Arabic}}
\flushleft{\begin{malayalam}
അത് ചിലരെ താഴ്ത്തുന്നതും മറ്റു ചിലരെ ഉയര്‍ത്തുന്നതുമാണ്.
\end{malayalam}}
\flushright{\begin{Arabic}
\quranayah[56][4]
\end{Arabic}}
\flushleft{\begin{malayalam}
അപ്പോള്‍ ഭൂമി കിടുകിടാ വിറക്കും.
\end{malayalam}}
\flushright{\begin{Arabic}
\quranayah[56][5]
\end{Arabic}}
\flushleft{\begin{malayalam}
പര്‍വതങ്ങള്‍ തകര്‍ന്ന് തരിപ്പണമാകും.
\end{malayalam}}
\flushright{\begin{Arabic}
\quranayah[56][6]
\end{Arabic}}
\flushleft{\begin{malayalam}
അങ്ങനെയത് പാറിപ്പറക്കുന്ന പൊടിപടലമായിത്തീരും.
\end{malayalam}}
\flushright{\begin{Arabic}
\quranayah[56][7]
\end{Arabic}}
\flushleft{\begin{malayalam}
അന്നു നിങ്ങള്‍ മൂന്നു വിഭാഗമായിരിക്കും.
\end{malayalam}}
\flushright{\begin{Arabic}
\quranayah[56][8]
\end{Arabic}}
\flushleft{\begin{malayalam}
വലതു പക്ഷക്കാര്‍! ആഹാ! എന്തായിരിക്കും അന്ന് വലതുപക്ഷക്കാരുടെ അവസ്ഥ!
\end{malayalam}}
\flushright{\begin{Arabic}
\quranayah[56][9]
\end{Arabic}}
\flushleft{\begin{malayalam}
ഇടതുപക്ഷക്കാര്‍! ഹാവൂ! എന്തായിരിക്കും ഇടതുപക്ഷത്തിന്റെ അവസ്ഥ?
\end{malayalam}}
\flushright{\begin{Arabic}
\quranayah[56][10]
\end{Arabic}}
\flushleft{\begin{malayalam}
പിന്നെ മുന്നേറിയവര്‍! അവര്‍ അവിടെയും മുന്‍നിരക്കാര്‍ തന്നെ!
\end{malayalam}}
\flushright{\begin{Arabic}
\quranayah[56][11]
\end{Arabic}}
\flushleft{\begin{malayalam}
അവരാണ് ദിവ്യസാമീപ്യം സിദ്ധിച്ചവര്‍.
\end{malayalam}}
\flushright{\begin{Arabic}
\quranayah[56][12]
\end{Arabic}}
\flushleft{\begin{malayalam}
അനുഗൃഹീതമായ സ്വര്‍ഗീയാരാമങ്ങളിലായിരിക്കും അവര്‍.
\end{malayalam}}
\flushright{\begin{Arabic}
\quranayah[56][13]
\end{Arabic}}
\flushleft{\begin{malayalam}
അവരോ മുന്‍ഗാമികളില്‍നിന്ന് കുറേ പേര്‍.
\end{malayalam}}
\flushright{\begin{Arabic}
\quranayah[56][14]
\end{Arabic}}
\flushleft{\begin{malayalam}
പിന്‍ഗാമികളില്‍നിന്ന് കുറച്ചും.
\end{malayalam}}
\flushright{\begin{Arabic}
\quranayah[56][15]
\end{Arabic}}
\flushleft{\begin{malayalam}
അവര്‍ പൊന്നുനൂലുകൊണ്ടുണ്ടാക്കിയ കട്ടിലുകളിലായിരിക്കും.
\end{malayalam}}
\flushright{\begin{Arabic}
\quranayah[56][16]
\end{Arabic}}
\flushleft{\begin{malayalam}
അവയിലവര്‍ മുഖാമുഖം ചാരിയിരിക്കുന്നവരായിരിക്കും.
\end{malayalam}}
\flushright{\begin{Arabic}
\quranayah[56][17]
\end{Arabic}}
\flushleft{\begin{malayalam}
നിത്യബാല്യം നേടിയവര്‍ അവര്‍ക്കിടയില്‍ ചുറ്റിക്കറങ്ങും.
\end{malayalam}}
\flushright{\begin{Arabic}
\quranayah[56][18]
\end{Arabic}}
\flushleft{\begin{malayalam}
ശുദ്ധ ഉറവുജലം നിറച്ച കോപ്പകളും കൂജകളും ചഷകങ്ങളുമായി.
\end{malayalam}}
\flushright{\begin{Arabic}
\quranayah[56][19]
\end{Arabic}}
\flushleft{\begin{malayalam}
അതവര്‍ക്ക് തലകറക്കമോ ലഹരിയോ ഉണ്ടാക്കുകയില്ല.
\end{malayalam}}
\flushright{\begin{Arabic}
\quranayah[56][20]
\end{Arabic}}
\flushleft{\begin{malayalam}
ഇഷ്ടാനുസരണം തെരഞ്ഞെടുക്കാന്‍ അവര്‍ക്കവിടെ പലയിനം പഴങ്ങളുമുണ്ടായിരിക്കും.
\end{malayalam}}
\flushright{\begin{Arabic}
\quranayah[56][21]
\end{Arabic}}
\flushleft{\begin{malayalam}
അവരാഗ്രഹിക്കുന്ന പക്ഷിമാംസങ്ങളും.
\end{malayalam}}
\flushright{\begin{Arabic}
\quranayah[56][22]
\end{Arabic}}
\flushleft{\begin{malayalam}
വിശാലാക്ഷികളായ സുന്ദരിമാരും.
\end{malayalam}}
\flushright{\begin{Arabic}
\quranayah[56][23]
\end{Arabic}}
\flushleft{\begin{malayalam}
അവരോ ശ്രദ്ധയോടെ സൂക്ഷിക്കപ്പെട്ട മുത്തുപോലുള്ളവര്‍.
\end{malayalam}}
\flushright{\begin{Arabic}
\quranayah[56][24]
\end{Arabic}}
\flushleft{\begin{malayalam}
ഇതൊക്കെയും അവര്‍ പ്രവര്‍ത്തിച്ചതിന്റെ പ്രതിഫലമായാണ് അവര്‍ക്കു ലഭിക്കുക.
\end{malayalam}}
\flushright{\begin{Arabic}
\quranayah[56][25]
\end{Arabic}}
\flushleft{\begin{malayalam}
അവരവിടെ അപശബ്ദങ്ങളോ പാപവാക്കുകളോ കേള്‍ക്കുകയില്ല.
\end{malayalam}}
\flushright{\begin{Arabic}
\quranayah[56][26]
\end{Arabic}}
\flushleft{\begin{malayalam}
സമാധാനം! സമാധാനം! എന്ന അഭിവാദ്യമല്ലാതെ.
\end{malayalam}}
\flushright{\begin{Arabic}
\quranayah[56][27]
\end{Arabic}}
\flushleft{\begin{malayalam}
വലതുപക്ഷം! ആഹാ; എന്താണ് ഈ വലതുപക്ഷക്കാരുടെ അവസ്ഥ?
\end{malayalam}}
\flushright{\begin{Arabic}
\quranayah[56][28]
\end{Arabic}}
\flushleft{\begin{malayalam}
അവര്‍ക്കുള്ളതാണ് മുള്ളില്ലാത്ത ഇലന്തമരത്തോട്ടം.
\end{malayalam}}
\flushright{\begin{Arabic}
\quranayah[56][29]
\end{Arabic}}
\flushleft{\begin{malayalam}
പടലകളുള്ള കുലകളോടു കൂടിയ വാഴ.
\end{malayalam}}
\flushright{\begin{Arabic}
\quranayah[56][30]
\end{Arabic}}
\flushleft{\begin{malayalam}
പടര്‍ന്നു പരന്നു കിടക്കുന്ന നിഴല്‍.
\end{malayalam}}
\flushright{\begin{Arabic}
\quranayah[56][31]
\end{Arabic}}
\flushleft{\begin{malayalam}
അവിരാമം ഒഴുകിക്കൊണ്ടിരിക്കുന്ന തെളിനീര്‍.
\end{malayalam}}
\flushright{\begin{Arabic}
\quranayah[56][32]
\end{Arabic}}
\flushleft{\begin{malayalam}
ധാരാളം പഴങ്ങള്‍;
\end{malayalam}}
\flushright{\begin{Arabic}
\quranayah[56][33]
\end{Arabic}}
\flushleft{\begin{malayalam}
അവയോ ഒരിക്കലും ഒടുക്കമില്ലാത്തവയും തീരേ തടയപ്പെടാത്തവയുമത്രെ.
\end{malayalam}}
\flushright{\begin{Arabic}
\quranayah[56][34]
\end{Arabic}}
\flushleft{\begin{malayalam}
ഉന്നതമായ മെത്തകളും.
\end{malayalam}}
\flushright{\begin{Arabic}
\quranayah[56][35]
\end{Arabic}}
\flushleft{\begin{malayalam}
അവര്‍ക്കുള്ള ഇണകള്‍ നാം പ്രത്യേക ശ്രദ്ധയോടെ സൃഷ്ടിച്ചവരാണ്.
\end{malayalam}}
\flushright{\begin{Arabic}
\quranayah[56][36]
\end{Arabic}}
\flushleft{\begin{malayalam}
അവരെ നാം നിത്യ കന്യകകളാക്കിയിരിക്കുന്നു.
\end{malayalam}}
\flushright{\begin{Arabic}
\quranayah[56][37]
\end{Arabic}}
\flushleft{\begin{malayalam}
ഒപ്പം സ്നേഹസമ്പന്നരും സമപ്രായക്കാരും.
\end{malayalam}}
\flushright{\begin{Arabic}
\quranayah[56][38]
\end{Arabic}}
\flushleft{\begin{malayalam}
ഇതൊക്കെയും വലതുപക്ഷക്കാര്‍ക്കുള്ളതാണ്.
\end{malayalam}}
\flushright{\begin{Arabic}
\quranayah[56][39]
\end{Arabic}}
\flushleft{\begin{malayalam}
അവരോ പൂര്‍വികരില്‍ നിന്ന് ധാരാളമുണ്ട്.
\end{malayalam}}
\flushright{\begin{Arabic}
\quranayah[56][40]
\end{Arabic}}
\flushleft{\begin{malayalam}
പിന്‍മുറക്കാരില്‍നിന്നും ധാരാളമുണ്ട്.
\end{malayalam}}
\flushright{\begin{Arabic}
\quranayah[56][41]
\end{Arabic}}
\flushleft{\begin{malayalam}
ഇടതു പക്ഷക്കാര്‍! എന്താണ് ഇടതുപക്ഷത്തിന്റെ അവസ്ഥ?
\end{malayalam}}
\flushright{\begin{Arabic}
\quranayah[56][42]
\end{Arabic}}
\flushleft{\begin{malayalam}
അവര്‍ തീക്കാറ്റിലായിരിക്കും. തിളച്ചു തുള്ളുന്ന വെള്ളത്തിലും!
\end{malayalam}}
\flushright{\begin{Arabic}
\quranayah[56][43]
\end{Arabic}}
\flushleft{\begin{malayalam}
കരിമ്പുകയുടെ ഇരുണ്ട നിഴലിലും.
\end{malayalam}}
\flushright{\begin{Arabic}
\quranayah[56][44]
\end{Arabic}}
\flushleft{\begin{malayalam}
അത് തണുപ്പോ സുഖമോ നല്‍കുകയില്ല.
\end{malayalam}}
\flushright{\begin{Arabic}
\quranayah[56][45]
\end{Arabic}}
\flushleft{\begin{malayalam}
കാരണമവര്‍ അതിന് മുമ്പ് സുഖഭോഗങ്ങളില്‍ മുഴുകിയവരായിരുന്നു.
\end{malayalam}}
\flushright{\begin{Arabic}
\quranayah[56][46]
\end{Arabic}}
\flushleft{\begin{malayalam}
കൊടും പാപങ്ങളില്‍ ആണ്ടു പൂണ്ടവരും.
\end{malayalam}}
\flushright{\begin{Arabic}
\quranayah[56][47]
\end{Arabic}}
\flushleft{\begin{malayalam}
അവര്‍ ചോദിക്കാറുണ്ടായിരുന്നു; "ഞങ്ങള്‍ മരിച്ച് മണ്ണും എല്ലുമായി മാറിയാല്‍ പിന്നെ വീണ്ടും ഉയിര്‍ത്തെഴുന്നേല്‍പിക്കപ്പെടുമെന്നോ?
\end{malayalam}}
\flushright{\begin{Arabic}
\quranayah[56][48]
\end{Arabic}}
\flushleft{\begin{malayalam}
ഞങ്ങളുടെ പൂര്‍വ പിതാക്കളും?”
\end{malayalam}}
\flushright{\begin{Arabic}
\quranayah[56][49]
\end{Arabic}}
\flushleft{\begin{malayalam}
പറയുക: ഉറപ്പായും മുന്‍ഗാമികളും പിന്‍ഗാമികളും.
\end{malayalam}}
\flushright{\begin{Arabic}
\quranayah[56][50]
\end{Arabic}}
\flushleft{\begin{malayalam}
ഒരു നിര്‍ണിത നാളിലെ നിശ്ചിത സമയത്ത് ഒരുമിച്ചു ചേര്‍ക്കപ്പെടുക തന്നെ ചെയ്യും.
\end{malayalam}}
\flushright{\begin{Arabic}
\quranayah[56][51]
\end{Arabic}}
\flushleft{\begin{malayalam}
പിന്നെ, അല്ലയോ സത്യനിഷേധികളായ ദുര്‍മാര്‍ഗികളേ,
\end{malayalam}}
\flushright{\begin{Arabic}
\quranayah[56][52]
\end{Arabic}}
\flushleft{\begin{malayalam}
നിശ്ചയമായും നിങ്ങള്‍ സഖൂം വൃക്ഷത്തില്‍നിന്നാണ് തിന്നേണ്ടി വരിക.
\end{malayalam}}
\flushright{\begin{Arabic}
\quranayah[56][53]
\end{Arabic}}
\flushleft{\begin{malayalam}
അങ്ങനെ നിങ്ങളതുകൊണ്ട് വയറു നിറയ്ക്കും.
\end{malayalam}}
\flushright{\begin{Arabic}
\quranayah[56][54]
\end{Arabic}}
\flushleft{\begin{malayalam}
അതിനു മേലെ തിളച്ചുമറിയുന്ന വെള്ളം കുടിക്കുകയും ചെയ്യും.
\end{malayalam}}
\flushright{\begin{Arabic}
\quranayah[56][55]
\end{Arabic}}
\flushleft{\begin{malayalam}
ദാഹിച്ചു വലഞ്ഞ ഒട്ടകത്തെപ്പോലെ നിങ്ങളത് മോന്തും.
\end{malayalam}}
\flushright{\begin{Arabic}
\quranayah[56][56]
\end{Arabic}}
\flushleft{\begin{malayalam}
പ്രതിഫല നാളില്‍ അവര്‍ക്കുള്ള സല്‍ക്കാരമതായിരിക്കും.
\end{malayalam}}
\flushright{\begin{Arabic}
\quranayah[56][57]
\end{Arabic}}
\flushleft{\begin{malayalam}
നാമാണ് നിങ്ങളെ സൃഷ്ടിച്ചത്. എന്നിട്ടും നിങ്ങളിതിനെ സത്യമായംഗീകരിക്കാത്തതെന്ത്?
\end{malayalam}}
\flushright{\begin{Arabic}
\quranayah[56][58]
\end{Arabic}}
\flushleft{\begin{malayalam}
നിങ്ങള്‍ സ്രവിക്കുന്ന ശുക്ളത്തെ സംബന്ധിച്ച് ആലോചിച്ചുവോ?
\end{malayalam}}
\flushright{\begin{Arabic}
\quranayah[56][59]
\end{Arabic}}
\flushleft{\begin{malayalam}
നിങ്ങളാണോ അതിനെ സൃഷ്ടിക്കുന്നത്? അതോ നാമോ സൃഷ്ടികര്‍മം നിര്‍വഹിക്കുന്നത്?
\end{malayalam}}
\flushright{\begin{Arabic}
\quranayah[56][60]
\end{Arabic}}
\flushleft{\begin{malayalam}
നിങ്ങള്‍ക്കിടയില്‍ മരണം നിശ്ചയിച്ചതും നാം തന്നെ. നമ്മെ മറികടക്കാനാരുമില്ല.
\end{malayalam}}
\flushright{\begin{Arabic}
\quranayah[56][61]
\end{Arabic}}
\flushleft{\begin{malayalam}
നിങ്ങള്‍ക്കുപകരം നിങ്ങളെപ്പോലുള്ളവരെ ഉണ്ടാക്കാനും നിങ്ങള്‍ക്കറിയാത്ത വിധം നിങ്ങളെ വീണ്ടും സൃഷ്ടിക്കാനും നമുക്കു കഴിയും.
\end{malayalam}}
\flushright{\begin{Arabic}
\quranayah[56][62]
\end{Arabic}}
\flushleft{\begin{malayalam}
ആദ്യത്തെ സൃഷ്ടിയെ സംബന്ധിച്ച് നിശ്ചയമായും നിങ്ങള്‍ക്കറിയാമല്ലോ. എന്നിട്ടും നിങ്ങള്‍ ചിന്തിച്ചറിയാത്തതെന്ത്?
\end{malayalam}}
\flushright{\begin{Arabic}
\quranayah[56][63]
\end{Arabic}}
\flushleft{\begin{malayalam}
നിങ്ങള്‍ വിളയിറക്കുന്നതിനെക്കുറിച്ച് ചിന്തിച്ചുവോ?
\end{malayalam}}
\flushright{\begin{Arabic}
\quranayah[56][64]
\end{Arabic}}
\flushleft{\begin{malayalam}
നിങ്ങളാണോ അതിനെ മുളപ്പിക്കുന്നത്? അതോ നാമോ മുളപ്പിക്കുന്നവന്‍?
\end{malayalam}}
\flushright{\begin{Arabic}
\quranayah[56][65]
\end{Arabic}}
\flushleft{\begin{malayalam}
നാം ഉദ്ദേശിച്ചിരുന്നെങ്കില്‍ അതിനെ തുരുമ്പാക്കി മാറ്റുമായിരുന്നു. അപ്പോള്‍ നിങ്ങള്‍ നിരാശയോടെ പറയുമായിരുന്നു:
\end{malayalam}}
\flushright{\begin{Arabic}
\quranayah[56][66]
\end{Arabic}}
\flushleft{\begin{malayalam}
"ഞങ്ങള്‍ കടക്കെണിയിലായല്ലോ.
\end{malayalam}}
\flushright{\begin{Arabic}
\quranayah[56][67]
\end{Arabic}}
\flushleft{\begin{malayalam}
"എന്നല്ല; ഞങ്ങള്‍ ഉപജീവനം വിലക്കപ്പെട്ടവരായിപ്പോയല്ലോ.”
\end{malayalam}}
\flushright{\begin{Arabic}
\quranayah[56][68]
\end{Arabic}}
\flushleft{\begin{malayalam}
നിങ്ങള്‍ കുടിക്കുന്ന വെള്ളത്തെക്കുറിച്ച് ചിന്തിച്ചുവോ?
\end{malayalam}}
\flushright{\begin{Arabic}
\quranayah[56][69]
\end{Arabic}}
\flushleft{\begin{malayalam}
നിങ്ങളാണോ കാര്‍മുകിലില്‍നിന്ന് വെള്ളമിറക്കിയത്? അതോ നാമോ അതിറക്കിയവന്‍!
\end{malayalam}}
\flushright{\begin{Arabic}
\quranayah[56][70]
\end{Arabic}}
\flushleft{\begin{malayalam}
നാം ഉദ്ദേശിച്ചിരുന്നെങ്കില്‍ അതിനെ ഉപ്പുവെള്ളമാക്കി മാറ്റുമായിരുന്നു. എന്നിട്ടും നിങ്ങള്‍ നന്ദി കാണിക്കാത്തതെന്ത്?
\end{malayalam}}
\flushright{\begin{Arabic}
\quranayah[56][71]
\end{Arabic}}
\flushleft{\begin{malayalam}
നിങ്ങള്‍ കത്തിക്കുന്ന തീയിനെക്കുറിച്ച് ചിന്തിച്ചുവോ?
\end{malayalam}}
\flushright{\begin{Arabic}
\quranayah[56][72]
\end{Arabic}}
\flushleft{\begin{malayalam}
നിങ്ങളാണോ അതിനുള്ള മരമുണ്ടാക്കിയത്? അതോ നാമോ അത് പടച്ചുണ്ടാക്കിയത്?
\end{malayalam}}
\flushright{\begin{Arabic}
\quranayah[56][73]
\end{Arabic}}
\flushleft{\begin{malayalam}
നാമതിനെ ഒരു പാഠമാക്കിയിരിക്കുന്നു. വഴിപോക്കര്‍ക്ക് ജീവിത വിഭവവും.
\end{malayalam}}
\flushright{\begin{Arabic}
\quranayah[56][74]
\end{Arabic}}
\flushleft{\begin{malayalam}
അതിനാല്‍ നീ നിന്റെ മഹാനായ നാഥന്റെ നാമം വാഴ്ത്തുക.
\end{malayalam}}
\flushright{\begin{Arabic}
\quranayah[56][75]
\end{Arabic}}
\flushleft{\begin{malayalam}
അല്ല; ഞാനിതാ നക്ഷത്ര സ്ഥാനങ്ങളെക്കൊണ്ട് സത്യം ചെയ്യുന്നു.
\end{malayalam}}
\flushright{\begin{Arabic}
\quranayah[56][76]
\end{Arabic}}
\flushleft{\begin{malayalam}
ഇത് മഹത്തായ ശപഥം തന്നെ; തീര്‍ച്ച. നിങ്ങള്‍ അറിയുന്നുവെങ്കില്‍!
\end{malayalam}}
\flushright{\begin{Arabic}
\quranayah[56][77]
\end{Arabic}}
\flushleft{\begin{malayalam}
ഉറപ്പായും ഇത് ആദരണീയമായ ഖുര്‍ആന്‍ തന്നെ.
\end{malayalam}}
\flushright{\begin{Arabic}
\quranayah[56][78]
\end{Arabic}}
\flushleft{\begin{malayalam}
സുരക്ഷിതമായ ഗ്രന്ഥത്തില്‍.
\end{malayalam}}
\flushright{\begin{Arabic}
\quranayah[56][79]
\end{Arabic}}
\flushleft{\begin{malayalam}
വിശുദ്ധരല്ലാത്ത ആര്‍ക്കും ഇതിനെ സ്പര്‍ശിക്കാനാവില്ല.
\end{malayalam}}
\flushright{\begin{Arabic}
\quranayah[56][80]
\end{Arabic}}
\flushleft{\begin{malayalam}
മുഴുലോകരുടെയും നാഥനില്‍ നിന്ന് അവതീര്‍ണമായതാണിത്.
\end{malayalam}}
\flushright{\begin{Arabic}
\quranayah[56][81]
\end{Arabic}}
\flushleft{\begin{malayalam}
എന്നിട്ടും ഈ വചനങ്ങളോടാണോ നിങ്ങള്‍ നിസ്സംഗത പുലര്‍ത്തുന്നത്.
\end{malayalam}}
\flushright{\begin{Arabic}
\quranayah[56][82]
\end{Arabic}}
\flushleft{\begin{malayalam}
നിങ്ങളുടെ വിഹിതം അതിനെ കള്ളമാക്കി തള്ളലാണോ?
\end{malayalam}}
\flushright{\begin{Arabic}
\quranayah[56][83]
\end{Arabic}}
\flushleft{\begin{malayalam}
ജീവന്‍ തൊണ്ടക്കുഴിയിലെത്തുമ്പോള്‍ നിങ്ങള്‍ക്ക് എന്തുകൊണ്ട് അതിനെ പിടിച്ചു നിര്‍ത്താനാവുന്നില്ല?
\end{malayalam}}
\flushright{\begin{Arabic}
\quranayah[56][84]
\end{Arabic}}
\flushleft{\begin{malayalam}
മരണം വരിക്കുന്നവനെ നിങ്ങള്‍ നോക്കി നില്‍ക്കാറുണ്ടല്ലോ.
\end{malayalam}}
\flushright{\begin{Arabic}
\quranayah[56][85]
\end{Arabic}}
\flushleft{\begin{malayalam}
അപ്പോള്‍ നിങ്ങളെക്കാള്‍ അവനോട് ഏറെ അടുത്തവന്‍ നാമാകുന്നു. എന്നാല്‍ നിങ്ങളത് കണ്ടറിയുന്നില്ല.
\end{malayalam}}
\flushright{\begin{Arabic}
\quranayah[56][86]
\end{Arabic}}
\flushleft{\begin{malayalam}
അഥവാ, നിങ്ങള്‍ ദൈവിക നിയമത്തിന് വിധേയരല്ലെങ്കില്‍.
\end{malayalam}}
\flushright{\begin{Arabic}
\quranayah[56][87]
\end{Arabic}}
\flushleft{\begin{malayalam}
നിങ്ങളെന്തുകൊണ്ട് ആ ജീവനെ തിരിച്ചുകൊണ്ടുവരുന്നില്ല. നിങ്ങള്‍ സത്യവാന്മാരെങ്കില്‍!
\end{malayalam}}
\flushright{\begin{Arabic}
\quranayah[56][88]
\end{Arabic}}
\flushleft{\begin{malayalam}
മരിക്കുന്നവന്‍ ദൈവസാമീപ്യം സിദ്ധിച്ചവനാണെങ്കില്‍.
\end{malayalam}}
\flushright{\begin{Arabic}
\quranayah[56][89]
\end{Arabic}}
\flushleft{\begin{malayalam}
അവന് അവിടെ ആശ്വാസവും വിശിഷ്ട വിഭവവും അനുഗൃഹീതമായ സ്വര്‍ഗീയാരാമവുമുണ്ടായിരിക്കും.
\end{malayalam}}
\flushright{\begin{Arabic}
\quranayah[56][90]
\end{Arabic}}
\flushleft{\begin{malayalam}
അഥവാ, അവന്‍ വലതുപക്ഷക്കാരില്‍ പെട്ടവനെങ്കില്‍.
\end{malayalam}}
\flushright{\begin{Arabic}
\quranayah[56][91]
\end{Arabic}}
\flushleft{\begin{malayalam}
“വലതുപക്ഷക്കാരില്‍ പെട്ട നിനക്കു സമാധാനം” എന്ന് സ്വാഗതം ചെയ്യപ്പെടും.
\end{malayalam}}
\flushright{\begin{Arabic}
\quranayah[56][92]
\end{Arabic}}
\flushleft{\begin{malayalam}
മറിച്ച്, ദുര്‍മാര്‍ഗികളായ സത്യനിഷേധികളില്‍പെട്ടവനെങ്കിലോ.
\end{malayalam}}
\flushright{\begin{Arabic}
\quranayah[56][93]
\end{Arabic}}
\flushleft{\begin{malayalam}
അവന്നുണ്ടാവുക തിളച്ചുമറിയുന്ന വെള്ളംകൊണ്ടുള്ള സല്‍ക്കാരമായിരിക്കും.
\end{malayalam}}
\flushright{\begin{Arabic}
\quranayah[56][94]
\end{Arabic}}
\flushleft{\begin{malayalam}
നരകത്തിലെ കത്തിയെരിയലും.
\end{malayalam}}
\flushright{\begin{Arabic}
\quranayah[56][95]
\end{Arabic}}
\flushleft{\begin{malayalam}
തീര്‍ച്ചയായും ഇതൊക്കെയും സുദൃഢമായ സത്യം തന്നെ.
\end{malayalam}}
\flushright{\begin{Arabic}
\quranayah[56][96]
\end{Arabic}}
\flushleft{\begin{malayalam}
അതിനാല്‍ നീ നിന്റെ മഹാനായ നാഥന്റെ നാമം വാഴ്ത്തുക.
\end{malayalam}}
\chapter{\textmalayalam{ഹദീദ്  ( ഇരുമ്പ് )}}
\begin{Arabic}
\Huge{\centerline{\basmalah}}\end{Arabic}
\flushright{\begin{Arabic}
\quranayah[57][1]
\end{Arabic}}
\flushleft{\begin{malayalam}
ആകാശ ഭൂമികളിലുള്ളവയൊക്കെയും അല്ലാഹുവിന്റെ മഹത്വം കീര്‍ത്തിക്കുന്നു. അവന്‍ അജയ്യനും യുക്തിജ്ഞനുമാണ്.
\end{malayalam}}
\flushright{\begin{Arabic}
\quranayah[57][2]
\end{Arabic}}
\flushleft{\begin{malayalam}
ആകാശ ഭൂമികളുടെ ആധിപത്യം അവന്നാണ്. അവന്‍ ജീവിപ്പിക്കുകയും മരിപ്പിക്കുകയും ചെയ്യുന്നു. അവന്‍ എല്ലാ കാര്യങ്ങള്‍ക്കും കഴിവുറ്റവന്‍.
\end{malayalam}}
\flushright{\begin{Arabic}
\quranayah[57][3]
\end{Arabic}}
\flushleft{\begin{malayalam}
ആദ്യനും അന്ത്യനും പുറവും അകവും അവന്‍ തന്നെ. അവന്‍ സകല സംഗതികളും അറിയുന്നവന്‍.
\end{malayalam}}
\flushright{\begin{Arabic}
\quranayah[57][4]
\end{Arabic}}
\flushleft{\begin{malayalam}
ആറു നാളുകളിലായി ആകാശഭൂമികളെ സൃഷ്ടിച്ചത് അവനാണ്. പിന്നെ അവന്‍ സിംഹാസനസ്ഥനായി. ഭൂമിയില്‍ വരുന്നതും അവിടെ നിന്ന് പോകുന്നതും ആകാശത്തുനിന്നിറങ്ങുന്നതും അതിലേക്ക് കയറിപ്പോകുന്നതും അവനറിയുന്നു. നിങ്ങളെവിടെയായാലും അവന്‍ നിങ്ങളോടൊപ്പമുണ്ട്. അല്ലാഹു നിങ്ങള്‍ ചെയ്തുകൊണ്ടിരിക്കുന്നതൊക്കെ കണ്ടറിയുന്നവനാണ്.
\end{malayalam}}
\flushright{\begin{Arabic}
\quranayah[57][5]
\end{Arabic}}
\flushleft{\begin{malayalam}
ആകാശ ഭൂമികളുടെ ആധിപത്യം അവന്നാണ്. കാര്യങ്ങളൊക്കെയും മടക്കപ്പെടുന്നതും അവങ്കലേക്കാണ്.
\end{malayalam}}
\flushright{\begin{Arabic}
\quranayah[57][6]
\end{Arabic}}
\flushleft{\begin{malayalam}
അവന്‍ രാവിനെ പകലിലും പകലിനെ രാവിലും ചേര്‍ക്കുന്നു. അവന്‍ ഹൃദയ രഹസ്യങ്ങളെല്ലാം അറിയുന്നവനാണ്.
\end{malayalam}}
\flushright{\begin{Arabic}
\quranayah[57][7]
\end{Arabic}}
\flushleft{\begin{malayalam}
നിങ്ങള്‍ അല്ലാഹുവിലും അവന്റെ ദൂതനിലും വിശ്വസിക്കുക. അവന്‍ നിങ്ങളെ പ്രതിനിധികളാക്കിയ സമ്പത്തില്‍നിന്ന് ചെലവഴിക്കുകയും ചെയ്യുക. നിങ്ങളില്‍നിന്ന് സത്യവിശ്വാസം സ്വീകരിക്കുകയും സമ്പത്ത് ചെലവഴിക്കുകയും ചെയ്തവരാരോ, അവര്‍ക്കു മഹത്തായ പ്രതിഫലമുണ്ട്.
\end{malayalam}}
\flushright{\begin{Arabic}
\quranayah[57][8]
\end{Arabic}}
\flushleft{\begin{malayalam}
നിങ്ങള്‍ക്ക് എന്തുപറ്റി? നിങ്ങളെന്തുകൊണ്ട് അല്ലാഹുവില്‍ വിശ്വസിക്കുന്നില്ല? നിങ്ങളുടെ നാഥനില്‍ വിശ്വസിക്കാന്‍ ദൈവദൂതന്‍ നിങ്ങളെ നിരന്തരം ക്ഷണിച്ചുകൊണ്ടിരുന്നിട്ടും. അല്ലാഹു, നിങ്ങളില്‍നിന്ന് ഉറപ്പുവാങ്ങിയിട്ടുമുണ്ടല്ലോ. നിങ്ങള്‍ സത്യവിശ്വാസികളെങ്കില്‍.
\end{malayalam}}
\flushright{\begin{Arabic}
\quranayah[57][9]
\end{Arabic}}
\flushleft{\begin{malayalam}
തന്റെ ദാസന് സുവ്യക്തമായ സൂക്തങ്ങള്‍ അവതരിപ്പിച്ചുകൊടുക്കുന്നത് അവനാണ്. നിങ്ങളെ ഇരുളില്‍ നിന്ന് വെളിച്ചത്തിലേക്ക് നയിക്കാനാണത്. അല്ലാഹു നിങ്ങളോട് ഏറെ ദയാലുവും കരുണയുള്ളവനുമാണ്.
\end{malayalam}}
\flushright{\begin{Arabic}
\quranayah[57][10]
\end{Arabic}}
\flushleft{\begin{malayalam}
അല്ലാഹുവിന്റെ മാര്‍ഗത്തില്‍ ചെലവഴിക്കാതിരിക്കാന്‍ നിങ്ങള്‍ക്കെന്തുണ്ട് ന്യായം? - ആകാശ ഭൂമികളുടെ സമസ്താവകാശവും അവനു മാത്രമായിരുന്നിട്ടും. നിങ്ങളില്‍ മക്കാ വിജയത്തിനു മുമ്പെ ചെലവഴിക്കുകയും സമരം നടത്തുകയും ചെയ്തവരാരോ, അവര്‍ക്ക് അതിനു ശേഷം ചെലവഴിക്കുകയും സമരം നടത്തുകയും ചെയ്തവരെക്കാളേറെ മഹത്തായ പദവിയുണ്ട്. എല്ലാവര്‍ക്കും ഏറ്റവും ഉത്തമമായ പ്രതിഫലം അല്ലാഹു വാഗ്ദാനം ചെയ്തിരിക്കുന്നു. നിങ്ങള്‍ ചെയ്തുകൊണ്ടിരിക്കുന്നതൊക്കെയും നന്നായറിയുന്നവനാണ് അല്ലാഹു.
\end{malayalam}}
\flushright{\begin{Arabic}
\quranayah[57][11]
\end{Arabic}}
\flushleft{\begin{malayalam}
അല്ലാഹുവിന് ഉത്തമമായ കടം കൊടുക്കാന്‍ ആരുണ്ട്? എങ്കില്‍ അല്ലാഹു അത് അനേകമിരട്ടിയായി തിരിച്ചുതരും. മാന്യമായ പ്രതിഫലത്തിനര്‍ഹനും അയാള്‍തന്നെ.
\end{malayalam}}
\flushright{\begin{Arabic}
\quranayah[57][12]
\end{Arabic}}
\flushleft{\begin{malayalam}
നീ വിശ്വാസികളെയും വിശ്വാസിനികളെയും കാണും ദിനം; അവരുടെ മുന്നിലും വലതുവശത്തും പ്രകാശം പ്രസരിച്ചുകൊണ്ടിരിക്കും. അന്നവരോട് പറയും: നിങ്ങള്‍ക്ക് ശുഭാശംസകള്‍! നിങ്ങള്‍ക്ക് താഴ്ഭാഗത്തൂടെ ആറുകളൊഴുകുന്ന സ്വര്‍ഗീയാരാമങ്ങളുണ്ട്. നിങ്ങളതില്‍ നിത്യവാസികളായിരിക്കും. അതൊരു മഹാഭാഗ്യം തന്നെ!
\end{malayalam}}
\flushright{\begin{Arabic}
\quranayah[57][13]
\end{Arabic}}
\flushleft{\begin{malayalam}
കപടവിശ്വാസികളും വിശ്വാസിനികളും സത്യവിശ്വാസികളോട് ഇവ്വിധം പറയുന്ന ദിനമാണത്: നിങ്ങള്‍ ഞങ്ങള്‍ക്കായി കാത്തു നില്‍ക്കണേ, നിങ്ങളുടെ വെളിച്ചത്തില്‍ നിന്ന് ഇത്തിരി ഞങ്ങളും അനുഭവിക്കട്ടെ. അപ്പോള്‍ അവരോട് പറയും: "നിങ്ങള്‍ നിങ്ങളുടെ പിറകിലേക്കു തന്നെ തിരിച്ചുപോവുക. എന്നിട്ട് വെളിച്ചം തേടുക.” അപ്പോള്‍ അവര്‍ക്കിടയില്‍ ഒരു ഭിത്തി ഉയര്‍ത്തപ്പെടും. അതിനൊരു കവാടമുണ്ടായിരിക്കും. അതിന്റെ അകഭാഗത്ത് കാരുണ്യവും പുറഭാഗത്ത് ശിക്ഷയുമായിരിക്കും.
\end{malayalam}}
\flushright{\begin{Arabic}
\quranayah[57][14]
\end{Arabic}}
\flushleft{\begin{malayalam}
അവര്‍ വിശ്വാസികളെ വിളിച്ച് ചോദിക്കും: "ഞങ്ങള്‍ നിങ്ങളുടെ കൂടെയായിരുന്നില്ലേ?” സത്യവിശ്വാസികള്‍ പറയും: "അതെ. പക്ഷേ, നിങ്ങള്‍ നിങ്ങളെത്തന്നെ നാശത്തിലാഴ്ത്തി. അവസരവാദനയം സ്വീകരിച്ചു. സന്ദേഹികളാവുകയും ചെയ്തു. അല്ലാഹുവിന്റെ തീരുമാനം വന്നെത്തുംവരെ വ്യാമോഹം നിങ്ങളെ വഞ്ചിതരാക്കി. അല്ലാഹുവിന്റെ കാര്യത്തില്‍ കൊടുംവഞ്ചകന്‍ നിങ്ങളെ ചതിച്ചു.
\end{malayalam}}
\flushright{\begin{Arabic}
\quranayah[57][15]
\end{Arabic}}
\flushleft{\begin{malayalam}
"അതിനാലിന്ന് നിങ്ങളില്‍നിന്നും സത്യനിഷേധികളില്‍നിന്നും പ്രായശ്ചിത്തം സ്വീകരിക്കുന്നതല്ല. നിങ്ങളുടെ സങ്കേതം നരകമത്രെ. അതു തന്നെയാണ് നിങ്ങളുടെ അഭയസ്ഥാനം. ആ മടക്കസ്ഥലം വളരെ ചീത്ത തന്നെ.”
\end{malayalam}}
\flushright{\begin{Arabic}
\quranayah[57][16]
\end{Arabic}}
\flushleft{\begin{malayalam}
സത്യവിശ്വാസികളുടെ ഹൃദയങ്ങള്‍ ദൈവസ്മരണയ്ക്കും തങ്ങള്‍ക്ക് അവതീര്‍ണമായ സത്യവേദത്തിനും വിധേയമാകാന്‍ സമയമായില്ലേ? മുമ്പ് വേദം കിട്ടിയവരെപ്പോലെ ആകാതിരിക്കാനും. കാലം കുറേയേറെ കടന്നുപോയതിനാല്‍ അവരുടെ ഹൃദയങ്ങള്‍ കടുത്തുപോയി. അവരിലേറെ പേരും അധാര്‍മികരാണ്.
\end{malayalam}}
\flushright{\begin{Arabic}
\quranayah[57][17]
\end{Arabic}}
\flushleft{\begin{malayalam}
അറിയുക: അല്ലാഹു ഭൂമിയെ അതിന്റെ മൃതാവസ്ഥക്കുശേഷം ജീവസ്സുറ്റതാക്കുന്നു. നാം നിങ്ങള്‍ക്ക് ഉറപ്പായും ദൃഷ്ടാന്തങ്ങള്‍ വിവരിച്ചു തന്നിരിക്കുന്നു. നിങ്ങള്‍ ചിന്തിച്ചു മനസ്സിലാക്കാന്‍.
\end{malayalam}}
\flushright{\begin{Arabic}
\quranayah[57][18]
\end{Arabic}}
\flushleft{\begin{malayalam}
ദാനധര്‍മം നല്‍കിയ സ്ത്രീ പുരുഷന്മാര്‍ക്കും അല്ലാഹുവിന് ഉത്തമമായ കടം കൊടുത്തവര്‍ക്കും അനേകമിരട്ടി തിരിച്ചു കിട്ടും. അവര്‍ക്ക് മാന്യമായ പ്രതിഫലമുണ്ട്.
\end{malayalam}}
\flushright{\begin{Arabic}
\quranayah[57][19]
\end{Arabic}}
\flushleft{\begin{malayalam}
അല്ലാഹുവിലും അവന്റെ ദൂതന്മാരിലും വിശ്വസിച്ചവരാരോ, അവരാണ് തങ്ങളുടെ നാഥന്റെ സന്നിധിയില്‍ സത്യസന്ധരും സത്യസാക്ഷികളും. അവര്‍ക്ക് അവരുടെ പ്രതിഫലമുണ്ട്; വെളിച്ചവും. എന്നാല്‍ സത്യനിഷേധികളാവുകയും നമ്മുടെ വചനങ്ങളെ തള്ളിപ്പറയുകയും ചെയ്തവരോ; അവര്‍ തന്നെയാണ് നരകാവകാശികള്‍.
\end{malayalam}}
\flushright{\begin{Arabic}
\quranayah[57][20]
\end{Arabic}}
\flushleft{\begin{malayalam}
അറിയുക: ഈ ലോക ജീവിതം വെറും കളിയും തമാശയും പുറംപൂച്ചും പരസ്പരമുള്ള പൊങ്ങച്ച പ്രകടനവും സമ്പത്തിലും സന്താനങ്ങളിലുമുള്ള പെരുമ നടിക്കലും മാത്രമാണ്. അതൊരു മഴപോലെയാണ്. അതുവഴിയുണ്ടാവുന്ന ചെടികള്‍ കര്‍ഷകരെ സന്തോഷഭരിതരാക്കുന്നു. പിന്നെ അതുണങ്ങുന്നു. അങ്ങനെയത് മഞ്ഞച്ചതായി നിനക്കു കാണാം. വൈകാതെ അത് തുരുമ്പായിത്തീരുന്നു. എന്നാല്‍, പരലോകത്തോ; കഠിനമായ ശിക്ഷയുണ്ട്. അല്ലാഹുവില്‍ നിന്നുള്ള പാപമോചനവും പ്രീതിയുമുണ്ട്. ഐഹികജീവിതം ചതിച്ചരക്കല്ലാതൊന്നുമല്ല.
\end{malayalam}}
\flushright{\begin{Arabic}
\quranayah[57][21]
\end{Arabic}}
\flushleft{\begin{malayalam}
നിങ്ങള്‍ മത്സരിച്ചു മുന്നേറുക; നിങ്ങളുടെ നാഥനില്‍ നിന്നുള്ള പാപമോചനത്തിലേക്കും ആകാശഭൂമികളെപ്പോലെ വിശാലമായ സ്വര്‍ഗത്തിലേക്കും. അത് അല്ലാഹുവിലും അവന്റെ ദൂതന്മാരിലും വിശ്വസിച്ചവര്‍ക്കായി തയ്യാറാക്കിയതാണ്. അത് അല്ലാഹുവിന്റെ അനുഗ്രഹമാണ്. അവനുദ്ദേശിക്കുന്നവര്‍ക്ക് അവനത് നല്‍കുന്നു. അല്ലാഹു അത്യുദാരന്‍ തന്നെ.
\end{malayalam}}
\flushright{\begin{Arabic}
\quranayah[57][22]
\end{Arabic}}
\flushleft{\begin{malayalam}
ഭൂമിയിലോ നിങ്ങളിലോ ഒരു വിപത്തും വന്നുഭവിക്കുന്നില്ല; നാമത് മുമ്പേ ഒരു ഗ്രന്ഥത്തില്‍ രേഖപ്പെടുത്തി വച്ചിട്ടല്ലാതെ. അത് അല്ലാഹുവിന് ഏറെ എളുപ്പമുള്ള കാര്യമാണല്ലോ.
\end{malayalam}}
\flushright{\begin{Arabic}
\quranayah[57][23]
\end{Arabic}}
\flushleft{\begin{malayalam}
നിങ്ങള്‍ക്കുണ്ടാകുന്ന നഷ്ടത്തിന്റെ പേരില്‍ ദുഃഖിക്കാതിരിക്കാനും നിങ്ങള്‍ക്ക് അവന്‍ തരുന്നതിന്റെ പേരില്‍ സ്വയം മറന്നാഹ്ളാദിക്കാതിരിക്കാനുമാണത്. പെരുമ നടിക്കുന്നവരെയും പൊങ്ങച്ചക്കാരെയും അല്ലാഹു ഇഷ്ടപ്പെടുന്നില്ല.
\end{malayalam}}
\flushright{\begin{Arabic}
\quranayah[57][24]
\end{Arabic}}
\flushleft{\begin{malayalam}
അവരോ, സ്വയം പിശുക്ക് കാണിക്കുന്നവരും പിശുക്കരാകാന്‍ മറ്റുള്ളവരെ പ്രേരിപ്പിക്കുന്നവരുമാണ്. ആരെങ്കിലും സന്മാര്‍ഗത്തില്‍നിന്ന് പിന്തിരിയുന്നുവെങ്കില്‍ അറിയുക: അല്ലാഹു ആശ്രയമാവശ്യമില്ലാത്തവനും സ്തുത്യര്‍ഹനുമാണ്.
\end{malayalam}}
\flushright{\begin{Arabic}
\quranayah[57][25]
\end{Arabic}}
\flushleft{\begin{malayalam}
നിശ്ചയമായും നാം നമ്മുടെ ദൂതന്മാരെ തെളിഞ്ഞ തെളിവുകളുമായി നിയോഗിച്ചിരിക്കുന്നു. അവരോടൊപ്പം വേദവും തുലാസ്സും അവതരിപ്പിച്ചിരിക്കുന്നു. മനുഷ്യര്‍ നീതി നിലനിര്‍ത്താന്‍. നാം ഇരുമ്പും ഇറക്കിക്കൊടുത്തിരിക്കുന്നു. അതില്‍ ഏറെ ആയോധനശക്തിയും ജനങ്ങള്‍ക്കുപകാരവുമുണ്ട്. അല്ലാഹുവെ നേരില്‍ കാണാതെ തന്നെ അവനെയും അവന്റെ ദൂതന്മാരെയും സഹായിക്കുന്നവരാരെന്ന് അവന്ന് കണ്ടറിയാനാണിത്. അല്ലാഹു കരുത്തുറ്റവനും അജയ്യനും തന്നെ; തീര്‍ച്ച.
\end{malayalam}}
\flushright{\begin{Arabic}
\quranayah[57][26]
\end{Arabic}}
\flushleft{\begin{malayalam}
നിശ്ചയമായും നാം നൂഹിനെയും ഇബ്റാഹീമിനെയും ദൂതന്മാരായി നിയോഗിച്ചു. അവരിരുവരുടെയും സന്തതികളില്‍ പ്രവാചകത്വവും വേദഗ്രന്ഥവും നല്‍കി. അവരില്‍ നേര്‍വഴി പ്രാപിച്ചവരുണ്ട്. എന്നാല്‍ ഏറെപ്പേരും കുറ്റവാളികളായിരുന്നു.
\end{malayalam}}
\flushright{\begin{Arabic}
\quranayah[57][27]
\end{Arabic}}
\flushleft{\begin{malayalam}
പിന്നീട് അവര്‍ക്കു പിറകെ നാം നിരന്തരം നമ്മുടെ ദൂതന്മാരെ നിയോഗിച്ചു. മര്‍യമിന്റെ മകന്‍ ഈസയെയും അയച്ചു. അദ്ദേഹത്തിനു നാം ഇഞ്ചീല്‍ നല്‍കി. അദ്ദേഹത്തെ അനുഗമിച്ചവരുടെ ഹൃദയങ്ങളില്‍ കൃപയും കാരുണ്യവും വിളയിച്ചു. എന്നാല്‍ അവര്‍ സ്വയം സന്യാസം കെട്ടിച്ചമച്ചു. നാം അവര്‍ക്കത് നിയമമാക്കിയിരുന്നില്ല. ദൈവപ്രീതി പ്രതീക്ഷിച്ച് അവര്‍ പുതുതായി ഉണ്ടാക്കിയതാണത്. എന്നിട്ടോ, അവരത് യഥാവിധി പാലിച്ചതുമില്ല. അപ്പോള്‍ അവരില്‍ സത്യവിശ്വാസം സ്വീകരിച്ചവര്‍ക്ക് നാം അര്‍ഹമായ പ്രതിഫലം നല്‍കി. അവരിലേറെ പേരും അധാര്‍മികരാണ്.
\end{malayalam}}
\flushright{\begin{Arabic}
\quranayah[57][28]
\end{Arabic}}
\flushleft{\begin{malayalam}
സത്യവിശ്വാസം സ്വീകരിച്ചവരേ, നിങ്ങള്‍ അല്ലാഹുവോട് ഭക്തിയുള്ളവരാവുക. അവന്റെ ദൂതനില്‍ വിശ്വസിക്കുക. എങ്കില്‍ തന്റെ അനുഗ്രഹത്തില്‍ നിന്ന് നിങ്ങള്‍ക്കവന്‍ രണ്ട് ഓഹരി നല്‍കും. നിങ്ങള്‍ക്ക് നടക്കാനാവശ്യമായ വെളിച്ചം സമ്മാനിക്കും. നിങ്ങള്‍ക്ക് മാപ്പേകും. അല്ലാഹു ഏറെ പൊറുക്കുന്നവനും ദയാപരനുമല്ലോ.
\end{malayalam}}
\flushright{\begin{Arabic}
\quranayah[57][29]
\end{Arabic}}
\flushleft{\begin{malayalam}
അല്ലാഹുവിന്റെ അനുഗ്രഹത്തില്‍നിന്ന് ഒന്നും തട്ടിയെടുക്കാന്‍ തങ്ങള്‍ക്കാവില്ലെന്നും അനുഗ്രഹം അല്ലാഹുവിന്റെ കൈയിലാണെന്നും അത് താനുദ്ദേശിക്കുന്നവര്‍ക്ക് അവന്‍ നല്‍കുമെന്നും, വേദവാഹകര്‍ അറിയുവാന്‍ വേണ്ടിയാണിത്. അല്ലാഹു അതിമഹത്തായ ഔദാര്യത്തിനുടമയാകുന്നു.
\end{malayalam}}
\chapter{\textmalayalam{  മുജാദില ( തര്‍ക്കിക്കുന്നവള്‍ )}}
\begin{Arabic}
\Huge{\centerline{\basmalah}}\end{Arabic}
\flushright{\begin{Arabic}
\quranayah[58][1]
\end{Arabic}}
\flushleft{\begin{malayalam}
തന്റെ ഭര്‍ത്താവിനെക്കുറിച്ച് നിന്നോട് തര്‍ക്കിക്കുകയും അല്ലാഹുവോട് ആവലാതിപ്പെടുകയും ചെയ്യുന്നവളുടെ വാക്കുകള്‍ അല്ലാഹു കേട്ടിരിക്കുന്നു; തീര്‍ച്ച. അല്ലാഹു നിങ്ങളിരുവരുടെയും സംഭാഷണം ശ്രവിക്കുന്നുണ്ട്. നിശ്ചയമായും അല്ലാഹു എല്ലാം കേള്‍ക്കുന്നവനും കാണുന്നവനുമാകുന്നു.
\end{malayalam}}
\flushright{\begin{Arabic}
\quranayah[58][2]
\end{Arabic}}
\flushleft{\begin{malayalam}
നിങ്ങളില്‍ ചിലര്‍ ഭാര്യമാരെ ളിഹാര്‍ ചെയ്യുന്നു. എന്നാല്‍ ആ ഭാര്യമാര്‍ അവരുടെ മാതാക്കളല്ല. അവരെ പ്രസവിച്ചവര്‍ മാത്രമാണ് അവരുടെ മാതാക്കള്‍. അതിനാല്‍ നീചവും വ്യാജവുമായ വാക്കുകളാണ് അവര്‍ പറയുന്നത്. അല്ലാഹു വളരെ വിട്ടുവീഴ്ച ചെയ്യുന്നവനാണ്. ഏറെ പൊറുക്കുന്നവനും.
\end{malayalam}}
\flushright{\begin{Arabic}
\quranayah[58][3]
\end{Arabic}}
\flushleft{\begin{malayalam}
തങ്ങളുടെ ഭാര്യമാരെ ളിഹാര്‍ ചെയ്യുകയും പിന്നീട് തങ്ങള്‍ പറഞ്ഞതില്‍നിന്ന് പിന്‍മാറുകയും ചെയ്യുന്നവര്‍; ഇരുവരും പരസ്പരം സ്പര്‍ശിക്കുംമുമ്പെ ഒരടിമയെ മോചിപ്പിക്കണം. നിങ്ങള്‍ക്കു നല്‍കുന്ന ഉപദേശമാണിത്. നിങ്ങള്‍ ചെയ്യുന്നതിനെക്കുറിച്ചൊക്കെ നന്നായറിയുന്നവനാണ് അല്ലാഹു.
\end{malayalam}}
\flushright{\begin{Arabic}
\quranayah[58][4]
\end{Arabic}}
\flushleft{\begin{malayalam}
ആര്‍ക്കെങ്കിലും അടിമയെ കിട്ടുന്നില്ലെങ്കില്‍ അവര്‍ ശാരീരിക ബന്ധം പുലര്‍ത്തും മുമ്പെ പുരുഷന്‍ രണ്ടു മാസം തുടര്‍ച്ചയായി നോമ്പനുഷ്ഠിക്കണം. ആര്‍ക്കെങ്കിലും അതിനും കഴിയാതെ വരുന്നുവെങ്കില്‍ അയാള്‍ അറുപത് അഗതികള്‍ക്ക് അന്നം നല്‍കണം. നിങ്ങള്‍ അല്ലാഹുവിലും അവന്റെ ദൂതനിലുമുള്ള വിശ്വാസം സംരക്ഷിക്കാനാണിത്. അല്ലാഹു നിശ്ചയിച്ച ചിട്ടകളാണിവ. സത്യനിഷേധികള്‍ക്ക് നോവേറിയ ശിക്ഷയുണ്ട്.
\end{malayalam}}
\flushright{\begin{Arabic}
\quranayah[58][5]
\end{Arabic}}
\flushleft{\begin{malayalam}
അല്ലാഹുവോടും അവന്റെ ദൂതനോടും വിരോധം വെച്ചുപുലര്‍ത്തുന്നവര്‍ തങ്ങളുടെ മുന്‍ഗാമികള്‍ നിന്ദിക്കപ്പെട്ടപോലെ നിന്ദിതരാകും. നാം വ്യക്തമായ തെളിവുകള്‍ അവതരിപ്പിച്ചുകഴിഞ്ഞിരിക്കുന്നു; ഉറപ്പായും സത്യനിഷേധികള്‍ക്ക് അപമാനകരമായ ശിക്ഷയുണ്ട്.
\end{malayalam}}
\flushright{\begin{Arabic}
\quranayah[58][6]
\end{Arabic}}
\flushleft{\begin{malayalam}
അല്ലാഹു സകലരെയും ഉയിര്‍ത്തെഴുന്നേല്‍പിക്കുകയും തങ്ങള്‍ ചെയ്തുകൊണ്ടിരുന്നതെല്ലാം അവരെ ഓര്‍മിപ്പിക്കുകയും ചെയ്യുന്ന ദിവസം. അവരതൊക്കെ മറന്നിരിക്കാമെങ്കിലും അല്ലാഹു എല്ലാം രേഖപ്പെടുത്തി വെച്ചിട്ടുണ്ട്. അല്ലാഹു സകലകാര്യങ്ങള്‍ക്കും സാക്ഷിയാണ്.
\end{malayalam}}
\flushright{\begin{Arabic}
\quranayah[58][7]
\end{Arabic}}
\flushleft{\begin{malayalam}
ആകാശഭൂമികളിലുള്ളതെല്ലാം അല്ലാഹു അറിയുന്നുണ്ടെന്ന് നീ മനസ്സിലാക്കുന്നില്ലേ? മൂന്നാളുകള്‍ക്കിടയിലൊരു രഹസ്യഭാഷണവും നടക്കുന്നില്ല; നാലാമനായി അല്ലാഹുവില്ലാതെ. അല്ലെങ്കില്‍ അഞ്ചാളുകള്‍ക്കിടയില്‍ സ്വകാര്യ ഭാഷണം നടക്കുന്നില്ല; ആറാമനായി അവനില്ലാതെ. എണ്ണം ഇതിനെക്കാള്‍ കുറയട്ടെ, കൂടട്ടെ, അവര്‍ എവിടെയുമാകട്ടെ, അല്ലാഹു അവരോടൊപ്പമുണ്ട്. പിന്നെ അവരെന്താണ് ചെയ്തുകൊണ്ടിരുന്നതെന്ന് പുനരുത്ഥാന നാളില്‍ അവരെ ഉണര്‍ത്തുകയും ചെയ്യും. അല്ലാഹു സര്‍വജ്ഞനാണ്; തീര്‍ച്ച.
\end{malayalam}}
\flushright{\begin{Arabic}
\quranayah[58][8]
\end{Arabic}}
\flushleft{\begin{malayalam}
വിലക്കപ്പെട്ട ഗൂഢാലോചന വീണ്ടും നടത്തുന്നവരെ നീ കണ്ടില്ലേ? പാപത്തിനും അതിക്രമത്തിനും ദൈവദൂതനെ ധിക്കരിക്കാനുമാണ് അവര്‍ ഗൂഢാലോചന നടത്തുന്നത്. അവര്‍ നിന്റെ അടുത്തുവന്നാല്‍ അല്ലാഹു നിന്നെ അഭിവാദ്യം ചെയ്തിട്ടില്ലാത്ത വിധം അവര്‍ നിന്നെ അഭിവാദ്യം ചെയ്യുന്നു. എന്നിട്ട്: “ഞങ്ങളിങ്ങനെ പറയുന്നതിന്റെ പേരില്‍ അല്ലാഹു ഞങ്ങളെ ശിക്ഷിക്കാത്തതെന്ത്” എന്ന് അവര്‍ സ്വയം ചോദിക്കുകയും ചെയ്യുന്നു. അവര്‍ക്കു അര്‍ഹമായ ശിക്ഷ നരകം തന്നെ. അവരതിലെരിയും. അവരെത്തുന്നിടം എത്ര ചീത്ത!
\end{malayalam}}
\flushright{\begin{Arabic}
\quranayah[58][9]
\end{Arabic}}
\flushleft{\begin{malayalam}
വിശ്വസിച്ചവരേ, നിങ്ങള്‍ രഹസ്യാലോചന നടത്തുകയാണെങ്കില്‍ അത് പാപത്തിനും അതിക്രമത്തിനും പ്രവാചകധിക്കാരത്തിനും വേണ്ടിയാവരുത്. നന്മയുടെയും ഭക്തിയുടെയും കാര്യത്തില്‍ പരസ്പരാലോചന നടത്തുക. നിങ്ങള്‍ ദൈവഭക്തരാവുക. അവസാനം നിങ്ങള്‍ ഒത്തുകൂടുക അവന്റെ സന്നിധിയിലാണല്ലോ.
\end{malayalam}}
\flushright{\begin{Arabic}
\quranayah[58][10]
\end{Arabic}}
\flushleft{\begin{malayalam}
ഗൂഢാലോചന തീര്‍ത്തും പൈശാചികം തന്നെ. വിശ്വാസികളെ ദുഃഖിതരാക്കാന്‍ വേണ്ടിയാണത്. എന്നാല്‍ അല്ലാഹുവിന്റെ അനുമതിയില്ലാതെ അതവര്‍ക്കൊരു ദ്രോഹവും വരുത്തുകയില്ല. സത്യവിശ്വാസികള്‍ അല്ലാഹുവില്‍ ഭരമേല്‍പിച്ചുകൊള്ളട്ടെ.
\end{malayalam}}
\flushright{\begin{Arabic}
\quranayah[58][11]
\end{Arabic}}
\flushleft{\begin{malayalam}
സത്യവിശ്വാസികളേ, സദസ്സുകളില്‍ മറ്റുള്ളവര്‍ക്കു സൌകര്യമൊരുക്കിക്കൊടുക്കാന്‍ നിങ്ങളോടാവശ്യപ്പെട്ടാല്‍ നിങ്ങള്‍ നീങ്ങിയിരുന്ന് ഇടം നല്‍കുക. എങ്കില്‍ അല്ലാഹു നിങ്ങള്‍ക്കും സൌകര്യമൊരുക്കിത്തരും. “പിരിഞ്ഞുപോവുക” എന്നാണ് നിങ്ങളോടാവശ്യപ്പെടുന്നതെങ്കില്‍ നിങ്ങള്‍ എഴുന്നേറ്റ് പോവുക. നിങ്ങളില്‍നിന്ന് സത്യവിശ്വാസം സ്വീകരിച്ചവരുടെയും അറിവു നല്‍കപ്പെട്ടവരുടെയും പദവികള്‍ അല്ലാഹു ഉയര്‍ത്തുന്നതാണ്. നിങ്ങള്‍ ചെയ്യുന്നതൊക്കെയും നന്നായറിയുന്നവനാണ് അല്ലാഹു.
\end{malayalam}}
\flushright{\begin{Arabic}
\quranayah[58][12]
\end{Arabic}}
\flushleft{\begin{malayalam}
വിശ്വസിച്ചവരേ, നിങ്ങള്‍ ദൈവദൂതനുമായി സ്വകാര്യസംഭാഷണം നടത്തുകയാണെങ്കില്‍ നിങ്ങളുടെ രഹസ്യഭാഷണത്തിനു മുമ്പായി വല്ലതും ദാനമായി നല്‍കുക. അതു നിങ്ങള്‍ക്ക് പുണ്യവും പവിത്രവുമത്രെ. അഥവാ, നിങ്ങള്‍ക്ക് അതിന് കഴിവില്ലെങ്കില്‍, അപ്പോള്‍ അല്ലാഹു ഏറെ പൊറുക്കുന്നവനും ദയാപരനും തന്നെ; തീര്‍ച്ച.
\end{malayalam}}
\flushright{\begin{Arabic}
\quranayah[58][13]
\end{Arabic}}
\flushleft{\begin{malayalam}
നിങ്ങളുടെ സ്വകാര്യ സംഭാഷണങ്ങള്‍ക്കു മുമ്പേ വല്ലതും ദാനം നല്‍കണമെന്നത് നിങ്ങള്‍ക്ക് വിഷമകരമായോ? നിങ്ങള്‍ അങ്ങനെ ചെയ്യാതിരിക്കുകയും അല്ലാഹു നിങ്ങളുടെ പശ്ചാത്താപം സ്വീകരിക്കുകയും ചെയ്തതിനാല്‍ നിങ്ങള്‍ നമസ്കാരം നിഷ്ഠയോടെ നിര്‍വഹിക്കുക. സകാത് നല്‍കുക. അല്ലാഹുവിനെയും അവന്റെ ദൂതനെയും അനുസരിക്കുക. നിങ്ങള്‍ ചെയ്യുന്നതൊക്കെയും നന്നായറിയുന്നവനാണ് അല്ലാഹു.
\end{malayalam}}
\flushright{\begin{Arabic}
\quranayah[58][14]
\end{Arabic}}
\flushleft{\begin{malayalam}
ദൈവകോപത്തിന്നിരയായ ജനത യുമായി ഉറ്റബന്ധം സ്ഥാപിച്ച കപടവിശ്വാസികളെ നീ കണ്ടില്ലേ? അവര്‍ നിങ്ങളില്‍ പെട്ടവരോ ജൂതന്മാരില്‍ പെട്ടവരോ അല്ല. അവര്‍ ബോധപൂര്‍വം കള്ളസത്യം ചെയ്യുകയാണ്.
\end{malayalam}}
\flushright{\begin{Arabic}
\quranayah[58][15]
\end{Arabic}}
\flushleft{\begin{malayalam}
അല്ലാഹു അവര്‍ക്ക് കൊടിയ ശിക്ഷ ഒരുക്കിവെച്ചിട്ടുണ്ട്. അവര്‍ ചെയ്തുകൊണ്ടിരിക്കുന്നത് തീര്‍ത്തും ചീത്ത തന്നെ.
\end{malayalam}}
\flushright{\begin{Arabic}
\quranayah[58][16]
\end{Arabic}}
\flushleft{\begin{malayalam}
തങ്ങളുടെ ശപഥങ്ങളെ അവര്‍ ഒരു മറയായുപയോഗിക്കുകയാണ്. അങ്ങനെ അവര്‍ ജനങ്ങളെ ദൈവമാര്‍ഗത്തില്‍നിന്ന് തെറ്റിക്കുന്നു. അതിനാലവര്‍ക്ക് നിന്ദ്യമായ ശിക്ഷയുണ്ട്.
\end{malayalam}}
\flushright{\begin{Arabic}
\quranayah[58][17]
\end{Arabic}}
\flushleft{\begin{malayalam}
തങ്ങളുടെ സമ്പത്തോ സന്താനങ്ങളോ അല്ലാഹുവില്‍നിന്ന് രക്ഷ നേടാന്‍ അവര്‍ക്ക് ഒട്ടും ഉപകരിക്കുകയില്ല. അവര്‍ നരകാവകാശികളാണ്. അവരവിടെ സ്ഥിരവാസികളായിരിക്കും.
\end{malayalam}}
\flushright{\begin{Arabic}
\quranayah[58][18]
\end{Arabic}}
\flushleft{\begin{malayalam}
അവരെയെല്ലാം അല്ലാഹു ഉയിര്‍ത്തെഴുന്നേല്പിക്കുന്ന ദിവസം അവര്‍ നിങ്ങളോട് ശപഥം ചെയ്യുന്നതുപോലെ അവനോടും ശപഥം ചെയ്യും. അതുകൊണ്ട് തങ്ങള്‍ക്ക് നേട്ടം കിട്ടുമെന്ന് അവര്‍ കരുതുകയും ചെയ്യും. അറിയുക: തീര്‍ച്ചയായും അവര്‍ കള്ളം പറയുന്നവര്‍ തന്നെ.
\end{malayalam}}
\flushright{\begin{Arabic}
\quranayah[58][19]
\end{Arabic}}
\flushleft{\begin{malayalam}
പിശാച് അവരെ തന്റെ പിടിയിലൊതുക്കിയിരിക്കുന്നു. അങ്ങനെ അല്ലാഹുവെ ഓര്‍ക്കുന്നതില്‍ നിന്ന് അവനവരെ മറപ്പിച്ചിരിക്കുന്നു. അവരാണ് പിശാചിന്റെ പാര്‍ട്ടി. അറിയുക: നഷ്ടം പറ്റുന്നത് പിശാചിന്റെ പാര്‍ട്ടിക്കാര്‍ക്കുതന്നെയാണ്.
\end{malayalam}}
\flushright{\begin{Arabic}
\quranayah[58][20]
\end{Arabic}}
\flushleft{\begin{malayalam}
അല്ലാഹുവോടും അവന്റെ ദൂതനോടും വിരോധം വെക്കുന്നവര്‍ പരമനിന്ദ്യരില്‍ പെട്ടവരത്രെ.
\end{malayalam}}
\flushright{\begin{Arabic}
\quranayah[58][21]
\end{Arabic}}
\flushleft{\begin{malayalam}
ഉറപ്പായും താനും തന്റെ ദൂതന്മാരും തന്നെയാണ് വിജയം വരിക്കുകയെന്ന് അല്ലാഹു വിധി എഴുതിക്കഴിഞ്ഞിരിക്കുന്നു. അല്ലാഹു സര്‍വശക്തനും അജയ്യനുമാണ്; തീര്‍ച്ച.
\end{malayalam}}
\flushright{\begin{Arabic}
\quranayah[58][22]
\end{Arabic}}
\flushleft{\begin{malayalam}
അല്ലാഹുവിലും അന്ത്യദിനത്തിലും വിശ്വസിക്കുന്ന ഒരു ജനത, അല്ലാഹുവോടും അവന്റെ ദൂതനോടും വിരോധം വെച്ചുപുലര്‍ത്തുന്നവരോട് സ്നേഹബന്ധം സ്ഥാപിക്കുന്നതായി നിനക്ക് കാണാനാവില്ല. ആ വിരോധം വെച്ചുപുലര്‍ത്തുന്നവര്‍ സ്വന്തം പിതാക്കന്മാരോ പുത്രന്മാരോ സഹോദരന്മാരോ മറ്റു കുടുംബക്കാരോ ആരായിരുന്നാലും ശരി. അവരുടെ മനസ്സുകളില്‍ അല്ലാഹു സത്യവിശ്വാസം സുദൃഢമാക്കുകയും തന്നില്‍നിന്നുള്ള ആത്മചൈതന്യത്താല്‍ അവരെ പ്രബലരാക്കുകയും ചെയ്തിരിക്കുന്നു. അവന്‍ അവരെ താഴ്ഭാഗത്തൂടെ അരുവികളൊഴുകുന്ന സ്വര്‍ഗീയാരാമങ്ങളില്‍ പ്രവേശിപ്പിക്കും. അതിലവര്‍ സ്ഥിരവാസികളായിരിക്കും. അല്ലാഹു അവരില്‍ സംതൃപ്തനായിരിക്കും. അല്ലാഹുവിനെ സംബന്ധിച്ച് അവരും സംതൃപ്തരായിരിക്കും. അവരാണ് അല്ലാഹുവിന്റെ കക്ഷി. അറിയുക; ഉറപ്പായും അല്ലാഹുവിന്റെ കക്ഷിക്കാര്‍ തന്നെയാണ് വിജയം വരിക്കുന്നവര്‍.
\end{malayalam}}
\chapter{\textmalayalam{ഹഷ്ര്‍ ( തുരത്തിയോടിക്കല്‍ )}}
\begin{Arabic}
\Huge{\centerline{\basmalah}}\end{Arabic}
\flushright{\begin{Arabic}
\quranayah[59][1]
\end{Arabic}}
\flushleft{\begin{malayalam}
ആകാശഭൂമികളിലുള്ളവയെല്ലാം അല്ലാഹുവെ കീര്‍ത്തിക്കുന്നു. അവന്‍ അജയ്യനും യുക്തിമാനുമത്രെ.
\end{malayalam}}
\flushright{\begin{Arabic}
\quranayah[59][2]
\end{Arabic}}
\flushleft{\begin{malayalam}
ഒന്നാമത്തെ പടപ്പുറപ്പാടില്‍ തന്നെ വേദക്കാരിലെ സത്യനിഷേധികളെ അവരുടെ പാര്‍പ്പിടങ്ങളില്‍ നിന്ന് പുറത്താക്കിയത് അവനാണ്. അവര്‍ പുറത്തുപോകുമെന്ന് നിങ്ങള്‍ കരുതിയിരുന്നില്ല. അവരോ, തങ്ങളുടെ കോട്ടകള്‍ അല്ലാഹുവില്‍ നിന്ന് തങ്ങളെ രക്ഷിക്കുമെന്ന് കരുതിക്കഴിയുകയായിരുന്നു. എന്നാല്‍ അവര്‍ തീരെ പ്രതീക്ഷിക്കാത്ത വഴിയിലൂടെ അല്ലാഹു അവരുടെ നേരെ ചെന്നു. അവന്‍ അവരുടെ മനസ്സുകളില്‍ പേടി പടര്‍ത്തി. അങ്ങനെ അവര്‍ സ്വന്തം കൈകള്‍ കൊണ്ടുതന്നെ തങ്ങളുടെ പാര്‍പ്പിടങ്ങള്‍ തകര്‍ത്തുകൊണ്ടിരുന്നു. സത്യവിശ്വാസികള്‍ തങ്ങളുടെ കൈകളാലും. അതിനാല്‍ കണ്ണുള്ളവരേ, ഇതില്‍നിന്ന് പാഠമുള്‍ക്കൊള്ളുക.
\end{malayalam}}
\flushright{\begin{Arabic}
\quranayah[59][3]
\end{Arabic}}
\flushleft{\begin{malayalam}
അല്ലാഹു അവര്‍ക്ക് നാടുകടത്തല്‍ ശിക്ഷ വിധിച്ചില്ലായിരുന്നെങ്കില്‍ അവന്‍ അവരെ ഈ ലോകത്തുവെച്ചുതന്നെ ശിക്ഷിക്കുമായിരുന്നു. പരലോകത്ത് അവര്‍ക്ക് നരകശിക്ഷയാണുണ്ടാവുക.
\end{malayalam}}
\flushright{\begin{Arabic}
\quranayah[59][4]
\end{Arabic}}
\flushleft{\begin{malayalam}
അവര്‍ അല്ലാഹുവിനെയും അവന്റെ ദൂതനെയും എതിര്‍ത്തതിനാലാണിത്. അല്ലാഹുവോട് വിരോധം വെച്ചുപുലര്‍ത്തുന്നവര്‍, അറിയണം: അല്ലാഹു കഠിനമായി ശിക്ഷിക്കുന്നവനാണ്.
\end{malayalam}}
\flushright{\begin{Arabic}
\quranayah[59][5]
\end{Arabic}}
\flushleft{\begin{malayalam}
നിങ്ങള്‍ ചില ഈത്തപ്പനകളെ മുറിച്ചിട്ടതും ചിലതിനെ അവയുടെ മുരടുകളില്‍ നിലനിര്‍ത്തിയതും അല്ലാഹുവിന്റെ അനുമതിയോടെ തന്നെയാണ്. അധര്‍മചാരികളെ അപമാനിതരാക്കാനാണത്.
\end{malayalam}}
\flushright{\begin{Arabic}
\quranayah[59][6]
\end{Arabic}}
\flushleft{\begin{malayalam}
അവരില്‍നിന്ന് അല്ലാഹു തന്റെ ദൂതന് അധീനപ്പെടുത്തിക്കൊടുത്ത ധനമുണ്ടല്ലോ; അതിനായി നിങ്ങള്‍ക്ക് കുതിരകളെയും ഒട്ടകങ്ങളെയും ഓടിക്കേണ്ടി വന്നില്ല. എന്നാല്‍, അല്ലാഹു അവനാഗ്രഹിക്കുന്നവരുടെ മേല്‍ തന്റെ ദൂതന്മാര്‍ക്ക് ആധിപത്യമേകുന്നു. അല്ലാഹു എല്ലാറ്റിനും കഴിവുറ്റവനല്ലോ.
\end{malayalam}}
\flushright{\begin{Arabic}
\quranayah[59][7]
\end{Arabic}}
\flushleft{\begin{malayalam}
വിവിധ നാടുകളില്‍നിന്ന് അല്ലാഹു അവന്റെ ദൂതന് നേടിക്കൊടുത്ത തൊക്കെയും അല്ലാഹുവിനും അവന്റെ ദൂതന്നും അടുത്ത ബന്ധുക്കള്‍ക്കും അനാഥകള്‍ക്കും അഗതികള്‍ക്കും വഴിപോക്കര്‍ക്കുമുള്ളതാണ്. സമ്പത്ത് നിങ്ങളിലെ ധനികര്‍ക്കിടയില്‍ മാത്രം ചുറ്റിക്കറങ്ങാതിരിക്കാനാണിത്. ദൈവദൂതന്‍ നിങ്ങള്‍ക്കു നല്‍കുന്നതെന്തോ അതു നിങ്ങള്‍ സ്വീകരിക്കുക. വിലക്കുന്നതെന്തോ അതില്‍നിന്ന് വിട്ടകലുകയും ചെയ്യുക. അല്ലാഹുവോട് ഭക്തിയുള്ളവരാവുക. അല്ലാഹു കഠിനമായി ശിക്ഷിക്കുന്നവന്‍ തന്നെ; തീര്‍ച്ച.
\end{malayalam}}
\flushright{\begin{Arabic}
\quranayah[59][8]
\end{Arabic}}
\flushleft{\begin{malayalam}
തങ്ങളുടെ വീടുകളില്‍നിന്നും സ്വത്തുക്കളില്‍നിന്നും പുറംതള്ളപ്പെട്ട് പലായനം ചെയ്തുവന്ന പാവങ്ങള്‍ക്കുമുള്ളതാണ് യുദ്ധമുതല്‍. അല്ലാഹുവിന്റെ ഔദാര്യവും പ്രീതിയും തേടുന്നവരാണവര്‍. അല്ലാഹുവിനെയും അവന്റെ ദൂതനെയും സഹായിക്കുന്നവരും. അവര്‍ തന്നെയാണ് സത്യസന്ധര്‍.
\end{malayalam}}
\flushright{\begin{Arabic}
\quranayah[59][9]
\end{Arabic}}
\flushleft{\begin{malayalam}
അവരെത്തും മുമ്പേ സത്യവിശ്വാസം സ്വീകരിക്കുകയും അവിടെ താമസിക്കുകയും ചെയ്തവര്‍ക്കുമുള്ളതാണ് ആ സമരാര്‍ജിത സമ്പത്ത്. പലായനം ചെയ്ത് തങ്ങളിലേക്കെത്തുന്നവരെ അവര്‍ സ്നേഹിക്കുന്നു. അവര്‍ക്കു നല്‍കിയ സമ്പത്തിനോട് ഇവരുടെ മനസ്സുകളില്‍ ഒട്ടും മോഹമില്ല. തങ്ങള്‍ക്കു തന്നെ അത്യാവശ്യമുണ്ടെങ്കില്‍ പോലും അവര്‍ സ്വന്തത്തെക്കാള്‍ മറ്റുള്ളവര്‍ക്ക് മുന്‍ഗണന നല്‍കുന്നു. സ്വമനസ്സിന്റെ പിശുക്കില്‍ നിന്ന് മോചിതരായവര്‍ ആരോ, അവര്‍തന്നെയാണ് വിജയം വരിച്ചവര്‍.
\end{malayalam}}
\flushright{\begin{Arabic}
\quranayah[59][10]
\end{Arabic}}
\flushleft{\begin{malayalam}
ഈ യുദ്ധമുതല്‍ അവര്‍ക്കു ശേഷം വന്നെത്തിയവര്‍ക്കുമുള്ളതാണ്. അവര്‍ ഇങ്ങനെ പ്രാര്‍ഥിക്കുന്നവരാണ്: "ഞങ്ങളുടെ നാഥാ, നീ ഞങ്ങള്‍ക്കും ഞങ്ങളുടെ മുമ്പെ സത്യവിശ്വാസം സ്വീകരിച്ച ഞങ്ങളുടെ സഹോദരങ്ങള്‍ക്കും പൊറുത്തുതരേണമേ! ഞങ്ങളുടെ മനസ്സുകളില്‍ വിശ്വാസികളോട് ഒട്ടും വെറുപ്പ് ഉണ്ടാക്കരുതേ. ഞങ്ങളുടെ നാഥാ! ഉറപ്പായും നീ ദയാപരനും പരമകാരുണികനുമല്ലോ.”
\end{malayalam}}
\flushright{\begin{Arabic}
\quranayah[59][11]
\end{Arabic}}
\flushleft{\begin{malayalam}
കാപട്യം കാണിച്ചവരെ നീ കണ്ടില്ലേ? വേദക്കാരിലെ സത്യനിഷേധികളായ സഹോദരങ്ങളോട് അവര്‍ പറയുന്നു: "നിങ്ങള്‍ നാടുകടത്തപ്പെടുകയാണെങ്കില്‍ നിശ്ചയമായും നിങ്ങളോടൊപ്പം ഞങ്ങളും പുറത്തുപോരും. നിങ്ങളുടെ കാര്യത്തില്‍ ഞങ്ങള്‍ ഒരിക്കലും മറ്റാരെയും അനുസരിക്കുകയില്ല. നിങ്ങള്‍ക്കെതിരെ യുദ്ധമുണ്ടായാല്‍ ഉറപ്പായും ഞങ്ങള്‍ നിങ്ങളെ സഹായിക്കും.” എന്നാല്‍ ഈ കപടന്മാര്‍ കള്ളം പറയുന്നവരാണെന്ന് അല്ലാഹു സാക്ഷ്യം വഹിക്കുന്നു.
\end{malayalam}}
\flushright{\begin{Arabic}
\quranayah[59][12]
\end{Arabic}}
\flushleft{\begin{malayalam}
അവര്‍ പുറത്താക്കപ്പെട്ടാല്‍ ഒരിക്കലും ഇക്കൂട്ടര്‍ കൂടെ പുറത്തു പോവുകയില്ല. അവര്‍ യുദ്ധത്തിന്നിരയായാല്‍ ഈ കപടന്മാര്‍ അവരെ സഹായിക്കുകയുമില്ല. അഥവാ; സഹായിക്കാനിറങ്ങിയാല്‍ തന്നെ പിന്തിരിഞ്ഞോടും; തീര്‍ച്ച. പിന്നെ, അവര്‍ക്ക് ഒരിടത്തുനിന്നും ഒരു സഹായവും ലഭിക്കുകയില്ല.
\end{malayalam}}
\flushright{\begin{Arabic}
\quranayah[59][13]
\end{Arabic}}
\flushleft{\begin{malayalam}
ആ കപടവിശ്വാസികളുടെ മനസ്സുകളില്‍ അല്ലാഹുവോടുള്ളതിലേറെ ഭയം നിങ്ങളോടാണ്. കാരണം, അവരൊട്ടും കാര്യബോധമില്ലാത്ത ജനതയാണെന്നതുതന്നെ.
\end{malayalam}}
\flushright{\begin{Arabic}
\quranayah[59][14]
\end{Arabic}}
\flushleft{\begin{malayalam}
ഭദ്രമായ കോട്ടകളോടുകൂടിയ പട്ടണങ്ങളില്‍ വെച്ചോ വന്‍മതിലുകള്‍ക്കു പിറകെ ഒളിച്ചിരുന്നോ അല്ലാതെ അവരൊരിക്കലും ഒന്നായി നിങ്ങളോട് യുദ്ധം ചെയ്യുകയില്ല. അവര്‍ക്കിടയില്‍ പരസ്പരപോര് അതിരൂക്ഷമത്രെ. അവര്‍ ഒറ്റക്കെട്ടാണെന്ന് നീ കരുതുന്നു. എന്നാല്‍ അവരുടെ മനസ്സുകള്‍ പലതാണ്. കാരണം, അവര്‍ കാര്യം ശരിയാംവിധം മനസ്സിലാക്കാത്തവരാണെന്നതുതന്നെ.
\end{malayalam}}
\flushright{\begin{Arabic}
\quranayah[59][15]
\end{Arabic}}
\flushleft{\begin{malayalam}
അവര്‍ അവരുടെ തൊട്ടുമുമ്പുള്ളവരെപ്പോലെത്തന്നെയാണ്. അവര്‍ തങ്ങളുടെ പ്രവര്‍ത്തനങ്ങളുടെ ദുഷ്ഫലം അനുഭവിച്ചുകഴിഞ്ഞു. ഇവര്‍ക്കും നോവേറിയ ശിക്ഷയുണ്ട്.
\end{malayalam}}
\flushright{\begin{Arabic}
\quranayah[59][16]
\end{Arabic}}
\flushleft{\begin{malayalam}
പിശാചിനെപ്പോലെയാണ്ഇവര്‍. നീ അവിശ്വാസിയാവുക എന്ന് പിശാച് മനുഷ്യനോട് പറയും. അങ്ങനെ മനുഷ്യന്‍ അവിശ്വാസിയായാല്‍ പിശാച് പറയും: "എനിക്ക് നീയുമായി ഒരു ബന്ധവുമില്ല. എനിക്കു പ്രപഞ്ചനാഥനായ അല്ലാഹുവിനെ ഭയമാണ്.”
\end{malayalam}}
\flushright{\begin{Arabic}
\quranayah[59][17]
\end{Arabic}}
\flushleft{\begin{malayalam}
ഇരുവരുടെയും പര്യവസാനം, നരകത്തില്‍ നിത്യവാസികളാവുക എന്നതത്രെ. അതാണ് അക്രമികള്‍ക്കുള്ള പ്രതിഫലം.
\end{malayalam}}
\flushright{\begin{Arabic}
\quranayah[59][18]
\end{Arabic}}
\flushleft{\begin{malayalam}
സത്യവിശ്വാസികളേ, നിങ്ങള്‍ ദൈവഭക്തരാവുക. നാളേക്കുവേണ്ടി താന്‍ തയ്യാറാക്കിയത് എന്തെന്ന് ഓരോ മനുഷ്യനും ആലോചിക്കട്ടെ. അല്ലാഹുവോട് ഭക്തിയുള്ളവരാവുക. നിങ്ങള്‍ ചെയ്യുന്നതൊക്കെയും നന്നായറിയുന്നവനാണ് അല്ലാഹു.
\end{malayalam}}
\flushright{\begin{Arabic}
\quranayah[59][19]
\end{Arabic}}
\flushleft{\begin{malayalam}
അല്ലാഹുവെ മറന്നതിനാല്‍, തങ്ങളെത്തന്നെ മറക്കുന്നവരാക്കി അല്ലാഹു മാറ്റിയ ജനത്തെപ്പോലെ ആകരുത് നിങ്ങള്‍. അവര്‍ തന്നെയാണ് ദുര്‍മാര്‍ഗികള്‍.
\end{malayalam}}
\flushright{\begin{Arabic}
\quranayah[59][20]
\end{Arabic}}
\flushleft{\begin{malayalam}
നരകവാസികളും സ്വര്‍ഗവാസികളും തുല്യരാവുകയില്ല. സ്വര്‍ഗവാസികളോ; അവര്‍തന്നെയാണ് വിജയികള്‍.
\end{malayalam}}
\flushright{\begin{Arabic}
\quranayah[59][21]
\end{Arabic}}
\flushleft{\begin{malayalam}
നാം ഈ ഖുര്‍ആനിനെ ഒരു പര്‍വതത്തിന്മേലാണ് ഇറക്കിയിരുന്നതെങ്കില്‍ ദൈവഭയത്താല്‍ അത് ഏറെ വിനീതമാകുന്നതും പൊട്ടിപ്പിളരുന്നതും നിനക്കു കാണാമായിരുന്നു. ഈ ഉദാഹരണങ്ങളെല്ലാം നാം മനുഷ്യര്‍ക്കായി വിവരിക്കുകയാണ്. അവര്‍ ആലോചിച്ചറിയാന്‍.
\end{malayalam}}
\flushright{\begin{Arabic}
\quranayah[59][22]
\end{Arabic}}
\flushleft{\begin{malayalam}
അവനാണ് അല്ലാഹു. അവനല്ലാതെ ദൈവമില്ല. കാണുന്നതും കാണാത്തതും അറിയുന്നവനാണവന്‍. അവന്‍ ദയാപരനും കരുണാമയനുമാണ്.
\end{malayalam}}
\flushright{\begin{Arabic}
\quranayah[59][23]
\end{Arabic}}
\flushleft{\begin{malayalam}
അവനാണ് അല്ലാഹു. അവനല്ലാതെ ദൈവമില്ല. രാജാധിരാജന്‍; പരമപവിത്രന്‍, സമാധാന ദായകന്‍, അഭയദാതാവ്, മേല്‍നോട്ടക്കാരന്‍, അജയ്യന്‍, പരമാധികാരി, സര്‍വോന്നതന്‍, എല്ലാം അവന്‍ തന്നെ. ജനം പങ്കുചേര്‍ക്കുന്നതില്‍ നിന്നെല്ലാം അല്ലാഹു ഏറെ പരിശുദ്ധനാണ്.
\end{malayalam}}
\flushright{\begin{Arabic}
\quranayah[59][24]
\end{Arabic}}
\flushleft{\begin{malayalam}
അവനാണ് അല്ലാഹു. സ്രഷ്ടാവും നിര്‍മാതാവും രൂപരചയിതാവും അവന്‍തന്നെ. വിശിഷ്ടനാമങ്ങളൊക്കെയും അവന്നുള്ളതാണ്. ആകാശഭൂമികളിലുള്ളവയെല്ലാം അവന്റെ മഹത്വം കീര്‍ത്തിച്ചുകൊണ്ടിരിക്കുന്നു. അവനാണ് അജയ്യനും യുക്തിജ്ഞനും.
\end{malayalam}}
\chapter{\textmalayalam{മുംതഹന (. പരീക്ഷിക്കപ്പെടേണ്ടവള്‍ )}}
\begin{Arabic}
\Huge{\centerline{\basmalah}}\end{Arabic}
\flushright{\begin{Arabic}
\quranayah[60][1]
\end{Arabic}}
\flushleft{\begin{malayalam}
വിശ്വസിച്ചവരേ, നിങ്ങള്‍ എന്റെയും നിങ്ങളുടെയും ശത്രുക്കളുമായി സ്നേഹബന്ധം സ്ഥാപിച്ച് അവരെ രക്ഷാധികാരികളാക്കരുത്. നിങ്ങള്‍ക്കു വന്നെത്തിയ സത്യത്തെ അവര്‍ തള്ളിപ്പറഞ്ഞിരിക്കുന്നു. നിങ്ങളുടെ നാഥനായ അല്ലാഹുവില്‍ വിശ്വസിച്ചുവെന്നതിനാല്‍ അവര്‍ ദൈവദൂതനെയും നിങ്ങളെയും നാടുകടത്തുന്നു. എന്റെ മാര്‍ഗത്തില്‍ സമരം ചെയ്യാനും എന്റെ പ്രീതി നേടാനും തന്നെയാണ് നിങ്ങള്‍ ഇറങ്ങിത്തിരിച്ചതെങ്കില്‍ അങ്ങനെ ചെയ്യരുത്. എന്നാല്‍ നിങ്ങള്‍ അവരുമായി സ്വകാര്യത്തില്‍ സ്നേഹബന്ധം പുലര്‍ത്തുകയാണ്. നിങ്ങള്‍ ഒളിഞ്ഞും തെളിഞ്ഞും ചെയ്യുന്നതെല്ലാം ഞാന്‍ നന്നായി അറിയുന്നുണ്ട്. നിങ്ങളില്‍ അങ്ങനെ ചെയ്യുന്നവര്‍ നിശ്ചയമായും നേര്‍വഴിയില്‍നിന്ന് തെറ്റിപ്പോയിരിക്കുന്നു.
\end{malayalam}}
\flushright{\begin{Arabic}
\quranayah[60][2]
\end{Arabic}}
\flushleft{\begin{malayalam}
നിങ്ങള്‍ അവരുടെ പിടിയില്‍ പെട്ടാല്‍ നിങ്ങളോട് കൊടിയ ശത്രുത കാണിക്കുന്നവരാണ് അവര്‍. കയ്യും നാവുമുപയോഗിച്ച് അവര്‍ നിങ്ങളെ ദ്രോഹിക്കും. നിങ്ങള്‍ സത്യനിഷേധികളായിത്തീര്‍ന്നെങ്കില്‍ എന്ന് അവരാഗ്രഹിക്കുന്നു.
\end{malayalam}}
\flushright{\begin{Arabic}
\quranayah[60][3]
\end{Arabic}}
\flushleft{\begin{malayalam}
ഉയിര്‍ത്തെഴുന്നേല്പു നാളില്‍ നിങ്ങളുടെ കുടുംബക്കാരോ മക്കളോ നിങ്ങള്‍ക്കൊട്ടും ഉപകരിക്കുകയില്ല. അന്ന് അല്ലാഹു നിങ്ങളെ അന്യോന്യം വേര്‍പിരിക്കും. അല്ലാഹു നിങ്ങള്‍ ചെയ്യുന്നതൊക്കെ കണ്ടറിയുന്നവനാണ്.
\end{malayalam}}
\flushright{\begin{Arabic}
\quranayah[60][4]
\end{Arabic}}
\flushleft{\begin{malayalam}
തീര്‍ച്ചയായും ഇബ്റാഹീമിലും അദ്ദേഹത്തോടൊപ്പമുള്ളവരിലും നിങ്ങള്‍ക്ക് മഹിതമായ മാതൃകയുണ്ട്. അവര്‍ തങ്ങളുടെ ജനതയോട് ഇവ്വിധം പറഞ്ഞ സന്ദര്‍ഭം: "നിങ്ങളുമായോ അല്ലാഹുവെക്കൂടാതെ നിങ്ങള്‍ ആരാധിക്കുന്നവയുമായോ ഞങ്ങള്‍ക്കൊരു ബന്ധവുമില്ല. ഞങ്ങള്‍ നിങ്ങളെ അവിശ്വസിച്ചിരിക്കുന്നു. നിങ്ങള്‍ ഏകനായ അല്ലാഹുവില്‍ വിശ്വസിക്കുന്നതുവരെ ഞങ്ങള്‍ക്കും നിങ്ങള്‍ക്കുമിടയില്‍ വെറുപ്പും വിരോധവും പ്രകടമത്രെ.” ഇതില്‍നിന്ന് വ്യത്യസ്തമായുള്ളത് ഇബ്റാഹീം തന്റെ പിതാവിനോടിങ്ങനെ പറഞ്ഞതു മാത്രമാണ്: “തീര്‍ച്ചയായും ഞാന്‍ താങ്കളുടെ പാപമോചനത്തിനായി പ്രാര്‍ഥിക്കാം. എന്നാല്‍ അല്ലാഹുവില്‍നിന്ന് അങ്ങയ്ക്ക് എന്തെങ്കിലും നേടിത്തരിക എന്നത് എന്റെ കഴിവില്‍ പെട്ടതല്ല.” അവര്‍ പ്രാര്‍ഥിച്ചു: "ഞങ്ങളുടെ നാഥാ! ഞങ്ങള്‍ നിന്നില്‍ മാത്രം ഭരമേല്പിക്കുന്നു. നിന്നിലേക്കു മാത്രം പശ്ചാത്തപിച്ചു മടങ്ങുന്നു. അവസാനം ഞങ്ങള്‍ വന്നെത്തുന്നതും നിന്റെ അടുത്തേക്കുതന്നെ.
\end{malayalam}}
\flushright{\begin{Arabic}
\quranayah[60][5]
\end{Arabic}}
\flushleft{\begin{malayalam}
"ഞങ്ങളുടെ നാഥാ! സത്യനിഷേധികള്‍ക്കുള്ള പരീക്ഷണങ്ങള്‍ക്ക് ഞങ്ങളെ നീ വിധേയരാക്കരുതേ! ഞങ്ങളുടെ നാഥാ! ഞങ്ങള്‍ക്കു നീ പൊറുത്തുതരേണമേ! നീ അജയ്യനും യുക്തിജ്ഞനുമല്ലോ.”
\end{malayalam}}
\flushright{\begin{Arabic}
\quranayah[60][6]
\end{Arabic}}
\flushleft{\begin{malayalam}
നിങ്ങള്‍ക്ക്, അല്ലാഹുവിലും അന്ത്യദിനത്തിലും പ്രതീക്ഷയര്‍പ്പിക്കുന്നവര്‍ക്ക്, അവരില്‍ ഉത്തമ മാതൃകയുണ്ട്. ആരെങ്കിലും അതിനെ നിരാകരിക്കുന്നുവെങ്കില്‍ അറിയുക; നിശ്ചയം, അല്ലാഹു ആശ്രയമാവശ്യമില്ലാത്തവനും സ്തുത്യര്‍ഹനുമാകുന്നു.
\end{malayalam}}
\flushright{\begin{Arabic}
\quranayah[60][7]
\end{Arabic}}
\flushleft{\begin{malayalam}
നിങ്ങള്‍ക്കും നിങ്ങള്‍ ശത്രുതപുലര്‍ത്തുന്നവര്‍ക്കുമിടയില്‍ അല്ലാഹു ഒരുവേള സൌഹൃദം സ്ഥാപിച്ചേക്കാം. അല്ലാഹു എല്ലാറ്റിനും കഴിവുറ്റവനാണ്. അല്ലാഹു ഏറെ പൊറുക്കുന്നവനും ദയാപരനുമാണ്.
\end{malayalam}}
\flushright{\begin{Arabic}
\quranayah[60][8]
\end{Arabic}}
\flushleft{\begin{malayalam}
മതത്തിന്റെ പേരില്‍ നിങ്ങളോട് പൊരുതുകയോ, നിങ്ങളുടെ വീടുകളില്‍നിന്ന് നിങ്ങളെ ആട്ടിപ്പുറത്താക്കുകയോ ചെയ്യാത്തവരോട് നന്മ ചെയ്യുന്നതും നീതി കാണിക്കുന്നതും അല്ലാഹു വിലക്കുന്നില്ല. നീതി കാട്ടുന്നവരെ തീര്‍ച്ചയായും അല്ലാഹു ഇഷ്ടപ്പെടുന്നു.
\end{malayalam}}
\flushright{\begin{Arabic}
\quranayah[60][9]
\end{Arabic}}
\flushleft{\begin{malayalam}
മതത്തിന്റെ പേരില്‍ നിങ്ങളോട് പൊരുതുകയും നിങ്ങളുടെ വീടുകളില്‍നിന്ന് നിങ്ങളെ പുറത്താക്കുകയും നിങ്ങളെ പുറത്താക്കാന്‍ പരസ്പരം സഹായിക്കുകയും ചെയ്തവരെ ആത്മമിത്രങ്ങളാക്കുന്നത് മാത്രമാണ് അല്ലാഹു വിലക്കിയിട്ടുള്ളത്. അത്തരക്കാരെ ആത്മമിത്രങ്ങളാക്കുന്നവരാരോ, അവര്‍ തന്നെയാണ് അക്രമികള്‍.
\end{malayalam}}
\flushright{\begin{Arabic}
\quranayah[60][10]
\end{Arabic}}
\flushleft{\begin{malayalam}
വിശ്വസിച്ചവരേ, വിശ്വാസിനികള്‍ അഭയം തേടി നിങ്ങളെ സമീപിച്ചാല്‍ അവരെ പരീക്ഷിച്ചു നോക്കുക. അവരുടെ വിശ്വാസവിശുദ്ധിയെ സംബന്ധിച്ച് അല്ലാഹു നന്നായറിയുന്നു. അവര്‍ യഥാര്‍ഥ വിശ്വാസിനികളാണെന്ന് ബോധ്യമായാല്‍ പിന്നെ നിങ്ങളവരെ സത്യനിഷേധികളിലേക്ക് തിരിച്ചയക്കരുത്. ആ വിശ്വാസിനികള്‍ സത്യനിഷേധികള്‍ക്ക് അനുവദിക്കപ്പെട്ടവരല്ല. ആ സത്യനിഷേധികള്‍ വിശ്വാസിനികള്‍ക്കും അനുവദനീയരല്ല. അവര്‍ വ്യയം ചെയ്തത് നിങ്ങള്‍ അവര്‍ക്ക് മടക്കിക്കൊടുക്കുക. നിങ്ങള്‍ അവരെ വിവാഹം ചെയ്യുന്നതിന് വിലക്കൊന്നുമില്ല- അവര്‍ക്ക് അവരുടെ വിവാഹമൂല്യം നല്‍കുകയാണെങ്കില്‍. സത്യനിഷേധിനികളുമായുള്ള വിവാഹബന്ധം നിങ്ങളും നിലനിര്‍ത്തരുത്. നിങ്ങളവര്‍ക്കു നല്‍കിയത് തിരിച്ചു ചോദിക്കുക. അവര്‍ ചെലവഴിച്ചതെന്തോ അതത്രയും അവരും ആവശ്യപ്പെട്ടുകൊള്ളട്ടെ. അതാണ് അല്ലാഹുവിന്റെ വിധി. അവന്‍ നിങ്ങള്‍ക്കിടയില്‍ വിധി കല്പിക്കുന്നു. അല്ലാഹു സര്‍വജ്ഞനും യുക്തിമാനുമാണ്.
\end{malayalam}}
\flushright{\begin{Arabic}
\quranayah[60][11]
\end{Arabic}}
\flushleft{\begin{malayalam}
സത്യനിഷേധികളിലേക്കു പോയ നിങ്ങളുടെ ഭാര്യമാര്‍ക്കു നല്‍കിയ വിവാഹമൂല്യം നിങ്ങള്‍ക്കു തിരിച്ചുകിട്ടാതെ നഷ്ടപ്പെടുകയും എന്നിട്ട് നിങ്ങള്‍ അനന്തര നടപടി സ്വീകരിക്കുകയും ചെയ്താല്‍ ആരുടെ ഭാര്യമാരാണോ നഷ്ടപ്പെട്ടത് അവര്‍ നല്‍കിയ വിവാഹമൂല്യത്തിനു തുല്യമായ തുക അവര്‍ക്കു നല്‍കുക. നിങ്ങള്‍ വിശ്വസിച്ചംഗീകരിച്ച അല്ലാഹുവോട് ഭക്തിയുള്ളവരാവുക.
\end{malayalam}}
\flushright{\begin{Arabic}
\quranayah[60][12]
\end{Arabic}}
\flushleft{\begin{malayalam}
പ്രവാചകരേ, അല്ലാഹുവില്‍ ഒന്നിനെയും പങ്കുചേര്‍ക്കുകയില്ല; മോഷ്ടിക്കുകയില്ല; വ്യഭിചരിക്കുകയില്ല; സന്താനഹത്യ നടത്തുകയില്ല; തങ്ങളുടെ കൈകാലുകള്‍ക്കിടയില്‍ യാതൊരു വ്യാജവും മെനഞ്ഞുണ്ടാക്കുകയില്ല; നല്ല കാര്യത്തിലൊന്നും നിന്നോട് അനുസരണക്കേട് കാണിക്കുകയില്ല എന്നിങ്ങനെ പ്രതിജ്ഞ ചെയ്തുകൊണ്ട് സത്യവിശ്വാസിനികള്‍ നിന്നെ സമീപിച്ചാല്‍ അവരുടെ പ്രതിജ്ഞ സ്വീകരിച്ചുകൊള്ളുക. അവര്‍ക്കുവേണ്ടി അല്ലാഹുവോട് പാപമോചനം തേടുക. അല്ലാഹു ഏറെ പൊറുക്കുന്നവനും ദയാപരനുമാകുന്നു.
\end{malayalam}}
\flushright{\begin{Arabic}
\quranayah[60][13]
\end{Arabic}}
\flushleft{\begin{malayalam}
സത്യവിശ്വാസികളേ, അല്ലാഹുവിന്റെ കോപത്തിനിരയായ ജനത്തെ നിങ്ങള്‍ രക്ഷാധികാരികളാക്കരുത്. അവര്‍ പരലോകത്തെപ്പറ്റി തീര്‍ത്തും നിരാശരായിരിക്കുന്നു. ശവക്കുഴിയിലുള്ളവരെ സംബന്ധിച്ച് സത്യനിഷേധികള്‍ നിരാശരായപോലെ.
\end{malayalam}}
\chapter{\textmalayalam{സ്വഫ്ഫ് ( അണി )}}
\begin{Arabic}
\Huge{\centerline{\basmalah}}\end{Arabic}
\flushright{\begin{Arabic}
\quranayah[61][1]
\end{Arabic}}
\flushleft{\begin{malayalam}
ആകാശഭൂമികളിലുള്ളവയൊക്കെയും അല്ലാഹുവിന്റെ മഹത്വം കീര്‍ത്തിച്ചിരിക്കുന്നു. അവന്‍ അജയ്യനും യുക്തിജ്ഞനും തന്നെ!
\end{malayalam}}
\flushright{\begin{Arabic}
\quranayah[61][2]
\end{Arabic}}
\flushleft{\begin{malayalam}
വിശ്വസിച്ചവരേ, നിങ്ങള്‍ ചെയ്യാത്തത് പറയുന്നതെന്തിനാണ്?
\end{malayalam}}
\flushright{\begin{Arabic}
\quranayah[61][3]
\end{Arabic}}
\flushleft{\begin{malayalam}
ചെയ്യാത്തത് പറഞ്ഞുകൊണ്ടിരിക്കുകയെന്നത് അല്ലാഹുവിന് ഏറെ വെറുപ്പുള്ള കാര്യമാണ്.
\end{malayalam}}
\flushright{\begin{Arabic}
\quranayah[61][4]
\end{Arabic}}
\flushleft{\begin{malayalam}
കരുത്തുറ്റ മതില്‍ക്കെട്ടുപോലെ അണിചേര്‍ന്ന് അല്ലാഹുവിന്റെ മാര്‍ഗത്തില്‍ അടരാടുന്നവരെയാണ് അവന്‍ ഏറെ ഇഷ്ടപ്പെടുന്നത്.
\end{malayalam}}
\flushright{\begin{Arabic}
\quranayah[61][5]
\end{Arabic}}
\flushleft{\begin{malayalam}
മൂസ തന്റെ ജനതയോട് പറഞ്ഞത് ഓര്‍ക്കുക: "എന്റെ ജനമേ, നിങ്ങളെന്തിനാണ് എന്നെ പ്രയാസപ്പെടുത്തുന്നത്? നിശ്ചയമായും നിങ്ങള്‍ക്കറിയാം; ഞാന്‍ നിങ്ങളിലേക്കുള്ള ദൈവദൂതനാണെന്ന്.” അങ്ങനെ അവര്‍ വഴിപിഴച്ചപ്പോള്‍ അല്ലാഹു അവരുടെ മനസ്സുകളെ നേര്‍വഴിയില്‍നിന്ന് വ്യതിചലിപ്പിച്ചു. അധര്‍മകാരികളെ അല്ലാഹു നേര്‍വഴിയിലാക്കുകയില്ല.
\end{malayalam}}
\flushright{\begin{Arabic}
\quranayah[61][6]
\end{Arabic}}
\flushleft{\begin{malayalam}
മര്‍യമിന്റെ മകന്‍ ഈസാ പറഞ്ഞത് ഓര്‍ക്കുക: "ഇസ്രായേല്‍ മക്കളേ, ഞാന്‍ നിങ്ങളിലേക്കുള്ള ദൈവദൂതനാണ്. എനിക്കു മുമ്പേ അവതീര്‍ണമായ തൌറാത്തിനെ സത്യപ്പെടുത്തുന്നവന്‍. എനിക്കുശേഷം ആഗതനാകുന്ന അഹ്മദ് എന്നു പേരുള്ള ദൈവദൂതനെ സംബന്ധിച്ച് സുവാര്‍ത്ത അറിയിക്കുന്നവനും.” അങ്ങനെ അദ്ദേഹം തെളിഞ്ഞ തെളിവുകളുമായി അവരുടെ അടുത്തു വന്നപ്പോള്‍ അവര്‍ പറഞ്ഞു: ഇതു വ്യക്തമായും ഒരു മായാജാലം തന്നെ.
\end{malayalam}}
\flushright{\begin{Arabic}
\quranayah[61][7]
\end{Arabic}}
\flushleft{\begin{malayalam}
അല്ലാഹുവിനെക്കുറിച്ച് കള്ളം കെട്ടിച്ചമച്ചവനെക്കാള്‍ കൊടിയ അക്രമി ആരുണ്ട്? അതും അവന്‍ ഇസ്ലാമിലേക്ക് ക്ഷണിക്കപ്പെട്ടുകൊണ്ടിരിക്കെ. അല്ലാഹു അക്രമികളായ ജനത്തെ നേര്‍വഴിയിലാക്കുകയില്ല.
\end{malayalam}}
\flushright{\begin{Arabic}
\quranayah[61][8]
\end{Arabic}}
\flushleft{\begin{malayalam}
തങ്ങളുടെ വായകൊണ്ട് അല്ലാഹുവിന്റെ പ്രകാശത്തെ ഊതിക്കെടുത്താനാണ് അവരുദ്ദേശിക്കുന്നത്. അല്ലാഹു തന്റെ പ്രകാശത്തെ പൂര്‍ണമായി പരത്തുകതന്നെ ചെയ്യും. സത്യനിഷേധികള്‍ക്ക് അതെത്ര അരോചകമാണെങ്കിലും!
\end{malayalam}}
\flushright{\begin{Arabic}
\quranayah[61][9]
\end{Arabic}}
\flushleft{\begin{malayalam}
അല്ലാഹുവാണ് തന്റെ ദൂതനെ നേര്‍മാര്‍ഗവും സത്യമതവുമായി നിയോഗിച്ചത്. മറ്റെല്ലാ ജീവിതക്രമങ്ങളെക്കാളും അതിനെ വിജയിപ്പിച്ചെടുക്കാന്‍. ബഹുദൈവാരാധകര്‍ക്ക് അതെത്ര അനിഷ്ടകരമാണെങ്കിലും.
\end{malayalam}}
\flushright{\begin{Arabic}
\quranayah[61][10]
\end{Arabic}}
\flushleft{\begin{malayalam}
വിശ്വസിച്ചവരേ, നോവേറിയ ശിക്ഷയില്‍ നിന്ന് നിങ്ങളെ രക്ഷിക്കുന്ന ഒരു വ്യാപാരത്തെക്കുറിച്ച് ഞാന്‍ നിങ്ങളെ അറിയിക്കട്ടെ?
\end{malayalam}}
\flushright{\begin{Arabic}
\quranayah[61][11]
\end{Arabic}}
\flushleft{\begin{malayalam}
നിങ്ങള്‍ അല്ലാഹുവിലും അവന്റെ ദൂതനിലും വിശ്വസിക്കലാണത്. നിങ്ങളുടെ സമ്പത്തും ശരീരവുമുപയോഗിച്ച് അല്ലാഹുവിന്റെ മാര്‍ഗത്തില്‍ സമരം ചെയ്യലും. അതാണ് നിങ്ങള്‍ക്ക് ഏറ്റവും ഉത്തമം. നിങ്ങള്‍ അറിയുന്നവരെങ്കില്‍.
\end{malayalam}}
\flushright{\begin{Arabic}
\quranayah[61][12]
\end{Arabic}}
\flushleft{\begin{malayalam}
എങ്കില്‍ അല്ലാഹു നിങ്ങളുടെ പാപങ്ങള്‍ നിങ്ങള്‍ക്കു പൊറുത്തുതരും. താഴ്ഭാഗത്തിലൂടെ ആറുകളൊഴുകുന്ന സ്വര്‍ഗീയാരാമങ്ങളില്‍ നിങ്ങളെ പ്രവേശിപ്പിക്കും. സ്ഥിരജീവിതത്തിനായൊരുക്കിയ സ്വര്‍ഗീയാരാമങ്ങളിലെ വിശിഷ്ടമായ വാസസ്ഥലങ്ങളില്‍ അവന്‍ നിങ്ങളെ പ്രവേശിപ്പിക്കും. ഇതത്രെ അതിമഹത്തായ വിജയം.
\end{malayalam}}
\flushright{\begin{Arabic}
\quranayah[61][13]
\end{Arabic}}
\flushleft{\begin{malayalam}
നിങ്ങളഭിലഷിക്കുന്ന മറ്റൊരനുഗ്രഹവും നിങ്ങള്‍ക്ക് അവന്‍ നല്‍കും. അല്ലാഹുവില്‍നിന്നുള്ള സഹായവും ആസന്നവിജയവുമാണത്. ഈ ശുഭവാര്‍ത്ത സത്യവിശ്വാസികളെ അറിയിക്കുക.
\end{malayalam}}
\flushright{\begin{Arabic}
\quranayah[61][14]
\end{Arabic}}
\flushleft{\begin{malayalam}
വിശ്വസിച്ചവരേ, നിങ്ങള്‍ അല്ലാഹുവിന്റെ സഹായികളാവുക? മര്‍യമിന്റെ മകന്‍ ഈസാ ഹവാരികളോട് ചോദിച്ചപോലെ: "ദൈവമാര്‍ഗത്തില്‍ എന്നെ സഹായിക്കാനാരുണ്ട്?” ഹവാരിക 1ള്‍ പറഞ്ഞു: "ഞങ്ങളുണ്ട് അല്ലാഹുവിന്റെ മാര്‍ഗത്തില്‍ സഹായികളായി.” അങ്ങനെ ഇസ്രായേല്‍ മക്കളില്‍ ഒരുവിഭാഗം വിശ്വസിക്കുകയും മറ്റൊരു വിഭാഗം അവിശ്വസിക്കുകയും ചെയ്തു. പിന്നെ, വിശ്വസിച്ചവര്‍ക്കു നാം അവരുടെ ശത്രുക്കളെ തുരത്താനുള്ള കരുത്ത് നല്‍കി. അങ്ങനെ അവര്‍ വിജയികളാവുകയും ചെയ്തു.
\end{malayalam}}
\chapter{\textmalayalam{ജുമുഅ}}
\begin{Arabic}
\Huge{\centerline{\basmalah}}\end{Arabic}
\flushright{\begin{Arabic}
\quranayah[62][1]
\end{Arabic}}
\flushleft{\begin{malayalam}
ആകാശഭൂമികളിലുള്ളവയൊക്കെയും അല്ലാഹുവെ കീര്‍ത്തിച്ചുകൊണ്ടിരിക്കുന്നു. അവന്‍ രാജാധിരാജനാണ്. പരമപരിശുദ്ധനാണ്. പ്രതാപിയാണ്. യുക്തിജ്ഞനും.
\end{malayalam}}
\flushright{\begin{Arabic}
\quranayah[62][2]
\end{Arabic}}
\flushleft{\begin{malayalam}
നിരക്ഷരര്‍ക്കിടയില്‍ അവരില്‍നിന്നു തന്നെ ദൂതനെ നിയോഗിച്ചത് അവനാണ്. അദ്ദേഹം അവര്‍ക്ക് അല്ലാഹുവിന്റെ സൂക്തങ്ങള്‍ ഓതിക്കേള്‍പ്പിക്കുന്നു. അവരെ സംസ്കരിക്കുകയും അവര്‍ക്ക് വേദവും തത്വജ്ഞാനവും അഭ്യസിപ്പിക്കുകയും ചെയ്യുന്നു. നേരത്തെ അവര്‍ വ്യക്തമായ വഴികേടിലായിരുന്നു.
\end{malayalam}}
\flushright{\begin{Arabic}
\quranayah[62][3]
\end{Arabic}}
\flushleft{\begin{malayalam}
ഇനിയും അവരോടൊപ്പം വന്നുചേര്‍ന്നിട്ടില്ലാത്ത മറ്റുള്ളവരിലേക്കു കൂടി നിയോഗിക്കപ്പെട്ടവനാണ് അദ്ദേഹം. അല്ലാഹു പ്രതാപിയും യുക്തിജ്ഞനുമല്ലോ.
\end{malayalam}}
\flushright{\begin{Arabic}
\quranayah[62][4]
\end{Arabic}}
\flushleft{\begin{malayalam}
പ്രവാചകത്വം അല്ലാഹുവിന്റെ അനുഗ്രഹമാണ്. അവനാഗ്രഹിക്കുന്നവര്‍ക്ക് അവനത് നല്‍കുന്നു. അതിമഹത്തായ അനുഗ്രഹത്തിനുടമയാണ് അല്ലാഹു.
\end{malayalam}}
\flushright{\begin{Arabic}
\quranayah[62][5]
\end{Arabic}}
\flushleft{\begin{malayalam}
തൌറാത്തിന്റെ വാഹകരാക്കുകയും എന്നിട്ടത് വഹിക്കാതിരിക്കുകയും ചെയ്തവരുടെ ഉപമയിതാ: ഗ്രന്ഥക്കെട്ടുകള്‍ പേറുന്ന കഴുതയെപ്പോലെയാണവര്‍. അല്ലാഹുവിന്റെ സൂക്തങ്ങളെ നിഷേധിച്ചു തള്ളിയവരുടെ ഉപമ വളരെ നീചം തന്നെ. ഇത്തരം അക്രമികളായ ജനത്തെ അല്ലാഹു നേര്‍വഴിയിലാക്കുകയില്ല.
\end{malayalam}}
\flushright{\begin{Arabic}
\quranayah[62][6]
\end{Arabic}}
\flushleft{\begin{malayalam}
പറയുക: ജൂതന്മാരായവരേ, മറ്റു മനുഷ്യരെയൊക്കെ മാറ്റിനിര്‍ത്തി, നിങ്ങള്‍ മാത്രമാണ് ദൈവത്തിന്റെ അടുത്ത ആള്‍ക്കാരെന്ന് വാദിക്കുന്നുവെങ്കില്‍ മരണം കൊതിക്കുക. നിങ്ങള്‍ സത്യവാദികളെങ്കില്‍!
\end{malayalam}}
\flushright{\begin{Arabic}
\quranayah[62][7]
\end{Arabic}}
\flushleft{\begin{malayalam}
എന്നാല്‍ അവരൊരിക്കലും അത് കൊതിക്കുന്നില്ല. അവരുടെ കരങ്ങള്‍ നേരത്തെ ചെയ്ത ദുഷ്കൃത്യങ്ങളാണതിനു കാരണം. അല്ലാഹു ഈ അക്രമികളെക്കുറിച്ച് നന്നായറിയുന്നവനാണ്.
\end{malayalam}}
\flushright{\begin{Arabic}
\quranayah[62][8]
\end{Arabic}}
\flushleft{\begin{malayalam}
പറയുക: ഏതൊരു മരണത്തില്‍ നിന്നാണോ നിങ്ങള്‍ ഓടിയകലാന്‍ ശ്രമിക്കുന്നത്; ഉറപ്പായും ആ മരണം നിങ്ങളെ പിടികൂടുക തന്നെ ചെയ്യും. പിന്നെ അകവും പുറവും നന്നായറിയുന്നവന്റെ മുന്നിലേക്ക് നിങ്ങള്‍ മടക്കപ്പെടും. നിങ്ങള്‍ പ്രവര്‍ത്തിച്ചുകൊണ്ടിരുന്നതിനെപ്പറ്റിയെല്ലാം അപ്പോള്‍ അവന്‍ നിങ്ങളെ വിശദമായി വിവരമറിയിക്കും.
\end{malayalam}}
\flushright{\begin{Arabic}
\quranayah[62][9]
\end{Arabic}}
\flushleft{\begin{malayalam}
വിശ്വസിച്ചവരേ, വെള്ളിയാഴ്ച ദിവസം നമസ്കാരത്തിന് വിളിക്കപ്പെട്ടാല്‍ ദൈവസ്മരണയിലേക്ക് തിടുക്കത്തോടെ ചെന്നെത്തുക. കച്ചവട കാര്യങ്ങളൊക്കെ മാറ്റിവെക്കുക. അതാണ് നിങ്ങള്‍ക്ക് ഉത്തമം. നിങ്ങള്‍ അറിയുന്നവരെങ്കില്‍!
\end{malayalam}}
\flushright{\begin{Arabic}
\quranayah[62][10]
\end{Arabic}}
\flushleft{\begin{malayalam}
പിന്നെ നമസ്കാരത്തില്‍നിന്നു വിരമിച്ചു കഴിഞ്ഞാല്‍ ഭൂമിയില്‍ പരക്കുക. അല്ലാഹുവിന്റെ അനുഗ്രഹം തേടുകയും അല്ലാഹുവെ ധാരാളമായി സ്മരിക്കുകയും ചെയ്യുക. നിങ്ങള്‍ വിജയം വരിച്ചേക്കാം.
\end{malayalam}}
\flushright{\begin{Arabic}
\quranayah[62][11]
\end{Arabic}}
\flushleft{\begin{malayalam}
വല്ല വ്യാപാര കാര്യമോ വിനോദവൃത്തിയോ കണ്ടാല്‍ നിന്നെ 1 നിന്ന നില്പില്‍ വിട്ടു അവര്‍ 2 അങ്ങോട്ട് തിരിയുന്നുവല്ലോ. പറയുക: അല്ലാഹുവിന്റെ പക്കലുള്ളത് വിനോദത്തെക്കാളും വ്യാപാരത്തെക്കാളും വിശിഷ്ടമാകുന്നു. വിഭവദാതാക്കളില്‍ അത്യുത്തമന്‍ അല്ലാഹു തന്നെ.
\end{malayalam}}
\chapter{\textmalayalam{മുനാഫിഖൂന്‍ ( കപടവിശ്വാസികള്‍ )}}
\begin{Arabic}
\Huge{\centerline{\basmalah}}\end{Arabic}
\flushright{\begin{Arabic}
\quranayah[63][1]
\end{Arabic}}
\flushleft{\begin{malayalam}
കപടവിശ്വാസികള്‍ നിന്റെ അടുത്തുവന്നാല്‍ അവര്‍ പറയും: "തീര്‍ച്ചയായും അങ്ങ് അല്ലാഹുവിന്റെ ദൂതനാണെന്ന് ഞങ്ങള്‍ സാക്ഷ്യം വഹിക്കുന്നു.” അല്ലാഹുവിന്നറിയാം, നിശ്ചയമായും നീ അവന്റെ ദൂതനാണെന്ന്. കപടവിശ്വാസികള്‍ കള്ളം പറയുന്നവരാണെന്ന് അല്ലാഹുവും സാക്ഷ്യം വഹിക്കുന്നു.
\end{malayalam}}
\flushright{\begin{Arabic}
\quranayah[63][2]
\end{Arabic}}
\flushleft{\begin{malayalam}
അവര്‍ തങ്ങളുടെ ശപഥങ്ങളെ പരിചയാക്കുകയാണ്. 1 അങ്ങനെ അവര്‍ അല്ലാഹുവിന്റെ മാര്‍ഗത്തില്‍നിന്ന് ജനത്തെ തടയുന്നു. അവര്‍ ചെയ്തുകൊണ്ടിരിക്കുന്നത് വളരെ നീചം തന്നെ.
\end{malayalam}}
\flushright{\begin{Arabic}
\quranayah[63][3]
\end{Arabic}}
\flushleft{\begin{malayalam}
അവര്‍ ആദ്യം വിശ്വസിക്കുകയും പിന്നെ അവിശ്വസിക്കുകയും ചെയ്തതിന്റെ ഫലമാണത്. അങ്ങനെ അവരുടെ ഹൃദയങ്ങള്‍ മുദ്രവെക്കപ്പെട്ടിരിക്കുന്നു. അതിനാല്‍ അവര്‍ക്കൊന്നും തിരിച്ചറിയാനാവുന്നില്ല.
\end{malayalam}}
\flushright{\begin{Arabic}
\quranayah[63][4]
\end{Arabic}}
\flushleft{\begin{malayalam}
നീ അവരെ കണ്ടാല്‍ അവരുടെ ആകാരം നിന്നെ വിസ്മയഭരിതനാക്കും. അവര്‍ സംസാരിച്ചാലോ അവരുടെ വാക്കുകള്‍ നീ കേട്ടിരുന്നുപോകും. ചാരിവെച്ച മരത്തടികള്‍ പോലെയാണ് അവര്‍. എല്ലാ ഒച്ചയും തങ്ങള്‍ക്കെതിരാണെന്ന് അവര്‍ കരുതുന്നു. അവര്‍ തന്നെയാണ് ശത്രു. അവരെ സൂക്ഷിക്കുക. അല്ലാഹു അവരെ തുലക്കട്ടെ. എവിടേക്കാണവര്‍ വഴിതെറ്റിപ്പോകുന്നത്?
\end{malayalam}}
\flushright{\begin{Arabic}
\quranayah[63][5]
\end{Arabic}}
\flushleft{\begin{malayalam}
“വരിക, ദൈവദൂതന്‍ നിങ്ങളുടെ പാപമോചനത്തിനായി പ്രാര്‍ഥിച്ചുകൊള്ളു”മെന്ന് പറഞ്ഞാല്‍ അവര്‍ തങ്ങളുടെ തല തിരിച്ചുകളയും. അഹങ്കാരപൂര്‍വം അവര്‍ വരാന്‍ വിസമ്മതിക്കുന്നതായി നിനക്കു കാണാം.
\end{malayalam}}
\flushright{\begin{Arabic}
\quranayah[63][6]
\end{Arabic}}
\flushleft{\begin{malayalam}
നീ അവരുടെ പാപമോചനത്തിന് പ്രാര്‍ഥിക്കുന്നതും പ്രാര്‍ഥിക്കാതിരിക്കുന്നതും അവരെ സംബന്ധിച്ചേടത്തോളം സമമാണ്. അല്ലാഹു അവര്‍ക്ക് പൊറുത്തുകൊടുക്കുകയില്ല. അധര്‍മികളായ ജനത്തെ അല്ലാഹു നേര്‍വഴിയിലാക്കുകയില്ല; തീര്‍ച്ച.
\end{malayalam}}
\flushright{\begin{Arabic}
\quranayah[63][7]
\end{Arabic}}
\flushleft{\begin{malayalam}
ദൈവദൂതന്റെ കൂടെയുള്ളവര്‍ക്ക്, അവരദ്ദേഹത്തെ വിട്ടുപിരിയും വരെ, നിങ്ങളൊന്നും ചെലവഴിക്കരുത് എന്ന് പറയുന്നവരാണല്ലോ അവര്‍. എന്നാല്‍ ആകാശഭൂമികളുടെ ഖജനാവുകള്‍ അല്ലാഹുവിന്റേതാണ്. പക്ഷേ, കപട വിശ്വാസികള്‍ ഇത് മനസ്സിലാക്കുന്നില്ല.
\end{malayalam}}
\flushright{\begin{Arabic}
\quranayah[63][8]
\end{Arabic}}
\flushleft{\begin{malayalam}
അവര്‍ പറയുന്നു: "ഞങ്ങള്‍ മദീനയില്‍ തിരിച്ചെത്തിയാല്‍ അവിടെ നിന്ന് പ്രതാപികള്‍ പതിതരെ പുറംതള്ളുകതന്നെ ചെയ്യും.” എന്നാല്‍ പ്രതാപമൊക്കെയും അല്ലാഹുവിനും അവന്റെ ദൂതന്നും സത്യവിശ്വാസികള്‍ക്കുമാണ്. പക്ഷേ, കപടവിശ്വാസികള്‍ അതറിയുന്നില്ല.
\end{malayalam}}
\flushright{\begin{Arabic}
\quranayah[63][9]
\end{Arabic}}
\flushleft{\begin{malayalam}
വിശ്വസിച്ചവരേ, നിങ്ങളുടെ സമ്പത്തും സന്താനങ്ങളും ദൈവചിന്തയില്‍നിന്ന് നിങ്ങളെ അശ്രദ്ധരാക്കാതിരിക്കട്ടെ. ആര്‍ അങ്ങനെ ചെയ്യുന്നുവോ, അവരത്രെ നഷ്ടം പറ്റിയവര്‍.
\end{malayalam}}
\flushright{\begin{Arabic}
\quranayah[63][10]
\end{Arabic}}
\flushleft{\begin{malayalam}
മരണം വന്നെത്തും മുമ്പേ നിങ്ങളോരോരുത്തരും നാം നല്‍കിയ വിഭവങ്ങളില്‍നിന്ന് ചെലവഴിക്കുക. അപ്പോഴ 2വന്‍ പറയും: എന്റെ നാഥാ, അടുത്ത ഒരവധി വരെ എനിക്ക് സമയം നീട്ടിത്തരാത്തതെന്ത്? എങ്കില്‍ ഞാന്‍ ദാനം നല്‍കാം; സജ്ജനങ്ങളിലുള്‍പ്പെട്ടവനാകാം.
\end{malayalam}}
\flushright{\begin{Arabic}
\quranayah[63][11]
\end{Arabic}}
\flushleft{\begin{malayalam}
അവധി ആസന്നമായാല്‍ പിന്നെ അല്ലാഹു ആര്‍ക്കും അത് നീട്ടിക്കൊടുക്കുകയില്ല. നിങ്ങള്‍ ചെയ്യുന്നതൊക്കെയും നന്നായറിയുന്നവനാണ് അല്ലാഹു.
\end{malayalam}}
\chapter{\textmalayalam{തഗാബൂന്‍ ( നഷ്ടം വെളിപ്പെടല്‍ )}}
\begin{Arabic}
\Huge{\centerline{\basmalah}}\end{Arabic}
\flushright{\begin{Arabic}
\quranayah[64][1]
\end{Arabic}}
\flushleft{\begin{malayalam}
ആകാശ ഭൂമികളിലുള്ളവയൊക്കെയും അല്ലാഹുവിനെ കീര്‍ത്തിച്ചുകൊണ്ടിരിക്കുന്നു. അവന്നാണ് ആധിപത്യം. അവന്നാണ് സര്‍വസ്തുതിയും. അവന്‍ എല്ലാറ്റിനും കഴിവുറ്റവന്‍.
\end{malayalam}}
\flushright{\begin{Arabic}
\quranayah[64][2]
\end{Arabic}}
\flushleft{\begin{malayalam}
അവനാണ് നിങ്ങളെ സൃഷ്ടിച്ചവന്‍. നിങ്ങളില്‍ സത്യനിഷേധികളുണ്ട്. സത്യവിശ്വാസികളുമുണ്ട്. നിങ്ങള്‍ ചെയ്യുന്നതൊക്കെയും അല്ലാഹു കണ്ടുകൊണ്ടേയിരിക്കുന്നു.
\end{malayalam}}
\flushright{\begin{Arabic}
\quranayah[64][3]
\end{Arabic}}
\flushleft{\begin{malayalam}
ആകാശഭൂമികളെ അവന്‍ യാഥാര്‍ഥ്യനിഷ്ഠയോടെ സൃഷ്ടിച്ചു. നിങ്ങള്‍ക്ക് അവന്‍ രൂപമേകി. നിങ്ങളുടെ രൂപം അവന്‍ ആകര്‍ഷകമാക്കുകയും ചെയ്തു. നിങ്ങളുടെ തിരിച്ചുപോക്ക് അവങ്കലേക്കാണ്.
\end{malayalam}}
\flushright{\begin{Arabic}
\quranayah[64][4]
\end{Arabic}}
\flushleft{\begin{malayalam}
ആകാശഭൂമികളിലുള്ളതൊക്കെയും അവനറിയുന്നു. നിങ്ങള്‍ ഒളിച്ചുവെക്കുന്നതും വെളിപ്പെടുത്തുന്നതുമെല്ലാം അവനറിയുന്നു. മനസ്സിലുള്ളതുപോലും അറിയുന്നവനാണ് അല്ലാഹു.
\end{malayalam}}
\flushright{\begin{Arabic}
\quranayah[64][5]
\end{Arabic}}
\flushleft{\begin{malayalam}
മുമ്പ് സത്യനിഷേധികളാവുകയും അങ്ങനെ തങ്ങളുടെ ദുര്‍വൃത്തികളുടെ കെടുതി അനുഭവിക്കുകയും ചെയ്തവരുടെ വിവരം നിങ്ങള്‍ക്ക് വന്നെത്തിയിട്ടില്ലേ? ഇനിയവര്‍ക്ക് നോവേറിയ ശിക്ഷയുമുണ്ട്.
\end{malayalam}}
\flushright{\begin{Arabic}
\quranayah[64][6]
\end{Arabic}}
\flushleft{\begin{malayalam}
അതെന്തുകൊണ്ടെന്നാല്‍ അവര്‍ക്കുള്ള ദൈവദൂതന്മാര്‍ വ്യക്തമായ തെളിവുകളുമായി അവരുടെ അടുത്ത് വന്നുകൊണ്ടിരുന്നു. അപ്പോഴൊക്കെ അവര്‍ പറഞ്ഞു: "കേവലം ഒരു മനുഷ്യന്‍ ഞങ്ങളെ വഴികാട്ടുകയോ?” അങ്ങനെ അവര്‍ അവിശ്വസിച്ചു. പിന്തിരിയുകയും ചെയ്തു. അല്ലാഹുവിന് അവരുടെ പിന്തുണ ആവശ്യമുണ്ടായിരുന്നില്ല. അല്ലാഹു ആശ്രയമാവശ്യമില്ലാത്തവനും സ്തുത്യര്‍ഹനുമാണ്.
\end{malayalam}}
\flushright{\begin{Arabic}
\quranayah[64][7]
\end{Arabic}}
\flushleft{\begin{malayalam}
സത്യനിഷേധികള്‍ വാദിച്ചു, തങ്ങളൊരിക്കലും ഉയിര്‍ത്തെഴുന്നേല്‍പിക്കപ്പെടുകയില്ലെന്ന്. പറയുക: എന്റെ നാഥന്‍ സാക്ഷി! നിങ്ങള്‍ ഉയിര്‍ത്തെഴുന്നേല്‍പിക്കപ്പെടുക തന്നെചെയ്യും. പിന്നീട് നിങ്ങള്‍ പ്രവര്‍ത്തിച്ചുകൊണ്ടിരുന്നതിനെപ്പറ്റി നിങ്ങളെ വിവരമറിയിക്കും; തീര്‍ച്ച. അത് അല്ലാഹുവിന് നന്നേ എളുപ്പമാണ്.
\end{malayalam}}
\flushright{\begin{Arabic}
\quranayah[64][8]
\end{Arabic}}
\flushleft{\begin{malayalam}
അതിനാല്‍ നിങ്ങള്‍ അല്ലാഹുവിലും അവന്റെ ദൂതനിലും നാം ഇറക്കിത്തന്ന വെളിച്ചത്തിലും വിശ്വസിക്കുക. നിങ്ങള്‍ ചെയ്തുകൊണ്ടിരിക്കുന്നതൊക്കെയും നന്നായറിയുന്നവനാണ് അല്ലാഹു.
\end{malayalam}}
\flushright{\begin{Arabic}
\quranayah[64][9]
\end{Arabic}}
\flushleft{\begin{malayalam}
ആ മഹാസംഗമ ദിവസം അല്ലാഹു നിങ്ങളെയെല്ലാം ഒരുമിച്ചുകൂട്ടുമ്പോള്‍; ഓര്‍ക്കുക, അതത്രെ ലാഭചേതങ്ങളുടെ ദിവസം. അല്ലാഹുവില്‍ വിശ്വസിക്കുകയും സല്‍ക്കര്‍മങ്ങളനുഷ്ഠിക്കുകയും ചെയ്തവന്റെ പാപങ്ങള്‍ അല്ലാഹു മായ്ച്ചു കളയും. താഴ്ഭാഗത്തൂടെ ആറുകളൊഴുകുന്ന സ്വര്‍ഗീയാരാമങ്ങളില്‍ അവരെ പ്രവേശിപ്പിക്കുകയും ചെയ്യും. അവരതില്‍ നിത്യവാസികളായിരിക്കും. അതാണ് അതിമഹത്തായ വിജയം.
\end{malayalam}}
\flushright{\begin{Arabic}
\quranayah[64][10]
\end{Arabic}}
\flushleft{\begin{malayalam}
സത്യത്തെ തള്ളിപ്പറയുകയും നമ്മുടെ സൂക്തങ്ങളെ കളവാക്കുകയും ചെയ്തവരോ; അവരാണ് നരകാവകാശികള്‍. അവരതില്‍ നിത്യവാസികളായിരിക്കും. എത്ര ചീത്ത സങ്കേതം!
\end{malayalam}}
\flushright{\begin{Arabic}
\quranayah[64][11]
\end{Arabic}}
\flushleft{\begin{malayalam}
അല്ലാഹുവിന്റെ അനുമതിയില്ലാതെ ആര്‍ക്കും ഒരാപത്തും സംഭവിക്കുന്നില്ല. അല്ലാഹുവില്‍ വിശ്വസിക്കുന്നവനാരോ, അവന്റെ മനസ്സിനെ അവന്‍ നേര്‍വഴിയിലാക്കുന്നു. അല്ലാഹു എല്ലാം അറിയുന്നവനാണ്.
\end{malayalam}}
\flushright{\begin{Arabic}
\quranayah[64][12]
\end{Arabic}}
\flushleft{\begin{malayalam}
നിങ്ങള്‍ അല്ലാഹുവിനെ അനുസരിക്കുക. പ്രവാചകനെയും അനുസരിക്കുക. അഥവാ, നിങ്ങള്‍ പിന്തിരിയുന്നുവെങ്കില്‍, സത്യസന്ദേശമെത്തിച്ചുതരിക എന്നതല്ലാതെ നമ്മുടെ ദൂതന് മറ്റു ബാധ്യതകളൊന്നുമില്ല.
\end{malayalam}}
\flushright{\begin{Arabic}
\quranayah[64][13]
\end{Arabic}}
\flushleft{\begin{malayalam}
അല്ലാഹു. അവനല്ലാതെ ദൈവമില്ല. അതിനാല്‍ സത്യവിശ്വാസികള്‍ അല്ലാഹുവില്‍ മാത്രം ഭരമേല്‍പിക്കട്ടെ.
\end{malayalam}}
\flushright{\begin{Arabic}
\quranayah[64][14]
\end{Arabic}}
\flushleft{\begin{malayalam}
വിശ്വസിച്ചവരേ, നിങ്ങളുടെ ഭാര്യമാരിലും മക്കളിലും നിങ്ങള്‍ക്ക് ശത്രുക്കളുണ്ട്. അതിനാല്‍ അവരെ സൂക്ഷിക്കുക. എന്നാല്‍ നിങ്ങള്‍ മാപ്പ് നല്‍കുകയും വിട്ടുവീഴ്ച കാണിക്കുകയും പൊറുത്തു കൊടുക്കുകയുമാണെങ്കില്‍, തീര്‍ച്ചയായും അറിയുക: അല്ലാഹു ഏറെ പൊറുക്കുന്നവനും ദയാപരനുമാണ്.
\end{malayalam}}
\flushright{\begin{Arabic}
\quranayah[64][15]
\end{Arabic}}
\flushleft{\begin{malayalam}
നിങ്ങളുടെ സമ്പത്തും സന്താനങ്ങളും ഒരു പരീക്ഷണം മാത്രമാണ്. അല്ലാഹുവിങ്കലത്രെ അതിമഹത്തായ പ്രതിഫലം.
\end{malayalam}}
\flushright{\begin{Arabic}
\quranayah[64][16]
\end{Arabic}}
\flushleft{\begin{malayalam}
അതിനാല്‍ ആവുന്നത്ര നിങ്ങള്‍ അല്ലാഹുവോട് ഭക്തിയുള്ളവരാവുക. കേള്‍ക്കുകയും അനുസരിക്കുകയും ചെയ്യുക. ധനം ചെലവു ചെയ്യുക. അതു നിങ്ങള്‍ക്കുതന്നെ ഗുണകരമായിരിക്കും. മനസ്സിന്റെ പിശുക്കില്‍നിന്ന് വിടുതി നേടുന്നവരാരോ അവരാകുന്നു വിജയികള്‍.
\end{malayalam}}
\flushright{\begin{Arabic}
\quranayah[64][17]
\end{Arabic}}
\flushleft{\begin{malayalam}
നിങ്ങള്‍ അല്ലാഹുവിന് ഉത്തമമായ കടം നല്‍കുകയാണെങ്കില്‍ അവന്‍ നിങ്ങള്‍ക്കത് ഇരട്ടിയായി മടക്കിത്തരും. നിങ്ങളുടെ പാപങ്ങള്‍ പൊറുക്കും. അല്ലാഹു ഏറെ നന്ദി കാണിക്കുന്നവനും ക്ഷമാലുവുമാകുന്നു.
\end{malayalam}}
\flushright{\begin{Arabic}
\quranayah[64][18]
\end{Arabic}}
\flushleft{\begin{malayalam}
തെളിഞ്ഞതും ഒളിഞ്ഞതും അറിയുന്നവനാണവന്‍. പ്രതാപിയും യുക്തിജ്ഞനും.
\end{malayalam}}
\chapter{\textmalayalam{ത്വലാഖ് ( വിവാഹ മോചനം )}}
\begin{Arabic}
\Huge{\centerline{\basmalah}}\end{Arabic}
\flushright{\begin{Arabic}
\quranayah[65][1]
\end{Arabic}}
\flushleft{\begin{malayalam}
നബിയേ, നിങ്ങള്‍ സ്ത്രീകളെ വിവാഹമോചനം ചെയ്യുകയാണെങ്കില്‍ അവര്‍ക്ക് ഇദ്ദഃ തുടങ്ങാനുള്ള അവസരത്തില്‍ വിവാഹമോചനം നടത്തുക. ഇദ്ദ കാലം നിങ്ങള്‍ കൃത്യമായി കണക്കാക്കുക. നിങ്ങളുടെ നാഥനായ അല്ലാഹുവെ സൂക്ഷിക്കുക. ഇദ്ദാ വേളയില്‍ അവരെ അവരുടെ വീടുകളില്‍നിന്ന് പുറംതള്ളരുത്. അവര്‍ സ്വയം ഇറങ്ങിപ്പോവുകയുമരുത്. അവര്‍ വ്യക്തമായ ദുര്‍വൃത്തിയിലേര്‍പ്പെട്ടാലല്ലാതെ. അല്ലാഹുവിന്റെ നിയമപരിധികളാണിവ. അല്ലാഹുവിന്റെ പരിധികള്‍ ലംഘിക്കുന്നവന്‍ തന്നോടു തന്നെയാണ് അക്രമം ചെയ്യുന്നത്. അതിനുശേഷം അല്ലാഹു വല്ല പുതിയ കാര്യവും ചെയ്തേക്കാം; നിനക്കത് അറിയില്ല.
\end{malayalam}}
\flushright{\begin{Arabic}
\quranayah[65][2]
\end{Arabic}}
\flushleft{\begin{malayalam}
അവരുടെ ഇദ്ദാ കാലാവധി എത്തിയാല്‍ നല്ല നിലയില്‍ അവരെ കൂടെ നിര്‍ത്തുക. അല്ലെങ്കില്‍ മാന്യമായ നിലയില്‍ അവരുമായി വേര്‍പിരിയുക. നിങ്ങളില്‍ നീതിമാന്‍മാരായ രണ്ടുപേരെ അതിനു സാക്ഷികളാക്കുക. "സാക്ഷികളേ, നിങ്ങള്‍ അല്ലാഹുവിന് വേണ്ടി നേരാംവിധം സാക്ഷ്യം വഹിക്കുക.” അല്ലാഹുവിലും അന്ത്യദിനത്തിലും വിശ്വസിക്കുന്നവര്‍ക്കു നല്‍കപ്പെടുന്ന ഉപദേശമാണിത്. അല്ലാഹുവോട് ഭക്തി കാണിക്കുന്നവന്ന് അല്ലാഹു രക്ഷാമാര്‍ഗമൊരുക്കിക്കൊടുക്കും.
\end{malayalam}}
\flushright{\begin{Arabic}
\quranayah[65][3]
\end{Arabic}}
\flushleft{\begin{malayalam}
അവന്‍ വിചാരിക്കാത്ത വിധം അവന് ആഹാരം നല്‍കും. എല്ലാം അല്ലാഹുവില്‍ അര്‍പ്പിക്കുന്നവന് അല്ലാഹു തന്നെ മതി. അല്ലാഹു അവന്റെ കാര്യം നിറവേറ്റുക തന്നെ ചെയ്യും. അല്ലാഹു എല്ലാ കാര്യങ്ങള്‍ക്കും ഒരു ക്രമം നിശ്ചയിച്ചിട്ടുണ്ട്.
\end{malayalam}}
\flushright{\begin{Arabic}
\quranayah[65][4]
\end{Arabic}}
\flushleft{\begin{malayalam}
നിങ്ങളുടെ സ്ത്രീകളില്‍ ആര്‍ത്തവം നിലച്ചവരുടെ ഇദ്ദാ കാര്യത്തില്‍ നിങ്ങള്‍ക്ക് സംശയമുണ്ടെങ്കില്‍ അറിയുക: അവരുടെ ഇദ്ദാകാലം മൂന്നു മാസമാണ്. ഋതുമതികളായിട്ടില്ലാത്തവരുടേതും ഇതുതന്നെ. ഗര്‍ഭിണികളുടെ കാലാവധി അവര്‍ പ്രസവിക്കുന്നതുവരെയാകുന്നു. ആര്‍ അല്ലാഹുവോട് ഭക്തിയുള്ളവരാകുന്നുവോ അവന്റെ കാര്യം അല്ലാഹു എളുപ്പമാക്കും.
\end{malayalam}}
\flushright{\begin{Arabic}
\quranayah[65][5]
\end{Arabic}}
\flushleft{\begin{malayalam}
ഇത് നിങ്ങള്‍ക്കായി അല്ലാഹു അവതരിപ്പിച്ച കല്‍പനയാണ്. അല്ലാഹുവോട് ഭക്തി കാണിക്കുന്നവന്റെ പാപങ്ങള്‍ അല്ലാഹു മായിച്ചുകളയുകയും അവന്റെ പ്രതിഫലം അവന്ന് മെച്ചപ്പെടുത്തിക്കൊടുക്കുകയും ചെയ്യും.
\end{malayalam}}
\flushright{\begin{Arabic}
\quranayah[65][6]
\end{Arabic}}
\flushleft{\begin{malayalam}
നിങ്ങളുടെ കഴിവിനൊത്തവിധം ഇദ്ദാവേളയില്‍ അവരെ നിങ്ങള്‍ താമസിക്കുന്നിടത്ത് തന്നെ താമസിപ്പിക്കുക. അവര്‍ക്ക് ഇടുക്കമുണ്ടാക്കുംവിധം നിങ്ങളവരെ പ്രയാസപ്പെടുത്തരുത്. അവര്‍ ഗര്‍ഭിണികളാണെങ്കില്‍ പ്രസവിക്കുന്നത് വരെ നിങ്ങളവര്‍ക്ക് ചെലവിന് കൊടുക്കുക. അവര്‍ നിങ്ങള്‍ക്കായി കുഞ്ഞുങ്ങളെ മുലയൂട്ടുന്നുവെങ്കില്‍ അവര്‍ക്ക് അതിനുള്ള പ്രതിഫലവും നല്‍കുക. അക്കാര്യം നിങ്ങള്‍ നല്ല നിലയില്‍ അന്യോന്യം കൂടിയാലോചിച്ച് തീരുമാനിക്കുക. എന്നാല്‍ നിങ്ങള്‍ക്കിരുവര്‍ക്കും അത് പ്രയാസകരമാവുകയാണെങ്കില്‍ അയാള്‍ക്കുവേണ്ടി മറ്റൊരുവള്‍ മുലയൂട്ടട്ടെ.
\end{malayalam}}
\flushright{\begin{Arabic}
\quranayah[65][7]
\end{Arabic}}
\flushleft{\begin{malayalam}
സമ്പന്നന്‍ തന്റെ കഴിവിനനുസരിച്ചു ചെലവു ചെയ്യണം. തന്റെ ഉപജീവനത്തിന് ഇടുക്കമനുഭവിക്കുന്നവന്‍ അല്ലാഹു അവന് നല്‍കിയതില്‍ നിന്ന് ചെലവിനു നല്‍കട്ടെ. അല്ലാഹു ആരെയും അയാള്‍ക്കേകിയ കഴിവില്‍ കവിഞ്ഞതിന് നിര്‍ബന്ധിക്കുന്നില്ല. പ്രയാസത്തിനു ശേഷം അല്ലാഹു എളുപ്പം ഉണ്ടാക്കിക്കൊടുക്കുന്നു.
\end{malayalam}}
\flushright{\begin{Arabic}
\quranayah[65][8]
\end{Arabic}}
\flushleft{\begin{malayalam}
എത്രയോ നാടുകള്‍, അവയുടെ നാഥന്റെയും അവന്റെ ദൂതന്‍മാരുടെയും കല്‍പനകള്‍ നിരാകരിച്ച് ധിക്കാരം പ്രവര്‍ത്തിച്ചു. അപ്പോള്‍ നാം അവയെ കര്‍ക്കശമായ വിചാരണക്കു വിധേയമാക്കി. കൊടിയ ശിക്ഷ നല്‍കുകയും ചെയ്തു.
\end{malayalam}}
\flushright{\begin{Arabic}
\quranayah[65][9]
\end{Arabic}}
\flushleft{\begin{malayalam}
അങ്ങനെ അവ ആ ചെയ്തികളുടെ കെടുതികളനുഭവിച്ചു. കൊടിയ നഷ്ടമായിരുന്നു അതിന്റെ അന്ത്യം.
\end{malayalam}}
\flushright{\begin{Arabic}
\quranayah[65][10]
\end{Arabic}}
\flushleft{\begin{malayalam}
അല്ലാഹു അവര്‍ക്ക് കഠിന ശിക്ഷ ഒരുക്കിവെച്ചിട്ടുണ്ട്. അതിനാല്‍ സത്യവിശ്വാസം സ്വീകരിച്ച ബുദ്ധിമാന്മാരേ, നിങ്ങള്‍ അല്ലാഹുവോട് ഭക്തിയുള്ളവരാവുക. നിശ്ചയം, അല്ലാഹു നിങ്ങള്‍ക്കായി ഒരുദ്ബോധനം നല്‍കിയിരിക്കുകയാണ്.
\end{malayalam}}
\flushright{\begin{Arabic}
\quranayah[65][11]
\end{Arabic}}
\flushleft{\begin{malayalam}
അല്ലാഹുവിന്റെ സുവ്യക്തമായ സൂക്തങ്ങള്‍ ഓതിക്കേള്‍പ്പിക്കുന്ന ഒരു ദൂതനെ അവന്‍ നിയോഗിച്ചിരിക്കുന്നു. വിശ്വസിക്കുകയും സല്‍ക്കര്‍മം പ്രവര്‍ത്തിക്കുകയും ചെയ്തവരെ ഇരുളില്‍നിന്ന് വെളിച്ചത്തിലേക്ക് നയിക്കാനാണത്. അല്ലാഹുവില്‍ വിശ്വസിക്കുകയും സല്‍ക്കര്‍മങ്ങള്‍ പ്രവര്‍ത്തിക്കുകയും ചെയ്യുന്നവനെ, അല്ലാഹു താഴ്ഭാഗത്തൂടെ ആറുകളൊഴുകുന്ന സ്വര്‍ഗീയാരാമങ്ങളില്‍ പ്രവേശിപ്പിക്കും. അവനതില്‍ നിത്യവാസിയായിരിക്കും. അല്ലാഹു അവന് വിശിഷ്ട വിഭവങ്ങളാണ് നല്‍കുക.
\end{malayalam}}
\flushright{\begin{Arabic}
\quranayah[65][12]
\end{Arabic}}
\flushleft{\begin{malayalam}
ഏഴ് ആകാശങ്ങളെ സൃഷ്ടിച്ചവനാണ് അല്ലാഹു. ഭൂമിയില്‍നിന്നും അതുപോലെയുള്ളവയെ അവന്‍ സൃഷ്ടിച്ചു. അവയ്ക്കിടയില്‍ അവന്റെ കല്‍പനകളിറങ്ങിക്കൊണ്ടിരിക്കുന്നു. അല്ലാഹു എല്ലാറ്റിനും കഴിവുറ്റവനാണെന്നും അവന്റെ അറിവ് സകല സംഗതികളെയും ചൂഴ്ന്നുനില്‍ക്കുന്നുവെന്നും നിങ്ങള്‍ മനസ്സിലാക്കാനാണ് ഇവ്വിധം വിശദീകരിക്കുന്നത്.
\end{malayalam}}
\chapter{\textmalayalam{തഹ് രീം ( നിഷിദ്ധമാക്കല്‍ )}}
\begin{Arabic}
\Huge{\centerline{\basmalah}}\end{Arabic}
\flushright{\begin{Arabic}
\quranayah[66][1]
\end{Arabic}}
\flushleft{\begin{malayalam}
നബിയേ, നീയെന്തിനാണ് ഭാര്യമാരുടെ പ്രീതി കാംക്ഷിച്ച് അല്ലാഹു അനുവദനീയമാക്കിയത് നിഷിദ്ധമാക്കുന്നത്? അല്ലാഹു ഏറെ പൊറുക്കുന്നവനും ദയാപരനും തന്നെ.
\end{malayalam}}
\flushright{\begin{Arabic}
\quranayah[66][2]
\end{Arabic}}
\flushleft{\begin{malayalam}
നിങ്ങളുടെ ശപഥങ്ങള്‍ക്കുള്ള പരിഹാരം അല്ലാഹു നിങ്ങള്‍ക്കു നിശ്ചയിച്ചു തന്നിരിക്കുന്നു. അല്ലാഹുവാണ് നിങ്ങളുടെ രക്ഷകന്‍. സര്‍വജ്ഞനും യുക്തിമാനുമാണ് അവന്‍.
\end{malayalam}}
\flushright{\begin{Arabic}
\quranayah[66][3]
\end{Arabic}}
\flushleft{\begin{malayalam}
പ്രവാചകന്‍ തന്റെ ഭാര്യമാരിലൊരാളോട് ഒരു രഹസ്യവര്‍ത്തമാനം പറഞ്ഞു. അവരത് മറ്റൊരാളെ അറിയിച്ചു. രഹസ്യം പരസ്യമായ വിവരം അല്ലാഹു പ്രവാചകനെ ധരിപ്പിച്ചു. അപ്പോള്‍ അദ്ദേഹം അതിലെ ചില വശങ്ങള്‍ ആ ഭാര്യയെ അറിയിച്ചു. ചിലവശം ഒഴിവാക്കുകയും ചെയ്തു. ഇക്കാര്യം പ്രവാചകന്‍ അവരോട് പറഞ്ഞപ്പോള്‍ ആരാണിത് താങ്കളെ അറിയിച്ചതെന്ന് അവര്‍ ചോദിച്ചു. പ്രവാചകന്‍ പറഞ്ഞു: സര്‍വജ്ഞനും സൂക്ഷ്മജ്ഞനുമായവനാണ് എന്നെ വിവരമറിയിച്ചത്.
\end{malayalam}}
\flushright{\begin{Arabic}
\quranayah[66][4]
\end{Arabic}}
\flushleft{\begin{malayalam}
നിങ്ങളിരുവരും അല്ലാഹുവിലേക്ക് പശ്ചാത്തപിച്ചു മടങ്ങുന്നുവെങ്കില്‍ അതാണ് നിങ്ങള്‍ക്കുത്തമം. കാരണം, നിങ്ങളിരുവരുടെയും മനസ്സുകള്‍ വ്യതിചലിച്ചു പോയിട്ടുണ്ട്. അഥവാ നിങ്ങളിരുവരും അദ്ദേഹത്തിനെതിരെ പരസ്പരം സഹായിക്കുകയാണെങ്കില്‍ അറിയുക: അല്ലാഹുവാണ് അദ്ദേഹത്തിന്റെ രക്ഷകന്‍. പിന്നെ ജിബ്രീലും സച്ചരിതരായ മുഴുവന്‍ സത്യവിശ്വാസികളും മലക്കുകളുമെല്ലാം അദ്ദേഹത്തിന്റെ സഹായികളാണ്.
\end{malayalam}}
\flushright{\begin{Arabic}
\quranayah[66][5]
\end{Arabic}}
\flushleft{\begin{malayalam}
പ്രവാചകന്‍ നിങ്ങളെയൊക്കെ വിവാഹമോചനം ചെയ്യുന്നുവെങ്കില്‍ പകരം അല്ലാഹു അദ്ദേഹത്തിന് നിങ്ങളെക്കാള്‍ നല്ലവരായ ഭാര്യമാരെ നല്‍കിയേക്കാം; മുസ്ലിംകളും സത്യവിശ്വാസിനികളും ഭയഭക്തരും പശ്ചാത്തപിക്കുന്നവരും ആരാധനാ നിരതരും വ്രതനിഷ്ഠരും വിധവകളും കന്യകകളുമായ സ്ത്രീകളെ.
\end{malayalam}}
\flushright{\begin{Arabic}
\quranayah[66][6]
\end{Arabic}}
\flushleft{\begin{malayalam}
വിശ്വസിച്ചവരേ, നിങ്ങളെയും നിങ്ങളുടെ കുടുംബാംഗങ്ങളെയും നരകാഗ്നിയില്‍നിന്ന് കാത്തുരക്ഷിക്കുക. അതിന്റെ ഇന്ധനം മനുഷ്യരും കല്ലുകളുമാണ്. അതിന്റെ മേല്‍നോട്ടത്തിന് പരുഷപ്രകൃതരും ശക്തരുമായ മലക്കുകളാണുണ്ടാവുക. അല്ലാഹുവിന്റെ ആജ്ഞകളെ അവര്‍ അല്‍പംപോലും ലംഘിക്കുകയില്ല. അവരോട് ആജ്ഞാപിക്കുന്നതൊക്കെ അതേപടി പ്രാവര്‍ത്തികമാക്കുന്നതുമാണ്.
\end{malayalam}}
\flushright{\begin{Arabic}
\quranayah[66][7]
\end{Arabic}}
\flushleft{\begin{malayalam}
സത്യനിഷേധികളേ, നിങ്ങളിന്ന് ഒഴികഴിവൊന്നും പറയാന്‍ നോക്കേണ്ട. നിങ്ങള്‍ ചെയ്തുകൊണ്ടിരുന്നതിന്റെ പ്രതിഫലം മാത്രമാണ് നിങ്ങള്‍ക്കിപ്പോള്‍ നല്‍കുന്നത്.
\end{malayalam}}
\flushright{\begin{Arabic}
\quranayah[66][8]
\end{Arabic}}
\flushleft{\begin{malayalam}
വിശ്വസിച്ചവരേ, നിങ്ങള്‍ അല്ലാഹുവോട് പശ്ചാത്തപിക്കുക. ആത്മാര്‍ഥമായ പശ്ചാത്താപം. നിങ്ങളുടെ നാഥന്‍ നിങ്ങളുടെ തിന്മകള്‍ മായിച്ചുകളയുകയും താഴ്ഭാഗത്തൂടെ ആറുകളൊഴുകുന്ന സ്വര്‍ഗീയാരാമങ്ങളില്‍ നിങ്ങളെ പ്രവേശിപ്പിക്കുകയും ചെയ്തേക്കാം. അല്ലാഹു തന്റെ പ്രവാചകനെയും കൂടെയുള്ള വിശ്വാസികളെയും നിന്ദിക്കാത്ത ദിനമാണത്. അവരുടെ മുന്നിലും വലതുഭാഗത്തും തങ്ങളുടെതന്നെ പ്രകാശം പ്രസരിച്ചുകൊണ്ടിരിക്കും. അവര്‍ പറയും: ഞങ്ങളുടെ നാഥാ! ഞങ്ങളുടെ പ്രകാശം ഞങ്ങള്‍ക്കു നീ പൂര്‍ത്തീകരിച്ചു തരേണമേ! ഞങ്ങളോട് നീ പൊറുക്കേണമേ! നീ എല്ലാറ്റിനും കഴിവുറ്റവന്‍തന്നെ; തീര്‍ച്ച.
\end{malayalam}}
\flushright{\begin{Arabic}
\quranayah[66][9]
\end{Arabic}}
\flushleft{\begin{malayalam}
പ്രവാചകരേ, സത്യനിഷേധികളോടും കപടവിശ്വാസികളോടും സമരം ചെയ്യുക. അവരോട് കര്‍ക്കശമായി പെരുമാറുക. അവരുടെ സങ്കേതം നരകമാകുന്നു. എത്ര ചീത്ത സങ്കേതം!
\end{malayalam}}
\flushright{\begin{Arabic}
\quranayah[66][10]
\end{Arabic}}
\flushleft{\begin{malayalam}
സത്യനിഷേധികള്‍ക്ക് ഉദാഹരണമായി അല്ലാഹു നൂഹിന്റെയും ലൂത്വിന്റെയും ഭാര്യമാരെ എടുത്തു കാണിക്കുന്നു. അവരിരുവരും സദ്വൃത്തരായ നമ്മുടെ രണ്ട് ദാസന്മാരുടെ ഭാര്യമാരായിരുന്നു. എന്നിട്ടും അവരിരുവരും തങ്ങളുടെ ഭര്‍ത്താക്കന്മാരെ വഞ്ചിച്ചു. അതിനാല്‍ അവരിരുവര്‍ക്കും അല്ലാഹുവിന്റെ ശിക്ഷയുടെ കാര്യത്തില്‍ ഭര്‍ത്താക്കന്മാരൊട്ടും ഉപകാരപ്പെട്ടില്ല. ഇരുവരോടും പറഞ്ഞത് ഇതായിരുന്നു: നരകയാത്രികരോടൊപ്പം നിങ്ങളിരുവരും അതില്‍ പ്രവേശിക്കുക.
\end{malayalam}}
\flushright{\begin{Arabic}
\quranayah[66][11]
\end{Arabic}}
\flushleft{\begin{malayalam}
സത്യവിശ്വാസികള്‍ക്ക് ഉദാഹരണമായി അല്ലാഹു ഫറവോന്റെ പത്നിയെ എടുത്തുകാണിക്കുന്നു. അവര്‍ അല്ലാഹുവോട് ഇങ്ങനെ പ്രാര്‍ഥിച്ചു: എന്റെ നാഥാ! എനിക്കു നിന്റെയടുത്ത് സ്വര്‍ഗത്തിലൊരു വീട് ഉണ്ടാക്കിത്തരേണമേ! ഫറവോനില്‍ നിന്നും അയാളുടെ ദുര്‍വൃത്തിയില്‍നിന്നും എന്നെ രക്ഷിക്കേണമേ! അക്രമികളായ ജനത്തില്‍നിന്നും എന്നെ നീ മോചിപ്പിക്കേണമേ!
\end{malayalam}}
\flushright{\begin{Arabic}
\quranayah[66][12]
\end{Arabic}}
\flushleft{\begin{malayalam}
ഇംറാന്റെ പുത്രി മര്‍യമിനെയും ഉദാഹരണമായി എടുത്തു കാണിക്കുന്നു. അവര്‍ തന്റെ ഗുഹ്യസ്ഥാനം കാത്തുസൂക്ഷിച്ചു. അപ്പോള്‍ നാം അതില്‍ നമ്മില്‍ നിന്നുള്ള ആത്മാവിനെ ഊതി. അവളോ തന്റെ നാഥനില്‍ നിന്നുള്ള വചനങ്ങളേയും വേദങ്ങളേയും സത്യപ്പെടുത്തി. അവള്‍ ഭക്തരില്‍ പെട്ടവളായിരുന്നു.
\end{malayalam}}
\chapter{\textmalayalam{മുല്‍ക്ക് ( അധിപത്യം )}}
\begin{Arabic}
\Huge{\centerline{\basmalah}}\end{Arabic}
\flushright{\begin{Arabic}
\quranayah[67][1]
\end{Arabic}}
\flushleft{\begin{malayalam}
ആധിപത്യം ആരുടെ കരങ്ങളിലാണോ അവന്‍ മഹത്വത്തിന്നുടമയത്രെ. അവന്‍ എല്ലാ കാര്യത്തിനും കഴിവുറ്റവനാണ്.
\end{malayalam}}
\flushright{\begin{Arabic}
\quranayah[67][2]
\end{Arabic}}
\flushleft{\begin{malayalam}
മരണവും ജീവിതവും സൃഷ്ടിച്ചവന്‍. കര്‍മ നിര്‍വഹണത്തില്‍ നിങ്ങളിലേറ്റം മികച്ചവരാരെന്ന് പരീക്ഷിക്കാനാണത്. അവന്‍ അജയ്യനാണ്. ഏറെ മാപ്പേകുന്നവനും.
\end{malayalam}}
\flushright{\begin{Arabic}
\quranayah[67][3]
\end{Arabic}}
\flushleft{\begin{malayalam}
ഏഴ് ആകാശങ്ങളെ ഒന്നിനുമീതെ മറ്റൊന്നായി സൃഷ്ടിച്ചവനാണവന്‍. ദയാപരനായ അവന്റെ സൃഷ്ടിയില്‍ ഒരുവിധ ഏറ്റക്കുറവും നിനക്കു കാണാനാവില്ല. ഒന്നുകൂടി നോക്കൂ. എവിടെയെങ്ങാനും വല്ല വിടവും കാണുന്നുണ്ടോ?
\end{malayalam}}
\flushright{\begin{Arabic}
\quranayah[67][4]
\end{Arabic}}
\flushleft{\begin{malayalam}
വീണ്ടും വീണ്ടും നോക്കൂ. നിന്റെ കണ്ണ് തോറ്റ് തളര്‍ന്ന് നിന്നിലേക്കു തന്നെ തിരികെ വരും, തീര്‍ച്ച.
\end{malayalam}}
\flushright{\begin{Arabic}
\quranayah[67][5]
\end{Arabic}}
\flushleft{\begin{malayalam}
തൊട്ടടുത്തുള്ള ആകാശത്തെ നാം വിളക്കുകളാല്‍ അലങ്കരിച്ചു. അവയെ പിശാചുക്കളെ തുരത്താനുള്ള ബാണങ്ങളുമാക്കി. അവര്‍ക്കായി കത്തിക്കാളുന്ന നരകശിക്ഷ ഒരുക്കിവെച്ചിട്ടുമുണ്ട്.
\end{malayalam}}
\flushright{\begin{Arabic}
\quranayah[67][6]
\end{Arabic}}
\flushleft{\begin{malayalam}
തങ്ങളുടെ നാഥനെ നിഷേധിക്കുന്നവര്‍ക്ക് നരകശിക്ഷയാണുള്ളത്. മടങ്ങിച്ചെല്ലാനുള്ള ആ ഇടം വളരെ ചീത്തതന്നെ.
\end{malayalam}}
\flushright{\begin{Arabic}
\quranayah[67][7]
\end{Arabic}}
\flushleft{\begin{malayalam}
അതിലേക്ക് എറിയപ്പെടുമ്പോള്‍ അതിന്റെ ഭീകര ഗര്‍ജനം അവര്‍ കേള്‍ക്കും. അത് തിളച്ചുമറിയുകയായിരിക്കും.
\end{malayalam}}
\flushright{\begin{Arabic}
\quranayah[67][8]
\end{Arabic}}
\flushleft{\begin{malayalam}
ക്ഷോഭത്താല്‍ അത് പൊട്ടിത്തെറിക്കാറാകും. ഓരോ സംഘവും അതില്‍ വലിച്ചെറിയപ്പെടുമ്പോള്‍ അതിന്റെ കാവല്‍ക്കാര്‍ അവരോട് ചോദിച്ചുകൊണ്ടിരിക്കും: "നിങ്ങളുടെയടുത്ത് മുന്നറിയിപ്പുകാരാരും വന്നിരുന്നില്ലേ?”
\end{malayalam}}
\flushright{\begin{Arabic}
\quranayah[67][9]
\end{Arabic}}
\flushleft{\begin{malayalam}
അവര്‍ പറയും: "ഞങ്ങളുടെ അടുത്ത് മുന്നറിയിപ്പുകാരന്‍ വന്നിരുന്നു. പക്ഷേ, ഞങ്ങള്‍ അദ്ദേഹത്തെ തള്ളിപ്പറഞ്ഞു. ഞങ്ങളിങ്ങനെ വാദിക്കുകയും ചെയ്തു: അല്ലാഹു ഒന്നും അവതരിപ്പിച്ചിട്ടില്ല. നിങ്ങള്‍ കൊടിയ വഴികേടില്‍ തന്നെയാണ്.”
\end{malayalam}}
\flushright{\begin{Arabic}
\quranayah[67][10]
\end{Arabic}}
\flushleft{\begin{malayalam}
അവര്‍ കേണുകൊണ്ടിരിക്കും: "അന്നത് കേള്‍ക്കുകയോ, അതേക്കുറിച്ച് ചിന്തിക്കുകയോ ചെയ്തിരുന്നെങ്കില്‍ ഇന്ന് ഈ നരകത്തീയില്‍ അകപ്പെട്ടവരില്‍ പെടുമായിരുന്നില്ല.”
\end{malayalam}}
\flushright{\begin{Arabic}
\quranayah[67][11]
\end{Arabic}}
\flushleft{\begin{malayalam}
അങ്ങനെ അവര്‍ കുറ്റം ഏറ്റുപറഞ്ഞു. നരകത്തീയിന്റെ ആള്‍ക്കാര്‍ക്കു ശാപം!
\end{malayalam}}
\flushright{\begin{Arabic}
\quranayah[67][12]
\end{Arabic}}
\flushleft{\begin{malayalam}
തങ്ങളുടെ നാഥനെ കാണാതെ തന്നെ ഭയപ്പെട്ടു ജീവിക്കുന്നവരോ, അവര്‍ക്ക് പാപമോചനവും മഹത്തായ പ്രതിഫലവുമുണ്ട്.
\end{malayalam}}
\flushright{\begin{Arabic}
\quranayah[67][13]
\end{Arabic}}
\flushleft{\begin{malayalam}
നിങ്ങളുടെ വാക്ക് നിങ്ങള്‍ രഹസ്യമാക്കുകയോ പരസ്യമാക്കുകയോ ചെയ്യുക. തീര്‍ച്ചയായും മനസ്സിലുള്ളതെല്ലാം അറിയുന്നവനാണവന്‍.
\end{malayalam}}
\flushright{\begin{Arabic}
\quranayah[67][14]
\end{Arabic}}
\flushleft{\begin{malayalam}
സൃഷ്ടിച്ചവന്‍ അറിയുകയില്ലെന്നോ! അവന്‍ രഹസ്യങ്ങളറിയുന്നവനും സൂക്ഷ്മജ്ഞാനിയുമാണ്.
\end{malayalam}}
\flushright{\begin{Arabic}
\quranayah[67][15]
\end{Arabic}}
\flushleft{\begin{malayalam}
അവനാണ് നിങ്ങള്‍ക്ക് ഭൂമിയെ അധീനപ്പെടുത്തിത്തന്നത്. അതിനാല്‍ അതിന്റെ വിരിമാറിലൂടെ നടന്നുകൊള്ളുക. അവന്‍ തന്ന വിഭവങ്ങളില്‍നിന്ന് ആഹരിക്കുക. നിങ്ങള്‍ ഉയിര്‍ത്തെഴുന്നേറ്റ് ചെല്ലുന്നതും അവങ്കലേക്കുതന്നെ.
\end{malayalam}}
\flushright{\begin{Arabic}
\quranayah[67][16]
\end{Arabic}}
\flushleft{\begin{malayalam}
ഉപരിലോകത്തുള്ളവന്‍ നിങ്ങളെ ഭൂമിയില്‍ ആഴ്ത്തിക്കളയുന്നതിനെയും അപ്പോള്‍ ഭൂമി ഇളകി മറിയുന്നതിനെയും സംബന്ധിച്ച് നിങ്ങളൊട്ടും ഭയപ്പെടുന്നില്ലേ?
\end{malayalam}}
\flushright{\begin{Arabic}
\quranayah[67][17]
\end{Arabic}}
\flushleft{\begin{malayalam}
അതല്ല; ഉപരിലോകത്തുള്ളവന്‍ നിങ്ങളുടെ മേല്‍ ചരലുകള്‍ ചൊരിയുന്ന കാറ്റിനെ അയക്കുന്നതിനെപ്പറ്റി നിങ്ങള്‍ക്കൊരു പേടിയുമില്ലേ? നമ്മുടെ താക്കീത് എങ്ങനെയുണ്ടെന്ന് വഴിയെ നിങ്ങള്‍ അറിയുകതന്നെ ചെയ്യും.
\end{malayalam}}
\flushright{\begin{Arabic}
\quranayah[67][18]
\end{Arabic}}
\flushleft{\begin{malayalam}
അവര്‍ക്ക് മുമ്പുണ്ടായിരുന്നവരും സത്യത്തെ തള്ളിപ്പറഞ്ഞിട്ടുണ്ട്. അപ്പോള്‍ എവ്വിധമായിരുന്നു എന്റെ ശിക്ഷ.
\end{malayalam}}
\flushright{\begin{Arabic}
\quranayah[67][19]
\end{Arabic}}
\flushleft{\begin{malayalam}
തങ്ങള്‍ക്കു മീതെ ചിറകുവിടര്‍ത്തിയും ഒതുക്കിയും പറക്കുന്ന പക്ഷികളെ അവര്‍ കാണുന്നില്ലേ. അവയെ താങ്ങിനിര്‍ത്തുന്നത് ദയാപരനായ ദൈവമല്ലാതാരുമല്ല. അവന്‍ എല്ലാ കാര്യങ്ങളും കണ്ടറിയുന്നവന്‍ തന്നെ; തീര്‍ച്ച.
\end{malayalam}}
\flushright{\begin{Arabic}
\quranayah[67][20]
\end{Arabic}}
\flushleft{\begin{malayalam}
പരമകാരുണികനായ ദൈവമല്ലാതെ നിങ്ങളെ സഹായിക്കാന്‍ കഴിയുന്ന സൈന്യമേതുണ്ട്? ഉറപ്പായും ഈ സത്യനിഷേധികള്‍ വഞ്ചനയിലകപ്പെട്ടിരിക്കുകയാണ്.
\end{malayalam}}
\flushright{\begin{Arabic}
\quranayah[67][21]
\end{Arabic}}
\flushleft{\begin{malayalam}
അല്ലാഹു അവന്റെ വിഭവം വിലക്കിയാല്‍ നിങ്ങള്‍ക്ക് അന്നം നല്‍കാന്‍ ആരുണ്ട്? യഥാര്‍ഥത്തില്‍ അവര്‍ ധിക്കാരത്തിലും പകയിലും ആണ്ടുപൂണ്ടിരിക്കുകയാണ്.
\end{malayalam}}
\flushright{\begin{Arabic}
\quranayah[67][22]
\end{Arabic}}
\flushleft{\begin{malayalam}
അല്ല, മുഖം നിലത്തുകുത്തി നടക്കുന്നവനോ നേര്‍വഴി പ്രാപിച്ചവന്‍? അതല്ല, സത്യപാതയിലൂടെ നിവര്‍ന്ന് നടക്കുന്നവനോ?
\end{malayalam}}
\flushright{\begin{Arabic}
\quranayah[67][23]
\end{Arabic}}
\flushleft{\begin{malayalam}
പറയുക: അവനാണ് നിങ്ങളെ സൃഷ്ടിച്ചത്. അവന്‍ നിങ്ങള്‍ക്ക് കേള്‍വിയും കാഴ്ചകളും ഹൃദയങ്ങളും ഉണ്ടാക്കി. നിങ്ങള്‍ നന്നെ കുറച്ചേ നന്ദികാണിക്കുന്നുള്ളൂ.
\end{malayalam}}
\flushright{\begin{Arabic}
\quranayah[67][24]
\end{Arabic}}
\flushleft{\begin{malayalam}
പറയുക: അവനാണ് നിങ്ങളെ ഭൂമിയില്‍ സൃഷ്ടിച്ചു വളര്‍ത്തിയത്. അവങ്കലേക്കു തന്നെയാണ് നിങ്ങള്‍ ഒരുമിച്ചുകൂട്ടപ്പെടുന്നതും.
\end{malayalam}}
\flushright{\begin{Arabic}
\quranayah[67][25]
\end{Arabic}}
\flushleft{\begin{malayalam}
അവര്‍ ചോദിക്കുന്നു: നിങ്ങള്‍ സത്യവാദികളെങ്കില്‍ എപ്പോഴാണ് ഈ വാഗ്ദാനം പുലരുക?
\end{malayalam}}
\flushright{\begin{Arabic}
\quranayah[67][26]
\end{Arabic}}
\flushleft{\begin{malayalam}
പറയുക: ആ അറിവ് അല്ലാഹുവിന്റെ അടുക്കല്‍ മാത്രം. ഞാന്‍ വ്യക്തമായ മുന്നറിയിപ്പു നല്‍കുന്നവനല്ലാതാരുമല്ല.
\end{malayalam}}
\flushright{\begin{Arabic}
\quranayah[67][27]
\end{Arabic}}
\flushleft{\begin{malayalam}
മുന്നറിയിപ്പായി പറയുന്ന കാര്യം അടുത്തെത്തിയതായി കാണുമ്പോള്‍ സത്യനിഷേധികളുടെ മുഖം വിഷാദമൂകമായി മാറും. അപ്പോള്‍ അവരോട് പറയും: "ഇതുതന്നെയാണ് നിങ്ങള്‍ ആവശ്യപ്പെട്ടുകൊണ്ടിരുന്നത്.”
\end{malayalam}}
\flushright{\begin{Arabic}
\quranayah[67][28]
\end{Arabic}}
\flushleft{\begin{malayalam}
ചോദിക്കുക: "നിങ്ങള്‍ ആലോചിച്ചിട്ടുണ്ടോ? എന്നെയും എന്നോടൊപ്പമുള്ളവരെയും അല്ലാഹു നശിപ്പിക്കുകയോ ഞങ്ങളോട് കരുണ കാണിക്കുകയോ ചെയ്തുവെന്നിരിക്കട്ടെ; എന്നാല്‍ നോവേറിയ ശിക്ഷയില്‍നിന്ന് സത്യനിഷേധികളെ രക്ഷിക്കാന്‍ ആരുണ്ട്?
\end{malayalam}}
\flushright{\begin{Arabic}
\quranayah[67][29]
\end{Arabic}}
\flushleft{\begin{malayalam}
പറയുക: അവനാണ് ദയാപരന്‍. ഞങ്ങള്‍ അവനില്‍ വിശ്വസിച്ചിരിക്കുന്നു. അവനെതന്നെയാണ് ഞങ്ങള്‍ ഭരമേല്‍പിച്ചതും. ആരാണ് വ്യക്തമായ വഴികേടിലെന്ന് വഴിയെ നിങ്ങളറിയുകതന്നെ ചെയ്യും.
\end{malayalam}}
\flushright{\begin{Arabic}
\quranayah[67][30]
\end{Arabic}}
\flushleft{\begin{malayalam}
ചോദിക്കുക: നിങ്ങള്‍ ആലോചിച്ചിട്ടുണ്ടോ? നിങ്ങളുടെ വെള്ളം വറ്റിപ്പോയാല്‍ ആരാണ് നിങ്ങള്‍ക്ക് തെളിനീരുറവ എത്തിക്കുക?
\end{malayalam}}
\chapter{\textmalayalam{ഖലം ( പേന )}}
\begin{Arabic}
\Huge{\centerline{\basmalah}}\end{Arabic}
\flushright{\begin{Arabic}
\quranayah[68][1]
\end{Arabic}}
\flushleft{\begin{malayalam}
നൂന്‍. പേനയും അവര്‍ എഴുതിവെക്കുന്നതും സാക്ഷി.
\end{malayalam}}
\flushright{\begin{Arabic}
\quranayah[68][2]
\end{Arabic}}
\flushleft{\begin{malayalam}
നിന്റെ നാഥന്റെ അനുഗ്രഹത്താല്‍ നീ ഭ്രാന്തനല്ല.
\end{malayalam}}
\flushright{\begin{Arabic}
\quranayah[68][3]
\end{Arabic}}
\flushleft{\begin{malayalam}
നിശ്ചയമായും നിനക്ക് നിലക്കാത്ത പ്രതിഫലമുണ്ട്.
\end{malayalam}}
\flushright{\begin{Arabic}
\quranayah[68][4]
\end{Arabic}}
\flushleft{\begin{malayalam}
നീ മഹത്തായ സ്വഭാവത്തിനുടമതന്നെ; തീര്‍ച്ച.
\end{malayalam}}
\flushright{\begin{Arabic}
\quranayah[68][5]
\end{Arabic}}
\flushleft{\begin{malayalam}
വൈകാതെ നീ കണ്ടറിയും. അവരും കണ്ടറിയും.
\end{malayalam}}
\flushright{\begin{Arabic}
\quranayah[68][6]
\end{Arabic}}
\flushleft{\begin{malayalam}
നിങ്ങളില്‍ ആരാണ് കുഴപ്പത്തിലായതെന്ന്?
\end{malayalam}}
\flushright{\begin{Arabic}
\quranayah[68][7]
\end{Arabic}}
\flushleft{\begin{malayalam}
നിശ്ചയമായും നിന്റെ നാഥന്‍ വഴി തെറ്റിയവരെ നന്നായറിയുന്നവനാണ്. നേര്‍വഴി പ്രാപിച്ചവരെയും അവനു നന്നായറിയാം.
\end{malayalam}}
\flushright{\begin{Arabic}
\quranayah[68][8]
\end{Arabic}}
\flushleft{\begin{malayalam}
അതിനാല്‍ നീ സത്യനിഷേധികളെ അനുസരിക്കരുത്.
\end{malayalam}}
\flushright{\begin{Arabic}
\quranayah[68][9]
\end{Arabic}}
\flushleft{\begin{malayalam}
നീ അല്‍പം അനുനയം കാണിച്ചെങ്കില്‍ തങ്ങള്‍ക്കും അനുനയം ആകാമായിരുന്നുവെന്ന് അവരാഗ്രഹിക്കുന്നു.
\end{malayalam}}
\flushright{\begin{Arabic}
\quranayah[68][10]
\end{Arabic}}
\flushleft{\begin{malayalam}
അടിക്കടി ആണയിട്ടുകൊണ്ടിരിക്കുന്ന അതിനീചനെ നീ അനുസരിക്കരുത്.
\end{malayalam}}
\flushright{\begin{Arabic}
\quranayah[68][11]
\end{Arabic}}
\flushleft{\begin{malayalam}
അവനോ ദൂഷണം പറയുന്നവന്‍, ഏഷണിയുമായി ചുറ്റിക്കറങ്ങുന്നവന്‍.
\end{malayalam}}
\flushright{\begin{Arabic}
\quranayah[68][12]
\end{Arabic}}
\flushleft{\begin{malayalam}
നന്മയെ തടയുന്നവന്‍, അതിക്രമി, മഹാപാപി.
\end{malayalam}}
\flushright{\begin{Arabic}
\quranayah[68][13]
\end{Arabic}}
\flushleft{\begin{malayalam}
ക്രൂരന്‍, പിന്നെ, പിഴച്ചു പെറ്റവനും.
\end{malayalam}}
\flushright{\begin{Arabic}
\quranayah[68][14]
\end{Arabic}}
\flushleft{\begin{malayalam}
അതിനു കാരണമോ സമൃദ്ധമായ സമ്പത്തും സന്താനങ്ങളുമുണ്ടെന്നതും.
\end{malayalam}}
\flushright{\begin{Arabic}
\quranayah[68][15]
\end{Arabic}}
\flushleft{\begin{malayalam}
നമ്മുടെ സൂക്തങ്ങള്‍ ഓതിക്കേള്‍പ്പിക്കപ്പെട്ടാല്‍ അവന്‍ പറയും: "ഇത് പൂര്‍വികരുടെ പുരാണ കഥകളാണ്.”
\end{malayalam}}
\flushright{\begin{Arabic}
\quranayah[68][16]
\end{Arabic}}
\flushleft{\begin{malayalam}
അടുത്തുതന്നെ അവന്റെ തുമ്പിക്കൈക്ക് നാം അടയാളമിടും.
\end{malayalam}}
\flushright{\begin{Arabic}
\quranayah[68][17]
\end{Arabic}}
\flushleft{\begin{malayalam}
ഇവരെ നാം പരീക്ഷണ വിധേയരാക്കിയിരിക്കുന്നു. തോട്ടക്കാരെ പരീക്ഷിച്ചപോലെ. തോട്ടത്തിലെ പഴങ്ങള്‍ പ്രഭാതത്തില്‍ തന്നെ പറിച്ചെടുക്കുമെന്ന് അവര്‍ ശപഥം ചെയ്ത സന്ദര്‍ഭം!
\end{malayalam}}
\flushright{\begin{Arabic}
\quranayah[68][18]
\end{Arabic}}
\flushleft{\begin{malayalam}
അവര്‍ ഒന്നും ഒഴിവാക്കിപ്പറഞ്ഞില്ല.
\end{malayalam}}
\flushright{\begin{Arabic}
\quranayah[68][19]
\end{Arabic}}
\flushleft{\begin{malayalam}
അങ്ങനെ അവര്‍ ഉറങ്ങവെ നിന്റെ നാഥനില്‍നിന്നുള്ള വിപത്ത് ആ തോട്ടത്തെ ബാധിച്ചു.
\end{malayalam}}
\flushright{\begin{Arabic}
\quranayah[68][20]
\end{Arabic}}
\flushleft{\begin{malayalam}
അത് വിളവെടുപ്പ് കഴിഞ്ഞ വയല്‍പോലെയായി.
\end{malayalam}}
\flushright{\begin{Arabic}
\quranayah[68][21]
\end{Arabic}}
\flushleft{\begin{malayalam}
പ്രഭാതവേളയില്‍ അവരന്യോന്യം വിളിച്ചുപറഞ്ഞു:
\end{malayalam}}
\flushright{\begin{Arabic}
\quranayah[68][22]
\end{Arabic}}
\flushleft{\begin{malayalam}
"നിങ്ങള്‍ വിളവെടുക്കുന്നുവെങ്കില്‍ നിങ്ങളുടെ കൃഷിയിടത്തേക്ക് നേരത്തെ തന്നെ പുറപ്പെട്ടുകൊള്ളുക.”
\end{malayalam}}
\flushright{\begin{Arabic}
\quranayah[68][23]
\end{Arabic}}
\flushleft{\begin{malayalam}
അന്യോന്യം സ്വകാര്യം പറഞ്ഞുകൊണ്ട് അവര്‍ പുറപ്പെട്ടു:
\end{malayalam}}
\flushright{\begin{Arabic}
\quranayah[68][24]
\end{Arabic}}
\flushleft{\begin{malayalam}
"ദരിദ്രവാസികളാരും ഇന്നവിടെ കടന്നുവരാനിടവരരുത്.”
\end{malayalam}}
\flushright{\begin{Arabic}
\quranayah[68][25]
\end{Arabic}}
\flushleft{\begin{malayalam}
അവരെ തടയാന്‍ തങ്ങള്‍ കഴിവുറ്റവരെന്നവണ്ണം അവര്‍ അവിടെയെത്തി.
\end{malayalam}}
\flushright{\begin{Arabic}
\quranayah[68][26]
\end{Arabic}}
\flushleft{\begin{malayalam}
എന്നാല്‍ തോട്ടം കണ്ടപ്പോള്‍ അവര്‍ വിലപിക്കാന്‍ തുടങ്ങി: "നാം വഴി തെറ്റിയിരിക്കുന്നു.
\end{malayalam}}
\flushright{\begin{Arabic}
\quranayah[68][27]
\end{Arabic}}
\flushleft{\begin{malayalam}
"അല്ല; നാം എല്ലാം നഷ്ടപ്പെട്ടവരായിരിക്കുന്നു.”
\end{malayalam}}
\flushright{\begin{Arabic}
\quranayah[68][28]
\end{Arabic}}
\flushleft{\begin{malayalam}
കൂട്ടത്തില്‍ മധ്യമ നിലപാട് സ്വീകരിച്ചയാള്‍ പറഞ്ഞു: "നിങ്ങള്‍ എന്തുകൊണ്ട് ദൈവകീര്‍ത്തനം നടത്തുന്നില്ലെന്ന് ഞാന്‍ ചോദിച്ചിരുന്നില്ലേ?”
\end{malayalam}}
\flushright{\begin{Arabic}
\quranayah[68][29]
\end{Arabic}}
\flushleft{\begin{malayalam}
അവര്‍ പറഞ്ഞു: "നമ്മുടെ നാഥന്‍ എത്ര പരിശുദ്ധന്‍! നിശ്ചയമായും നാം അക്രമികളായിരിക്കുന്നു.”
\end{malayalam}}
\flushright{\begin{Arabic}
\quranayah[68][30]
\end{Arabic}}
\flushleft{\begin{malayalam}
അങ്ങനെ അവരന്യോന്യം പഴിചാരാന്‍ തുടങ്ങി.
\end{malayalam}}
\flushright{\begin{Arabic}
\quranayah[68][31]
\end{Arabic}}
\flushleft{\begin{malayalam}
അവര്‍ വിലപിച്ചു: "നമ്മുടെ നാശം! നിശ്ചയമായും നാം അതിക്രമികളായിരിക്കുന്നു.
\end{malayalam}}
\flushright{\begin{Arabic}
\quranayah[68][32]
\end{Arabic}}
\flushleft{\begin{malayalam}
"നമ്മുടെ നാഥന്‍ ഇതിനെക്കാള്‍ നല്ലത് നമുക്ക് പകരം നല്‍കിയേക്കാം. നിശ്ചയമായും നാം നമ്മുടെ നാഥനില്‍ പ്രതീക്ഷയര്‍പ്പിക്കുന്നവരാകുന്നു.”
\end{malayalam}}
\flushright{\begin{Arabic}
\quranayah[68][33]
\end{Arabic}}
\flushleft{\begin{malayalam}
ഇവ്വിധമാണ് ഇവിടത്തെ ശിക്ഷ. പരലോക ശിക്ഷയോ കൂടുതല്‍ കഠിനവും. അവര്‍ അറിഞ്ഞിരുന്നെങ്കില്‍!
\end{malayalam}}
\flushright{\begin{Arabic}
\quranayah[68][34]
\end{Arabic}}
\flushleft{\begin{malayalam}
ഉറപ്പായും ദൈവ ഭക്തര്‍ക്ക് തങ്ങളുടെ നാഥന്റെയടുക്കല്‍ അനുഗൃഹീതമായ സ്വര്‍ഗീയാരാമങ്ങളുണ്ട്.
\end{malayalam}}
\flushright{\begin{Arabic}
\quranayah[68][35]
\end{Arabic}}
\flushleft{\begin{malayalam}
അപ്പോള്‍ മുസ്ലിംകളോടു നാം കുറ്റവാളികളെപ്പോലെയാണോ പെരുമാറുക?
\end{malayalam}}
\flushright{\begin{Arabic}
\quranayah[68][36]
\end{Arabic}}
\flushleft{\begin{malayalam}
നിങ്ങള്‍ക്കെന്തുപറ്റി? എങ്ങനെയൊക്കെയാണ് നിങ്ങള്‍ തീര്‍പ്പു കല്‍പിക്കുന്നത്.
\end{malayalam}}
\flushright{\begin{Arabic}
\quranayah[68][37]
\end{Arabic}}
\flushleft{\begin{malayalam}
അതല്ല, നിങ്ങളുടെ വശം വല്ല വേദപുസ്തകവുമുണ്ടോ? നിങ്ങളതില്‍ പഠനം നടത്തിക്കൊണ്ടിരിക്കുകയാണോ?
\end{malayalam}}
\flushright{\begin{Arabic}
\quranayah[68][38]
\end{Arabic}}
\flushleft{\begin{malayalam}
നിങ്ങള്‍ ആഗ്രഹിക്കുന്നതൊക്കെ നിങ്ങള്‍ക്ക് അതിലുണ്ടെന്നോ?
\end{malayalam}}
\flushright{\begin{Arabic}
\quranayah[68][39]
\end{Arabic}}
\flushleft{\begin{malayalam}
അതല്ലെങ്കില്‍ നിങ്ങള്‍ തീരുമാനിക്കുന്നതു തന്നെ നിങ്ങള്‍ക്ക് ലഭിക്കുമെന്നതിന് ഉയിര്‍ത്തെഴുന്നേല്‍പുനാള്‍ വരെ നിലനില്‍ക്കുന്ന വല്ല കരാറും നമ്മുടെ പേരില്‍ നിങ്ങള്‍ക്കുണ്ടോ?
\end{malayalam}}
\flushright{\begin{Arabic}
\quranayah[68][40]
\end{Arabic}}
\flushleft{\begin{malayalam}
അവരോട് ചോദിക്കുക: തങ്ങളില്‍ ആരാണ് അതിന്റെ ഉത്തരവാദിത്തം ഏല്‍ക്കുന്നത്?
\end{malayalam}}
\flushright{\begin{Arabic}
\quranayah[68][41]
\end{Arabic}}
\flushleft{\begin{malayalam}
അതല്ല, അവര്‍ക്ക് വല്ല പങ്കുകാരുമുണ്ടോ? എങ്കില്‍ അവരുടെ പങ്കാളികളെ അവരിങ്ങ് കൊണ്ടുവരട്ടെ. അവര്‍ സത്യവാദികളെങ്കില്‍!
\end{malayalam}}
\flushright{\begin{Arabic}
\quranayah[68][42]
\end{Arabic}}
\flushleft{\begin{malayalam}
കണങ്കാല്‍ വെളിവാക്കപ്പെടുംനാള്‍; അന്നവര്‍ സാഷ്ടാംഗം പ്രണമിക്കാന്‍ വിളിക്കപ്പെടും. എന്നാല്‍ അവര്‍ക്കതിനു സാധ്യമാവില്ല.
\end{malayalam}}
\flushright{\begin{Arabic}
\quranayah[68][43]
\end{Arabic}}
\flushleft{\begin{malayalam}
അന്നവരുടെ നോട്ടം കീഴ്പോട്ടായിരിക്കും. അപമാനം അവരെ ആവരണം ചെയ്യും. നേരത്തെ അവര്‍ പ്രണാമമര്‍പ്പിക്കാന്‍ വിളിക്കപ്പെട്ടിരുന്നല്ലോ. അന്നവര്‍ സുരക്ഷിതരുമായിരുന്നു.
\end{malayalam}}
\flushright{\begin{Arabic}
\quranayah[68][44]
\end{Arabic}}
\flushleft{\begin{malayalam}
അതിനാല്‍ ഈ വചനങ്ങളെ തള്ളിപ്പറയുന്നവരുടെ കാര്യം എനിക്കു വിട്ടുതരിക. അവരറിയാത്ത വിധം നാമവരെ പടിപടിയായി പിടികൂടും.
\end{malayalam}}
\flushright{\begin{Arabic}
\quranayah[68][45]
\end{Arabic}}
\flushleft{\begin{malayalam}
നാമവര്‍ക്ക് സാവകാശം നല്‍കിയിരിക്കുകയാണ്. എന്റെ തന്ത്രം ഭദ്രം തന്നെ; തീര്‍ച്ച.
\end{malayalam}}
\flushright{\begin{Arabic}
\quranayah[68][46]
\end{Arabic}}
\flushleft{\begin{malayalam}
അല്ല; നീ അവരോട് വല്ല പ്രതിഫലവും ആവശ്യപ്പെടുന്നുണ്ടോ? അങ്ങനെ അവര്‍ കടബാധ്യതയാല്‍ കഷ്ടപ്പെടുകയാണോ?
\end{malayalam}}
\flushright{\begin{Arabic}
\quranayah[68][47]
\end{Arabic}}
\flushleft{\begin{malayalam}
അതല്ലെങ്കില്‍ അവരുടെ വശം വല്ല അഭൌതിക ജ്ഞാനവുമുണ്ടോ? അവര്‍ അത് എഴുതിയെടുക്കുകയാണോ?
\end{malayalam}}
\flushright{\begin{Arabic}
\quranayah[68][48]
\end{Arabic}}
\flushleft{\begin{malayalam}
അതിനാല്‍ നീ നിന്റെ നാഥന്റെ തീരുമാനങ്ങള്‍ക്കായി ക്ഷമയോടെ കാത്തിരിക്കുക. നീ ആ മത്സ്യക്കാരനെപ്പോലെ ആകരുത്. അദ്ദേഹം കൊടും ദുഃഖിതനായി പ്രാര്‍ഥിച്ച സന്ദര്‍ഭം ഓര്‍ക്കുക.
\end{malayalam}}
\flushright{\begin{Arabic}
\quranayah[68][49]
\end{Arabic}}
\flushleft{\begin{malayalam}
തന്റെ നാഥനില്‍നിന്നുള്ള അനുഗ്രഹം രക്ഷക്കെത്തിയിരുന്നില്ലെങ്കില്‍ അദ്ദേഹം ആ പാഴ്മണല്‍ക്കാട്ടില്‍ ആക്ഷേപിതനായി ഉപേക്ഷിക്കപ്പെടുമായിരുന്നു.
\end{malayalam}}
\flushright{\begin{Arabic}
\quranayah[68][50]
\end{Arabic}}
\flushleft{\begin{malayalam}
അവസാനം അദ്ദേഹത്തിന്റെ നാഥന്‍ അദ്ദേഹത്തെ തെരഞ്ഞെടുത്തു. അങ്ങനെ സജ്ജനങ്ങളിലുള്‍പ്പെടുത്തുകയും ചെയ്തു.
\end{malayalam}}
\flushright{\begin{Arabic}
\quranayah[68][51]
\end{Arabic}}
\flushleft{\begin{malayalam}
ഈ ഉദ്ബോധനം കേള്‍ക്കുമ്പോള്‍ സത്യനിഷേധികള്‍ നീ നിന്റെ കാലിടറി വീഴുമാറ് നിന്നെ തുറിച്ചു നോക്കുന്നു. ഇവന്‍ ഒരു മുഴു ഭ്രാന്തന്‍ തന്നെയെന്ന് പുലമ്പുകയും ചെയ്യുന്നു.
\end{malayalam}}
\flushright{\begin{Arabic}
\quranayah[68][52]
\end{Arabic}}
\flushleft{\begin{malayalam}
എന്നാലിത് മുഴുലോകര്‍ക്കുമുള്ള ഒരുദ്ബോധനമല്ലാതൊന്നുമല്ല.
\end{malayalam}}
\chapter{\textmalayalam{ഹാഖ ( യഥാര്‍ത്ഥ സംഭവം )}}
\begin{Arabic}
\Huge{\centerline{\basmalah}}\end{Arabic}
\flushright{\begin{Arabic}
\quranayah[69][1]
\end{Arabic}}
\flushleft{\begin{malayalam}
അനിവാര്യ സംഭവം!
\end{malayalam}}
\flushright{\begin{Arabic}
\quranayah[69][2]
\end{Arabic}}
\flushleft{\begin{malayalam}
എന്താണ് ആ അനിവാര്യ സംഭവം?
\end{malayalam}}
\flushright{\begin{Arabic}
\quranayah[69][3]
\end{Arabic}}
\flushleft{\begin{malayalam}
ആ അനിവാര്യ സംഭവമെന്തെന്ന് നിനക്കെന്തറിയാം?
\end{malayalam}}
\flushright{\begin{Arabic}
\quranayah[69][4]
\end{Arabic}}
\flushleft{\begin{malayalam}
സമൂദും ആദും ആ കൊടും വിപത്തിനെ തള്ളിപ്പറഞ്ഞു.
\end{malayalam}}
\flushright{\begin{Arabic}
\quranayah[69][5]
\end{Arabic}}
\flushleft{\begin{malayalam}
എന്നിട്ടോ സമൂദ് ഗോത്രം കൊടും കെടുതിയാല്‍ നശിപ്പിക്കപ്പെട്ടു.
\end{malayalam}}
\flushright{\begin{Arabic}
\quranayah[69][6]
\end{Arabic}}
\flushleft{\begin{malayalam}
ആദ് ഗോത്രം അത്യുഗ്രമായി ആഞ്ഞടിച്ച കൊടുങ്കാറ്റിനാലും നാമാവശേഷമായി.
\end{malayalam}}
\flushright{\begin{Arabic}
\quranayah[69][7]
\end{Arabic}}
\flushleft{\begin{malayalam}
ഏഴു രാവും എട്ടു പകലും ഇടതടവില്ലാതെ അല്ലാഹു അതിനെ അവരുടെ നേരെ തിരിച്ചുവിട്ടു. അപ്പോള്‍ നുരുമ്പിയ ഈത്തപ്പനത്തടികള്‍ പോലെ ആ കാറ്റിലവര്‍ ഉയിരറ്റു കിടക്കുന്നത് നിനക്ക് കാണാമായിരുന്നു.
\end{malayalam}}
\flushright{\begin{Arabic}
\quranayah[69][8]
\end{Arabic}}
\flushleft{\begin{malayalam}
അവരുടേതായി വല്ലതും ബാക്കിയായത് നീ കാണുന്നുണ്ടോ?
\end{malayalam}}
\flushright{\begin{Arabic}
\quranayah[69][9]
\end{Arabic}}
\flushleft{\begin{malayalam}
ഫറവോനും അവനു മുമ്പുള്ളവരും കീഴ്മേല്‍ മറിക്കപ്പെട്ട നാടുകളും അതേ കുറ്റകൃത്യം തന്നെ ചെയ്തു.
\end{malayalam}}
\flushright{\begin{Arabic}
\quranayah[69][10]
\end{Arabic}}
\flushleft{\begin{malayalam}
അവരൊക്കെയും തങ്ങളുടെ നാഥന്റെ ദൂതനെ ധിക്കരിച്ചു. അപ്പോള്‍ അവന്‍ അവരെ കഠിന ശിക്ഷയാല്‍ പിടികൂടുകയായിരുന്നു.
\end{malayalam}}
\flushright{\begin{Arabic}
\quranayah[69][11]
\end{Arabic}}
\flushleft{\begin{malayalam}
പ്രളയം പരിധി കടന്നപ്പോള്‍ നിങ്ങളെ നാം കപ്പലില്‍ കയറ്റി രക്ഷിച്ചു.
\end{malayalam}}
\flushright{\begin{Arabic}
\quranayah[69][12]
\end{Arabic}}
\flushleft{\begin{malayalam}
ആ സംഭവത്തെ നാം നിങ്ങള്‍ക്ക് ഓര്‍ക്കാനുള്ള ഒന്നാക്കാനാണത്. സൂക്ഷ്മതയുള്ള കാതുകള്‍ എക്കാലത്തേക്കും ഒരോര്‍മയാക്കാനും.
\end{malayalam}}
\flushright{\begin{Arabic}
\quranayah[69][13]
\end{Arabic}}
\flushleft{\begin{malayalam}
പിന്നെ കാഹളത്തില്‍ ഒരൂത്ത് ഊതപ്പെട്ടാല്‍.
\end{malayalam}}
\flushright{\begin{Arabic}
\quranayah[69][14]
\end{Arabic}}
\flushleft{\begin{malayalam}
ഭൂമിയും പര്‍വതങ്ങളും പൊക്കിയെടുത്ത് രണ്ടിനെയും ഒറ്റയടിക്ക് ഇടിച്ചു തരിപ്പണമാക്കിയാല്‍.
\end{malayalam}}
\flushright{\begin{Arabic}
\quranayah[69][15]
\end{Arabic}}
\flushleft{\begin{malayalam}
അന്നാണ് അനിവാര്യ സംഭവം നടക്കുക.
\end{malayalam}}
\flushright{\begin{Arabic}
\quranayah[69][16]
\end{Arabic}}
\flushleft{\begin{malayalam}
അന്ന് ആകാശം പൊട്ടിപ്പിളരുന്നു. അന്നത് നന്നേ ദുര്‍ബലമായിരിക്കും.
\end{malayalam}}
\flushright{\begin{Arabic}
\quranayah[69][17]
\end{Arabic}}
\flushleft{\begin{malayalam}
മലക്കുകള്‍ അതിന്റെ നാനാഭാഗങ്ങളിലുമുണ്ടായിരിക്കും. നിന്റെ നാഥന്റെ സിംഹാസനം എട്ടുപേര്‍ തങ്ങള്‍ക്കു മുകളിലായി ചുമക്കും.
\end{malayalam}}
\flushright{\begin{Arabic}
\quranayah[69][18]
\end{Arabic}}
\flushleft{\begin{malayalam}
അന്ന് നിങ്ങള്‍ ദൈവസന്നിധിയില്‍ ഹാജരാക്കപ്പെടും. നിങ്ങളില്‍ നിന്ന് ഒരു രഹസ്യം പോലും മറഞ്ഞു കിടക്കുകയില്ല.
\end{malayalam}}
\flushright{\begin{Arabic}
\quranayah[69][19]
\end{Arabic}}
\flushleft{\begin{malayalam}
അപ്പോള്‍ കര്‍മപുസ്തകം തന്റെവലതു കയ്യില്‍ കിട്ടിയവന്‍ പറയും: "ഇതാ എന്റെ കര്‍മപുസ്തകം; വായിച്ചു നോക്കൂ.
\end{malayalam}}
\flushright{\begin{Arabic}
\quranayah[69][20]
\end{Arabic}}
\flushleft{\begin{malayalam}
"എന്റെ വിചാരണയെ ഞാന്‍ നേരിടേണ്ടിവരുമെന്ന് എനിക്കുറപ്പുണ്ടായിരുന്നു.”
\end{malayalam}}
\flushright{\begin{Arabic}
\quranayah[69][21]
\end{Arabic}}
\flushleft{\begin{malayalam}
അങ്ങനെ അവന്‍ സംതൃപ്തമായ ജീവിതത്തിലെത്തുന്നു.
\end{malayalam}}
\flushright{\begin{Arabic}
\quranayah[69][22]
\end{Arabic}}
\flushleft{\begin{malayalam}
ഉന്നതമായ സ്വര്‍ഗത്തില്‍.
\end{malayalam}}
\flushright{\begin{Arabic}
\quranayah[69][23]
\end{Arabic}}
\flushleft{\begin{malayalam}
അതിലെ പഴങ്ങള്‍ വളരെ അടുത്തായി തൂങ്ങിക്കിടക്കുന്നുണ്ടായിരിക്കും.
\end{malayalam}}
\flushright{\begin{Arabic}
\quranayah[69][24]
\end{Arabic}}
\flushleft{\begin{malayalam}
കഴിഞ്ഞ നാളുകളില്‍ നിങ്ങള്‍ ചെയ്തിരുന്നതിന്റെ പ്രതിഫലമായി ഇതാ തൃപ്തിയോടെ തിന്നുകയും കുടിക്കുകയും ചെയ്യുക.
\end{malayalam}}
\flushright{\begin{Arabic}
\quranayah[69][25]
\end{Arabic}}
\flushleft{\begin{malayalam}
എന്നാല്‍ ഇടതു കൈയില്‍ കര്‍മപുസ്തകം കിട്ടുന്നവനോ, അവന്‍ പറയും: കഷ്ടം! എനിക്കെന്റെ കര്‍മപുസ്തകം കിട്ടിയില്ലായിരുന്നെങ്കില്‍!
\end{malayalam}}
\flushright{\begin{Arabic}
\quranayah[69][26]
\end{Arabic}}
\flushleft{\begin{malayalam}
എന്റെ കണക്ക് എന്തെന്ന് ഞാന്‍ അറിഞ്ഞിരുന്നില്ലെങ്കില്‍!
\end{malayalam}}
\flushright{\begin{Arabic}
\quranayah[69][27]
\end{Arabic}}
\flushleft{\begin{malayalam}
മരണം എല്ലാറ്റിന്റെയും ഒടുക്കമായിരുന്നെങ്കില്‍!
\end{malayalam}}
\flushright{\begin{Arabic}
\quranayah[69][28]
\end{Arabic}}
\flushleft{\begin{malayalam}
എന്റെ ധനം എനിക്കൊട്ടും ഉപകരിച്ചില്ല.
\end{malayalam}}
\flushright{\begin{Arabic}
\quranayah[69][29]
\end{Arabic}}
\flushleft{\begin{malayalam}
എന്റെ അധികാരങ്ങളൊക്കെയും എനിക്ക് നഷ്ടപ്പെട്ടു.
\end{malayalam}}
\flushright{\begin{Arabic}
\quranayah[69][30]
\end{Arabic}}
\flushleft{\begin{malayalam}
അപ്പോള്‍ കല്പനയുണ്ടാകുന്നു: നിങ്ങള്‍ അവനെ പിടിച്ച് കുരുക്കിലിടൂ.
\end{malayalam}}
\flushright{\begin{Arabic}
\quranayah[69][31]
\end{Arabic}}
\flushleft{\begin{malayalam}
പിന്നെ നരകത്തീയിലെറിയൂ.
\end{malayalam}}
\flushright{\begin{Arabic}
\quranayah[69][32]
\end{Arabic}}
\flushleft{\begin{malayalam}
എന്നിട്ട് എഴുപതു മുഴം നീളമുള്ള ചങ്ങലകൊണ്ട് കെട്ടിവരിയൂ.
\end{malayalam}}
\flushright{\begin{Arabic}
\quranayah[69][33]
\end{Arabic}}
\flushleft{\begin{malayalam}
അവന്‍ മഹാനായ അല്ലാഹുവില്‍ വിശ്വസിച്ചിരുന്നില്ല.
\end{malayalam}}
\flushright{\begin{Arabic}
\quranayah[69][34]
\end{Arabic}}
\flushleft{\begin{malayalam}
അഗതികള്‍ക്ക് അന്നം നല്‍കാന്‍ പ്രേരിപ്പിച്ചിരുന്നുമില്ല.
\end{malayalam}}
\flushright{\begin{Arabic}
\quranayah[69][35]
\end{Arabic}}
\flushleft{\begin{malayalam}
അതിനാല്‍ അവനിന്നിവിടെ ഒരു മിത്രവുമില്ല.
\end{malayalam}}
\flushright{\begin{Arabic}
\quranayah[69][36]
\end{Arabic}}
\flushleft{\begin{malayalam}
ഒരാഹാരവുമില്ല. വ്രണങ്ങളുടെ പൊറ്റയല്ലാതെ.
\end{malayalam}}
\flushright{\begin{Arabic}
\quranayah[69][37]
\end{Arabic}}
\flushleft{\begin{malayalam}
പാപികളല്ലാതെ അതു തിന്നുകയില്ല.
\end{malayalam}}
\flushright{\begin{Arabic}
\quranayah[69][38]
\end{Arabic}}
\flushleft{\begin{malayalam}
വേണ്ടാ, നിങ്ങള്‍ കാണുന്ന സകല വസ്തുക്കളെക്കൊണ്ടും ഞാന്‍ സത്യം ചെയ്യുന്നു.
\end{malayalam}}
\flushright{\begin{Arabic}
\quranayah[69][39]
\end{Arabic}}
\flushleft{\begin{malayalam}
നിങ്ങള്‍ക്കു കാണാനാവാത്തവയെക്കൊണ്ടും.
\end{malayalam}}
\flushright{\begin{Arabic}
\quranayah[69][40]
\end{Arabic}}
\flushleft{\begin{malayalam}
തീര്‍ച്ചയായും ഇത് മാന്യനായ ദൈവദൂതന്റെ വചനങ്ങളാണ്.
\end{malayalam}}
\flushright{\begin{Arabic}
\quranayah[69][41]
\end{Arabic}}
\flushleft{\begin{malayalam}
ഇത് കവിവാക്യമല്ല. നിങ്ങള്‍ കുറച്ചേ വിശ്വസിക്കുന്നുള്ളൂ.
\end{malayalam}}
\flushright{\begin{Arabic}
\quranayah[69][42]
\end{Arabic}}
\flushleft{\begin{malayalam}
ഇത് ജ്യോത്സ്യന്റെ വാക്കുമല്ല. നന്നെക്കുറച്ചേ നിങ്ങള്‍ ആലോചിക്കുന്നുള്ളൂ.
\end{malayalam}}
\flushright{\begin{Arabic}
\quranayah[69][43]
\end{Arabic}}
\flushleft{\begin{malayalam}
ഇത് ലോകനാഥനില്‍ നിന്ന് അവതീര്‍ണമായതാണ്.
\end{malayalam}}
\flushright{\begin{Arabic}
\quranayah[69][44]
\end{Arabic}}
\flushleft{\begin{malayalam}
പ്രവാചകന്‍ നമ്മുടെ മേല്‍ വല്ലതും കെട്ടിച്ചമച്ച് പറയുകയാണെങ്കില്‍.
\end{malayalam}}
\flushright{\begin{Arabic}
\quranayah[69][45]
\end{Arabic}}
\flushleft{\begin{malayalam}
അദ്ദേഹത്തിന്റെ വലംകൈ നാം പിടിക്കുമായിരുന്നു.
\end{malayalam}}
\flushright{\begin{Arabic}
\quranayah[69][46]
\end{Arabic}}
\flushleft{\begin{malayalam}
എന്നിട്ട് അദ്ദേഹത്തിന്റെ ജീവനാഡി മുറിച്ചു കളയുമായിരുന്നു.
\end{malayalam}}
\flushright{\begin{Arabic}
\quranayah[69][47]
\end{Arabic}}
\flushleft{\begin{malayalam}
അപ്പോള്‍ നിങ്ങളിലാര്‍ക്കും അദ്ദേഹത്തില്‍നിന്ന് നമ്മുടെ ശിക്ഷയെ തടയാനാവില്ല.
\end{malayalam}}
\flushright{\begin{Arabic}
\quranayah[69][48]
\end{Arabic}}
\flushleft{\begin{malayalam}
ഉറപ്പായും ഇത് ഭക്തന്മാര്‍ക്ക് ഒരുദ്ബോധനമാണ്.
\end{malayalam}}
\flushright{\begin{Arabic}
\quranayah[69][49]
\end{Arabic}}
\flushleft{\begin{malayalam}
നിശ്ചയമായും നമുക്കറിയാം; നിങ്ങളില്‍ ഇതിനെ തള്ളിപ്പറയുന്നവരുണ്ട്.
\end{malayalam}}
\flushright{\begin{Arabic}
\quranayah[69][50]
\end{Arabic}}
\flushleft{\begin{malayalam}
തീര്‍ച്ചയായും അത്തരം സത്യനിഷേധികള്‍ക്കിത് ദുഃഖകാരണം തന്നെ.
\end{malayalam}}
\flushright{\begin{Arabic}
\quranayah[69][51]
\end{Arabic}}
\flushleft{\begin{malayalam}
നിശ്ചയമായും ഇത് സുദൃഢമായ സത്യമാണ്.
\end{malayalam}}
\flushright{\begin{Arabic}
\quranayah[69][52]
\end{Arabic}}
\flushleft{\begin{malayalam}
അതിനാല്‍ നീ നിന്റെ അത്യുന്നതനായ നാഥന്റെ നാമം കീര്‍ത്തിച്ചുകൊണ്ടിരിക്കുക.
\end{malayalam}}
\chapter{\textmalayalam{മആരിജ് ( കയറുന്ന വഴികള്‍ )}}
\begin{Arabic}
\Huge{\centerline{\basmalah}}\end{Arabic}
\flushright{\begin{Arabic}
\quranayah[70][1]
\end{Arabic}}
\flushleft{\begin{malayalam}
സംഭവിക്കാനിരിക്കുന്ന ശിക്ഷയെ സംബന്ധിച്ച് ഒരന്വേഷകന്‍ ആരാഞ്ഞുവല്ലോ.
\end{malayalam}}
\flushright{\begin{Arabic}
\quranayah[70][2]
\end{Arabic}}
\flushleft{\begin{malayalam}
അത് സത്യനിഷേധികള്‍ക്കുള്ളതാണ്. അതിനെ തടയുന്ന ആരുമില്ല.
\end{malayalam}}
\flushright{\begin{Arabic}
\quranayah[70][3]
\end{Arabic}}
\flushleft{\begin{malayalam}
ചവിട്ടുപടികളുടെ ഉടമയായ അല്ലാഹുവില്‍ നിന്നുള്ളതാണത്.
\end{malayalam}}
\flushright{\begin{Arabic}
\quranayah[70][4]
\end{Arabic}}
\flushleft{\begin{malayalam}
മലക്കുകളും പരിശുദ്ധാത്മാവും അവന്റെ സന്നിധിയിലേക്ക് കയറിപ്പോകുന്നു. അമ്പതിനായിരം കൊല്ലം ദൈര്‍ഘ്യമുള്ള ഒരു ദിനത്തില്‍
\end{malayalam}}
\flushright{\begin{Arabic}
\quranayah[70][5]
\end{Arabic}}
\flushleft{\begin{malayalam}
അതിനാല്‍ ക്ഷമിക്കുക. മനോഹരമായ ക്ഷമ.
\end{malayalam}}
\flushright{\begin{Arabic}
\quranayah[70][6]
\end{Arabic}}
\flushleft{\begin{malayalam}
അവരത് അകലെയായാണ് കാണുന്നത്.
\end{malayalam}}
\flushright{\begin{Arabic}
\quranayah[70][7]
\end{Arabic}}
\flushleft{\begin{malayalam}
നാമോ അടുത്തായും കാണുന്നു.
\end{malayalam}}
\flushright{\begin{Arabic}
\quranayah[70][8]
\end{Arabic}}
\flushleft{\begin{malayalam}
അന്ന് ആകാശം ഉരുകിയ ലോഹം പോലെയാകും.
\end{malayalam}}
\flushright{\begin{Arabic}
\quranayah[70][9]
\end{Arabic}}
\flushleft{\begin{malayalam}
മലകള്‍ കടഞ്ഞെടുത്ത രോമം പോലെയും.
\end{malayalam}}
\flushright{\begin{Arabic}
\quranayah[70][10]
\end{Arabic}}
\flushleft{\begin{malayalam}
അന്ന് ഒരുറ്റവനും തന്റെ തോഴനെ തേടുകയില്ല.
\end{malayalam}}
\flushright{\begin{Arabic}
\quranayah[70][11]
\end{Arabic}}
\flushleft{\begin{malayalam}
അവരന്യോന്യം കാണുന്നുണ്ടാകും. അപ്പോള്‍ കുറ്റവാളി കൊതിച്ചുപോകും: അന്നാളിലെ ശിക്ഷയില്‍നിന്നൊഴിവാകാന്‍ മക്കളെ പണയം നല്‍കിയാലോ!
\end{malayalam}}
\flushright{\begin{Arabic}
\quranayah[70][12]
\end{Arabic}}
\flushleft{\begin{malayalam}
സഹധര്‍മിണിയെയും സഹോദരനെയും നല്‍കിയാലോ!
\end{malayalam}}
\flushright{\begin{Arabic}
\quranayah[70][13]
\end{Arabic}}
\flushleft{\begin{malayalam}
തനിക്ക് അഭയമേകിപ്പോന്ന കുടുംബത്തെയും.
\end{malayalam}}
\flushright{\begin{Arabic}
\quranayah[70][14]
\end{Arabic}}
\flushleft{\begin{malayalam}
ഭൂമിയിലുള്ള മറ്റെല്ലാറ്റിനെയും. അങ്ങനെ താന്‍ രക്ഷപ്പെട്ടിരുന്നെങ്കില്‍!
\end{malayalam}}
\flushright{\begin{Arabic}
\quranayah[70][15]
\end{Arabic}}
\flushleft{\begin{malayalam}
വേണ്ട! അത് കത്തിക്കാളുന്ന നരകത്തീയാണ്.
\end{malayalam}}
\flushright{\begin{Arabic}
\quranayah[70][16]
\end{Arabic}}
\flushleft{\begin{malayalam}
തൊലി ഉരിച്ചു കളയുന്ന തീ!
\end{malayalam}}
\flushright{\begin{Arabic}
\quranayah[70][17]
\end{Arabic}}
\flushleft{\begin{malayalam}
സത്യത്തോട് പുറം തിരിയുകയും പിന്തിരിഞ്ഞു പോവുകയും ചെയ്തവരെ അത് വിളിച്ചുവരുത്തും.
\end{malayalam}}
\flushright{\begin{Arabic}
\quranayah[70][18]
\end{Arabic}}
\flushleft{\begin{malayalam}
ധനം ശേഖരിച്ച് സൂക്ഷിച്ചുവെച്ചവരെയും.
\end{malayalam}}
\flushright{\begin{Arabic}
\quranayah[70][19]
\end{Arabic}}
\flushleft{\begin{malayalam}
മനുഷ്യന്‍ ക്ഷമ കെട്ടവനായാണ് സൃഷ്ടിക്കപ്പെട്ടത്.
\end{malayalam}}
\flushright{\begin{Arabic}
\quranayah[70][20]
\end{Arabic}}
\flushleft{\begin{malayalam}
വിപത്ത് വരുമ്പോള്‍ അവന്‍ വെപ്രാളം കാട്ടും.
\end{malayalam}}
\flushright{\begin{Arabic}
\quranayah[70][21]
\end{Arabic}}
\flushleft{\begin{malayalam}
നേട്ടം കിട്ടിയാലോ കെട്ടിപ്പൂട്ടിവെക്കും.
\end{malayalam}}
\flushright{\begin{Arabic}
\quranayah[70][22]
\end{Arabic}}
\flushleft{\begin{malayalam}
നമസ്കരിക്കുന്നവരൊഴികെ.
\end{malayalam}}
\flushright{\begin{Arabic}
\quranayah[70][23]
\end{Arabic}}
\flushleft{\begin{malayalam}
അവര്‍ നമസ്കാരത്തില്‍ നിഷ്ഠ പുലര്‍ത്തുന്നവരാണ്.
\end{malayalam}}
\flushright{\begin{Arabic}
\quranayah[70][24]
\end{Arabic}}
\flushleft{\begin{malayalam}
അവരുടെ ധനത്തില്‍ നിര്‍ണിതമായ അവകാശമുണ്ട് --
\end{malayalam}}
\flushright{\begin{Arabic}
\quranayah[70][25]
\end{Arabic}}
\flushleft{\begin{malayalam}
ചോദിച്ചെത്തുന്നവര്‍ക്കും പ്രാഥമികാവശ്യങ്ങള്‍ക്കു വകയില്ലാത്തവര്‍ക്കും.
\end{malayalam}}
\flushright{\begin{Arabic}
\quranayah[70][26]
\end{Arabic}}
\flushleft{\begin{malayalam}
വിധിദിനം സത്യമാണെന്ന് അംഗീകരിക്കുന്നവരാണവര്‍.
\end{malayalam}}
\flushright{\begin{Arabic}
\quranayah[70][27]
\end{Arabic}}
\flushleft{\begin{malayalam}
തങ്ങളുടെ നാഥന്റെ ശിക്ഷയെ പേടിക്കുന്നവരും.
\end{malayalam}}
\flushright{\begin{Arabic}
\quranayah[70][28]
\end{Arabic}}
\flushleft{\begin{malayalam}
അവരുടെ നാഥന്റെ ശിക്ഷയെക്കുറിച്ച് നിര്‍ഭയരാകാവതല്ല; തീര്‍ച്ച.
\end{malayalam}}
\flushright{\begin{Arabic}
\quranayah[70][29]
\end{Arabic}}
\flushleft{\begin{malayalam}
അവര്‍ തങ്ങളുടെ സദാചാരനിഷ്ഠ സംരക്ഷിച്ചു പോരുന്നവരാണ്.
\end{malayalam}}
\flushright{\begin{Arabic}
\quranayah[70][30]
\end{Arabic}}
\flushleft{\begin{malayalam}
തങ്ങളുടെ ഭാര്യമാരിലോ അധീനതയിലുള്ളവരിലോ ഒഴികെ. ഇവരുമായി ബന്ധപ്പെടുന്നത് ആക്ഷേപാര്‍ഹമല്ല.
\end{malayalam}}
\flushright{\begin{Arabic}
\quranayah[70][31]
\end{Arabic}}
\flushleft{\begin{malayalam}
എന്നാല്‍ അതിനപ്പുറം ആഗ്രഹിക്കുന്നവരാരോ അവരത്രെ അതിക്രമികള്‍.
\end{malayalam}}
\flushright{\begin{Arabic}
\quranayah[70][32]
\end{Arabic}}
\flushleft{\begin{malayalam}
തങ്ങളുടെ വശമുള്ള സൂക്ഷിപ്പുസ്വത്തുക്കള്‍ സംരക്ഷിക്കുന്നവരും കരാര്‍ പാലിക്കുന്നവരുമാണവര്‍.
\end{malayalam}}
\flushright{\begin{Arabic}
\quranayah[70][33]
\end{Arabic}}
\flushleft{\begin{malayalam}
തങ്ങളുടെ സാക്ഷ്യങ്ങള്‍ സത്യസന്ധമായി പൂര്‍ത്തീകരിക്കുന്നവരും.
\end{malayalam}}
\flushright{\begin{Arabic}
\quranayah[70][34]
\end{Arabic}}
\flushleft{\begin{malayalam}
നമസ്കാരം നിഷ്ഠയോടെ നിര്‍വഹിക്കുന്നവരും.
\end{malayalam}}
\flushright{\begin{Arabic}
\quranayah[70][35]
\end{Arabic}}
\flushleft{\begin{malayalam}
അവര്‍ സ്വര്‍ഗത്തില്‍ അത്യധികം ആദരിക്കപ്പെടുന്നവരായിരിക്കും.
\end{malayalam}}
\flushright{\begin{Arabic}
\quranayah[70][36]
\end{Arabic}}
\flushleft{\begin{malayalam}
ഈ സത്യനിഷേധികള്‍ക്ക് എന്തുപറ്റി? നിന്റെ നേരെ പാഞ്ഞുവരികയാണല്ലോ അവര്‍.
\end{malayalam}}
\flushright{\begin{Arabic}
\quranayah[70][37]
\end{Arabic}}
\flushleft{\begin{malayalam}
ഇടത്തുനിന്നും വലത്തുനിന്നും കൂട്ടം കൂട്ടമായി.
\end{malayalam}}
\flushright{\begin{Arabic}
\quranayah[70][38]
\end{Arabic}}
\flushleft{\begin{malayalam}
അവരോരോരുത്തരും താന്‍ അനുഗൃഹീത സ്വര്‍ഗത്തില്‍ കടക്കുമെന്ന് കൊതിക്കുകയാണോ?
\end{malayalam}}
\flushright{\begin{Arabic}
\quranayah[70][39]
\end{Arabic}}
\flushleft{\begin{malayalam}
ഒരിക്കലുമില്ല! അവര്‍ക്കുതന്നെ നന്നായറിയാവുന്ന വസ്തുവില്‍ നിന്നാണ് നാമവരെ പടച്ചത്.
\end{malayalam}}
\flushright{\begin{Arabic}
\quranayah[70][40]
\end{Arabic}}
\flushleft{\begin{malayalam}
വേണ്ട, ഉദയാസ്തമയ സ്ഥാനങ്ങളുടെ നാഥന്റെ പേരില്‍ ഞാനിതാ സത്യം ചെയ്യുന്നു. നിസ്സംശയം നാം കഴിവുറ്റവനാണ്.
\end{malayalam}}
\flushright{\begin{Arabic}
\quranayah[70][41]
\end{Arabic}}
\flushleft{\begin{malayalam}
അവരുടെ സ്ഥാനത്ത് അവരെക്കാള്‍ ഉത്തമമായ ജനതയെ കൊണ്ടുവരാന്‍ ; നമ്മെ ആരും മറികടക്കുകയില്ല.
\end{malayalam}}
\flushright{\begin{Arabic}
\quranayah[70][42]
\end{Arabic}}
\flushleft{\begin{malayalam}
അതിനാല്‍ അവരെ വിട്ടേക്കുക. അവര്‍ക്കു താക്കീതു നല്‍കപ്പെട്ട ദിനം വരുംവരെ അവര്‍ തങ്ങളുടെ തോന്നിവാസങ്ങളിലും ദുര്‍വൃത്തികളിലും മുഴുകി കഴിയട്ടെ.
\end{malayalam}}
\flushright{\begin{Arabic}
\quranayah[70][43]
\end{Arabic}}
\flushleft{\begin{malayalam}
അവര്‍ തങ്ങളുടെ ശവകുടീരങ്ങളില്‍ നിന്ന് പുറപ്പെട്ട് ഓടിയണയുന്ന ദിനമാണത്. തങ്ങളുടെ ലക്ഷ്യസ്ഥാനമായ നാട്ടക്കുറിയിലേക്ക് ഓടിയൊഴുകുന്ന പോലെ.
\end{malayalam}}
\flushright{\begin{Arabic}
\quranayah[70][44]
\end{Arabic}}
\flushleft{\begin{malayalam}
കണ്ണുകള്‍ താണുപോയ അവസ്ഥയിലായിരിക്കും അന്നവര്‍. അപമാനം അവരെ ആവരണം ചെയ്തിരിക്കും. അവര്‍ക്ക് മുന്നറിയിപ്പു നല്‍കപ്പെട്ടിരുന്ന ദിനം അതത്രെ.
\end{malayalam}}
\chapter{\textmalayalam{നൂഹ്}}
\begin{Arabic}
\Huge{\centerline{\basmalah}}\end{Arabic}
\flushright{\begin{Arabic}
\quranayah[71][1]
\end{Arabic}}
\flushleft{\begin{malayalam}
നൂഹിനെ നാം തന്റെ ജനതയിലേക്ക് ദൂതനായി നിയോഗിച്ചു. “നോവേറിയ ശിക്ഷ വന്നെത്തും മുമ്പെ നിന്റെ ജനത്തിന് മുന്നറിയിപ്പ് നല്‍കുക”യെന്ന നിര്‍ദേശത്തോടെ.
\end{malayalam}}
\flushright{\begin{Arabic}
\quranayah[71][2]
\end{Arabic}}
\flushleft{\begin{malayalam}
അദ്ദേഹം പറഞ്ഞു: "എന്റെ ജനമേ, ഞാന്‍ നിങ്ങള്‍ക്ക് വ്യക്തമായ മുന്നറിയിപ്പു നല്‍കുന്നവനാണ്.
\end{malayalam}}
\flushright{\begin{Arabic}
\quranayah[71][3]
\end{Arabic}}
\flushleft{\begin{malayalam}
"അതിനാല്‍ നിങ്ങള്‍ അല്ലാഹുവിന് വഴിപ്പെടുക. അവനെ സൂക്ഷിക്കുക. എന്നെ അനുസരിക്കുക.
\end{malayalam}}
\flushright{\begin{Arabic}
\quranayah[71][4]
\end{Arabic}}
\flushleft{\begin{malayalam}
"എങ്കില്‍ അല്ലാഹു നിങ്ങള്‍ക്ക് നിങ്ങളുടെ പാപങ്ങള്‍ പൊറുത്തുതരും. ഒരു നിശ്ചിത അവധിവരെ നിങ്ങള്‍ക്ക് ജീവിക്കാനവസരം നല്‍കും. അല്ലാഹുവിന്റെ അവധി ആഗതമായാല്‍ പിന്നെയൊട്ടും പിന്തിക്കുകയില്ല; തീര്‍ച്ച. നിങ്ങള്‍ അതറിഞ്ഞിരുന്നെങ്കില്‍.”
\end{malayalam}}
\flushright{\begin{Arabic}
\quranayah[71][5]
\end{Arabic}}
\flushleft{\begin{malayalam}
നൂഹ് പറഞ്ഞു: "നാഥാ, രാവും പകലും ഞാനെന്റെ ജനത്തെ വിളിച്ചും.
\end{malayalam}}
\flushright{\begin{Arabic}
\quranayah[71][6]
\end{Arabic}}
\flushleft{\begin{malayalam}
"എന്നാല്‍ എന്റെ ക്ഷണം അവരെ കൂടുതല്‍ അകറ്റുകയാണുണ്ടായത്.
\end{malayalam}}
\flushright{\begin{Arabic}
\quranayah[71][7]
\end{Arabic}}
\flushleft{\begin{malayalam}
"നീ അവര്‍ക്ക് മാപ്പേകാനായി ഞാന്‍ അവരെ വിളിച്ചപ്പോഴൊക്കെയും അവര്‍ കാതില്‍ വിരല്‍ തിരുകുകയും വസ്ത്രം കൊണ്ട് മൂടുകയുമായിരുന്നു. അവര്‍ തങ്ങളുടെ ദുശ്ശാഠ്യത്തിലുറച്ചുനിന്നു. അങ്ങേയറ്റം അഹങ്കരിക്കുകയും ചെയ്തു.
\end{malayalam}}
\flushright{\begin{Arabic}
\quranayah[71][8]
\end{Arabic}}
\flushleft{\begin{malayalam}
"വീണ്ടും ഞാനവരെ ഉറക്കെ വിളിച്ചു.
\end{malayalam}}
\flushright{\begin{Arabic}
\quranayah[71][9]
\end{Arabic}}
\flushleft{\begin{malayalam}
"പിന്നെ പരസ്യമായും വളരെ രഹസ്യമായും ഉദ്ബോധനം നല്‍കി.
\end{malayalam}}
\flushright{\begin{Arabic}
\quranayah[71][10]
\end{Arabic}}
\flushleft{\begin{malayalam}
"ഞാന്‍ ആവശ്യപ്പെട്ടു: നിങ്ങള്‍ നിങ്ങളുടെ നാഥനോട് മാപ്പിനപേക്ഷിക്കുക. അവന്‍ ഏറെ പൊറുക്കുന്നവനാണ്.
\end{malayalam}}
\flushright{\begin{Arabic}
\quranayah[71][11]
\end{Arabic}}
\flushleft{\begin{malayalam}
"അവന്‍ നിങ്ങള്‍ക്ക് ധാരാളം മഴ വീഴ്ത്തിത്തരും.
\end{malayalam}}
\flushright{\begin{Arabic}
\quranayah[71][12]
\end{Arabic}}
\flushleft{\begin{malayalam}
"സമ്പത്തും സന്താനങ്ങളും കൊണ്ട് നിങ്ങളെ സഹായിക്കും. നിങ്ങള്‍ക്ക് തോട്ടങ്ങളുണ്ടാക്കിത്തരും. അരുവികളൊരുക്കിത്തരും.”
\end{malayalam}}
\flushright{\begin{Arabic}
\quranayah[71][13]
\end{Arabic}}
\flushleft{\begin{malayalam}
നിങ്ങള്‍ക്കെന്തുപറ്റി? നിങ്ങള്‍ക്ക് അല്ലാഹുവിന്റെ മഹത്വം ഒട്ടും അംഗീകരിക്കാനാവുന്നില്ലല്ലോ.
\end{malayalam}}
\flushright{\begin{Arabic}
\quranayah[71][14]
\end{Arabic}}
\flushleft{\begin{malayalam}
നിങ്ങളെ വിവിധ ഘട്ടങ്ങളിലൂടെ സൃഷ്ടിച്ചു വളര്‍ത്തിയത് അവനാണ്.
\end{malayalam}}
\flushright{\begin{Arabic}
\quranayah[71][15]
\end{Arabic}}
\flushleft{\begin{malayalam}
അല്ലാഹു ഒന്നിനുമീതെ മറ്റൊന്നായി ഏഴ് ആകാശങ്ങളെ സൃഷ്ടിച്ചതെങ്ങനെയെന്ന് നിങ്ങള്‍ കാണുന്നില്ലേ?
\end{malayalam}}
\flushright{\begin{Arabic}
\quranayah[71][16]
\end{Arabic}}
\flushleft{\begin{malayalam}
അതില്‍ വെളിച്ചമായി ചന്ദ്രനെ ഉണ്ടാക്കി. വിളക്കായി സൂര്യനെയും.
\end{malayalam}}
\flushright{\begin{Arabic}
\quranayah[71][17]
\end{Arabic}}
\flushleft{\begin{malayalam}
അല്ലാഹു നിങ്ങളെ ഭൂമിയില്‍നിന്ന് മുളപ്പിച്ചു വളര്‍ത്തി.
\end{malayalam}}
\flushright{\begin{Arabic}
\quranayah[71][18]
\end{Arabic}}
\flushleft{\begin{malayalam}
പിന്നെ അവന്‍ നിങ്ങളെ അതിലേക്കുതന്നെ മടക്കുന്നു. വീണ്ടും നിങ്ങളെ പുനരുജ്ജീവിപ്പിച്ച് പുറപ്പെടുവിക്കുന്നതാണ്.
\end{malayalam}}
\flushright{\begin{Arabic}
\quranayah[71][19]
\end{Arabic}}
\flushleft{\begin{malayalam}
അല്ലാഹു നിങ്ങള്‍ക്കായി ഭൂമിയെ വിരിപ്പാക്കിയിരിക്കുന്നു.
\end{malayalam}}
\flushright{\begin{Arabic}
\quranayah[71][20]
\end{Arabic}}
\flushleft{\begin{malayalam}
നിങ്ങള്‍ അതിലെ വിശാലമായ വഴികളിലൂടെ സഞ്ചരിക്കാന്‍.
\end{malayalam}}
\flushright{\begin{Arabic}
\quranayah[71][21]
\end{Arabic}}
\flushleft{\begin{malayalam}
നൂഹ് പറഞ്ഞു: "എന്റെ നാഥാ! ഇവരെന്നെ ധിക്കരിച്ചു. എന്നിട്ടവര്‍ പിന്‍പറ്റിയതോ തന്റെ സ്വത്തും സന്താനവും വഴി നഷ്ടമല്ലാതൊന്നും വര്‍ധിപ്പിക്കാത്തവനെയും.
\end{malayalam}}
\flushright{\begin{Arabic}
\quranayah[71][22]
\end{Arabic}}
\flushleft{\begin{malayalam}
"അവര്‍ കൊടിയ കുതന്ത്രമാണ് കാണിച്ചത്.
\end{malayalam}}
\flushright{\begin{Arabic}
\quranayah[71][23]
\end{Arabic}}
\flushleft{\begin{malayalam}
"അവര്‍ ജനത്തോടു പറഞ്ഞു: “നിങ്ങള്‍ നിങ്ങളുടെ ദൈവങ്ങളെ വെടിയരുത്. വദ്ദിനെയും സുവാഇനെയും യഗൂസിനെയും യഊഖിനെയും നസ്റിനെയും കൈവിടരുത്.”
\end{malayalam}}
\flushright{\begin{Arabic}
\quranayah[71][24]
\end{Arabic}}
\flushleft{\begin{malayalam}
"അവരിങ്ങനെ വളരെപ്പേരെ വഴിപിഴപ്പിച്ചു. ഈ അതിക്രമകാരികള്‍ക്ക് നീ വഴികേടല്ലാതൊന്നും വര്‍ധിപ്പിച്ചുകൊടുക്കരുതേ.”
\end{malayalam}}
\flushright{\begin{Arabic}
\quranayah[71][25]
\end{Arabic}}
\flushleft{\begin{malayalam}
തങ്ങളുടെ തന്നെ തെറ്റിനാല്‍ അവരെ മുക്കിക്കൊന്നു. പിന്നെ അവര്‍ നരകത്തില്‍ പ്രവേശിപ്പിക്കപ്പെടുകയും ചെയ്തു. അപ്പോള്‍ അല്ലാഹുവെ കൂടാതെ ഒരു സഹായിയെയും അവര്‍ക്കവിടെ കണ്ടുകിട്ടിയില്ല.
\end{malayalam}}
\flushright{\begin{Arabic}
\quranayah[71][26]
\end{Arabic}}
\flushleft{\begin{malayalam}
നൂഹ് പ്രാര്‍ഥിച്ചു: "നാഥാ! ഈ സത്യനിഷേധികളിലൊരുത്തനെയും ഈ ഭൂമുഖത്ത് ബാക്കിവെക്കരുതേ!
\end{malayalam}}
\flushright{\begin{Arabic}
\quranayah[71][27]
\end{Arabic}}
\flushleft{\begin{malayalam}
"നീ അവരെ വെറുതെ വിട്ടാല്‍ ഇനിയുമവര്‍ നിന്റെ ദാസന്മാരെ വഴിപിഴപ്പിക്കും. തെമ്മാടികള്‍ക്കും നിഷേധികള്‍ക്കുമല്ലാതെ അവര്‍ ജന്മം നല്‍കുകയുമില്ല.
\end{malayalam}}
\flushright{\begin{Arabic}
\quranayah[71][28]
\end{Arabic}}
\flushleft{\begin{malayalam}
"നാഥാ! എനിക്കും എന്റെ മാതാപിതാക്കള്‍ക്കും വിശ്വാസികളായി എന്റെ ഭവനത്തില്‍ കടന്നുവരുന്നവര്‍ക്കും സത്യവിശ്വാസികള്‍ക്കും വിശ്വാസിനികള്‍ക്കും നീ പൊറുത്തു തരേണമേ! അതിക്രമികള്‍ക്ക് നാശമല്ലാതൊന്നും വര്‍ധിപ്പിച്ചുകൊടുക്കരുതേ!”
\end{malayalam}}
\chapter{\textmalayalam{ജിന്ന് ( ജിന്ന് വര്‍ഗ്ഗം )}}
\begin{Arabic}
\Huge{\centerline{\basmalah}}\end{Arabic}
\flushright{\begin{Arabic}
\quranayah[72][1]
\end{Arabic}}
\flushleft{\begin{malayalam}
പറയുക: ജിന്നുകളില്‍ കുറേ പേര്‍ ഖുര്‍ആന്‍ കേട്ടുവെന്ന് എനിക്ക് ദിവ്യബോധനം ലഭിച്ചിരിക്കുന്നു. അങ്ങനെ അവര്‍ പറഞ്ഞു: "വിസ്മയകരമായ ഒരു ഖുര്‍ആന്‍ ഞങ്ങള്‍ കേട്ടിരിക്കുന്നു.
\end{malayalam}}
\flushright{\begin{Arabic}
\quranayah[72][2]
\end{Arabic}}
\flushleft{\begin{malayalam}
"അത് നേര്‍വഴിയിലേക്ക് നയിക്കുന്നു. അതിനാല്‍ ഞങ്ങളതില്‍ വിശ്വസിച്ചിരിക്കുന്നു. ഞങ്ങള്‍ ഞങ്ങളുടെ നാഥനില്‍ ആരെയും പങ്കുചേര്‍ക്കുകയില്ല.
\end{malayalam}}
\flushright{\begin{Arabic}
\quranayah[72][3]
\end{Arabic}}
\flushleft{\begin{malayalam}
"നമ്മുടെ നാഥന്റെ മഹത്വം അത്യുന്നതമത്രെ. അവന്‍ സഖിയെയോ സന്താനത്തെയോ സ്വീകരിച്ചിട്ടില്ല.
\end{malayalam}}
\flushright{\begin{Arabic}
\quranayah[72][4]
\end{Arabic}}
\flushleft{\begin{malayalam}
"ഞങ്ങളുടെ കൂട്ടത്തിലെ വിവരം കെട്ടവര്‍ അല്ലാഹുവെക്കുറിച്ച് കള്ളം പറയാറുണ്ടായിരുന്നു.
\end{malayalam}}
\flushright{\begin{Arabic}
\quranayah[72][5]
\end{Arabic}}
\flushleft{\begin{malayalam}
"മനുഷ്യരും ജിന്നുകളും അല്ലാഹുവെക്കുറിച്ച് ഒരിക്കലും കള്ളം പറയില്ലെന്നാണ് ഞങ്ങള്‍ കരുതിയിരുന്നത്.
\end{malayalam}}
\flushright{\begin{Arabic}
\quranayah[72][6]
\end{Arabic}}
\flushleft{\begin{malayalam}
"മനുഷ്യരില്‍ ചിലര്‍ ജിന്നുകളില്‍ ചിലരോട് ശരണം തേടാറുണ്ടായിരുന്നു. അതവരില്‍ 1 അഹങ്കാരം വളര്‍ത്തി.
\end{malayalam}}
\flushright{\begin{Arabic}
\quranayah[72][7]
\end{Arabic}}
\flushleft{\begin{malayalam}
"അല്ലാഹു ആരെയും പ്രവാചകനായി നിയോഗിക്കില്ലെന്ന് നിങ്ങള്‍ കരുതിയ പോലെ അവരും കരുതിയിരുന്നു.
\end{malayalam}}
\flushright{\begin{Arabic}
\quranayah[72][8]
\end{Arabic}}
\flushleft{\begin{malayalam}
"ഞങ്ങള്‍ ആകാശത്തെ തൊട്ടുനോക്കി. അപ്പോഴത് കരുത്തരായ കാവല്‍ക്കാരാലും തീജ്വാലകളാലും നിറഞ്ഞുനില്‍ക്കുന്നതായി ഞങ്ങള്‍ക്കനുഭവപ്പെട്ടു.
\end{malayalam}}
\flushright{\begin{Arabic}
\quranayah[72][9]
\end{Arabic}}
\flushleft{\begin{malayalam}
"ആകാശത്തിലെ ചില ഇരിപ്പിടങ്ങളില്‍ മുമ്പ് ഞങ്ങള്‍ കേള്‍ക്കാന്‍ ഇരിക്കാറുണ്ടായിരുന്നു. എന്നാല്‍ ഇപ്പോള്‍ ആരെങ്കിലും കട്ടുകേള്‍ക്കുകയാണെങ്കില്‍, തന്നെ ശ്രദ്ധിച്ചു കൊണ്ടിരിക്കുന്ന തീജ്വാലയെ അവന്ന് നേരിടേണ്ടിവരും.
\end{malayalam}}
\flushright{\begin{Arabic}
\quranayah[72][10]
\end{Arabic}}
\flushleft{\begin{malayalam}
"ഭൂമിയിലുള്ളവര്‍ക്ക് നാശം വരുത്താനാണോ ഉദ്ദേശിച്ചത്, അതല്ല അവരുടെ നാഥന്‍ അവരെ നേര്‍വഴിയിലാക്കാനാണോ ഇച്ഛിച്ചതെന്ന് ഞങ്ങള്‍ക്കറിയില്ല.
\end{malayalam}}
\flushright{\begin{Arabic}
\quranayah[72][11]
\end{Arabic}}
\flushleft{\begin{malayalam}
"ഞങ്ങളോ, ഞങ്ങളില്‍ സച്ചരിതരുണ്ട്. അല്ലാത്തവരും ഞങ്ങളിലുണ്ട്. ഞങ്ങള്‍ ഭിന്നമാര്‍ഗക്കാരാണ്.
\end{malayalam}}
\flushright{\begin{Arabic}
\quranayah[72][12]
\end{Arabic}}
\flushleft{\begin{malayalam}
"ഭൂമിയില്‍ വെച്ച് അല്ലാഹുവെ പരാജയപ്പെടുത്താനോ, ഓടിപ്പോയി അവനെ തോല്‍പിക്കാനോ സാധ്യമല്ലെന്ന് ഞങ്ങള്‍ മനസ്സിലാക്കിയിരിക്കുന്നു.
\end{malayalam}}
\flushright{\begin{Arabic}
\quranayah[72][13]
\end{Arabic}}
\flushleft{\begin{malayalam}
"സന്മാര്‍ഗം കേട്ടപ്പോള്‍തന്നെ ഞങ്ങളതില്‍ വിശ്വസിച്ചു. തന്റെ നാഥനില്‍ വിശ്വസിക്കുന്നവനാരോ, അവന് ഒരുവിധ നഷ്ടമോ പീഡനമോ ഉണ്ടാവുമെന്ന് ഭയപ്പെടേണ്ടതില്ല.
\end{malayalam}}
\flushright{\begin{Arabic}
\quranayah[72][14]
\end{Arabic}}
\flushleft{\begin{malayalam}
"ഞങ്ങളില്‍ വഴിപ്പെട്ട് ജീവിക്കുന്നവരുണ്ട്. വഴിവിട്ട് ജീവിക്കുന്നവരുമുണ്ട്. ആര്‍ കീഴ്പ്പെട്ട് ജീവിക്കുന്നുവോ അവര്‍ നേര്‍വഴി ഉറപ്പാക്കിയിരിക്കുന്നു.
\end{malayalam}}
\flushright{\begin{Arabic}
\quranayah[72][15]
\end{Arabic}}
\flushleft{\begin{malayalam}
"വഴിവിട്ട് ജീവിക്കുന്നവരോ അവര്‍ നരകത്തീയിലെ വിറകായിത്തീരും.”
\end{malayalam}}
\flushright{\begin{Arabic}
\quranayah[72][16]
\end{Arabic}}
\flushleft{\begin{malayalam}
അവര്‍ നേര്‍വഴിയില്‍ തന്നെ ഉറച്ചു നില്‍ക്കുകയാണെങ്കില്‍ നാം അവര്‍ക്ക് കുടിക്കാന്‍ ധാരാളമായി വെള്ളം നല്‍കും.
\end{malayalam}}
\flushright{\begin{Arabic}
\quranayah[72][17]
\end{Arabic}}
\flushleft{\begin{malayalam}
അതിലൂടെ നാം അവരെ പരീക്ഷിക്കാനാണത്. തന്റെ നാഥന്റെ ഉദ്ബോധനത്തെ നിരാകരിച്ച് ജീവിക്കുന്നവരെ അവന്‍ അതികഠിന ശിക്ഷയിലകപ്പെടുത്തും.
\end{malayalam}}
\flushright{\begin{Arabic}
\quranayah[72][18]
\end{Arabic}}
\flushleft{\begin{malayalam}
പള്ളികള്‍ അല്ലാഹുവിന്നുള്ളതാണ്. അതിനാല്‍ അല്ലാഹുവോടൊപ്പം മറ്റാരെയും വിളിച്ചു പ്രാര്‍ഥിക്കരുത്.
\end{malayalam}}
\flushright{\begin{Arabic}
\quranayah[72][19]
\end{Arabic}}
\flushleft{\begin{malayalam}
ദൈവദാസന്‍ ദൈവത്തോട് പ്രാര്‍ഥിക്കാനായി എഴുന്നേറ്റു നിന്നപ്പോള്‍ സത്യനിഷേധികള്‍ അയാള്‍ക്കു ചുറ്റും തടിച്ചുകൂടുമാറായി.
\end{malayalam}}
\flushright{\begin{Arabic}
\quranayah[72][20]
\end{Arabic}}
\flushleft{\begin{malayalam}
പറയുക: ഞാന്‍ എന്റെ നാഥനെ മാത്രമേ വിളിച്ചു പ്രാര്‍ഥിക്കുകയുള്ളൂ. ആരെയും അവന്റെ പങ്കാളിയാക്കുകയില്ല.
\end{malayalam}}
\flushright{\begin{Arabic}
\quranayah[72][21]
\end{Arabic}}
\flushleft{\begin{malayalam}
പറയുക: നിങ്ങള്‍ക്ക് എന്തെങ്കിലും ഗുണമോ ദോഷമോ വരുത്താന്‍ എനിക്കാവില്ല.
\end{malayalam}}
\flushright{\begin{Arabic}
\quranayah[72][22]
\end{Arabic}}
\flushleft{\begin{malayalam}
പറയുക: അല്ലാഹുവിന്റെ ശിക്ഷയില്‍നിന്ന് എന്നെ രക്ഷിക്കാന്‍ ആര്‍ക്കുമാവില്ല. അവനല്ലാതെ ഒരഭയസ്ഥാനവും ഞാന്‍ കാണുന്നില്ല.
\end{malayalam}}
\flushright{\begin{Arabic}
\quranayah[72][23]
\end{Arabic}}
\flushleft{\begin{malayalam}
അല്ലാഹുവില്‍നിന്നുള്ള വിധികളും അവന്റെ സന്ദേശവും എത്തിക്കുകയെന്നതല്ലാതെ ഒരു ദൌത്യവും എനിക്കില്ല. ആര്‍ അല്ലാഹുവെയും അവന്റെ ദൂതനെയും ധിക്കരിക്കുന്നുവോ തീര്‍ച്ചയായും അവന്നുള്ളത് നരകത്തീയാണ്. അവരതില്‍ നിത്യവാസികളായിരിക്കും.
\end{malayalam}}
\flushright{\begin{Arabic}
\quranayah[72][24]
\end{Arabic}}
\flushleft{\begin{malayalam}
ഈ ജനത്തിന് മുന്നറിയിപ്പ് നല്‍കിയ കാര്യം നേരില്‍ കാണുമ്പോള്‍ അവര്‍ക്ക് ബോധ്യമാകും: ആരുടെ സഹായിയാണ് ദുര്‍ബലനെന്നും ആരുടെ സംഘമാണ് എണ്ണത്തില്‍ കുറവെന്നും.
\end{malayalam}}
\flushright{\begin{Arabic}
\quranayah[72][25]
\end{Arabic}}
\flushleft{\begin{malayalam}
പറയുക: നിങ്ങള്‍ക്ക് താക്കീതു നല്‍കപ്പെട്ട ശിക്ഷ ആസന്നമാണോ അതല്ല എന്റെ നാഥന്‍ അതിനു നീണ്ട അവധി നിശ്ചയിക്കുന്നുണ്ടോ എന്നൊന്നും എനിക്കറിയില്ല.
\end{malayalam}}
\flushright{\begin{Arabic}
\quranayah[72][26]
\end{Arabic}}
\flushleft{\begin{malayalam}
അവന്‍ അഭൌതിക കാര്യം അറിയുന്നവനാണ്. എന്നാല്‍ അവന്‍ തന്റെ അഭൌതിക കാര്യങ്ങള്‍ ആര്‍ക്കും വെളിവാക്കിക്കൊടുക്കുകയില്ല.
\end{malayalam}}
\flushright{\begin{Arabic}
\quranayah[72][27]
\end{Arabic}}
\flushleft{\begin{malayalam}
അവന്‍ തൃപ്തിപ്പെട്ട് അംഗീകരിച്ച ദൂതന്നൊഴികെ. അദ്ദേഹത്തിന്റെ മുന്നിലും പിന്നിലും അവന്‍ കാവല്‍ക്കാരെ ഏര്‍പ്പെടുത്തുന്നു.
\end{malayalam}}
\flushright{\begin{Arabic}
\quranayah[72][28]
\end{Arabic}}
\flushleft{\begin{malayalam}
അവര്‍ തങ്ങളുടെ നാഥന്റെ സന്ദേശങ്ങള്‍ എത്തിച്ചുകൊടുത്തിരിക്കുന്നുവെന്ന് അവനറിയാനാണിത്. അവരുടെ വശമുള്ളതിനെപ്പറ്റി അവന്ന് നന്നായറിയാം. എല്ലാ വസ്തുക്കളുടെയും എണ്ണം അവന്‍ തിട്ടപ്പെടുത്തിയിരിക്കുന്നു.
\end{malayalam}}
\chapter{\textmalayalam{മുസമ്മില്‍ ( വസ്ത്രത്താല്‍ മൂടിയവന്‍ )}}
\begin{Arabic}
\Huge{\centerline{\basmalah}}\end{Arabic}
\flushright{\begin{Arabic}
\quranayah[73][1]
\end{Arabic}}
\flushleft{\begin{malayalam}
മൂടിപ്പുതച്ചവനേ,
\end{malayalam}}
\flushright{\begin{Arabic}
\quranayah[73][2]
\end{Arabic}}
\flushleft{\begin{malayalam}
രാത്രിയില്‍ എഴുന്നേറ്റ് നമസ്കരിക്കുക -കുറച്ചുനേരമൊഴികെ.
\end{malayalam}}
\flushright{\begin{Arabic}
\quranayah[73][3]
\end{Arabic}}
\flushleft{\begin{malayalam}
അതായത് രാവിന്റെ പാതി. അല്ലെങ്കില്‍ അതില്‍ അല്‍പം കുറക്കുക.
\end{malayalam}}
\flushright{\begin{Arabic}
\quranayah[73][4]
\end{Arabic}}
\flushleft{\begin{malayalam}
അല്ലെങ്കില്‍ അല്‍പം വര്‍ധിപ്പിക്കുക. ഖുര്‍ആന്‍ നിര്‍ത്തി നിര്‍ത്തി സാവധാനം ഓതുക.
\end{malayalam}}
\flushright{\begin{Arabic}
\quranayah[73][5]
\end{Arabic}}
\flushleft{\begin{malayalam}
നിനക്കു നാം ഭാരിച്ച വചനം അവതരിപ്പിക്കുന്നതാണ്.
\end{malayalam}}
\flushright{\begin{Arabic}
\quranayah[73][6]
\end{Arabic}}
\flushleft{\begin{malayalam}
രാത്രിയില്‍ ഉണര്‍ന്നെഴുന്നേറ്റുള്ള നമസ്കാരം ഏറെ ഹൃദയസാന്നിധ്യം ഉളവാക്കുന്നതാണ്. സംസാരം സത്യനിഷ്ഠമാക്കുന്നതും.
\end{malayalam}}
\flushright{\begin{Arabic}
\quranayah[73][7]
\end{Arabic}}
\flushleft{\begin{malayalam}
പകല്‍സമയത്ത് നിനക്ക് ദീര്‍ഘമായ ജോലിത്തിരക്കുണ്ടല്ലോ.
\end{malayalam}}
\flushright{\begin{Arabic}
\quranayah[73][8]
\end{Arabic}}
\flushleft{\begin{malayalam}
നിന്റെ നാഥന്റെ നാമം സ്മരിക്കുക. മറ്റെല്ലാറ്റില്‍നിന്നും വിട്ടൊഴിഞ്ഞ് അവനില്‍ മാത്രം മുഴുകുക.
\end{malayalam}}
\flushright{\begin{Arabic}
\quranayah[73][9]
\end{Arabic}}
\flushleft{\begin{malayalam}
അവന്‍ ഉദയാസ്തമയ സ്ഥലങ്ങളുടെ ഉടമയാണ്. അവനല്ലാതെ ദൈവമില്ല. അതിനാല്‍ അവനെ മാത്രം ഭരമേല്‍പിക്കുക.
\end{malayalam}}
\flushright{\begin{Arabic}
\quranayah[73][10]
\end{Arabic}}
\flushleft{\begin{malayalam}
സത്യനിഷേധികള്‍ പറയുന്നതൊക്കെ ക്ഷമിക്കുക. അവരില്‍ നിന്ന് മാന്യമായി വിട്ടകന്നു നില്‍ക്കുക.
\end{malayalam}}
\flushright{\begin{Arabic}
\quranayah[73][11]
\end{Arabic}}
\flushleft{\begin{malayalam}
സമ്പന്നരായ ഈ നിഷേധികളുടെ കൈകാര്യം എനിക്ക് വിട്ടേക്കുക. അവര്‍ക്ക് ഈ അവസ്ഥയില്‍ അല്പം കൂടി സമയം അനുവദിക്കുക.
\end{malayalam}}
\flushright{\begin{Arabic}
\quranayah[73][12]
\end{Arabic}}
\flushleft{\begin{malayalam}
തീര്‍ച്ചയായും നമ്മുടെ അടുക്കല്‍ കാല്‍ച്ചങ്ങലകളും കത്തിക്കാളുന്ന നരകത്തീയുമുണ്ട്.
\end{malayalam}}
\flushright{\begin{Arabic}
\quranayah[73][13]
\end{Arabic}}
\flushleft{\begin{malayalam}
ചങ്കില്‍ കുടുങ്ങുന്ന ആഹാരവും നോവേറിയ ശിക്ഷയും.
\end{malayalam}}
\flushright{\begin{Arabic}
\quranayah[73][14]
\end{Arabic}}
\flushleft{\begin{malayalam}
ഭൂമിയും മലകളും വിറകൊള്ളുകയും പര്‍വതങ്ങള്‍ മണല്‍ക്കൂനകള്‍പോലെ ചിതറിപ്പോവുകയും ചെയ്യുന്ന ദിനമാണത്.
\end{malayalam}}
\flushright{\begin{Arabic}
\quranayah[73][15]
\end{Arabic}}
\flushleft{\begin{malayalam}
ഉറപ്പായും നിങ്ങളിലേക്ക് നാം ഒരു ദൂതനെ നിയോഗിച്ചിരിക്കുന്നു- നിങ്ങള്‍ക്ക് സാക്ഷിയായി. ഫറവോന്റെ അടുത്തേക്ക് ദൂതനെ അയച്ചപോലെ.
\end{malayalam}}
\flushright{\begin{Arabic}
\quranayah[73][16]
\end{Arabic}}
\flushleft{\begin{malayalam}
ഫറവോന്‍ ആ ദൂതനെ ധിക്കരിച്ചു. അതിനാല്‍ അവനെ നാം ശക്തമായ ഒരു പിടുത്തം പിടിച്ചു.
\end{malayalam}}
\flushright{\begin{Arabic}
\quranayah[73][17]
\end{Arabic}}
\flushleft{\begin{malayalam}
നിങ്ങള്‍ സത്യത്തെ നിഷേധിക്കുകയാണെങ്കില്‍ കൊച്ചു കുട്ടികളെക്കൂടി നരച്ചവരാക്കുന്ന ആ ദിനത്തെ നിങ്ങള്‍ക്ക് എങ്ങനെ കരുതിയിരിക്കാനാവും?
\end{malayalam}}
\flushright{\begin{Arabic}
\quranayah[73][18]
\end{Arabic}}
\flushleft{\begin{malayalam}
ആകാശം പൊട്ടിപ്പിളരുന്ന ദിനമാണത്. അല്ലാഹുവിന്റെ വാഗ്ദാനം പൂര്‍ത്തീകരിക്കപ്പെടുകതന്നെ ചെയ്യും.
\end{malayalam}}
\flushright{\begin{Arabic}
\quranayah[73][19]
\end{Arabic}}
\flushleft{\begin{malayalam}
ഇത് ഒരുദ്ബോധനമാണ്. അതിനാല്‍ ഇഷ്ടമുള്ളവന്‍ തന്റെ നാഥങ്കലേക്കുള്ള മാര്‍ഗം അവലംബിച്ചു കൊള്ളട്ടെ.
\end{malayalam}}
\flushright{\begin{Arabic}
\quranayah[73][20]
\end{Arabic}}
\flushleft{\begin{malayalam}
നിന്റെ നാഥന്നറിയാം: നീയും നിന്റെ കൂടെയുള്ളവരിലൊരു സംഘവും രാവിന്റെ മിക്കവാറും മൂന്നില്‍ രണ്ടു ഭാഗവും ചിലപ്പോള്‍ പാതിഭാഗവും മറ്റു ചിലപ്പോള്‍ മൂന്നിലൊരു ഭാഗവും നിന്ന് നമസ്കരിക്കുന്നുണ്ട്. രാപ്പകലുകള്‍ കണക്കാക്കുന്നത് അല്ലാഹുവാണ്. നിങ്ങള്‍ക്കത് കൃത്യമായി കണക്കാക്കാന്‍ കഴിയില്ലെന്ന് അവന്നറിയാം. അതിനാല്‍ നിങ്ങള്‍ക്ക് ഇളവ് നല്‍കിയിരിക്കുന്നു. അതുകൊണ്ട് ഖുര്‍ആനില്‍നിന്ന് നിങ്ങള്‍ക്ക് കഴിയുംവിധം പാരായണം ചെയ്ത് നമസ്കാരം നിര്‍വഹിക്കുക. നിങ്ങളില്‍ ചിലര്‍ രോഗികളാണ്. വേറെ ചിലര്‍ അല്ലാഹുവിന്റെ അനുഗ്രഹമന്വേഷിച്ച് ഭൂമിയില്‍ സഞ്ചരിക്കുന്നവരാണ്. ഇനിയും ചിലര്‍ അല്ലാഹുവിന്റെ മാര്‍ഗത്തില്‍ പോരാടുന്നവരും. ഇത് അവന് നന്നായറിയാം. അതിനാല്‍ ഖുര്‍ആനില്‍നിന്ന് സൌകര്യപ്രദമായത് പാരായണം ചെയ്യുക. നമസ്കാരം നിഷ്ഠയോടെ നിര്‍വഹിക്കുക. സകാത്ത് നല്‍കുക. അല്ലാഹുവിന്ന് ഉത്തമമായ കടം കൊടുക്കുക. നിങ്ങള്‍ സ്വന്തത്തിനുവേണ്ടി മുന്‍കൂട്ടി ചെയ്യുന്ന നന്മകളൊക്കെയും അല്ലാഹുവിങ്കല്‍ ഏറെ ഗുണമുള്ളതായി നിങ്ങള്‍ക്കു കണ്ടെത്താം. മഹത്തായ പ്രതിഫലമുള്ളതായും. നിങ്ങള്‍ അല്ലാഹുവോട് മാപ്പപേക്ഷിക്കുക. തീര്‍ച്ചയായും അല്ലാഹു ഏറെ പൊറുക്കുന്നവനും ദയാപരനുമാണ്.
\end{malayalam}}
\chapter{\textmalayalam{മുദ്ദഥിര്‍ ( പുതച്ച് മൂടിയവന്‍ )}}
\begin{Arabic}
\Huge{\centerline{\basmalah}}\end{Arabic}
\flushright{\begin{Arabic}
\quranayah[74][1]
\end{Arabic}}
\flushleft{\begin{malayalam}
പുതച്ചു മൂടിയവനേ!
\end{malayalam}}
\flushright{\begin{Arabic}
\quranayah[74][2]
\end{Arabic}}
\flushleft{\begin{malayalam}
എഴുന്നേല്‍ക്കുക. ജനത്തിന് മുന്നറിയിപ്പ് നല്‍കുക.
\end{malayalam}}
\flushright{\begin{Arabic}
\quranayah[74][3]
\end{Arabic}}
\flushleft{\begin{malayalam}
നിന്റെ നാഥന്റെ മഹത്വം വാഴ്ത്തുക.
\end{malayalam}}
\flushright{\begin{Arabic}
\quranayah[74][4]
\end{Arabic}}
\flushleft{\begin{malayalam}
നിന്റെ വസ്ത്രങ്ങള്‍ വൃത്തിയാക്കുക.
\end{malayalam}}
\flushright{\begin{Arabic}
\quranayah[74][5]
\end{Arabic}}
\flushleft{\begin{malayalam}
അഴുക്കുകളില്‍നിന്ന് അകന്നു നില്‍ക്കുക.
\end{malayalam}}
\flushright{\begin{Arabic}
\quranayah[74][6]
\end{Arabic}}
\flushleft{\begin{malayalam}
കൂടുതല്‍ തിരിച്ചുകിട്ടാന്‍ കൊതിച്ച് നീ ഔദാര്യം കാണിക്കരുത്.
\end{malayalam}}
\flushright{\begin{Arabic}
\quranayah[74][7]
\end{Arabic}}
\flushleft{\begin{malayalam}
നിന്റെ നാഥന്നുവേണ്ടി ക്ഷമ പാലിക്കുക.
\end{malayalam}}
\flushright{\begin{Arabic}
\quranayah[74][8]
\end{Arabic}}
\flushleft{\begin{malayalam}
പിന്നെ കാഹളം ഊതപ്പെട്ടാല്‍.
\end{malayalam}}
\flushright{\begin{Arabic}
\quranayah[74][9]
\end{Arabic}}
\flushleft{\begin{malayalam}
അന്ന് ഏറെ പ്രയാസമേറിയ ദിനമായിരിക്കും.
\end{malayalam}}
\flushright{\begin{Arabic}
\quranayah[74][10]
\end{Arabic}}
\flushleft{\begin{malayalam}
സത്യനിഷേധികള്‍ക്ക് ഒട്ടും സുഖകരമല്ലാത്ത ദിവസം!
\end{malayalam}}
\flushright{\begin{Arabic}
\quranayah[74][11]
\end{Arabic}}
\flushleft{\begin{malayalam}
ഞാന്‍ തനിയെ സൃഷ്ടിച്ച ആ മനുഷ്യനെ എനിക്കിങ്ങു വിട്ടുതരിക.
\end{malayalam}}
\flushright{\begin{Arabic}
\quranayah[74][12]
\end{Arabic}}
\flushleft{\begin{malayalam}
നാമവന് ധാരാളം ധനം നല്‍കി.
\end{malayalam}}
\flushright{\begin{Arabic}
\quranayah[74][13]
\end{Arabic}}
\flushleft{\begin{malayalam}
എന്തിനും പോന്ന മക്കളെയും.
\end{malayalam}}
\flushright{\begin{Arabic}
\quranayah[74][14]
\end{Arabic}}
\flushleft{\begin{malayalam}
അവനാവശ്യമായ സൌകര്യങ്ങളെല്ലാം ഞാനൊരുക്കിക്കൊടുത്തു.
\end{malayalam}}
\flushright{\begin{Arabic}
\quranayah[74][15]
\end{Arabic}}
\flushleft{\begin{malayalam}
എന്നിട്ടും ഞാന്‍ ഇനിയും കൂടുതല്‍ കൊടുക്കണമെന്ന് അവന്‍ കൊതിക്കുന്നു.
\end{malayalam}}
\flushright{\begin{Arabic}
\quranayah[74][16]
\end{Arabic}}
\flushleft{\begin{malayalam}
ഇല്ല; അവന്‍ നമ്മുടെ വചനങ്ങളുടെ കടുത്ത ശത്രുവായിരിക്കുന്നു.
\end{malayalam}}
\flushright{\begin{Arabic}
\quranayah[74][17]
\end{Arabic}}
\flushleft{\begin{malayalam}
വൈകാതെ തന്നെ നാമവനെ ക്ളേശമേറിയ ഒരു കയറ്റം കയറാനിടവരുത്തും.
\end{malayalam}}
\flushright{\begin{Arabic}
\quranayah[74][18]
\end{Arabic}}
\flushleft{\begin{malayalam}
അവന്‍ ചിന്തിച്ചു. ചിലത് ചെയ്യാനുറച്ചു.
\end{malayalam}}
\flushright{\begin{Arabic}
\quranayah[74][19]
\end{Arabic}}
\flushleft{\begin{malayalam}
അതിനാലവന് ശാപം. എങ്ങനെ ചെയ്യാനാണവനുറച്ചത്?
\end{malayalam}}
\flushright{\begin{Arabic}
\quranayah[74][20]
\end{Arabic}}
\flushleft{\begin{malayalam}
വീണ്ടും അവനു നാശം! എങ്ങനെ പ്രവര്‍ത്തിക്കാനാണവന്‍ തീരുമാനിച്ചത്.
\end{malayalam}}
\flushright{\begin{Arabic}
\quranayah[74][21]
\end{Arabic}}
\flushleft{\begin{malayalam}
പിന്നെ അവനൊന്നു നോക്കി.
\end{malayalam}}
\flushright{\begin{Arabic}
\quranayah[74][22]
\end{Arabic}}
\flushleft{\begin{malayalam}
എന്നിട്ട് മുഖം കോട്ടി. നെറ്റി ചുളിച്ചു.
\end{malayalam}}
\flushright{\begin{Arabic}
\quranayah[74][23]
\end{Arabic}}
\flushleft{\begin{malayalam}
പിന്നെ പിന്തിരിയുകയും അഹങ്കരിക്കുകയും ചെയ്തു.
\end{malayalam}}
\flushright{\begin{Arabic}
\quranayah[74][24]
\end{Arabic}}
\flushleft{\begin{malayalam}
എന്നിട്ട് അവന്‍ പുലമ്പി: ഈ ഖുര്‍ആന്‍ പരമ്പരാഗതമായ മായാജാലമല്ലാതൊന്നുമല്ല.
\end{malayalam}}
\flushright{\begin{Arabic}
\quranayah[74][25]
\end{Arabic}}
\flushleft{\begin{malayalam}
ഇത് വെറും മനുഷ്യവചനം മാത്രം.
\end{malayalam}}
\flushright{\begin{Arabic}
\quranayah[74][26]
\end{Arabic}}
\flushleft{\begin{malayalam}
അടുത്തുതന്നെ നാമവനെ നരകത്തീയിലെരിയിക്കും.
\end{malayalam}}
\flushright{\begin{Arabic}
\quranayah[74][27]
\end{Arabic}}
\flushleft{\begin{malayalam}
നരകത്തീ എന്താണെന്ന് നിനക്കെന്തറിയാം?
\end{malayalam}}
\flushright{\begin{Arabic}
\quranayah[74][28]
\end{Arabic}}
\flushleft{\begin{malayalam}
അത് ഒന്നും ബാക്കിവെക്കുകയില്ല. ഒന്നിനെയും ഒഴിവാക്കുകയുമില്ല.
\end{malayalam}}
\flushright{\begin{Arabic}
\quranayah[74][29]
\end{Arabic}}
\flushleft{\begin{malayalam}
അത് തൊലി കരിച്ചുകളയും.
\end{malayalam}}
\flushright{\begin{Arabic}
\quranayah[74][30]
\end{Arabic}}
\flushleft{\begin{malayalam}
അതിന്റെ ചുമതലക്കാരായി പത്തൊമ്പത് പേരുണ്ട്.
\end{malayalam}}
\flushright{\begin{Arabic}
\quranayah[74][31]
\end{Arabic}}
\flushleft{\begin{malayalam}
നാം നരകത്തിന് ഇവ്വിധം ചുമതലക്കാരായി നിശ്ചയിച്ചത് മലക്കുകളെ മാത്രമാണ്. അവരുടെ എണ്ണം സത്യനിഷേധികള്‍ക്കുള്ള നമ്മുടെ ഒരു പരീക്ഷണം മാത്രമാണ്; വേദാവകാശികള്‍ക്ക് ദൃഢബോധ്യം വരാനും സത്യവിശ്വാസികള്‍ക്ക് വിശ്വാസം വര്‍ധിക്കാനുമാണിത്. വേദക്കാരും സത്യവിശ്വാസികളും സംശയത്തിലകപ്പെടാതിരിക്കാനും. അതോടൊപ്പം സത്യനിഷേധികളും രോഗബാധിതമായ മനസ്സിനുടമകളും, അല്ലാഹു ഇതുകൊണ്ട് എന്തൊരുപമയാണ് ഉദ്ദേശിച്ചത് എന്നു പറയാനുമാണ്. ഇവ്വിധം അല്ലാഹു താനിഛിക്കുന്നവരെ വഴികേടിലാക്കുന്നു. താനുദ്ദേശിക്കുന്നവരെ നേര്‍വഴിയിലാക്കുകയും ചെയ്യുന്നു. നിന്റെ നാഥന്റെ സൈന്യങ്ങളെ സംബന്ധിച്ച് അവനല്ലാതെ ആരുമറിയില്ല. ഇത് മനുഷ്യര്‍ക്ക് ഒരുദ്ബോധനമല്ലാതൊന്നുമല്ല.
\end{malayalam}}
\flushright{\begin{Arabic}
\quranayah[74][32]
\end{Arabic}}
\flushleft{\begin{malayalam}
നിസ്സംശയം, ചന്ദ്രനാണ് സത്യം.
\end{malayalam}}
\flushright{\begin{Arabic}
\quranayah[74][33]
\end{Arabic}}
\flushleft{\begin{malayalam}
രാത്രിയാണ് സത്യം- അത് പിന്നിടുമ്പോള്‍.
\end{malayalam}}
\flushright{\begin{Arabic}
\quranayah[74][34]
\end{Arabic}}
\flushleft{\begin{malayalam}
പ്രഭാതമാണ് സത്യം- അത് പ്രശോഭിതമാവുമ്പോള്‍.
\end{malayalam}}
\flushright{\begin{Arabic}
\quranayah[74][35]
\end{Arabic}}
\flushleft{\begin{malayalam}
നരകം ഗൌരവമുള്ള കാര്യങ്ങളിലൊന്നുതന്നെ; തീര്‍ച്ച.
\end{malayalam}}
\flushright{\begin{Arabic}
\quranayah[74][36]
\end{Arabic}}
\flushleft{\begin{malayalam}
മനുഷ്യര്‍ക്കൊരു താക്കീതും!
\end{malayalam}}
\flushright{\begin{Arabic}
\quranayah[74][37]
\end{Arabic}}
\flushleft{\begin{malayalam}
നിങ്ങളില്‍ മുന്നോട്ടുവരാനോ പിന്നോട്ടു പോകാനോ ആഗ്രഹിക്കുന്ന ഏവര്‍ക്കുമുള്ള താക്കീത്.
\end{malayalam}}
\flushright{\begin{Arabic}
\quranayah[74][38]
\end{Arabic}}
\flushleft{\begin{malayalam}
ഓരോ മനുഷ്യനും താന്‍ നേടിയതിന് ബാധ്യസ്ഥനാണ്.
\end{malayalam}}
\flushright{\begin{Arabic}
\quranayah[74][39]
\end{Arabic}}
\flushleft{\begin{malayalam}
വലതു കൈയില്‍ കര്‍മ്മപുസ്തകം കിട്ടുന്നവരൊഴികെ.
\end{malayalam}}
\flushright{\begin{Arabic}
\quranayah[74][40]
\end{Arabic}}
\flushleft{\begin{malayalam}
അവര്‍ സ്വര്‍ഗത്തോപ്പുകളിലായിരിക്കും. അവരന്വേഷിക്കും,
\end{malayalam}}
\flushright{\begin{Arabic}
\quranayah[74][41]
\end{Arabic}}
\flushleft{\begin{malayalam}
കുറ്റവാളികളോട്:
\end{malayalam}}
\flushright{\begin{Arabic}
\quranayah[74][42]
\end{Arabic}}
\flushleft{\begin{malayalam}
"നിങ്ങളെ നരകത്തിലെത്തിച്ചത് എന്താണ്?”
\end{malayalam}}
\flushright{\begin{Arabic}
\quranayah[74][43]
\end{Arabic}}
\flushleft{\begin{malayalam}
അവര്‍ പറയും: "ഞങ്ങള്‍ നമസ്കരിക്കുന്നവരായിരുന്നില്ല.
\end{malayalam}}
\flushright{\begin{Arabic}
\quranayah[74][44]
\end{Arabic}}
\flushleft{\begin{malayalam}
"അഗതികള്‍ക്ക് ആഹാരം കൊടുക്കുന്നവരുമായിരുന്നില്ല.
\end{malayalam}}
\flushright{\begin{Arabic}
\quranayah[74][45]
\end{Arabic}}
\flushleft{\begin{malayalam}
"പാഴ്മൊഴികളില്‍ മുഴുകിക്കഴിഞ്ഞവരോടൊപ്പം ഞങ്ങളും അതില്‍ വ്യാപൃതരായിരുന്നു.
\end{malayalam}}
\flushright{\begin{Arabic}
\quranayah[74][46]
\end{Arabic}}
\flushleft{\begin{malayalam}
"പ്രതിഫല നാളിനെ ഞങ്ങള്‍ നിഷേധിച്ചിരുന്നു.
\end{malayalam}}
\flushright{\begin{Arabic}
\quranayah[74][47]
\end{Arabic}}
\flushleft{\begin{malayalam}
"മരണം ഞങ്ങളില്‍ വന്നെത്തുംവരെ.”
\end{malayalam}}
\flushright{\begin{Arabic}
\quranayah[74][48]
\end{Arabic}}
\flushleft{\begin{malayalam}
അന്നേരം ശുപാര്‍ശകരുടെ ശുപാര്‍ശ അവര്‍ക്കൊട്ടും ഉപകരിക്കുകയില്ല.
\end{malayalam}}
\flushright{\begin{Arabic}
\quranayah[74][49]
\end{Arabic}}
\flushleft{\begin{malayalam}
എന്നിട്ടും അവര്‍ക്കെന്തുപറ്റി? അവര്‍ ഈ ഉദ്ബോധനത്തില്‍നിന്ന് തെന്നിമാറുകയാണ്.
\end{malayalam}}
\flushright{\begin{Arabic}
\quranayah[74][50]
\end{Arabic}}
\flushleft{\begin{malayalam}
വിറളിപിടിച്ച കഴുതകളെപ്പോലെയാണവര്‍ --
\end{malayalam}}
\flushright{\begin{Arabic}
\quranayah[74][51]
\end{Arabic}}
\flushleft{\begin{malayalam}
സിംഹത്തെ ഭയന്ന് വിരണ്ടോടുന്ന.
\end{malayalam}}
\flushright{\begin{Arabic}
\quranayah[74][52]
\end{Arabic}}
\flushleft{\begin{malayalam}
അല്ല; അവരിലോരോരുത്തരും ആഗ്രഹിക്കുന്നു: തനിക്ക് അല്ലാഹുവില്‍നിന്ന് തുറന്ന ഏടുകളുള്ള വേദപുസ്തകം ലഭിക്കണമെന്ന്.
\end{malayalam}}
\flushright{\begin{Arabic}
\quranayah[74][53]
\end{Arabic}}
\flushleft{\begin{malayalam}
ഒരിക്കലുമില്ല. അവര്‍ക്ക് പരലോകത്തെ പേടിയില്ല എന്നതാണ് സത്യം.
\end{malayalam}}
\flushright{\begin{Arabic}
\quranayah[74][54]
\end{Arabic}}
\flushleft{\begin{malayalam}
അറിയുക! ഉറപ്പായും ഇത് ഒരുദ്ബോധനമാണ്.
\end{malayalam}}
\flushright{\begin{Arabic}
\quranayah[74][55]
\end{Arabic}}
\flushleft{\begin{malayalam}
അതിനാല്‍ ഇഷ്ടമുള്ളവന്‍ ഇതോര്‍ക്കട്ടെ.
\end{malayalam}}
\flushright{\begin{Arabic}
\quranayah[74][56]
\end{Arabic}}
\flushleft{\begin{malayalam}
അല്ലാഹു ഇഛിക്കുന്നുവെങ്കിലല്ലാതെ അവരത് സ്വീകരിക്കുകയില്ല. അവനാകുന്നു ഭക്തിക്കര്‍ഹന്‍. പാപമോചനത്തിനുടമയും അവന്‍ തന്നെ.
\end{malayalam}}
\chapter{\textmalayalam{ഖിയാമ ( ഉയിര്‍ത്തെഴുന്നേല്‍പ്പ് )}}
\begin{Arabic}
\Huge{\centerline{\basmalah}}\end{Arabic}
\flushright{\begin{Arabic}
\quranayah[75][1]
\end{Arabic}}
\flushleft{\begin{malayalam}
ഉയിര്‍ത്തെഴുന്നേല്‍പ് നാളുകൊണ്ട് ഞാന്‍ സത്യം ചെയ്യുന്നു.
\end{malayalam}}
\flushright{\begin{Arabic}
\quranayah[75][2]
\end{Arabic}}
\flushleft{\begin{malayalam}
കുറ്റപ്പെടുത്തുന്ന മനസ്സാക്ഷിയെ ക്കൊണ്ടും ഞാന്‍ സത്യം ചെയ്യുന്നു.
\end{malayalam}}
\flushright{\begin{Arabic}
\quranayah[75][3]
\end{Arabic}}
\flushleft{\begin{malayalam}
മനുഷ്യന്‍ വിചാരിക്കുന്നുവോ, നമുക്ക് അവന്റെ എല്ലുകളെ ഒരുമിച്ചുകൂട്ടാനാവില്ലെന്ന്?
\end{malayalam}}
\flushright{\begin{Arabic}
\quranayah[75][4]
\end{Arabic}}
\flushleft{\begin{malayalam}
എന്നാല്‍, നാം അവന്റെ വിരല്‍ത്തുമ്പുപോലും കൃത്യമായി നിര്‍മിക്കാന്‍ പോന്നവനാണ്.
\end{malayalam}}
\flushright{\begin{Arabic}
\quranayah[75][5]
\end{Arabic}}
\flushleft{\begin{malayalam}
എന്നിട്ടും മനുഷ്യന്‍ തന്റെ വരുംകാല ജീവിതത്തില്‍ ദുര്‍വൃത്തികള്‍ ചെയ്യാനുദ്ദേശിക്കുന്നു.
\end{malayalam}}
\flushright{\begin{Arabic}
\quranayah[75][6]
\end{Arabic}}
\flushleft{\begin{malayalam}
ഈ ഉയിര്‍ത്തെഴുന്നേല്പു ദിനം എന്നാണെന്ന് അവന്‍ ചോദിക്കുന്നു.
\end{malayalam}}
\flushright{\begin{Arabic}
\quranayah[75][7]
\end{Arabic}}
\flushleft{\begin{malayalam}
കണ്ണ് അഞ്ചിപ്പോവുകയും,
\end{malayalam}}
\flushright{\begin{Arabic}
\quranayah[75][8]
\end{Arabic}}
\flushleft{\begin{malayalam}
ചന്ദ്രന്‍ കെട്ടുപോവുകയും,
\end{malayalam}}
\flushright{\begin{Arabic}
\quranayah[75][9]
\end{Arabic}}
\flushleft{\begin{malayalam}
സൂര്യചന്ദ്രന്മാര്‍ ഒരുമിച്ചു കൂട്ടപ്പെടുകയും ചെയ്താല്‍.
\end{malayalam}}
\flushright{\begin{Arabic}
\quranayah[75][10]
\end{Arabic}}
\flushleft{\begin{malayalam}
അന്ന് ഈ മനുഷ്യന്‍ പറയും: എവിടേക്കാണ് ഓടി രക്ഷപ്പെടുകയെന്ന്.
\end{malayalam}}
\flushright{\begin{Arabic}
\quranayah[75][11]
\end{Arabic}}
\flushleft{\begin{malayalam}
ഇല്ല! ഒരു രക്ഷയുമില്ല.
\end{malayalam}}
\flushright{\begin{Arabic}
\quranayah[75][12]
\end{Arabic}}
\flushleft{\begin{malayalam}
അന്ന് നിന്റെ നാഥന്റെ മുന്നില്‍ തന്നെ ചെന്നു നില്‍ക്കേണ്ടിവരും.
\end{malayalam}}
\flushright{\begin{Arabic}
\quranayah[75][13]
\end{Arabic}}
\flushleft{\begin{malayalam}
അന്നാളില്‍ മനുഷ്യന്‍ താന്‍ ചെയ്യരുതാത്തത് ചെയ്തതിനെ സംബന്ധിച്ചും ചെയ്യേണ്ടത് ചെയ്യാത്തതിനെപ്പറ്റിയും അറിയുന്നു.
\end{malayalam}}
\flushright{\begin{Arabic}
\quranayah[75][14]
\end{Arabic}}
\flushleft{\begin{malayalam}
എന്നല്ല, അന്ന് മനുഷ്യന്‍ തനിക്കെതിരെ തന്നെ തെളിവായിത്തീരുന്നു.
\end{malayalam}}
\flushright{\begin{Arabic}
\quranayah[75][15]
\end{Arabic}}
\flushleft{\begin{malayalam}
അവന്‍ എന്തൊക്കെ ഒഴികഴിവു സമര്‍പ്പിച്ചാലും ശരി.
\end{malayalam}}
\flushright{\begin{Arabic}
\quranayah[75][16]
\end{Arabic}}
\flushleft{\begin{malayalam}
ഖുര്‍ആന്‍ പെട്ടെന്ന് മനഃപാഠമാക്കാനായി നീ നാവു പിടപ്പിക്കേണ്ടതില്ല.
\end{malayalam}}
\flushright{\begin{Arabic}
\quranayah[75][17]
\end{Arabic}}
\flushleft{\begin{malayalam}
അതിന്റെ സമാഹരണവും അത് ഓതിത്തരലും നമ്മുടെ ബാധ്യതയാണ്.
\end{malayalam}}
\flushright{\begin{Arabic}
\quranayah[75][18]
\end{Arabic}}
\flushleft{\begin{malayalam}
അങ്ങനെ നാം ഓതിത്തന്നാല്‍ ആ പാരായണത്തെ നീ പിന്തുടരുക.
\end{malayalam}}
\flushright{\begin{Arabic}
\quranayah[75][19]
\end{Arabic}}
\flushleft{\begin{malayalam}
തുടര്‍ന്നുള്ള അതിന്റെ വിശദീകരണവും നമ്മുടെ ചുമതല തന്നെ.
\end{malayalam}}
\flushright{\begin{Arabic}
\quranayah[75][20]
\end{Arabic}}
\flushleft{\begin{malayalam}
എന്നാല്‍ അങ്ങനെയല്ല; നിങ്ങള്‍ താല്‍ക്കാലിക നേട്ടം കൊതിക്കുന്നു.
\end{malayalam}}
\flushright{\begin{Arabic}
\quranayah[75][21]
\end{Arabic}}
\flushleft{\begin{malayalam}
പരലോകത്തെ അവഗണിക്കുകയും ചെയ്യുന്നു.
\end{malayalam}}
\flushright{\begin{Arabic}
\quranayah[75][22]
\end{Arabic}}
\flushleft{\begin{malayalam}
അന്ന് ചില മുഖങ്ങള്‍ പ്രസന്നങ്ങളായിരിക്കും.
\end{malayalam}}
\flushright{\begin{Arabic}
\quranayah[75][23]
\end{Arabic}}
\flushleft{\begin{malayalam}
തങ്ങളുടെ നാഥനെ നോക്കിക്കൊണ്ടിരിക്കുന്നവയും.
\end{malayalam}}
\flushright{\begin{Arabic}
\quranayah[75][24]
\end{Arabic}}
\flushleft{\begin{malayalam}
മറ്റു ചില മുഖങ്ങളന്ന് കറുത്തിരുണ്ടവയായിരിക്കും.
\end{malayalam}}
\flushright{\begin{Arabic}
\quranayah[75][25]
\end{Arabic}}
\flushleft{\begin{malayalam}
തങ്ങളുടെ മേല്‍ വന്‍ വിപത്ത് വന്നു വീഴാന്‍ പോവുകയാണെന്ന് അവ അറിയുന്നു.
\end{malayalam}}
\flushright{\begin{Arabic}
\quranayah[75][26]
\end{Arabic}}
\flushleft{\begin{malayalam}
മാത്രമല്ല; ജീവന്‍ തൊണ്ടക്കുഴിയിലെത്തുകയും,
\end{malayalam}}
\flushright{\begin{Arabic}
\quranayah[75][27]
\end{Arabic}}
\flushleft{\begin{malayalam}
മന്ത്രിക്കാനാരുണ്ട് എന്ന ചോദ്യമുയരുകയും,
\end{malayalam}}
\flushright{\begin{Arabic}
\quranayah[75][28]
\end{Arabic}}
\flushleft{\begin{malayalam}
ഇത് തന്റെ വേര്‍പാടാണെന്ന് മനസ്സിലാവുകയും,
\end{malayalam}}
\flushright{\begin{Arabic}
\quranayah[75][29]
\end{Arabic}}
\flushleft{\begin{malayalam}
കണങ്കാലുകള്‍ തമ്മില്‍ കൂടിച്ചേര്‍ന്ന് കെട്ടിപ്പിണയുകയും ചെയ്യുമ്പോള്‍.
\end{malayalam}}
\flushright{\begin{Arabic}
\quranayah[75][30]
\end{Arabic}}
\flushleft{\begin{malayalam}
അതാണ് നിന്റെ നാഥങ്കലേക്ക് നയിക്കപ്പെടുന്ന ദിനം.
\end{malayalam}}
\flushright{\begin{Arabic}
\quranayah[75][31]
\end{Arabic}}
\flushleft{\begin{malayalam}
എന്നാല്‍ അവന്‍ സത്യമംഗീകരിച്ചില്ല. നമസ്കരിച്ചതുമില്ല.
\end{malayalam}}
\flushright{\begin{Arabic}
\quranayah[75][32]
\end{Arabic}}
\flushleft{\begin{malayalam}
മറിച്ച്, നിഷേധിക്കുകയും പിന്തിരിയുകയും ചെയ്തു.
\end{malayalam}}
\flushright{\begin{Arabic}
\quranayah[75][33]
\end{Arabic}}
\flushleft{\begin{malayalam}
എന്നിട്ട് അഹങ്കാരത്തോടെ സ്വന്തക്കാരുടെ അടുത്തേക്ക് പോയി.
\end{malayalam}}
\flushright{\begin{Arabic}
\quranayah[75][34]
\end{Arabic}}
\flushleft{\begin{malayalam}
അതുതന്നെയാണ് നിനക്ക് ഏറ്റം പറ്റിയതും ഉചിതവും.
\end{malayalam}}
\flushright{\begin{Arabic}
\quranayah[75][35]
\end{Arabic}}
\flushleft{\begin{malayalam}
അതെ, അതുതന്നെയാണ് നിനക്കേറ്റം പറ്റിയതും ഉചിതവും.
\end{malayalam}}
\flushright{\begin{Arabic}
\quranayah[75][36]
\end{Arabic}}
\flushleft{\begin{malayalam}
മനുഷ്യന്‍ കരുതുന്നോ; അവനെ വെറുതെയങ്ങ് വിട്ടേക്കുമെന്ന്?
\end{malayalam}}
\flushright{\begin{Arabic}
\quranayah[75][37]
\end{Arabic}}
\flushleft{\begin{malayalam}
അവന്‍, തെറിച്ചു വീണ നിസ്സാരമായ ഒരിന്ദ്രിയകണം മാത്രമായിരുന്നില്ലേ?
\end{malayalam}}
\flushright{\begin{Arabic}
\quranayah[75][38]
\end{Arabic}}
\flushleft{\begin{malayalam}
പിന്നെയത് ഭ്രൂണമായി. അനന്തരം അല്ലാഹു അവനെ സൃഷ്ടിച്ചു സംവിധാനിച്ചു.
\end{malayalam}}
\flushright{\begin{Arabic}
\quranayah[75][39]
\end{Arabic}}
\flushleft{\begin{malayalam}
അങ്ങനെ അവനതില്‍ നിന്ന് ആണും പെണ്ണുമായി ഇണകളെ ഉണ്ടാക്കി.
\end{malayalam}}
\flushright{\begin{Arabic}
\quranayah[75][40]
\end{Arabic}}
\flushleft{\begin{malayalam}
അതൊക്കെ ചെയ്തവന്‍ മരിച്ചവരെ വീണ്ടും ജീവിപ്പിക്കാന്‍ പോന്നവനല്ലെന്നോ?
\end{malayalam}}
\chapter{\textmalayalam{ഇന്‍സാന്‍ ( മനുഷ്യന്‍ )}}
\begin{Arabic}
\Huge{\centerline{\basmalah}}\end{Arabic}
\flushright{\begin{Arabic}
\quranayah[76][1]
\end{Arabic}}
\flushleft{\begin{malayalam}
താന്‍ പറയത്തക്ക ഒന്നുമല്ലാതിരുന്ന ഒരു കാലഘട്ടം മനുഷ്യന് കഴിഞ്ഞുപോയിട്ടില്ലേ?
\end{malayalam}}
\flushright{\begin{Arabic}
\quranayah[76][2]
\end{Arabic}}
\flushleft{\begin{malayalam}
മനുഷ്യനെ നാം കൂടിക്കലര്‍ന്ന ദ്രവകണ ത്തില്‍നിന്ന് സൃഷ്ടിച്ചു; നമുക്ക് അവനെ പരീക്ഷിക്കാന്‍. അങ്ങനെ നാമവനെ കേള്‍വിയും കാഴ്ചയുമുള്ളവനാക്കി.
\end{malayalam}}
\flushright{\begin{Arabic}
\quranayah[76][3]
\end{Arabic}}
\flushleft{\begin{malayalam}
ഉറപ്പായും നാമവന് വഴികാണിച്ചു കൊടുത്തിരിക്കുന്നു. അവന് നന്ദിയുള്ളവനാകാം. നന്ദികെട്ടവനുമാകാം.
\end{malayalam}}
\flushright{\begin{Arabic}
\quranayah[76][4]
\end{Arabic}}
\flushleft{\begin{malayalam}
ഉറപ്പായും സത്യനിഷേധികള്‍ക്കു നാം ചങ്ങലകളും വിലങ്ങുകളും കത്തിക്കാളുന്ന നരകത്തീയും ഒരുക്കിവെച്ചിരിക്കുന്നു.
\end{malayalam}}
\flushright{\begin{Arabic}
\quranayah[76][5]
\end{Arabic}}
\flushleft{\begin{malayalam}
സുകര്‍മികളോ, തീര്‍ച്ചയായും അവര്‍ കര്‍പ്പൂരം ചേര്‍ത്ത പാനീയം നിറച്ച ചഷകത്തില്‍നിന്ന് പാനം ചെയ്യുന്നതാണ്.
\end{malayalam}}
\flushright{\begin{Arabic}
\quranayah[76][6]
\end{Arabic}}
\flushleft{\begin{malayalam}
അത് ഒരുറവയായിരിക്കും. ദൈവദാസന്മാര്‍ അതില്‍നിന്നാണ് കുടിക്കുക. അവരതിനെ ഇഷ്ടാനുസൃതം കൈവഴികളായി ഒഴുക്കിക്കൊണ്ടിരിക്കും.
\end{malayalam}}
\flushright{\begin{Arabic}
\quranayah[76][7]
\end{Arabic}}
\flushleft{\begin{malayalam}
അവര്‍; നേര്‍ച്ചകള്‍ നിറവേറ്റുന്നവരാണ്. ഒരു ഭീകരനാളിനെ പേടിക്കുന്നവരും. വിപത്ത് പടര്‍ന്നു പിടിക്കുന്ന നാളിനെ.
\end{malayalam}}
\flushright{\begin{Arabic}
\quranayah[76][8]
\end{Arabic}}
\flushleft{\begin{malayalam}
ആഹാരത്തോട് ഏറെ പ്രിയമുള്ളതോടൊപ്പം അവരത് അഗതിക്കും അനാഥക്കും ബന്ധിതന്നും നല്‍കുന്നു.
\end{malayalam}}
\flushright{\begin{Arabic}
\quranayah[76][9]
\end{Arabic}}
\flushleft{\begin{malayalam}
അവര്‍ പറയും: "അല്ലാഹുവിന്റെ പ്രീതിക്കുവേണ്ടി മാത്രമാണ് ഞങ്ങള്‍ നിങ്ങള്‍ക്ക് അന്നമേകുന്നത്. നിങ്ങളില്‍നിന്ന് എന്തെങ്കിലും പ്രതിഫലമോ നന്ദിയോ ഞങ്ങള്‍ പ്രതീക്ഷിക്കുന്നില്ല.
\end{malayalam}}
\flushright{\begin{Arabic}
\quranayah[76][10]
\end{Arabic}}
\flushleft{\begin{malayalam}
"ഞങ്ങളുടെ നാഥനില്‍ നിന്നുള്ള ദുസ്സഹവും ഭീകരവുമായ ഒരു നാളിനെ ഞങ്ങള്‍ ഭയപ്പെടുന്നു.”
\end{malayalam}}
\flushright{\begin{Arabic}
\quranayah[76][11]
\end{Arabic}}
\flushleft{\begin{malayalam}
അതിനാല്‍ ആ നാളിന്റെ നാശത്തില്‍നിന്ന് അല്ലാഹു അവരെ കാത്തുരക്ഷിച്ചു. അവര്‍ക്ക് സമാശ്വാസവും സന്തോഷവും സമ്മാനിച്ചു.
\end{malayalam}}
\flushright{\begin{Arabic}
\quranayah[76][12]
\end{Arabic}}
\flushleft{\begin{malayalam}
അവര്‍ ക്ഷമ പാലിച്ചതിനാല്‍ പ്രതിഫലമായി അവനവര്‍ക്ക് പൂന്തോപ്പുകളും പട്ടുടുപ്പുകളും പ്രദാനം ചെയ്തു.
\end{malayalam}}
\flushright{\begin{Arabic}
\quranayah[76][13]
\end{Arabic}}
\flushleft{\begin{malayalam}
അവരവിടെ ഉയര്‍ന്ന മഞ്ചങ്ങളില്‍ ചാരിയിരിക്കും. അത്യുഷ്ണമോ അതിശൈത്യമോ അനുഭവിക്കുകയില്ല.
\end{malayalam}}
\flushright{\begin{Arabic}
\quranayah[76][14]
\end{Arabic}}
\flushleft{\begin{malayalam}
സ്വര്‍ഗീയഛായ അവര്‍ക്കു മേല്‍ തണല്‍ വിരിക്കും. അതിലെ പഴങ്ങള്‍, പറിച്ചെടുക്കാന്‍ പാകത്തില്‍ അവരുടെ അധീനതയിലായിരിക്കും.
\end{malayalam}}
\flushright{\begin{Arabic}
\quranayah[76][15]
\end{Arabic}}
\flushleft{\begin{malayalam}
വെള്ളിപ്പാത്രങ്ങളും സ്ഫടികക്കോപ്പകളുമായി പരിചാരകര്‍ അവര്‍ക്കിടയില്‍ ചുറ്റിക്കറങ്ങിക്കൊണ്ടിരിക്കും.
\end{malayalam}}
\flushright{\begin{Arabic}
\quranayah[76][16]
\end{Arabic}}
\flushleft{\begin{malayalam}
ആ സ്ഫടികവും വെള്ളിമയമായിരിക്കും. പരിചാരകര്‍ അവ കണിശതയോടെ കണക്കാക്കിവെക്കുന്നു.
\end{malayalam}}
\flushright{\begin{Arabic}
\quranayah[76][17]
\end{Arabic}}
\flushleft{\begin{malayalam}
ഇഞ്ചിനീരിന്റെ ചേരുവ ചേര്‍ത്ത പാനീയം അവര്‍ക്കവിടെ കുടിക്കാന്‍ കിട്ടും.
\end{malayalam}}
\flushright{\begin{Arabic}
\quranayah[76][18]
\end{Arabic}}
\flushleft{\begin{malayalam}
അത് സ്വര്‍ഗത്തിലെ ഒരരുവിയില്‍ നിന്നുള്ളതാണ്. സല്‍സബീല്‍ എന്നാണ് അതിനെ വിളിക്കുക.
\end{malayalam}}
\flushright{\begin{Arabic}
\quranayah[76][19]
\end{Arabic}}
\flushleft{\begin{malayalam}
നിത്യബാല്യം നല്‍കപ്പെട്ട കുട്ടികള്‍ അവര്‍ക്കിടയിലൂടെ ചുറ്റിനടന്നുകൊണ്ടിരിക്കും. അവരെ കണ്ടാല്‍ ചിതറിത്തെറിച്ച മുത്തുകളായേ നിനക്ക് തോന്നൂ.
\end{malayalam}}
\flushright{\begin{Arabic}
\quranayah[76][20]
\end{Arabic}}
\flushleft{\begin{malayalam}
സ്വര്‍ഗത്തില്‍ മഹത്തായ അനുഗ്രഹങ്ങളും ഒരു മഹാസാമ്രാജ്യത്തിന്റെ അവസ്ഥയും നിനക്കു കാണാം.
\end{malayalam}}
\flushright{\begin{Arabic}
\quranayah[76][21]
\end{Arabic}}
\flushleft{\begin{malayalam}
അവിടെ നേര്‍ത്തുമിനുത്ത പച്ചവില്ലൂസും കട്ടിയുള്ള പട്ടുടവയുമാണ് അവരെ അണിയിക്കുക. അവര്‍ക്ക് അവിടെ വെള്ളിവളകള്‍ അണിയിക്കുന്നതാണ്. അവരുടെ നാഥന്‍ അവരെ പരിശുദ്ധമായ പാനീയം കുടിപ്പിക്കുകയും ചെയ്യും.
\end{malayalam}}
\flushright{\begin{Arabic}
\quranayah[76][22]
\end{Arabic}}
\flushleft{\begin{malayalam}
ഇതാണ് നിങ്ങള്‍ക്കുള്ള പ്രതിഫലം; തീര്‍ച്ച. നിങ്ങളുടെ പ്രവര്‍ത്തനങ്ങള്‍ നന്ദിപൂര്‍വം സ്വീകരിക്കപ്പെട്ടവയത്രെ.
\end{malayalam}}
\flushright{\begin{Arabic}
\quranayah[76][23]
\end{Arabic}}
\flushleft{\begin{malayalam}
ഉറപ്പായും ഈ ഖുര്‍ആന്‍ നിനക്ക് നാം അല്‍പാല്‍പമായി ഇറക്കിത്തന്നിരിക്കുന്നു.
\end{malayalam}}
\flushright{\begin{Arabic}
\quranayah[76][24]
\end{Arabic}}
\flushleft{\begin{malayalam}
അതിനാല്‍ നീ നിന്റെ നാഥന്റെ തീരുമാനത്തെ ക്ഷമയോടെ കാത്തിരിക്കുക. അവരിലെ കുറ്റവാളിയെയോ സത്യനിഷേധിയെയോ നീ അനുസരിക്കരുത്.
\end{malayalam}}
\flushright{\begin{Arabic}
\quranayah[76][25]
\end{Arabic}}
\flushleft{\begin{malayalam}
നിന്റെ നാഥന്റെ നാമം കാലത്തും വൈകുന്നേരവും സ്മരിക്കുക.
\end{malayalam}}
\flushright{\begin{Arabic}
\quranayah[76][26]
\end{Arabic}}
\flushleft{\begin{malayalam}
രാത്രിയില്‍ അവന്ന് സാഷ്ടാംഗം പ്രണമിക്കുക. നീണ്ട നിശാവേളകളില്‍ അവന്റെ മഹത്വം കീര്‍ത്തിക്കുക.
\end{malayalam}}
\flushright{\begin{Arabic}
\quranayah[76][27]
\end{Arabic}}
\flushleft{\begin{malayalam}
എന്നാല്‍ ഇക്കൂട്ടര്‍, ക്ഷണികമായ ഐഹിക നേട്ടമാണ് ഇഷ്ടപ്പെടുന്നത്. വരാനിരിക്കുന്ന ഭാരമേറിയ നാളിന്റെ കാര്യമവര്‍ പിറകോട്ട് തട്ടിമാറ്റുന്നു.
\end{malayalam}}
\flushright{\begin{Arabic}
\quranayah[76][28]
\end{Arabic}}
\flushleft{\begin{malayalam}
നാമാണ് അവരെ സൃഷ്ടിച്ചത്. അവരുടെ ശരീരഘടനക്ക് കരുത്തേകിയതും നാം തന്നെ. നാം ഇഛിക്കുന്നുവെങ്കില്‍ അവരുടെ രൂപം അപ്പാടെ മാറ്റിമറിക്കാവുന്നതാണ്.
\end{malayalam}}
\flushright{\begin{Arabic}
\quranayah[76][29]
\end{Arabic}}
\flushleft{\begin{malayalam}
തീര്‍ച്ചയായും ഇത് ഒരു ഉദ്ബോധനമാണ്. അതിനാല്‍ ഇഷ്ടമുള്ളവന്‍ തന്റെ നാഥങ്കലേക്കുള്ള മാര്‍ഗമവലംബിക്കട്ടെ.
\end{malayalam}}
\flushright{\begin{Arabic}
\quranayah[76][30]
\end{Arabic}}
\flushleft{\begin{malayalam}
അല്ലാഹു ഇഛിക്കുന്നുവെങ്കിലല്ലാതെ നിങ്ങള്‍ക്ക് അതിഷ്ടപ്പെടാനാവില്ല. നിശ്ചയമായും അല്ലാഹു സര്‍വജ്ഞനും യുക്തിമാനുമാണ്.
\end{malayalam}}
\flushright{\begin{Arabic}
\quranayah[76][31]
\end{Arabic}}
\flushleft{\begin{malayalam}
താനിഛിക്കുന്നവരെ അല്ലാഹു തന്റെ അനുഗ്രഹത്തില്‍ പ്രവേശിപ്പിക്കുന്നു. അക്രമികള്‍ക്കോ, നോവേറിയ ശിക്ഷയാണ് അവന്‍ ഒരുക്കിവെച്ചിരിക്കുന്നത്.
\end{malayalam}}
\chapter{\textmalayalam{മുര്‍സലാത്ത് ( അയക്കപ്പെടുന്നവര്‍ )}}
\begin{Arabic}
\Huge{\centerline{\basmalah}}\end{Arabic}
\flushright{\begin{Arabic}
\quranayah[77][1]
\end{Arabic}}
\flushleft{\begin{malayalam}
തുടര്‍ച്ചയായി അയക്കപ്പെടുന്നവ സത്യം.
\end{malayalam}}
\flushright{\begin{Arabic}
\quranayah[77][2]
\end{Arabic}}
\flushleft{\begin{malayalam}
പിന്നെ കൊടുങ്കാറ്റായി ആഞ്ഞുവീശുന്നവ സത്യം.
\end{malayalam}}
\flushright{\begin{Arabic}
\quranayah[77][3]
\end{Arabic}}
\flushleft{\begin{malayalam}
പരക്കെപരത്തുന്നവ സത്യം.
\end{malayalam}}
\flushright{\begin{Arabic}
\quranayah[77][4]
\end{Arabic}}
\flushleft{\begin{malayalam}
പിന്നെ അതിനെ വേര്‍തിരിച്ച് വിവേചിക്കുന്നവ സത്യം.
\end{malayalam}}
\flushright{\begin{Arabic}
\quranayah[77][5]
\end{Arabic}}
\flushleft{\begin{malayalam}
ദിവ്യസന്ദേശം ഇട്ടുകൊടുക്കുന്നവസത്യം.
\end{malayalam}}
\flushright{\begin{Arabic}
\quranayah[77][6]
\end{Arabic}}
\flushleft{\begin{malayalam}
ഒഴികഴിവായോ, താക്കീതായോ.
\end{malayalam}}
\flushright{\begin{Arabic}
\quranayah[77][7]
\end{Arabic}}
\flushleft{\begin{malayalam}
നിങ്ങളോട് വാഗ്ദാനം ചെയ്യപ്പെടുന്നത് സംഭവിക്കുക തന്നെ ചെയ്യും.
\end{malayalam}}
\flushright{\begin{Arabic}
\quranayah[77][8]
\end{Arabic}}
\flushleft{\begin{malayalam}
നക്ഷത്രങ്ങളുടെ പ്രകാശം അണഞ്ഞില്ലാതാവുകയും,
\end{malayalam}}
\flushright{\begin{Arabic}
\quranayah[77][9]
\end{Arabic}}
\flushleft{\begin{malayalam}
ആകാശം പിളര്‍ന്ന് പോവുകയും,
\end{malayalam}}
\flushright{\begin{Arabic}
\quranayah[77][10]
\end{Arabic}}
\flushleft{\begin{malayalam}
പര്‍വതങ്ങള്‍ ഉടഞ്ഞുപൊടിയുകയും,
\end{malayalam}}
\flushright{\begin{Arabic}
\quranayah[77][11]
\end{Arabic}}
\flushleft{\begin{malayalam}
ദൂതന്മാരുടെ വരവ് നിശ്ചയിക്കപ്പെടുകയും ചെയ്താല്‍.
\end{malayalam}}
\flushright{\begin{Arabic}
\quranayah[77][12]
\end{Arabic}}
\flushleft{\begin{malayalam}
ഏതൊരു ദിനത്തിലേക്കാണ് അത് നിശ്ചയിക്കപ്പെട്ടിരിക്കുന്നത്?
\end{malayalam}}
\flushright{\begin{Arabic}
\quranayah[77][13]
\end{Arabic}}
\flushleft{\begin{malayalam}
വിധി തീര്‍പ്പിന്റെ ദിനത്തിലേക്ക്.
\end{malayalam}}
\flushright{\begin{Arabic}
\quranayah[77][14]
\end{Arabic}}
\flushleft{\begin{malayalam}
വിധി തീര്‍പ്പിന്റെ ദിനമെന്തെന്ന് നിനക്കെന്തറിയാം?
\end{malayalam}}
\flushright{\begin{Arabic}
\quranayah[77][15]
\end{Arabic}}
\flushleft{\begin{malayalam}
അന്നാളില്‍ സത്യനിഷേധികള്‍ക്ക് കൊടിയ നാശം!
\end{malayalam}}
\flushright{\begin{Arabic}
\quranayah[77][16]
\end{Arabic}}
\flushleft{\begin{malayalam}
മുന്‍ഗാമികളെ നാം നശിപ്പിച്ചില്ലേ?
\end{malayalam}}
\flushright{\begin{Arabic}
\quranayah[77][17]
\end{Arabic}}
\flushleft{\begin{malayalam}
അവര്‍ക്കു പിറകെ പിന്‍ഗാമികളെയും നാം നശിപ്പിക്കും.
\end{malayalam}}
\flushright{\begin{Arabic}
\quranayah[77][18]
\end{Arabic}}
\flushleft{\begin{malayalam}
കുറ്റവാളികളെ നാം അങ്ങനെയാണ് ചെയ്യുക.
\end{malayalam}}
\flushright{\begin{Arabic}
\quranayah[77][19]
\end{Arabic}}
\flushleft{\begin{malayalam}
അന്നാളില്‍ സത്യനിഷേധികള്‍ക്ക് കൊടിയ നാശം!
\end{malayalam}}
\flushright{\begin{Arabic}
\quranayah[77][20]
\end{Arabic}}
\flushleft{\begin{malayalam}
നിസ്സാരമായ ദ്രാവകത്തില്‍നിന്നല്ലേ നിങ്ങളെ നാം സൃഷ്ടിച്ചത്?
\end{malayalam}}
\flushright{\begin{Arabic}
\quranayah[77][21]
\end{Arabic}}
\flushleft{\begin{malayalam}
എന്നിട്ടു നാമതിനെ സുരക്ഷിതമായ ഒരിടത്തു സൂക്ഷിച്ചു.
\end{malayalam}}
\flushright{\begin{Arabic}
\quranayah[77][22]
\end{Arabic}}
\flushleft{\begin{malayalam}
ഒരു നിശ്ചിത അവധി വരെ.
\end{malayalam}}
\flushright{\begin{Arabic}
\quranayah[77][23]
\end{Arabic}}
\flushleft{\begin{malayalam}
അങ്ങനെ നാം എല്ലാം കൃത്യമായി നിര്‍ണയിച്ചു. നാം എത്രനല്ല നിര്‍ണയക്കാരന്‍.
\end{malayalam}}
\flushright{\begin{Arabic}
\quranayah[77][24]
\end{Arabic}}
\flushleft{\begin{malayalam}
അന്നാളില്‍ സത്യനിഷേധികള്‍ക്ക് കൊടിയ നാശം.
\end{malayalam}}
\flushright{\begin{Arabic}
\quranayah[77][25]
\end{Arabic}}
\flushleft{\begin{malayalam}
ഭൂമിയെ നാം എല്ലാവരെയും ഉള്‍ക്കൊള്ളുന്നതാക്കിയില്ലേ?
\end{malayalam}}
\flushright{\begin{Arabic}
\quranayah[77][26]
\end{Arabic}}
\flushleft{\begin{malayalam}
ജീവിച്ചിരിക്കുന്നവരെയും മരിച്ചവരെയും.
\end{malayalam}}
\flushright{\begin{Arabic}
\quranayah[77][27]
\end{Arabic}}
\flushleft{\begin{malayalam}
ഭൂമിയില്‍ നാം ഉയര്‍ന്ന പര്‍വതങ്ങളുണ്ടാക്കി. നിങ്ങള്‍ക്ക് നാം കുടിക്കാന്‍ തെളിനീര്‍ നല്‍കി.
\end{malayalam}}
\flushright{\begin{Arabic}
\quranayah[77][28]
\end{Arabic}}
\flushleft{\begin{malayalam}
അന്നാളില്‍ സത്യനിഷേധികള്‍ക്ക് കൊടിയ നാശം.
\end{malayalam}}
\flushright{\begin{Arabic}
\quranayah[77][29]
\end{Arabic}}
\flushleft{\begin{malayalam}
അവരോട് പറയും: നിങ്ങളെന്നും നിഷേധിച്ചു തള്ളിയിരുന്ന ഒന്നില്ലേ; അതിലേക്ക് പോയിക്കൊള്ളുക.
\end{malayalam}}
\flushright{\begin{Arabic}
\quranayah[77][30]
\end{Arabic}}
\flushleft{\begin{malayalam}
മൂന്ന് ശാഖകളുള്ള ഒരുതരം നിഴലിലേക്ക് പോയിക്കൊള്ളുക.
\end{malayalam}}
\flushright{\begin{Arabic}
\quranayah[77][31]
\end{Arabic}}
\flushleft{\begin{malayalam}
അത് തണല്‍ നല്‍കുന്നതല്ല. തീ ജ്വാലയില്‍നിന്ന് രക്ഷ നല്‍കുന്നതുമല്ല.
\end{malayalam}}
\flushright{\begin{Arabic}
\quranayah[77][32]
\end{Arabic}}
\flushleft{\begin{malayalam}
അത് കൂറ്റന്‍ കെട്ടിടം പോലെ തോന്നിക്കുന്ന തീപ്പൊരി വിതറിക്കൊണ്ടിരിക്കും.
\end{malayalam}}
\flushright{\begin{Arabic}
\quranayah[77][33]
\end{Arabic}}
\flushleft{\begin{malayalam}
അത് കടും മഞ്ഞയുള്ള ഒട്ടകങ്ങളെപ്പോലെയിരിക്കും.
\end{malayalam}}
\flushright{\begin{Arabic}
\quranayah[77][34]
\end{Arabic}}
\flushleft{\begin{malayalam}
അന്നാളില്‍ സത്യനിഷേധികള്‍ക്ക് കൊടിയ നാശം.
\end{malayalam}}
\flushright{\begin{Arabic}
\quranayah[77][35]
\end{Arabic}}
\flushleft{\begin{malayalam}
അവര്‍ക്ക് ഒരക്ഷരം ഉരിയാടാനാവാത്ത ദിനമാണത്.
\end{malayalam}}
\flushright{\begin{Arabic}
\quranayah[77][36]
\end{Arabic}}
\flushleft{\begin{malayalam}
എന്തെങ്കിലും ഒഴികഴിവു പറയാന്‍ അവര്‍ക്ക് അനുവാദം നല്‍കപ്പെടുന്നതുമല്ല.
\end{malayalam}}
\flushright{\begin{Arabic}
\quranayah[77][37]
\end{Arabic}}
\flushleft{\begin{malayalam}
അന്നാളില്‍ സത്യനിഷേധികള്‍ക്ക് കൊടിയ നാശം.
\end{malayalam}}
\flushright{\begin{Arabic}
\quranayah[77][38]
\end{Arabic}}
\flushleft{\begin{malayalam}
വിധി തീര്‍പ്പിന്റെ ദിനമാണത്. നിങ്ങളെയും നിങ്ങളുടെ മുന്‍ഗാമികളെയും നാം ഒരുമിച്ചുകൂട്ടിയിരിക്കുന്നു.
\end{malayalam}}
\flushright{\begin{Arabic}
\quranayah[77][39]
\end{Arabic}}
\flushleft{\begin{malayalam}
നിങ്ങളുടെ വശം വല്ല തന്ത്രവുമുണ്ടെങ്കില്‍ ആ തന്ത്രമിങ്ങ് പ്രയോഗിച്ചു കൊള്ളുക.
\end{malayalam}}
\flushright{\begin{Arabic}
\quranayah[77][40]
\end{Arabic}}
\flushleft{\begin{malayalam}
അന്നാളില്‍ സത്യനിഷേധികള്‍ക്ക് കൊടിയ നാശം.
\end{malayalam}}
\flushright{\begin{Arabic}
\quranayah[77][41]
\end{Arabic}}
\flushleft{\begin{malayalam}
ഭക്തരോ, അന്ന് തണലുകളിലും അരുവികളിലുമായിരിക്കും.
\end{malayalam}}
\flushright{\begin{Arabic}
\quranayah[77][42]
\end{Arabic}}
\flushleft{\begin{malayalam}
അവര്‍ക്കിഷ്ടപ്പെട്ട പഴങ്ങളോടൊപ്പവും.
\end{malayalam}}
\flushright{\begin{Arabic}
\quranayah[77][43]
\end{Arabic}}
\flushleft{\begin{malayalam}
അപ്പോള്‍ അവരെ അറിയിക്കും: സംതൃപ്തിയോടെ തിന്നുകയും കുടിക്കുകയും ചെയ്യുക. നിങ്ങള്‍ പ്രവര്‍ത്തിച്ചിരുന്നതിന്റെ പ്രതിഫലമാണിത്.
\end{malayalam}}
\flushright{\begin{Arabic}
\quranayah[77][44]
\end{Arabic}}
\flushleft{\begin{malayalam}
ഇവ്വിധമാണ് നാം സുകര്‍മികള്‍ക്ക് പ്രതിഫലം നല്‍കുക.
\end{malayalam}}
\flushright{\begin{Arabic}
\quranayah[77][45]
\end{Arabic}}
\flushleft{\begin{malayalam}
അന്നാളില്‍ സത്യനിഷേധികള്‍ക്ക് കൊടിയ നാശം.
\end{malayalam}}
\flushright{\begin{Arabic}
\quranayah[77][46]
\end{Arabic}}
\flushleft{\begin{malayalam}
അവരെ അറിയിക്കും: നിങ്ങള്‍ തിന്നുകൊള്ളുക. സുഖിച്ചു കൊള്ളുക. ഇത്തിരി കാലം മാത്രം. നിങ്ങള്‍ പാപികളാണ്; തീര്‍ച്ച.
\end{malayalam}}
\flushright{\begin{Arabic}
\quranayah[77][47]
\end{Arabic}}
\flushleft{\begin{malayalam}
അന്നാളില്‍ സത്യനിഷേധികള്‍ക്ക് കൊടിയ നാശം.
\end{malayalam}}
\flushright{\begin{Arabic}
\quranayah[77][48]
\end{Arabic}}
\flushleft{\begin{malayalam}
അവരോട് അല്ലാഹുവിന്റെ മുമ്പില്‍ കുമ്പിടാന്‍ കല്‍പിച്ചാല്‍ അവര്‍ കുമ്പിടുന്നില്ല.
\end{malayalam}}
\flushright{\begin{Arabic}
\quranayah[77][49]
\end{Arabic}}
\flushleft{\begin{malayalam}
അന്നാളില്‍ സത്യനിഷേധികള്‍ക്ക് കൊടിയ നാശം
\end{malayalam}}
\flushright{\begin{Arabic}
\quranayah[77][50]
\end{Arabic}}
\flushleft{\begin{malayalam}
ഈ ഖുര്‍ആന്നപ്പുറം ഏതു വേദത്തിലാണ് അവരിനി വിശ്വസിക്കുക?
\end{malayalam}}
\chapter{\textmalayalam{നബഅ് ( വൃത്താന്തം )}}
\begin{Arabic}
\Huge{\centerline{\basmalah}}\end{Arabic}
\flushright{\begin{Arabic}
\quranayah[78][1]
\end{Arabic}}
\flushleft{\begin{malayalam}
ഏതിനെപ്പറ്റിയാണ് അവരന്യോന്യം ചോദിച്ചുകൊണ്ടിരിക്കുന്നത്?
\end{malayalam}}
\flushright{\begin{Arabic}
\quranayah[78][2]
\end{Arabic}}
\flushleft{\begin{malayalam}
അതിഭയങ്കരമായ വാര്‍ത്തയെപ്പറ്റി തന്നെ.
\end{malayalam}}
\flushright{\begin{Arabic}
\quranayah[78][3]
\end{Arabic}}
\flushleft{\begin{malayalam}
അതിലവര്‍ ഭിന്നാഭിപ്രായക്കാരാണ്.
\end{malayalam}}
\flushright{\begin{Arabic}
\quranayah[78][4]
\end{Arabic}}
\flushleft{\begin{malayalam}
വേണ്ട; വൈകാതെ അവരറിയുകതന്നെ ചെയ്യും.
\end{malayalam}}
\flushright{\begin{Arabic}
\quranayah[78][5]
\end{Arabic}}
\flushleft{\begin{malayalam}
വീണ്ടും വേണ്ട; ഉറപ്പായും അവരറിയും.
\end{malayalam}}
\flushright{\begin{Arabic}
\quranayah[78][6]
\end{Arabic}}
\flushleft{\begin{malayalam}
ഭൂമിയെ നാം മെത്തയാക്കിയില്ലേ?
\end{malayalam}}
\flushright{\begin{Arabic}
\quranayah[78][7]
\end{Arabic}}
\flushleft{\begin{malayalam}
മലകളെ ആണികളും?
\end{malayalam}}
\flushright{\begin{Arabic}
\quranayah[78][8]
\end{Arabic}}
\flushleft{\begin{malayalam}
നിങ്ങളെ നാം ഇണകളായി സൃഷ്ടിച്ചു.
\end{malayalam}}
\flushright{\begin{Arabic}
\quranayah[78][9]
\end{Arabic}}
\flushleft{\begin{malayalam}
നിങ്ങളുടെ ഉറക്കത്തെ നാം വിശ്രമമാക്കി.
\end{malayalam}}
\flushright{\begin{Arabic}
\quranayah[78][10]
\end{Arabic}}
\flushleft{\begin{malayalam}
രാവിനെ വസ്ത്രമാക്കി.
\end{malayalam}}
\flushright{\begin{Arabic}
\quranayah[78][11]
\end{Arabic}}
\flushleft{\begin{malayalam}
പകലിനെ ജീവിതവേളയാക്കി.
\end{malayalam}}
\flushright{\begin{Arabic}
\quranayah[78][12]
\end{Arabic}}
\flushleft{\begin{malayalam}
നിങ്ങള്‍ക്കു മേലെ ഭദ്രമായ ഏഴാകാശങ്ങളെ നാം നിര്‍മിച്ചു.
\end{malayalam}}
\flushright{\begin{Arabic}
\quranayah[78][13]
\end{Arabic}}
\flushleft{\begin{malayalam}
കത്തിജ്ജ്വലിക്കുന്ന ഒരു വിളക്കും നാം സ്ഥാപിച്ചു.
\end{malayalam}}
\flushright{\begin{Arabic}
\quranayah[78][14]
\end{Arabic}}
\flushleft{\begin{malayalam}
കാര്‍മുകിലില്‍നിന്ന് കുത്തിയൊഴുകും വെള്ളമിറക്കി.
\end{malayalam}}
\flushright{\begin{Arabic}
\quranayah[78][15]
\end{Arabic}}
\flushleft{\begin{malayalam}
അതുവഴി ധാന്യവും ചെടികളും ഉല്‍പാദിപ്പിക്കാന്‍.
\end{malayalam}}
\flushright{\begin{Arabic}
\quranayah[78][16]
\end{Arabic}}
\flushleft{\begin{malayalam}
ഇടതൂര്‍ന്ന തോട്ടങ്ങളും.
\end{malayalam}}
\flushright{\begin{Arabic}
\quranayah[78][17]
\end{Arabic}}
\flushleft{\begin{malayalam}
നിശ്ചയമായും വിധിദിനം സമയനിര്‍ണിതമാണ്.
\end{malayalam}}
\flushright{\begin{Arabic}
\quranayah[78][18]
\end{Arabic}}
\flushleft{\begin{malayalam}
കാഹളം ഊതുന്ന ദിനമാണത്. അപ്പോള്‍ നിങ്ങള്‍ കൂട്ടംകൂട്ടമായി വന്നെത്തും.
\end{malayalam}}
\flushright{\begin{Arabic}
\quranayah[78][19]
\end{Arabic}}
\flushleft{\begin{malayalam}
ആകാശം തുറക്കപ്പെടും. അത് അനേകം കവാടങ്ങളായിത്തീരും.
\end{malayalam}}
\flushright{\begin{Arabic}
\quranayah[78][20]
\end{Arabic}}
\flushleft{\begin{malayalam}
പര്‍വതങ്ങള്‍ ഇളകി നീങ്ങും. അവ മരീചികയാകും.
\end{malayalam}}
\flushright{\begin{Arabic}
\quranayah[78][21]
\end{Arabic}}
\flushleft{\begin{malayalam}
നിശ്ചയമായും നരകത്തീ പതിസ്ഥലമാണ്.
\end{malayalam}}
\flushright{\begin{Arabic}
\quranayah[78][22]
\end{Arabic}}
\flushleft{\begin{malayalam}
അതിക്രമികളുടെ സങ്കേതം.
\end{malayalam}}
\flushright{\begin{Arabic}
\quranayah[78][23]
\end{Arabic}}
\flushleft{\begin{malayalam}
അവരതില്‍ യുഗങ്ങളോളം വസിക്കും.
\end{malayalam}}
\flushright{\begin{Arabic}
\quranayah[78][24]
\end{Arabic}}
\flushleft{\begin{malayalam}
കുളിരോ കുടിനീരോ അവരവിടെ അനുഭവിക്കുകയില്ല.
\end{malayalam}}
\flushright{\begin{Arabic}
\quranayah[78][25]
\end{Arabic}}
\flushleft{\begin{malayalam}
തിളക്കുന്ന വെള്ളവും ചലവുമല്ലാതെ.
\end{malayalam}}
\flushright{\begin{Arabic}
\quranayah[78][26]
\end{Arabic}}
\flushleft{\begin{malayalam}
അര്‍ഹിക്കുന്ന പ്രതിഫലം.
\end{malayalam}}
\flushright{\begin{Arabic}
\quranayah[78][27]
\end{Arabic}}
\flushleft{\begin{malayalam}
തീര്‍ച്ചയായും അവര്‍ വിചാരണ പ്രതീക്ഷിക്കുന്നവരായിരുന്നില്ല.
\end{malayalam}}
\flushright{\begin{Arabic}
\quranayah[78][28]
\end{Arabic}}
\flushleft{\begin{malayalam}
നമ്മുടെ താക്കീതുകളെ അവര്‍ അപ്പാടെ കള്ളമാക്കി തള്ളി.
\end{malayalam}}
\flushright{\begin{Arabic}
\quranayah[78][29]
\end{Arabic}}
\flushleft{\begin{malayalam}
എല്ലാ കാര്യവും നാം കൃത്യമായി രേഖപ്പെടുത്തി വെച്ചിട്ടുണ്ട്.
\end{malayalam}}
\flushright{\begin{Arabic}
\quranayah[78][30]
\end{Arabic}}
\flushleft{\begin{malayalam}
അതിനാല്‍ നിങ്ങള്‍ അനുഭവിച്ചുകൊള്ളുക. നിങ്ങള്‍ക്കു ശിക്ഷയല്ലാതൊന്നും വര്‍ധിപ്പിച്ചു തരാനില്ല.
\end{malayalam}}
\flushright{\begin{Arabic}
\quranayah[78][31]
\end{Arabic}}
\flushleft{\begin{malayalam}
ഭക്തന്മാര്‍ക്ക് വിജയം ഉറപ്പ്.
\end{malayalam}}
\flushright{\begin{Arabic}
\quranayah[78][32]
\end{Arabic}}
\flushleft{\begin{malayalam}
അവര്‍ക്ക് സ്വര്‍ഗത്തോപ്പുകളും മുന്തിരിക്കുലകളുമുണ്ട്.
\end{malayalam}}
\flushright{\begin{Arabic}
\quranayah[78][33]
\end{Arabic}}
\flushleft{\begin{malayalam}
തുടുത്ത മാറിടമുള്ള തുല്യവയസ്കരായ തരുണികളും.
\end{malayalam}}
\flushright{\begin{Arabic}
\quranayah[78][34]
\end{Arabic}}
\flushleft{\begin{malayalam}
നിറഞ്ഞ കോപ്പകളും.
\end{malayalam}}
\flushright{\begin{Arabic}
\quranayah[78][35]
\end{Arabic}}
\flushleft{\begin{malayalam}
അവരവിടെ പൊയ്മൊഴികളോ വിടുവാക്കുകളോ കേള്‍ക്കുകയില്ല.
\end{malayalam}}
\flushright{\begin{Arabic}
\quranayah[78][36]
\end{Arabic}}
\flushleft{\begin{malayalam}
നിന്റെ നാഥനില്‍ നിന്നുള്ള പ്രതിഫലമായാണത്. അവരര്‍ഹിക്കുന്ന സമ്മാനം.
\end{malayalam}}
\flushright{\begin{Arabic}
\quranayah[78][37]
\end{Arabic}}
\flushleft{\begin{malayalam}
അവന്‍, ആകാശഭൂമികളുടെയും അവയ്ക്കിടയിലുള്ളവയുടെയും നാഥനാണ്. ദയാപരന്‍. അവനുമായി നേരില്‍ സംഭാഷണം നടത്താനാര്‍ക്കുമാവില്ല.
\end{malayalam}}
\flushright{\begin{Arabic}
\quranayah[78][38]
\end{Arabic}}
\flushleft{\begin{malayalam}
ജിബ്രീലും മറ്റു മലക്കുകളും അണിനിരക്കും ദിനമാണ് അതുണ്ടാവുക. അന്നാര്‍ക്കും സംസാരിക്കാനാവില്ല; കരുണാനിധിയായ നാഥന്‍ അനുവാദം നല്‍കിയവന്നും സത്യം പറഞ്ഞവന്നുമൊഴികെ.
\end{malayalam}}
\flushright{\begin{Arabic}
\quranayah[78][39]
\end{Arabic}}
\flushleft{\begin{malayalam}
അതത്രെ സത്യദിനം. അതിനാല്‍ ഇഷ്ടമുള്ളവന്‍ തന്റെ നാഥങ്കലേക്ക് മടങ്ങാനുള്ള മാര്‍ഗമവലംബിക്കട്ടെ.
\end{malayalam}}
\flushright{\begin{Arabic}
\quranayah[78][40]
\end{Arabic}}
\flushleft{\begin{malayalam}
ആസന്നമായ ശിക്ഷയെ സംബന്ധിച്ച് തീര്‍ച്ചയായും നാം നിങ്ങള്‍ക്ക് താക്കീതു നല്‍കിയിരിക്കുന്നു. മനുഷ്യന്‍ തന്റെ ഇരു കരങ്ങളും ചെയ്തുവെച്ചത് നോക്കിക്കാണും ദിനം. അന്ന് സത്യനിഷേധി പറയും: "ഞാന്‍ മണ്ണായിരുന്നെങ്കില്‍ എത്ര നന്നായേനെ.”
\end{malayalam}}
\chapter{\textmalayalam{നാസിയാത്ത് ( ഊരിയെടുക്കുന്നവ )}}
\begin{Arabic}
\Huge{\centerline{\basmalah}}\end{Arabic}
\flushright{\begin{Arabic}
\quranayah[79][1]
\end{Arabic}}
\flushleft{\begin{malayalam}
മുങ്ങിച്ചെന്ന് ഊരിയെടുക്കുന്നവ സത്യം.
\end{malayalam}}
\flushright{\begin{Arabic}
\quranayah[79][2]
\end{Arabic}}
\flushleft{\begin{malayalam}
സൌമ്യമായി പുറത്തേക്കെടുക്കുന്നവ സത്യം.
\end{malayalam}}
\flushright{\begin{Arabic}
\quranayah[79][3]
\end{Arabic}}
\flushleft{\begin{malayalam}
ശക്തിയായി നീന്തുന്നവ സത്യം.
\end{malayalam}}
\flushright{\begin{Arabic}
\quranayah[79][4]
\end{Arabic}}
\flushleft{\begin{malayalam}
എന്നിട്ട് മുന്നോട്ടു കുതിക്കുന്നവ സത്യം.
\end{malayalam}}
\flushright{\begin{Arabic}
\quranayah[79][5]
\end{Arabic}}
\flushleft{\begin{malayalam}
കാര്യങ്ങള്‍ നിയന്ത്രിക്കുന്നവ സത്യം!
\end{malayalam}}
\flushright{\begin{Arabic}
\quranayah[79][6]
\end{Arabic}}
\flushleft{\begin{malayalam}
ഘോരസംഭവം പ്രകമ്പനം സൃഷ്ടിക്കും ദിനം;
\end{malayalam}}
\flushright{\begin{Arabic}
\quranayah[79][7]
\end{Arabic}}
\flushleft{\begin{malayalam}
അതിന്റെ പിറകെ മറ്റൊരു പ്രകമ്പനവുമുണ്ടാകും.
\end{malayalam}}
\flushright{\begin{Arabic}
\quranayah[79][8]
\end{Arabic}}
\flushleft{\begin{malayalam}
അന്നു ചില ഹൃദയങ്ങള്‍ പിടയുന്നവയായിരിക്കും.
\end{malayalam}}
\flushright{\begin{Arabic}
\quranayah[79][9]
\end{Arabic}}
\flushleft{\begin{malayalam}
അവരുടെ കണ്ണുകള്‍ പേടിച്ചരണ്ടിരിക്കും.
\end{malayalam}}
\flushright{\begin{Arabic}
\quranayah[79][10]
\end{Arabic}}
\flushleft{\begin{malayalam}
അവര്‍ ചോദിക്കുന്നു: "ഉറപ്പായും നാം പൂര്‍വാവസ്ഥയിലേക്ക് മടക്കപ്പെടുമെന്നോ?
\end{malayalam}}
\flushright{\begin{Arabic}
\quranayah[79][11]
\end{Arabic}}
\flushleft{\begin{malayalam}
"നാം നുരുമ്പിയ എല്ലുകളായ ശേഷവും?”
\end{malayalam}}
\flushright{\begin{Arabic}
\quranayah[79][12]
\end{Arabic}}
\flushleft{\begin{malayalam}
അവര്‍ ഘോഷിക്കുന്നു: "എങ്കിലതൊരു തുലഞ്ഞ തിരിച്ചു പോക്കു തന്നെ.”
\end{malayalam}}
\flushright{\begin{Arabic}
\quranayah[79][13]
\end{Arabic}}
\flushleft{\begin{malayalam}
എന്നാല്‍ അതൊരു ഘോര ശബ്ദം മാത്രമായിരിക്കും.
\end{malayalam}}
\flushright{\begin{Arabic}
\quranayah[79][14]
\end{Arabic}}
\flushleft{\begin{malayalam}
അപ്പോഴേക്കും അവര്‍ ഭൂതലത്തിലെത്തിയിരിക്കും.
\end{malayalam}}
\flushright{\begin{Arabic}
\quranayah[79][15]
\end{Arabic}}
\flushleft{\begin{malayalam}
മൂസായുടെ വര്‍ത്തമാനം നിനക്ക് വന്നെത്തിയോ?
\end{malayalam}}
\flushright{\begin{Arabic}
\quranayah[79][16]
\end{Arabic}}
\flushleft{\begin{malayalam}
വിശുദ്ധമായ ത്വുവാ താഴ്വരയില്‍ വെച്ച് തന്റെ നാഥന്‍ അദ്ദേഹത്തെ വിളിച്ചു കല്പിച്ചതോര്‍ക്കുക:
\end{malayalam}}
\flushright{\begin{Arabic}
\quranayah[79][17]
\end{Arabic}}
\flushleft{\begin{malayalam}
"നീ ഫറവോന്റെ അടുത്തേക്ക് പോവുക. അവന്‍ അതിക്രമിയായിരിക്കുന്നു.
\end{malayalam}}
\flushright{\begin{Arabic}
\quranayah[79][18]
\end{Arabic}}
\flushleft{\begin{malayalam}
"എന്നിട്ട് അയാളോട് ചോദിക്കുക: “നീ വിശുദ്ധി വരിക്കാന്‍ തയ്യാറുണ്ടോ?
\end{malayalam}}
\flushright{\begin{Arabic}
\quranayah[79][19]
\end{Arabic}}
\flushleft{\begin{malayalam}
“ഞാന്‍ നിന്നെ നിന്റെ നാഥനിലേക്കു വഴിനടത്താനും അങ്ങനെ നിനക്കു ദൈവഭക്തനാകാനും?”
\end{malayalam}}
\flushright{\begin{Arabic}
\quranayah[79][20]
\end{Arabic}}
\flushleft{\begin{malayalam}
മൂസാ അയാള്‍ക്ക് മഹത്തായ ഒരടയാളം കാണിച്ചുകൊടുത്തു.
\end{malayalam}}
\flushright{\begin{Arabic}
\quranayah[79][21]
\end{Arabic}}
\flushleft{\begin{malayalam}
അപ്പോള്‍ അയാളതിനെ കളവാക്കുകയും ധിക്കരിക്കുകയും ചെയ്തു.
\end{malayalam}}
\flushright{\begin{Arabic}
\quranayah[79][22]
\end{Arabic}}
\flushleft{\begin{malayalam}
പിന്നീട് അയാള്‍ എതിര്‍ശ്രമങ്ങള്‍ക്കായി തിരിഞ്ഞു നടന്നു.
\end{malayalam}}
\flushright{\begin{Arabic}
\quranayah[79][23]
\end{Arabic}}
\flushleft{\begin{malayalam}
അങ്ങനെ ജനങ്ങളെ ഒരുമിച്ചുകൂട്ടി ഇങ്ങനെ വിളംബരം ചെയ്തു:
\end{malayalam}}
\flushright{\begin{Arabic}
\quranayah[79][24]
\end{Arabic}}
\flushleft{\begin{malayalam}
അവന്‍ പ്രഖ്യാപിച്ചു: ഞാനാണ് നിങ്ങളുടെ പരമോന്നത നാഥന്‍.
\end{malayalam}}
\flushright{\begin{Arabic}
\quranayah[79][25]
\end{Arabic}}
\flushleft{\begin{malayalam}
അപ്പോള്‍ അല്ലാഹു അവനെ പിടികൂടി. മറുലോകത്തെയും ഈലോകത്തെയും ശിക്ഷക്കിരയാക്കാന്‍.
\end{malayalam}}
\flushright{\begin{Arabic}
\quranayah[79][26]
\end{Arabic}}
\flushleft{\begin{malayalam}
നിശ്ചയമായും ദൈവഭയമുള്ളവര്‍ക്ക് ഇതില്‍ ഗുണപാഠമുണ്ട്.
\end{malayalam}}
\flushright{\begin{Arabic}
\quranayah[79][27]
\end{Arabic}}
\flushleft{\begin{malayalam}
നിങ്ങളെ സൃഷ്ടിക്കുന്നതോ ആകാശത്തെ സൃഷ്ടിക്കുന്നതോ ഏതാണ് കൂടുതല്‍ പ്രയാസകരം? അവന്‍ അതുണ്ടാക്കി.
\end{malayalam}}
\flushright{\begin{Arabic}
\quranayah[79][28]
\end{Arabic}}
\flushleft{\begin{malayalam}
അതിന്റെ വിതാനം ഉയര്‍ത്തുകയും അങ്ങനെ അതിനെ കുറ്റമറ്റതാക്കുകയും ചെയ്തു.
\end{malayalam}}
\flushright{\begin{Arabic}
\quranayah[79][29]
\end{Arabic}}
\flushleft{\begin{malayalam}
അതിലെ രാവിനെ അവന്‍ ഇരുളുള്ളതാക്കി. പകലിനെ ഇരുളില്‍നിന്ന് പുറത്തെടുക്കുകയും ചെയ്തു.
\end{malayalam}}
\flushright{\begin{Arabic}
\quranayah[79][30]
\end{Arabic}}
\flushleft{\begin{malayalam}
അതിനുശേഷം ഭൂമിയെ പരത്തി വിടര്‍ത്തി.
\end{malayalam}}
\flushright{\begin{Arabic}
\quranayah[79][31]
\end{Arabic}}
\flushleft{\begin{malayalam}
ഭൂമിയില്‍നിന്ന് അതിന്റെ വെള്ളവും സസ്യങ്ങളും പുറത്തുകൊണ്ടുവന്നു.
\end{malayalam}}
\flushright{\begin{Arabic}
\quranayah[79][32]
\end{Arabic}}
\flushleft{\begin{malayalam}
മലകളെ ഉറപ്പിച്ചു നിര്‍ത്തി.
\end{malayalam}}
\flushright{\begin{Arabic}
\quranayah[79][33]
\end{Arabic}}
\flushleft{\begin{malayalam}
നിങ്ങള്‍ക്കും നിങ്ങളുടെ കന്നുകാലികള്‍ക്കും വിഭവമായി.
\end{malayalam}}
\flushright{\begin{Arabic}
\quranayah[79][34]
\end{Arabic}}
\flushleft{\begin{malayalam}
എന്നാല്‍ ആ ഘോര വിപത്ത് വന്നെത്തിയാല്‍!
\end{malayalam}}
\flushright{\begin{Arabic}
\quranayah[79][35]
\end{Arabic}}
\flushleft{\begin{malayalam}
മനുഷ്യന്‍ താന്‍ പ്രയത്നിച്ചു നേടിയതിനെക്കുറിച്ചോര്‍ക്കുന്ന ദിനം!
\end{malayalam}}
\flushright{\begin{Arabic}
\quranayah[79][36]
\end{Arabic}}
\flushleft{\begin{malayalam}
കാഴ്ചക്കാര്‍ക്കായി നരകം വെളിപ്പെടുത്തും നാള്‍.
\end{malayalam}}
\flushright{\begin{Arabic}
\quranayah[79][37]
\end{Arabic}}
\flushleft{\begin{malayalam}
അപ്പോള്‍; ആര്‍ അതിക്രമം കാണിക്കുകയും,
\end{malayalam}}
\flushright{\begin{Arabic}
\quranayah[79][38]
\end{Arabic}}
\flushleft{\begin{malayalam}
ഈ ലോക ജീവിതത്തിന് അളവറ്റ പ്രാധാന്യം നല്‍കുകയും ചെയ്തുവോ,
\end{malayalam}}
\flushright{\begin{Arabic}
\quranayah[79][39]
\end{Arabic}}
\flushleft{\begin{malayalam}
അവന്റെ സങ്കേതം കത്തിക്കാളുന്ന നരകത്തീയാണ്; തീര്‍ച്ച.
\end{malayalam}}
\flushright{\begin{Arabic}
\quranayah[79][40]
\end{Arabic}}
\flushleft{\begin{malayalam}
എന്നാല്‍ ആര്‍ തന്റെ നാഥന്റെ പദവിയെ പേടിക്കുകയും ആത്മാവി നെ ശാരീരികേഛകളില്‍ നിന്ന് വിലക്കി നിര്‍ത്തുകയും ചെയ്തുവോ,
\end{malayalam}}
\flushright{\begin{Arabic}
\quranayah[79][41]
\end{Arabic}}
\flushleft{\begin{malayalam}
ഉറപ്പായും അവന്റെ മടക്കസ്ഥാനം സ്വര്‍ഗമാണ്.
\end{malayalam}}
\flushright{\begin{Arabic}
\quranayah[79][42]
\end{Arabic}}
\flushleft{\begin{malayalam}
ആ അന്ത്യ സമയത്തെ സംബന്ധിച്ച് അവര്‍ നിന്നോട് ചോദിക്കുന്നു. അതെപ്പോഴാണുണ്ടാവുകയെന്ന്.
\end{malayalam}}
\flushright{\begin{Arabic}
\quranayah[79][43]
\end{Arabic}}
\flushleft{\begin{malayalam}
നീ അതേക്കുറിച്ച് എന്തുപറയാനാണ്?
\end{malayalam}}
\flushright{\begin{Arabic}
\quranayah[79][44]
\end{Arabic}}
\flushleft{\begin{malayalam}
അതേക്കുറിച്ച് അന്തിമമായ അറിവ് നിന്റെ നാഥങ്കല്‍ മാത്രമത്രെ.
\end{malayalam}}
\flushright{\begin{Arabic}
\quranayah[79][45]
\end{Arabic}}
\flushleft{\begin{malayalam}
നീ അതിനെ ഭയക്കുന്നവര്‍ക്കുള്ള താക്കീതുകാരന്‍ മാത്രം!
\end{malayalam}}
\flushright{\begin{Arabic}
\quranayah[79][46]
\end{Arabic}}
\flushleft{\begin{malayalam}
അതിനെ അവര്‍ കാണും നാള്‍, ഇവിടെ ഒരു സായാഹ്നമോ പ്രഭാതമോ അല്ലാതെ താമസിച്ചിട്ടില്ലെന്ന് അവര്‍ക്ക് തോന്നിപ്പോകും.
\end{malayalam}}
\chapter{\textmalayalam{അബസ ( മുഖം ചുളിച്ചു )}}
\begin{Arabic}
\Huge{\centerline{\basmalah}}\end{Arabic}
\flushright{\begin{Arabic}
\quranayah[80][1]
\end{Arabic}}
\flushleft{\begin{malayalam}
അദ്ദേഹം നെറ്റിചുളിച്ചു, മുഖം തിരിച്ചു.
\end{malayalam}}
\flushright{\begin{Arabic}
\quranayah[80][2]
\end{Arabic}}
\flushleft{\begin{malayalam}
കുരുടന്റെ വരവു കാരണം.
\end{malayalam}}
\flushright{\begin{Arabic}
\quranayah[80][3]
\end{Arabic}}
\flushleft{\begin{malayalam}
നിനക്കെന്തറിയാം? ഒരുവേള അവന്‍ വിശുദ്ധി വരിച്ചെങ്കിലോ?
\end{malayalam}}
\flushright{\begin{Arabic}
\quranayah[80][4]
\end{Arabic}}
\flushleft{\begin{malayalam}
അഥവാ, ഉപദേശം ശ്രദ്ധിക്കുകയും ആ ഉപദേശം അയാള്‍ക്ക് ഉപകരിക്കുകയും ചെയ്തേക്കാമല്ലോ.
\end{malayalam}}
\flushright{\begin{Arabic}
\quranayah[80][5]
\end{Arabic}}
\flushleft{\begin{malayalam}
എന്നാല്‍ താന്‍പോരിമ നടിച്ചവനോ;
\end{malayalam}}
\flushright{\begin{Arabic}
\quranayah[80][6]
\end{Arabic}}
\flushleft{\begin{malayalam}
അവന്റെ നേരെ നീ ശ്രദ്ധ തിരിച്ചു.
\end{malayalam}}
\flushright{\begin{Arabic}
\quranayah[80][7]
\end{Arabic}}
\flushleft{\begin{malayalam}
അവന്‍ നന്നായില്ലെങ്കില്‍ നിനക്കെന്ത്?
\end{malayalam}}
\flushright{\begin{Arabic}
\quranayah[80][8]
\end{Arabic}}
\flushleft{\begin{malayalam}
എന്നാല്‍ നിന്നെത്തേടി ഓടി വന്നവനോ,
\end{malayalam}}
\flushright{\begin{Arabic}
\quranayah[80][9]
\end{Arabic}}
\flushleft{\begin{malayalam}
അവന്‍ ദൈവഭയമുള്ളവനാണ്.
\end{malayalam}}
\flushright{\begin{Arabic}
\quranayah[80][10]
\end{Arabic}}
\flushleft{\begin{malayalam}
എന്നിട്ടും നീ അവന്റെ കാര്യത്തില്‍ അശ്രദ്ധ കാണിച്ചു.
\end{malayalam}}
\flushright{\begin{Arabic}
\quranayah[80][11]
\end{Arabic}}
\flushleft{\begin{malayalam}
അറിയുക: ഇതൊരുദ്ബോധനമാ ണ്.
\end{malayalam}}
\flushright{\begin{Arabic}
\quranayah[80][12]
\end{Arabic}}
\flushleft{\begin{malayalam}
അതിനാല്‍ മനസ്സുള്ളവര്‍ ഇതോര്‍ക്കട്ടെ.
\end{malayalam}}
\flushright{\begin{Arabic}
\quranayah[80][13]
\end{Arabic}}
\flushleft{\begin{malayalam}
ആദരണീയമായ ഏടുകളിലാണിതുള്ളത്.
\end{malayalam}}
\flushright{\begin{Arabic}
\quranayah[80][14]
\end{Arabic}}
\flushleft{\begin{malayalam}
ഉന്നതങ്ങളും വിശുദ്ധങ്ങളുമായ ഏടുകളില്‍.
\end{malayalam}}
\flushright{\begin{Arabic}
\quranayah[80][15]
\end{Arabic}}
\flushleft{\begin{malayalam}
ചില സന്ദേശവാഹകരുടെ കൈകളിലാണവ;
\end{malayalam}}
\flushright{\begin{Arabic}
\quranayah[80][16]
\end{Arabic}}
\flushleft{\begin{malayalam}
അവര്‍ മാന്യരും മഹത്തുക്കളുമാണ്.
\end{malayalam}}
\flushright{\begin{Arabic}
\quranayah[80][17]
\end{Arabic}}
\flushleft{\begin{malayalam}
മനുഷ്യന്‍ തുലയട്ടെ. അവനിത്ര നന്ദിയില്ലാത്തവനായതെന്ത്?
\end{malayalam}}
\flushright{\begin{Arabic}
\quranayah[80][18]
\end{Arabic}}
\flushleft{\begin{malayalam}
ഏതൊരു വസ്തുവില്‍ നിന്നാണവനെ പടച്ചത്?
\end{malayalam}}
\flushright{\begin{Arabic}
\quranayah[80][19]
\end{Arabic}}
\flushleft{\begin{malayalam}
ഒരു ബീജ കണത്തില്‍നിന്നാണവനെ സൃഷ്ടിച്ചത്. അങ്ങനെ ക്രമാനുസൃതം രൂപപ്പെടുത്തി.
\end{malayalam}}
\flushright{\begin{Arabic}
\quranayah[80][20]
\end{Arabic}}
\flushleft{\begin{malayalam}
എന്നിട്ട് അല്ലാഹു അവന്ന് വഴി എളുപ്പമാക്കിക്കൊടുത്തു.
\end{malayalam}}
\flushright{\begin{Arabic}
\quranayah[80][21]
\end{Arabic}}
\flushleft{\begin{malayalam}
പിന്നീട് അവനെ മരിപ്പിച്ചു. മറമാടുകയും ചെയ്തു.
\end{malayalam}}
\flushright{\begin{Arabic}
\quranayah[80][22]
\end{Arabic}}
\flushleft{\begin{malayalam}
പിന്നെ അല്ലാഹു ഇഛിക്കുമ്പോള്‍ അവനെ ഉയിര്‍ത്തെഴുന്നേല്‍പിക്കുന്നു.
\end{malayalam}}
\flushright{\begin{Arabic}
\quranayah[80][23]
\end{Arabic}}
\flushleft{\begin{malayalam}
അല്ല, അല്ലാഹു കല്പിച്ചത് അവന്‍ നിര്‍വഹിച്ചില്ല.
\end{malayalam}}
\flushright{\begin{Arabic}
\quranayah[80][24]
\end{Arabic}}
\flushleft{\begin{malayalam}
മനുഷ്യന്‍ തന്റെ ആഹാരത്തെ സംബന്ധിച്ച് ആലോചിക്കട്ടെ.
\end{malayalam}}
\flushright{\begin{Arabic}
\quranayah[80][25]
\end{Arabic}}
\flushleft{\begin{malayalam}
നാം ധാരാളമായി മഴവെള്ളം വീഴ്ത്തി.
\end{malayalam}}
\flushright{\begin{Arabic}
\quranayah[80][26]
\end{Arabic}}
\flushleft{\begin{malayalam}
പിന്നെ നാം മണ്ണ് കീറിപ്പിളര്‍ത്തി.
\end{malayalam}}
\flushright{\begin{Arabic}
\quranayah[80][27]
\end{Arabic}}
\flushleft{\begin{malayalam}
അങ്ങനെ നാമതില്‍ ധാന്യത്തെ മുളപ്പിച്ചു.
\end{malayalam}}
\flushright{\begin{Arabic}
\quranayah[80][28]
\end{Arabic}}
\flushleft{\begin{malayalam}
മുന്തിരിയും പച്ചക്കറികളും.
\end{malayalam}}
\flushright{\begin{Arabic}
\quranayah[80][29]
\end{Arabic}}
\flushleft{\begin{malayalam}
ഒലീവും ഈത്തപ്പനയും.
\end{malayalam}}
\flushright{\begin{Arabic}
\quranayah[80][30]
\end{Arabic}}
\flushleft{\begin{malayalam}
ഇടതൂര്‍ന്ന തോട്ടങ്ങളും.
\end{malayalam}}
\flushright{\begin{Arabic}
\quranayah[80][31]
\end{Arabic}}
\flushleft{\begin{malayalam}
പഴങ്ങളും പുല്‍പടര്‍പ്പുകളും.
\end{malayalam}}
\flushright{\begin{Arabic}
\quranayah[80][32]
\end{Arabic}}
\flushleft{\begin{malayalam}
നിങ്ങള്‍ക്കും നിങ്ങളുടെ കന്നുകാലികള്‍ക്കും ആഹാരമായി.
\end{malayalam}}
\flushright{\begin{Arabic}
\quranayah[80][33]
\end{Arabic}}
\flushleft{\begin{malayalam}
എന്നാല്‍ ആ ഘോര ശബ്ദം വന്നുഭവിച്ചാല്‍.
\end{malayalam}}
\flushright{\begin{Arabic}
\quranayah[80][34]
\end{Arabic}}
\flushleft{\begin{malayalam}
അതുണ്ടാവുന്ന ദിനം മനുഷ്യന്‍ തന്റെ സഹോദരനെ വെടിഞ്ഞോടും.
\end{malayalam}}
\flushright{\begin{Arabic}
\quranayah[80][35]
\end{Arabic}}
\flushleft{\begin{malayalam}
മാതാവിനെയും പിതാവിനെയും.
\end{malayalam}}
\flushright{\begin{Arabic}
\quranayah[80][36]
\end{Arabic}}
\flushleft{\begin{malayalam}
ഭാര്യയെയും മക്കളെയും.
\end{malayalam}}
\flushright{\begin{Arabic}
\quranayah[80][37]
\end{Arabic}}
\flushleft{\begin{malayalam}
അന്ന് അവരിലോരോരുത്തര്‍ക്കും സ്വന്തം കാര്യം നോക്കാനുണ്ടാകും.
\end{malayalam}}
\flushright{\begin{Arabic}
\quranayah[80][38]
\end{Arabic}}
\flushleft{\begin{malayalam}
അന്നു ചില മുഖങ്ങള്‍ പ്രസന്നങ്ങളായിരിക്കും;
\end{malayalam}}
\flushright{\begin{Arabic}
\quranayah[80][39]
\end{Arabic}}
\flushleft{\begin{malayalam}
ചിരിക്കുന്നവയും സന്തോഷപൂര്‍ണ്ണങ്ങളും.
\end{malayalam}}
\flushright{\begin{Arabic}
\quranayah[80][40]
\end{Arabic}}
\flushleft{\begin{malayalam}
മറ്റു ചില മുഖങ്ങള്‍ അന്ന് പൊടി പുരണ്ടിരിക്കും;
\end{malayalam}}
\flushright{\begin{Arabic}
\quranayah[80][41]
\end{Arabic}}
\flushleft{\begin{malayalam}
ഇരുള്‍ മുറ്റിയും.
\end{malayalam}}
\flushright{\begin{Arabic}
\quranayah[80][42]
\end{Arabic}}
\flushleft{\begin{malayalam}
അവര്‍ തന്നെയാണ് സത്യനിഷേധികളും തെമ്മാടികളും.
\end{malayalam}}
\chapter{\textmalayalam{തക് വീര്‍  ( ചുറ്റിപ്പൊതിയല്‍ )}}
\begin{Arabic}
\Huge{\centerline{\basmalah}}\end{Arabic}
\flushright{\begin{Arabic}
\quranayah[81][1]
\end{Arabic}}
\flushleft{\begin{malayalam}
സൂര്യനെ ചുറ്റിപ്പൊതിയുമ്പോള്‍,
\end{malayalam}}
\flushright{\begin{Arabic}
\quranayah[81][2]
\end{Arabic}}
\flushleft{\begin{malayalam}
നക്ഷത്രങ്ങള്‍ ഉതിര്‍ന്നുവീഴുമ്പോള്‍,
\end{malayalam}}
\flushright{\begin{Arabic}
\quranayah[81][3]
\end{Arabic}}
\flushleft{\begin{malayalam}
പര്‍വതങ്ങള്‍ ചലിച്ചു നീങ്ങുമ്പോള്‍,
\end{malayalam}}
\flushright{\begin{Arabic}
\quranayah[81][4]
\end{Arabic}}
\flushleft{\begin{malayalam}
പൂര്‍ണ ഗര്‍ഭിണികളായ ഒട്ടകങ്ങള്‍ പോലും ഉപേക്ഷിക്കപ്പെടുമ്പോള്‍,
\end{malayalam}}
\flushright{\begin{Arabic}
\quranayah[81][5]
\end{Arabic}}
\flushleft{\begin{malayalam}
വന്യമൃഗങ്ങള്‍ ഒരുമിച്ചു കൂടുമ്പോള്‍
\end{malayalam}}
\flushright{\begin{Arabic}
\quranayah[81][6]
\end{Arabic}}
\flushleft{\begin{malayalam}
കടലുകള്‍ കത്തിപ്പടരുമ്പോള്‍,
\end{malayalam}}
\flushright{\begin{Arabic}
\quranayah[81][7]
\end{Arabic}}
\flushleft{\begin{malayalam}
ആത്മാക്കള്‍ ഇണങ്ങിച്ചേരുമ്പോള്‍,
\end{malayalam}}
\flushright{\begin{Arabic}
\quranayah[81][8]
\end{Arabic}}
\flushleft{\begin{malayalam}
കുഴിച്ചുമൂടപ്പെട്ട പെണ്‍കുട്ടിയോട് ചോദിക്കുമ്പോള്‍.
\end{malayalam}}
\flushright{\begin{Arabic}
\quranayah[81][9]
\end{Arabic}}
\flushleft{\begin{malayalam}
ഏതൊരു പാപത്തിന്റെ പേരിലാണ് താന്‍ വധിക്കപ്പെട്ടതെന്ന്.
\end{malayalam}}
\flushright{\begin{Arabic}
\quranayah[81][10]
\end{Arabic}}
\flushleft{\begin{malayalam}
കര്‍മ പുസ്തകത്തിലെ താളുകള്‍ നിവര്‍ത്തുമ്പോള്‍.
\end{malayalam}}
\flushright{\begin{Arabic}
\quranayah[81][11]
\end{Arabic}}
\flushleft{\begin{malayalam}
ആകാശത്തിന്റെ ആവരണം അഴിച്ചുമാറ്റുമ്പോള്‍.
\end{malayalam}}
\flushright{\begin{Arabic}
\quranayah[81][12]
\end{Arabic}}
\flushleft{\begin{malayalam}
നരകത്തീ ആളിക്കത്തുമ്പോള്‍.
\end{malayalam}}
\flushright{\begin{Arabic}
\quranayah[81][13]
\end{Arabic}}
\flushleft{\begin{malayalam}
സ്വര്‍ഗം അരികെ കൊണ്ടുവരുമ്പോള്‍.
\end{malayalam}}
\flushright{\begin{Arabic}
\quranayah[81][14]
\end{Arabic}}
\flushleft{\begin{malayalam}
അന്ന് ഓരോരുത്തനും താന്‍ എന്തുമായാണ് എത്തിയതെന്നറിയും.
\end{malayalam}}
\flushright{\begin{Arabic}
\quranayah[81][15]
\end{Arabic}}
\flushleft{\begin{malayalam}
പിന്‍വാങ്ങിക്കൊണ്ടിരിക്കുന്ന നക്ഷത്രങ്ങള്‍ സാക്ഷി.
\end{malayalam}}
\flushright{\begin{Arabic}
\quranayah[81][16]
\end{Arabic}}
\flushleft{\begin{malayalam}
അവ മുന്നോട്ടു സഞ്ചരിക്കുന്നവയും പിന്നീട് അപ്രത്യക്ഷമാകുന്നവയുമത്രെ.
\end{malayalam}}
\flushright{\begin{Arabic}
\quranayah[81][17]
\end{Arabic}}
\flushleft{\begin{malayalam}
വിടപറയുന്ന രാവ് സാക്ഷി.
\end{malayalam}}
\flushright{\begin{Arabic}
\quranayah[81][18]
\end{Arabic}}
\flushleft{\begin{malayalam}
വിടര്‍ന്നുവരുന്ന പ്രഭാതം സാക്ഷി.
\end{malayalam}}
\flushright{\begin{Arabic}
\quranayah[81][19]
\end{Arabic}}
\flushleft{\begin{malayalam}
ഉറപ്പായും ഇത് ആദരണീയനായ ഒരു ദൂതന്റെ വചനം തന്നെ.
\end{malayalam}}
\flushright{\begin{Arabic}
\quranayah[81][20]
\end{Arabic}}
\flushleft{\begin{malayalam}
പ്രബലനും സിംഹാസനത്തിന്റെ ഉടമയുടെ അടുത്ത് ഉന്നത സ്ഥാനമുള്ളവനുമാണദ്ദേഹം.
\end{malayalam}}
\flushright{\begin{Arabic}
\quranayah[81][21]
\end{Arabic}}
\flushleft{\begin{malayalam}
അവിടെ അനുസരിക്കപ്പെടുന്നവനും വിശ്വസ്തനുമാണ്.
\end{malayalam}}
\flushright{\begin{Arabic}
\quranayah[81][22]
\end{Arabic}}
\flushleft{\begin{malayalam}
നിങ്ങളുടെ കൂട്ടുകാരന്‍ ഭ്രാന്തനല്ല.
\end{malayalam}}
\flushright{\begin{Arabic}
\quranayah[81][23]
\end{Arabic}}
\flushleft{\begin{malayalam}
ഉറപ്പായും അദ്ദേഹം ജിബ്രീലിനെ തെളിഞ്ഞ ചക്രവാളത്തില്‍ വെച്ചു കണ്ടിട്ടുണ്ട്.
\end{malayalam}}
\flushright{\begin{Arabic}
\quranayah[81][24]
\end{Arabic}}
\flushleft{\begin{malayalam}
അദ്ദേഹം അദൃശ്യവാര്‍ത്തകളുടെ കാര്യത്തില്‍ പിശുക്ക് കാട്ടുന്നവനല്ല.
\end{malayalam}}
\flushright{\begin{Arabic}
\quranayah[81][25]
\end{Arabic}}
\flushleft{\begin{malayalam}
ഇത് ശപിക്കപ്പെട്ട പിശാചിന്റെ വചനവുമല്ല.
\end{malayalam}}
\flushright{\begin{Arabic}
\quranayah[81][26]
\end{Arabic}}
\flushleft{\begin{malayalam}
എന്നിട്ടും നിങ്ങളെങ്ങോട്ടാണു പോയിക്കൊണ്ടിരിക്കുന്നത്.
\end{malayalam}}
\flushright{\begin{Arabic}
\quranayah[81][27]
\end{Arabic}}
\flushleft{\begin{malayalam}
ഇത് ലോകര്‍ക്കാകെയുള്ള ഉദ്ബോധനമല്ലാതൊന്നുമല്ല;
\end{malayalam}}
\flushright{\begin{Arabic}
\quranayah[81][28]
\end{Arabic}}
\flushleft{\begin{malayalam}
നിങ്ങളില്‍ നേര്‍വഴിയില്‍ നടക്കാനാഗ്രഹിക്കുന്നവര്‍ക്ക്.
\end{malayalam}}
\flushright{\begin{Arabic}
\quranayah[81][29]
\end{Arabic}}
\flushleft{\begin{malayalam}
എന്നാല്‍ മുഴുലോകരുടെയും നാഥനായ അല്ലാഹു ഇഛിക്കുന്നതല്ലാതൊന്നും നിങ്ങള്‍ക്ക് ആഗ്രഹിക്കാനാവില്ല.
\end{malayalam}}
\chapter{\textmalayalam{ഇന്‍ഫിത്വാര്‍ ( പൊട്ടിക്കീറല്‍ )}}
\begin{Arabic}
\Huge{\centerline{\basmalah}}\end{Arabic}
\flushright{\begin{Arabic}
\quranayah[82][1]
\end{Arabic}}
\flushleft{\begin{malayalam}
ആകാശം പൊട്ടിപ്പിളരുമ്പോള്‍,
\end{malayalam}}
\flushright{\begin{Arabic}
\quranayah[82][2]
\end{Arabic}}
\flushleft{\begin{malayalam}
നക്ഷത്രങ്ങള്‍ ഉതിര്‍ന്നു വീഴുമ്പോള്‍,
\end{malayalam}}
\flushright{\begin{Arabic}
\quranayah[82][3]
\end{Arabic}}
\flushleft{\begin{malayalam}
കടലുകള്‍ കര തകര്‍ത്തൊഴുകുമ്പോള്‍,
\end{malayalam}}
\flushright{\begin{Arabic}
\quranayah[82][4]
\end{Arabic}}
\flushleft{\begin{malayalam}
കുഴിമാടങ്ങള്‍ കീഴ്മേല്‍ മറിയുമ്പോള്‍,
\end{malayalam}}
\flushright{\begin{Arabic}
\quranayah[82][5]
\end{Arabic}}
\flushleft{\begin{malayalam}
ഓരോ ആത്മാവും താന്‍ നേരത്തെ പ്രവര്‍ത്തിച്ചതും പിന്നേക്ക് മാറ്റി വെച്ചതും എന്തെന്നറിയും.
\end{malayalam}}
\flushright{\begin{Arabic}
\quranayah[82][6]
\end{Arabic}}
\flushleft{\begin{malayalam}
അല്ലയോ മനുഷ്യാ, അത്യുദാരനായ നിന്റെ നാഥന്റെ കാര്യത്തില്‍ നിന്നെ ചതിയില്‍ പെടുത്തിയതെന്താണ്?
\end{malayalam}}
\flushright{\begin{Arabic}
\quranayah[82][7]
\end{Arabic}}
\flushleft{\begin{malayalam}
അവനോ, നിന്നെ സൃഷ്ടിക്കുകയും ശ്രദ്ധയോടെ ചിട്ടപ്പെടുത്തുകയും, എല്ലാം സന്തുലിതമാക്കുകയും ചെയ്തവന്‍.
\end{malayalam}}
\flushright{\begin{Arabic}
\quranayah[82][8]
\end{Arabic}}
\flushleft{\begin{malayalam}
താനുദ്ദേശിച്ച വിധം നിന്നെ രൂപപ്പെടുത്തിയവന്‍.
\end{malayalam}}
\flushright{\begin{Arabic}
\quranayah[82][9]
\end{Arabic}}
\flushleft{\begin{malayalam}
അല്ല; എന്നിട്ടും നിങ്ങള്‍ രക്ഷാശിക്ഷാ നടപടികളെ തള്ളിപ്പറയുന്നു.
\end{malayalam}}
\flushright{\begin{Arabic}
\quranayah[82][10]
\end{Arabic}}
\flushleft{\begin{malayalam}
സംശയമില്ല; നിങ്ങളെ നിരീക്ഷിക്കുന്ന ചില മേല്‍നോട്ടക്കാരുണ്ട്
\end{malayalam}}
\flushright{\begin{Arabic}
\quranayah[82][11]
\end{Arabic}}
\flushleft{\begin{malayalam}
സമാദരണീയരായ ചില എഴുത്തുകാര്‍.
\end{malayalam}}
\flushright{\begin{Arabic}
\quranayah[82][12]
\end{Arabic}}
\flushleft{\begin{malayalam}
നിങ്ങള്‍ ചെയ്യുന്നതൊക്കെയും അവരറിയുന്നു.
\end{malayalam}}
\flushright{\begin{Arabic}
\quranayah[82][13]
\end{Arabic}}
\flushleft{\begin{malayalam}
സുകര്‍മികള്‍ സുഖാനുഗ്രഹങ്ങളില്‍ തന്നെയായിരിക്കും; തീര്‍ച്ച.
\end{malayalam}}
\flushright{\begin{Arabic}
\quranayah[82][14]
\end{Arabic}}
\flushleft{\begin{malayalam}
കുറ്റവാളികള്‍ ആളിക്കത്തുന്ന നരകത്തീയിലും.
\end{malayalam}}
\flushright{\begin{Arabic}
\quranayah[82][15]
\end{Arabic}}
\flushleft{\begin{malayalam}
വിധിദിനത്തില്‍ അവരതിലെത്തിച്ചേരും.
\end{malayalam}}
\flushright{\begin{Arabic}
\quranayah[82][16]
\end{Arabic}}
\flushleft{\begin{malayalam}
അവര്‍ക്ക് അതില്‍നിന്ന് മാറി നില്‍ക്കാനാവില്ല.
\end{malayalam}}
\flushright{\begin{Arabic}
\quranayah[82][17]
\end{Arabic}}
\flushleft{\begin{malayalam}
വിധിദിനം എന്തെന്ന് നിനക്കെന്തറിയാം?
\end{malayalam}}
\flushright{\begin{Arabic}
\quranayah[82][18]
\end{Arabic}}
\flushleft{\begin{malayalam}
വീണ്ടും ചോദിക്കട്ടെ: വിധിദിനമെന്തെന്ന് നിനക്കെന്തറിയാം?
\end{malayalam}}
\flushright{\begin{Arabic}
\quranayah[82][19]
\end{Arabic}}
\flushleft{\begin{malayalam}
ആര്‍ക്കും മറ്റൊരാള്‍ക്കുവേണ്ടി ഒന്നും ചെയ്യാനാവാത്ത ദിനമാണത്. അന്ന് തീരുമാനാധികാരമൊക്കെ അല്ലാഹുവിന് മാത്രമായിരിക്കും.
\end{malayalam}}
\chapter{\textmalayalam{മുതഫ്ഫിഫീന്‍ ( അളവില്‍ കുറയ്ക്കുന്നവന്‍ )}}
\begin{Arabic}
\Huge{\centerline{\basmalah}}\end{Arabic}
\flushright{\begin{Arabic}
\quranayah[83][1]
\end{Arabic}}
\flushleft{\begin{malayalam}
കള്ളത്താപ്പുകാര്‍ക്ക് നാശം!
\end{malayalam}}
\flushright{\begin{Arabic}
\quranayah[83][2]
\end{Arabic}}
\flushleft{\begin{malayalam}
അവര്‍ ജനങ്ങളില്‍നിന്ന് അളന്നെടുക്കുമ്പോള്‍ തികവു വരുത്തും.
\end{malayalam}}
\flushright{\begin{Arabic}
\quranayah[83][3]
\end{Arabic}}
\flushleft{\begin{malayalam}
ജനങ്ങള്‍ക്ക് അളന്നോ തൂക്കിയോ കൊടുക്കുമ്പോള്‍ കുറവു വരുത്തുകയും ചെയ്യും.
\end{malayalam}}
\flushright{\begin{Arabic}
\quranayah[83][4]
\end{Arabic}}
\flushleft{\begin{malayalam}
അവരോര്‍ക്കുന്നില്ലേ; തങ്ങള്‍ ഉയിര്‍ത്തെഴുന്നേല്‍പിക്കപ്പെടുന്നവരാണെന്ന്.
\end{malayalam}}
\flushright{\begin{Arabic}
\quranayah[83][5]
\end{Arabic}}
\flushleft{\begin{malayalam}
ഭീകരമായ ഒരു ദിനത്തില്‍.
\end{malayalam}}
\flushright{\begin{Arabic}
\quranayah[83][6]
\end{Arabic}}
\flushleft{\begin{malayalam}
പ്രപഞ്ചനാഥങ്കല്‍ ജനം വന്നു നില്‍ക്കുന്ന ദിനം.
\end{malayalam}}
\flushright{\begin{Arabic}
\quranayah[83][7]
\end{Arabic}}
\flushleft{\begin{malayalam}
സംശയമില്ല; കുറ്റവാളികളുടെ കര്‍മ്മരേഖ സിജ്ജീനില്‍ തന്നെ.
\end{malayalam}}
\flushright{\begin{Arabic}
\quranayah[83][8]
\end{Arabic}}
\flushleft{\begin{malayalam}
സിജ്ജീന്‍ എന്നാല്‍ എന്തെന്ന് നിനക്കെന്തറിയാം?
\end{malayalam}}
\flushright{\begin{Arabic}
\quranayah[83][9]
\end{Arabic}}
\flushleft{\begin{malayalam}
അതൊരു ലിഖിത രേഖയാണ്.
\end{malayalam}}
\flushright{\begin{Arabic}
\quranayah[83][10]
\end{Arabic}}
\flushleft{\begin{malayalam}
അന്നാളില്‍ നാശം സത്യനിഷേധികള്‍ക്കാണ്.
\end{malayalam}}
\flushright{\begin{Arabic}
\quranayah[83][11]
\end{Arabic}}
\flushleft{\begin{malayalam}
അവരോ, പ്രതിഫലനാളിനെ കള്ളമാക്കി തള്ളുന്നവര്‍.
\end{malayalam}}
\flushright{\begin{Arabic}
\quranayah[83][12]
\end{Arabic}}
\flushleft{\begin{malayalam}
അതിക്രമിയും അപരാധിയുമല്ലാതെ ആരും അതിനെ തള്ളിപ്പറയുകയില്ല.
\end{malayalam}}
\flushright{\begin{Arabic}
\quranayah[83][13]
\end{Arabic}}
\flushleft{\begin{malayalam}
നമ്മുടെ സന്ദേശം ഓതിക്കേള്‍പ്പിക്കുമ്പോള്‍ അവന്‍ പറയും: “ഇത് പൂര്‍വികരുടെ പൊട്ടക്കഥകളാണ്.”
\end{malayalam}}
\flushright{\begin{Arabic}
\quranayah[83][14]
\end{Arabic}}
\flushleft{\begin{malayalam}
അല്ല. അവര്‍ ചെയ്തുകൂട്ടുന്ന കുറ്റങ്ങള്‍ അവരുടെ ഹൃദയങ്ങളിന്മേല്‍ കറയായി പറ്റിപ്പിടിച്ചിരിക്കുകയാണ്.
\end{malayalam}}
\flushright{\begin{Arabic}
\quranayah[83][15]
\end{Arabic}}
\flushleft{\begin{malayalam}
നിസ്സംശയം; ആ ദിനത്തിലവര്‍ തങ്ങളുടെ നാഥനെ ദര്‍ശിക്കുന്നത് വിലക്കപ്പെടും.
\end{malayalam}}
\flushright{\begin{Arabic}
\quranayah[83][16]
\end{Arabic}}
\flushleft{\begin{malayalam}
പിന്നെയവര്‍ കത്തിക്കാളുന്ന നരകത്തീയില്‍ കടന്നെരിയും.
\end{malayalam}}
\flushright{\begin{Arabic}
\quranayah[83][17]
\end{Arabic}}
\flushleft{\begin{malayalam}
പിന്നീട് അവരോട് പറയും: നിങ്ങളെന്നും നിഷേധിച്ചുകൊണ്ടിരുന്ന ശിക്ഷയാണിത്.
\end{malayalam}}
\flushright{\begin{Arabic}
\quranayah[83][18]
\end{Arabic}}
\flushleft{\begin{malayalam}
സംശയമില്ല; സുകര്‍മികളുടെ കര്‍മ്മരേഖ ഇല്ലിയ്യീനിലാണ്.
\end{malayalam}}
\flushright{\begin{Arabic}
\quranayah[83][19]
\end{Arabic}}
\flushleft{\begin{malayalam}
ഇല്ലിയ്യീനെ സംബന്ധിച്ച് നിനക്കെന്തറിയാം?
\end{malayalam}}
\flushright{\begin{Arabic}
\quranayah[83][20]
\end{Arabic}}
\flushleft{\begin{malayalam}
അതൊരു ലിഖിത രേഖയാണ്.
\end{malayalam}}
\flushright{\begin{Arabic}
\quranayah[83][21]
\end{Arabic}}
\flushleft{\begin{malayalam}
ദൈവസാമീപ്യം സിദ്ധിച്ചവര്‍ അതിനു സാക്ഷികളായിരിക്കും.
\end{malayalam}}
\flushright{\begin{Arabic}
\quranayah[83][22]
\end{Arabic}}
\flushleft{\begin{malayalam}
സുകര്‍മികള്‍ സുഖാനുഗ്രഹങ്ങളിലായിരിക്കും.
\end{malayalam}}
\flushright{\begin{Arabic}
\quranayah[83][23]
\end{Arabic}}
\flushleft{\begin{malayalam}
ചാരുമഞ്ചങ്ങളിലിരുന്ന് അവരെല്ലാം നോക്കിക്കാണും.
\end{malayalam}}
\flushright{\begin{Arabic}
\quranayah[83][24]
\end{Arabic}}
\flushleft{\begin{malayalam}
അവരുടെ മുഖങ്ങളില്‍ ദിവ്യാനുഗ്രഹങ്ങളുടെ ശോഭ നിനക്കു കണ്ടറിയാം.
\end{malayalam}}
\flushright{\begin{Arabic}
\quranayah[83][25]
\end{Arabic}}
\flushleft{\begin{malayalam}
അടച്ചു മുദ്രവെച്ച പാത്രങ്ങളിലെ പവിത്ര മദ്യം അവര്‍ കുടിപ്പിക്കപ്പെടും.
\end{malayalam}}
\flushright{\begin{Arabic}
\quranayah[83][26]
\end{Arabic}}
\flushleft{\begin{malayalam}
അതിന്റെ മുദ്ര കസ്തൂരികൊണ്ടായിരിക്കും. മത്സരിക്കുന്നവര്‍ അതിനായി മത്സരിക്കട്ടെ.
\end{malayalam}}
\flushright{\begin{Arabic}
\quranayah[83][27]
\end{Arabic}}
\flushleft{\begin{malayalam}
ആ പാനീയത്തിന്റെ ചേരുവ തസ്നീം ആയിരിക്കും.
\end{malayalam}}
\flushright{\begin{Arabic}
\quranayah[83][28]
\end{Arabic}}
\flushleft{\begin{malayalam}
അതോ, ദിവ്യസാന്നിധ്യം സിദ്ധിച്ചവര്‍ കുടിക്കുന്ന ഉറവയാണത്.
\end{malayalam}}
\flushright{\begin{Arabic}
\quranayah[83][29]
\end{Arabic}}
\flushleft{\begin{malayalam}
കുറ്റവാളികള്‍ സത്യവിശ്വാസികളെ കളിയാക്കി ചിരിക്കുമായിരുന്നു.
\end{malayalam}}
\flushright{\begin{Arabic}
\quranayah[83][30]
\end{Arabic}}
\flushleft{\begin{malayalam}
അവരുടെ അരികിലൂടെ നടന്നുപോകുമ്പോള്‍ അവര്‍ പരിഹാസത്തോടെ കണ്ണിറുക്കുമായിരുന്നു.
\end{malayalam}}
\flushright{\begin{Arabic}
\quranayah[83][31]
\end{Arabic}}
\flushleft{\begin{malayalam}
അവര്‍ തങ്ങളുടെ കുടുംബങ്ങളിലേക്ക് രസിച്ചുല്ലസിച്ചാണ് തിരിച്ചു ചെന്നിരുന്നത്.
\end{malayalam}}
\flushright{\begin{Arabic}
\quranayah[83][32]
\end{Arabic}}
\flushleft{\begin{malayalam}
അവര്‍ സത്യവിശ്വാസികളെ കണ്ടാല്‍ പരസ്പരം പറയുമായിരുന്നു: "ഇക്കൂട്ടര്‍ വഴിപിഴച്ചവര്‍ തന്നെ; തീര്‍ച്ച.”
\end{malayalam}}
\flushright{\begin{Arabic}
\quranayah[83][33]
\end{Arabic}}
\flushleft{\begin{malayalam}
സത്യവിശ്വാസികളുടെ മേല്‍നോട്ടക്കാരായി ഇവരെയാരും ചുമതലപ്പെടുത്തിയിട്ടില്ല.
\end{malayalam}}
\flushright{\begin{Arabic}
\quranayah[83][34]
\end{Arabic}}
\flushleft{\begin{malayalam}
എന്നാലന്ന് ആ സത്യവിശ്വാസികള്‍ സത്യനിഷേധികളെ കളിയാക്കിച്ചിരിക്കും.
\end{malayalam}}
\flushright{\begin{Arabic}
\quranayah[83][35]
\end{Arabic}}
\flushleft{\begin{malayalam}
അവര്‍ ചാരുമഞ്ചങ്ങളിലിരുന്ന് ഇവരെ നോക്കിക്കൊണ്ടിരിക്കും;
\end{malayalam}}
\flushright{\begin{Arabic}
\quranayah[83][36]
\end{Arabic}}
\flushleft{\begin{malayalam}
സത്യനിഷേധികള്‍ക്ക് അവര്‍ പ്രവര്‍ത്തിച്ചുകൊണ്ടിരുന്നതിന് പ്രതിഫലം കിട്ടിക്കഴിഞ്ഞോ എന്ന്.
\end{malayalam}}
\chapter{\textmalayalam{ഇന്ഷിഖാഖ് ( പൊട്ടിപിളരല്‍ )}}
\begin{Arabic}
\Huge{\centerline{\basmalah}}\end{Arabic}
\flushright{\begin{Arabic}
\quranayah[84][1]
\end{Arabic}}
\flushleft{\begin{malayalam}
ആകാശം പൊട്ടിപ്പിളരുമ്പോള്‍!
\end{malayalam}}
\flushright{\begin{Arabic}
\quranayah[84][2]
\end{Arabic}}
\flushleft{\begin{malayalam}
അത് തന്റെ നാഥന്ന് കീഴ്പ്പെടുമ്പോള്‍!- അത് അങ്ങനെ ചെയ്യാന്‍ കടപ്പെട്ടിരിക്കുന്നുവല്ലോ.
\end{malayalam}}
\flushright{\begin{Arabic}
\quranayah[84][3]
\end{Arabic}}
\flushleft{\begin{malayalam}
ഭൂമി പരത്തപ്പെടുമ്പോള്‍-
\end{malayalam}}
\flushright{\begin{Arabic}
\quranayah[84][4]
\end{Arabic}}
\flushleft{\begin{malayalam}
അതിനകത്തുള്ളതിനെ പുറത്തേക്ക് തള്ളുകയും അത് ശൂന്യമായിത്തീരുകയും.
\end{malayalam}}
\flushright{\begin{Arabic}
\quranayah[84][5]
\end{Arabic}}
\flushleft{\begin{malayalam}
അത് അതിന്റെ നാഥന്ന് കീഴ്പ്പെടുകയും ചെയ്യുമ്പോള്‍! -അങ്ങനെ ചെയ്യാന്‍ അത് കടപ്പെട്ടിരിക്കുന്നുവല്ലോ.
\end{malayalam}}
\flushright{\begin{Arabic}
\quranayah[84][6]
\end{Arabic}}
\flushleft{\begin{malayalam}
അല്ലയോ മനുഷ്യാ; നീ നിന്റെ നാഥനിലേക്ക് കടുത്ത ക്ളേശത്തോടെ ചെന്നെത്തുന്നവനാണ്; അങ്ങനെ അവനുമായി കണ്ടുമുട്ടുന്നവനും.
\end{malayalam}}
\flushright{\begin{Arabic}
\quranayah[84][7]
\end{Arabic}}
\flushleft{\begin{malayalam}
എന്നാല്‍ തന്റെ കര്‍മപുസ്തകം വലതു കയ്യില്‍ നല്‍കപ്പെടുന്നവനോ;
\end{malayalam}}
\flushright{\begin{Arabic}
\quranayah[84][8]
\end{Arabic}}
\flushleft{\begin{malayalam}
അവന് നിസ്സാരമായ വിചാരണയേ ഉണ്ടാവുകയുള്ളൂ.
\end{malayalam}}
\flushright{\begin{Arabic}
\quranayah[84][9]
\end{Arabic}}
\flushleft{\begin{malayalam}
അവന്‍ തന്റെ വീട്ടുകാരുടെ അടുത്തേക്ക് സന്തോഷത്തോടെ മടങ്ങിച്ചെല്ലും.
\end{malayalam}}
\flushright{\begin{Arabic}
\quranayah[84][10]
\end{Arabic}}
\flushleft{\begin{malayalam}
എന്നാല്‍ കര്‍മപുസ്തകം തന്റെ പിന്‍ഭാഗത്തൂടെ നല്‍കപ്പെടുന്നവനോ;
\end{malayalam}}
\flushright{\begin{Arabic}
\quranayah[84][11]
\end{Arabic}}
\flushleft{\begin{malayalam}
അവന്‍ “നാശമേ”യെന്ന് വിലപിച്ചു കൊണ്ടിരിക്കും.
\end{malayalam}}
\flushright{\begin{Arabic}
\quranayah[84][12]
\end{Arabic}}
\flushleft{\begin{malayalam}
ആളിക്കത്തുന്ന നരകത്തീയില്‍ കത്തിയെരിയും.
\end{malayalam}}
\flushright{\begin{Arabic}
\quranayah[84][13]
\end{Arabic}}
\flushleft{\begin{malayalam}
തീര്‍ച്ചയായും അവന്‍ തന്റെ കുടുംബക്കാര്‍ക്കിടയില്‍ ആഹ്ളാദത്തോടെ കഴിയുന്നവനായിരുന്നു.
\end{malayalam}}
\flushright{\begin{Arabic}
\quranayah[84][14]
\end{Arabic}}
\flushleft{\begin{malayalam}
താന്‍ മടങ്ങിവരില്ലെന്നാണ് അവന്‍ കരുതിയത്.
\end{malayalam}}
\flushright{\begin{Arabic}
\quranayah[84][15]
\end{Arabic}}
\flushleft{\begin{malayalam}
എന്നാല്‍ ഉറപ്പായും അവന്റെ നാഥന്‍ അവനെ സൂക്ഷ്മമായി നിരീക്ഷിക്കുന്നവനായിരുന്നു.
\end{malayalam}}
\flushright{\begin{Arabic}
\quranayah[84][16]
\end{Arabic}}
\flushleft{\begin{malayalam}
ഞാനിതാ സത്യംചെയ്യുന്നു; സൂര്യാസ്തമയ സമയത്തെ ശോഭകൊണ്ട്.
\end{malayalam}}
\flushright{\begin{Arabic}
\quranayah[84][17]
\end{Arabic}}
\flushleft{\begin{malayalam}
രാത്രിയും അതുള്‍ക്കൊള്ളുന്നതുകൊണ്ടും.
\end{malayalam}}
\flushright{\begin{Arabic}
\quranayah[84][18]
\end{Arabic}}
\flushleft{\begin{malayalam}
ചന്ദ്രന്‍ സാക്ഷി- അതു പൂര്‍ണത പ്രാപിക്കുമ്പോള്‍.
\end{malayalam}}
\flushright{\begin{Arabic}
\quranayah[84][19]
\end{Arabic}}
\flushleft{\begin{malayalam}
നിശ്ചയമായും നിങ്ങള്‍ പടിപടിയായി പുരോഗമിച്ചുകൊണ്ടിരിക്കും.
\end{malayalam}}
\flushright{\begin{Arabic}
\quranayah[84][20]
\end{Arabic}}
\flushleft{\begin{malayalam}
എന്നിട്ടും അവര്‍ക്കെന്തു പറ്റി, അവര്‍ വിശ്വസിക്കുന്നില്ലല്ലോ?
\end{malayalam}}
\flushright{\begin{Arabic}
\quranayah[84][21]
\end{Arabic}}
\flushleft{\begin{malayalam}
ഖുര്‍ആന്‍ ഓതിക്കേള്‍പിക്കുമ്പോള്‍ സാഷ്ടാംഗം പ്രണമിക്കുന്നുമില്ല.
\end{malayalam}}
\flushright{\begin{Arabic}
\quranayah[84][22]
\end{Arabic}}
\flushleft{\begin{malayalam}
എന്നല്ല; സത്യനിഷേധികള്‍ അതിനെ കളവാക്കി തള്ളുകയാണ്.
\end{malayalam}}
\flushright{\begin{Arabic}
\quranayah[84][23]
\end{Arabic}}
\flushleft{\begin{malayalam}
അവര്‍ മനസ്സില്‍ സൂക്ഷിക്കുന്നവയൊക്കെയും നന്നായറിയുന്നവനാണ് അല്ലാഹു.
\end{malayalam}}
\flushright{\begin{Arabic}
\quranayah[84][24]
\end{Arabic}}
\flushleft{\begin{malayalam}
അതിനാല്‍ നീ അവര്‍ക്ക് നോവേറിയ ശിക്ഷയെ സംബന്ധിച്ച് വിവരമറിയിക്കുക.
\end{malayalam}}
\flushright{\begin{Arabic}
\quranayah[84][25]
\end{Arabic}}
\flushleft{\begin{malayalam}
സത്യവിശ്വാസം സ്വീകരിക്കുകയും സല്‍ക്കര്‍മങ്ങള്‍ പ്രവര്‍ത്തിക്കുകയും ചെയ്തവര്‍ക്കൊഴികെ. അവര്‍ക്ക് അറുതിയില്ലാത്ത പ്രതിഫലമുണ്ട്.
\end{malayalam}}
\chapter{\textmalayalam{ബുറൂജ് ( നക്ഷത്രമണ്ഡലങ്ങള്‍ )}}
\begin{Arabic}
\Huge{\centerline{\basmalah}}\end{Arabic}
\flushright{\begin{Arabic}
\quranayah[85][1]
\end{Arabic}}
\flushleft{\begin{malayalam}
നക്ഷത്രങ്ങളുള്ള ആകാശം സാക്ഷി.
\end{malayalam}}
\flushright{\begin{Arabic}
\quranayah[85][2]
\end{Arabic}}
\flushleft{\begin{malayalam}
വാഗ്ദാനം ചെയ്യപ്പെട്ട ആ ദിനം സാക്ഷി.
\end{malayalam}}
\flushright{\begin{Arabic}
\quranayah[85][3]
\end{Arabic}}
\flushleft{\begin{malayalam}
സാക്ഷിയും സാക്ഷ്യം നില്‍ക്കപ്പെടുന്ന കാര്യവും സാക്ഷി.
\end{malayalam}}
\flushright{\begin{Arabic}
\quranayah[85][4]
\end{Arabic}}
\flushleft{\begin{malayalam}
കിടങ്ങിന്റെ ആള്‍ക്കാര്‍ നശിച്ചിരിക്കുന്നു.
\end{malayalam}}
\flushright{\begin{Arabic}
\quranayah[85][5]
\end{Arabic}}
\flushleft{\begin{malayalam}
വിറക് നിറച്ച തീക്കുണ്ഡത്തിന്റെ ആള്‍ക്കാര്‍.
\end{malayalam}}
\flushright{\begin{Arabic}
\quranayah[85][6]
\end{Arabic}}
\flushleft{\begin{malayalam}
അവര്‍ അതിന്റെ മേല്‍നോട്ടക്കാരായി ഇരുന്ന സന്ദര്‍ഭം.
\end{malayalam}}
\flushright{\begin{Arabic}
\quranayah[85][7]
\end{Arabic}}
\flushleft{\begin{malayalam}
സത്യവിശ്വാസികള്‍ക്കെതിരെ തങ്ങള്‍ ചെയ്തുകൊണ്ടിരുന്നതിന് അവര്‍ സാക്ഷികളായിരുന്നു.
\end{malayalam}}
\flushright{\begin{Arabic}
\quranayah[85][8]
\end{Arabic}}
\flushleft{\begin{malayalam}
അവര്‍ക്ക് വിശ്വാസികളുടെ മേല്‍ ഒരു കുറ്റവും ആരോപിക്കാനുണ്ടായിരുന്നില്ല; സ്തുത്യര്‍ഹനും അജയ്യനുമായ അല്ലാഹുവില്‍ വിശ്വസിച്ചു എന്നതല്ലാതെ.
\end{malayalam}}
\flushright{\begin{Arabic}
\quranayah[85][9]
\end{Arabic}}
\flushleft{\begin{malayalam}
അവനോ, ആകാശ ഭൂമികളുടെ മേല്‍ ആധിപത്യമുള്ളവനത്രെ. അല്ലാഹു എല്ലാ കാര്യങ്ങള്‍ക്കും സാക്ഷിയാണ്.
\end{malayalam}}
\flushright{\begin{Arabic}
\quranayah[85][10]
\end{Arabic}}
\flushleft{\begin{malayalam}
സത്യവിശ്വാസികളെയും വിശ്വാസിനികളെയും മര്‍ദിക്കുകയും എന്നിട്ട് പശ്ചാത്തപിക്കാതിരിക്കുകയും ചെയ്തവരുണ്ടല്ലോ, ഉറപ്പായും അവര്‍ക്ക് നരകശിക്ഷയുണ്ട്. ചുട്ടു കരിക്കുന്ന ശിക്ഷ.
\end{malayalam}}
\flushright{\begin{Arabic}
\quranayah[85][11]
\end{Arabic}}
\flushleft{\begin{malayalam}
എന്നാല്‍ സത്യവിശ്വാസം സ്വീകരിച്ച് സല്‍ക്കര്‍മങ്ങള്‍ പ്രവര്‍ത്തിക്കുന്നവര്‍ക്ക് താഴ്ഭാഗത്തൂടെ ആറുകളൊഴുകുന്ന സ്വര്‍ഗീയാരാമങ്ങളാണുള്ളത്. അതത്രെ അതിമഹത്തായ വിജയം!
\end{malayalam}}
\flushright{\begin{Arabic}
\quranayah[85][12]
\end{Arabic}}
\flushleft{\begin{malayalam}
തീര്‍ച്ചയായും നിന്റെ നാഥന്റെ പിടുത്തം കഠിനം തന്നെ.
\end{malayalam}}
\flushright{\begin{Arabic}
\quranayah[85][13]
\end{Arabic}}
\flushleft{\begin{malayalam}
സൃഷ്ടികര്‍മം ആരംഭിച്ചതും ആവര്‍ത്തിക്കുന്നതും അവനാണ്.
\end{malayalam}}
\flushright{\begin{Arabic}
\quranayah[85][14]
\end{Arabic}}
\flushleft{\begin{malayalam}
അവന്‍ ഏറെ പൊറുക്കുന്നവനാണ്. സ്നേഹിക്കുന്നവനും.
\end{malayalam}}
\flushright{\begin{Arabic}
\quranayah[85][15]
\end{Arabic}}
\flushleft{\begin{malayalam}
സിംഹാസനത്തിനുടമയും മഹാനും.
\end{malayalam}}
\flushright{\begin{Arabic}
\quranayah[85][16]
\end{Arabic}}
\flushleft{\begin{malayalam}
താന്‍ ഉദ്ദേശിക്കുന്നതൊക്കെ ചെയ്യുന്നവനും.
\end{malayalam}}
\flushright{\begin{Arabic}
\quranayah[85][17]
\end{Arabic}}
\flushleft{\begin{malayalam}
ആ സൈന്യത്തിന്റെ കഥ നിനക്കറിയാമോ?
\end{malayalam}}
\flushright{\begin{Arabic}
\quranayah[85][18]
\end{Arabic}}
\flushleft{\begin{malayalam}
ഫറോവയുടെയും ഥമൂദിന്റെയും കഥ.
\end{malayalam}}
\flushright{\begin{Arabic}
\quranayah[85][19]
\end{Arabic}}
\flushleft{\begin{malayalam}
എന്നാല്‍; സത്യനിഷേധികള്‍ എല്ലാം കള്ളമാക്കി തള്ളുന്നതില്‍ വ്യാപൃതരാണ്.
\end{malayalam}}
\flushright{\begin{Arabic}
\quranayah[85][20]
\end{Arabic}}
\flushleft{\begin{malayalam}
അല്ലാഹു അവരെ പിറകിലൂടെ വലയം ചെയ്തുകൊണ്ടിരിക്കുന്നവനാണ്.
\end{malayalam}}
\flushright{\begin{Arabic}
\quranayah[85][21]
\end{Arabic}}
\flushleft{\begin{malayalam}
എന്നാലിത് അതിമഹത്തായ ഖുര്‍ആനാണ്.
\end{malayalam}}
\flushright{\begin{Arabic}
\quranayah[85][22]
\end{Arabic}}
\flushleft{\begin{malayalam}
സുരക്ഷിതമായ ഒരു ഫലകത്തിലാണ് ഇതുള്ളത്.
\end{malayalam}}
\chapter{\textmalayalam{ത്വാരിഖ് ( രാത്രിയില്‍ വരുന്നത് )}}
\begin{Arabic}
\Huge{\centerline{\basmalah}}\end{Arabic}
\flushright{\begin{Arabic}
\quranayah[86][1]
\end{Arabic}}
\flushleft{\begin{malayalam}
ആകാശം സാക്ഷി. രാവില്‍ പ്രത്യക്ഷപ്പെടുന്നതും സാക്ഷി.
\end{malayalam}}
\flushright{\begin{Arabic}
\quranayah[86][2]
\end{Arabic}}
\flushleft{\begin{malayalam}
രാവില്‍ പ്രത്യക്ഷപ്പെടുന്നതെന്തെന്ന് നിനക്കെന്തറിയാം?
\end{malayalam}}
\flushright{\begin{Arabic}
\quranayah[86][3]
\end{Arabic}}
\flushleft{\begin{malayalam}
തുളച്ചുകയറും നക്ഷത്രമാണത്.
\end{malayalam}}
\flushright{\begin{Arabic}
\quranayah[86][4]
\end{Arabic}}
\flushleft{\begin{malayalam}
ഒരുമേല്‍നോട്ടക്കാരനില്ലാതെ ഈ ലോകത്ത് ഒരു മനുഷ്യനുമില്ല.
\end{malayalam}}
\flushright{\begin{Arabic}
\quranayah[86][5]
\end{Arabic}}
\flushleft{\begin{malayalam}
മനുഷ്യന്‍ ചിന്തിച്ചു നോക്കട്ടെ; ഏതില്‍നിന്നാണവന്‍ സൃഷ്ടിക്കപ്പെട്ടതെന്ന്.
\end{malayalam}}
\flushright{\begin{Arabic}
\quranayah[86][6]
\end{Arabic}}
\flushleft{\begin{malayalam}
അവന്‍ സൃഷ്ടിക്കപ്പെട്ടത് സ്രവിക്കപ്പെടുന്ന വെള്ളത്തില്‍നിന്നാണ്.
\end{malayalam}}
\flushright{\begin{Arabic}
\quranayah[86][7]
\end{Arabic}}
\flushleft{\begin{malayalam}
മുതുകെല്ലിന്റെയും മാറെല്ലിന്റെയും ഇടയിലാണതിന്റെ ഉറവിടം.
\end{malayalam}}
\flushright{\begin{Arabic}
\quranayah[86][8]
\end{Arabic}}
\flushleft{\begin{malayalam}
അവനെ തിരികെ കൊണ്ടുവരാന്‍ കഴിവുറ്റവനാണ് അല്ലാഹു.
\end{malayalam}}
\flushright{\begin{Arabic}
\quranayah[86][9]
\end{Arabic}}
\flushleft{\begin{malayalam}
രഹസ്യങ്ങള്‍ വിലയിരുത്തപ്പെടും ദിനമാണതുണ്ടാവുക.
\end{malayalam}}
\flushright{\begin{Arabic}
\quranayah[86][10]
\end{Arabic}}
\flushleft{\begin{malayalam}
അന്നവന് എന്തെങ്കിലും കഴിവോ സഹായിയോ ഉണ്ടാവില്ല.
\end{malayalam}}
\flushright{\begin{Arabic}
\quranayah[86][11]
\end{Arabic}}
\flushleft{\begin{malayalam}
മഴപൊഴിക്കും മാനം സാക്ഷി.
\end{malayalam}}
\flushright{\begin{Arabic}
\quranayah[86][12]
\end{Arabic}}
\flushleft{\begin{malayalam}
സസ്യങ്ങള്‍ കിളുര്‍പ്പിക്കും ഭൂമി സാക്ഷി!
\end{malayalam}}
\flushright{\begin{Arabic}
\quranayah[86][13]
\end{Arabic}}
\flushleft{\begin{malayalam}
നിശ്ചയമായും ഇതൊരു നിര്‍ണായക വചനമാണ്.
\end{malayalam}}
\flushright{\begin{Arabic}
\quranayah[86][14]
\end{Arabic}}
\flushleft{\begin{malayalam}
ഇത് തമാശയല്ല.
\end{malayalam}}
\flushright{\begin{Arabic}
\quranayah[86][15]
\end{Arabic}}
\flushleft{\begin{malayalam}
അവര്‍ കുതന്ത്രം പ്രയോഗിച്ചുകൊണ്ടിരിക്കും.
\end{malayalam}}
\flushright{\begin{Arabic}
\quranayah[86][16]
\end{Arabic}}
\flushleft{\begin{malayalam}
നാമും തന്ത്രം പ്രയോഗിക്കും.
\end{malayalam}}
\flushright{\begin{Arabic}
\quranayah[86][17]
\end{Arabic}}
\flushleft{\begin{malayalam}
അതിനാല്‍ സത്യനിഷേധികള്‍ക്ക് നീ അവധി നല്‍കുക. ഇത്തിരി നേരം അവര്‍ക്ക് സമയമനുവദിക്കുക.
\end{malayalam}}
\chapter{\textmalayalam{അഅ്അലാ ( അത്യുന്നതന്‍ )}}
\begin{Arabic}
\Huge{\centerline{\basmalah}}\end{Arabic}
\flushright{\begin{Arabic}
\quranayah[87][1]
\end{Arabic}}
\flushleft{\begin{malayalam}
അത്യുന്നതനായ നിന്റെ നാഥന്റെ നാമം കീര്‍ത്തിക്കുക.
\end{malayalam}}
\flushright{\begin{Arabic}
\quranayah[87][2]
\end{Arabic}}
\flushleft{\begin{malayalam}
അവനോ സൃഷ്ടിച്ച് സന്തുലിതമാക്കിയവന്‍.
\end{malayalam}}
\flushright{\begin{Arabic}
\quranayah[87][3]
\end{Arabic}}
\flushleft{\begin{malayalam}
ക്രമീകരിച്ച് നേര്‍വഴി കാണിച്ചവന്‍;
\end{malayalam}}
\flushright{\begin{Arabic}
\quranayah[87][4]
\end{Arabic}}
\flushleft{\begin{malayalam}
മേച്ചില്‍പ്പുറങ്ങള്‍ ഒരുക്കിയവന്‍.
\end{malayalam}}
\flushright{\begin{Arabic}
\quranayah[87][5]
\end{Arabic}}
\flushleft{\begin{malayalam}
എന്നിട്ടവനതിനെ ഉണങ്ങിക്കരിഞ്ഞ ചവറാക്കി.
\end{malayalam}}
\flushright{\begin{Arabic}
\quranayah[87][6]
\end{Arabic}}
\flushleft{\begin{malayalam}
നിനക്കു നാം ഓതിത്തരും. നീയത് മറക്കുകയില്ല;
\end{malayalam}}
\flushright{\begin{Arabic}
\quranayah[87][7]
\end{Arabic}}
\flushleft{\begin{malayalam}
അല്ലാഹു ഇഛിച്ചതൊഴികെ. പരസ്യവും രഹസ്യവും അവനറിയുന്നു.
\end{malayalam}}
\flushright{\begin{Arabic}
\quranayah[87][8]
\end{Arabic}}
\flushleft{\begin{malayalam}
എളുപ്പമായ വഴി നിനക്കു നാം ഒരുക്കിത്തരാം.
\end{malayalam}}
\flushright{\begin{Arabic}
\quranayah[87][9]
\end{Arabic}}
\flushleft{\begin{malayalam}
അതിനാല്‍ നീ ഉദ്ബോധിപ്പിക്കുക- ഉദ്ബോധനം ഉപകരിക്കുമെങ്കില്‍!
\end{malayalam}}
\flushright{\begin{Arabic}
\quranayah[87][10]
\end{Arabic}}
\flushleft{\begin{malayalam}
ദൈവഭയമുള്ളവന്‍ ഉദ്ബോധനം ഉള്‍ക്കൊള്ളും.
\end{malayalam}}
\flushright{\begin{Arabic}
\quranayah[87][11]
\end{Arabic}}
\flushleft{\begin{malayalam}
കൊടിയ നിര്‍ഭാഗ്യവാന്‍ അതില്‍ നിന്ന് അകലുകയും ചെയ്യും.
\end{malayalam}}
\flushright{\begin{Arabic}
\quranayah[87][12]
\end{Arabic}}
\flushleft{\begin{malayalam}
അവനോ, കഠിനമായ നരകത്തീയില്‍ കിടന്നെരിയുന്നവന്‍.
\end{malayalam}}
\flushright{\begin{Arabic}
\quranayah[87][13]
\end{Arabic}}
\flushleft{\begin{malayalam}
പിന്നീട് അവനതില്‍ മരിക്കുകയില്ല; ജീവിക്കുകയുമില്ല.
\end{malayalam}}
\flushright{\begin{Arabic}
\quranayah[87][14]
\end{Arabic}}
\flushleft{\begin{malayalam}
തീര്‍ച്ചയായും വിശുദ്ധി വരിച്ചവന്‍ വിജയിച്ചു.
\end{malayalam}}
\flushright{\begin{Arabic}
\quranayah[87][15]
\end{Arabic}}
\flushleft{\begin{malayalam}
അവന്‍ തന്റെ നാഥന്റെ നാമമോര്‍ത്തു. അങ്ങനെ അവന്‍ നമസ്കരിച്ചു.
\end{malayalam}}
\flushright{\begin{Arabic}
\quranayah[87][16]
\end{Arabic}}
\flushleft{\begin{malayalam}
എന്നാല്‍ നിങ്ങള്‍ ഈ ലോക ജീവിതത്തിനാണ് പ്രാമുഖ്യം നല്‍കുന്നത്.
\end{malayalam}}
\flushright{\begin{Arabic}
\quranayah[87][17]
\end{Arabic}}
\flushleft{\begin{malayalam}
പരലോകമാണ് ഏറ്റം ഉത്തമവും ഏറെ ശാശ്വതവും.
\end{malayalam}}
\flushright{\begin{Arabic}
\quranayah[87][18]
\end{Arabic}}
\flushleft{\begin{malayalam}
സംശയം വേണ്ടാ, ഇത് പൂര്‍വ വേദങ്ങളിലുമുണ്ട്.
\end{malayalam}}
\flushright{\begin{Arabic}
\quranayah[87][19]
\end{Arabic}}
\flushleft{\begin{malayalam}
അഥവാ, ഇബ്റാഹീമിന്റെയും മൂസായുടെയും ഗ്രന്ഥത്താളുകളില്‍!
\end{malayalam}}
\chapter{\textmalayalam{ഗാശിയ ( മൂടുന്ന സംഭവം )}}
\begin{Arabic}
\Huge{\centerline{\basmalah}}\end{Arabic}
\flushright{\begin{Arabic}
\quranayah[88][1]
\end{Arabic}}
\flushleft{\begin{malayalam}
ആവരണം ചെയ്യുന്ന മഹാവിപത്തിന്റെ വാര്‍ത്ത നിനക്കു വന്നെത്തിയോ?
\end{malayalam}}
\flushright{\begin{Arabic}
\quranayah[88][2]
\end{Arabic}}
\flushleft{\begin{malayalam}
അന്ന് ചില മുഖങ്ങള്‍ പേടിച്ചരണ്ടവയായിരിക്കും.
\end{malayalam}}
\flushright{\begin{Arabic}
\quranayah[88][3]
\end{Arabic}}
\flushleft{\begin{malayalam}
അധ്വാനിച്ച് തളര്‍ന്നവയും.
\end{malayalam}}
\flushright{\begin{Arabic}
\quranayah[88][4]
\end{Arabic}}
\flushleft{\begin{malayalam}
ചുട്ടെരിയും നരകത്തിലവര്‍ ചെന്നെത്തും.
\end{malayalam}}
\flushright{\begin{Arabic}
\quranayah[88][5]
\end{Arabic}}
\flushleft{\begin{malayalam}
തിളച്ചു മറിയുന്ന ഉറവയില്‍നിന്നാണവര്‍ക്ക് കുടിക്കാന്‍ കിട്ടുക.
\end{malayalam}}
\flushright{\begin{Arabic}
\quranayah[88][6]
\end{Arabic}}
\flushleft{\begin{malayalam}
കയ്പുള്ള മുള്‍ചെടിയില്‍ നിന്നല്ലാതെ അവര്‍ക്കൊരാഹാരവുമില്ല.
\end{malayalam}}
\flushright{\begin{Arabic}
\quranayah[88][7]
\end{Arabic}}
\flushleft{\begin{malayalam}
അത് ശരീരത്തെ പോഷിപ്പിക്കില്ല. വിശപ്പിനു ശമനമേകുകയുമില്ല.
\end{malayalam}}
\flushright{\begin{Arabic}
\quranayah[88][8]
\end{Arabic}}
\flushleft{\begin{malayalam}
എന്നാല്‍ മറ്റു ചില മുഖങ്ങള്‍ അന്ന് പ്രസന്നങ്ങളായിരിക്കും.
\end{malayalam}}
\flushright{\begin{Arabic}
\quranayah[88][9]
\end{Arabic}}
\flushleft{\begin{malayalam}
തങ്ങളുടെ കര്‍മങ്ങളെക്കുറിച്ച് സംതൃപ്തരും.
\end{malayalam}}
\flushright{\begin{Arabic}
\quranayah[88][10]
\end{Arabic}}
\flushleft{\begin{malayalam}
അവര്‍ അത്യുന്നതമായ സ്വര്‍ഗീയാരാമത്തിലായിരിക്കും.
\end{malayalam}}
\flushright{\begin{Arabic}
\quranayah[88][11]
\end{Arabic}}
\flushleft{\begin{malayalam}
വിടുവാക്കുകള്‍ അവിടെ കേള്‍ക്കുകയില്ല.
\end{malayalam}}
\flushright{\begin{Arabic}
\quranayah[88][12]
\end{Arabic}}
\flushleft{\begin{malayalam}
അവിടെ ഒഴുകുന്ന അരുവിയുണ്ട്.
\end{malayalam}}
\flushright{\begin{Arabic}
\quranayah[88][13]
\end{Arabic}}
\flushleft{\begin{malayalam}
ഉയര്‍ത്തിയൊരുക്കിയ മഞ്ചങ്ങളും.
\end{malayalam}}
\flushright{\begin{Arabic}
\quranayah[88][14]
\end{Arabic}}
\flushleft{\begin{malayalam}
തയ്യാറാക്കിവെച്ച പാനപാത്രങ്ങളും.
\end{malayalam}}
\flushright{\begin{Arabic}
\quranayah[88][15]
\end{Arabic}}
\flushleft{\begin{malayalam}
നിരത്തിവെച്ച തലയണകളും.
\end{malayalam}}
\flushright{\begin{Arabic}
\quranayah[88][16]
\end{Arabic}}
\flushleft{\begin{malayalam}
പരത്തിവെച്ച പരവതാനികളും.
\end{malayalam}}
\flushright{\begin{Arabic}
\quranayah[88][17]
\end{Arabic}}
\flushleft{\begin{malayalam}
അവര്‍ നോക്കുന്നില്ലേ? ഒട്ടകത്തെ; അതിനെ എങ്ങനെ സൃഷ്ടിച്ചുവെന്ന്?
\end{malayalam}}
\flushright{\begin{Arabic}
\quranayah[88][18]
\end{Arabic}}
\flushleft{\begin{malayalam}
ആകാശത്തെ; അതിനെ എവ്വിധം ഉയര്‍ത്തിയെന്ന്?
\end{malayalam}}
\flushright{\begin{Arabic}
\quranayah[88][19]
\end{Arabic}}
\flushleft{\begin{malayalam}
പര്‍വതങ്ങളെ, അവയെ എങ്ങനെ സ്ഥാപിച്ചുവെന്ന്?
\end{malayalam}}
\flushright{\begin{Arabic}
\quranayah[88][20]
\end{Arabic}}
\flushleft{\begin{malayalam}
ഭൂമിയെ, അതിനെ എങ്ങനെ വിശാലമാക്കിയെന്ന്?
\end{malayalam}}
\flushright{\begin{Arabic}
\quranayah[88][21]
\end{Arabic}}
\flushleft{\begin{malayalam}
അതിനാല്‍ നീ ഉദ്ബോധിപ്പിക്കുക. നീ ഒരുദ്ബോധകന്‍ മാത്രമാണ്.
\end{malayalam}}
\flushright{\begin{Arabic}
\quranayah[88][22]
\end{Arabic}}
\flushleft{\begin{malayalam}
നീ അവരുടെ മേല്‍ നിര്‍ബന്ധം ചെലുത്തുന്നവനല്ല.
\end{malayalam}}
\flushright{\begin{Arabic}
\quranayah[88][23]
\end{Arabic}}
\flushleft{\begin{malayalam}
ആര്‍ പിന്തിരിയുകയും സത്യത്തെ തള്ളിപ്പറയുകയും ചെയ്യുന്നുവോ,
\end{malayalam}}
\flushright{\begin{Arabic}
\quranayah[88][24]
\end{Arabic}}
\flushleft{\begin{malayalam}
അവനെ അല്ലാഹു കഠിനമായി ശിക്ഷിക്കും.
\end{malayalam}}
\flushright{\begin{Arabic}
\quranayah[88][25]
\end{Arabic}}
\flushleft{\begin{malayalam}
നിശ്ചയമായും നമ്മുടെ അടുത്തേക്കാണ് അവരുടെ മടക്കം.
\end{malayalam}}
\flushright{\begin{Arabic}
\quranayah[88][26]
\end{Arabic}}
\flushleft{\begin{malayalam}
പിന്നെ അവരുടെ വിചാരണയും നമ്മുടെ ചുമതലയിലാണ്
\end{malayalam}}
\chapter{\textmalayalam{ഫജ്ര്‍ ( പ്രഭാതം )}}
\begin{Arabic}
\Huge{\centerline{\basmalah}}\end{Arabic}
\flushright{\begin{Arabic}
\quranayah[89][1]
\end{Arabic}}
\flushleft{\begin{malayalam}
പ്രഭാതം സാക്ഷി.
\end{malayalam}}
\flushright{\begin{Arabic}
\quranayah[89][2]
\end{Arabic}}
\flushleft{\begin{malayalam}
പത്തു രാവുകള്‍ സാക്ഷി.
\end{malayalam}}
\flushright{\begin{Arabic}
\quranayah[89][3]
\end{Arabic}}
\flushleft{\begin{malayalam}
ഇരട്ടയും ഒറ്റയും സാക്ഷി.
\end{malayalam}}
\flushright{\begin{Arabic}
\quranayah[89][4]
\end{Arabic}}
\flushleft{\begin{malayalam}
രാവു സാക്ഷി- അതു കടന്നുപോയിക്കൊണ്ടിരിക്കെ.
\end{malayalam}}
\flushright{\begin{Arabic}
\quranayah[89][5]
\end{Arabic}}
\flushleft{\begin{malayalam}
കാര്യമറിയുന്നവന് അവയില്‍ ശപഥമുണ്ടോ?
\end{malayalam}}
\flushright{\begin{Arabic}
\quranayah[89][6]
\end{Arabic}}
\flushleft{\begin{malayalam}
ആദ് ജനതയെ നിന്റെ നാഥന്‍ എന്തു ചെയ്തുവെന്ന് നീ കണ്ടില്ലേ?
\end{malayalam}}
\flushright{\begin{Arabic}
\quranayah[89][7]
\end{Arabic}}
\flushleft{\begin{malayalam}
ഉന്നതസ്തൂപങ്ങളുടെ ഉടമകളായ ഇറം ഗോത്രത്തെ?
\end{malayalam}}
\flushright{\begin{Arabic}
\quranayah[89][8]
\end{Arabic}}
\flushleft{\begin{malayalam}
അവരെപ്പോലെ ശക്തരായൊരു ജനത മറ്റൊരു നാട്ടിലും സൃഷ്ടിക്കപ്പെട്ടിട്ടില്ല.
\end{malayalam}}
\flushright{\begin{Arabic}
\quranayah[89][9]
\end{Arabic}}
\flushleft{\begin{malayalam}
താഴ്വരകളില്‍ പാറവെട്ടിപ്പൊളിച്ച് പാര്‍പ്പിടങ്ങളുണ്ടാക്കിയ ഥമൂദ് ഗോത്രത്തെയും.
\end{malayalam}}
\flushright{\begin{Arabic}
\quranayah[89][10]
\end{Arabic}}
\flushleft{\begin{malayalam}
ആണികളുടെ ആളായ ഫറവോനെയും.
\end{malayalam}}
\flushright{\begin{Arabic}
\quranayah[89][11]
\end{Arabic}}
\flushleft{\begin{malayalam}
അവരോ, ആ നാടുകളില്‍ അതിക്രമം പ്രവര്‍ത്തിച്ചവരായിരുന്നു.
\end{malayalam}}
\flushright{\begin{Arabic}
\quranayah[89][12]
\end{Arabic}}
\flushleft{\begin{malayalam}
അവരവിടെ കുഴപ്പം പെരുപ്പിച്ചു.
\end{malayalam}}
\flushright{\begin{Arabic}
\quranayah[89][13]
\end{Arabic}}
\flushleft{\begin{malayalam}
അപ്പോള്‍ നിന്റെ നാഥന്‍ അവര്‍ക്കുമേല്‍ ശിക്ഷയുടെ ചാട്ടവാര്‍ വര്‍ഷിച്ചു.
\end{malayalam}}
\flushright{\begin{Arabic}
\quranayah[89][14]
\end{Arabic}}
\flushleft{\begin{malayalam}
നിന്റെ നാഥന്‍ പതിസ്ഥലത്തു തന്നെയുണ്ട്; തീര്‍ച്ച.
\end{malayalam}}
\flushright{\begin{Arabic}
\quranayah[89][15]
\end{Arabic}}
\flushleft{\begin{malayalam}
എന്നാല്‍ മനുഷ്യനെ അവന്റെ നാഥന്‍ പരീക്ഷിക്കുകയും, അങ്ങനെ അവനെ ആദരിക്കുകയും അനുഗ്രഹിക്കുകയും ചെയ്താല്‍ അവന്‍ പറയും: “എന്റെ നാഥന്‍ എന്നെ ആദരിച്ചിരിക്കുന്നു.”
\end{malayalam}}
\flushright{\begin{Arabic}
\quranayah[89][16]
\end{Arabic}}
\flushleft{\begin{malayalam}
എന്നാല്‍ അല്ലാഹു അവനെ പരീക്ഷിക്കുകയും, അങ്ങനെ അവന്റെ ജീവിതവിഭവം പരിമിതപ്പെടുത്തുകയും ചെയ്താലോ, അവന്‍ പറയും: “എന്റെ നാഥന്‍ എന്നെ നിന്ദിച്ചിരിക്കുന്നു.”
\end{malayalam}}
\flushright{\begin{Arabic}
\quranayah[89][17]
\end{Arabic}}
\flushleft{\begin{malayalam}
കാര്യം അതല്ല; നിങ്ങള്‍ അനാഥയെ പരിഗണിക്കുന്നില്ല.
\end{malayalam}}
\flushright{\begin{Arabic}
\quranayah[89][18]
\end{Arabic}}
\flushleft{\begin{malayalam}
അഗതിക്ക് അന്നം നല്‍കാന്‍ പ്രേരിപ്പിക്കുന്നുമില്ല.
\end{malayalam}}
\flushright{\begin{Arabic}
\quranayah[89][19]
\end{Arabic}}
\flushleft{\begin{malayalam}
പാരമ്പര്യമായിക്കിട്ടിയ സ്വത്ത് വാരിക്കൂട്ടി വെട്ടിവിഴുങ്ങുകയും ചെയ്യുന്നു.
\end{malayalam}}
\flushright{\begin{Arabic}
\quranayah[89][20]
\end{Arabic}}
\flushleft{\begin{malayalam}
ധനത്തെ നിങ്ങള്‍ അതിരറ്റ് സ്നേഹിക്കുന്നു.
\end{malayalam}}
\flushright{\begin{Arabic}
\quranayah[89][21]
\end{Arabic}}
\flushleft{\begin{malayalam}
അതല്ല; ഭൂമിയാകെ ഇടിച്ചു നിരപ്പാക്കുകയും,
\end{malayalam}}
\flushright{\begin{Arabic}
\quranayah[89][22]
\end{Arabic}}
\flushleft{\begin{malayalam}
നിന്റെ നാഥനും അണിയണിയായി മലക്കുകളും വരികയും,
\end{malayalam}}
\flushright{\begin{Arabic}
\quranayah[89][23]
\end{Arabic}}
\flushleft{\begin{malayalam}
അന്ന് നരകത്തെ നിങ്ങളുടെ അടുത്തേക്ക് കൊണ്ടുവരികയും ചെയ്യുമ്പോള്‍; അന്ന് മനുഷ്യന് എല്ലാം ഓര്‍മവരും. ആ സമയത്ത് ഓര്‍മ വന്നിട്ടെന്തു കാര്യം?
\end{malayalam}}
\flushright{\begin{Arabic}
\quranayah[89][24]
\end{Arabic}}
\flushleft{\begin{malayalam}
അവന്‍ പറയും: അയ്യോ, എന്റെ ഈ ജീവിതത്തിനായി ഞാന്‍ നേരത്തെ ചെയ്തുവെച്ചിരുന്നെങ്കില്‍.
\end{malayalam}}
\flushright{\begin{Arabic}
\quranayah[89][25]
\end{Arabic}}
\flushleft{\begin{malayalam}
അന്നാളില്‍ അല്ലാഹു ശിക്ഷിക്കും വിധം മറ്റാരും ശിക്ഷിക്കുകയില്ല.
\end{malayalam}}
\flushright{\begin{Arabic}
\quranayah[89][26]
\end{Arabic}}
\flushleft{\begin{malayalam}
അവന്‍ പിടിച്ചുകെട്ടുംപോലെ മറ്റാരും പിടിച്ചുകെട്ടുകയുമില്ല.
\end{malayalam}}
\flushright{\begin{Arabic}
\quranayah[89][27]
\end{Arabic}}
\flushleft{\begin{malayalam}
അല്ലയോ ശാന്തി നേടിയ ആത്മാവേ.
\end{malayalam}}
\flushright{\begin{Arabic}
\quranayah[89][28]
\end{Arabic}}
\flushleft{\begin{malayalam}
നീ നിന്റെ നാഥങ്കലേക്ക് തൃപ്തിപ്പെട്ടവനായും തൃപ്തി നേടിയവനായും തിരിച്ചു ചെല്ലുക.
\end{malayalam}}
\flushright{\begin{Arabic}
\quranayah[89][29]
\end{Arabic}}
\flushleft{\begin{malayalam}
അങ്ങനെ എന്റെ ഉത്തമ ദാസന്മാരുടെ കൂട്ടത്തില്‍ പ്രവേശിച്ചു കൊള്ളുക.
\end{malayalam}}
\flushright{\begin{Arabic}
\quranayah[89][30]
\end{Arabic}}
\flushleft{\begin{malayalam}
എന്റെ സ്വര്‍ഗത്തില്‍ പ്രവേശിച്ചുകൊള്ളുക.
\end{malayalam}}
\chapter{\textmalayalam{ബലദ് ( രാജ്യം )}}
\begin{Arabic}
\Huge{\centerline{\basmalah}}\end{Arabic}
\flushright{\begin{Arabic}
\quranayah[90][1]
\end{Arabic}}
\flushleft{\begin{malayalam}
അങ്ങനെയല്ല; ഈ മക്കാനഗരം സാക്ഷി.
\end{malayalam}}
\flushright{\begin{Arabic}
\quranayah[90][2]
\end{Arabic}}
\flushleft{\begin{malayalam}
നീ ഈ നഗരത്തില്‍ താമസിക്കുന്നവനല്ലോ.
\end{malayalam}}
\flushright{\begin{Arabic}
\quranayah[90][3]
\end{Arabic}}
\flushleft{\begin{malayalam}
ജനയിതാവും അവന്‍ ജന്മമേകിയതും സാക്ഷി.
\end{malayalam}}
\flushright{\begin{Arabic}
\quranayah[90][4]
\end{Arabic}}
\flushleft{\begin{malayalam}
നിശ്ചയം; നാം മനുഷ്യനെ സൃഷ്ടിച്ചത് ക്ളേശമനുഭവിക്കുന്നവനായാണ്.
\end{malayalam}}
\flushright{\begin{Arabic}
\quranayah[90][5]
\end{Arabic}}
\flushleft{\begin{malayalam}
തന്നെ പിടികൂടാനാര്‍ക്കും കഴിയില്ലെന്നാണോ അവന്‍ കരുതുന്നത്?
\end{malayalam}}
\flushright{\begin{Arabic}
\quranayah[90][6]
\end{Arabic}}
\flushleft{\begin{malayalam}
അവന്‍ അവകാശപ്പെട്ടു; താന്‍ ധാരാളം ധനം തുലച്ചെന്ന്.
\end{malayalam}}
\flushright{\begin{Arabic}
\quranayah[90][7]
\end{Arabic}}
\flushleft{\begin{malayalam}
അവന്‍ കരുതുന്നുവോ; അവനെ ആരും കാണുന്നില്ലെന്ന്.
\end{malayalam}}
\flushright{\begin{Arabic}
\quranayah[90][8]
\end{Arabic}}
\flushleft{\begin{malayalam}
അവനു നാം കണ്ണിണകള്‍ നല്‍കിയില്ലേ?;
\end{malayalam}}
\flushright{\begin{Arabic}
\quranayah[90][9]
\end{Arabic}}
\flushleft{\begin{malayalam}
നാവും ചുണ്ടിണകളും?
\end{malayalam}}
\flushright{\begin{Arabic}
\quranayah[90][10]
\end{Arabic}}
\flushleft{\begin{malayalam}
തെളിഞ്ഞ രണ്ടു വഴികള്‍ നാമവന് കാണിച്ചുകൊടുത്തില്ലേ?
\end{malayalam}}
\flushright{\begin{Arabic}
\quranayah[90][11]
\end{Arabic}}
\flushleft{\begin{malayalam}
എന്നിട്ടും അവന്‍ മലമ്പാത താണ്ടിക്കടന്നില്ല.
\end{malayalam}}
\flushright{\begin{Arabic}
\quranayah[90][12]
\end{Arabic}}
\flushleft{\begin{malayalam}
മലമ്പാത എന്തെന്ന് നിനക്കെന്തറിയാം?
\end{malayalam}}
\flushright{\begin{Arabic}
\quranayah[90][13]
\end{Arabic}}
\flushleft{\begin{malayalam}
അത് അടിമയുടെ മോചനമാണ്.
\end{malayalam}}
\flushright{\begin{Arabic}
\quranayah[90][14]
\end{Arabic}}
\flushleft{\begin{malayalam}
അല്ലെങ്കില്‍ കൊടും വറുതി നാളിലെ അന്നദാനം.
\end{malayalam}}
\flushright{\begin{Arabic}
\quranayah[90][15]
\end{Arabic}}
\flushleft{\begin{malayalam}
അടുത്ത ബന്ധുവായ അനാഥയ്ക്ക്.
\end{malayalam}}
\flushright{\begin{Arabic}
\quranayah[90][16]
\end{Arabic}}
\flushleft{\begin{malayalam}
അല്ലെങ്കില്‍ പട്ടിണിക്കാരനായ മണ്ണുപുരണ്ട അഗതിക്ക്.
\end{malayalam}}
\flushright{\begin{Arabic}
\quranayah[90][17]
\end{Arabic}}
\flushleft{\begin{malayalam}
പിന്നെ സത്യവിശ്വാസം സ്വീകരിക്കുകയും ക്ഷമയും കാരുണ്യവും പരസ്പരം ഉപദേശിക്കുകയും ചെയ്തവരില്‍ ഉള്‍പ്പെടലുമാണ്.
\end{malayalam}}
\flushright{\begin{Arabic}
\quranayah[90][18]
\end{Arabic}}
\flushleft{\begin{malayalam}
അവര്‍ തന്നെയാണ് വലതു പക്ഷക്കാര്‍.
\end{malayalam}}
\flushright{\begin{Arabic}
\quranayah[90][19]
\end{Arabic}}
\flushleft{\begin{malayalam}
നമ്മുടെ സൂക്തങ്ങളെ തള്ളിപ്പറഞ്ഞവരോ, അവര്‍ ഇടതുപക്ഷക്കാരും.
\end{malayalam}}
\flushright{\begin{Arabic}
\quranayah[90][20]
\end{Arabic}}
\flushleft{\begin{malayalam}
അവര്‍ക്കുമേല്‍ മൂടപ്പെട്ട നരകമുണ്ട്.
\end{malayalam}}
\chapter{\textmalayalam{ശംസ് ( സൂര്യന്‍ )}}
\begin{Arabic}
\Huge{\centerline{\basmalah}}\end{Arabic}
\flushright{\begin{Arabic}
\quranayah[91][1]
\end{Arabic}}
\flushleft{\begin{malayalam}
സൂര്യനും അതിന്റെ ശോഭയും സാക്ഷി.
\end{malayalam}}
\flushright{\begin{Arabic}
\quranayah[91][2]
\end{Arabic}}
\flushleft{\begin{malayalam}
ചന്ദ്രന്‍ സാക്ഷി, അത് സൂര്യനെ പിന്തുടരുമ്പോള്‍!
\end{malayalam}}
\flushright{\begin{Arabic}
\quranayah[91][3]
\end{Arabic}}
\flushleft{\begin{malayalam}
പകല്‍ സാക്ഷി, അത് സൂര്യനെ തെളിയിച്ചുകാണിക്കുമ്പോള്‍!
\end{malayalam}}
\flushright{\begin{Arabic}
\quranayah[91][4]
\end{Arabic}}
\flushleft{\begin{malayalam}
രാവു സാക്ഷി, അത് സൂര്യനെ മൂടുമ്പോള്‍!
\end{malayalam}}
\flushright{\begin{Arabic}
\quranayah[91][5]
\end{Arabic}}
\flushleft{\begin{malayalam}
ആകാശവും അതിനെ നിര്‍മിച്ചു നിര്‍ത്തിയതും സാക്ഷി.
\end{malayalam}}
\flushright{\begin{Arabic}
\quranayah[91][6]
\end{Arabic}}
\flushleft{\begin{malayalam}
ഭൂമിയും അതിനെ പരത്തിയതും സാക്ഷി.
\end{malayalam}}
\flushright{\begin{Arabic}
\quranayah[91][7]
\end{Arabic}}
\flushleft{\begin{malayalam}
ആത്മാവും അതിനെ ക്രമപ്പെടുത്തിയതും സാക്ഷി.
\end{malayalam}}
\flushright{\begin{Arabic}
\quranayah[91][8]
\end{Arabic}}
\flushleft{\begin{malayalam}
അങ്ങനെ അതിന് ധര്‍മത്തെയും അധര്‍മത്തെയും സംബന്ധിച്ച ബോധം നല്‍കിയതും.
\end{malayalam}}
\flushright{\begin{Arabic}
\quranayah[91][9]
\end{Arabic}}
\flushleft{\begin{malayalam}
തീര്‍ച്ചയായും അത്മാവിനെ സംസ്കരിച്ചവന്‍ വിജയിച്ചു.
\end{malayalam}}
\flushright{\begin{Arabic}
\quranayah[91][10]
\end{Arabic}}
\flushleft{\begin{malayalam}
അതിനെ മലിനമാക്കിയവന്‍ പരാജയപ്പെട്ടു.
\end{malayalam}}
\flushright{\begin{Arabic}
\quranayah[91][11]
\end{Arabic}}
\flushleft{\begin{malayalam}
ഥമൂദ് ഗോത്രം ധിക്കാരം കാരണം സത്യത്തെ തള്ളിക്കളഞ്ഞു.
\end{malayalam}}
\flushright{\begin{Arabic}
\quranayah[91][12]
\end{Arabic}}
\flushleft{\begin{malayalam}
അവരിലെ പരമ ദുഷ്ടന്‍ ഇറങ്ങിത്തിരിച്ചപ്പോള്‍.
\end{malayalam}}
\flushright{\begin{Arabic}
\quranayah[91][13]
\end{Arabic}}
\flushleft{\begin{malayalam}
ദൈവദൂതന്‍ അവരോട് പറഞ്ഞു: “ഇത് അല്ലാഹുവിന്റെ ഒട്ടകം. അതിന്റെ ജലപാനം 1 തടയാതിരിക്കുക.”
\end{malayalam}}
\flushright{\begin{Arabic}
\quranayah[91][14]
\end{Arabic}}
\flushleft{\begin{malayalam}
അവരദ്ദേഹത്തെ ധിക്കരിച്ചു. ഒട്ടകത്തെ അറുത്തു. അവരുടെ പാപം കാരണം അവരുടെ നാഥന്‍ അവരെ ഒന്നടങ്കം നശിപ്പിച്ചു. ശിക്ഷ അവര്‍ക്കെല്ലാം ഒരുപോലെ നല്കുകയും ചെയ്തു.
\end{malayalam}}
\flushright{\begin{Arabic}
\quranayah[91][15]
\end{Arabic}}
\flushleft{\begin{malayalam}
ഈ നടപടിയുടെ പരിണതി അവനൊട്ടും ഭയപ്പെടുന്നില്ല.
\end{malayalam}}
\chapter{\textmalayalam{ലൈല്‍ ( രാത്രി )}}
\begin{Arabic}
\Huge{\centerline{\basmalah}}\end{Arabic}
\flushright{\begin{Arabic}
\quranayah[92][1]
\end{Arabic}}
\flushleft{\begin{malayalam}
രാത്രി സാക്ഷി, അത് പ്രപഞ്ചത്തെ മൂടുമ്പോള്‍.
\end{malayalam}}
\flushright{\begin{Arabic}
\quranayah[92][2]
\end{Arabic}}
\flushleft{\begin{malayalam}
പകല്‍ സാക്ഷി, അത് തെളിയുമ്പോള്‍.
\end{malayalam}}
\flushright{\begin{Arabic}
\quranayah[92][3]
\end{Arabic}}
\flushleft{\begin{malayalam}
ആണിനെയും പെണ്ണിനെയും സൃഷ്ടിച്ചതു സാക്ഷി.
\end{malayalam}}
\flushright{\begin{Arabic}
\quranayah[92][4]
\end{Arabic}}
\flushleft{\begin{malayalam}
തീര്‍ച്ചയായും നിങ്ങളുടെ പ്രവര്‍ത്തനം പലവിധമാണ്.
\end{malayalam}}
\flushright{\begin{Arabic}
\quranayah[92][5]
\end{Arabic}}
\flushleft{\begin{malayalam}
അതിനാല്‍ ആര്‍ ദാനം നല്‍കുകയും ഭക്തനാവുകയും,
\end{malayalam}}
\flushright{\begin{Arabic}
\quranayah[92][6]
\end{Arabic}}
\flushleft{\begin{malayalam}
അത്യുത്തമമായതിനെ സത്യപ്പെടുത്തുകയും ചെയ്യുന്നുവോ.
\end{malayalam}}
\flushright{\begin{Arabic}
\quranayah[92][7]
\end{Arabic}}
\flushleft{\begin{malayalam}
അവനു നാം ഏറ്റം എളുപ്പമായതിലേക്ക് വഴിയൊരുക്കിക്കൊടുക്കും.
\end{malayalam}}
\flushright{\begin{Arabic}
\quranayah[92][8]
\end{Arabic}}
\flushleft{\begin{malayalam}
എന്നാല്‍ ആര്‍ പിശുക്കുകാണിക്കുകയും സ്വയം പൂര്‍ണതനടിക്കുകയും,
\end{malayalam}}
\flushright{\begin{Arabic}
\quranayah[92][9]
\end{Arabic}}
\flushleft{\begin{malayalam}
അത്യുത്തമമായതിനെ തള്ളിപ്പറയുകയും ചെയ്യുന്നുവോ,
\end{malayalam}}
\flushright{\begin{Arabic}
\quranayah[92][10]
\end{Arabic}}
\flushleft{\begin{malayalam}
അവനെ നാം ഏറ്റം ക്ളേശകരമായതില്‍ കൊണ്ടെത്തിക്കും.
\end{malayalam}}
\flushright{\begin{Arabic}
\quranayah[92][11]
\end{Arabic}}
\flushleft{\begin{malayalam}
അവന്‍ നാശത്തിനിരയാകുമ്പോള്‍ അവന്റെ ധനം അവന്ന് ഉപകരിക്കുകയില്ല.
\end{malayalam}}
\flushright{\begin{Arabic}
\quranayah[92][12]
\end{Arabic}}
\flushleft{\begin{malayalam}
സംശയമില്ല; നാമാണ് നേര്‍വഴി കാണിച്ചു തരേണ്ടത്.
\end{malayalam}}
\flushright{\begin{Arabic}
\quranayah[92][13]
\end{Arabic}}
\flushleft{\begin{malayalam}
തീര്‍ച്ചയായും നമ്മുക്കുള്ളതാണ് പരലോകവും ഈ ലോകവും.
\end{malayalam}}
\flushright{\begin{Arabic}
\quranayah[92][14]
\end{Arabic}}
\flushleft{\begin{malayalam}
അതിനാല്‍ കത്തിയെരിയും നരകത്തീയിനെക്കുറിച്ച് ഞാന്‍ നിങ്ങള്‍ക്ക് മുന്നറിയിപ്പ് നല്‍കിയിരിക്കുന്നു.
\end{malayalam}}
\flushright{\begin{Arabic}
\quranayah[92][15]
\end{Arabic}}
\flushleft{\begin{malayalam}
പരമ നിര്‍ഭാഗ്യവാനല്ലാതെ അതില്‍ പ്രവേശിക്കുകയില്ല.
\end{malayalam}}
\flushright{\begin{Arabic}
\quranayah[92][16]
\end{Arabic}}
\flushleft{\begin{malayalam}
സത്യത്തെ തള്ളിക്കളഞ്ഞവനും അതില്‍നിന്ന് പിന്മാറിയവനുമാണവന്‍.
\end{malayalam}}
\flushright{\begin{Arabic}
\quranayah[92][17]
\end{Arabic}}
\flushleft{\begin{malayalam}
പരമഭക്തന്‍ അതില്‍നിന്ന് അകറ്റപ്പെടും.
\end{malayalam}}
\flushright{\begin{Arabic}
\quranayah[92][18]
\end{Arabic}}
\flushleft{\begin{malayalam}
ധനം വ്യയം ചെയ്ത് വിശുദ്ധി വരിക്കുന്നവനാണവന്‍.
\end{malayalam}}
\flushright{\begin{Arabic}
\quranayah[92][19]
\end{Arabic}}
\flushleft{\begin{malayalam}
പ്രത്യുപകാരം നല്‍കപ്പെടേണ്ട ഒരൌദാര്യവും അവന്റെ വശം ആര്‍ക്കുമില്ല.
\end{malayalam}}
\flushright{\begin{Arabic}
\quranayah[92][20]
\end{Arabic}}
\flushleft{\begin{malayalam}
അത്യുന്നതനായ തന്റെ നാഥന്റെ പ്രീതിയെ സംബന്ധിച്ച പ്രതീക്ഷയല്ലാതെ.
\end{malayalam}}
\flushright{\begin{Arabic}
\quranayah[92][21]
\end{Arabic}}
\flushleft{\begin{malayalam}
വഴിയെ അയാള്‍ സംതൃപ്തനാകും; തീര്‍ച്ച.
\end{malayalam}}
\chapter{\textmalayalam{ളുഹാ ( പൂര്‍വ്വാഹ്നം )}}
\begin{Arabic}
\Huge{\centerline{\basmalah}}\end{Arabic}
\flushright{\begin{Arabic}
\quranayah[93][1]
\end{Arabic}}
\flushleft{\begin{malayalam}
പകലിന്റെ ആദ്യപാതി സാക്ഷി.
\end{malayalam}}
\flushright{\begin{Arabic}
\quranayah[93][2]
\end{Arabic}}
\flushleft{\begin{malayalam}
രാവു സാക്ഷി; അത് പ്രശാന്തമായാല്‍.
\end{malayalam}}
\flushright{\begin{Arabic}
\quranayah[93][3]
\end{Arabic}}
\flushleft{\begin{malayalam}
നിന്റെ നാഥന്‍ നിന്നെ വെടിഞ്ഞിട്ടില്ല. വെറുത്തിട്ടുമില്ല.
\end{malayalam}}
\flushright{\begin{Arabic}
\quranayah[93][4]
\end{Arabic}}
\flushleft{\begin{malayalam}
തീര്‍ച്ചയായും വരാനുള്ളതാണ് വന്നെത്തിയതിനെക്കാള്‍ നിനക്കുത്തമം.
\end{malayalam}}
\flushright{\begin{Arabic}
\quranayah[93][5]
\end{Arabic}}
\flushleft{\begin{malayalam}
വൈകാതെ തന്നെ നിന്റെ നാഥന്‍ നിനക്കു നല്‍കും; അപ്പോള്‍ നീ സംതൃപ്തനാകും.
\end{malayalam}}
\flushright{\begin{Arabic}
\quranayah[93][6]
\end{Arabic}}
\flushleft{\begin{malayalam}
നിന്നെ അനാഥനായി കണ്ടപ്പോള്‍ അവന്‍ നിനക്ക് അഭയമേകിയില്ലേ?
\end{malayalam}}
\flushright{\begin{Arabic}
\quranayah[93][7]
\end{Arabic}}
\flushleft{\begin{malayalam}
നിന്നെ വഴിയറിയാത്തവനായി കണ്ടപ്പോള്‍ അവന്‍ നിന്നെ നേര്‍വഴിയിലാക്കിയില്ലേ?
\end{malayalam}}
\flushright{\begin{Arabic}
\quranayah[93][8]
\end{Arabic}}
\flushleft{\begin{malayalam}
നിന്നെ ദരിദ്രനായി കണ്ടപ്പോള്‍ അവന്‍ നിന്നെ സമ്പന്നനാക്കിയില്ലേ?
\end{malayalam}}
\flushright{\begin{Arabic}
\quranayah[93][9]
\end{Arabic}}
\flushleft{\begin{malayalam}
അതിനാല്‍ അനാഥയോട് നീ കാഠിന്യം കാട്ടരുത്.
\end{malayalam}}
\flushright{\begin{Arabic}
\quranayah[93][10]
\end{Arabic}}
\flushleft{\begin{malayalam}
ചോദിച്ചു വരുന്നവനെ വിരട്ടിയോട്ടരുത്.
\end{malayalam}}
\flushright{\begin{Arabic}
\quranayah[93][11]
\end{Arabic}}
\flushleft{\begin{malayalam}
നിന്റെ നാഥന്റെ അനുഗ്രഹത്തെ കീര്‍ത്തിച്ചുകൊള്ളുക.
\end{malayalam}}
\chapter{\textmalayalam{ശര്‍ഹ് ( വിശാലമാക്കല്‍ )}}
\begin{Arabic}
\Huge{\centerline{\basmalah}}\end{Arabic}
\flushright{\begin{Arabic}
\quranayah[94][1]
\end{Arabic}}
\flushleft{\begin{malayalam}
നിന്റെ ഹൃദയം നിനക്കു നാം വിശാലമാക്കിയില്ലേ?
\end{malayalam}}
\flushright{\begin{Arabic}
\quranayah[94][2]
\end{Arabic}}
\flushleft{\begin{malayalam}
നിന്റെ ഭാരം നിന്നില്‍ നിന്നിറക്കി വെച്ചില്ലേ?
\end{malayalam}}
\flushright{\begin{Arabic}
\quranayah[94][3]
\end{Arabic}}
\flushleft{\begin{malayalam}
നിന്റെ മുതുകിനെ ഞെരിച്ചുകൊണ്ടിരുന്ന ഭാരം.
\end{malayalam}}
\flushright{\begin{Arabic}
\quranayah[94][4]
\end{Arabic}}
\flushleft{\begin{malayalam}
നിന്റെ കീര്‍ത്തി നാം ഉയര്‍ത്തിത്തരികയും ചെയ്തു.
\end{malayalam}}
\flushright{\begin{Arabic}
\quranayah[94][5]
\end{Arabic}}
\flushleft{\begin{malayalam}
അതിനാല്‍ തീര്‍ച്ചയായും പ്രയാസത്തോടൊപ്പം എളുപ്പവുമുണ്ട്.
\end{malayalam}}
\flushright{\begin{Arabic}
\quranayah[94][6]
\end{Arabic}}
\flushleft{\begin{malayalam}
നിശ്ചയം, പ്രയാസത്തോടൊപ്പമാണ് എളുപ്പം.
\end{malayalam}}
\flushright{\begin{Arabic}
\quranayah[94][7]
\end{Arabic}}
\flushleft{\begin{malayalam}
അതിനാല്‍ ഒന്നില്‍ നിന്നൊഴിവായാല്‍ മറ്റൊന്നില്‍ മുഴുകുക.
\end{malayalam}}
\flushright{\begin{Arabic}
\quranayah[94][8]
\end{Arabic}}
\flushleft{\begin{malayalam}
നിന്റെ നാഥനില്‍ പ്രതീക്ഷ അര്‍പ്പിക്കുകയും ചെയ്യുക.
\end{malayalam}}
\chapter{\textmalayalam{തീന്‍ ( അത്തി )}}
\begin{Arabic}
\Huge{\centerline{\basmalah}}\end{Arabic}
\flushright{\begin{Arabic}
\quranayah[95][1]
\end{Arabic}}
\flushleft{\begin{malayalam}
അത്തിയും ഒലീവും സാക്ഷി.
\end{malayalam}}
\flushright{\begin{Arabic}
\quranayah[95][2]
\end{Arabic}}
\flushleft{\begin{malayalam}
സീനാമല സാക്ഷി.
\end{malayalam}}
\flushright{\begin{Arabic}
\quranayah[95][3]
\end{Arabic}}
\flushleft{\begin{malayalam}
നിര്‍ഭീതമായ ഈ മക്കാനഗരം സാക്ഷി.
\end{malayalam}}
\flushright{\begin{Arabic}
\quranayah[95][4]
\end{Arabic}}
\flushleft{\begin{malayalam}
തീര്‍ച്ചയായും മനുഷ്യനെ നാം മികവുറ്റ ഘടനയില്‍ സൃഷ്ടിച്ചു.
\end{malayalam}}
\flushright{\begin{Arabic}
\quranayah[95][5]
\end{Arabic}}
\flushleft{\begin{malayalam}
പിന്നെ നാമവനെ പതിതരില്‍ പതിതനാക്കി.
\end{malayalam}}
\flushright{\begin{Arabic}
\quranayah[95][6]
\end{Arabic}}
\flushleft{\begin{malayalam}
സത്യവിശ്വാസം സ്വീകരിച്ചവരെയും സല്‍ക്കര്‍മങ്ങള്‍ പ്രവര്‍ത്തിച്ചവരെയുമൊഴികെ. അവര്‍ക്ക്, അറുതിയില്ലാത്ത പ്രതിഫലമുണ്ട്.
\end{malayalam}}
\flushright{\begin{Arabic}
\quranayah[95][7]
\end{Arabic}}
\flushleft{\begin{malayalam}
എന്നിട്ടും രക്ഷാശിക്ഷകളുടെ കാര്യത്തില്‍ നിന്നെ കള്ളമാക്കുന്നതെന്ത്?
\end{malayalam}}
\flushright{\begin{Arabic}
\quranayah[95][8]
\end{Arabic}}
\flushleft{\begin{malayalam}
വിധികര്‍ത്താക്കളില്‍ ഏറ്റവും നല്ല വിധികര്‍ത്താവ് അല്ലാഹുവല്ലയോ?
\end{malayalam}}
\chapter{\textmalayalam{അലഖ് ( ഭ്രൂണം )}}
\begin{Arabic}
\Huge{\centerline{\basmalah}}\end{Arabic}
\flushright{\begin{Arabic}
\quranayah[96][1]
\end{Arabic}}
\flushleft{\begin{malayalam}
വായിക്കുക, സൃഷ്ടിച്ച നിന്റെ നാഥന്റെ നാമത്തില്‍.
\end{malayalam}}
\flushright{\begin{Arabic}
\quranayah[96][2]
\end{Arabic}}
\flushleft{\begin{malayalam}
ഒട്ടിപ്പിടിക്കുന്നതില്‍നിന്ന് അവന്‍ മനുഷ്യനെ സൃഷ്ടിച്ചു.
\end{malayalam}}
\flushright{\begin{Arabic}
\quranayah[96][3]
\end{Arabic}}
\flushleft{\begin{malayalam}
വായിക്കുക! നിന്റെ നാഥന്‍ അത്യുദാരനാണ്.
\end{malayalam}}
\flushright{\begin{Arabic}
\quranayah[96][4]
\end{Arabic}}
\flushleft{\begin{malayalam}
പേനകൊണ്ടു പഠിപ്പിച്ചവന്‍.
\end{malayalam}}
\flushright{\begin{Arabic}
\quranayah[96][5]
\end{Arabic}}
\flushleft{\begin{malayalam}
മനുഷ്യനെ അവനറിയാത്തത് അവന്‍ പഠിപ്പിച്ചു.
\end{malayalam}}
\flushright{\begin{Arabic}
\quranayah[96][6]
\end{Arabic}}
\flushleft{\begin{malayalam}
സംശയമില്ല; മനുഷ്യന്‍ അതിക്രമിയായിരിക്കുന്നു.
\end{malayalam}}
\flushright{\begin{Arabic}
\quranayah[96][7]
\end{Arabic}}
\flushleft{\begin{malayalam}
തനിക്കുതാന്‍പോന്നവനായി കണ്ടതിനാല്‍.
\end{malayalam}}
\flushright{\begin{Arabic}
\quranayah[96][8]
\end{Arabic}}
\flushleft{\begin{malayalam}
നിശ്ചയം, മടക്കം നിന്റെ നാഥങ്കലേക്കാണ്.
\end{malayalam}}
\flushright{\begin{Arabic}
\quranayah[96][9]
\end{Arabic}}
\flushleft{\begin{malayalam}
തടയുന്നവനെ നീ കണ്ടോ?
\end{malayalam}}
\flushright{\begin{Arabic}
\quranayah[96][10]
\end{Arabic}}
\flushleft{\begin{malayalam}
നമ്മുടെ ദാസനെ, അവന്‍ നമസ്കരിക്കുമ്പോള്‍
\end{malayalam}}
\flushright{\begin{Arabic}
\quranayah[96][11]
\end{Arabic}}
\flushleft{\begin{malayalam}
നീ കണ്ടോ? ആ അടിമ നേര്‍വഴിയില്‍ തന്നെയാണ്;
\end{malayalam}}
\flushright{\begin{Arabic}
\quranayah[96][12]
\end{Arabic}}
\flushleft{\begin{malayalam}
അഥവാ, ഭക്തി ഉപദേശിക്കുന്നവനാണ്!
\end{malayalam}}
\flushright{\begin{Arabic}
\quranayah[96][13]
\end{Arabic}}
\flushleft{\begin{malayalam}
നീ കണ്ടോ? ഈ തടയുന്നവന്‍ സത്യത്തെ തള്ളിക്കളയുകയും പുറംതിരിഞ്ഞു നില്‍ക്കുകയും ചെയ്തവനാണ്!
\end{malayalam}}
\flushright{\begin{Arabic}
\quranayah[96][14]
\end{Arabic}}
\flushleft{\begin{malayalam}
അല്ലാഹു എല്ലാം കാണുന്നുവെന്ന് അവന്‍ അറിയുന്നില്ലേ.
\end{malayalam}}
\flushright{\begin{Arabic}
\quranayah[96][15]
\end{Arabic}}
\flushleft{\begin{malayalam}
സംശയം വേണ്ട; അവനിത് അവസാനിപ്പിക്കുന്നില്ലെങ്കില്‍; അവന്റെ കുടുമ നാം പിടിച്ചു വലിക്കുക തന്നെ ചെയ്യും.
\end{malayalam}}
\flushright{\begin{Arabic}
\quranayah[96][16]
\end{Arabic}}
\flushleft{\begin{malayalam}
കള്ളം പറയുകയും; പാപം പ്രവര്‍ത്തിക്കുകയും ചെയ്യുന്ന കുടുമ!
\end{malayalam}}
\flushright{\begin{Arabic}
\quranayah[96][17]
\end{Arabic}}
\flushleft{\begin{malayalam}
അപ്പോഴവന്‍ തന്റെ ആളുകളെ വിളിക്കട്ടെ.
\end{malayalam}}
\flushright{\begin{Arabic}
\quranayah[96][18]
\end{Arabic}}
\flushleft{\begin{malayalam}
നാം നമ്മുടെ ശിക്ഷാകാര്യങ്ങളുടെ ചുമതലക്കാരെയും വിളിക്കാം.
\end{malayalam}}
\flushright{\begin{Arabic}
\quranayah[96][19]
\end{Arabic}}
\flushleft{\begin{malayalam}
അരുത്! നീ അവന് വഴങ്ങരുത്. നീ സാഷ്ടാംഗം പ്രണമിക്കുക! നമ്മുടെ സാമീപ്യം നേടുക
\end{malayalam}}
\chapter{\textmalayalam{ഖദ്ര്‍ ( നിര്‍ണയം )}}
\begin{Arabic}
\Huge{\centerline{\basmalah}}\end{Arabic}
\flushright{\begin{Arabic}
\quranayah[97][1]
\end{Arabic}}
\flushleft{\begin{malayalam}
തീര്‍ച്ചയായും നാം ഈ ഖുര്‍ആനിനെ വിധി നിര്‍ണായക രാവില്‍ അവതരിപ്പിച്ചു.
\end{malayalam}}
\flushright{\begin{Arabic}
\quranayah[97][2]
\end{Arabic}}
\flushleft{\begin{malayalam}
വിധി നിര്‍ണായക രാവ് എന്തെന്ന് നിനക്കെന്തറിയാം?
\end{malayalam}}
\flushright{\begin{Arabic}
\quranayah[97][3]
\end{Arabic}}
\flushleft{\begin{malayalam}
വിധി നിര്‍ണായക രാവ് ആയിരം മാസത്തെക്കാള്‍ മഹത്തരമാണ്.
\end{malayalam}}
\flushright{\begin{Arabic}
\quranayah[97][4]
\end{Arabic}}
\flushleft{\begin{malayalam}
ആ രാവില്‍ മലക്കുകളും ജിബ്രീലും ഇറങ്ങി വരുന്നു. സമസ്ത കാര്യങ്ങളെയും സംബന്ധിച്ച തങ്ങളുടെ നാഥന്റെ ഉത്തരവുമായി.
\end{malayalam}}
\flushright{\begin{Arabic}
\quranayah[97][5]
\end{Arabic}}
\flushleft{\begin{malayalam}
പുലരൊളി വിരിയും വരെ അത് പ്രശാന്തമായിരിക്കും.
\end{malayalam}}
\chapter{\textmalayalam{ബയ്യിന ( വ്യക്തമായ തെളിവ് )}}
\begin{Arabic}
\Huge{\centerline{\basmalah}}\end{Arabic}
\flushright{\begin{Arabic}
\quranayah[98][1]
\end{Arabic}}
\flushleft{\begin{malayalam}
വേദക്കാരും ബഹുദൈവ വിശ്വാസികളുമായ സത്യനിഷേധികള്‍ വ്യക്തമായ തെളിവ് വന്നെത്തും വരെ തങ്ങളുടെ വഴിയില്‍ ഉറച്ചുനിന്നു.
\end{malayalam}}
\flushright{\begin{Arabic}
\quranayah[98][2]
\end{Arabic}}
\flushleft{\begin{malayalam}
അല്ലാഹുവില്‍ നിന്നുള്ള ദൂതന്‍ പവിത്രമായ ഗ്രന്ഥത്താളുകള്‍ വായിച്ചു കേള്‍പ്പിക്കുന്നത് വരെ.
\end{malayalam}}
\flushright{\begin{Arabic}
\quranayah[98][3]
\end{Arabic}}
\flushleft{\begin{malayalam}
ആ ഗ്രന്ഥത്താളുകളില്‍ സത്യനിഷ്ഠമായ പ്രമാണങ്ങളുണ്ട്.
\end{malayalam}}
\flushright{\begin{Arabic}
\quranayah[98][4]
\end{Arabic}}
\flushleft{\begin{malayalam}
വേദം നല്‍കപ്പെട്ടവര്‍ ഭിന്നിച്ചിട്ടില്ല. അവര്‍ക്കു വ്യക്തമായ തെളിവ് വന്നെത്തിയ ശേഷമല്ലാതെ.
\end{malayalam}}
\flushright{\begin{Arabic}
\quranayah[98][5]
\end{Arabic}}
\flushleft{\begin{malayalam}
വിധേയത്വം അല്ലാഹുവിനു മാത്രമാക്കി അവനെ മാത്രം വഴിപ്പെട്ട് നേര്‍വഴിയില്‍ ജീവിക്കാനല്ലാതെ അവരോട് കല്‍പിച്ചിട്ടില്ല. ഒപ്പം നമസ്കാരം നിഷ്ഠയോടെ നിര്‍വഹിക്കാനും സകാത് നല്‍കാനും. അതാണ് ചൊവ്വായ ജീവിതക്രമം.
\end{malayalam}}
\flushright{\begin{Arabic}
\quranayah[98][6]
\end{Arabic}}
\flushleft{\begin{malayalam}
തീര്‍ച്ചയായും വേദക്കാരും ബഹുദൈവ വിശ്വാസികളുമായ സത്യനിഷേധികള്‍ നരകത്തീയിലാണ്. അവരതില്‍ സ്ഥിരവാസികളായിരിക്കും. അവരാണ് സൃഷ്ടികളിലേറ്റം നികൃഷ്ടര്‍.
\end{malayalam}}
\flushright{\begin{Arabic}
\quranayah[98][7]
\end{Arabic}}
\flushleft{\begin{malayalam}
എന്നാല്‍ സത്യവിശ്വാസം സ്വീകരിക്കുകയും സല്‍ക്കര്‍മങ്ങള്‍ പ്രവര്‍ത്തിക്കുകയും ചെയ്തവരോ, അവരാണ് സൃഷ്ടികളിലേറ്റം ശ്രേഷ്ഠര്‍.
\end{malayalam}}
\flushright{\begin{Arabic}
\quranayah[98][8]
\end{Arabic}}
\flushleft{\begin{malayalam}
അവര്‍ക്ക് അവരുടെ നാഥങ്കല്‍ അര്‍ഹമായ പ്രതിഫലമുണ്ട്. താഴ്ഭാഗത്തൂടെ ആറുകളൊഴുകുന്ന സ്വര്‍ഗീയാരാമങ്ങള്‍. അവരതില്‍ എക്കാലവും സ്ഥിരവാസികളായിരിക്കും. അല്ലാഹു അവരെക്കുറിച്ച് തൃപ്തനായിരിക്കും. അവര്‍ അല്ലാഹുവിലും സംതൃപ്തരായിരിക്കും. ഇതെല്ലാം തങ്ങളുടെ നാഥനെ ഭയപ്പെടുന്നവര്‍ക്കുള്ളതാണ്.
\end{malayalam}}
\chapter{\textmalayalam{സല്‍സല ( പ്രകമ്പനം )}}
\begin{Arabic}
\Huge{\centerline{\basmalah}}\end{Arabic}
\flushright{\begin{Arabic}
\quranayah[99][1]
\end{Arabic}}
\flushleft{\begin{malayalam}
ഭൂമി അതിശക്തിയായി വിറകൊണ്ടാല്‍.
\end{malayalam}}
\flushright{\begin{Arabic}
\quranayah[99][2]
\end{Arabic}}
\flushleft{\begin{malayalam}
ഭൂമി അതിന്റെ ഭാരങ്ങള്‍ പുറംതള്ളിയാല്‍.
\end{malayalam}}
\flushright{\begin{Arabic}
\quranayah[99][3]
\end{Arabic}}
\flushleft{\begin{malayalam}
മനുഷ്യന്‍ ചോദിക്കും: അതിനെന്തു പറ്റി?
\end{malayalam}}
\flushright{\begin{Arabic}
\quranayah[99][4]
\end{Arabic}}
\flushleft{\begin{malayalam}
അന്നാളില്‍ ഭൂമി അതിന്റെ വിവരമൊക്കെ പറഞ്ഞറിയിക്കും.
\end{malayalam}}
\flushright{\begin{Arabic}
\quranayah[99][5]
\end{Arabic}}
\flushleft{\begin{malayalam}
നിന്റെ നാഥന്‍ അതിനു ബോധനം നല്‍കിയതിനാലാണിത്.
\end{malayalam}}
\flushright{\begin{Arabic}
\quranayah[99][6]
\end{Arabic}}
\flushleft{\begin{malayalam}
അന്നാളില്‍ ജനം പല സംഘങ്ങളായി പുറപ്പെടും; തങ്ങളുടെ പ്രവര്‍ത്തനഫലങ്ങള്‍ നേരില്‍ കാണാന്‍.
\end{malayalam}}
\flushright{\begin{Arabic}
\quranayah[99][7]
\end{Arabic}}
\flushleft{\begin{malayalam}
അതിനാല്‍, അണുത്തൂക്കം നന്മ ചെയ്തവന്‍ അത് കാണും.
\end{malayalam}}
\flushright{\begin{Arabic}
\quranayah[99][8]
\end{Arabic}}
\flushleft{\begin{malayalam}
അണുത്തൂക്കം തിന്മ ചെയ്തവന്‍ അതും കാണും.
\end{malayalam}}
\chapter{\textmalayalam{ആദിയാത് ( ഓടുന്നവ )}}
\begin{Arabic}
\Huge{\centerline{\basmalah}}\end{Arabic}
\flushright{\begin{Arabic}
\quranayah[100][1]
\end{Arabic}}
\flushleft{\begin{malayalam}
കിതച്ചോടുന്നവ സാക്ഷി.
\end{malayalam}}
\flushright{\begin{Arabic}
\quranayah[100][2]
\end{Arabic}}
\flushleft{\begin{malayalam}
അങ്ങനെ കുളമ്പുരസി തീപ്പൊരി പറത്തുന്നവ സാക്ഷി.
\end{malayalam}}
\flushright{\begin{Arabic}
\quranayah[100][3]
\end{Arabic}}
\flushleft{\begin{malayalam}
പുലര്‍ച്ചെ ആക്രമണം നടത്തുന്നവ സാക്ഷി.
\end{malayalam}}
\flushright{\begin{Arabic}
\quranayah[100][4]
\end{Arabic}}
\flushleft{\begin{malayalam}
അങ്ങനെ പൊടിപടലം ഇളക്കിവിടുന്നവ സാക്ഷി.
\end{malayalam}}
\flushright{\begin{Arabic}
\quranayah[100][5]
\end{Arabic}}
\flushleft{\begin{malayalam}
ശത്രുക്കള്‍ക്കു നടുവില്‍ കടന്നുചെല്ലുന്നവ സാക്ഷി.
\end{malayalam}}
\flushright{\begin{Arabic}
\quranayah[100][6]
\end{Arabic}}
\flushleft{\begin{malayalam}
തീര്‍ച്ചയായും മനുഷ്യന്‍ തന്റെ നാഥനോട് നന്ദിയില്ലാത്തവനാണ്
\end{malayalam}}
\flushright{\begin{Arabic}
\quranayah[100][7]
\end{Arabic}}
\flushleft{\begin{malayalam}
ഉറപ്പായും അവന്‍ തന്നെ ഈ നന്ദികേടിനു സാക്ഷിയാണ്;
\end{malayalam}}
\flushright{\begin{Arabic}
\quranayah[100][8]
\end{Arabic}}
\flushleft{\begin{malayalam}
ധനത്തോടുള്ള അവന്റെ ആര്‍ത്തി അതികഠിനം തന്നെ;
\end{malayalam}}
\flushright{\begin{Arabic}
\quranayah[100][9]
\end{Arabic}}
\flushleft{\begin{malayalam}
അവന്‍ അറിയുന്നില്ലേ? ഖബറുകളിലുള്ളവ ഇളക്കിമറിക്കപ്പെടുകയും.
\end{malayalam}}
\flushright{\begin{Arabic}
\quranayah[100][10]
\end{Arabic}}
\flushleft{\begin{malayalam}
ഹൃദയങ്ങളിലുള്ളവ വെളിവാക്കപ്പെടുകയും ചെയ്യുമ്പോള്‍.
\end{malayalam}}
\flushright{\begin{Arabic}
\quranayah[100][11]
\end{Arabic}}
\flushleft{\begin{malayalam}
സംശയമില്ല; അന്നാളില്‍ അവരുടെ നാഥന്‍ അവരെപ്പറ്റി സൂക്ഷ്മമായി അറിയുന്നവനാണ്.
\end{malayalam}}
\chapter{\textmalayalam{അല്‍ ഖാരിഅ ( ഭയങ്കര സംഭവം )}}
\begin{Arabic}
\Huge{\centerline{\basmalah}}\end{Arabic}
\flushright{\begin{Arabic}
\quranayah[101][1]
\end{Arabic}}
\flushleft{\begin{malayalam}
ഭയങ്കര സംഭവം!
\end{malayalam}}
\flushright{\begin{Arabic}
\quranayah[101][2]
\end{Arabic}}
\flushleft{\begin{malayalam}
എന്താണാ ഭയങ്കര സംഭവം?
\end{malayalam}}
\flushright{\begin{Arabic}
\quranayah[101][3]
\end{Arabic}}
\flushleft{\begin{malayalam}
ആ ഭയങ്കര സംഭവം ഏതെന്ന് നിനക്കെന്തറിയാം?
\end{malayalam}}
\flushright{\begin{Arabic}
\quranayah[101][4]
\end{Arabic}}
\flushleft{\begin{malayalam}
അന്ന് മനുഷ്യര്‍ ചിന്നിച്ചിതറിയ പാറ്റപോലെയാകും.
\end{malayalam}}
\flushright{\begin{Arabic}
\quranayah[101][5]
\end{Arabic}}
\flushleft{\begin{malayalam}
പര്‍വതങ്ങള്‍ കടഞ്ഞ കമ്പിളി രോമം പോലെയും.
\end{malayalam}}
\flushright{\begin{Arabic}
\quranayah[101][6]
\end{Arabic}}
\flushleft{\begin{malayalam}
അപ്പോള്‍ ആരുടെ തുലാസിന്റെ തട്ട് കനം തൂങ്ങുന്നുവോ,
\end{malayalam}}
\flushright{\begin{Arabic}
\quranayah[101][7]
\end{Arabic}}
\flushleft{\begin{malayalam}
അവന് സംതൃപ്തമായ ജീവിതമുണ്ട്.
\end{malayalam}}
\flushright{\begin{Arabic}
\quranayah[101][8]
\end{Arabic}}
\flushleft{\begin{malayalam}
ആരുടെ തുലാസിന്‍ തട്ട് കനം കുറയുന്നുവോ,
\end{malayalam}}
\flushright{\begin{Arabic}
\quranayah[101][9]
\end{Arabic}}
\flushleft{\begin{malayalam}
അവന്റെ സങ്കേതം ഹാവിയ ആയിരിക്കും.
\end{malayalam}}
\flushright{\begin{Arabic}
\quranayah[101][10]
\end{Arabic}}
\flushleft{\begin{malayalam}
ഹാവിയ ഏതെന്ന് നിനക്കെന്തറിയാം?
\end{malayalam}}
\flushright{\begin{Arabic}
\quranayah[101][11]
\end{Arabic}}
\flushleft{\begin{malayalam}
അത് കൊടും ചൂടുള്ള നരകത്തീ തന്നെ.
\end{malayalam}}
\chapter{\textmalayalam{തകാഥുര്‍ (പെരുമ നടിക്കല്‍ )}}
\begin{Arabic}
\Huge{\centerline{\basmalah}}\end{Arabic}
\flushright{\begin{Arabic}
\quranayah[102][1]
\end{Arabic}}
\flushleft{\begin{malayalam}
പരസ്പരം പെരുമനടിക്കല്‍ നിങ്ങളെ അശ്രദ്ധരാക്കിയിരിക്കുന്നു.
\end{malayalam}}
\flushright{\begin{Arabic}
\quranayah[102][2]
\end{Arabic}}
\flushleft{\begin{malayalam}
നിങ്ങള്‍ ശവക്കുഴികള്‍ സന്ദര്‍ശിക്കും വരെ.
\end{malayalam}}
\flushright{\begin{Arabic}
\quranayah[102][3]
\end{Arabic}}
\flushleft{\begin{malayalam}
സംശയം വേണ്ട; നിങ്ങളതറിയുകതന്നെ ചെയ്യും.
\end{malayalam}}
\flushright{\begin{Arabic}
\quranayah[102][4]
\end{Arabic}}
\flushleft{\begin{malayalam}
വീണ്ടും സംശയം വേണ്ട; നിങ്ങളതറിയുകതന്നെ ചെയ്യും.
\end{malayalam}}
\flushright{\begin{Arabic}
\quranayah[102][5]
\end{Arabic}}
\flushleft{\begin{malayalam}
നിസ്സംശയം! നിങ്ങള്‍ ദൃഢമായ അറിവ് നേടിയിരുന്നെങ്കില്‍!
\end{malayalam}}
\flushright{\begin{Arabic}
\quranayah[102][6]
\end{Arabic}}
\flushleft{\begin{malayalam}
നരകത്തെ നിങ്ങള്‍ നേരില്‍ കാണുകതന്നെ ചെയ്യും.
\end{malayalam}}
\flushright{\begin{Arabic}
\quranayah[102][7]
\end{Arabic}}
\flushleft{\begin{malayalam}
പിന്നെ നിങ്ങളതിനെ ഉറപ്പായും കണ്ണുകൊണ്ട് കാണുകതന്നെ ചെയ്യും.
\end{malayalam}}
\flushright{\begin{Arabic}
\quranayah[102][8]
\end{Arabic}}
\flushleft{\begin{malayalam}
പിന്നീട് നിങ്ങളനുഭവിച്ച സുഖാനുഗ്രഹങ്ങളെപ്പറ്റി അന്ന് നിങ്ങളെ ചോദ്യം ചെയ്യും.
\end{malayalam}}
\chapter{\textmalayalam{അസ്വര്‍ ( കാലം )}}
\begin{Arabic}
\Huge{\centerline{\basmalah}}\end{Arabic}
\flushright{\begin{Arabic}
\quranayah[103][1]
\end{Arabic}}
\flushleft{\begin{malayalam}
കാലം സാക്ഷി.
\end{malayalam}}
\flushright{\begin{Arabic}
\quranayah[103][2]
\end{Arabic}}
\flushleft{\begin{malayalam}
തീര്‍ച്ചയായും മനുഷ്യരൊക്കെയും നഷ്ടത്തിലാണ്.
\end{malayalam}}
\flushright{\begin{Arabic}
\quranayah[103][3]
\end{Arabic}}
\flushleft{\begin{malayalam}
സത്യവിശ്വാസം സ്വീകരിച്ചവരും സല്‍ക്കര്‍മങ്ങള്‍ പ്രവര്‍ത്തിച്ചവരും; സത്യം സ്വീകരിക്കാനും ക്ഷമ പാലിക്കാനും പരസ്പരമുപദേശിച്ചവരുമൊഴികെ.
\end{malayalam}}
\chapter{\textmalayalam{ഹുമസ (കുത്തിപ്പറയുന്നവര്‍ )}}
\begin{Arabic}
\Huge{\centerline{\basmalah}}\end{Arabic}
\flushright{\begin{Arabic}
\quranayah[104][1]
\end{Arabic}}
\flushleft{\begin{malayalam}
കുത്തുവാക്ക് പറയുന്നവനൊക്കെയും നാശം! അവഹേളിക്കുന്നവന്നും!
\end{malayalam}}
\flushright{\begin{Arabic}
\quranayah[104][2]
\end{Arabic}}
\flushleft{\begin{malayalam}
അവനോ, ധനം ഒരുക്കൂട്ടുകയും അത് എണ്ണിക്കണക്കാക്കുകയും ചെയ്യുന്നവനാണ്.
\end{malayalam}}
\flushright{\begin{Arabic}
\quranayah[104][3]
\end{Arabic}}
\flushleft{\begin{malayalam}
ധനം തന്നെ അനശ്വരനാക്കിയതായി അവന്‍ കരുതുന്നു.
\end{malayalam}}
\flushright{\begin{Arabic}
\quranayah[104][4]
\end{Arabic}}
\flushleft{\begin{malayalam}
സംശയം വേണ്ട; അവന്‍ ഹുത്വമയില്‍ എറിയപ്പെടുക തന്നെ ചെയ്യും.
\end{malayalam}}
\flushright{\begin{Arabic}
\quranayah[104][5]
\end{Arabic}}
\flushleft{\begin{malayalam}
ഹുത്വമ എന്തെന്ന് നിനക്കറിയാമോ?
\end{malayalam}}
\flushright{\begin{Arabic}
\quranayah[104][6]
\end{Arabic}}
\flushleft{\begin{malayalam}
അല്ലാഹുവിന്റെ കത്തിപ്പടരും നരകത്തീയാണത്.
\end{malayalam}}
\flushright{\begin{Arabic}
\quranayah[104][7]
\end{Arabic}}
\flushleft{\begin{malayalam}
ഹൃദയങ്ങളിലേക്ക് കത്തിപ്പടരുന്നത്.
\end{malayalam}}
\flushright{\begin{Arabic}
\quranayah[104][8]
\end{Arabic}}
\flushleft{\begin{malayalam}
അത് അവരുടെ മേല്‍ മൂടിയിരിക്കും;
\end{malayalam}}
\flushright{\begin{Arabic}
\quranayah[104][9]
\end{Arabic}}
\flushleft{\begin{malayalam}
നാട്ടിനിര്‍ത്തിയ സ്തംഭങ്ങളില്‍ അവര്‍ ബന്ധിതരായിരിക്കെ.
\end{malayalam}}
\chapter{\textmalayalam{ഫീല്‍ ( ആന )}}
\begin{Arabic}
\Huge{\centerline{\basmalah}}\end{Arabic}
\flushright{\begin{Arabic}
\quranayah[105][1]
\end{Arabic}}
\flushleft{\begin{malayalam}
ആനക്കാരെ നിന്റെ നാഥന്‍ ചെയ്തതെങ്ങനെയെന്ന് നീ കണ്ടില്ലേ?
\end{malayalam}}
\flushright{\begin{Arabic}
\quranayah[105][2]
\end{Arabic}}
\flushleft{\begin{malayalam}
അവരുടെ കുതന്ത്രം അവന്‍ പാഴാക്കിയില്ലേ?
\end{malayalam}}
\flushright{\begin{Arabic}
\quranayah[105][3]
\end{Arabic}}
\flushleft{\begin{malayalam}
അവരുടെ നേരെ അവന്‍ പക്ഷിക്കൂട്ടങ്ങളെ അയച്ചു.
\end{malayalam}}
\flushright{\begin{Arabic}
\quranayah[105][4]
\end{Arabic}}
\flushleft{\begin{malayalam}
ചുട്ടെടുത്ത കല്ലുകള്‍കൊണ്ട് ആ പറവകള്‍ അവരെ എറിഞ്ഞുകൊണ്ടിരുന്നു.
\end{malayalam}}
\flushright{\begin{Arabic}
\quranayah[105][5]
\end{Arabic}}
\flushleft{\begin{malayalam}
അങ്ങനെ അല്ലാഹു അവരെ ചവച്ചരച്ച കച്ചിത്തുരുമ്പുപോലെയാക്കി.
\end{malayalam}}
\chapter{\textmalayalam{ഖുറൈഷ്}}
\begin{Arabic}
\Huge{\centerline{\basmalah}}\end{Arabic}
\flushright{\begin{Arabic}
\quranayah[106][1]
\end{Arabic}}
\flushleft{\begin{malayalam}
ഖുറൈശികളെ ഇണക്കിയതിനാല്‍
\end{malayalam}}
\flushright{\begin{Arabic}
\quranayah[106][2]
\end{Arabic}}
\flushleft{\begin{malayalam}
അഥവാ ശൈത്യകാലത്തെയും ഉഷ്ണകാലത്തെയും യാത്രയോടുള്ള അവരുടെ ഇണക്കം.
\end{malayalam}}
\flushright{\begin{Arabic}
\quranayah[106][3]
\end{Arabic}}
\flushleft{\begin{malayalam}
അതിനാല്‍ ഈ കഅ്ബാമന്ദിരത്തിന്റെ നാഥന് അവര്‍ വഴിപ്പെടട്ടെ.
\end{malayalam}}
\flushright{\begin{Arabic}
\quranayah[106][4]
\end{Arabic}}
\flushleft{\begin{malayalam}
അവര്‍ക്ക് വിശപ്പടക്കാന്‍ ആഹാരവും പേടിക്കു പകരം നിര്‍ഭയത്വവും നല്‍കിയവനാണവന്‍.
\end{malayalam}}
\chapter{\textmalayalam{മാഊന്‍ (  പരോപകാര വസ്തുക്കള്‍ )}}
\begin{Arabic}
\Huge{\centerline{\basmalah}}\end{Arabic}
\flushright{\begin{Arabic}
\quranayah[107][1]
\end{Arabic}}
\flushleft{\begin{malayalam}
മതത്തെ നിഷേധിക്കുന്നവനെ നീ കണ്ടോ?
\end{malayalam}}
\flushright{\begin{Arabic}
\quranayah[107][2]
\end{Arabic}}
\flushleft{\begin{malayalam}
അത് അനാഥയെ ആട്ടിയകറ്റുന്നവനാണ്.
\end{malayalam}}
\flushright{\begin{Arabic}
\quranayah[107][3]
\end{Arabic}}
\flushleft{\begin{malayalam}
അഗതിയുടെ അന്നം കൊടുക്കാന്‍ പ്രേരിപ്പിക്കാത്തവനും.
\end{malayalam}}
\flushright{\begin{Arabic}
\quranayah[107][4]
\end{Arabic}}
\flushleft{\begin{malayalam}
അതിനാല്‍ നമസ്കാരക്കാര്‍ക്ക് നാശം!
\end{malayalam}}
\flushright{\begin{Arabic}
\quranayah[107][5]
\end{Arabic}}
\flushleft{\begin{malayalam}
അവരോ, തങ്ങളുടെ നമസ്കാര കാര്യത്തില്‍ അശ്രദ്ധരാണ്.
\end{malayalam}}
\flushright{\begin{Arabic}
\quranayah[107][6]
\end{Arabic}}
\flushleft{\begin{malayalam}
അവര്‍ ആളുകളെ കാണിക്കാന്‍ വേണ്ടി ചെയ്യുന്നവരാണ്.
\end{malayalam}}
\flushright{\begin{Arabic}
\quranayah[107][7]
\end{Arabic}}
\flushleft{\begin{malayalam}
നിസ്സാരമായ സഹായം പോലും മുടക്കുന്നവരും.
\end{malayalam}}
\chapter{\textmalayalam{കൌഥര്‍‍ ( ധാരാളം )}}
\begin{Arabic}
\Huge{\centerline{\basmalah}}\end{Arabic}
\flushright{\begin{Arabic}
\quranayah[108][1]
\end{Arabic}}
\flushleft{\begin{malayalam}
നിനക്കു നാം ധാരാളം നന്മ നല്‍കിയിരിക്കുന്നു.
\end{malayalam}}
\flushright{\begin{Arabic}
\quranayah[108][2]
\end{Arabic}}
\flushleft{\begin{malayalam}
അതിനാല്‍ നീ നിന്റെ നാഥന്ന് നമസ്കരിക്കുക. അവന്ന് ബലിയര്‍പ്പിക്കുക.
\end{malayalam}}
\flushright{\begin{Arabic}
\quranayah[108][3]
\end{Arabic}}
\flushleft{\begin{malayalam}
നിശ്ചയം നിന്നോട് ശത്രുത പുലര്‍ത്തുന്നവന്‍ തന്നെയാണ് വാലറ്റവന്‍.
\end{malayalam}}
\chapter{\textmalayalam{കാഫിറൂന്‍ ( സത്യനിഷേധികള്‍ )}}
\begin{Arabic}
\Huge{\centerline{\basmalah}}\end{Arabic}
\flushright{\begin{Arabic}
\quranayah[109][1]
\end{Arabic}}
\flushleft{\begin{malayalam}
പറയുക: അല്ലയോ സത്യനിഷേധികളേ,
\end{malayalam}}
\flushright{\begin{Arabic}
\quranayah[109][2]
\end{Arabic}}
\flushleft{\begin{malayalam}
നിങ്ങള്‍ ആരാധിക്കുന്നവയെ ഞാന്‍ ആരാധിക്കുന്നില്ല.
\end{malayalam}}
\flushright{\begin{Arabic}
\quranayah[109][3]
\end{Arabic}}
\flushleft{\begin{malayalam}
ഞാന്‍ ആരാധിക്കുന്നതിനെ ആരാധിക്കുന്നവരല്ല നിങ്ങള്‍.
\end{malayalam}}
\flushright{\begin{Arabic}
\quranayah[109][4]
\end{Arabic}}
\flushleft{\begin{malayalam}
നിങ്ങള്‍ ആരാധിക്കുന്നവയെ ആരാധിക്കുന്നവനല്ല ഞാന്‍.
\end{malayalam}}
\flushright{\begin{Arabic}
\quranayah[109][5]
\end{Arabic}}
\flushleft{\begin{malayalam}
ഞാന്‍ ആരാധിക്കുന്നതിനെ ആരാധിക്കുന്നവരല്ല നിങ്ങളും.
\end{malayalam}}
\flushright{\begin{Arabic}
\quranayah[109][6]
\end{Arabic}}
\flushleft{\begin{malayalam}
നിങ്ങള്‍ക്ക് നിങ്ങളുടെ മതം. എനിക്ക് എന്റെ മതം.
\end{malayalam}}
\chapter{\textmalayalam{നസ്ര്‍ ( സഹായം )}}
\begin{Arabic}
\Huge{\centerline{\basmalah}}\end{Arabic}
\flushright{\begin{Arabic}
\quranayah[110][1]
\end{Arabic}}
\flushleft{\begin{malayalam}
അല്ലാഹുവിന്റെ സഹായവും വിജയവും വന്നെത്തിയാല്‍;
\end{malayalam}}
\flushright{\begin{Arabic}
\quranayah[110][2]
\end{Arabic}}
\flushleft{\begin{malayalam}
ജനം കൂട്ടംകൂട്ടമായി ദൈവിക മതത്തില്‍ കടന്നുവരുന്നത് നീ കാണുകയും ചെയ്താല്‍;
\end{malayalam}}
\flushright{\begin{Arabic}
\quranayah[110][3]
\end{Arabic}}
\flushleft{\begin{malayalam}
നിന്റെ നാഥനെ നീ സ്തുതിച്ച് വാഴ്ത്തുക. അവനോട് പാപമോചനം തേടുക. തീര്‍ച്ചയായും അവന്‍ പശ്ചാത്താപം സ്വീകരിക്കുന്നവനാണ്.
\end{malayalam}}
\chapter{\textmalayalam{മസദ് (  ഈന്തപ്പനനാര് )}}
\begin{Arabic}
\Huge{\centerline{\basmalah}}\end{Arabic}
\flushright{\begin{Arabic}
\quranayah[111][1]
\end{Arabic}}
\flushleft{\begin{malayalam}
അബൂലഹബിന്റെ ഇരു കരങ്ങളും നശിക്കട്ടെ. അവന്‍ നശിച്ചിരിക്കുന്നു.
\end{malayalam}}
\flushright{\begin{Arabic}
\quranayah[111][2]
\end{Arabic}}
\flushleft{\begin{malayalam}
അവന്റെ സ്വത്തോ അവന്‍ സമ്പാദിച്ചതോ അവന്നൊട്ടും ഉപകരിച്ചില്ല.
\end{malayalam}}
\flushright{\begin{Arabic}
\quranayah[111][3]
\end{Arabic}}
\flushleft{\begin{malayalam}
ആളിക്കത്തുന്ന നരകത്തിലവന്‍ ചെന്നെത്തും.
\end{malayalam}}
\flushright{\begin{Arabic}
\quranayah[111][4]
\end{Arabic}}
\flushleft{\begin{malayalam}
വിറക് ചുമക്കുന്ന അവന്റെ ഭാര്യയും.
\end{malayalam}}
\flushright{\begin{Arabic}
\quranayah[111][5]
\end{Arabic}}
\flushleft{\begin{malayalam}
അവളുടെ കഴുത്തില്‍ ഈന്തപ്പന നാരുകൊണ്ടുള്ള കയറുണ്ട്.
\end{malayalam}}
\chapter{\textmalayalam{ഇഖ് ലാസ് ( നിഷ്കളങ്കത )}}
\begin{Arabic}
\Huge{\centerline{\basmalah}}\end{Arabic}
\flushright{\begin{Arabic}
\quranayah[112][1]
\end{Arabic}}
\flushleft{\begin{malayalam}
പറയുക, അവനാണ് അല്ലാഹു. അവന്‍ ഏകനാണ്.
\end{malayalam}}
\flushright{\begin{Arabic}
\quranayah[112][2]
\end{Arabic}}
\flushleft{\begin{malayalam}
അല്ലാഹു ആരെയും ആശ്രയിക്കാത്തവനാണ്. ഏവരാലും ആശ്രയിക്കപ്പെടുന്നവനും.
\end{malayalam}}
\flushright{\begin{Arabic}
\quranayah[112][3]
\end{Arabic}}
\flushleft{\begin{malayalam}
അവന്‍ പിതാവോ പുത്രനോ അല്ല.
\end{malayalam}}
\flushright{\begin{Arabic}
\quranayah[112][4]
\end{Arabic}}
\flushleft{\begin{malayalam}
അവനു തുല്യനായി ആരുമില്ല.
\end{malayalam}}
\chapter{\textmalayalam{ഫലഖ് ( പുലരി )}}
\begin{Arabic}
\Huge{\centerline{\basmalah}}\end{Arabic}
\flushright{\begin{Arabic}
\quranayah[113][1]
\end{Arabic}}
\flushleft{\begin{malayalam}
പറയുക: പ്രഭാതത്തിന്റെ നാഥനോട് ഞാന്‍ ശരണം തേടുന്നു.
\end{malayalam}}
\flushright{\begin{Arabic}
\quranayah[113][2]
\end{Arabic}}
\flushleft{\begin{malayalam}
അവന്‍ സൃഷ്ടിച്ചവയുടെ ദ്രോഹത്തില്‍നിന്ന്.
\end{malayalam}}
\flushright{\begin{Arabic}
\quranayah[113][3]
\end{Arabic}}
\flushleft{\begin{malayalam}
ഇരുള്‍ മൂടുമ്പോഴത്തെ രാവിന്റെ ദ്രോഹത്തില്‍നിന്ന്.
\end{malayalam}}
\flushright{\begin{Arabic}
\quranayah[113][4]
\end{Arabic}}
\flushleft{\begin{malayalam}
കെട്ടുകളില്‍ ഊതുന്നവരുടെ ദ്രോഹത്തില്‍നിന്ന്.
\end{malayalam}}
\flushright{\begin{Arabic}
\quranayah[113][5]
\end{Arabic}}
\flushleft{\begin{malayalam}
അസൂയാലു അസൂയ കാണിച്ചാലുള്ള ദ്രോഹത്തില്‍നിന്ന്.
\end{malayalam}}
\chapter{\textmalayalam{നാസ് ( ജനങ്ങള്‍ )}}
\begin{Arabic}
\Huge{\centerline{\basmalah}}\end{Arabic}
\flushright{\begin{Arabic}
\quranayah[114][1]
\end{Arabic}}
\flushleft{\begin{malayalam}
പറയുക: ഞാന്‍ ശരണം തേടുന്നു, ജനങ്ങളുടെ നാഥനോട്.
\end{malayalam}}
\flushright{\begin{Arabic}
\quranayah[114][2]
\end{Arabic}}
\flushleft{\begin{malayalam}
ജനങ്ങളുടെ രാജാവിനോട്.
\end{malayalam}}
\flushright{\begin{Arabic}
\quranayah[114][3]
\end{Arabic}}
\flushleft{\begin{malayalam}
ജനങ്ങളുടെ ആരാധ്യനോട്.
\end{malayalam}}
\flushright{\begin{Arabic}
\quranayah[114][4]
\end{Arabic}}
\flushleft{\begin{malayalam}
ദുര്‍ബോധനം നടത്തി പിന്മാറുന്ന പിശാചിന്റെ ദ്രോഹത്തില്‍നിന്ന്.
\end{malayalam}}
\flushright{\begin{Arabic}
\quranayah[114][5]
\end{Arabic}}
\flushleft{\begin{malayalam}
അവന്‍, മനുഷ്യ മനസ്സുകളില്‍ ദുര്‍ബോധനം നടത്തുന്നവനാണ്.
\end{malayalam}}
\flushright{\begin{Arabic}
\quranayah[114][6]
\end{Arabic}}
\flushleft{\begin{malayalam}
മനുഷ്യരിലും ജിന്നുകളിലും പെട്ടവനും.
\end{malayalam}}


%\quransurah[108] % Surah Al-Kauther

%%\surna[110]\quransurah*[110]  % Surah Al-Nasr

%The below typeset 104th surah through 113th surah.
%%\quransurah*[104-113]

%\makeatletter
%\surna[\qt@surah@default]\quransurah*  % Surah Al-Ikhlas
%\makeatother

%\quranayah[33][33]
%\quranayah*[76][1-22]

%%You can typeset whole of Holy Quran with below commands.
%%\newcounter{ct}
%%\forloop{ct}{1}{\value{ct} < 115} {\test{\value{ct}}}

%%You can typeset whole of Holy Quran with below commands.
%%\newcounter{jz}
%%\forloop{jz}{1}{\value{jz} < 31} {\quranjuz[\value{jz}]}

%%\quranjuz*[28-30]
%\hfill  صفحة  \arabic{pg} \par
%\quranpage*[\value{pg}]\vfill}

%\quranhizb*[117-120]

%\quranquarter*[1-4]
%\quranquarter*[239-240]

%\quranruku[313]
%\quranruku[556]

%\quranmanzil*[2]


%%\surna[1]\qurantext  % Surah Al-Hamd

%%\surna[1]\qurantext* % Surah Al-Hamd

%%\surna[114]\qurantext[6231-6236]  % Surah Al-Nas

%%\surna[114]\qurantext*[6231-6236]  % Surah Al-Nas

%%%\surna[2]\qurantext*[8-293] % Surah Al-Baqara

%\qurantext[1-6236] % The whole of Holy Quran
\end{document}
